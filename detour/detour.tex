\documentclass[lotsofwhite]{patmorin}
\usepackage{charter}
%\documentclass[lotsofwhite,ccfonts]{patmorin}
\usepackage{bbold}
\usepackage{graphicx}
\usepackage{ipe}
%\usepackage[colorlinks=true, pdfstartview=FitV, linkcolor=black,
%            citecoloa=black, urlcolor=black]{hyperref}
\usepackage{amsthm}

\newcommand{\email}[1]{\texttt{#1}}
\input{pat.tex}

\title{\MakeUppercase{Computing the Maximum Detour and Spanning 
	Ratio of 2 and 3-Dimensional Paths, Trees and Cycles}%
	\thanks{This research was partly funded by CRM, FCAR, MITACS, 
		and NSERC.}}

\author{Pankaj K. Agarwal \\ \textit{Duke University} \and
	Rolf Klein \\ \textit{FernUniversit\"at Hagen} \and
	Christian Knauer \\ \textit{FernUniversit\"at Hagen} \and
	Stefan Langerman \\ \textit{McGill University} \and
	Pat Morin \\ \textit{Carleton University} \and
	Micha Sharir \\ \textit{Tel Aviv University} \and
	Michael Soss \\ \textit{Chemical Computing Group}}
\date{}

\begin{document}
\maketitle

\begin{abstract}
The maximum detour and spanning ratio of an embedded graph $G$ are
values that measure how well $G$ approximates Euclidean space and the
complete Euclidean graph, respectively.  In this paper we describe
$O(n\log n)$ time algorithms for computing the maximum detour and
spanning ratio of a planar polygonal path.  These algorithms solve
open problems posed in at least two previous works \cite{ekll01,ns01}.
We also generalize these algorithms to find a $O(n\log^2 n)$ time
algorithms for computing the maximum detour or spanning ratio of
planar trees and cycles.  For cycles, paths and trees embedded in 3
dimensions, we give subquadratic time algorithms for computing the
maximum detour and spanning ratio.
\end{abstract}

\section{Introduction}

Let $G=(V,E)$ be an embedded connected graph with $n$ vertices and $m$
edges.  Specifically, the vertex set $V$ consists of points in
$\mathbb{R}^d$, and $E$ consists of closed line segments whose
endpoints are in $V$. Let $s$ and $t$ be two points in $\cup
E$.\footnote{Here and thoughout, $\cup S$ is shorthand for
$\bigcup_{x\in S} x$.}  We denote by $\|st\|$ the Euclidean distance
between $s$ and $t$ and by $\|st\|_G$ the length of the shortest path
from $s$ to $t$ in $G$.  The detour between two points $s,t\in\cup E$
is defined as
\[
	D(G,s,t) = \frac{\|st\|_G}{\|st\|} \enspace .
\]
The \emph{maximum detour} $D(G)$ of $G$ is the maximum detour
over all pairs of points in $G$, i.e.,
\[
       D(G) = \max\{ D(G,s,t): {s,t\in\cup E, 
	s\neq t} \} \enspace .
\]
The \emph{spanning ratio} or \emph{stretch factor} $S(G)$ of $G$ is
the maximum detour over all pairs of vertices of $G$, i.e.,
\[
  	S(G) = \max\{ D(G,s,t) : {s,t\in V,\, s\neq t}\} \enspace .
\]

The maximum detour and spanning ratio play important roles in the
analysis of online routing algorithms \cite{bm01,ik95} and the
construction of spanners \cite{e99}.  In the former case, the goal is
to find paths that minimize maximum detour.  In the latter, the goal
is to construct graphs with few edges that minimize the spanning
ratio.

\subsection{Related Work}

Recently, researchers have become interested in computing the maximum
detour and spanning ratio of embedded graphs.  In any dimension $d$,
the spanning ratio can be computed by computing the shortest paths in
$G$ between all pairs of vertices and using this to compute the detour
between each pair of vertices.  Similarly, the maximum detour can be
computed by computing the detour between each pair of edges.  The
approaches require at least $\Omega(n^2)$, respectively $\Omega(m^2)$
computation time.  Surprisingly, these are the best known results for
computing the exact maximum detour or spanning ratio.  Even if the
input graph $G$ is a path in $\mathbb{R}^2$, no subquadratic time
algorithms are known.

Narasimhan and Smid \cite{ns01} study the problem of approximating the
spanning ratio in a graph and give $O(n\log n)$ time algorithms that
can $(1-\epsilon)$-approximate the spanning ratio when $G$ is a path,
a cycle or a tree.  More generally, they show that, after $O(n\log n)$
preprocessing, the problem of approximating the spanning ratio can be
reduced to $O(n)$ approximate shortest path queries on $G$.  Their
results hold even for graphs embedded in $\mathbb{R}^d$.  The authors
also show that approximating the spanning ratio requires $\Omega(n\log
n)$ time in the algebraic decision tree model of computation.

Ebbers-Baumann \etal\ \cite{ekll01} study the problem of approximating
the maximum detour of a polygonal path and give an
$O(\frac{n}{\epsilon}\log n)$ time algorithm that finds a
$(1-\epsilon)$-approximation to the maximum detour.  Their algorithm
also generalizes to planar cycles, with the same running time.

\subsection{New Results}

In this paper we give randomized algorithms with $O(n\log n)$ expected
running time that compute the exact spanning ratio or maximum detour
of a polygonal path with $n$ vertices.  These are the first
subquadratic time algorithms for finding the exact spanning ratio or
maximum detour, and they solve open problems posed in at least two
papers \cite{ekll01,ns01}.  We show that these algorithms can be
extended to give $O(n\log^2 n)$ time algorithms for computing the
maximum detour and spanning ratio of planar cycles and trees.

We solve these problems by reducing the associated decision problem to
computing the upper envelope of a set of identical cones in
$\mathbb{R}^3$.  In the case of spanning ratio, the set of cones is
finite, and the upper envelope that we compute is actually an
additively-weighted Voronoi diagram of points in the plane.  In the
case of maximum detour, the set of cones is infinite, and corresponds
to computing the additively-weighted Voronoi diagram of line segments
in the plane, a diagram that seems not to have been considered
previously.  We then apply a general optimization technique of Chan
\cite{c99} to convert the decision algorithms into optimization
algorithms.

We also consider the problem of computing the maximum detour and
spanning ratio of 3-dimensional chains and show that this problem can
be solved in subquadratic time.  Using the same extensions as in the
planar case, this leads to subquadratic time algorithms for
3-dimensional trees and cycles.  We also show that it is unlikely that
an $O(n\log^c n)$ time algorithm exists for computing the maximum
detour of 3-dimensional chains, since this problem is at least as hard
as Hopcroft's problem.

Some of these results, namely those on planar paths, trees and cycles
have appeared in a preliminary form in a conference proceedings
\cite{lms02}.

The remainder of the paper is organized as follows: In \secref{cones}
we show how to reduce the decision problems to an upper envelope
computation.  In \secref{bats} we give an algorithm for computing the
upper envelope of a set of objects called bats that is required for
solving the maximum detour decision problem.  In \secref{finishing} we
describe how to use these decision algorithms to obtain optimization
algorithms.  In \secref{trees-cycles} we extend these algorithms to
trees and cycles.  Finally, in \secref{conclusions} we
summarize and conclude with open problems.


\section{A Problem on Cones}\seclabel{cones}

For a point $p\in\mathbb{R}^3$, denote by $p_z$ the $z$-coordinate of
$p$ and by $p_{xy}$, the projection of $p$ onto the $x,y$-plane.
Given a polygonal path $P$ in $\mathbb{R}^2$ whose vertices are
$v_1,\ldots,v_n$ we lift it to a polygonal path $P'$ in $\mathbb{R}^3$
by assigning to each point $p$ in $P$ a $z$-coordinate equal to its
distance along the path from $v_1$, i.e., for each point $p\in P$,
$P'$ has a point $p'$ such that $p_{xy}'=p$ and $p'_z=\|v_1p\|_P$.

In this way, we obtain a $z$-monotone polygonal path $P'$ with
vertices $u_1,\ldots,u_n$ such that for any two points $p',q'\in P'$,
the unique path between the corresponding points $p,q\in P$ has length
$\|pq\|_P=\|p'_z-q'_z\|$.

Consider the following construction.  At each vertex $u_i$ of $P'$ we
place the apex of a cone $c_i$ that points downwards with its axis of
rotation parallel to the $z$-axis and that spans an angle of
$2\arctan(1/d)$.  We call such a cone the $d$-cone of $u_i$.  Now, if
some cone $c_i$ contains some vertex $u_j$ then $D(P,v_i,v_j)\ge d$
(see \figref{cones}).  Conversely, if there exists a pair of vertices
$v_i$, $v_j$ such that $D(P,v_i,v_j)\ge d$, then either $c_i$ contains
$u_j$ or $c_j$ contains $u_{i}$.  Thus, the problem of determining
whether the spanning ratio of $P$ is greater than or equal to $d$ is
reducible to the problem of determining whether any cone $c_i$
contains any vertex $u_j$.

\begin{figure}
\centeripe{cone}
\caption{If $c_i$ contains $u_j$ then $D(P,v_i,v_j)\ge d$.}
\figlabel{cones}
\end{figure}

The \emph{upper envelope} of the cones $c_1,\ldots,c_n$ is the
bivariate function $f(x,y)=\max\{z:(x,y,z)\in c_i,\mbox{ for some
$i$}\}$.  From this definition it follows that $u_i$ is contained in
some cone if and only if $u_i$ does not appear on the upper envelope,
i.e., $f(u_i)\neq u_{i,z}$.  The upper envelope of identical and
identically oriented cones has been given at least two other names:
additively-weighted Voronoi diagram \cite{f87} and Johnson-Mehl
crystal growth diagram \cite{jm39}.  It is known that $f$ consists of
$O(n)$ pieces and can be computed in $O(n\log n)$ time using a sweep
line algorithm \cite{f87}.  Thus, the decision problem of determining
whether the spanning ratio of $P$ is at least $d$ can be solved in
$O(n\log n)$ time.

Next we turn to the problem of determining whether the maximum detour
of $G$ is at least $d$.  For this problem we use the same construction
except that we place a cone with its apex on \emph{every} point of
$P'$, not just the vertices.  The decision problem then reduces to the
question: Does every point on $P'$ appear on the upper envelope of
these (infinitely many) cones.  Of course, computationally, we do not
compute the upper envelope of infinitely many cones.  Instead, we
compute the upper envelope of $n$ \emph{bats}, where a \emph{bat} is
the convex hull of two cones with apexes on the endpoints of an edge
of $P$.  We call the edge that defines a bat the \emph{core} of the
bat.

Thus, for both maximum detour and spanning ratio, the associated
decision problem can be solved by computing the upper envelope of a
suitably chosen set of objects, either cones or bats. To the best of
our knowledge, no algorithm exists for computing the upper envelope of
a set of bats.  In the following section, we show that we can
nevertheless solve the decision problem.

\section{The Upper Envelope of a Set of Bats}\seclabel{bats}

We say that a point $p$ in the core of some bat is \emph{redundant} if
$p$ is contained in some other bat.  We say that the input is
\emph{redundant} if some bat contains a redundant point and the input
is \emph{non-redundant} otherwise.  It is clear that solving the
decision problem associated with detour is equivalent to determining
whether the input is redundant or non-redundant.  In the preliminary
version of this paper, we sketched a plane sweep algorithm for
answering this question \cite{lms02}.  Here, we present an alternative
solution.  We begin with a structural lemma due to Ebbers-Baumann
\etal\ \cite{ekll01}.

\begin{lem}[Ebbers-Baumann \etal]\lemlabel{vertex-edge}
For a planar polygonal chain $P$, the maximum detour is attained by a
vertex and some other point on $P$, i.e., $D(P)=D(P,v_i,q)$ where
$v_i$ is a vertex of $P$ and $q$ is some other point of $P$.
Furthermore, the segment $v_iq$ intersects $P$ only at its endpoints.
\end{lem}

The consequence of \lemref{vertex-edge} is that it allows us to use
the additively weighted Voronoi diagram of points rather than line
segments.  Suppose that $v_i$ appears after $q$ on $P$.  Then, for any
value of $d$ less than $D(P)$, the $d$-cone of $v'_i$ contains $q'$.
If $v_i$ appears before $q$ on $P$ then we can reverse the direction
of $P$ to obtain a path $\overleftarrow{P}$ and the same statement is
true in $\overleftarrow{P}$.  Therefore, we can solve our decision
problem by computing the upper-envelope of the $d$-cones defined by
the vertices of of $P$ (respectively, $\overleftarrow{P}$) and
checking whether $P'$ (respectively, $\overleftarrow{P}'$) is
completely above the upper envelope of the $d$-cones.  If both paths
are above their respective upper envelopes then the answer to the
decision problem is no, otherwise it is yes.

Computing the upper envelope of $n$ $d$-cones (i.e., the additively
weighted Voronoi diagram of a set of a $n$ points) can be done in
$O(n\log n)$ time using Fortune's algorithm \cite{f87}, but testing
whether $P'$ lies above this diagram is slightly more complicated.
The straightforward way to solve this problem would be to compute, for
each Voronoi cell, the portion of $P$ contained in that cell, and then
test if each of these segments intersects the $d$-cone that defines
the cell. Unfortunately, the number of intersections between $P$ and
the Voronoi cells may be $\Omega(n^2)$, so this approach leads to a
quadratic-time algorithm.

A more efficient algorithm is obtained by using the second part of
\lemref{vertex-edge}.  For each vertex $v_i$ of $P$, we only require
the portion of $P$ that intersects the Voronoi cell of $v_i$ and is
visible from $v_i$, call this $P_i$.  The
Combination~Lemma~\cite{gss89,sa95} says that in total, all $P_i$
consist of only $O(n)$ segments and can be computed in $O(n\log n)$
time using the red-blue merging technique \cite{sa95}.  Once the sets
$P_i$ are computed we need only test for each segment in $P_i$,
whether it intersects the $d$-cone of $v_i$. Since each of these steps
can be done in $O(n\log n)$ time, we obtain the following result.


\comment{ In this section we show how to compute the upper envelope of
a set of bats using a sweep line algorithm.  This algorithm is
essentially a modification of Fortune's algorithm for computing
additively-weighted Voronoi diagrams and Voronoi diagrams of line
segments \cite{f87}.

This is fortunate, since the upper envelope of $n$
bats can have $\Omega(n^2)$ complexity (see \figref{quadratic}), so any
approach that requires computing the upper envelope is doomed to have
quadratic running time in the worst case.

\begin{figure}
\centeripe{quad}
\caption{The upper envelope of $n$ bats can have $\Omega(n^2)$ complexity.}
\figlabel{quadratic}
\end{figure}

We describe an algorithm that takes as input a set of $n$ bats which
are the union of cones that span angles of $2\alpha\le\pi/2$ and
either reports that the input is redundant or correctly computes the
upper envelope of the input.  The algorithm sweeps a plane $P$ through
space and maintains, at all times, the intersection of $P$ with the
upper envelope $E$.  The plane $P$ is parallel to the $x$-axis and
forms an angle of $\pi-\alpha$ with the $x,y$-plane (see
\figref{sweep-plane}).  The reason we sweep with such a plane is that
no bat $b$ can contribute to $P\cap E$ until $P$ has swept over some
point in the core of $b$.

\begin{figure}
\centeripe{plane}
\caption{The sweep plane makes an angle of $\pi-\alpha$ with the $x,y$-plane.}
\figlabel{sweep-plane}
\end{figure}

To understand the structure of $P\cap E$, it is helpful to note that
the boundary of a bat consists of four pieces (see \figref{bat}): two
conic pieces and two planar pieces.  It is easy to verify that the
intersection $P\cap E$ is a weakly $x$-monotone curve consisting of
pieces of parabolas and lines.  Therefore, its pieces can be stored in
a balanced binary tree sorted by $x$-coordinate.  The intersection
$P\cap E$ consists of parabolic arcs (where $P$ intersects conic
pieces) and straight line segments (where $P$ intersects linear
pieces).

\begin{figure}
\IpeFit{1.5in}\centeripe{bat}
\caption{The boundary of a bat consists of two conic pieces and
	two planar pieces.}
\figlabel{bat}
\end{figure}

As $P$ sweeps through space, $P\cap E$ changes continuously.
Therefore, we store $P\cap E$ symbolically so that each arc and
segment is represented by its equation as a function of the position
of the plane $P$.  The progress of the sweep plane is controlled by a
priority queue $Q$.  Initially, $Q$ contains $2n$ events corresponding
to the times at which the sweep plane passes over each endpoint of the
core of a bat.

During the sweep, some arcs or segments of $P\cap E$ may disappear as
they become obscured by neighbouring arcs.  Since each arc and segment
is parameterized by the position of the plane $P$, it is a constant
time operation to determine the time (if any) that an arc will be
obscured by its neighbours. In the following discussion, when we
insert and delete arcs and segments from $P\cap E$, it is implicit
that we recompute the times at which arcs in the neighbourhood of the
inserted/deleted arc are obscured and insert these times into $Q$. For
further details refer to Fortune's original paper \cite{f87}.

During the sweep, we process three types of events:

\begin{enumerate}
\item \emph{$P$ sweeps over a point $p$ that is the first endpoint of
the core of some bat $b$.}  Refer to \figref{events}.a.  In this case,
we first check if $p$ is below $P\cap E$.  If so, then $p$ is
contained in the bat that intersects $P$ directly above $p$, so we
quit and report that the input is redundant.  Otherwise, we add four
objects to $P\cap E$.  These objects are two line segments
representing the intersection of $P$ with the two planar pieces of $b$
and two parabolic arcs representing the intersection of $P$ with the
cone whose apex is at $p$.

\begin{figure}
\begin{center}\begin{tabular}{cc}
\IpeFit{2.1in}\Ipe{event-2} & \IpeFit{1.9in}\Ipe{event-1} \\
(a) & (b) 
\end{tabular}\end{center}
\caption{Handling type~1 and type~2 events.}
\figlabel{events}
\end{figure}

\item \emph{$P$ sweeps over a point $p$ that is the last endpoint of
the core of some bat $b$.}  Refer to \figref{events}.b.  In this case,
we add two parabolic arcs to $P\cap E$ that correspond to the
intersection of $P$ with the cone whose apex is at $p$.

\item \emph{An arc or segment disappears from $P\cap E$.}  In this
case, we remove from $Q$ any events associated with the arc or
segment.  In the case of a segment, we also check if one endpoint of
the segment corresponds to a point in the core of a bat.  If so, then
that point is not part of $E$ so we can quit and report that the input
is redundant.
\end{enumerate}

To see that the above algorithm runs in $O(n\log n)$ time, we observe
that there are only $O(n)$ type~1 and 2 events.  Each of these can be
easily implemented in $O(\log n)$ time, and each such event adds
$O(1)$ arcs or segments to $P\cap E$.  Each type~3 event can also be
implemented in $O(\log n)$ time, and deletes one element from $P\cap
E$.  Therefore, there are only $O(n)$ type~3 events and the entire
algorithm runs in $O(n\log n)$ time.

To see that the algorithms is correct for non-redundant inputs we can
use arguments which are standard by now \cite{f87}.  In particular, we
can show that there is a direct correspondence between the events
processed by the algorithm and changes to the combinatorial structure
of $P\cap E$.  Suppose therefore that the input is redundant and let
$p$ be the first redundant point swept over by $P$.  Either $p$ is an
endpoint of a core or it is in the interior of a core.  In the former
case, it will be handled as a type~1 event while in the latter case it
will be handled as a type~3 event.

In either case, all input previously swept over by $P$ is
non-redundant, so the intersection $P\cap E$ has been correctly
computed.  If $p$ is an endpoint of a core it will then be processed
as a type~1 event and the algorithm will correctly detect that $p$ is
below $P\cap E$ and is therefore redundant.  If $p$ is in the interior
of a core it will be processed as a type~3 event and the algorithm
will correctly detect that the input is redundant.  Therefore, either
the algorithm correctly computes the upper envelope (if the input is
non-redundant) or correctly reports that the input is redundant.
}


\begin{lem}
There exists an algorithm requiring $O(n\log n)$ time and $O(n)$ space
that tests whether a set of $n$ bats is redundant or non-redundant.
\end{lem}

\section{Optimization}\seclabel{finishing}

So far we have given all the tools required for solving the decision
problems associated with finding the maximum detour and spanning ratio
of a path.  More specifically, we have solved the problem: Given a set
of segments (possibly points) in 3 space, does there exist a cone $c$
with angular radius $2\arctan(1/d)$, center of rotation parallel to
the $z$-axis and apex on one of the segments such that $c$ intersects
or contains another segment.  The optimization problem is that of
finding the largest value of $d$ for which such a cone exists.

To solve the optimization problem we apply the randomized reduction of
Chan \cite{c99}, which requires only that we (1)~be able to solve the
decision problem in $f(n)=\Omega(n^\epsilon)$ time, for some constant
$\epsilon>0$, and (2)~partition the problem into $r$ subproblems, each
of size at most $\alpha n$, $\alpha < 1$ such that the optimal
solution is the maximum of the solutions to the $r$ subproblems.  The
reduction works by considering the subproblems in random order and
recursively solving a subproblem only if its solution is larger than
the current maximum (which can be tested by the decision algorithm).
The resulting optimization algorithm has running time $O(f(n))$.

We have already shown how to do (1) in $f(n) = O(n\log n)$ time.  To
do (2), we simply note that we can partition our set of segments into
three sets $A$, $B$, and $C$, each of size rougly $n/3$.  The optimal
solution is then the maximum of the solutions to $A\cup B$, $B\cup C$
and $A\cup C$.  Since there are only $r=3$ subproblems and each has
size $\alpha n=\frac{2}{3}n$ we have satisfied the conditions required
to use Chan's optimization technique.

\begin{thm}
The maximum detour and spanning ratio of a planar path with $n$
vertices can be computed in $O(n\log n)$ expected time.
\end{thm}

\noindent\textbf{Remark:} Using the parametric search technique
\cite{m83}, it is possible to devise $O(n\log^c n)$ time deterministic
algorithms for computing the maximum detour or spanning ratio of a
planar path, for constant $c$.  However, it seems difficult to apply
parametric search to obtain an $O(n\log n)$ time algorithm.

\section{Planar Trees and Cycles}\seclabel{trees-cycles}

In this section we show how the tools developed for planar paths can
be used for solving problems on more complicated types of objects.  In
our discussions, we consider only the problem of computing the maximum
detour, although it is clear that the algorithms also apply to the
problem of computing the spanning ratio.

\subsection{Planar Trees}

Let $T$ be a tree embedded in the plane and assume $T$ is rooted at a
vertex $v$ such that $T\setminus\{v\}$ has no component with more than
$n/2$ vertices.  Such a vertex is easily found in linear time. Refer
to \figref{ab}.  We partition the children of $v$ into two sets $A$
and $B$.  Let $T_A$, respectively $T_B$, denote the tree induced by
$v$ and all vertices having ancestors in $A$, respectively $B$.  The
partition $A$, $B$ is chosen so that
$\frac{1}{4}n\le\|T_A\|,\|T_B\|\le\frac{3}{4}n$.  Since no descendent of
$v$ is the root of a subtree with size more than $\frac{n}{2}$, such a
partition can be found with a greedy algorithm.

\begin{figure}
\centeripe{ab}
\caption{Partitioning into subtrees $T_A$ and $T_B$.}
\figlabel{ab}
\end{figure}

We lift $T$ into a 3-dimensional tree $T'$ in the following way.  Each
point $p\in T_A$ is assigned a $z$-coordinate equal to $\|vp\|_T$.  Each
point $p\in T_B$ is assigned a $z$-coordinate equal to $-\|vp\|_T$.
This gives us a 3-dimensional tree $T'$ consisting of points $T_A'$
above the $x,y$ plane and points $T_B'$ below the $x,y$ plane.

This lifting has the property that for any point $p'\in T_A'$ and any
point $q'\in T_B'$ the distance between the corresponding points $p$
and $q$ in $T$ is equal to the difference in $z$-coordinates of $p'$
and $q'$, i.e., $\|pq\|_T=\|p_z'-q_z'\|$.  Furthermore, for two points
$p,q$ both in $T_A$ or both in $T_B$, $\|pq\|_T\ge\|p_z'-q_z'\|$.  It
follows that if we run the maximum detour algorithm of the previous
section on the tree $T'$, the algorithm will respond with the correct
answer $D(T)$ if there is a pair of points $p\in T_A$ and $q\in T_B$
such that $D(T)=D(T,p,q)$.  If this is not the case, then the
algorithm may report a value less than $D(T)$, but will never report a
larger value.

Therefore, we can compute $D(T)$ using a recursive algorithm.  We run
the maximum detour algorithm on the tree $T'$, make two recursive
calls on $T_A$ and $T_B$ and output the maximum of the three values
obtained.  To see that this algorithm correctly computes $D(T)$,
consider the pair $p,q\in T$ that maximizes $D(T,p,q)$.  If $p\in A$
and $q\in B$ (or vice versa), the correct value of $D(T,p,q)$ is found
when we run the maximum detour algorithm on $T'$.  Otherwise, $p,q\in
T_A$ or $p,q\in T_B$ and is correctly reported by one of the recursive
calls.

The running time of the above algorithm is given by the recurrence
\[ 
   T(n)=T(n-k+1)+T(k)+O(n\log n) \enspace , 
\]
with $n/4\le k\le 3n/4$, which solves to $O(n\log^2
n)$.

\begin{thm}
The maximum detour or spanning ratio of a planar tree with $n$
vertices can be computed in $O(n\log^2 n)$ expected time.
\end{thm}

\subsection{Planar Cycles}

To compute the maximum detour of a planar cycle, we use a similar
divide and conquer strategy.  However, before beginning, we must
take care of one small technicality.

\begin{lem}
Let $C$ be a planar cycle of length 1.  Then the maximum detour in $C$
is obtained by two points $p$ and $q$ such that
(a)~$\|pq\|_C = 1/2$ or (b)~$p$ is a vertex of $C$.
\end{lem}

The maximum detour over all pairs of points $p$ and $q$ such that
$\|pq\|_C=1/2$ can be computed in linear time by scanning two points
around $C$ while maintaining the invariant that their distance along
$C$ is always equal to $1/2$.  Therefore, we can concentrate on
computing the maximum detour over all points $p,q\in C$ where $p$ is
vertex of $C$.

We do this by solving a more general problem. Let $C$ be a planar
cycle of length 1 and let $s$, $t$, $u$ and $v$ be four points on $C$
that appear in counterclockwise order and such that $\|su\|_C =
\|tv\|_C = 1/2$. (See \figref{cycle}.) Let $C_{ab}$ denote the
polygonal chain obtained by traversing $C$ counterclockwise beginning
at $a$ and ending at $b$.

\begin{figure}
\centeripe{cycle}
\caption{Finding the maximum detour over all $p,q\in C_{st}\cup C_{uv}$.}
\figlabel{cycle}
\end{figure}

Suppose that we wish to compute the maximum detour between all points
$p$ and $q$ where $p,q\in C_{st}\cup C_{uv}$ and define $m$ as the
number of edges in $C_{st}\cup C_{uv}$.  We will show how to solve
this problem in $O(m\log^2 m)$ time.  To solve the original problem of
computing the maximum detour of $C$, we simply set $v=s$ so that
$t=u$.

Let $w$ be any point in $C_{st}$.  Then there is a unique point $x\in
C_{uv}$ such that $\|wx\|_C=1/2$.  There are four possible locations
for the points $p$ and $q$ that maximize the detour. These are
(1)~$p,q\in C_{wt}\cup C_{ux}$, (2)~$p,q\in C_{xv}\cup C_{sw}$,
(3)~$p,q\in C_{wt}\cup C_{xv}$, or (4)~$p,q\in C_{sw}\cup C_{ux}$.

Observe that in Case~1, we only consider a subchain of $C$ that has
length $1/2$, so that any shortest path in $C$ between two points of
$C_{wt}\cup C_{ux}$ remains in $C_{wx}$.  Therefore, we can resolve
Case~1 by lifting $C_{wt}$ and $C_{ux}$ into three space where the $z$
coordinate of any point in the lifting is it's distance, along $C$
from $t$.  Applying the algorithm for chains to this lifting
determines the maximum detour between any two points $p,q\in
C_{wt}\cup C_{ux}$ in $O(m\log m)$ time.  The same is true for case~2,
so it can also be resolved in $O(m\log m)$ time.

In Cases~3 and 4, we are left with problems of the same form as the
original problem, so we can recursively solve these.  Since we can
always find a point $w$ such that $C_{wt}\cup C_{xv}$ and $C_{sw}\cup
C_{ux}$ each contain at least $m/4$ vertices, the running time of the
resulting algorithm is given by
\[
 T(m) = T(m-k) + T(k) + O(m\log m) \enspace ,
\] 
with $m/4\le k\le 3m/4$, which solves to $O(m\log^2 m)$.

\begin{thm}\thmlabel{three}
The maximum detour or spanning ratio of a planar cycle with $n$
vertices can be computed in $O(n\log^2 n)$ time.
\end{thm}

\section{Three Dimensional Objects}

\ldots

Finally, we show that computing the maximum detour of a 3-d path is as
hard as Hopcroft's problem: Given $n$ lines $l_1,\ldots,l_n$ and $n$
points $p_1,\ldots,p_n$ in the the plane, determine if any line
contains any point. There is an abundance of evidence that suggests
that Hopcroft's problem has an $\Omega(n^{4/3})$ lower bound
\cite{e95}.  The best known upper bound in any reasonable model of
computation is $O(n^{4/3}2^{O(\log^* n)})$ \cite{m93}.

Refer to \figref{hopcroft} for what follows.  To reduce an instance of
Hopcroft's problem to that of computing the maximum detour of a 3-d
path, we will build a 3-d path that is self-intersecting, i.e., has
infinite maximum detour, if and only if the answer to Hopcroft's
problem is yes.  We begin by computing a bounding rectangle $R$ that
contains all input points and contains all points at which two input
lines intersect.  This can easily be done in $O(n\log n)$ time.

Next, we take the intersection of each $l_i$ with $R$ and lift it
orthogonally so that it lies in the plane $z=i$ to obtain a line
segment $l'_i$.  Then we transform each input point $p_i$ to a line
segment $p'_i$ that is parallel to the $z$-axis, has one endpoint on
the plane $z=0$, the other endpoint on the plane $z=n+1$ and whose
projection onto the $x,y$ plane is $p_i$.

\begin{figure}
\centeripe{hopcroft}
\caption{Hopcrofts's problem can be solved by computing the maximum
detour of a 3-d path. Red points and lines show the input to
Hopcroft's problem, the blue part of the chain corresponds to input
lines and the purple part of the chain corresponds to input
points.}\figlabel{hopcroft}
\end{figure}

This gives us a set of line segments such that the answer to
Hopcroft's problem is yes if and only if two of the line segments
intersect.  All that remains is to make a path that contains all these
segments without introducting any additional crossings.  To do this,
we connect each $l'_i$ to $l'_{i+1}$ using a path whose projection onto
the $x,y$ plane is contained in $R$ and that stays between the planes
$z=i$ and $z=i+1$.  It is clear that each such connection can be done
using $O(1)$ edges and that no two such connections intersect, nor do
they intersect any other line segments.

Next we sort $p'_1,\ldots,p'_n$ lexicographically by $x,y$ and make
these into a polygonal chain where $p'_i$ is connected to $p'_{i+1}$
with a line segment in the plane $z=0$ if $i$ is even and a line
segment in the plane $z=n+1$ if $i$ is odd.  This chain is clearly not
self-intersecting, since it's projection onto the $x,y$ plane is
monotone.

Finally, we connect the unconnected endpoint of $l'_n$ to the
unconnected endpoint of $p'_1$ with a path that stays between the
planes $z=n$ and $z=n+1$ and whose projection onto the $x,y$ plane is
contained in the boundary of $R$, except for one vertical segment that
joins the bottom edge of $R$ to $p_1$.  Again, this path uses $O(1)$
edges and does not introduce any additional crossings.  Therefore, the
resulting chain if self-intersecting, i.e., has infinite maximum
detour, if and only if the answer to Hopcroft's problem is yes.

\begin{thm}
An algorithm with running time $f(n)$ for computing the maximum detour
of 3-dimensional polygonal chain with $n$ vertices implies an $O(n\log
n+ f(n))$ time algorithm for Hopcroft's problem.
\end{thm}

\section{Conclusions}\seclabel{conclusions}

We have given $O(n\log n)$ time algorithms for computing the maximum
detour and spanning ratio of planar paths.  These algorithms lead to
an $O(n\log^2 n)$ time algorithms for computing the maximum detour and
spanning ratio of a planar trees and cycles.  In three dimensions, we
have given subquadratic algorithms for computing the maximum detour of
paths, trees and cycles.  Previously, no subquadratic time algorithms
were known for any of these problems.

The main idea in this paper is that of reducing maximum detour and
spanning ratio problems to problems on sets of cones and bats,
respectively.  The upper envelope of bats, i.e., the
additively-weighted Voronoi diagram of line segments, may have other
applications.  It would be interesting to find an output sensitive
algorithm for the case in which the input is redundant.  This seems
difficult to do with plane sweep, since it requires a 4th type of
event that occurs when a core reappears on the upper envelope.
Maintaining these events can take quadratic time even when the
complexity of the upper envelope is linear.

There are many open problems in this new area.  The most obvious is:
Which other classes of graphs admit subquadratic time algorithms for
computing the maximum detour or spanning ratio?


\bibliography{detour}
\bibliographystyle{plain}


\end{document}
