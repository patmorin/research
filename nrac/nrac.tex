\documentclass[lotsofwhite,charterfonts]{patmorin}
\usepackage{amsopn}
\input{pat}

\DeclareMathOperator{\ch}{ch}


\title{Nearly Right-Angle Crossing Graphs}
\author{Vida Dujmovi\'c 
	\and Joachim Gudmundsson
	\and Pat morin
	\and Thomas Wolle}

\begin{document}
\maketitle

\begin{abstract}
A plane graph $G$ is an $\alpha$ angle crossing ($\alpha$AC) graph if
every crossing pair of edges in $G$ intersect at an angle greater of at
least $\alpha$. 
Eades \etal\ studied $\pi/2$-angle crossing graphs, called
right angle crossing (RAC) graphs, and showed that any RAC graph with $n$
vertices has at most $4n-10$ edges and that there exists infinitely values
of $n$ for which there exists a RAC graph with $n$ vertices and $4n-10$
edges.
In the current work, we give upper and lower bounds for the number of edges
in $\alpha$AC graphs for all $0 < \alpha < \pi/2$.
\end{abstract}

\section{Introduction}

For $\alpha > \pi/3$, an $\alpha$AC embedding has no three edges that
mutually intersect since, otherwise, one of the pairs of edges must
intersect at an angle that is at most $\pi/3$.  Geometric graphs with no
three pairwise crossing edges are known as \emph{quasiplanar graphs}
\cite{aapps97}. Ackerman and Tardos \cite[Theorem~5]{atXX} have shown that
any quasiplanar graph on $n$ vertices has at most $6.5n - 20$ edges. 

For $\alpha > \pi/4$, an $\alpha$AC graph has no 4 pairwise crossing edges.
Ackerman has shown that any such graph has at most $36n - 72$ edges.  It
remains an open problem whether, for any $k\ge 5$, a graph with no
$k$-pairwise crossing edges has a linear number of edges.  The best known
upper bound of $O(n\log n)$ on the number of edges in such a graph is due
to Valtr \cite[Theorem~3]{v99}.

In this paper we prove upper and lower bounds on the number of edges in
$\alpha$AC graphs.  In \secref{uniform} we show that, for any $0< \alpha
<\pi/2$, the maximum number of edges in a $\alpha$AC graph is at most
$(\pi/\alpha)(3n-6)$.  In \secref{lower-bounds}, we give constructions that
essentially match this upper bound when $\alpha = \pi/2k-\epsilon$ or when
$\alpha = \pi/ 3k-\epsilon$, for any integer $k \ge 1$ and any $\epsilon >
0$.  Finally, in \secref{charging} we use a charging argument like that
used by Ackerman and Tardos to prove that, for $2\pi/5 < \alpha < \pi/2$,
the number of edges in a $\alpha$AC graph is bounded by $6n-12$.


\section{A Uniform Upper Bound}

\begin{thm}
Let $G$ be a $\alpha$AC graph with $n$ vertices, for some $0<\alpha<\pi/2$.
Then $G$ has at most $(\pi/\alpha)(3n-6)$ edges.
\end{thm}

\begin{proof}
Define the \emph{direction} of an edge $xy$ whose lower endpoint is $x$ (in
the case of a horizontal edge, take $x$ as the left endpoint) as the angle
between $\angle wxy$ where $w=x+(1,0)$.  The direction of an edge $xy$ is
therefore a real number in the interval $[0,\pi)$.  Now, take a random
rotation $G'$ of $G$ and partition the edges of $G'$ into groups
$G_1,\ldots,G_{r}$ where $r=\lceil\pi/\alpha\rceil$, and $G_i$ contains all
edges of $G'$ in the interval $[\alpha(i-1),\alpha i)$.

Note that no two edges of $G_i$ cross each other, so each $G_i$ is a planar
graph which therefore contains
at most $3n-6$ edges.  Furthermore, since $G'$ is a random rotation, the
expected number of edges in $G_{r}$ is $(\pi\bmod \alpha) |E(G)|$.  In
particular, there must exist some rotation $G'$ of $G$ such that $|E(G')|
\le (\pi\bmod \alpha) |E(G)|$.  Therefore,
\begin{equation}
   E(G) \le \lfloor \pi/\alpha \rfloor(3n-6) + (\pi\bmod\alpha)|E(G)| \enspace .
   \eqlabel{uniform}
\end{equation}
Rearranging \eqref{uniform} yields
\[  
  |E(G)| 
    \le  \frac{\lfloor \pi/\alpha \rfloor(3n-6)}{1-\pi\bmod\alpha} 
    = (\pi/\alpha)(3n-6) \enspace ,
\]
as required.
\end{proof}

\section{Lower Bounds}

\section{Charging Arguments}

In this section we derive upper bounds using charging arguments like those
used by Ackerman and Tardos \cite{atXX} and Ackerman \cite{atXX}.  Let $G$
be an $\alpha$AC graph.  We denote by $G'$ the planar graph obtained by
introducing a vertex at each point in which a pair of edges in $G$ crosses
(thereby subdividing) two edges of $G$.  

For a face $f$ of $G'$, denote by $|f|$ the number of steps taken while
traversing the boundary of $f$ is counterclockwise order so that, if an
edge is crossed twice during the traversal, then it contributes twice to
$|f|$.  Let $v(f)$ denote the number of steps of this traversal during
which an vertex of $G$ (as opposed to a vertex introduced in $G'$) is
encountered.  For each face $f$ of $G'$ define the \emph{initial charge} of
$f$ as
\[
    \ch(f) = |f| + v(f) - 4  \enspace .
\]
Ackerman and Tardos show, using two applications of Euler's formula, that
\[
    \sum_{f\in G'} \ch(f) = 4n-8 \enspace .
\]

As a warm-up, we offer a short proof that is almost as good as 
alternate proof of Eade's \etal\ 's upper bound.

\begin{thm}
A RAC graph with $n$ vertices has at most $4n-9$ edges.
\end{thm}

\begin{proof}
We claim that, for every face $f$ of $G'$, $\ch(f)\ge v(G)/2$.  To see
this, observe that the claim is certainly true if $|f| \ge 4$.  On the
other hand, if $|f|=3$ then, by the RAC property, $v(f) \ge 2$, so it is
also true in this case.

Therefore,
\[
 4n-8 = \sum_{f\in G'} \ch(f) \ge
 \sum_{f\in G'} v(f)/2 = \sum_{v\in G} \deg(v)/2 = |E(G)| \enspace ,
\]
which proves an upper bound of $4n-8$.  To improve the bound slightly,
observe that the outerface of $G'$ has at least 3 vertices of $G$, so its
charge is at least $3+v(f)-4 \ge v(f)/2+1/2$, for $v(f)\ge 3$, so we obtain
\[
4n-8-1/2 \ge |E(G)|
\]
so $|E(G)| \le 4n-9$ since $|E(G)|$ and $n$ are integers.
\end{proof}


Next, we prove an upper bound for $\alpha > 2\pi/5$ that improves on the
$6.5n-20$ upper bound given by Ackerman and Tardos.

\begin{thm}
Let $G$ be an $\alpha$AC graph with $n$ vertices, for $\alpha > 2\pi/5$.
Then $G$ has at most $6n-12$ edges.
\end{thm}

\begin{proof}
We will redistribute the charge in the $G'$ to obtain a new charge $\ch'$
such that $\ch'(f) \ge v(f)/3$ for every face $f$ of $G$.  In this way, we
get
\[
 4n-8 = \sum_{f\in G'} \ch(f) 
   = \sum_{f\in G'} \ch'(f) \ge
 \sum_{f\in G'} v(f)/3 = \sum_{v\in G} \deg(v)/3 = 2|E(G)|/3 \enspace ,
\]
which we rewrite to get $|E(G)| \le 6n-12$.

The charge $\ch'(f)$ is obtained as follows.  Let $f$ be any face of $G'$
such that $|f|=3$ and $v(f)=1$.   That is, $f$ is a triangle formed by two
edges $e_1$ and $e_2$ that meet at a vertex $x$ of $G$ and an edge $e$ that
crosses $e_1$ and $e_2$.

Imagine walking along the bisector of $e_1$ and $e_2$ (starting in the
interior of $f$) until reaching a face $f'$ such that $|v(f')| \ge 1$.  To
see why this happens observe that, if $|f'|\ge 5$ then the $\alpha$AC
condition implies that $v(f)\ge 1$.  Similarly, since no three edges of $G$
are pairwise crossing, if we encounter a face $f'$ with $|f'|=3$ then
$v(f')\ge 1$.  Thus, while walking we encounter a sequence of faces $f'$
such that $|f'|=4$ and $v(f')=0$.  This process must end since it will
eventually encounter a face that contains one of the endpoints of $e_1$ or
$e_2$.


that are quadrilaterals

From $f'$ we subtract a charge of
$1/3$ and add this to the (currently 0) charge of $f$.  We say that $f'$
discharges through the edge

  By doing this for
each such face $f$, we obtain the new charge $\ch'$.

Clearly $\sum_{f}\ch(f) = \sum_{f}\ch'(f)$.  What remains is to show that
$\ch'(f) \ge v(f)/3$ for every face $f$.  By construction, this is true if 



\in V(G)$.
\end{proof}




\end{document}
