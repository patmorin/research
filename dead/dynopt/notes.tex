\documentclass[charterfonts,lotsofwhite]{patmorin}
 
%\usepackage{amsthm}

\newcommand{\centeripe}[1]{\begin{center}\Ipe{#1}\end{center}}
\newcommand{\comment}[1]{}

\newcommand{\centerpsfig}[1]{\centerline{\psfig{#1}}}

\newcommand{\seclabel}[1]{\label{sec:#1}}
\newcommand{\Secref}[1]{Section~\ref{sec:#1}}
\newcommand{\secref}[1]{\mbox{Section~\ref{sec:#1}}}

\newcommand{\alglabel}[1]{\label{alg:#1}}
\newcommand{\Algref}[1]{Algorithm~\ref{alg:#1}}
\newcommand{\algref}[1]{\mbox{Algorithm~\ref{alg:#1}}}

\newcommand{\applabel}[1]{\label{app:#1}}
\newcommand{\Appref}[1]{Appendix~\ref{app:#1}}
\newcommand{\appref}[1]{\mbox{Appendix~\ref{app:#1}}}

\newcommand{\tablabel}[1]{\label{tab:#1}}
\newcommand{\Tabref}[1]{Table~\ref{tab:#1}}
\newcommand{\tabref}[1]{Table~\ref{tab:#1}}

\newcommand{\figlabel}[1]{\label{fig:#1}}
\newcommand{\Figref}[1]{Figure~\ref{fig:#1}}
\newcommand{\figref}[1]{\mbox{Figure~\ref{fig:#1}}}

\newcommand{\eqlabel}[1]{\label{eq:#1}}
\newcommand{\eqref}[1]{(\ref{eq:#1})}

\newtheorem{thm}{Theorem}{\bfseries}{\itshape}
\newcommand{\thmlabel}[1]{\label{thm:#1}}
\newcommand{\thmref}[1]{Theorem~\ref{thm:#1}}

\newtheorem{lem}{Lemma}{\bfseries}{\itshape}
\newcommand{\lemlabel}[1]{\label{lem:#1}}
\newcommand{\lemref}[1]{Lemma~\ref{lem:#1}}

\newtheorem{cor}{Corollary}{\bfseries}{\itshape}
\newcommand{\corlabel}[1]{\label{cor:#1}}
\newcommand{\corref}[1]{Corollary~\ref{cor:#1}}

\newtheorem{obs}{Observation}{\bfseries}{\itshape}
\newcommand{\obslabel}[1]{\label{obs:#1}}
\newcommand{\obsref}[1]{Observation~\ref{obs:#1}}

\newtheorem{assumption}{Assumption}{\bfseries}{\rm}
\newenvironment{ass}{\begin{assumption}\rm}{\end{assumption}}
\newcommand{\asslabel}[1]{\label{ass:#1}}
\newcommand{\assref}[1]{Assumption~\ref{ass:#1}}

\newcommand{\proclabel}[1]{\label{alg:#1}}
\newcommand{\procref}[1]{Procedure~\ref{alg:#1}}

\newtheorem{rem}{Remark}
\newtheorem{op}{Open Problem}

\newcommand{\etal}{\emph{et al}}

\newcommand{\voronoi}{Vorono\u\i}
\newcommand{\ceil}[1]{\left\lceil #1 \right\rceil}
\newcommand{\floor}[1]{\left\lfloor #1 \right\rfloor}



\title{\MakeUppercase{On the Splay Tree Dynamic Optimality Conjecture}}
\author{Pat Morin}
\date{}

\begin{document}
\maketitle

\begin{abstract}
This document contains some ideas about how to develop data structures
that achieve Wilbur's lower bound for accessing a sequence.
\end{abstract}


\section{Wilbur's Lower Bound}

Let $T$ be a binary search tree containing the elements $1,\ldots,n$.
Let $A=a_1,\ldots,a_m$ be a sequence consisting of elements in
$\{1,\ldots,n\}$.  Let $T_i$ denote the binary search tree obtained by
inserting $a_i,a_{i-1},\ldots,a_1$.  Let $C_i(x)$ be cost of searching
for $x$ in $T_i$ and let $W_i(x)$ denote the number of alternations in
the search path for $x$ in $T_i$.

\begin{thm}[Wilbur's Lower Bound]
The cost of accessing the sequence $A$ using a binary search tree is
$\Omega(\sum_{i=1}^{m} W_{i-1}(a_i))$ independent of the update
algorihtm used on the search tree.
\end{thm}

\section{Decomposing Splay Trees}


\begin{lem}[Sequential Access Lemma]\lemlabel{sequential}
The cost of accessing the sequence $0,\ldots,n-1$ on a splay tree is $O(n)$.
\end{lem}

We will show how the Sequential Access Lemma can be used to show that
splay trees access the "shuffle sequence" sequence $a_1,\ldots,a_n$
where $a_i=\floor{i/n}+\sqrt{n}(i\bmod \sqrt{n})$ in $O(n)$ time.
Consider the following two-level data structure.  Our level-1 data
structure is a splay tree of size $\sqrt{n}$.  This splay tree
contains $\sqrt{n}$ level-2 splay trees each of which contains a
consecutive group of $\sqrt{n}$ elements.

Observe that the shuffle sequence iterates $\sqrt{n}$ time accessing each group
in order in order.  Therefore, by \lemref{sequential}, the cost of accessing
the groups in the level-1 splay tree is $\sqrt{n}\times O(\sqrt{n})=O(n)$.
Furthermore, within each group the elements are accessed in order so the cost
of accesses within each level-2 splay tree is $O(\sqrt{n})$ for a total cost of
$O(n)$ since there are $O(\sqrt{n})$ groups.

Now, if we delve more deeply into the internals of this two-level data
structure, we see that we can slightly reorder the splay tree operations
without affecting their running times.

\begin{enumerate}
\item Search for the requested group in the level-1 data structure.
\item Search for the requested element in the level-2 data structure for that group.
\item Splay in the level-2 data structure.
\item Splay in the level-1 data structure.
\end{enumerate}

Now, observe that the splay in Step~3 brings the requested element to the root
of its level-2 data structure and the splay in Step~4 brings the requested
group to the root of the level-1 data structure.  However, the concatenation of
Steps~3 and 4 is simply a splay of the requested element in the entire data
structure.  In other words, this two level data structure is simply a splay
tree!

\section{Using Treaps to Match Wilbur's Lower Bound}

To match Wilbur's lower bound we can try using treaps to maintain the
trees $T_1,\ldots,T_m$.  Suppose we have the tree $T_{i-1}$ and we
want to access $a_i$.  We can certainly do this using $C_{i-1}(a_i)$
comparisons since we have the tree $T_{i-1}$ available to us (can we
do this faster).  Now, we want to update $T_{i-1}$ to get the tree
$T_i$.  Normally, we would do this by doing rotations about node $a_i$
until it becomes the root, and this would cost $O(C_{i-1}(a_i))$.
However, we can actually do all these rotations by performing only
$O(W_{i-1}(a_i))$ pointer updates.  To see this, observe that rotating
a node up a string of right (or left children) only affects the
pointers at the ends of the string:

\begin{center}
\begin{minipage}{2.5in}
\begin{verbatim}
                      x
   /                 / \
  a                 a   Y
 / \               / \
A   b             A   b
   / \               / \
  B   c      =>     B   c
     / \               / \
    C   d             C   d
       / \               / \
      D   x             D   X
         / \
        X   Y
\end{verbatim}
\end{minipage}
\end{center}

This seems to imply that, with the necessary data structures, this
updating could be done in $O(W_{i-1}(a_i)\log\log n)$ time, or maybe
faster.  Thus, all that would be needed is a faster way to search in
$T_{i-1}$ to get a data structure comes close to Wilbur's lower bound.

\section{Using a Potential Function that Compares the Wilbur Tree
$\bf T_i$ and the Splay Tree}



\end{document}

