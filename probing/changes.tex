\documentclass[lotsofwhite]{patmorin}

\title{\MakeUppercase{Revision Document for}\\
	\MakeUppercase{Testing the Quality of Manufactured Disks and Balls}}
\author{}
\date{}

\newenvironment{comment}{\noindent\bf}{}
\newenvironment{response}{\noindent}{}

\begin{document}
\maketitle

\noindent
Reviewers comments are in bold, responses are in a normal type-face.

\section*{Reviewer 1 (bose-rev1.txt)}
%%%%%%%%%%%%%%%%%%%%%%%%%%%%%%%%%%%%%%%%%%%%%%%%%%%%%%%%%%%%%%%%%%%%%%
\begin{comment}
Abstract: Give some, possibly informal, definition for "quality".
\end{comment}

\begin{response}
Such a definition was added.
\end{response}

%%%%%%%%%%%%%%%%%%%%%%%%%%%%%%%%%%%%%%%%%%%%%%%%%%%%%%%%%%%%%%%%%%%%%%
\begin{comment}
Abstract/Introduction: Make your results more explicit, also in
comparing them to previous results. For example, you tell in the
conclusion only, that your runtimes are the same as in Mehlhorn et
al. Also, the ball testing algorithm is not mentioned at all in the
abstract (and only a single sentence in the introduction)!
\end{comment}

\begin{response}
The results for balls are now included in the abstract.  Both results
are in the introduction and it is explicitly stated that we match
Mehlhorn et al.'s running time.
\end{response}

%%%%%%%%%%%%%%%%%%%%%%%%%%%%%%%%%%%%%%%%%%%%%%%%%%%%%%%%%%%%%%%%%%%%%%
\begin{comment}
Introduction: For those people who are not so much into computational
metrology, it could help to provide a bit more information about the
finger probe model: why is this a good/realistic model? And how
exactly is the ray shooting defined? A picture might help; it took me
some time to figure out where these rays are supposed to be shot from,
in particular for Procedure 2 on page 8.
\end{comment}

\begin{response}
A description of laser rangefinders and coordinate measuring machines
was added.  A figure was also added.
\end{response}


%%%%%%%%%%%%%%%%%%%%%%%%%%%%%%%%%%%%%%%%%%%%%%%%%%%%%%%%%%%%%%%%%%%%%%
\begin{comment}
The bibliography is outdated/incomplete: for the following papers,
there are journal versions (some of them out for long time)

\begin{itemize}
\item Agarwal et. al[1], DCG 21:373-388 (1999). 
\item de Berg et al.[4] Computer-Aided Design 30:267--275 (1998).
\item Ebara[6], IEICE Transactions on Information Systems, 1992.
\item Garcia-Lopez et al.[7], DCG 20:389-402 (1998), which even seems to, 
  at least partly, answer Problem 1 mentioned in Conclusions. 
\item Schoemer et al.[12], Algorithmica 27:170-186 (2000).
\item Swanson[14], CGTA vol. 5(4) from 1995. 
\end{itemize}

The following papers and results therein deserve mentioning:

\begin{itemize}
\item Agarwal et al., Approximation and exact algorithms for minimum-width
annuli and shells, SOCG 1999.
\item P. Ramos, Computing roundness is easy if the set is almost round,
SOCG 1999.
\item Kumar and Sivakumar, Roundness estimation via random sampling, SODA
1999.
\item Chan, Approximating the diameter, width, smallest enclosing
cylinder, and minimum-width annulus, SOCG 2000.
\item Agarwal et al., Exact and approximation algorithms for minimum-width
cylindrical shells, SODA 2000.
\end{itemize}

\end{comment}

\begin{response}
The list of publications was updated.  The open problems in the
conclusions were removed.
\end{response}


%%%%%%%%%%%%%%%%%%%%%%%%%%%%%%%%%%%%%%%%%%%%%%%%%%%%%%%%%%%%%%%%%%%%%%
\begin{comment}
Furthermore, when talking about enclosing annuli, it should be stated
that there is a difference between minimum width and minimum area.
Since the former problem seems to be considerably harder, one might
also ask, whether the minimum area enclosing annulus could not serve
as a reasonable estimate for roundness for practical
purposes. (Minimum area enclosing annulus is an LP-type problem and
can thus be solved in linear time in any fixed dimension, see
Gaertner, Exact Arithmetic at Low Cost, CGTA vol.13(1999), for an
exact implementation.)
\end{comment}

\begin{response}
The roundness measure used in our paper is neither minimum-width
annulus nor minimum-area annulus. Both of these measure are
inappropriate for metrology since they make no distinction between a
perfect circle with radius 1 and a perfect circle with radius 2.  As
such, we cite these papers but state that they are computing different
roundness measures than the one used in our paper.
\end{response}

%%%%%%%%%%%%%%%%%%%%%%%%%%%%%%%%%%%%%%%%%%%%%%%%%%%%%%%%%%%%%%%%%%%%%%
\begin{comment}
p2/l 22: teh $\rightarrow$ the
\end{comment}

\begin{response}
This was corrected.
\end{response}

%%%%%%%%%%%%%%%%%%%%%%%%%%%%%%%%%%%%%%%%%%%%%%%%%%%%%%%%%%%%%%%%%%%%%%
\begin{comment}
p3: You should be more precise when using the term center. Initially,
you are talking about "a center", i.e., there could be several such
points. At some point you switch to "the center", implying there is
exactly one. I assume, for the existence of a center, compactness is
sufficient, but what else is needed to make it unique?
\end{comment}

\begin{response}
All references to ``the center'' were relplaced with references to ``a
center''.  It is not clear (and not proven anywhere) that there is a
unique center.
\end{response}

%%%%%%%%%%%%%%%%%%%%%%%%%%%%%%%%%%%%%%%%%%%%%%%%%%%%%%%%%%%%%%%%%%%%%%
\begin{comment}
p3: Fix the (in)equality in Assumption 2/Observation 1. May $1-\delta$
be equal to $\alpha$ or not?
\end{comment}

\begin{response}
The inequality $1-\delta>\alpha$ was removed, since it is not needed.
This does not affect any results, since they only require a
lower-bound on $\alpha$.
\end{response}


%%%%%%%%%%%%%%%%%%%%%%%%%%%%%%%%%%%%%%%%%%%%%%%%%%%%%%%%%%%%%%%%%%%%%%
\begin{comment}
p4: In Procedure 1, you can easily safe a line: exchange line 4 and 5
and remove line 13. In fact, since Delta is not used anywhere except
for defining r and R, you could also remove it completely.
\end{comment}

\begin{response}
The procedure was modified and references to line numbers in the
procedure were updated.
\end{response}


%%%%%%%%%%%%%%%%%%%%%%%%%%%%%%%%%%%%%%%%%%%%%%%%%%%%%%%%%%%%%%%%%%%%%%
\begin{comment}
p4: I do not see why the inequalities in (6) and (7) hold. It seems
that these values can become arbitrarily large, if delta and alpha are
sufficiently small. Moreover, a factor of pi got lost in (6) compared
to (9) and (16). 
\end{comment}

\begin{response}
The reviewer is correct, a $\pi$ factor in the numerator and an
$\alpha$ factor in the denominator got lost while trying to simplify
the expression.  This has been corrected.
\end{response}


%%%%%%%%%%%%%%%%%%%%%%%%%%%%%%%%%%%%%%%%%%%%%%%%%%%%%%%%%%%%%%%%%%%%%%
\begin{comment}
general: Please include into the statement of your Lemmata which of
Assumption 1 or 2 are needed.
\end{comment}

\begin{response}
Assumptions 1 and 2 were added to the statements of all lemmata.
\end{response}

%%%%%%%%%%%%%%%%%%%%%%%%%%%%%%%%%%%%%%%%%%%%%%%%%%%%%%%%%%%%%%%%%%%%%%
\begin{comment}
general: I would prefer the squareroot symbol over $^{1/2}$, since
matching the corresponding parentheses is more difficult, in
particular if there are several parentheses around. But maybe this is
a matter of taste.
\end{comment}

\begin{response}
The formulas for $f(n)$, $f'(n)$ and $f''(n)$ were changed to use the
square root symbol, but the formulas in the proof of Theorem~4 were
kept as is, since they become much more cluttered if we use the square
root symbol.
\end{response}


%%%%%%%%%%%%%%%%%%%%%%%%%%%%%%%%%%%%%%%%%%%%%%%%%%%%%%%%%%%%%%%%%%%%%%
\begin{comment}
p5/l -2: an bad $\rightarrow$ a bad
\end{comment}

\begin{response}
Fixed.
\end{response}


%%%%%%%%%%%%%%%%%%%%%%%%%%%%%%%%%%%%%%%%%%%%%%%%%%%%%%%%%%%%%%%%%%%%%%
\begin{comment}
p7/Lemma 4: Why do you write $O(1/n)$ instead of $f'(n)$?
\end{comment}

\begin{response}
$O(1/n)$ was changed to $f'(n)$.
\end{response}

%%%%%%%%%%%%%%%%%%%%%%%%%%%%%%%%%%%%%%%%%%%%%%%%%%%%%%%%%%%%%%%%%%%%%%
\begin{comment}
p11/ Section 4.2.1: Why don't you use parallel slices of the same
height to subdivide the sphere? (These would have the same surface
area.) 
\end{comment}

\begin{response}
We're not sure what this comment means.  We do the subdivision based
on spherical (polar) coordinates, because were are interested in
bounding the angles between the centers of pieces.  Defining the
pieces by angles seems like the easiest way to do this.
\end{response}

%%%%%%%%%%%%%%%%%%%%%%%%%%%%%%%%%%%%%%%%%%%%%%%%%%%%%%%%%%%%%%%%%%%%%%
\begin{comment}
p13/Theorem 5: "using the algorithm of Duncan et al." Can you be a bit
more specific? There are several algorithms described in that paper,
some of which give approximate solutions, or with high probability. It
might also be helpful to mention this algorithm and its
runtime/requirements in the introduction already.
\end{comment}

\begin{response}
The theorem is now cited more specifically.
\end{response}


%%%%%%%%%%%%%%%%%%%%%%%%%%%%%%%%%%%%%%%%%%%%%%%%%%%%%%%%%%%%%%%%%%%%%%
\begin{comment}
p14/Theorem 6: Please argue why $qual(I') = -\Psi$
\end{comment}

\begin{response}
A proof of this claim was added.
\end{response}


%%%%%%%%%%%%%%%%%%%%%%%%%%%%%%%%%%%%%%%%%%%%%%%%%%%%%%%%%%%%%%%%%%%%%%
\begin{comment}
p15/l 1: circle $\rightarrow$ sphere
\end{comment}

\begin{response}
Fixed.
\end{response}


\section*{Reviewer 2 (bose-rev2.txt)}

\begin{comment}
Section 1: page 2, 3rd para: Certainly, the assumptions are weaker than 
those made by Mehlhorn et al., but it is certainly also debatable, whether 
they are {\em much} weaker. I have no evidence, that the weaker results will 
lead to tremendous improvements in practice. By the way, what happens, if 
the algorithm is applied to parts that do not satisfy the assumptions?
\end{comment}

\begin{response}
Mehlhorn et al.\ agree that their convex assumption is overly
restrictive and suggest replacing this assumption with a restriction
on curvature as an open problem.
\end{response}

\begin{comment}
Section 2: I suggest not to use $O$ to denote the origin in order to
avoid confusion with Big-Oh notation. $o$ is better, but suboptimal,
for similar reasons. May be $o_+$ or ...
\end{comment}

\begin{response}
The notation was changed to make $o_+$ denote the origin
\end{response}

\begin{comment}
Assumption 2: Is $1 - \delta = \alpha$ allowed, c.f. proof of
Observation 1.  ``Observations'' are rarely proved, better call it
``Lemma'', if you provide a proof.
\end{comment}

\begin{response}
This was addressed in response to reviewer 1.
\end{response}

\begin{comment}
Section 3: $\pi$ is missing in (6) !
\end{comment}

\begin{response}
This was  addressed in response to reviewer 1.
\end{response}


\begin{comment}
Moreover, I find having $(1+\delta)^2$ instead of $1 +2\delta +
\delta^2$ more intuitive in (6). Similarly I would prefer to have
$(1+3\delta)\pi/n (...)$ in (17).
\end{comment}

\begin{response}
This was  addressed in response to reviewer 1.
\end{response}

\begin{comment}
As for notation, $f'(n)$ might be misleading, cause it looks like 
derivation.
\end{comment}

\begin{response}
This was changed to $g(n)$.
\end{response}

\begin{comment}
roof of Lemma 3: By Lemma 2, not by Lemma 1.
\end{comment}

\begin{response}
Fixed.
\end{response}

\begin{comment}
proof of Lemma 4, (23): $p$ varies over all points in 2D space, not over all 
points in $I$.
\end{comment}

\begin{response}
This was fixed to $p\in\mathbb{R}^2$, though it is clear that the
quality is maximized when $p\in I$.
\end{response}

\begin{comment}
proof of Lemma 4: $O(1/n)$ is used instead of $f'(n)$.  reformat the
proof please, since this is not two-column format, there is no need
for line breaks and there is also no need for separate numbering.
\end{comment}

\begin{response}
The $O(1/n)$ was fixed in response to reviewer 1.  The formulas were
also reformatted.
\end{response}

\begin{comment}
page 10, case 2: $z = z(p_1)$ instead of $y = y(p_1)$.
\end{comment}

\begin{response}
This was corrected.
\end{response}

\begin{comment}
page 10, case 2, line 2: ``intersection the plane'' there seems to be something
missing.
\end{comment}

\begin{response}
Was changed to ``intersection of the plane.''
\end{response}

\begin{comment}
page 10, case 2: I don't see a dashed line in Figure (3) in my copy, sorry.\end{comment}

\begin{response}
The figure was modified to correct this.
\end{response}

\begin{comment}
page 10: Suddenly, $\delta \leq 1/30$; It used to be $1/21$.
\end{comment}

\begin{response}
The value $1/30$ is correct.  The value $1/29$ was leftover from
merging two papers. This has been corrected.
\end{response}

\begin{comment}
Section 4: page 13: There seems to be something missing at the
beginning of 4.3 !
\end{comment}

\begin{response}
The sentence at the beginning fo section 4.3 was finished.
\end{response}

\begin{comment}
Section 5: Why $\alpha 8 \psi$ and not $8\alpha\psi$, which is more natural?
\end{comment}

\begin{response}
This was changed to $8\alpha\Psi$.
\end{response}

\begin{comment}
page 14: the length OF $bd(I)$ between $p_1$ and $p_2$.
\end{comment}

\begin{response}
This was corrected.
\end{response}

\begin{comment}
end of Section 5 on page 15: What is \verb+$w(\Psi)$+ ?
\end{comment}

\begin{response}
Actually, this is $\omega(\Psi)$, a function of $\Psi$ that is
asymptotically larger than $\Psi$.
\end{response}


\begin{comment}
Section 6: problem 2 is less relevant, since testing cylinders is not
discussed here. So is this a left-over from pervious versions or
intentionally added?
\end{comment}

\begin{response}
The section on open problems was removed.
\end{response}



\begin{comment}
References: Shouldn't [4] be [2]?
\end{comment}

\begin{response}
Yes, this was a result of the way BibTeX alphabetizes de~Berg.
\end{response}

\begin{comment}
Page numbers for [7]?
\end{comment}

\begin{response}
This reference was updated to the journal version in response to the
first reviewer's comments.
\end{response}


\section*{Reviewer 3 (bose-rev3.txt)}

\begin{comment}
The abstract is lousy. State what the contribution is, i.e. what is 
new with respect of state of the art. 3D is not mentioned at all here.
\end{comment}

\begin{response}
The abstract was expanded and the contribution is stated more clearly.
\end{response}

\begin{comment}
Also in the text, be very explicit about what is new, and how much it 
differs from known stuff. The paper has a contribution, it is not
shocking, but be explicit about it.
\end{comment}

\begin{response}
The original contribution is now explained more clearly in the
introduction (page 2, second paragraph from the bottom).
\end{response}


\end{document}