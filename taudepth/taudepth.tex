\documentclass[lotsofwhite]{patmorin}
\usepackage{amsfonts,graphicx}

 
%\usepackage{amsthm}

\newcommand{\centeripe}[1]{\begin{center}\Ipe{#1}\end{center}}
\newcommand{\comment}[1]{}

\newcommand{\centerpsfig}[1]{\centerline{\psfig{#1}}}

\newcommand{\seclabel}[1]{\label{sec:#1}}
\newcommand{\Secref}[1]{Section~\ref{sec:#1}}
\newcommand{\secref}[1]{\mbox{Section~\ref{sec:#1}}}

\newcommand{\alglabel}[1]{\label{alg:#1}}
\newcommand{\Algref}[1]{Algorithm~\ref{alg:#1}}
\newcommand{\algref}[1]{\mbox{Algorithm~\ref{alg:#1}}}

\newcommand{\applabel}[1]{\label{app:#1}}
\newcommand{\Appref}[1]{Appendix~\ref{app:#1}}
\newcommand{\appref}[1]{\mbox{Appendix~\ref{app:#1}}}

\newcommand{\tablabel}[1]{\label{tab:#1}}
\newcommand{\Tabref}[1]{Table~\ref{tab:#1}}
\newcommand{\tabref}[1]{Table~\ref{tab:#1}}

\newcommand{\figlabel}[1]{\label{fig:#1}}
\newcommand{\Figref}[1]{Figure~\ref{fig:#1}}
\newcommand{\figref}[1]{\mbox{Figure~\ref{fig:#1}}}

\newcommand{\eqlabel}[1]{\label{eq:#1}}
\newcommand{\eqref}[1]{(\ref{eq:#1})}

\newtheorem{thm}{Theorem}{\bfseries}{\itshape}
\newcommand{\thmlabel}[1]{\label{thm:#1}}
\newcommand{\thmref}[1]{Theorem~\ref{thm:#1}}

\newtheorem{lem}{Lemma}{\bfseries}{\itshape}
\newcommand{\lemlabel}[1]{\label{lem:#1}}
\newcommand{\lemref}[1]{Lemma~\ref{lem:#1}}

\newtheorem{cor}{Corollary}{\bfseries}{\itshape}
\newcommand{\corlabel}[1]{\label{cor:#1}}
\newcommand{\corref}[1]{Corollary~\ref{cor:#1}}

\newtheorem{obs}{Observation}{\bfseries}{\itshape}
\newcommand{\obslabel}[1]{\label{obs:#1}}
\newcommand{\obsref}[1]{Observation~\ref{obs:#1}}

\newtheorem{assumption}{Assumption}{\bfseries}{\rm}
\newenvironment{ass}{\begin{assumption}\rm}{\end{assumption}}
\newcommand{\asslabel}[1]{\label{ass:#1}}
\newcommand{\assref}[1]{Assumption~\ref{ass:#1}}

\newcommand{\proclabel}[1]{\label{alg:#1}}
\newcommand{\procref}[1]{Procedure~\ref{alg:#1}}

\newtheorem{rem}{Remark}
\newtheorem{op}{Open Problem}

\newcommand{\etal}{\emph{et al}}

\newcommand{\voronoi}{Vorono\u\i}
\newcommand{\ceil}[1]{\left\lceil #1 \right\rceil}
\newcommand{\floor}[1]{\left\lfloor #1 \right\rfloor}



\title{\MakeUppercase{Fast Algorithms for $\tau$-Depth}}
\author{The Radcliffe Data Depth Workshop}
\date{}

\newcommand{\taudepth}{$\tau$-depth}
\newcommand{\Taudepth}{The $\tau$-depth}
\newcommand{\tdepth}{d_\tau}
\newcommand{\tukdepth}{d_{\frac{1}{2}}}

\begin{document}
\maketitle
\begin{abstract}
Efficient algorithms for finding the $\tau$-deepest point
in a finite point set are presented.
\end{abstract}

\section{Introduction}

We denote
by $e_i$ the unit vector parallel to the $i$th coordinate axis in the
positive direction.
For a non-zero vector $u$ and a point $p$ we denote by $H_u(p)$ the
closed halfspace with $p$ on its boundary and having inner normal $u$.
For any measure $\mu:2^{\R^d}\mapsto\R$ and any $1/2\le\tau < 1$,
the \emph{\taudepth} of $p\in\R^d$ with respect to $\mu$ is defined as 
\begin{equation}
    \tdepth(p,\mu) = \min\{ \tau \times \mu(H_u(p)\cap H_{e_1}(p)) 
           + (1-\tau)\times \mu(H_u(p)\cap H_{-e_1}(p)): \|u\|=1\} \enspace .
    \eqlabel{taudepth}
\end{equation}
For the special case $\tau=1/2$, \eqref{taudepth} simplifies to
\[
    \tukdepth(p,S) = \frac{1}{2}\min\{ \mu(H_u(p)) : \|u\|=1\} \enspace .
    \eqlabel{tukdepth}
\]
and multiplying by 2 yields the well-known \emph{halfspace depth}.
Often, we are concerned with the depth of a finite set
$S\subseteq\mathbb{R}^d$ of points.  In such cases we use the measure
$\mu(X)=|X\cap S|$.

\Taudepth\ is a generalization of halfspace depth in which mass to the
right of $p$ is assigned more weight than mass to the left of $p$.
However, \taudepth\ is not simply a weighted version of Tukey depth as
the weight of a mass varies depending on the first coordinate of the
point $p$ being considered.  Indeed, while the halfspace depth
contours are convex, the same is not true, in general, of the
\taudepth\ contours (see \figref{example}).

\begin{figure}
\begin{center} \includegraphics[scale=2]{example} \end{center}
\caption{The \taudepth\ contours of a set of 5 points, for $\tau=3/4$. The
white region has depth $1/4$, the gray region has depth $1/2$, and the
black region has depth $3/4$.}
\figlabel{example}
\end{figure}

\section{The Algorithm}

In order to leverage the existing work that has been done on efficient
algorithms for halfspace depth, we study the relationship between
halfspace depth and \taudepth.  For any real number $x$, define the
derived measure $\mu_x$ as
\[
           \mu_x(X) = \tau\times \mu(X\cap H_{e_1}(xe_1)) 
               + (1-\tau)\times \mu(X\cap H_{-e_1}(xe_1)) \enspace .
\]
The following lemma



\end{document}
