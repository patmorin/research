\documentclass{patmorin}
\usepackage{pat,graphicx,amsmath}
\usepackage[mathlines]{lineno}
\linenumbers
\listfiles

\newcommand{\depth}{\mathrm{depth}}

\title{\MakeUppercase{Working Title: Distribution-Sensitive Everything}}

\author{Prosenjit~Bose, 
        Luc~Devroye,
	Karim~Dou\"{\i}eb, 
	Vida~Dujmovi\'c, 
	James~King, and 
	Pat~Morin}

\begin{document}
\maketitle

\begin{abstract}
Let $\mathcal{P}:\R^d\rightarrow\{1,\ldots,k\}$ be a query problem over
$\R^d$ for which there exists a data structure $\mathcal{S}$ that can
compute $P(q)$ in $O(\log n)$ time for any query point $q$.  Let $D$ be
a probability measure over $\R^d$ representing a distribution of queries.
We describe a data structure $T=T(\mathcal{P},D)$ of size $o(n)$ that can
be used as a filter that quickly computes $\mathcal{P}(q)$ for some query
values in $\R^d$ and relies on $\mathcal{S}$ for the remaining queries.
With this filter, the expected query time for a point drawn according
to $D$ is $O(1+ H^*)$, where $H^*$ is a lower-bound on the expected cost
of any linear decision tree that solves $\mathcal{P}$.

This result has many applications, including distribution-sensitive data
structures for point location in 2-d, point-in-polytope testing in 3-d,
nearest-neighbour queries in $\R^d$, point-location in arrangements of
hyperplanes in $\R^d$, and many other geometric problems that can be
solved in the linear-decision tree model.
\end{abstract}

\section{Introduction}

Let $\mathcal{P}:\R^d\rightarrow\{1,\ldots,k\}$ be a query problem over
$\R^d$ for which there exists a data structure $\mathcal{S}$ that can
compute $P(q)$ in $O(\log n)$ time for any query point $q$.  Let $D$ be
a probability measure over $\R^d$ representing a distribution of queries.
We describe a data structure $T=T(\mathcal{P},D)$ of size $o(n)$ that can
be used as a filter that quickly computes $\mathcal{P}(q)$ for some query
values in $\R^d$ and relies on $\mathcal{S}$ for the remaining queries.
With this filter, the expected query time for a point drawn according
to $D$ is $O(1+ H^*)$, where $H^*$ is a lower-bound on the expected cost
of any linear decision tree that solves $\mathcal{P}$.


The remainder of this paper is organized as follows: \Secref{prelim}
presents some preliminary definitions and results. \Secref{filter}
presents the data structure and algorithms for constructing it.
\Secref{applications} presents some of the geometric applications of this
data structure.  Finally, \secref{conclusions} summarizes and concludes
with open problems.

\section{Preliminaries}
\seclabel{prelim}

Throughout this paper, the underlying dimension $d$ is a constant, and
other constants defined throughout the paper may (implicitly) depend
on $d$.  A \emph{simplex} in $\R^d$ is the common intersection of a
set of at most $d+1$ closed halfspaces in $\R^d$. Note that, under this
definition, simplices need not be bounded and $\R^d$ itself is a simplex.

Throughout this paper, we assume an underlying probability measure
$D$ over $\R^d$.  All expectations and probabilities are (implicitly)
with respect to $D$.  For any subset $X\subseteq\R^d$, $\Pr(X)$ refers
to $D(X)$.  We use the notation $D_{|X}$ to denote the distribution $D$
conditioned on $X$, i.e., $D_{|X}(Y)=\Pr(Y\mid X)=\Pr(X\cap Y)/\Pr(X)$
for all $Y\subseteq\R^d$.  If $\Delta$ is a partition of $\R^d$, then
the \emph{entropy} of $\Delta$, denoted $H(\Delta)$ is
\[
    H(\Delta) = \sum_{t\in \Delta} \Pr(t)\log(1/\Pr(t)) \enspace .
\]
The probability measure $D$ is used as an input to our algorithms.
We assume that the algorithm has access to $D$ through two oracles.
Oracle~A allows, for any simplex $t$, to determine $\Pr(t)$ in
constant time.  Oracle~B allows us to draw a random sample $p$ from the
distribution $D$.

A \emph{query problem} over $\R^d$ is a function
$\mathcal{P}:\R^d\rightarrow A$ where $A$ is some set of \emph{answers}.
A \emph{linear decision tree} for $\mathcal{P}$ is a rooted ordered
binary tree in which each internal node $v$ is labelled with a linear
inequality $a_{v,1}x_1 + a_{v,2}x_2 + \cdots a_{v,d}x_d + a_{v,d+1} >
0$, and each leaf $\ell$ is labelled with an element of $A$.  A query
point $p=(x_1,\ldots,x_d)$ follows a root-to-leaf path, proceeding
to the right child of $v$ if it satisfies the inequality and the left
child of $v$ if it does not.  For a linear decision tree $T$ and a point
$p\in\R^d$, we denote by $T(p)$ the label of the leaf on the root-to-leaf
path for $p$ in $T$.  A linear decision tree \emph{solves} $\mathcal{P}$
if $T(p)=\mathcal{P}(p)$ for all $p\in\R^d$.  The \emph{(expected) cost}
of a linear decision tree is the expected depth of the leaf reached when
$p$ is drawn according to the probability measure $D$.

\section{The Data Structure}
\seclabel{data-structure}

In this section we describe our data structure for point location in
disconnected planar subdivisions.  The first tool we use is simplicial
partitions, from the field of geometric range searching:

\begin{thm}[Matou\v{s}ek 1992]\thmlabel{point-partition}
There exists a constant $c$ such that, for any set $S$ of $m$
points in $\R^d$ and any constant $r$, there exists a sequence
$\langle \Delta_1,\ldots,\Delta_r\rangle$ of closed simplices such that
  \begin{enumerate}
    \item $\bigcup_{i=1}^r \Delta_i = \R^d$,
    \item $\left|\Delta_i \cap S\setminus
    \left(\bigcup_{j=1}^{i-1}\Delta_j\right)\right| \le 2m/r$, and
    \item For any hyperplane $\ell$, there are at most $cr^{1-1/d}$ elements of
  $\{\Delta_1,\ldots,\Delta_r\}$ whose interiors intersect $\ell$.
  \end{enumerate}
  The sequence of simplices $\Delta_1,\ldots,\Delta_r$ can be computed
  in $O(m)$ time.
\end{thm}

Note that Part~2 of \thmref{point-partition} is not in the original
statement of the theorem, but follows from Matou\v{s}ek's incremental
construction of $\Delta_1,\ldots,\Delta_r$ \cite{m92}.  We require a
slightly more restricted version of this Theorem in which the $\Delta_i$
are restricted to lie in containing simplex $t$:

\begin{cor}\corlabel{point-partition-2}
There exists a constant $c$ such that, for any set $S$ of $m$ points
contained in some simplex $\Delta \subseteq \R^d$ and any constant $r<m$,
there exists a sequence
$\langle \Delta_1,\ldots,\Delta_{r'}\rangle$ of closed simplices such that
  \begin{enumerate}
    \item $\bigcup_{i=1}^{r'} \Delta_i = \Delta$,
    \item $\left|\Delta_i \cap S\setminus
    \left(\bigcup_{j=1}^{i-1}\Delta_j\right)\right| \le 2m/r$, and
    \item For any hyperplane $\ell$, there are at most $cr^{1-1/d}$ elements of
  $\{\Delta_1,\ldots,\Delta_{r'}\}$ whose interiors intersect $\ell$.
  \end{enumerate}
  The sequence of simplices $\Delta_1,\ldots,\Delta_{r'}$ has $r'=O(r)$
  elements and can be computed in $O(m)$ time.
\end{cor}

\begin{proof}
The theorem is obtained by applying \thmref{point-partition} and then
further subdividing the resulting sequence $\Delta_1,\ldots,\Delta_r$
of triangles.  For each $Delta_i$, we compute the common
intersection $P_i=\Delta_i \cap \Delta$. The resulting polytope $P_i$
is the common intersection of at most $2d$ halfspaces and, by the
Upper Bound Theorem \cite{X}, has $O(d^{\floor{d/2}})$ vertices
and can therefore be partitioned into $O(d^{\floor{d/2}})$ simplices
$\Delta_{i,1},\ldots,\Delta_{i,r_i}$ using the bottom-vertex triangulation
\cite{S}.

The resulting sequence of $r'=O(r)$ simplices certainly has properties 1
and 2 of the theorem.  The sequence also has property 3, though the
constant $c$ is multiplied by a factor of $O(d^{\floor{d/2}})=O(1)$.
\end{proof}

Let $\Delta_i^*$ denote the increment difference
$\Delta_i^*=\Delta_i\setminus\bigcup_{j=1}^{i-1}\Delta_j$.
Let $S$ be a set of $m$ points contained in some simplex
$\Delta$. Then, a \emph{partition tree} $T_{S,\Delta}$ for
$S$ is a rooted ordered tree obtained by recursively applying
\corref{point-partition-2}.  The root of $T_{S,\Delta}$ has $r'$
children corresponding to the simplices $\Delta_1,\ldots,\Delta_{r'}$
obtained by applying \corref{point-partition-2} to $S$ and $\Delta$.
The $i$th child of the root is itself the root of a partition tree
$T_{S\cap\Delta_i^*,\Delta_i}$. This recursive process stops when a
node is empty of points or its depth exceeds $\floor{\alpha\log_r m}$,
for some appropriately chosen constant $0< \alpha < 1$.

Note that every node $v$ in $T_S$ is naturally associated with a
simplex $\Delta(v)$ that was obtained from \corref{point-partition-2} and
that generated $v$. Within a particular level (and across) levels,
these simplices are not necessarily disjoint.  For a point $p\in\R^d$,
the \emph{search path} for $p$ in $T_S$ starts at the root and proceeds
to the first child $i$ such $p\in\Delta_i$ (note that this implies
$p\in\Delta_i^*$) and this process is applied recursively until reaching
a leaf of $T_S$.  In this way, for every $v$ of the partition tree there
is a maximal subset $\Xi(v)\subseteq \R^d$ such that the search path for
every point in $p\in\Xi(v)$ contains $v$.  (Note that, unlike $\Delta(v)$,
$\Xi(v)$ is not necessarily convex or even connected.)

The following theorem summarizes the properties of the partition tree
$T_{S,\R^2}$:

\begin{thm}\thmlabel{point-partition-tree}
Let $T_S$ denote the partition tree described above and let $V_i$
denote the set of nodes of $T_S$ at distance $i$ from the root.
The partition tree $T_S$ has the following properties, for every
$i\in\{0,\ldots,\lfloor\alpha\log_r n\rfloor\}$:
\begin{enumerate}
  \item $\{\Xi(v) : v\in V_i\}$ is a partition of $\R^2$,  
  \item For every node $v\in V_i$, $|S\cap\Xi(v)| \le m(2/r)^i$, and 
  \item For any hyperplane $\ell$, the number of nodes in $V_i$ such that
        $\ell$ intersects the interior of $\Delta(v)$ is at most
        $(cr^{1-1/d})^i$.
  \end{enumerate}
  The partition tree $T_S$ can be constructed in $O(m\log m)$ time.
\end{thm}

\noindent\textbf{Remark:} The use of \corref{point-partition-2} (as
opposed to \thmref{point-partition} is crucial for the construction
of $T_{S,\R^2}$ since, otherwise, the resulting tree will not have
Property 3.

Restating \thmref{point-partition} in terms of probability distributions,
we have:

\begin{thm}\thmlabel{prob-partition-tree}
Let $S$ be a sample of $m$ points in $\R^d$ drawn i.i.d. according
to probability measure $D$, let $T_{S,\R^2}$ denote the partition
tree given by \thmref{point-partition-tree}, and let $V_i$ denote the
set of nodes of $T_{S,\R^2}$ at depth $i$.  With probability at least
$1-O(m^{\gamma})$, $T_{S,\R^2}$ has the following properties, for every
$i\in\{0,\ldots,\lfloor\alpha\log_r n\rfloor\}$:
\begin{enumerate}
  \item $\{r(v) : v\in V_i\}$ is a partition of $\R^2$,  
  \item For every node $v\in V_i$, $\Pr(\Xi(v)) \le (2/r)^i+O(m^{-\delta})$, 
        and 
  \item For any hyperplane $\ell$, the number of nodes in $V_i$ such that
       $\ell$ intersects the interior of $\Delta(v)$ is at most
       $(cr^{1-1/d})^i$.
\end{enumerate}
The partition tree $T_D$ can be constructed in $O(m\log m)$ time using 
$O(m)$ calls to Oracle~B.
\end{thm}

\begin{proof}
Use oracle $B$ to obtain a sample $S$ of $O(m)$ elements from $D$ and
then apply \thmref{point-partition-tree}.
\end{proof}

Note that, so far, we have not considered the query problem $\mathcal{P}$
at all. A \emph{filter tree} $T_{\mathcal{P},D}$ for $(\mathcal{P},D)$
is obtained in the following way:  We construct a partition tree $T_D$
described in \thmref{prob-partition-tree}, using the value $m=n^{\tau}$
for some parameter $0 < \tau < 1$.  Next, we inspect each node $v$ of
$T_D$ and determine if $\mathcal{P}(p)=\mathcal{P}(q)$ for all pairs of
points $p,q\in \Delta(v)$. If so, we call $v$ a \emph{terminal} node and
we label $v$ with the label $\ell(v) = \mathcal{P}(p)$.  Otherwise, $v$
is a \emph{non-terminal} node.

Using a filter tree to answer a query $p\in \R^d$, is easy: We follow
the search path for $p$ in $T_{\mathcal{P},D}$ until we reach a terminal
node $v$, in which case we output $\ell(v)$, or we reach a non-terminal
leaf $w$ after $\lfloor \alpha\log_r m\rfloor=O(\log n)$ steps, in which
case we rely on a backup structure to report $\mathcal{P}(v)$ in $O(\log
n)$ time.  The correctness of this procedure follows immediately from
the definition of terminal and non-terminal nodes.  In the next section,
we analyze the performance of filter trees.



\section{Analysis}
\seclabel{analysis}

An \emph{$i$-set} of a rooted tree $T$ is a set of vertices in $T$ all
of which are at distance at most $i$ from the root of $T$ and in which no
vertex in the set is the ancestor of any other vertex in the set.  We say
that a set of regions $X=\{X_1,\ldots,X_m\}$, $X_i\subseteq\R^d$, is in
\emph{$k$-general position} if there is no hyperplane that intersects $k$
or more elements of $X$.

\begin{lem}\lemlabel{independent}
  Let $T_{\mathcal{P},D}$ be the filter tree defined in
  \secref{data-structure}, let $V$ be an $i$-set of $T_{\mathcal{P},D}$,
  and let $k>1$ be a constant.  Then $V$ contains a subset $V'\subseteq V$
  such that $\{\Delta(v): v\in V'\}$ is in $k$-general position and
  $|V'|\ge |V|/(cr)^{i(d/k+1-1/d)}$ for some constant $c$ independent of $r$.
\end{lem}

\begin{proof}
  We will prove the lemma using the probabilistic method \cite{as08}.
  Let $V'$ be a Bernoulli sample of $V$ where each element is selected
  independently with probability $p=\frac{1}{4}(cr)^{-i(d/k+1-1/d)}$. We
  will prove that
  \[
     \Pr\left\{
        \mbox{$V'$ is in $k$-general position 
          and $|V'| \ge \floor{p|V|}$}
      \right\} > 0 \enspace ,
  \]
  thus proving the existence of a set $V'$ satisfying the conditions of
  the lemma.

  Consider any hyperplane $\ell$. Condition~3 of
  \thmref{prob-partition-tree}, implies that $\ell$ intersects the
  interior of at most $(cr^{1-1/d})^{i}$ elements of $V$.
  The probability that $\ell$ intersects the interior of $k$ or more
  elements of $V'$ is therefore no more than
  \[
    \binom{((cr)^{1-1/d})^{i}}{k}\cdot p^k
    \le (cr)^{i(k-k/d)}p^k
  \]
  The nodes in $V$ define a \emph{test set} $L$ of $O(|V|^d)\le (cr)^{di}$
  hyperplanes such that $V'$ is in $k$-general position if and only
  if no hyperplane in $L$ intersects $k$ or more elements of $V'$. The
  probability that \emph{any} hyperplane in $L$ intersects $k$ or more
  elements of $V'$ is therefore at most
  \[
    (cr)^{di}\cdot(cr)^{i(k-k/d)}p^k = (cr)^{i(d+k-k/d)}p^k \le 1/4 \enspace .
  \]
  The above argument shows that the nodes in $V'$ are quite likely to
  be in $k$-general position. To see that $V'$ is sufficiently large,
  we simply observe that $|V'|$ is a $\mathrm{binomal}(|V|,p)$
  random variable and therefore has median value at least
  $\lfloor{p|V|}\rfloor=\Omega(|V|/(cr)^{i(d/k+1-1/d)})$.  In particular,
  $\Pr\{|V'|\ge \lfloor{p|V|}\rfloor\}\ge 1/2$.  Therefore,
  \[
     \Pr\left\{
        \mbox{$V'$ is in $k$-general position 
          and $|V'|=\Omega(|V|/r^{i(1/2+\epsilon+\delta)})$}
      \right\} \ge 1- (1/4 + 1/2) > 0 \enspace .
  \]
  This completes the proof.
\end{proof}

We are now ready to show that the search time in the filter tree
is a lower bound on the expected cost of any linear decision tree that
solves $\mathcal{P}$.

\begin{lem}\lemlabel{lower-bound}
  Let $T_{\mathcal{P},D}$ be the filter tree defined in
  \secref{data-structure}, let $L$ denote the set of leaves of $T$,
  and let $H^*$ be the expected cost of some linear decision tree $T^*$
  that solves $\mathcal{P}$.  Then $H^* = \Omega(H(\{\Xi(v):v\in L\})-1)$.
\end{lem}

\begin{proof}
  This proof mixes the ideas from the proofs of Lemma~3 by Dujmovi\'c
  \etal\ \cite{dhm09} and Lemma~4 by Collette \etal\ \cite{cdilm08}.

  Let $T'$ be the tree obtained from $T$ by removing all terminal leaves,
  and let $L'$ denote the set of leaves of $T'$.  Note that $L'$ is a Steiner
  triangulation of $G$ and that 
  \[  
     H(L') = H(L) - O(\log r) = H(L) - O(1)
  \]
  since each simplex in $L'$ is partitioned in $O(r^2)$ simplices in $L$. 

  Partition $L'$ into groups $G_1,G_2,\ldots$, where $G_i$
  contains all leaves $v$ such that $1/2^{i-1} \ge \Pr(v) \ge
  1/2^{i}$.  Further partition each group $G_i$ into subgroups
  $G_{i,1},\ldots,G_{i,t_i}$ with the property that each group $G_{i,j}$
  with $j\in\{1,\ldots,t_i-1\}$ is in $k$-general position and has size
  at least $2^{\gamma i}$ for some constant $\gamma > 0$. Furthermore,
  the final group, $G_{i,t_i}$ has size at most $O(2^{\beta i})$, for some
  constant $\beta < 1$.  This partitioning is accomplished by repeatedly
  applying \lemref{independent} to remove a subset $G_{i,j}\subseteq
  G_{i}$ that is in $k$-general position and has size $2^{\gamma i}$,
  stopping the process once the size of $G_i$ drops below $2^{\beta
  i}$. This works provided that we choose $\beta$, $k$, and $r$ so
  that $\beta > ((\log r)/(\log r - 1))(1/2+\epsilon+4/k)$ and set
  $\gamma=\beta - ((\log r)/(\log r - 1))(1/2+\epsilon+4/k)$.

  Now, consider any Steiner triangulation $\Delta^*$ of $G$ and let
  $t$ be a simplex in $\Delta^*$.  Note that $t$ cannot contain any
  simplex in $L'$ since each element in $L'$ is non-terminal in $T$
  and therefore its interior intersects at least two faces of $G$.
  Therefore, any subgroup $G_{i,j}$ intersected by $t$ must intersect one
  of $t$'s three edges. Since each $G_{i,j}$ is in $k$-general position,
  this means that $t$ intersects at most $3k$ elements of $G_{i,j}$.
  It follows \cite[Lemma~3]{cdilm09} that
  \[
    H^* \ge H(L') 
       - H(\{\cup G_{i,j}:i\in\N,\, j \in\{1,\ldots,t_{i,j}\}) 
       - O(1) \enspace .
  \]
  Thus, all that remains is to upper-bound the contribution of $\bar{H}=H(\{\cup G_{i,j}:i\in\N,\, j \in\{1,\ldots,t_{i,j}\})$.
  \begin{eqnarray*}
    \bar{H} &= & H(\{\cup G_{i,j}:i\in\N,\, j \in\{1,\ldots,t_{i,j}\}) \\
     & = & \sum_{i=1}^\infty \sum_{j=1}^{t_{i}} 
         \Pr(\cup G_{i,j})\log(1/\Pr(\cup G_{i,j})) \\
   & = & \sum_{i=1}^\infty
        \left( 
          \sum_{j=1}^{t_{i}-1} 
             \Pr(\cup G_{i,j})\log(1/\Pr(\cup G_{i,j}) 
             + \Pr(\cup G_{i,t_i})\log(1/\Pr(\cup G_{i,t_i}))
        \right) \\
   & \le & \sum_{i=1}^\infty
        \left( 
          \sum_{j=1}^{t_{i}-1} 
             \Pr(\cup G_{i,j})\log(2^{i-\alpha i})
             + i 2^{\beta i - i + 1}
        \right) \\
    & \le & (1-\alpha)H(L') + O(1) \enspace .
  \end{eqnarray*}
  Thus, we have 
  \[  
     H^* \ge H(L') - \bar{H} -O(1) \ge \alpha H(L') - O(1) 
         \ge \alpha H(L) - O(1) = \Omega(H(L) - 1) 
  \]
  as required.
\end{proof}

\begin{thm}
  Let $G$ be a (possibly disconnected) planar subdivision of size $n$
  and let $D$ be a probability measure over $\R^d$.  There exists a data
  structure $T$ that, given $G$ and $D$, can be constructed in $O(n)$
  time, has $O(n)$ size, and can answer point location queries in $G$
  in $O(H^*)$ expected time, where $H^*$ is the expected time to answer
  point location queries in $G$ using any linear decision tree.
\end{thm}

\begin{proof}
  The data structure is, of course, the partition tree $T$ of
  \secref{data-structure} and some backup structure that can answer
  queries in $O(\log n)$ worst case time in case a query reaches a
  non-terminal leaf of $T$.  The expected time answer queries in $T$ is
  \[
     \sum_{t\in L} \Pr(t)O(\depth_T(t)) = \sum_{t\in L}\Pr(t)O(\log(1/\Pr(t))) = O(H(L)) \enspace .
  \]
  On the other hand, by \lemref{lower-bound} and \thmref{triangulation},
  the expected time required by any linear decision tree for answering
  queries in $G$ is
  \[
      H^* = \Omega(H(L) - 1) \enspace ,
  \]
  which completes the proof.
\end{proof}

We finish by observing that the tree $T$ in \secref{data-structure}
has sublinear size. Indeed, for any constant $0 \le d \le 1$, we can
construct a tree $T$ of size $O(n^d)$ that satisfies the conditions
of \lemref{lower-bound}.  Thus, we can think of $T$ as a sublinear
sized filter that can take any point location structure with $O(\log n)$
worst-case query time and make it into a distribution-sensitive data
structure.  In particular, one can combine $T$ with the succinct point
location structure of Bose \etal\ \cite[Theorem~2]{bchmm09}, to obtain the
following result:

\begin{thm}
  Let $G$ be a (possibly disconnected) planar subdivision of size $n$
  and let $D$ be a probability measure over $\R^d$.  There exists a data
  structure $T$ that, given $G$ and $D$, can be constructed in $O(n)$
  time and can answer point location queries in $G$ in $O(H^*)$ expected
  time, where $H^*$ is the expected time to answer point location queries
  in $G$ using any linear decision tree.  This structure is represented
  as a permutation of the vertices of $G$ and an additional $o(n)$ bits.
\end{thm}




\bibliographystyle{plain}
\bibliography{everything}
\end{document}
