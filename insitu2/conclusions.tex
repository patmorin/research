%% -*- LaTeX -*- This is LaTeX2e code



\section{Conclusions}\label{sec:conc}
In this paper, we presented a number of space-efficient tools. In
particular, we presented a scheme for transforming a recursive
function in standard form to one in space-efficient form reducing
the amount of extra memory used to traverse the recursion tree.  We
provided a simple way to stably select a set of elements within an
ordered array, and {\em undo-ing} the stable selection.  We also
provided a tool for easily selecting in linear time with constant
extra space, the $k$-th smallest element in one dimension from an
array of points in $2$ or more dimensions when the points are sorted
in another dimension.  All of these tools are applied to solve
several geometric problems that have solutions using some form of
divide-and-conquer. Specifically, we address the problem of nearest
neighbor, bichromatic nearest neighbor and orthogonal line segment
intersection.

We conclude with two open problems. First, can one
devise a deterministic counter-part to the randomized algorithm presented
for computing the bi-chromatic closest pair for points separated by a vertical line?
Second, is it possible to solve the orthogonal line segment intersection problem
with only $O(1)$ extra memory?

%%% Local Variables:
%%% mode: latex
%%% TeX-master: "paper"
%%% End:
