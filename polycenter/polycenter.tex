\documentclass{article}
\usepackage{graphicx,amsopn,amsthm,amsfonts}


\DeclareMathOperator{\area}{area}
\DeclareMathOperator{\comp}{comp}
\newcommand{\eps}{\epsilon}

 
%\usepackage{amsthm}

\newcommand{\centeripe}[1]{\begin{center}\Ipe{#1}\end{center}}
\newcommand{\comment}[1]{}

\newcommand{\centerpsfig}[1]{\centerline{\psfig{#1}}}

\newcommand{\seclabel}[1]{\label{sec:#1}}
\newcommand{\Secref}[1]{Section~\ref{sec:#1}}
\newcommand{\secref}[1]{\mbox{Section~\ref{sec:#1}}}

\newcommand{\alglabel}[1]{\label{alg:#1}}
\newcommand{\Algref}[1]{Algorithm~\ref{alg:#1}}
\newcommand{\algref}[1]{\mbox{Algorithm~\ref{alg:#1}}}

\newcommand{\applabel}[1]{\label{app:#1}}
\newcommand{\Appref}[1]{Appendix~\ref{app:#1}}
\newcommand{\appref}[1]{\mbox{Appendix~\ref{app:#1}}}

\newcommand{\tablabel}[1]{\label{tab:#1}}
\newcommand{\Tabref}[1]{Table~\ref{tab:#1}}
\newcommand{\tabref}[1]{Table~\ref{tab:#1}}

\newcommand{\figlabel}[1]{\label{fig:#1}}
\newcommand{\Figref}[1]{Figure~\ref{fig:#1}}
\newcommand{\figref}[1]{\mbox{Figure~\ref{fig:#1}}}

\newcommand{\eqlabel}[1]{\label{eq:#1}}
\newcommand{\eqref}[1]{(\ref{eq:#1})}

\newtheorem{thm}{Theorem}{\bfseries}{\itshape}
\newcommand{\thmlabel}[1]{\label{thm:#1}}
\newcommand{\thmref}[1]{Theorem~\ref{thm:#1}}

\newtheorem{lem}{Lemma}{\bfseries}{\itshape}
\newcommand{\lemlabel}[1]{\label{lem:#1}}
\newcommand{\lemref}[1]{Lemma~\ref{lem:#1}}

\newtheorem{cor}{Corollary}{\bfseries}{\itshape}
\newcommand{\corlabel}[1]{\label{cor:#1}}
\newcommand{\corref}[1]{Corollary~\ref{cor:#1}}

\newtheorem{obs}{Observation}{\bfseries}{\itshape}
\newcommand{\obslabel}[1]{\label{obs:#1}}
\newcommand{\obsref}[1]{Observation~\ref{obs:#1}}

\newtheorem{assumption}{Assumption}{\bfseries}{\rm}
\newenvironment{ass}{\begin{assumption}\rm}{\end{assumption}}
\newcommand{\asslabel}[1]{\label{ass:#1}}
\newcommand{\assref}[1]{Assumption~\ref{ass:#1}}

\newcommand{\proclabel}[1]{\label{alg:#1}}
\newcommand{\procref}[1]{Procedure~\ref{alg:#1}}

\newtheorem{rem}{Remark}
\newtheorem{op}{Open Problem}

\newcommand{\etal}{\emph{et al}}

\newcommand{\voronoi}{Vorono\u\i}
\newcommand{\ceil}[1]{\left\lceil #1 \right\rceil}
\newcommand{\floor}[1]{\left\lfloor #1 \right\rfloor}



\title{A Generalized Winternitz Theorem}
\author{Prosenjit Bose \and 
	Paz Carmi \and
	Ferran Hurtado \and
	Pat Morin}

\begin{document}
\maketitle
\begin{abstract}
We prove that, for every simple polygon $P$ having $k\ge 1$ reflex
vertices, there exists a point $q\in P$ such that every half-polygon
that contains $q$ contains nearly $1/2(k+1)$ times the area of $P$.
We also give a family of examples showing that this result is the best
possible.
\end{abstract}

\section{Introduction}

\emph{Winternitz' Theorem} \cite[pp.~54--55]{b23} is a classic theorem
in convex geometry that has been rediscovered many times
\cite{e55b,ll35,n45,n58,yb51}.  Winternitz' Theorem states that, for
any convex polygon $P$, there exists a point $q\in P$ such that any
halfspace that contains $q$ contains at least $4/9$ of the area of
$P$.  The dissection of a triangle into 9 similar triangles shown in
\figref{winternitz} can easily be used to show that the bound of $4/9$
is tight when $P$ is a triangle.

\begin{figure}
  \begin{center}
    \includegraphics{winternitz}
  \end{center}
  \caption{A triangle has maximum halfspace depth $4/9$.}
  \figlabel{winternitz}
\end{figure}


In this paper, we consider a generalization of Winternitz' Theorem to
the case when $P$ is a simple polygon.  A \emph{chord} of a simple
polygon $P$ is a closed line segment whose interior is contained in
the interior of $P$ and whose endpoints are on the boundary of $P$.
If $c$ is a chord of $P$ then $P\setminus c$ has two components $P^+$
and $P^-$.  We call the closure of these polygons \emph{half-polygons}
of $P$.  We define the \emph{depth} of a point $q\in P$ as 
\[
     \delta_P(q) = \min\{\area(h\cap P) : \mbox{$h$ is a half-polygon
	of $P$ that contains $q$} \} \enspace .
\]
Winternitz' Theorem states that, if $P$ is convex then there exists a
point $q\in P$ with $\delta_P(q)\ge (4/9)\area(P)$.  

Winternitz' Theorem is closely related to the \emph{Centerpoint
Theorem} \cite{pa95,m02} which states that for any set $S$ of $n$
points in $\R^2$ there exists a point $q\in\R^2$ such that every
closed halfplane that contains $q$ contains at least $n/3$
points of $S$.  The Centerpoint Theorem is easily derived from Helly's
Theorem \cite{e93} by considering all halfplanes that contain at least
$2n/3$ points of $S$ and taking $q$ to be in their common
intersection.

Helly's Theorem also holds
for half-polygons of $P$.  In particular, if $P_1,\ldots,P_n$ are
half-polygons of $P$ and $P_i\cap P_j\cap P_k\neq \emptyset$ for any
$1\le i < j < k\le n$ then $\bigcap_{i=1}^n P_i\neq \emptyset$.
Therefore we might expect that there always exists a point $q$ with
$\delta_P(q)\ge \area(P)/3$.  However, this intuition turns out to be
false.

\begin{thm}
\thmlabel{lowerbound}
For any $\epsilon > 0$ and any simple polygon $P$ with $k \ge 1$
reflex vertices, there exists a point $q\in P$ such that
$\delta_P(q)\ge \area(P)/2(k+1)-\epsilon$.
\end{thm}

The lower bound of \thmref{lowerbound} is essentially the best
possible:

\begin{thm}
\thmlabel{upperbound}
For every integer $k\ge 1$ and every $\epsilon > 0$,
there exists a polygon $P$ with $k$ reflex vertices, such that no point
in $P$ has depth greater than  $\area(P)/2(k+1) + \epsilon$.
\end{thm}

In \secref{lowerbound} a proof of \thmref{lowerbound} is given.
\Secref{upperbound} presents a family of simple polygons that prove
\thmref{upperbound}.

\section{The Lower Bound}
\seclabel{lowerbound}

For simplicity, we will prove a discrete version of
\thmref{lowerbound} that is a polygonal analog of the Centerpoint
Theorem.  In the discrete version, we are given a polygon $P$ and a
finite set of points $N$ in the interior of $P$, such that no point of
$N$ is collinear with 2 vertices of $P$.  We call $N$ a \emph{general
set of points} in $P$. The \emph{$N$-depth} of a point $q\in P$ is
defined as 
\[
     \delta_{P,N}(q) = \min\{|h\cap N| : \mbox{$h$ is a half-polygon
	of $P$ that contains $q$} \} \enspace .
\]

\comment{
We first prove the theorem for the case where there is exactly one
reflex vertex ($k=1$).
\begin{clm} 
\clmlabel{onereflex}
Let $P$ be a simple polygon having one reflex vertex and let $N$ be a
general set of points in $P$.  Then there exists a point $q\in P$ such
that $\delta_{P,N}(q) \ge |N|/4$.
\end{clm}
%
\begin{proof}
Refer to \figref{onereflex}.  Let $r$ be $P$'s only reflex vertex.
Draw a chord $R$ with one endpoint on $r$ that divides $P$ into two
sub-polygons $P_1$ and $P_2$, such that each sub-polygon contains
$|N|/2$ points.  By the planar Ham-Sandwich Theorem \cite{m03}, there
is a line $\ell$ that splits both $P_1$ and $P_2$ into sub-polygons,
each of which contains $|N|/4$ points.  Let $o$ be the intersection
point of $\ell$ with the supporting line of $R$.

\begin{figure}
  \begin{center}
     \begin{tabular}{cc}
       \includegraphics{lb1} & \includegraphics{lb2} \\
       (1) & (2)
     \end{tabular}
  \end{center}
  \caption{Cases 1 and 2 in the proof of \clmref{onereflex}.}
  \figlabel{onereflex}
\end{figure}


\begin{enumerate}

\item Point $o$ intersects $R$. In this case,
we observe that each half-polygon that contains $o$ contains at least
one of the four subpolygons defined by the removal of $\ell$ and $R$.
Therefore, Therefore, the depth of point $o$ is at least $|N|/4$. 

\item Point $o$ does not intersect $R$.  Without loss of generality,
assume $P_1$ is a convex polygon (at least one of the two polygons
$P_1$ and $P_2$ is convex.) Draw a chord $R'$ with one endpoint at $r$
and that divides $P_1$ into two sub-polygons $P'_1$ and $P''_1$, each
contains $|N|/4$ points. Let $T$ be a chord that is contained in $P_1$
and that separates all points of $N\cap P_1$ from $r$. Let $o$ denote
the intersection of $T$ and $R'$ and observe that any half-polygon
that contains $o$ contains at least one of $P_1'$, $P_1''$, or the
portion of $P_2$ cut off by $L$.  Since each of these subpolygons
contains at least $|N|/4$ points, the depth of $o$ is therefore at
least $|N|/4$.
\end{enumerate} 
\end{proof}
}

The following claim generalizes the Centerpoint Theorem.  Also, by
taking the point set $N$ to be (sufficiently close to) the vertices of
a (sufficiently dense) grid, the claim establishes
\thmref{lowerbound}. 

\begin{clm} 
\clmlabel{bigclaim}
Let $P$ be a simple polygon having $k\ge 1$ reflex vertices and let
$N$ be a general set of points in $P$.  Then there exists a point
$q\in P$ such that $\delta_{P,N}(q) \ge |N|/2(k+1)$.
\end{clm}
\begin{proof}
Refer to \figref{bigproof} for what follows.  
Divide polygon $P$ into at most $k+1$ convex sub-polygons by
iteratively adding a chord on each reflex vertex so that it becomes a
convex vertex in each of the two subpolygons generated.  Let $P^*$ be
a convex sub-polygon that contains at least $|N|/(k+1)$ points of $N$.
Let $Q$ be the component of $P\setminus P^*$ containing the largest
number of points of $N$.  Observe that $|(P^*\cup Q)\cap N|\ge
2|N|/(k+1)$.  Let $r_1r_2$ be the line segment that is common to
$P^*$ and $Q$.  Define $r_1'r_2'$ to be a chord of $P^*$ parallel to
$r_1r_2$ and that separates exactly $|N|/(k+1)$ points of
$N\cap P^*$ from $r_1r_2$. 
The chord $r_1'r_2'$ separates $P^*\cup Q$
into two sub-polygons, $P'$ and $Q'$, where $P'\subseteq P^*$.
Observe that $|Q'\cap N|\ge |P'\cap N| = |N|/(k+1)$

\begin{figure}
  \begin{center}
    \includegraphics{bigproof}
  \end{center}
  \caption{The Proof of \clmref{bigclaim}.}
  \figlabel{bigproof}
\end{figure}

The point, $q$, of high depth we are searching for will be on the
segment $r_1'r_2'$. The remainder of the proof uses a fairly standard
technique that can be used, for example, to prove the Planar Ham
Sandwich Theorem \cite{m03}. However, unlike most applications of this
technique we do not have the continuity that is usually required to
use this technique.  We therefore take special care to explain it in
detail.
 
For $0< t< 1$, let $q_t = tr_1'+(1-t)r_2'$.  Let $C_t$ be the chord of
$P'\cup Q'$ that contains $q_t$ and that bisects $P'\cap N$.  If
$|P'\cap N|$ is odd, the $C_t$ is unique and always contains a point
of $N$.  Otherwise, we can make $C_t$ unique by defining it to be
equidistant from the nearest points of $P\cap N$ on its left and
right.

Let $Q'_t$ denote the component of $Q'\setminus C_t$ that contains
$r_1'$ and let $\overline{Q}'_t = Q'\setminus Q'_t$.
Observe that, for all sufficiently small $\epsilon > 0$, $Q'_\eps\cap
N=\emptyset$ and $Q'_{1-eps} = Q'\cap N$.  Furthermore, $|Q'_t\cap N|$
is an increasing function of $t$.
Therefore, there is some value $t^*$, $0 < t^* < 1$, such that, for
all $\delta > 0$, 
$|Q'_{t^*+\delta}\cap N|\ge |N|/2(k+1)$ and 
$|\overline{Q}'_{t^*-\delta}\cap N|\ge |N|/2(k+1)$.

We claim that $\delta_{N,P}(q_{t^*}) \ge |N|/2(k+1)$.   To see
why this is so, observe that $C_{t^*}$ partitions $P'$ into two
half-polygons, $P_1'$ and $P_2'$, each of which contains $|N|/2(k+1)$
points.  Any half-polygon that contains $q_{t^*}$ but does not contain
either $P_1'$ or $P_2'$ must contain at least one of 
$Q'_{t^*+\delta}$ or 
$\overline{Q}'_{t^*-\delta}$ for some $\delta > 0$.
Therefore, $\delta_{N,O}(q_{t^*}) \ge |N|/2(k+1)$.
\end{proof}

\section{The Upper Bound}
\seclabel{upperbound}

Next we proceed with the proof of \thmref{upperbound}.

\begin{proof}[Proof (of \thmref{upperbound})]
Refer to \figref{upperbound}.
Our construction is parameterized by a value $c <1/2$.
The construction begins by constructing a spiral, with $k+1$ segments
$s_1,\ldots,s_{k+1}$, where segment $s_i$ has length $1+\ceil{i/2}c$
and creates an angle of $\pi/2$ with $s_{i+1}$.  Next, 
we expand the segments $s_1,\ldots,s_k$ inwards so
that each segment $s_i$ becomes a rectangle $R_i$ of the same length
as $s_i$, but whose area is $c$.  It is easy to verify that the
union of these rectangles is a simple polygon with $k$ reflex
vertices.  Furthermore, the area of the intersection of any two
rectangles $R_i$ and $R_i+1$ is at most $c^2$.  Finally, we replace
each reflex vertex with two convex vertices and one reflex vertex as
shown in \figref{upperbound}.b. Suppose the reflex vertex $v$ occurs
at the intersection of a horizontal rectangle $H$ and a vertical
rectangle $V$.  Then the location of the vertex is chosen so that its
$y$-coordinate bisects $H$ and its $x$-coordinate bisects $V$.  By
choosing the two convex vertices sufficiently close together, this
decreases the area of $P$ by at most $\delta$ for any constant $\delta
> 0$.  Denote
the resulting simple polygon by $P$.

\begin{figure}
  \begin{center}
    \begin{tabular}{c@{\hspace{1cm}}c}
      \includegraphics{ubx1} & \includegraphics{ubx2} \\
      (a) & (b)
    \end{tabular}
  \end{center}
  \caption{The construction for the proof of \thmref{upperbound} with $c=1/4$.}
  \figlabel{upperbound}
\end{figure}

Consider the path, shown in \figref{upperbound}.b, that passes through
every reflex vertex and nearly bisects $R_1$ and $R_{k+1}$.  This path
partitions $P$ into $k+2$ pieces.  One of these pieces has area at
most $c(k+1)/2$ and the other $k+1$ pieces have area at most $c/2$.
Each of the small pieces is a half-polygon of $P$, so any point $q$
contained in such a piece has $\delta_P(q)\le c/2$.  On the other
hand, any point contained in the large piece is also contained in a
half-polygon of $p$ whose area is at most $c/2$.  Therefore, 
$\delta_P(q)\le c/2$ for any $q\in P$.  Finally, observe that the area
of $P$ is at least
\[
\area(P) \ge (k+1)c - k(c^2+\delta) \ge (k+1)(c-c^2-\delta)
\]
Therefore, 
\[
\frac{\delta_P(q)}{\area(P)} \le
\left(\frac{1}{2(k+1)}\right)\left(\frac{1}{1-c-\delta/c}\right) \enspace .
\]
Selecting $\delta=c^2$ and $c$ sufficiently small completes the proof.
\end{proof}

\bibliographystyle{plain}
\bibliography{polycenter}


\end{document}

\end{document}
