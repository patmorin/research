\documentclass{patmorin}
%\usepackage{charter}
\usepackage{textcase}
\usepackage{amsmath,amsthm}
\usepackage{pat}

\newcommand{\mst}{\mathit{MST}}

\title{\MakeTextUppercase{A Note on Interference in Random Point Sets}}
\author{BIRS 12w5004}

\begin{document}
\maketitle

\begin{abstract}
  We show that, in any constant dimension $d$, given a set of $n$
  i.u.d. points in the $d$ dimensional cube, it is possible, with high
  probability, to construct a connected graph on these points whose
  maximum interference is $O((\log n)^{1/3})$.
\end{abstract}

\section{Introduction}


Let $V=\{x_1,\ldots,x_n\}$ be a set of $n$ points in $\R^d$ and let
$G=(V,E)$ be a graph with vertex set $V$.  The graph $G$ defines a set,
$B(G)$, of closed balls $B_1,\ldots,B_n$, where $B_i$ has center $x_i$
and radius
\[
   r_i = \max\{\|x_ix_j\| : j\in\{1,\ldots,n\}\} \enspace .
\]
In words, $B_i$ is just large enough to enclose all of $x_i$'s neighbours
in $G$.  We define the \emph{interference} at a point, $p$, as the
number of
these balls that contain $p$, i.e.,
\[
    I(p,G) = |\{B\in B(G) : p\in B\}| \enspace .
\]
We define the \emph{(maximum receiver-centric) interference} of $G$ as the
maximum inteference at any vertex of $G$, i.e.,
\[
   I(G) = \max\{I(x,G) : x\in V\} \enspace .
\]
One of the goals of network design is to build, given $V$, a connected
graph $G=(V,E)$ such that $I(G)$ is minimized.  Thus, it is natural to consider the interference as a property of a given point set, $V$, defined as
\[
  I(V) = \min\{I(G) : \mbox{$G=(V,E)$ is connected}\} \enspace .
\]
The purpose of current paper is to prove the following result:
\begin{thm}\thmlabel{main}
  Let $V$ be a set of $n$ points independently and uniformly distributed
  in $[0,1]^d$.  Then, with high probability, $I(V)\in O((\log n)^{1/3})$.
\end{thm}

\subsection{Related Work}

In this section, we survey previous work on the problem of bounding
the interference of worst-case and random point sets.  A summary of the
results described in this section is given in \figref{related}.  In the statements of all results in this section, $|V|=n$.

\begin{figure}
\begin{center}
  \begin{tabular}{|l|l|r@{, }l|}\hline
    Ref. & Dimension & \multicolumn{2}{c|}{Statement} \\ \hline
    \cite{vR05} & $d=1$ & for all $V$ & $I(V)\in O(n^{1/2})$ \\
    \cite{vR05} & $d\ge 1$ & there exists $V$ & $I(V)\in \Omega(n^{1/2})$ \\
    \cite{ht08} & $d=2$ & for all $V$ & $I(V)\in O(n^{1/2})$ \\
    \cite{ht08} & $d\ge 3$ & for all $V$ & $I(V)\in O((n\log n)^{1/2})$ \\
    \cite{kkmns10} & $d= 1$ & for $V$ i.u.d. in $[0,1]$ & $I(\mst(V))\in \Theta((\log n)^{1/2})$ w.h.p. \\
    \cite{kkmns10,vR05} & $d = 1$ & for $V$ i.u.d. in $[0,1]$  & $I(V)\in\Omega((\log n)^{1/4})$ w.h.p.  \\
    \cite{kdh11} & $d\ge 2$ & for $V$ i.u.d. in $[0,1]^d$ & $I(\mst(V))\in O(\log n)$ w.h.p.  \\
    Here & $d\ge 1$ & for $V$ i.u.d. in $[0,1]^d$  & $I(V)\in O((\log n)^{1/3})$ w.h.p.  \\ \hline
  \end{tabular}
\end{center}
\caption{Previous and new results on interference in geometric networks.}
\figlabel{related}
\end{figure}

The definition of interference used in this paper was introduced by
von~Rickenbach \etal\ \cite{vR05} who proved upper and
lower bounds on the interference of one dimensional point sets:
\begin{thm}\thmlabel{sqrtnlower}
For any integer $d\ge 1$, there exists $V\subset\R^{d}$, such
that $I(V)=\Omega(n^{1/2})$.
\end{thm}
The point set, $V$, in this lower-bound consists of any sequence of
points $x_1,\ldots,x_n$, all on a line, such that $\|x_{i+1}x_i\| \le (1/2)\|x_{i}x_{i-1}\|$,
for all $i\in\{2,\ldots,n-1\}$.  That is, the gaps between consecutive
points decrease exponentially.

This lower bound is matched by an upper-bound:
\begin{thm}\thmlabel{twod-upper}
For all $V\subset\R$, $I(V)\in O(n^{1/2})$.
\end{thm}
The upper bound in \thmref{twod-upper} is obtained by selecting $n^{1/2}$
vertices to act as \emph{hubs}, connecting the hubs into any connected
network and then having each of the remaining nodes connect to its
nearest hub.  This idea was extended to two and higher dimensions
by Halld\'orsson and Tokuyama \cite{ht08}, by using a special type of
$(n^{-1/2})$-net as the set of hubs:
\begin{thm}\thmlabel{sqrtn2d}
For all $V\subset\R^2$, $I(V)\in O(n^{1/2})$.  
\end{thm}
\begin{thm}\thmlabel{sqrtndd}
  For any constant $d\ge 3$, and for all $V\subset\R^d$, $I(V)=O((n\log
  n)^{1/2})$.
\end{thm}

Several authors have shown that the interference of a point set is
related to the (logarithm of) the ratio between the longest and shortest
distance defined by the point set.  In particular, different versions
of the following theorem have been proven by Halld\'orsson and Tokuyama
\cite{ht08}, and Khabbazian, Durocher, and Haghnegahdar
\cite{kdh11}:
\begin{thm}\thmlabel{log}
  For any constant integer $d\ge 1$, and for all $V\subset\R^d$
  $I(V)=O(\log D)$, where $D=\max\{\|xy\|: \{x,y\}\subseteq V\}/\min\{\|xy\|:
  \{x,y\}\subseteq V\}$.
\end{thm}
(A version of this theorem even applies when $V$ is a set of elements
in a metric space of bounded doubling dimension \cite{msz11}.)  The
\emph{minimum spanning tree} of $V$ is the connected graph, $\mst(V)$,
of minimum total edge length.  At least two of the proof of \thmref{log}
proceed by showing that $I(\mst(V))=O(\log D)$.  A strengthening of
this theorem, that is implict in a proof due to Maheshwari, Smid, and
Zeh \cite{msz11} is that the numerator in the definition of $D$ can be
replaced with the length of the longest edge in $\mst(V)$.

\thmref{log} suggests that point sets with very high interference are
unlikely to occur in practice.  This intuition is born out by the results
of Kranakis \etal\ \cite{kkmns10}, who show that high interference is
unlikely to occur in random point sets in one dimension:
\begin{thm}\thmlabel{sqrtlogn}
  Let $V$ be a set of $n$ points independently and uniformly distributed
  in $[0,1]$.  Then, with high probability, $I(\mst(V))\in \Theta((\log
  n)^{1/2})$.
\end{thm}
Note that, in this one-dimensional case, the minimum spanning tree,
$\mst(V)$, is simply a path that connects the points of $V$ in order,
from left to right.  The lower-bound in \thmref{sqrtlogn} is proven
by showing that, with high probability, $V$ contains a subsequence
of points like that used in the proof of \thmref{sqrtnlower},
whose length is $\Omega((\log n)^{1/2})$.  Combining the proofs of
Theorems~\ref{thm:sqrtnlower} and \ref{thm:sqrtlogn} one obtains the
following result:
\begin{thm}
  Let $V$ be a set of $n$ points independently and uniformly distributed
  in $[0,1]$.  Then, with high probability, $I(V)\in \Omega((\log
  n)^{1/4})$.
\end{thm}

In higher dimensions, Khabbazian, Durocher, and Haghnegahdar use their
version of \thmref{log} to show:
\begin{thm}
  Let $V$ be a set of $n$ points independently and uniformly distributed
  in $[0,1]^d$.  Then, with high probability, $I(\mst(V))\in O(\log n)$.
\end{thm}
This result follows from two observations: (1)~no pair of points in $V$
is at distance greater than $\sqrt{d}$, and (2)~with high probability,
the minimum distance between any pair of points is $\Omega(n^{-2})$.  Therefore,
with high probability
\[
   D \le \sqrt{d}/\Omega(n^{-2}) = O(n^2)
\]
and applying \thmref{log} completes the proof.

In the remainder of this paper, we prove \thmref{main}.

\section{Proof of \thmref{main}}

In this section, we prove \thmref{main}.  However, before we do this,
we state a slightly modified version of \thmref{log} that is needed in
our proof.

\begin{lem}\lemlabel{log}
  Let $V\subset\R^d$ be a set of points such that no ball of unit
  diameter contains more than $c$ points of $V$, for some constant $c$
  and suppose that $I(MST(V))=r$ for some integer $r>1$.  Then $MST(V)$
  contains an edge of length $\Omega(c^r)$, for some constant $c=c(d)>1$
  depending only on $d$.
\end{lem}

\begin{proof}[Proof Sketch]
The case $c=0$ corresponds to the statement that the minimum interpoint
distance in $V$ is at least 1.  This case is considered in Lemma~3 of
\cite{msz11}, where the authors show that, for any point, $x$, the set
of balls $B_i\in B(\mst(V))$ that contain $x$ can be partitioned into
$O(\log D)$ classes, where the $i$th class contains balls with radii
in $[2^i,2^{i+1})$.

\noindent\ldots
\end{proof}


\begin{proof}[Proof of \thmref{main}]
For ease of exposition, we will prove the theorem for $d=2$, and mention,
later, the modifications required for the cases $d=1$ and $d\ge 3$.
Partition $[0,1]^2$ into square \emph{cells} of area $1/nt$ for some
value $t$ to be specified later.  Let $N_i$ denote the number of points
that are contained in the $i$th cell.  Then $N_i$ is binomial
with mean $\mu=1/t$.  Recall Chernoff's Bounds \cite{c52} on the tails
of binomial random variables:
\[
  \Pr\{N_i \ge (1+\delta)\mu\} 
    \le \left(\frac{e^\delta}{(1+\delta)^{1+\delta}}\right)^\mu \enspace .
\]
In our setting, we have, 
\begin{align*}
  \Pr\{N_i \ge k\} 
    & = \Pr\{N_i \ge kt\mu\} \\
    & \le \left(\frac{e^{kt}}{(kt)^{kt}}\right)^{1/t} \\
    & = \frac{e^{k}}{(kt)^{k}} \\
    & \le \frac{1}{t^{k}} & \text{for $k\ge e$} \\
    & \le \frac{1}{n^{c+2}} & \text{for $t=2^{(\log n)^{1/3}}$ and $k=(c+2)(\log n)^{2/3}$.} \\
\end{align*}
Note that the number of cells is no more than $nt\le
n^2$, for sufficiently large $n$.  Therefore, by the union bound, the
probability that there exists any cell containing more than $k$ points
is at most $n^{-c}$.

Within each non-empty cell, we apply \thmref{sqrtn2d} to
connect the vertices in the $i$th cell into a connected graph $G_i$
with $I(G_i)=O(\sqrt{N_i})$.  In fact, a somewhat stronger result holds,
namely that $\max\{I(x,G_i) : x\in\R^2\}=O(\sqrt{N_i})$.  Notice that
each edge in $G_i$ has length at most $\sqrt{2/nt}$.  Stated another
way, in $\bigcup_i G_i$, any point, $x$, receives interference only
from cells within distance $\sqrt{2/nt}$ of the cell containing $x$.
There are only 13 such cells, so
\[
  \max\left\{I\left(x,\bigcup_iG_i\right) : x\in\R^2\right\}=O(\sqrt{k}) 
    = O((\log n)^{1/3})
\]
with high probability.

Thus far, the points within each cell are connected to each other and
the maximum interference, over all points in $\R^2$, is $O(\sqrt{k})$.
To connect the cells to each other, we select one point from each
non-empty cell and connect these using a minimum spanning tree, $T$.
What remains is to show that the additional interference caused by the
addition of the edges in $T$ does not exceed $O((\log n)^{1/3})$.

Suppose that $I(x,T)=r$, for some point $x\in\R^2$.  There are at most
4 vertices in $T$ whose distance to $x$ is less than $1/\sqrt{nt}$.
Therefore, by \lemref{log}, $T$ must contain an edge of length at least
$c^r/\sqrt{nt}$, for some $c>1$.  

A well-known property of minimum spanning trees is that, for any edge
$uv$ in $T$, the open disk with diameter $uv$ does not contain any
vertices of $T$.  In our setting, this means that there is an open disk,
$B$, of radius $c^r/2\sqrt{nt}$ such that every cell contained in $B$
contains no point of $V$.  Inside of $B$ is another empty disk $B'$ of
radius $c^r/(2\sqrt{nt})-\sqrt{2/nt}$ whose center is also the center
of some cell.

At least one quarter of the area of $B'$ is contained in $[0,1]^2$,
so the number of cells completely contained in $B'$ is $\pi c^{2r}/16 -
O(c^{r}/\sqrt{nt})$.  By decreasing $c$ slightly, and only considering
$r$ larger than a sufficiently large constant, $r_0$, we can simplify
this number of cells to $\pi c^{2r}/16$.

For a fixed disk $B'$, the probability that the $\pi c^{2r}/16$ cells
defined by $B'$ are empty of points in $V$ is at most
\begin{align*}
 p 
  & \le (1-\pi c^{2r}/{16nt})^{n} \\
  & \le \exp(-\pi c^{2r}/16t) \\
 & \le 1/n^{2+c'} \enspace ,
\end{align*}
for $r\ge(\log_c2)(\log(16/\pi)+\log t + \log(2+c')+\log\ln n)$.  By the union bound, the
probability that there exists any such $B'$ is at most
$pnt\le 1/n^{c'}$.  Since we can choose $r\in O(\log t+\log\log n) = O((\log
n)^{1/3})$, this completes the proof.
\end{proof}

\end{document}
