\documentclass{article}
\usepackage{amsmath,amsthm}

\newtheorem{lem}{Lemma}
\newtheorem{thm}{Theorem}
\newtheorem{cor}{Corollary}

\begin{document}
A sequence of numbers $X_0,\ldots,X_k$ forms a \emph{$k$-frame} if
\[
     X_0 \le 1
\]
and
\[
     X_{i-1}/4 \le X_i \le X_{i-1}/2 \enspace ,
\]
for all $i\in\{1,\ldots,k\}$.  Later, we will show that a $k$-frame
implies a node that has interference $k$.  For now, we start by showing
that frames are not that unlikely:

\begin{lem}
If $X_0,\ldots,X_k$ are a sequence of independent Exponential(1) random
variables, then the probability that $X_0,\ldots,X_k$ form a $k$-frame
is at least $2^{-O(k^2)}$.
\end{lem}

\begin{proof}
Recall that an Exponential(1) random variable $X$ has cumulative
distribution function
\[
   \Pr\{X \le x\} = 1-e^{-x} \enspace .
\]

Next, observe that, in a frame,
\[
                 4^{-i} \le X_i \le 2^{-i}  \enspace ,
\]
for all $i\in\{0,\ldots,k\}$.  Let $F(X)$ be the event ``$X$ is a frame.''
Then,
\begin{align*}
     \Pr\{F(X_0,\ldots,X_{i+1}) \mid F(X_0,\ldots,X_{i})\} 
        & = \Pr\{X_{i+1} \in [X_{i}/4,X_{i}/2] \mid F(X_0,\ldots,X_{i})\} \\
        & \ge \Pr\{X_{i+1} \in [4^{-i}/4,4^{-i}/2]\} \\
        & = \Pr\{X_{i+1} \in [2^{-(2i+2)},2^{-(2i+1)}]\} \\
        & = \exp(-2^{-(2i+2)}) - \exp(2^{-(2i+1)}) \\
        & \sim -2^{-(2i+2)} + 2^{-(2i+1)} \\
        & = 2^{-(2i+2)} \enspace .
\end{align*}
Therefore,
\begin{align*}
     \Pr\{F(X_0,\ldots,X_{k})\}
   & = \Pr\{X_0\le 1\}
         \cdot\prod_{i=1}^k \Pr\{F(X_0,\ldots,X_{i})
                                 \mid F(X_0,\ldots,X_{i-1})\} \\
   & = (1-1/e)
         \cdot\prod_{i=1}^k \Pr\{F(X_0,\ldots,X_{i})
                                 \mid F(X_0,\ldots,X_{i-1})\} \\
   & \ge (1-1/e)\cdot\prod_{i=1}^k 2^{-2i} \\
   & = (1-1/e)\cdot2^{-\sum_{i=1}^k 2i} \\
   & = (1-1/e)\cdot2^{-k(k+1)} \\
   & = 2^{-O(k^2)} \enspace ,
\end{align*}
as required.
\end{proof}

%\begin{thm}
%Let $S$ be a set of $n$ real numbers each selected independently and
%uniformly from the real interval $[0,1]$.  Then, the expected number
%of elements of $S$ that have interference at least $k$ is at least
%$n2^{-O(k^2)}$.
%\end{thm}
%
%\begin{proof}
%Let $X_0,\ldots,X_n$ be a sequence of Exponential(1) random variables,
%let $x_i=\sum_{j=0}^{i-1} X_i$, and let $\tilde x_i = x_i/x_{n+1}$.
%It is well-known that the sequence $\tilde x_1,\ldots,\tilde x_n$ has
%the same distribution that one would obtain by sorting the points of $S$.
%Thus, it suffices to study the set $S=\{\tilde x_1,\ldots,\tilde x_n\}$.
%
%Observe that, for any $i > k$, if $X_{i-k-1},\ldots,X_{i-1}$ is a
%$k$-frame, then $\tilde x_i$ has interference at least $k$.  Therefore
%\[
%  \Pr\{\mbox{$\tilde x_i$ has interference at least $k$}\} \ge 2^{-O(k^2)}
%    \enspace ,
%\]
%for all $i\in\{1,\ldots,n\}$.\footnote{For $i <k$, we can reverse
%the argument.} Therefore, the expected number of elements that have
%interference at least $k$ is $n2^{-O(k^2)}$, as required.
%\end{proof}

This doesn't quite prove that elements with interference at least
$\sqrt{\alpha\log n}$ have a high probability of occuring.  For that, we
can consider blocks of $k=\sqrt{\alpha\log n}$ consecutive elements.  The
last element in each block has probability $n^{-O(\alpha)}$ of being the
last element in a $k$-frame, and this is independent of whether any other
block ends with the last element in a $k$-frame.  Using this, we obtain:

\begin{cor}
Let $S$ be a set of $n$ real numbers each selected independently and
uniformly from the real interval $[0,1]$.  Then, for any $\alpha > 0$,
the probability that no element of $S$ has interference at
least $\sqrt{\alpha\log n}$ is at most $e^{-n^{1-O(\alpha)}}$.
\end{cor}

\end{document}

