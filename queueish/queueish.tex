\documentclass[lotsofwhite,ccfonts]{patmorin}

 
%\usepackage{amsthm}

\newcommand{\centeripe}[1]{\begin{center}\Ipe{#1}\end{center}}
\newcommand{\comment}[1]{}

\newcommand{\centerpsfig}[1]{\centerline{\psfig{#1}}}

\newcommand{\seclabel}[1]{\label{sec:#1}}
\newcommand{\Secref}[1]{Section~\ref{sec:#1}}
\newcommand{\secref}[1]{\mbox{Section~\ref{sec:#1}}}

\newcommand{\alglabel}[1]{\label{alg:#1}}
\newcommand{\Algref}[1]{Algorithm~\ref{alg:#1}}
\newcommand{\algref}[1]{\mbox{Algorithm~\ref{alg:#1}}}

\newcommand{\applabel}[1]{\label{app:#1}}
\newcommand{\Appref}[1]{Appendix~\ref{app:#1}}
\newcommand{\appref}[1]{\mbox{Appendix~\ref{app:#1}}}

\newcommand{\tablabel}[1]{\label{tab:#1}}
\newcommand{\Tabref}[1]{Table~\ref{tab:#1}}
\newcommand{\tabref}[1]{Table~\ref{tab:#1}}

\newcommand{\figlabel}[1]{\label{fig:#1}}
\newcommand{\Figref}[1]{Figure~\ref{fig:#1}}
\newcommand{\figref}[1]{\mbox{Figure~\ref{fig:#1}}}

\newcommand{\eqlabel}[1]{\label{eq:#1}}
\newcommand{\eqref}[1]{(\ref{eq:#1})}

\newtheorem{thm}{Theorem}{\bfseries}{\itshape}
\newcommand{\thmlabel}[1]{\label{thm:#1}}
\newcommand{\thmref}[1]{Theorem~\ref{thm:#1}}

\newtheorem{lem}{Lemma}{\bfseries}{\itshape}
\newcommand{\lemlabel}[1]{\label{lem:#1}}
\newcommand{\lemref}[1]{Lemma~\ref{lem:#1}}

\newtheorem{cor}{Corollary}{\bfseries}{\itshape}
\newcommand{\corlabel}[1]{\label{cor:#1}}
\newcommand{\corref}[1]{Corollary~\ref{cor:#1}}

\newtheorem{obs}{Observation}{\bfseries}{\itshape}
\newcommand{\obslabel}[1]{\label{obs:#1}}
\newcommand{\obsref}[1]{Observation~\ref{obs:#1}}

\newtheorem{assumption}{Assumption}{\bfseries}{\rm}
\newenvironment{ass}{\begin{assumption}\rm}{\end{assumption}}
\newcommand{\asslabel}[1]{\label{ass:#1}}
\newcommand{\assref}[1]{Assumption~\ref{ass:#1}}

\newcommand{\proclabel}[1]{\label{alg:#1}}
\newcommand{\procref}[1]{Procedure~\ref{alg:#1}}

\newtheorem{rem}{Remark}
\newtheorem{op}{Open Problem}

\newcommand{\etal}{\emph{et al}}

\newcommand{\voronoi}{Vorono\u\i}
\newcommand{\ceil}[1]{\left\lceil #1 \right\rceil}
\newcommand{\floor}[1]{\left\lfloor #1 \right\rfloor}



\title{Periodic Dictionaries}
\author{Prosenjit Bose \\ Carleton University \and
	John Iacono \\ Brookly Polytechnic \and
	Pat Morin \\ Carleton University}
\date{}

\begin{document}
\maketitle
\begin{abstract}
We describe a comparison-based dictionary data structure for storing a
set of $n$ keys.  Using this data structure, the time to execute an
access sequence $a_1,\ldots,a_m$ is $O(m(\log\mu+\log^* n))$.  Here,
$\mu=\sum_{i=1}^m (n-w_i)/m$, where $w_i$ is the number of distinct
elements in $a_j,\ldots,a_i$ and $j<i$ is the largest value such that
$a_j=a_i$.
\end{abstract}

\section{Introduction}

Comparison-based dictionary data structures are one of the most
well-studied topics in computer science.  In their simplest form,
these data structures store a set $k_1,\ldots,k_n$ of keys from some
total order $<$ so that searching for a particular key can be done
efficiently.  The only operations performed on keys are comparisons
between pairs of keys.

Let $a_1,\ldots,a_m$ be a sequence of keys, where each
$a_i\in\{k_1,\ldots,k_n\}$.  We define the value $w_i$ as follows: If
there exists a value $j<i$ such that $a_j=a_i$ then $w_i$ is the
number of distinct elements in $a_j\ldots,a_i$, otherwise $w_i=n-1$.
Note that, in the first case, $w_i$ represents the number of distinct
keys accessed since the last access to $a_i$.

Let $\delta$ be the average value of $w_i$, i.e., $\delta=\sum_{i=1}^m
w_i/m$.  The value $\delta$ is closely related to the \emph{empirical
entropy} of the sequence $a_1,\ldots,a_m$ and considerable effort has
been put into finding data structures that can access sequences in
$O(m\log\delta)$ time.  Sleator and Tarjan \cite{st85} showed that
splay trees can access any sequence $a_1,\ldots,a_m$ in
$O(m\log\delta)$ time.  Goodrich \cite{g00} shows that the same result
can be obtained using a different tree-based structure.  Iacono
\cite{i01} showed that a data structure using a set of balanced binary
trees of sizes $2^{2^i}$, $1\le i\le \log\log n$ achieves the same
result, but also has $O(\log n)$ worst case access time.

In this paper, we study a different, almost inverse, measure of
complexity on the sequence $a_1,\ldots,a_m$.  Let $q_i=n-w_i$, so that
$q_i$ is the number of keys in $\{k_1,\ldots,k_n\}$ \emph{not}
accessed since the last access to $a_i$.  Then, we define $\mu$ as the
average value of $q_i$, i.e., $\mu=\sum_{i=1}^m q_i/m$.  The goal of
the current paper is to develop a data structure that can access the
sequence $a_1,\ldots,a_m$ in $O(m\log\mu)$ time.

Note that $\mu$ is small when accesses are periodic in nature.  For an
example of such a periodic sequence, consider a driver's licensing
office where motorists must come every four years to renew their
driver's license.  During any 4 year period, most files will be
accessed and the files accessed on any given day are likely to have
been last accessed approximately 4 years previously.  Thus, in the
period between two accesses to a particular file $k_i$, most of the
other files will have been accessed.

Iacono and Langerman \cite{il02} define queueish dictionaries and
priority queues.  A dictionary is queueish if the (amortized) cost to
perform access $a_i$ is $O(\log q_i)$.  It follows that a queueish
dictionary can access $a_1,\ldots,a_m$ in $O(m\log\mu)$ time.
Unfortunately, their implementation of queueish dictionaries requires
$O(\log q_i+\log\log n)$ time to access $a_i$, which means that the
cost to access the sequence $a_1,\ldots,a_m$ is $O(m(\log\mu +
\log\log n))$.

In this paper we give a dictionary data structure that can access any
sequence $a_1,\ldots,a_m$ in $O(m(\log\mu +\log^* n))$ time, where
$\log^*$ is the iterated logarithm function.  A simple modification of
this data structure gives a data structure that can access the
sequence $a_1,\ldots,a_m$ in $O(m(\log\min\{\mu,\delta\} + \log^*n))$
time.  This is somewhat surprising, since $\mu$ and $\delta$ seem to
be in conflict with each other.

\section{The Data Structure}

We describe the data structure in several steps.  We first place some
conditions on $\mu$ and assume that it's value is known in advance and
then we gradually relax our assumptions.


\subsection{When $\mu \ge \log n/\log \mu$ and known}

Assume the value $\mu$ is known in advance, and that $\mu\ge \log n$.
We can also assume that $\mu\le \sqrt{n}$, otherwise any balanced
binary tree data structure can access the sequence $a_1,\ldots,a_m$ in
$O(m\log \mu)=O(m\log n)$ time.

For this case, our data structure consists of two dictionaries $D_1$
and $D_2$ and a list $L_2$.  At time $i$, the dictionary $D_1$
contains the $\lceil\mu\rceil^2$ least recently accessed keys.  The
dictionary $D_2$ always contains all keys.  The list $L_2$ contains
the subset of $\{k_1,\ldots,k_n\}$ that are not contained in $D_1$,
sorted by the order of most recent access time.  Each key stored in
$D_2$ maintains a pointer the node (if any) of $L_2$ that contains the
same key.

When searching for a value $a_i$ we first search in $D_1$. If we find
$a_i$ then we delete it from $D_1$, append it to $L_2$, remove the
last element of $L_2$ and insert this element into $D_1$.  Otherwise
(we don't find $a_i$ in $D_1$) we search for $a_i$ in $D_2$.  When we
find it, we remove its current location in $L_2$ and move it to the
front of $L_2$.

In the first case, the cost of a search is $O(\log\mu)$.  In the
second case, the cost of a search is $O(\log n)$.  Thus, the total
cost to access $a_1,\ldots,a_m$ is
\[
  T(a_1,\ldots,a_n) = x\times O(\log \mu) + y\times O(\log n) \enspace ,
\]
where $a$, respectively $b$, is the number of accesses that finish in
$D_1$, respectively, $D_2$.  However, since $\mu$ is the average value
of $q_i$ and $D_1$ contains the $\lceil\mu\rceil^2$ least recently
accessed elements, $y\le m/\mu$.  Therefore,
\begin{eqnarray*}
  T(a_1,\ldots,a_n) & \le & \left(m-\frac{m}{\mu}\right)\times O(\log \mu) 
	+ \frac{m}{\mu}\times O(\log n) \\
	& \le & m\times O(\log\mu) + O(m\log\mu) \\
	& = & O(m\log \mu) \enspace ,
\end{eqnarray*}

\begin{thm}
The above data structure accesses the sequence $a_1,\ldots,a_m$ in
$O(m\log\mu)$ time.
\end{thm}

\subsection{When $\mu < \log n$ and known}

When $\mu<\log n$, we use a similar approach.  We maintain
dictionaries $D_1,D_2,\ldots,D_k$, where $D_1$ contains
$\lceil\mu\rceil^2$ keys and $D_i$, $1<i\le k$ contains at most $\mu
t(i,\lceil\mu\rceil^2)$ keys.\footnote{The notation $t(i,x)$ denote
the base 2 tower of $x$ of size $i$, i.e.,
$t(i,x)=\left.2^{2^{\cdot^{\cdot^{\cdot^x}}}}\right\}i$.} We also
maintain lists $L_1,\ldots,L_k$, where the elements of these lists
form a partition of the keys $k_1,\ldots,k_n$.  We maintain the invariant that for all lists, except $L_k$, the elements of $L_i$ are a subset of the elements of $D_i$



\subsection{When $\mu$ is unknown}

\bibliography{queueish} \bibliographystyle{plain}


\end{document}