\documentclass[12pt]{article}
\usepackage{amsthm}
\usepackage{fullpage}
\usepackage{ipe}

\newcommand{\centerpsfig}[1]{\centerline{\psfig{#1}}}
\newcommand{\centeripe}[1]{\begin{center}\Ipe{#1}\end{center}}
\newcommand{\comment}[1]{}

\newcommand{\seclabel}[1]{\label{sec:#1}}
\newcommand{\Secref}[1]{Section~\ref{sec:#1}}
\newcommand{\secref}[1]{\mbox{Section~\ref{sec:#1}}}

\newcommand{\alglabel}[1]{\label{alg:#1}}
\newcommand{\Algref}[1]{Algorithm~\ref{alg:#1}}
\newcommand{\algref}[1]{\mbox{Algorithm~\ref{alg:#1}}}

\newcommand{\applabel}[1]{\label{app:#1}}
\newcommand{\Appref}[1]{Appendix~\ref{app:#1}}
\newcommand{\appref}[1]{\mbox{Appendix~\ref{app:#1}}}

\newcommand{\tablabel}[1]{\label{tab:#1}}
\newcommand{\Tabref}[1]{Table~\ref{tab:#1}}
\newcommand{\tabref}[1]{Table~\ref{tab:#1}}

\newcommand{\figlabel}[1]{\label{fig:#1}}
\newcommand{\Figref}[1]{Figure~\ref{fig:#1}}
\newcommand{\figref}[1]{\mbox{Figure~\ref{fig:#1}}}

\newcommand{\eqlabel}[1]{\label{eq:#1}}
\newcommand{\eqref}[1]{(\ref{eq:#1})}

\newtheorem{thm}{Theorem}{\bfseries}{\itshape}
\newcommand{\thmlabel}[1]{\label{thm:#1}}
\newcommand{\thmref}[1]{Theorem~\ref{thm:#1}}

\newtheorem{lem}{Lemma}{\bfseries}{\itshape}
\newcommand{\lemlabel}[1]{\label{lem:#1}}
\newcommand{\lemref}[1]{Lemma~\ref{lem:#1}}

\newtheorem{cor}{Corollary}{\bfseries}{\itshape}
\newcommand{\corlabel}[1]{\label{cor:#1}}
\newcommand{\corref}[1]{Corollary~\ref{cor:#1}}

\newtheorem{obs}{Observation}{\bfseries}{\itshape}
\newcommand{\obslabel}[1]{\label{obs:#1}}
\newcommand{\obsref}[1]{Observation~\ref{obs:#1}}

\newtheorem{assumption}{Assumption}{\bfseries}{\rm}
\newenvironment{ass}{\begin{assumption}\rm}{\end{assumption}}
\newcommand{\asslabel}[1]{\label{ass:#1}}
\newcommand{\assref}[1]{Assumption~\ref{ass:#1}}

\newcommand{\proclabel}[1]{\label{alg:#1}}
\newcommand{\procref}[1]{Procedure~\ref{alg:#1}}

\newcommand{\proj}{\mathit{proj}}
\newcommand{\pbar}{\overline{p}}
\newcommand{\xbar}{\overline{x}}


\title{Projections of 3-Dimensional Chains and Polygons\thanks{This research was supported by the Natural Sciences and Engineering Research Council of Canada.}}
\author{Prosenjit Bose \and
	Henk Meijer \and
	Pat Morin \and
	Jason Morrison \and 
	Mike Soss}
\date{}

\begin{document}
\maketitle

\begin{abstract}
We show that there exists 3-dimensional polygonal chains having
projections in 3 pairwise orthogonal directions that look like
polygons, but there do not exist 3-dimensional polygons having
projections in 3 pairwise orthogonal directions that look like chains.
\end{abstract}

%%%%%%%%%%%%%%%%%%%%%%%%%%%%%%%%%%%%%%%%%%%%%%%%%%%%%%%%%%%%%%%%%%%%%%%
\section{Introduction}

Orthogonal projection from 3 orthogonal viewpoints is used in
engineering diagrams as a tool to provide 2-dimensional (paper)
descriptions of 3-dimensional objects.  However, if the directions of
projection are poorly chosen, the resulting description can be
misleading.  In this paper we study orthogonal projections of some of
the simplest geometric objects imaginable, namely simple polygons and
polygonal chains.  

A {\em polygon} $P$ is a sequence of line segments
$e_0,\ldots,e_{n-1}$ such that $e_i=(v_{i-1},v_i)$ and $e_{(i+1)\bmod
n}=(v_{i},v_{i+1})$ have only the point $v_i$ in common, for all $0 <
i\le n-1$. $P$ is {\em simple} if $e_i$ and $e_j$ have no points in
common forall $i$ and $j$ such that $j-i\bmod n > 1$.  A {\em
polygonal chain} $C$ is a sequence of line segments
$e_{0},\ldots,e_{n-1}$ such that $e_i=(v_{i-1},v_i)$ and
$e_{i+1}=(v_{i},v_{i+1})$ have only the point $v_i$ in common, for all
$0< i< n-1$.  $C$ is {\em simple} if $e_i$ and $e_j$ have no points in
common forall $i$ and $j$ such that $j>i$ and $j-i > 1$.

We say that a polygon (respectively chain) is {\em 3-dimensional} if
the points that define it are 3-dimensional and {\em planar} if the
points that define it are 2-dimensional.  Although chains and polygons
are defined as sets of line segments, it is often convenient to think
of them as sets of points.  Therefore, with a slight abuse of
notation, we sometimes talk about them as if they were sets of points,
so long as the meaning is clear.

For a direction $d$, we denote by $\proj_d(X)$ the orthogonal
projection of the object $X$ onto a plane orthogonal to $d$.  If
$\proj_d(X)$ is a polygon (respectively, chain), we will say that $X$
{\em looks like} a polygon (respectively, chain) from direction $d$.

In the remainder of the paper, we consider the following questions:

\begin{enumerate}
\item Are there simple 3-dimensional polygonal chains that look like
simple planar polygons from 3 orthogonal directions?

\item Are there simple 3-dimensional polygons that look like simple
planar polygonal chains from 3 orthogonal directions?
\end{enumerate}

One might be tempted to believe that these questions are symmetric in
some way, however, we will see that the answer to the former is yes
while the answer to the latter is no.  Before we proceed, we note that
the word ``orthogonal'' could easily be replaced by ``linearly
independent'' in the above questions without changing their answers,
through the use of a linear transformation.  However, we will continue
to specify orthogonal since it is easier to visualize and understand.

In \secref{chains} we exhibit a simple polygonal chain that looks like
a simple planar polyon from three orthogonal directions.  In
\secref{polygons} we give a proof that there is no simple polygon that
looks like a simple chain from three orthogonal directions.



%%%%%%%%%%%%%%%%%%%%%%%%%%%%%%%%%%%%%%%%%%%%%%%%%%%%%%%%%%%%%%%%%%%%%%%
\section{Chains That Look Like Polygons}\seclabel{chains}

A 3-dimensional simple polygonal chain that looks like a simple planar
polygon from 3 orthogonal viewing directions is given by the following
sequence of vertex coordinates (left to right, top to bottom).
\begin{center}\begin{tabular}{llllll}
$(0,0,0)$&
$(-2,0,0)$&
$(-2,0,-1)$&
$(1,0,-1)$&
$(1,1,-2)$&
$(-3,1,-2)$ \\
$(-3,0,-1)$&
$(-3,0,1)$&
$(1,0,1)$&
$(1,1,0)$&
$(-1,1,0)$
\end{tabular}\end{center}
A picture of this chain (perspective view) is shown in
\figref{chain}.A.  The three orthogonal projections of the chain are
given in Figures~\ref{fig:chain}.B-C.  The existence of this chain
proves the following:

\begin{figure}
\begin{center}\begin{tabular}{cc}
\Ipe{perspective.ipe} & \Ipe{view1.ipe} \\
(A) & (B) \\ \\
\Ipe{view2.ipe} & \Ipe{view3.ipe} \\
(C) & (D)
\end{tabular}\end{center}
\caption{A chain that looks like a polygon from 3 orthogonal directions.}
\figlabel{chain}
\end{figure}

\begin{thm}
There exists a simple 3-dimensional chain $C$ and three orthogonal
directions $d_1$, $d_2$ and $d_3$ such that $\proj_{d_1}(C)$,
$\proj_{d_2}(C)$ and $\proj_{d_3}(C)$ are all simple planar polygons.
\end{thm}

%%%%%%%%%%%%%%%%%%%%%%%%%%%%%%%%%%%%%%%%%%%%%%%%%%%%%%%%%%%%%%%%%%%%%%%
\section{Polygons That Look Like Chains}\seclabel{polygons}

In this section, we will prove that there is no simple polygon that
looks like a chain from 3 orthonal directions.  However, before we
begin, it is worth looking at an example of a chain that look like
polygons in 2 directions.  Such an example is the ``napkin holder''
$N$ of \figref{napkin}.  An interesting property of this example is
that the two endpoints of the chain $\proj_{d_1}(N)$ lie on a common
line parallel to $d_2$.  In our proof, we will see that this is not
coincidental, and that this would be true for any example. Since there
is only one direction $d_2$ with this property, this rules out the
possibility of a third orthogonal direction.

\begin{figure}

\end{figure}

We now proceed with our proof.  Suppose, by way of contradiction, that
there exists a simple 3-dimensional polygon $P$ and three orthogonal
directions $d_1$, $d_2$ and $d_3$ such that $\proj_{d_1}(P)$,
$\proj_{d_2}(P)$ and $\proj_{d_3}(P)$ are simple planar polygonal
chains.  Assume, wlog, that $d_1$ is parallel to the $y$-axis.

The following definition, which we have attempted to illustrate in
\figref{curtain}, will be crucial to our argument.  The {\em curtain}
$c$ is a surface homeomorphic to a plane, parallel to $d_1$ and
containing $P$.  It is helpful to think of a simple polygon $P'$ drawn
on a lined piece of paper which is then folded with creases parallel
to the lines to obtain the 3-dimensional polygon $P$.  The lines on
the paper would be parallel to $d_1$, and the paper itself (if it were
infinitely large) would be the curtain $c$.

\begin{figure}
\centeripe{curtain.ipe}
\caption{A curtain containing a polygon.}
\figlabel{curtain}
\end{figure}

Consider the endpoints $a$ and $b$ of $\proj_{d_1}(P)$.  Since $d_2$
and $d_3$ are distinct, it must be the case that
$\proj_{d_2}(a)\neq\proj_{d_2}(b)$ or
$\proj_{d_3}(a)\neq\proj_{d_3}(b)$.  Therefore, assume wlog that
$\proj_{d_2}(a)\neq\proj_{d_2}(b)$.  Let $p$ be the shortest path, in
$P$, from point $a'\in P$ such that $\proj_{d_1}(a')=a$ to a point
$b'\in P$ such that $\proj_{d_1}(b')=b$.  Let $\pbar$ the part of $P$
not contained in $p$, i.e., the complement of $p$.

\begin{figure}
\begin{center}\begin{tabular}{c@{\hspace{2cm}}c}
$\proj_{d_1}(P)$ & $\proj_{d_2}(P)$ \\
\Ipe{d1.ipe} & \Ipe{d2.ipe} 
\end{tabular}\end{center}
\caption{$\proj_{d_1}(P)$ and $\proj_{d_2}(P)$.}
\figlabel{ab}
\end{figure}

Next we consider $\proj_{d_2}(p)$.  Since $p$ is a shortest path, it
must be that $\proj_{d_2}(p)$ corresponds to the unique path from
$\proj_{d_2}(a')$ to $\proj_{d_2}(b')$ in $\proj_{d_2}(P)$.  We claim
that $\proj_{d_2}(\pbar)$ must contain $\proj_{d_2}(p)$.  Indeed, if
it did not, then there would be two different paths from
$\proj_{d_2}(a')$ to $\proj_{d_2}(b')$, implying a cycle somewhere in
$\proj_{d_2}(P)$, and contradicting the assumption that
$\proj_{d_2}(P)$ is a simple planar polygonal chain.

The next observation is that the curtain $c$ can be unfolded to give
the simple planar polygon $P'$ (this would be the polygon drawn on the
paper, see \figref{pprime}).  The points $a'$ and $b'$ correspond to
points $a''$ and $b''$ in $P'$ that are extreme points (in the $x$ and
$-x$ directions, say).  The paths $p$ and $\pbar$ correspond to the
two parts $p'$ and $\pbar'$, respectively, of $P'$ separated by $a''$
and $b''$.

\begin{figure}
\centeripe{unfolded.ipe}
\caption{The polygon $P'$ obtained by unfolding $c$.}
\figlabel{pprime}
\end{figure}

Note that unfolding $c$ to obtain $P'$ does not change the
$y$-coordinates of the points in $P$.  We claim therefore that for
every point $x\in p'$, there exists a point $\xbar\in\pbar'$ with the
same $y$-coordinate.  This follows from the fact that
$\proj_{d_2}(\pbar)$ contains $\proj_{d_2}(p)$.  Therefore, the two
parts $p'$ and $\pbar'$ of $P'$ have the same maximum $y$-coordinate.
Furthermore, $p'$ and $\pbar'$ are both contained between the two
vertical lines supporting $a''$ and $b''$.  But this is a
contradiction, since then $P'$ cannot possibly be simple.  We conclude
that no such polygon $P$ exists.

We have just proven:

\begin{thm}
There does not exist a simple 3-dimensional polygon $P$ and three
orthogonal directions $d_1$, $d_2$ and $d_3$ such that
$\proj_{d_1}(P)$, $\proj_{d_2}(P)$ and $\proj_{d_3}(P)$ are all simple
planar polygonal chains.
\end{thm}


%%%%%%%%%%%%%%%%%%%%%%%%%%%%%%%%%%%%%%%%%%%%%%%%%%%%%%%%%%%%%%%%%%%%%%%
\section{Conclusions}

In conclusion, we make some remarks about our results.

If we generalize the definition of a curtain to allow for {\em closed
curtains} corresponding to polygons, then Theorems~1 and 2 can be be
thought of as theorems about curtains.  Theorem~1 states that there
exist 3 orthogonal closed curtains whose intersection is a polygonal
chain, while Theorem~2 states that there do not exist 3 orthogonal
curtains whose intersection contains a polygon.

In our proof of Theorem~2, we made no use of the fact that $P$ was
polygon, only that it was simple.  Therefore, the result immediately
generalizes to simple open and closed curves.

Although we have shown that there are no polygons that look like
chains from 3 orthogonal directions, there may be polygons that look
like trees from 3 orthogonal directions.  We are currently unable to
find an example or a counterexample.

\end{document}


