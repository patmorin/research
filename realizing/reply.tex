\documentclass{article}
\usepackage{fullpage}


\begin{document}
\centerline{\textbf{Reply to Referee 1}}

Note, the referee's comments in this file are mangled due to cutting
and pasting from a PDF file.

%%%%%%%%%%%%%%%%%%%%%%%%%%%%%%%%%%%%%%%%%%%%%%%%%%%%%%%%%%%%%%%%%%%%%%%%
\ \ \vspace{2ex}\hrule\vspace{2ex}
\noindent\textbf{Referee's Comment:}

page 2 standard number-theoretical notations: I do not fully agree that these
notations are so standardized: Often, Zk denotes the ring and not just the
set of integers form 0 to k−1. And sometimes 0 belongs to N, sometimes
it does not.

\noindent\textbf{Authors' Response:}

The word standard has been removed.

%%%%%%%%%%%%%%%%%%%%%%%%%%%%%%%%%%%%%%%%%%%%%%%%%%%%%%%%%%%%%%%%%%%%%%%%
\ \ \vspace{2ex}\hrule\vspace{2ex}
\noindent\textbf{Referee's Comment:}

page 2 and further places There are many places where l... should be re-
placed by ℓ....

\noindent\textbf{Authors' Response:}

These typos have been hunted down and fixed.

%%%%%%%%%%%%%%%%%%%%%%%%%%%%%%%%%%%%%%%%%%%%%%%%%%%%%%%%%%%%%%%%%%%%%%%%
\ \ \vspace{2ex}\hrule\vspace{2ex}
\noindent\textbf{Referee's Comment:}

Fig. 2 I think it would help to extend vertical segments on the left boundary
to the bottom line.


\noindent\textbf{Authors' Response:}

Done.


%%%%%%%%%%%%%%%%%%%%%%%%%%%%%%%%%%%%%%%%%%%%%%%%%%%%%%%%%%%%%%%%%%%%%%%%
\ \ \vspace{2ex}\hrule\vspace{2ex}
\noindent\textbf{Referee's Comment:}

Fig. 2, caption Replace subscript k by m.

\noindent\textbf{Authors' Response:}

Done.


%%%%%%%%%%%%%%%%%%%%%%%%%%%%%%%%%%%%%%%%%%%%%%%%%%%%%%%%%%%%%%%%%%%%%%%%
\ \ \vspace{2ex}\hrule\vspace{2ex}
\noindent\textbf{Referee's Comment:}

Proof of Lemma 1 Definition of r: Replace ℓi by ti.

\noindent\textbf{Authors' Response:}

Done.

%%%%%%%%%%%%%%%%%%%%%%%%%%%%%%%%%%%%%%%%%%%%%%%%%%%%%%%%%%%%%%%%%%%%%%%%
\ \ \vspace{2ex}\hrule\vspace{2ex}
\noindent\textbf{Referee's Comment:}

Proof of Lemma 1 Better write O(r) · k/r = O(k) because × is already used
with matrix dimension.

\noindent\textbf{Authors' Response:}

Done.

%%%%%%%%%%%%%%%%%%%%%%%%%%%%%%%%%%%%%%%%%%%%%%%%%%%%%%%%%%%%%%%%%%%%%%%%
\ \ \vspace{2ex}\hrule\vspace{2ex}
\noindent\textbf{Referee's Comment:}

Proof of Lemma 1 Last paragraph: This part should be more elaborated,
since it is important for grasping the ideas of the algorithm(s). For ex-
ample, you should say more explicitly what you mean by a half-line (in
this discrete setting). And you might re-write the line equation such that
it becomes more obvious that we have a line.

\noindent\textbf{Authors' Response:}

We have rewritten the expression for the halfline as a set of points
that satisfy a certain equation, stating precisely the domain of the
points.

%%%%%%%%%%%%%%%%%%%%%%%%%%%%%%%%%%%%%%%%%%%%%%%%%%%%%%%%%%%%%%%%%%%%%%%%
\ \ \vspace{2ex}\hrule\vspace{2ex}
\noindent\textbf{Referee's Comment:}
Fig. 3 There seems to be a bug in the drawing: The “y-values” of the half-lines
at the j = 0 and “j = 10” should be and not those at j = 0 and j = 9 !

Fig. 3 There is an inconsistency: In (2), we have j = 1, . . . , r, here we have
j = 0, . . . , r − 1, similarly for lines 7-9 in the pseudo-code.

\noindent\textbf{Authors' Response:}

The figure has been fixed so that $j$ runs from $1,\ldots,r$

%%%%%%%%%%%%%%%%%%%%%%%%%%%%%%%%%%%%%%%%%%%%%%%%%%%%%%%%%%%%%%%%%%%%%%%%
\ \ \vspace{2ex}\hrule\vspace{2ex}
\noindent\textbf{Referee's Comment:}

Fig. 3 At j = 9, the point on the lower envelope is wrong. It should be the
right endpoint of the dashed segment.

Fig. 3, caption Replace ℓi by ti.

page 4, pseudo-code 1: r gets k/ gcd(k, t ) ?

\noindent\textbf{Authors' Response:}

Good eye for detail!  This have all been fixed.


%%%%%%%%%%%%%%%%%%%%%%%%%%%%%%%%%%%%%%%%%%%%%%%%%%%%%%%%%%%%%%%%%%%%%%%%
\ \ \vspace{2ex}\hrule\vspace{2ex}
\noindent\textbf{Referee's Comment:}

page 4, pseudo-code Regarding readability, the naming of identifiers is a
mess. m was already used in this context; you might use μ or anything
else not used elsewhere. In lines 4 and 5, please replace j by x. Of course,
the present code is not wrong, but the change will increase consistency
with the proof of Lemma 1 and hence increase readability a lot.

\noindent\textbf{Authors' Response:}

The pseudocode has been updated to make it more readable.

%%%%%%%%%%%%%%%%%%%%%%%%%%%%%%%%%%%%%%%%%%%%%%%%%%%%%%%%%%%%%%%%%%%%%%%%
\ \ \vspace{2ex}\hrule\vspace{2ex}
\noindent\textbf{Referee's Comment:}

Theorem 1 There is no explanation regarding the running time(s). This holds
for sections 3 and 4 as well. By the way, don’t you have to prove (or to
mention) that the numbers in the sets Di(y) are not growing too much?

\noindent\textbf{Authors' Response:}

We have added a note at the introduction stating that the model of
computation we use is the $k$-bit word RAM and the implications this
has.  We have also added a short explanation before the theorem that
all the arithmetic operations performed by the algorithm can be done
in constant time in this model of computation.


%%%%%%%%%%%%%%%%%%%%%%%%%%%%%%%%%%%%%%%%%%%%%%%%%%%%%%%%%%%%%%%%%%%%%%%%
\ \ \vspace{2ex}\hrule\vspace{2ex}
\noindent\textbf{Referee's Comment:}

page 4, Sec. 3 Replace equivalences classes by equivalence classes

\noindent\textbf{Authors' Response:}

Done.

%%%%%%%%%%%%%%%%%%%%%%%%%%%%%%%%%%%%%%%%%%%%%%%%%%%%%%%%%%%%%%%%%%%%%%%%
\ \ \vspace{2ex}\hrule\vspace{2ex}
\noindent\textbf{Referee's Comment:}

page 5 No need to mention 1 with Di(y) ?

\noindent\textbf{Authors' Response:}

It seems that we do, otherwise we are taking the minimum of an empty
set, which is undefined.  Later, we show how to construct $D_{i+1}$
from $D_i$ so it turns out to be important that $D_i$ has no undefined
values.

%%%%%%%%%%%%%%%%%%%%%%%%%%%%%%%%%%%%%%%%%%%%%%%%%%%%%%%%%%%%%%%%%%%%%%%%
\ \ \vspace{2ex}\hrule\vspace{2ex}
\noindent\textbf{Referee's Comment:}

Proof of Lemma 3 Replace qk/r by qk/r−1 .

\noindent\textbf{Authors' Response:}

Done.

%%%%%%%%%%%%%%%%%%%%%%%%%%%%%%%%%%%%%%%%%%%%%%%%%%%%%%%%%%%%%%%%%%%%%%%%
\ \ \vspace{2ex}\hrule\vspace{2ex}
\noindent\textbf{Referee's Comment:}

Proof of Lemma 3 The coupling in Di(q0,j , j + c) is quite confusing at first
reading since it is not motivated.

\noindent\textbf{Authors' Response:}

A footnote has been added to help clarify this.


%%%%%%%%%%%%%%%%%%%%%%%%%%%%%%%%%%%%%%%%%%%%%%%%%%%%%%%%%%%%%%%%%%%%%%%%
\ \ \vspace{2ex}\hrule\vspace{2ex}
\noindent\textbf{Referee's Comment:}

Sec. 3 and 4 Figures and examples would help in these parts, too.

\noindent\textbf{Authors' Response:}

A figure (with example) has been added in Section 3.  This figure is
also referenced in Section 4.

\end{document}



















