
%% -*- LaTeX -*- This is LaTeX2e code



\section{Introduction}

Researchers have studied space-efficient algorithms since the early
70's. Examples include merging, (multiset) sorting, and partitioning
problems;
see~\cite{geffert:merging,katajainen:multisets,katajainen:partitioning}.
Br\"onnimann \etal~\cite{bronnimann:convex} were the first to consider
space-efficient geometric algorithms and showed how to compute the
convex hull of a planar set of $n$ points in \Oh{n \log h} time using
\Oh{1} extra space, where $h$ denotes the size of the output.
Recently, Chen and Chan~\cite{chen:space-efficient} addressed the
problem of computing the intersections among $n$ line segments, giving
an algorithm that runs in \Oh{(n+k)\log^2 n} time using \Oh{\log^2n}
extra space where $k$ is the number of intersections and an algorithm
that runs in \Oh{(n+k)\log n} time using \Oh{1} extra space but the
initial input is destroyed.  Br\"onnimann, Chan and
Chen~\cite{bronnimann:inplace} developed some space efficient data
structures and used them to solve a number of geometric problems such
as convex hull, Delaunay triangulation and nearest neighbor queries.
In this paper, we develop a number of space-efficient tools outlined
in Section \ref{sec:tools} that are of interest in their own right. We
then apply these tools to several geometric problems that have
solutions using some form of divide-and-conquer. Specifically, we
address the problems of closest pairs, bichromatic closest pair 
and orthogonal line segment intersection.

\ournote{Jit:  Can you please add a sentence for Br\"onnimann \etal~\cite{bronnimann:inplace}? I forgot my proceedings at home.}

\subsection{The Model}

The goal is to design algorithms that use very little extra space over
and above the space used for the input to the algorithm. The input is
assumed to be stored in an array of size $n$, thereby allowing random
access.  The specifics of the input are outlined with each problem
addressed.  We assume that a constant size memory can hold a constant
number of words. Each word can hold one pointer, or an \Oh{\log n} bit
integer, and a constant number of words can hold one element of the
input array. The extra memory used by an algorithm is measured in
terms of the number of extra words. In certain cases, the output may
be much larger than the size of the input. For example, given a set of
$n$ line segments, if the goal is to output all the intersections,
there may be as many as \Wm{n^2}. In such cases, we consider the
output memory to be write-only space usable for output but cannot be
used as extra storage space by the algorithm.  This model has been
used by Chen and Chan~\cite{chen:space-efficient} for variable size
output, space-efficient algorithms and accurately models algorithms
that have output streams with write-only buffer space.

The remainder of the paper is organized as follows. In Section
\ref{sec:tools}, we outline the space-efficient tools that will be
useful in the solution of the geometric problems addressed. In Section
\ref{sec:nn}, we present an \Oh{n \log n} time algorithm to solve the
closest pair problem using only \Oh{1} extra memory. In the following
subsection, we solve the bichromatic closest pair problem in \Oh{n
\log n} time using only \Oh{1} extra memory. The solution is
randomized but the extra memory used is still \Oh{1} in the worst
case.  Section \ref{sec:orth} presents a solution to the orthogonal
line segment intersection problem in \Oh{n\log n + k} time using
\Oh{\log n} extra space where $n$ is the number of line segments, and
$k$ is the number of intersections.  Finally, conclusions and open
problems are in Section \ref{sec:conc}.


%\begin{definition}
%  An algorithm that needs \Oh{1} extra working space is called
%  \emph{in-place}, and an algorithm that needs \Oh{\log_2 n} extra
%  space is called \emph{in situ}.
%\end{definition}
%
%Recently, Raman~\etal~\cite{raman:dynamic,raman:indexable} considered 
%\emph{succinct} representations of ordered sets that allowed for
%various dictionary operations.
%
%%% Local Variables:
%%% mode: latex
%%% TeX-master: "paper"
%%% End:
