\documentclass[lotsofwhite,charterfonts]{patmorin}
\usepackage{amsopn}
 
%\usepackage{amsthm}

\newcommand{\centeripe}[1]{\begin{center}\Ipe{#1}\end{center}}
\newcommand{\comment}[1]{}

\newcommand{\centerpsfig}[1]{\centerline{\psfig{#1}}}

\newcommand{\seclabel}[1]{\label{sec:#1}}
\newcommand{\Secref}[1]{Section~\ref{sec:#1}}
\newcommand{\secref}[1]{\mbox{Section~\ref{sec:#1}}}

\newcommand{\alglabel}[1]{\label{alg:#1}}
\newcommand{\Algref}[1]{Algorithm~\ref{alg:#1}}
\newcommand{\algref}[1]{\mbox{Algorithm~\ref{alg:#1}}}

\newcommand{\applabel}[1]{\label{app:#1}}
\newcommand{\Appref}[1]{Appendix~\ref{app:#1}}
\newcommand{\appref}[1]{\mbox{Appendix~\ref{app:#1}}}

\newcommand{\tablabel}[1]{\label{tab:#1}}
\newcommand{\Tabref}[1]{Table~\ref{tab:#1}}
\newcommand{\tabref}[1]{Table~\ref{tab:#1}}

\newcommand{\figlabel}[1]{\label{fig:#1}}
\newcommand{\Figref}[1]{Figure~\ref{fig:#1}}
\newcommand{\figref}[1]{\mbox{Figure~\ref{fig:#1}}}

\newcommand{\eqlabel}[1]{\label{eq:#1}}
\newcommand{\eqref}[1]{(\ref{eq:#1})}

\newtheorem{thm}{Theorem}{\bfseries}{\itshape}
\newcommand{\thmlabel}[1]{\label{thm:#1}}
\newcommand{\thmref}[1]{Theorem~\ref{thm:#1}}

\newtheorem{lem}{Lemma}{\bfseries}{\itshape}
\newcommand{\lemlabel}[1]{\label{lem:#1}}
\newcommand{\lemref}[1]{Lemma~\ref{lem:#1}}

\newtheorem{cor}{Corollary}{\bfseries}{\itshape}
\newcommand{\corlabel}[1]{\label{cor:#1}}
\newcommand{\corref}[1]{Corollary~\ref{cor:#1}}

\newtheorem{obs}{Observation}{\bfseries}{\itshape}
\newcommand{\obslabel}[1]{\label{obs:#1}}
\newcommand{\obsref}[1]{Observation~\ref{obs:#1}}

\newtheorem{assumption}{Assumption}{\bfseries}{\rm}
\newenvironment{ass}{\begin{assumption}\rm}{\end{assumption}}
\newcommand{\asslabel}[1]{\label{ass:#1}}
\newcommand{\assref}[1]{Assumption~\ref{ass:#1}}

\newcommand{\proclabel}[1]{\label{alg:#1}}
\newcommand{\procref}[1]{Procedure~\ref{alg:#1}}

\newtheorem{rem}{Remark}
\newtheorem{op}{Open Problem}

\newcommand{\etal}{\emph{et al}}

\newcommand{\voronoi}{Vorono\u\i}
\newcommand{\ceil}[1]{\left\lceil #1 \right\rceil}
\newcommand{\floor}[1]{\left\lfloor #1 \right\rfloor}



\DeclareMathOperator{\lft}{left}
\DeclareMathOperator{\rght}{right}
\DeclareMathOperator{\tp}{up}
\DeclareMathOperator{\bttm}{down}


\title{\MakeUppercase{Biased Range Trees}}
\author{Vida Dujmovi\'c
	\and John Howat
	\and Pat Morin}

\begin{document}
\maketitle
\begin{abstract}
A data structure, called a \emph{biased range tree}, 
is presented that preprocesses a set $S$ of $n$ points in $\R^2$
and a query distribution $D$ for 2-sided orthogonal range counting
queries.  The expected query time for this data structure under the
distribution $D$ is competitive with the optimal decision tree for $S$
and $D$.   The memory and preprocessing requirements of the data
structure are $O(n\log n)$.
\end{abstract}

\section{Introduction}

Let $S$ be a set of $n$ points in $\R^2$ and let $D$ be a probability
measure over $\R^2$.  A \emph{2-sided range counting query} over $S$
asks, for a query point $q=(q_1,q_2)$, the number of points in
$(p_1,p_2)\in S$ such $p_1 \ge q_1$ and $p_2 \ge q_2$.  A 2-sided
range counting query \emph{has distribution $D$} if the query point
$q$ is chosen from the probability measure $D$.  If $T$ is a data
structure for answering 2-sided range queries over $S$ then we denote
by $\mu_D(T)$ the expected time, using $T$, to answering a range query
with distribution $D$.  The current paper is concerned with
preprocessing a point set $S$ and a distribution $D$ to build a data
structure $T$ that minimizes $\mu_D(T)$.

\subsection{Previous Work}

\begin{enumerate}

\item The topic of geometric range queries is a huge field
\cite{agarwal_erickson}.

\item Range trees \cite{bentley} use $O(n\log n)$ space and
preprocessing and answer range queries in $O(\log^2 n)$
worst-case time.  When combined with fractional cascading, the query
time can be reduced to $O(\log n)$.

\item $k$-d trees use $O(n)$ space and answer range queries in
$O(\sqrt{n})$ time.

\item Get the two SICOMP papers by Chazelle.
\end{enumerate}

\subsection{New Results}

In the current paper we present a data structure of size $O(n\log n)$,
that can be constructed in $O(n\log n)$ time, and that answers range
counting (or semigroup) queries in $O(\mu_D(T))$ expected time, where
$T$ is any comparison tree that answers range counting queries over $S$.
In particular, $T$ could be the comparison tree that minimizes
$\mu_D(T)$ implying that the expected query time of our data structure
is as fast as the fastest comparison tree.

Note that we do not place any restrictions on the comparison tree $T$.
Thus, our data structure, while requiring only $O(n\log n)$ space, is
competitive with any data structure that can be represented as a
comparison tree, without any limits of the space required by that data
structure.

\section{Preliminaries}

\begin{itemize}
\item A \emph{rectangle} $R(a,b,c,d)$ is defined as
\[
    R(a,b,c,d) = \{ (x,y) : \mbox{$a\le x < b$ and $c \le y < d$}\}
	\enspace .
\]
Notice that this implies that rectangles are open on their top and
right sides but closed on their bottom and left sides.  We also allow
unbounded rectangles by setting $a,c=-\infty$ and/or $b,d=\infty$.
Therefore, under this definition, rectangles can have 0, 1, 2, 3,
or 4
sides.

\item Definition of decision tree and comparison tree
\item Shannon's entropy lower bound
\item Lemma that says that entropy decreases when the support is
partitioned
\item Lower bound in terms of the entropy of the ray shooting
arrangement
\end{itemize}

\section{Biased Range Trees}

In this section we present our data structure that, like range trees
is a two level structure consisting of a primary tree whose nodes
store secondary structures.  Throughout this section, for the sake of
simplicity, we will assume that the probability measure $D$ is smooth
in the sense that, if $X$ is a 0 or 1 dimensional set then
$\Pr\{X\}=0$.  This assumption is only used to simplify the
exposition, and at the end of \secref{analysis} we show how it can be
removed.

\subsection{The Primary Tree}

The primary tree is a 2-dimensional $k$-d tree \cite{X} $T=T(D,S)$.
Each node $v$ of $T$ is associated with a rectangle $r(v)$.  The
rectangle associated with the root of $T$ is all of $\R^2$. The two
children of a node $v$ are associated with the two rectangles obtained
by splitting $r(v)$ with a horizontal or vertical line depending on
whether the distance from $v$ to the root of $T$ is even or odd,
respectively.  For a node $v$ that is split with a vertical line, we
denote its children by $\lft(v)$ and $\rght(v)$ and we call them the
left and right children of $v$.  For a node $v$ that is split with a
horizontal line, we denote its children by $\top(v)$ and $\bttm(v)$
and we call them the up and down children of $v$. 

The line used to split the rectangle $r(v)$ into two rectangles $r_1$
and $r_2$ is selected so that $\Pr\{r_1\}=\Pr\{r_2\}=\Pr\{r(v)\}/2$.
This splitting process is applied recursively to any node $v$ such as
long as $r(v)$ contains more than one point of $S$ and $\Pr\{r(v)\}
\ge 1/n$.  This naturally partitions the leaves of $T$ into two
classes; the \emph{bad} leaves are those leaves $v$ such that
$|r(v)\cap S| > 1$ and the remaining leaves are the \emph{good}
leaves.  Note that, in $T$ a node $v$ at level $i$ has
$\Pr\{r(v)\}=1/2^i$ and hence the height of the structure is at most
$\log n+1$.


\subsection{The Catalogues}

The nodes of the tree $T$ are augmented with additional data
structures called \emph{catalogues}.  The $x$-catalogue $C_x(v)$
(respectively, $y$-catalogue $C_y(v)$) contains the set of points
$r(v)\cap S$ sorted by $x$ (respectively, $y$) coordinate.

In addition to the catalogues, fractional cascading \cite{X} is used
to help navigate between specific pairs of catalogues.  In order to
understand the flow of information within catalogues, it is useful to
understand a little about the query algorithm.  To determine the
number of points of $S$ in $r(v)$, for a query point $q$ contained in
$r(\lft(v))$, it suffices to recursively search for $q$ in $\lft(v)$
and then locate $y(q)$ in $C_y(\rght(v))$.  Thus, after recursing in
$\lft(v)$ we would like to have enough information to locate $y(q)$ in
$C_y(\rght(v))$ in $O(1)$ time.  Note that the structure is very
asymmetric in this sense since answering a query $q\in r(\rght(v))$
does not require locating $q\in C_y(\lft(v))$ or even locating $q\in
C_y(v)$.



Similarly, a query
that recurses in $\bttm(v)$ should have gathered enough information
to locate $x(q)$ in $C_x(\tp(v))$.


If the query algorithm
has already recursively performed a range counting query $q$ on $\lft(v)$
(respectively, $bttm(v)$) the answer to entire query can be obtained
by locating $x(q)$ in the $C_y(\rght(v))$

Suppose that a node $v$ is split with a vertical line into two
children $\lft(v)$ and $\rght(v)$.  Let $LR^i(v)$ denote the set of
nodes in the subtree rooted at $\lft(v)$ that can be reached by a path
from $\lft(v)$ of length $2i$ that does not take any edge from a node
to its left child.  Define $RL^i(v)$ in a symmetric manner, reversing
the roles of left and right. For each $i$, the $y$-catalogues of nodes
in $RL^i(v)$ is cascaded into the appropriate nodes of 
$LR^i(v)$. 

  Suppose
further that $\lft(v)$ is split into two children $\tp(\lft(v))$ and
$\bttm(\lft(v))$.  Then, every second point of $C_y(\rght(v))$ is
cascaded into one of $\tp(\lft(v))$ or $\bttm(\lft(v))$ depending on
whether its horizontal projection intersects the horizontal projection
of $\tp(\lft(v))$ or $\bttm(\lft(v))$, respectively.  With the extra
information provided by this fractional cascading, if the search
algorithm has determined the location of $q_2$ in $C_y(\tp(\lft(v)))$
or $C_y(\bttm(\lft(v)))$ then, in $O(1)$ time, it can determine the
location of $q_2$ in $\rght(v)$.

Similarly, suppose a node $v$ is split with a horizontal line into two
children $\tp(v)$ and $\bttm(v)$.  Suppose further that $\bttm(v)$ is
split into two children $\lft(\bttm(v))$ and $\rght(\bttm(v))$.  Then,
every second point of $C_x(\tp(v))$ is cascaded into one of
$\lft(\bttm(v))$ or $\rght(\bttm(v))$ depending on whether its
horizontal projection intersects the horizontal projection of
$\lft(\bttm(v))$ or $\rght(\bttm(v))$, respectively.  Again, if the
search algorithm has located $q_1$ in $\lft(\bttm(v))$ or
$C_x(\rght(\bttm(v)))$ then it can determine, in $O(1)$ time, the location
of $q_1$ in $C_x(\tp(v))$.

An important property of our structure is that the points in $C_x(v)$
are all contained in the 3-sided range bounded by the left, bottom,
and right sides of $r(v)$.  Similarly, the points in $C_y(v)$ are all
contained in the 3-sided range bounded by the top, left, and bottom
sides of $r(v)$.

In addition their catalogues, the good leaves of $T$ have additional
indexing structures associated with them.  Recall that each good leaf
$v$ of $T$ has $|r(v)\cap S| \le 1$. However, $C_x(v)$ and $C_y(v)$
could still be quite large since they contain elements that were
cascaded from nodes higher in $T$.  For each good leaf $v$, we
therefore maintain two biased binary search trees $T_x(v)$ and
$T_y(v)$ over the intervals defined by the $x$, respectively, $y$,
coordinates of the points in $C_x(v)$, respectively, $C_y(v)$.  The
weight of an interval $(x_1,x_2)$ is the probability that a query
point $q=(q_1,q_2)$ drawn from the distribution $D\mid r(v)$ has
$q_1\in(x_1,x_2)$.   Similarly, the weight of an interval $(y_1,y_2)$
is the probability that $q_2\in (y_1,y_2)$.

\subsection{The Query Algorithm}

To answer a 2-sided range query $q=(q_1,q_2)$ using $T$ we proceed in
three phases:

\begin{enumerate}

\item The algorithm navigates the tree $T$ from top to bottom to
locate the unique leaf $v$ of $T$ such that $q\in r(v)$.  This takes
$O(d_T(q))$ time, where $d_T(q)$ is the depth of the node $v$.


\item If the leaf $v$ is a bad leaf then the algorithm performs a
range query using the range tree $R(S)$ and the query algorithm is
done.  Otherwise ($v$ is a good leaf) the algorithm uses $T_x(v)$ and
$T_y(v)$ to locate $q_1$ and $q_2$, respectively, in $C_x(v)$ and
$C_y(v)$, respectively.

\item The walks back from $v$ to the root of $T$.

\end{enumerate}

\section{Optimality of Biased Range Trees}

\subsection{The Catalogue Location Step}


\subsection{The Tree Searching Step}

\section{Summary and Conclusions}

\end{document}
