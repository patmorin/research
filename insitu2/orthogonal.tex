%% -*- LaTeX -*- This is LaTeX2e code

\section{Orthogonal line segment intersection}\label{sec:orth}

In this section, we present a space-efficient algorithm for the
orthogonal line segment intersection problem.  Since this problem has
a variable output size $k \in \Oh{n^2}$ we consider the output memory
model to be a write-only space usable only for output and not for
temporary space.  This model has been used by Chen and
Chan~\cite{chen:space-efficient} for variable size output,
space-efficient algorithms and accurately models algorithms that have
output streams with write-only buffer space.

The input to our algorithm is an array $V[0,n-1]$ of $n$ vertical
segments and an array $H[0,m-1]$ of $m$ horizontal segments. The
goal is to output all of the intersections between horizontal and
vertical segments.  Our solution is an algorithm in standard recursive
form. Therefore, we use \Oh{\log n} extra memory and our computation
time is \Oh{n \log n + k} where $k$ is the number of intersections.
This is an improvement over the algorithm by Chen and
Chan\cite{chen:space-efficient}, however they solve the more general
problem of enumerating all intersections among a general set of
line segments.  It is unclear how to convert our solution to
space-efficient recursive form because it is difficult to pre-compute
all of the base instances.  Once we have described the algorithm,
the reason for this difficulty will become clear.



\begin{algorithm}
  \caption{OLSI($V[a,b],H[c,d]$): Report all intersections between vertical line segments in $V[a,b]$ and horizontal line segments $H[c,d]$.} \label{alg:orth}
  \begin{algorithmic}[1]
    \REQUIRE segments in $V$ are sorted by $Y$-coordinate of the top vertices of the segments and segments in $H$ are sorted by $Y$-coordinate.
    \ENSURE all intersections between a segment in $H$ and a segment in $V$ are reported.
    \IF{$b-a$ is 1} 
       \STATE \COMMENT{There are 2 vertical segments in $V[a,b]$.}
       \STATE Let $X_{min}$ be the minimum $X$-coordinate of the segments in $V[a,b]$.
       \STATE Let $X_{max}$ be the maximum $X$-coordinate of all segments in $V[a,b]$.
       \STATE Use \textsc{SortedSubsetSelection}, to select all horizontal segments that completely span the slab $[X_{min},X_{max}]$.
              The segments are placed in $H[c,m_2]$.
       \STATE Scan $V[a,b]$ and $H[c,m_2]$ to report all intersections.
       \STATE UNDO Selection of horizontal segments spanning the slab. We recover $H[c,d]$ from $H[c,m_2]$.
    \ELSE
       \STATE Let $X_{min}$ be the minimum $X$-coordinate of all segments in $V[a,b]$.
       \STATE Let $X_{med}$ be the median $X$-coordinate of all segments in $V[a,b]$.
       \STATE Let $X_{max}$ be the maximum $X$-coordinate of all segments in $V[a,b]$.
       \STATE \COMMENT{$X_{min}$ and $X_{max}$ represent the boundaries of the current vertical slab being processed}.
       \STATE Use \textsc{SortedSubsetSelection}, to select all horizontal segments that completely span the slab $[X_{min},X_{max}]$.
              The segments are placed in $H[c,m_2]$.
              \COMMENT{These are the horizontal segments that are completely spanning the current slab: $[X_{min},X_{max}]$}.
       \STATE Scan $V[a,b]$ and $H[c,m_2]$ to report all intersections.
       \STATE UNDO Selection of horizontal segments spanning the slab. We recover $H[c,d]$ from $H[c,m_2]$.
       \STATE Use \textsc{SortedSubsetSelection} to select all vertical segments in the slab $[X_{min},X_{med}]$.
              The segments are placed in $V[a,m_1]$.
       \STATE Use \textsc{SortedSubsetSelection}, to select all horizontal segments that are in the slab $[X_{min},X_{med}]$ or span the slab $[X_{min},X_{med}]$ but
              DO NOT span the slab $[X_{min},X_{max}]$.
              The segments are placed in $H[c,m_2]$.
       \STATE OLSI($V[a,m_1],H[c,m_2]$).
       \STATE UNDO Selection of horizontal segments to recover $H[c,d]$ from $H[c,m_2]$.
       \STATE UNDO Selection of vertical segments to recover $V[a,b]$ from $V[a,m_1]$.
       \STATE Use \textsc{SortedSubsetSelection} to select all vertical segments in the slab $[X_{med},X_{max}]$.
              The segments are placed in $V[a,m_1]$.
       \STATE Use \textsc{SortedSubsetSelection}, to select all horizontal segments that are in the slab $[X_{med},X_{max}]$ or span the slab $[X_{med},X_{max}]$ but
              DO NOT span the slab $[X_{min},X_{max}]$.
              The segments are placed in $H[c,m_2]$.
       \STATE OLSI($V[a,m_1],H[c,m_2]$).
   \ENDIF
  \end{algorithmic}
\end{algorithm}

The above algorithm is in standard recursive form. Since the recursion tree is balanced, the running time is \Oh{n \log n} 
and the amount of extra memory
required is \Oh{\log n} for the recursion stack. Prior to invoking the recursive algorithm, the segments in $H$ and $V$ are sorted 
in \Oh{n \log n} time. The key point is that an intersection between a horizontal and vertical segment is
reported only when the vertical segment is contained in the current slab and the horizontal segment completely spans the vertical slab.
All of the selection steps and their undo counter-parts are performed using SortedSubsetSelection in linear time with
constant extra memory. Selecting the
segments with maximum and minimum $X$-coordinate in $V[a,b]$ is trivial. Selecting the median can be done in linear expected time with \Oh{1}
extra memory using the algorithm described in Section \ref{sec:sel} or deterministically in linear time and \Oh{1} extra memory using
the algorithm described in Appendix \ref{sec:detkmed}.
The only steps of the algorithm still requiring explanation are Steps 6 and 14: how to report all of the intersections between vertical
segments within a slab and horizontal segments spanning a slab.

Prior to the scan, the vertical segments in $V[a,b]$ are sorted
by $Y$-coordinate of the upper vertices of the segments and the
horizontal segments in $H[c,m_2]$ are sorted by $Y$-coordinate. For
each vertical segment with endpoints $p,q$, we locate the index $i$
in $H$ of the horizontal segment with largest $Y$-coordinate whose
value is at most the $Y$-coordinate of $p$.  Starting at $i$, we
sequentially scan $H$ until we find the index $j$ of the first
horizontal segment whose $Y$-coordinate is more than the $Y$-coordinate
of $q$. Now, all of the segments in $H[i,j-1]$ intersect segment
$pq$. Given $i$, the number of segments of $H$ that are verified
is one more than the number of intersections. A simple scan, similar to the
merging of two sorted lists, shows that all of the $i$'s can be found in linear time.
Since a vertical segment only occurs in \Oh{\log n} slabs, the total amount of time
spent in this step is \Oh{n \log n + k} where $k$ is the number of intersections reports.
We outline the scan below.

\begin{algorithm}
  \caption{Scan($V[a,b],H[c,d]$): Report all intersections between vertical line segments in $V[a,b]$ and horizontal line
           segments in $H[c,d]$.} \label{alg:scan}
  \begin{algorithmic}[1]
    \REQUIRE All vertical segments are within a vertical slab and all the horizontal segments span the slab. All segments in
             $V$ are sorted by $Y$-coordinate of the upper vertex and all segments in $H$ are sorted by $Y$-coordinate.
    \ENSURE all intersections between a segment in $H$ and a segment in $V$ are reported. The order of the segments in $V$ and $H$
            remains unchanged.

\STATE C := a \COMMENT{C represents the index of the current vertical segment}.
\STATE T := index of first $Y$-coordinate in $H[c,d]$ that is smaller than $Y$ coordinate of $V[C]$.
\STATE R := T.
\WHILE{$C \leq b$}
      \WHILE{$Y$-coordinate of $H[R] > Y$-coordinate of $V[C]$}
            \STATE Report intersection and increment R by 1.
      \ENDWHILE
      \STATE increment C by 1.
      \STATE update T and set $R: = T$. 
\ENDWHILE

  \end{algorithmic}
\end{algorithm}


%% state theorem
\begin{theorem}
Given a set of $n$ vertical line segments in an array $V$ and a set of $m$ horizontal line segments in an array $H$, all
intersections between horizontal and vertical segments can be reported in \Oh{(n + m)\log n + k} time using \Oh{\log n} extra space where $k$ is
the number of intersections reported.
\end{theorem}

The main difficutly in transforming this algorithm into space-efficient recursive form stems from the difficulty of pre-computing
all of the base instances. Since a horizontal line segment can span a logarithmic number of slabs, we are unable
to precompute the base instances in a simple way.


%%% Local Variables:
%%% mode: latex
%%% TeX-master: "paper"
%%% End:

