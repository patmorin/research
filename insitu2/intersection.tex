%% -*- LaTeX -*- This is LaTeX2e code

\subsubsection{Orthogonal Line Segment Intersection}


In this section, we present a space-efficient algorithm for the
Orthogonal Line Segment Intersection problem. Our algorithm can be
seen a variant of the (external-memory) \emph{distribution sweeping}
approach taken by Goodrich \etal~\cite{goodrich:external}.

The (internal memory) version of this algorithm is a top-down
divide-and-conquer algorithm, that is, the (algorithmic) work is done
prior to recursion. As a precondition, assume that all vertical
segments are sorted according to the $y$-coordinate of their lower
endpoint and that all horizontal segments are sorted according to
theit $y$-coordinate. In each step of the recursion, the set of
vertical segments is split according to the median $x$-coordinate.
These sets define two \emph{slabs} that are swept top-down. All
horizontal segments that completely span a slab are pushed onto a
stack corresponding to their stack. Whenever (the lower endpoint of) a
vertical segment is encountered, the stack corresponding to the slab
containing the segment is scanned and all intersecting segments are
reported. After this sweep all horizontal segments (or fragments
thereof) not completely spanning a slab are distributed to the
corresponding slab which in turn is processed recursively.

The first observation that helps making this algorithm \emph{in-place}
is that the ``stack'' used in the top-down sweep is never accessed
using \emph{push}-operations. Instead, all horizontal segments are
pushed onto this stack in sorted order. This means that any sorted
(part of an) array in connection with a single pointer indicating the
current ``top'' of the ``stack'' can be used to implement this part of
the algorithm.

A much more challenging problem is that the recursion tree
corresponding to the algorithm is traversed in an \emph{inorder}
fashion, i.e., the general framework of Section~\ref{sec:dandc}
cannot be used. Our approach\ldots

\begin{itemize}
\item  Do ``revertable'' median-split according to $x$.
\item  ``Left'' segments are sorted in the opposite direction as 
  ``right'' segments. Alternate reversal on each level of the
  recursion, i.e., always revert the ``left'' segments w.r.t. to the
  last order encountered.
\item Claim: Can always reconstruct left, right, and median
  $x$-coordinate. 
\end{itemize}

\begin{itemize}
\item  Unless it is clear, how to report intersections
  space-efficently, just do counting\ldots
\item  This algorithm uses median-finding which can be done
  \emph{in-place} using the algorithm of Carlsson and 
  Sundstr\"om~\cite{carlsson:linear}. 
\item  Divide and conquer without cutting vertical segments that cross
  the ``divide'' line. Instead, move each such segment down one
  level when starting recursion and bring it back up when returning
  from recursion. Consequence: each segment is stored exactly once.
\item Breakdown of the algorithm:
  \begin{itemize}
  \item Sort all horizontal segments by their $y$-coordinate. These
    horizontal segments are stored in an array $h[1\ldots H]$.
  \item  Sort all vertical segments by their lower endpoint. These
    vertical segments are stored in an array $v[1\ldots V]$.
  \item  Maintain four index pointers: $h_b,h_e$ to indicate the
    subarray of array $h$ currently under inspection, and $v_b,v_e$ to
    indicate the subarray of array $v$ currently under inspection.
    Initially: $h_b = v_b = 1$, $h_e = H$, and $v_e=V$.
  \item repeat:
    \begin{itemize}
    \item  Use median-find to find the median $m$ of the $x$ coordinates
      of the vertical segments stored in $v[v_b,v_e]$.
    \item In-place partition the vertical segments such as to respect
      the median while maintaining their
      $y$-order~\cite{katajainen:partitioning} (\Oh{1} extra space).
      
    \item The array $a[1\ldots n/2]$ in each slab is subdivided into
      three parts using two indices $1\leq c\leq d \leq n/2+1$ such that
      the elements $a[c\ldots d-1]$ are on the stack, the elements
      $a[d\ldots n/2]$ have not been swept yet. Initially, $c=d=1$.
    \item \ldots
    \item Scan the array $a[c\ldots d-1]$ with index $i$: if $a[i]$ is
      not crossed by the segment, set $c=i+1$, else report the
      intersection. Correctness: At each point in time, the segments in
      the array are ordered by their lower endpoint.
    \item  
    \end{itemize}
    until $v_b=1$ and $v_e = V$ (i.e. until the recursion tree has
    been traversed completely.
  \end{itemize}
\end{itemize}

%%% Local Variables:
%%% mode: latex
%%% TeX-master: "paper"
%%% End:

