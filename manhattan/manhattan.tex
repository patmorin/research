\documentclass[lotsofwhite]{patmorin}

\input{pat}

\title{\MakeUppercase{Manhattan Graph Drawing}}
\author{Stefan Langerman \and Pat Morin \and Alexander Wolff}
\date{}

\begin{document}
\maketitle

Let $G=(V,E)$ be a finite undirected graph with vertex set
$V\subset\R^2$.  We consider the problem of embedding $G$ in the plane
so that
\begin{enumerate}
\item $v_i$ is drawn as a point at location $v_i$,
\item $e_i$ is drawn as an $x$ and $y$ monotone curve, and
\item no two edges intersect except possibly at their common endpoints.
\end{enumerate}
We call such an embedding a \emph{Manhattan embedding} of $G$.

\section{The Algorithm}

A \emph{geodesic segment} is a curve whose intersection with any
vertical or horizontal line consists of at most one point. A
\emph{geodesic (Jordan) arc} is an infinite geodesic segment.

The \emph{precedence graph} $P(G)=(E,X)$ is a directed graph whose
vertices are the edges of $V$ and whose edges represent above/below
relationships.  An edge $(e_1,e_2)$ is present in $G'$ if and only if
for every pair of geodesic Jordan arcs $a_1$ and $a_2$ such that
\begin{enumerate}
\item $a_i$ contains both endpoints of $e_i$, for $i\in\{1,2\}$ and
\item $a_1\cap a_2=\emptyset$
\end{enumerate}
$a_1$ is above $a_2$ (see \figref{X}).
\notice{What about when one the edges can't be completed without
intersecting the other?}

\begin{lem}
If $e_1\prec e_2$ then either $e_1$ and $e_2$ are separated by a
vertices line
\end{lem}

\begin{lem}
The graph $G$ has a Manhattan embedding if and only if $P(G)$ is
acyclic.
\end{lem}

\begin{proof}
If $P(G)$ contains a cycle then\ldots

If $P(G)$ is acyclic then topologically sort the edges of $G$ to
obtain a total order $(E,\prec)$. We show how to obtain a Manhattan
drawing of $G$ using a \emph{plane sweep} \cite{boXX}.  We sweep a
vertical line from left to right over the vertices of $G$.  During the
sweep, there are three kinds of edges of $G$:
\begin{enumerate}
\item The \emph{complete edges} have both endpoints to the left of
the sweep line.  These edges are already drawn as geodesic segments.

\item The \emph{partial edges} have one endpoint on each side of the
sweep line.  These edges are partially drawn as a geodesic arc that
begins at the vertex to the left of the line and ends on the sweep
line.

\item The \emph{untouched edges} have both endpoints to the right of
the sweep line.  These edges are not drawn at all.
\end{enumerate}

Sweep line events occur at the vertices of $G$.  Between the vertices
of $G$, the partial edges are extended horizontally as the sweep line
moves.  When the sweep line reaches a vertex of $u$ zero or more new
partial edges are created at $u$, zero or more partial edges are
terminated at $u$ and become complete edges, and a number of
prexisting partial edges have a vertical edge inserted so that they
pass above or below $u$, as appropriate. 

More precisely, let $e_1,\ldots,e_k$ be the partial edges of $G$,
ordered from top to bottom, before the sweep line encounters $u$ and
let $f_1,\ldots,f_\ell$ be the edges whose left endpoint is $u$.  Note
that $e_1\prec e_2\prec\cdots\prec e_k$ and that there is some value
$1<i\le k$ such that $e_i\prec f_1,\ldots,f_\ell\prec e_{i+1}$.  

\end{proof}

\end{document}

