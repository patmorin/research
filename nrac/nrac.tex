\documentclass[lotsofwhite,charterfonts]{patmorin}
\usepackage{amsopn,graphicx,pat}

\DeclareMathOperator{\ch}{ch}
\DeclareMathOperator{\ind}{indeg}
\DeclareMathOperator{\outd}{outdeg}


\title{\MakeUppercase{Notes on Nearly Right-Angle Crossing Graphs}}
\author{Vida Dujmovi\'c 
	\and Joachim Gudmundsson
	\and Pat morin
	\and Thomas Wolle}

\begin{document}
\maketitle

\begin{abstract}
A plane graph $G$ is an $\alpha$ angle crossing ($\alpha$AC) graph if
every pair of crossing edges in $G$ intersect at an angle of at least
$\alpha$.  Eades \etal\ studied $\pi/2$-angle crossing graphs, called
right angle crossing (RAC) graphs, and showed that any RAC graph with
$n$ vertices has at most $4n-10$ edges and that there exists infinitely
values of $n$ for which there exists a RAC graph with $n$ vertices and
$4n-10$ edges.  In the current work, we give upper and lower bounds for
the number of edges in $\alpha$AC graphs for all $0 < \alpha < \pi/2$.
\end{abstract}

\section{Introduction}

For the purposes of this work, a \emph{plane graph} is a graph drawn in the
plane using straight line segments to represents its edges, with no vertex
in the interior of any edge.  A plane graph $G$ is an \emph{$\alpha$ angle
crossing} (\emph{$\alpha$AC}) graph if every pair of crossing edges in $G$
intersect at an angle of at least $\alpha$.

For $\alpha\ge 70^\circ$, $\alpha$AC graphs are of particular practical
relevance. This is supported by the work of Hong \etal\ \cite{hXX} who
showed, through eye-tracking experiments, that crossings that occur at
angles of greater than $70^\circ$ have very little effect on humans'
abilities to interpret graphs.  Therefore, graph drawings with crossing are
not bad, so long as the crossings occur with large angles between them.

This raises the question of how much information can be conveyed in such
graph drawings.  Clearly, $\alpha$AC graphs are more general than planar
graphs, but how much more so?  One measure of generality is the maximum
number of edges such a graph can represent.  Euler's Formula implies that a
planar graph with $n$ vertices has at most $3n-6$ edges.  How many edges
can an $\alpha$AC graph have?

\subsection{Previous Work}

Eades \etal\ studied $\pi/2$-angle crossing graphs, called right angle
crossing (RAC) graphs, and showed that any RAC graph with $n$ vertices has
at most $4n-10$ edges and that there exists infinitely values of $n$ for
which there exists a RAC graph with $n$ vertices and $4n-10$ edges. 

For $\alpha > \pi/3$, an $\alpha$AC graph has no three edges that
mutually intersect since, otherwise, one of the pairs of edges must
intersect at an angle that is at most $\pi/3$.  Geometric graphs with no
three pairwise crossing edges are known as \emph{quasiplanar graphs}
\cite{aapps97}. Ackerman and Tardos \cite[Theorem~5]{atXX} have shown that
any quasiplanar graph on $n$ vertices has at most $6.5n - 20$ edges. 

For $\alpha > \pi/4$, an $\alpha$AC graph has no 4 pairwise crossing edges.
Ackerman has shown that any such graph has at most $36n - 72$ edges.  It
remains an open problem whether, for any $k\ge 5$, a graph with no
$k$-pairwise crossing edges has a linear number of edges.  The best known
upper bound of $O(n\log n)$ on the number of edges in such a graph is due
to Valtr \cite[Theorem~3]{v99}.

\subsection{New Results}

The current paper gives upper and lower bounds on the number of edges in
$\alpha$AC graphs.  In \secref{uniform} we show that, for any $0< \alpha
<\pi/2$, the maximum number of edges in a $\alpha$AC graph is at most
$(\pi/\alpha)(3n-6)$.  In \secref{lower-bounds}, we give constructions that
essentially match this upper bound when $\alpha = \pi/2k-\epsilon$ or when
$\alpha = \pi/ 3k-\epsilon$, for any integer $k \ge 1$ and any $\epsilon >
0$.\footnote{Pat: I'm not so sure about this.}  Finally, in
\secref{charging} we use a charging argument like that used by Ackerman and
Tardos to prove that, for $2\pi/5 < \alpha < \pi/2$, the number of edges in
a $\alpha$AC graph is bounded by $6n-12$.


\section{A Uniform Upper Bound}
\seclabel{uniform}

In this section, we give an upper bound of $(\pi/\alpha)(3n-6)$ on the
number of edges in an $\alpha$AC graph.  This upper bound captures the
intuition that an $\alpha$AC is the union of $\pi/\alpha$ planar graphs.
The only trouble with this intuition is that $\pi/\alpha$ is not
necessarily an integer so we have a problem of determining the number of
edges in a fraction of a planar graph.

\begin{thm}
Let $G$ be a $\alpha$AC graph with $n$ vertices, for some $0<\alpha<\pi/2$.
Then $G$ has at most $(\pi/\alpha)(3n-6)$ edges.
\end{thm}

\begin{proof}
Define the \emph{direction} of an edge $xy$ whose lower endpoint is $x$ (in
the case of a horizontal edge, take $x$ as the left endpoint) as the angle
between $\angle wxy$ where $w=x+(1,0)$.  The direction of an edge $xy$ is
therefore a real number in the interval $[0,\pi)$.  Now, take a random
rotation $G'$ of $G$ and partition the edges of $G'$ into groups
$G_1,\ldots,G_{r}$ where $r=\lceil\pi/\alpha\rceil$, and $G_i$ contains all
edges of $G'$ in the interval $[\alpha(i-1),\alpha i)$.

Note that no two edges of $G_i$ cross each other, so each $G_i$ is
a planar graph that, by Euler's Formula, has at most $3n-6$ edges.
Furthermore, since $G'$ is a random rotation, the expected number of
edges in $G_{r}$ is $(\pi\bmod \alpha) |E(G)|$.  In particular, there
must exist some rotation $G'$ of $G$ such that $|E(G')| \le (\pi\bmod
\alpha) |E(G)|$.  Therefore,
\begin{equation}
   E(G) \le \lfloor \pi/\alpha \rfloor(3n-6) + (\pi\bmod\alpha)|E(G)| \enspace .
   \eqlabel{uniform}
\end{equation}
Rearranging \eqref{uniform} yields
\[  
  |E(G)| 
    \le  \frac{\lfloor \pi/\alpha \rfloor(3n-6)}{1-\pi\bmod\alpha} 
    = (\pi/\alpha)(3n-6) \enspace ,
\]
as required.
\end{proof}

\section{Lower Bounds}
\seclabel{lower-bounds}

\begin{thm}
For any $\epsilon > 0$,  there exists
$(\pi/2-\epsilon)AC$ graphs that have $n$ vertices and $6n- o(n)$ edges.
\end{thm}

\begin{proof}
The construction is based on the square lattice.
\end{proof}

\begin{thm}
For any $\epsilon > 0$, there exists
$(\pi/3-\epsilon)AC$ graphs that have $n$ vertices and $9n- o(n)$ edges.
\end{thm}

\begin{proof}
The construction is based on the hexagonal lattice.
\end{proof}

\begin{thm}
For any $\epsilon > 0$, there exists
$(\pi/4-\epsilon)AC$ graphs that have $n$ vertices and $12n- o(n)$ edges.
\end{thm}

\begin{proof}
The construction is based on the square lattice.
\end{proof}

\begin{thm}
For any $\epsilon > 0$, there exists
$(\pi/6-\epsilon)AC$ graphs that have $n$ vertices and $18n- o(n)$ edges.
\end{thm}

\begin{proof}
The construction is based on the hexagonal lattice.
\end{proof}

Can we generalize this to the result claimed in the introduction?


\section{Charging Arguments}
\seclabel{charging}

In this section we derive upper bounds using charging arguments like those
used by Ackerman and Tardos \cite{atXX} and Ackerman \cite{atXX}.  Let $G$
be an $\alpha$AC graph.  We denote by $G'$ the planar graph obtained by
introducing a vertex at each point in which a pair of edges in $G$ crosses
(thereby subdividing) two edges of $G$.  

For a face $f$ of $G'$, denote by $|f|$ the number of steps taken while
traversing the boundary of $f$ in counterclockwise order so that, if an
edge is crossed twice during the traversal, then it contributes twice to
$|f|$.  Let $v(f)$ denote the number of steps of this traversal during
which a vertex of $G$ (as opposed to a vertex introduced in $G'$) is
encountered.  For each face $f$ of $G'$ define the \emph{initial charge} of
$f$ as
\[
    \ch(f) = |f| + v(f) - 4  \enspace .
\]
Ackerman and Tardos show, using two applications of Euler's formula, that
\[
    \sum_{f\in G'} \ch(f) = 4n-8 \enspace .
\]
We call a face $f$ of $G'$ a $k$-\emph{shape} if $v(f)=k$ and $f$ is a
\emph{shape}.  For example, a 2-pentagon is a face of $G'$ with $|f|=5$ and
$v(f)=2$.

As a warm-up, and introduction to charging arguments, we offer an alternate
proof of Eade's \etal\ 's upper bound.

\begin{thm}
A RAC graph with $n \ge 4$ vertices has at most $4n-10$ edges.
\end{thm}

\begin{proof}
Let $G$ be a maximal RAC graph on $n$ vertices, and define $G'$ and $\ch$
as above.  We claim that, for every face $f$ of $G'$, $\ch(f)\ge v(G)/2$.
To see this, observe that the claim is certainly true if $|f| \ge 4$.  On
the other hand, if $|f|=3$ then, by the RAC property, $v(f) \ge 2$, so it
is also true in this case.
Therefore,
\[
 4n-8 = \sum_{f\in G'} \ch(f) \ge
 \sum_{f\in G'} v(f)/2 = \sum_{v\in G} \deg(v)/2 = |E(G)| \enspace ,
\]
which proves that $E(G)\le 4n-8$.  

To improve the above bound, observe that, since $G$ is maximal all vertices
on the outer face, $f$, of $G'$ are vertices of $G$.  If $|f| \ge 4$ then
$\ch(f) \ge v(f)/2 + 2$, so in this case, proceeding as above, we have
\[
    4n-8-2 \ge |E(G)| 
\]
and we are done.  Otherwise, the outer face of $G'$ is a 3-triangle and
$\ch(f) = v(f)/2 + 1/2$.  Consider the internal faces of $G'$ incident on
the three edges of $f$.  Because $G$ is maximal, and $n\ge 4$, there must
be three such faces and each of these three faces, $f'$, has $v(f') \ge 2$.
Furthermore, at most one of these faces is a 2-triangle.\footnote{This is
proven by a simple geometric argument that shows for any triangle $f$,
two right-angle triangles that are interior to $f$ and each share an
edge with $f$ must overlap.}
A straightforward case analysis shows that the other
two faces must have $\ch(f') \ge v(f')/2 + 1/2$, with equality if and only
if $f'$ is a 3-triangle.  Therefore, we have
\[
    4n-8-3/2 \ge |E(G)| 
\]
which, implies that $|E(G)| \le 4n-10$ since $|E(G)|$ is an integer.
\end{proof}


Next, we prove an upper bound for $\alpha > 2\pi/5$ that improves on the
$6.5n-20$ upper bound that follows from Ackerman and Tardos' bound on
quasiplanar graphs.

\begin{thm}\thmlabel{six-n}
Let $G$ be an $\alpha$AC graph with $n$ vertices, for $\alpha > 2\pi/5$.
Then $G$ has at most $6n-12$ edges.
\end{thm}

\begin{proof}
We will redistribute the charge in the $G'$ to obtain a new charge $\ch'$
such that $\ch'(f) \ge v(f)/3$ for every face $f$ of $G$.  In this way, we
get
\[
 4n-8 = \sum_{f\in G'} \ch(f) 
   = \sum_{f\in G'} \ch'(f) \ge
 \sum_{f\in G'} v(f)/3 = \sum_{v\in G} \deg(v)/3 = 2|E(G)|/3 \enspace ,
\]
which we rewrite to get $|E(G)| \le 6n-12$.

The charge $\ch'(f)$ is obtained as follows.  Let $f$ be any 1-triangle of
$G'$.  (Note that $\ch(f) = 0$.)   That is, $f$ is a triangle formed by two
edges $e_1$ and $e_2$ that meet at a vertex $x$ of $G$ and an edge $e$ that
crosses $e_1$ and $e_2$.
Imagine walking along the bisector of $e_1$ and $e_2$ (starting in the
interior of $f$) until reaching a face $f'$ such that $f'$ is not a
0-quadrilateral.  To see why such an $f'$ exists, observe that if we
encounter nothing but 0-quadrilaterals we will eventually reach a face that
contains an endpoint of $e_1$ or $e_2$ and is therefore not a
0-quadrilateral.

Adjust the charges at $f$ and $f'$ by subtracting $1/3$ from $\ch(f')$ and
adding $1/3$ to $\ch(f)$.  It is helpful to think of the charge as leaving
$f'$ through the last edge $e'$ traversed in the walk.  Note that neither
endpoint of $e'$ is a vertex of $G$.  This implies that for a face $f'$,
the amount of charge that leaves $f'$ is at most 
\begin{equation}
   \ell(f') \le \left\{
            \begin{array}{ll}
              |f'|         & \mbox{if $v(f')=0$} \\
              |f'| - v(f') - 1 & \mbox{otherwise.}
            \end{array}
          \right.
  \eqlabel{leaving}
\end{equation}
Let $\ch'$ be the charge obtained after performing this redistribution of
charge for every 1-triangle $f$.  We claim that $\ch'(f) \ge v(f)/3$.  To
see this, we need only run through a few cases that can be verified using
\eqref{leaving} and the following observations:

\begin{enumerate}
\item If $|f|\ge 6$, then $\ell(f) \le |f|/3$, so $\ch'(f) \ge v(f) \ge v(f)/3$.

\item If $|f|=5$, then $v(f) \ge 1$ since, otherwise, $f$ has two edges on
its boundary that cross at an angle of less than or equal to $2\pi/5$.

\item If $|f|=4$, and $f$ is a 0-quadrilateral then $\ell(f)=0$, by construction.

\item If $|f|=3$, and $f$ is a 1-triangle then $\ch'(f)=1/3$, by construction. 

\item If $|f|=3$ then $v(f)\ge 1$ since, otherwise, $f$ has two edges on
its boundary that cross at an angle less of at most $\pi/3 < 2\pi/5$.
\end{enumerate}
This completes the proof.
\end{proof}


\section{Notes}

\thmref{six-n} appears to be true even for $\alpha > \pi/3$, but we can't
prove it yet.  The problem occurs because 0-pentagons can finish with
a charge of $-2/3$ or $-1/3$. (See \figref{pentagons}.) 

\begin{figure}
  \begin{center}
    \includegraphics{pentagons}
  \end{center}
  \caption{Pentagrams lead to 0 pentagons with negative charge.}
  \figlabel{pentagons}
\end{figure}

A proof could maybe look for extra charge near the vertices of the
penta\emph{gram} that created this pentagon, but it's easy to make gadgets
so that the faces surrounding those vertices have no extra charge.  Another
option is to look for extra charge near the vertices of the penta\emph{gon}
itself. Again, it's not too hard to to make them have no extra charge.
(See \figref{p2}.)

\begin{figure}
  \begin{center}
    \includegraphics{p2}
  \end{center}
  \caption{Pentagrams can have 0 extra charge at their vertices and
           0 extra charge at the vertices of a pentagon.}
  \figlabel{p2}
\end{figure}


We've also tried the option of following the Ackerman-Tardos proof more
closely. Namely, we distribute charge so that $\ch'(f)\ge v(f)/5$ and then
prove that there is leftover charge at the faces around each vertex.  For
this to give a bound of $6n$ we would need the extra charge at each vertex
to be $8/5$.  Unfortunately, the limiting case in Ackerman-Tardos is $7/5$
and this is realizable even with crossing angles arbitrarily close to
$\pi/2$. (See \figref{at-bound}.)

\begin{figure}
  \begin{center}
    \includegraphics{at-bound}
  \end{center}
  \caption{The Ackerman-Tardos proof can't even prove a bound of $6n$ for
           crossing angles of $\pi/2-\epsilon$.}
  \figlabel{at-bound}
\end{figure}

Finally, we can take a more global approach.  Discharging rules define a
directed graph among the faces (and possibly vertices) of $G'$.  An edge
$ab$ indicates that a charge of $x$ travels from $a$ to $b$, for some
number $x$ ($x=1/3$ in our argument).  The graph has to respect some flow
rules.  For example, in \thmref{six-n} we have 
\[
    \outd(a) - \ind(a) \le 3(|f| + 2v(f)/3 - 4) \enspace ,
\]
where $\ind$ and $\outd$ denote the in and out degree.  The goal would be
to define discharging paths recursively and then show that the recursion
terminates (i.e., that the resulting graph is acyclic) and that the flow
rule is satisfied.








\end{document}
