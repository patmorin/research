\documentclass{article}
\usepackage{amsthm,amsfonts}

 
%\usepackage{amsthm}

\newcommand{\centeripe}[1]{\begin{center}\Ipe{#1}\end{center}}
\newcommand{\comment}[1]{}

\newcommand{\centerpsfig}[1]{\centerline{\psfig{#1}}}

\newcommand{\seclabel}[1]{\label{sec:#1}}
\newcommand{\Secref}[1]{Section~\ref{sec:#1}}
\newcommand{\secref}[1]{\mbox{Section~\ref{sec:#1}}}

\newcommand{\alglabel}[1]{\label{alg:#1}}
\newcommand{\Algref}[1]{Algorithm~\ref{alg:#1}}
\newcommand{\algref}[1]{\mbox{Algorithm~\ref{alg:#1}}}

\newcommand{\applabel}[1]{\label{app:#1}}
\newcommand{\Appref}[1]{Appendix~\ref{app:#1}}
\newcommand{\appref}[1]{\mbox{Appendix~\ref{app:#1}}}

\newcommand{\tablabel}[1]{\label{tab:#1}}
\newcommand{\Tabref}[1]{Table~\ref{tab:#1}}
\newcommand{\tabref}[1]{Table~\ref{tab:#1}}

\newcommand{\figlabel}[1]{\label{fig:#1}}
\newcommand{\Figref}[1]{Figure~\ref{fig:#1}}
\newcommand{\figref}[1]{\mbox{Figure~\ref{fig:#1}}}

\newcommand{\eqlabel}[1]{\label{eq:#1}}
\newcommand{\eqref}[1]{(\ref{eq:#1})}

\newtheorem{thm}{Theorem}{\bfseries}{\itshape}
\newcommand{\thmlabel}[1]{\label{thm:#1}}
\newcommand{\thmref}[1]{Theorem~\ref{thm:#1}}

\newtheorem{lem}{Lemma}{\bfseries}{\itshape}
\newcommand{\lemlabel}[1]{\label{lem:#1}}
\newcommand{\lemref}[1]{Lemma~\ref{lem:#1}}

\newtheorem{cor}{Corollary}{\bfseries}{\itshape}
\newcommand{\corlabel}[1]{\label{cor:#1}}
\newcommand{\corref}[1]{Corollary~\ref{cor:#1}}

\newtheorem{obs}{Observation}{\bfseries}{\itshape}
\newcommand{\obslabel}[1]{\label{obs:#1}}
\newcommand{\obsref}[1]{Observation~\ref{obs:#1}}

\newtheorem{assumption}{Assumption}{\bfseries}{\rm}
\newenvironment{ass}{\begin{assumption}\rm}{\end{assumption}}
\newcommand{\asslabel}[1]{\label{ass:#1}}
\newcommand{\assref}[1]{Assumption~\ref{ass:#1}}

\newcommand{\proclabel}[1]{\label{alg:#1}}
\newcommand{\procref}[1]{Procedure~\ref{alg:#1}}

\newtheorem{rem}{Remark}
\newtheorem{op}{Open Problem}

\newcommand{\etal}{\emph{et al}}

\newcommand{\voronoi}{Vorono\u\i}
\newcommand{\ceil}[1]{\left\lceil #1 \right\rceil}
\newcommand{\floor}[1]{\left\lfloor #1 \right\rfloor}



\title{Random Sampling and Tukey Depth}

\begin{document}
\maketitle

\section{A Lemma}
For a set $A$ of hyperplanes in $\R^d$ and two points $p,q\in\R^d$,
define the $A$-distance from $p$ to $q$, denoted $\|pq\|_A$, as the
number of hyperplanes in $A$ that separate $p$ from $q$.  Define the
$A$-ball of radius $r$ centered at $p$ as
\[ 
   B_A(p,r) = \{q\in\R^d : \|pq\|_A \le r\} \enspace .
\]

\begin{lem}[Welzl --- Lemma 2.1]\lemlabel{ball-2}
Let $A$ be an arrangement of $n$ lines in the plane, let $0\le r\le
n/2$, and let $p$ be a point in the interior of a face of $A$.  Then,
\[
    |B_A(p,r) \cap V(A)| \ge {r+1 \choose 2} \enspace .
\]
\end{lem}

Here is a first extension to $\R^d$:
\begin{lem}\lemlabel{ball-d}
Let $A$ be an arrangement of $n$ planes in $\R^d$ ($d\ge 2$),  let
$0\le r\le n/2$, and let $p$ be a point in the interior of a cell of
$A$.  Then,
\[
    |B_A(p,r) \cap V(A)| \ge \frac{2^{d-1}r^d}{d!d!} \enspace .
\]
\end{lem}
\begin{proof}
The proof is by induction on $d$, with the base case being
\lemref{ball-2}.

Select a directed line $\ell$, containing $p$, not intersecting any
$(d-1)$-dimensional face of $A$ and such that the number of 
intersections of $\ell$ with
$A$ on each side of $p$ is at least $r$.  (The existence of such a
line is guaranteed by a simple continuity argument.)  Let
$\pi_0,\ldots,\pi_{r}$ be the hyperplanes of $A$ that intersect 
$\ell$ after $p$ ordered by increasing distance from $p$.

The inductive hypothesis and the triangle inequality imply that, on
the hyperplane
$\pi_i$, there are at least
\[
  \frac{2^{d-2}(r-i)^{d-1}}{(d-1)!(d-1)!} 
\]
vertices of $A$ whose $A$-distance from $p$ does not exceed $r$.
Counting this way, we count 
\[
  \sum_{i=0}^r \frac{2^{d-2}i^{d-1}}{(d-1)!(d-1)!} 
     \ge \frac{2^{d-2}r^{d}}{d!(d-1)!}
\]
vertices of $A$ that are at $A$-distance at most $r$ from $p$.
Repeating the same argument on the hyperplanes
that intersect $\ell$
before $p$ doubles the above quantity.  Since this counts each vertex
of $A$ at most $d$ times we obtain the desired result.
\end{proof}


\section{The Algorithm}

We can use \lemref{ball-d} to obtain a very simple additive
$r$-approximation algorithm for computing $\td(p,S)$.  In the dual,
the point $p$ becomes a hyperplane (a $(d-1)$-flat) $\pi$ and the
hyperplane $h$ containing $p$ and cutting off $\td(p,S)$ points from
$S$ is a point $q\in \pi$.
The
intersection of $\pi$ with the dual hyperplanes of $S$ gives a
$(d-1)$-dimensional arrangement $A$ \ldots 
Sample a bit then do sort-and-scan.
\end{document}



