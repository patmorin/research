\documentclass[lotsofwhite]{patmorin}



\begin{document}
We want a data structure that supports the following operations:

\begin{enumerate}
\item $\textsc{Build}(x_1,\ldots,x_m)$.  Build a data structure that
can manipulate a polynomial and quickly evaluate it at any $x_i$,
$1\le i\le m$.

\item $\textsc{Add}(c)$.  If the data structure currently represents
the polynomial $c_nx^n+\cdots+c_1x+c_0$ then modify it so that it
represents $c_nx^n+\cdots+c_1x+(c_0+c)$.

\item $\textsc{Multiply}$.  If the data structure currently represents
$c_nx^n+\cdots+c_1x+c_0$ then modify it so that it represents
$c_nx^{n+1}+\cdots+c_1x^2+c_0x$.

\item $\textsc{Evaluate}(i)$.  If the data structure currently
represents $c_nx^n+\cdots+c_1x+c_0$ then output $c_nx_i^n+\cdots
+c_1x_i+c_0$.
\end{enumerate}

In order to do this we use a grouping scheme with $O(\log n)$ groups.
The group $G_i$ is responsible for the subpolynomial:
\[
    P_i(x) =c_kx^k + c_{k-1}x^{k-1} + \cdots + c_{k-l}x^{k-l}  \enspace .
\]
where we maintain the invariant that $k$ is in the range
$[2^{i-1},2^{i}]$ and $l$ is in the range $[2^{i-2},2^{i-1}]$.
Obviously this can be done in such a way that the subpolynomial
assigned to $G_i$ only has to change after every $\Theta(2^{i})$
\textsc{Multiply} operations.

Each group $G_i$ will maintain a subset of indices
$j\in\{1,\ldots,m\}$ for which it has precomputed the value of
$P_i(x_j)$.  To execute the command $\textsc{Evaluate}(j)$ some groups
$G_i$ will have $P_i(x_j)$ precomputed and some groups won't.  For
those that do, we will use the precomputed value.  For those that
don't, we will compute $P_i$ at $2^i$ different values in
$x_1,\ldots,x_m$ that we haven't already precomputed, including the
value $x_j$.  This takes $\Theta(2^i\log 2^i)$ time using the FFT
convolution technique for evaluating a degree $2^i$ polynomial at
$2^i$ different locations.

To analyze the this data structure we let $K$ denote the total number
of pairs $(i,j)$ such that the value $P_i(x_j)$ has not yet been
precomputed.  Note that $0\le K\le m\log n$.   Let $\phi(D)$ denote
the usual potential function used for showing that logarithmic
hierarchies have $O(\log n)$ amortized insertion time.\footnote{For
insertion only hierarchies, this is usually $\phi(G_0,\ldots,G_{\log
n})= \sum_{i=0}^{\log n}
|G_i|(\log n - i)$.} Then we use
the potential function
\[
 \phi'(D) = c_1\phi(D)\log n + c_2K\log n  \enspace .
\]
Note that, with this potential function, the amortized cost of an
\textsc{Evaluate} operation is $O(\log n)$ since the expensive
precomputation steps are accompanied by a drop in the value of $K$. All that remains to show is that the amortized cost of
\textsc{Multiply} is not too high.  But this is trivial, since
rebuilding a group $G_i$ increases $K$ by at most $|G_i|$ so this
increase in potential gets killed off by the $\phi(D)$ term.  This
gives an amortized insertion time of $O(\log^2 n)$ and an amortized
query time of $O(\log n)$.

Again, we can also easily augment this data structure to support the operation
\begin{enumerate}
\item \textsc{Divide}.  If the data structure currently represents
$c_nx^n+\cdots+c_1x + c_0$ then modify it so that it represents
$c_nx^{n-1}+\cdots+c_2x + c_1$.
\end{enumerate}


\end{document}
