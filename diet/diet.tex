\documentclass{DIKU-article}
\usepackage{charter}
\usepackage[dvips]{graphicx}

%\renewcommand{\baselinestretch}{1.3}
% 
%\usepackage{amsthm}

\newcommand{\centeripe}[1]{\begin{center}\Ipe{#1}\end{center}}
\newcommand{\comment}[1]{}

\newcommand{\centerpsfig}[1]{\centerline{\psfig{#1}}}

\newcommand{\seclabel}[1]{\label{sec:#1}}
\newcommand{\Secref}[1]{Section~\ref{sec:#1}}
\newcommand{\secref}[1]{\mbox{Section~\ref{sec:#1}}}

\newcommand{\alglabel}[1]{\label{alg:#1}}
\newcommand{\Algref}[1]{Algorithm~\ref{alg:#1}}
\newcommand{\algref}[1]{\mbox{Algorithm~\ref{alg:#1}}}

\newcommand{\applabel}[1]{\label{app:#1}}
\newcommand{\Appref}[1]{Appendix~\ref{app:#1}}
\newcommand{\appref}[1]{\mbox{Appendix~\ref{app:#1}}}

\newcommand{\tablabel}[1]{\label{tab:#1}}
\newcommand{\Tabref}[1]{Table~\ref{tab:#1}}
\newcommand{\tabref}[1]{Table~\ref{tab:#1}}

\newcommand{\figlabel}[1]{\label{fig:#1}}
\newcommand{\Figref}[1]{Figure~\ref{fig:#1}}
\newcommand{\figref}[1]{\mbox{Figure~\ref{fig:#1}}}

\newcommand{\eqlabel}[1]{\label{eq:#1}}
\newcommand{\eqref}[1]{(\ref{eq:#1})}

\newtheorem{thm}{Theorem}{\bfseries}{\itshape}
\newcommand{\thmlabel}[1]{\label{thm:#1}}
\newcommand{\thmref}[1]{Theorem~\ref{thm:#1}}

\newtheorem{lem}{Lemma}{\bfseries}{\itshape}
\newcommand{\lemlabel}[1]{\label{lem:#1}}
\newcommand{\lemref}[1]{Lemma~\ref{lem:#1}}

\newtheorem{cor}{Corollary}{\bfseries}{\itshape}
\newcommand{\corlabel}[1]{\label{cor:#1}}
\newcommand{\corref}[1]{Corollary~\ref{cor:#1}}

\newtheorem{obs}{Observation}{\bfseries}{\itshape}
\newcommand{\obslabel}[1]{\label{obs:#1}}
\newcommand{\obsref}[1]{Observation~\ref{obs:#1}}

\newtheorem{assumption}{Assumption}{\bfseries}{\rm}
\newenvironment{ass}{\begin{assumption}\rm}{\end{assumption}}
\newcommand{\asslabel}[1]{\label{ass:#1}}
\newcommand{\assref}[1]{Assumption~\ref{ass:#1}}

\newcommand{\proclabel}[1]{\label{alg:#1}}
\newcommand{\procref}[1]{Procedure~\ref{alg:#1}}

\newtheorem{rem}{Remark}
\newtheorem{op}{Open Problem}

\newcommand{\etal}{\emph{et al}}

\newcommand{\voronoi}{Vorono\u\i}
\newcommand{\ceil}[1]{\left\lceil #1 \right\rceil}
\newcommand{\floor}[1]{\left\lfloor #1 \right\rfloor}


\newcommand{\comment}[1]{}
\newcommand{\floor}[1]{\left\lfloor #1 \right\rfloor}

\newcommand{\Findmin}{\mbox{$\mathit{find}$\textnormal{-}}\allowbreak{}\mbox{$\mathit{minimum}$}}
\newcommand{\Findmax}{\mbox{$\mathit{find}$\textnormal{-}}\allowbreak{}\mbox{$\mathit{maximum}$}}
\newcommand{\Member}{\mbox{$\mathit{find}$\textnormal{-}}\allowbreak{}\mbox{$\mathit{member}$}}
\newcommand{\Predecessor}{\mbox{$\mathit{find}$\textnormal{-}}\allowbreak{}\mbox{$\mathit{predecessor}$}}
\newcommand{\Successor}{\mbox{$\mathit{find}$\textnormal{-}}\allowbreak{}\mbox{$\mathit{successor}$}}
\newcommand{\Insert}{\mbox{$\mathit{insert}$}}
\newcommand{\Extract}{\mbox{$\mathit{extract}$}}
\newcommand{\Extractmin}{\mbox{$\mathit{extract}$\textnormal{-}$\mathit{minimum}$}}
\newcommand{\Extractmax}{\mbox{$\mathit{extract}$\textnormal{-}$\mathit{maximum}$}}
\newcommand{\Meld}{\mbox{$\mathit{meld}$}}
\newcommand{\Merge}{\mbox{$\mathit{merge}$}}
\newcommand{\Split}{\mbox{$\mathit{split}$}}
\newcommand{\Decrease}{\mbox{$\mathit{decrease}$}}
\newcommand{\Begin}{\mbox{$\mathit{begin}$}}
\newcommand{\End}{\mbox{$\mathit{end}$}}
\newcommand{\Borrow}{\mbox{$\mathit{borrow}$}}

\newcommand{\seclabel}[1]{\label{sec:#1}}
\newcommand{\secref}[1]{\mbox{Section~\ref{sec:#1}}}

\newcommand{\Null}{\mbox{$\mathit{null}$}}
\newcommand{\Operator}{\mbox{$\mathit{operator}$}}
\newcommand{\Pred}{\mathrm{pred}}
\newcommand{\Succ}{\mathrm{succ}} 
\newcommand{\Elem}{\mathrm{elem}} 
\newcommand{\Size}{\mathrm{size}}

\newcommand{\ceils}[1]{\lceil #1 \rceil}

\titlehead{Putting your data structure on a diet}
\authorhead{Herv\'e Br\"onnimann, Jyrki Katajainen, and Pat Morin}

\title{Putting your data structure on a diet}

\author{Herv\'e Br\"onnimann\inst{1}
\and
Jyrki Katajainen\inst{2}\fnmsep\thanks{%
Partially supported by the Danish Natural Science
Research Council under contract 272-05-0272
(project ``Generic programming---algorithms and tools'').%
}
\and
Pat Morin\inst{3}%
}

\institute{Department of Computer and Information Science, Polytechnic University\\
Six Metrotech, Brooklyn NY 11201, USA; \email{hbr@poly.edu}
\and
Department of Computing, University of Copenhagen\\
Universitetsparken 1, 2100 Copenhagen East, Denmark; \email{jyrki@diku.dk}
\and
School of Computer Science,
Carleton University\\
1125 Colonel By Drive,
Ottawa, Ontario, Canada K1S 5B6; 
\email{morin@cs.carleton.ca}%
}

\dates{CPH STL Report 2006-X, October 2006. Version: \today }

\begin{document}
\maketitle
\begin{abstract}
Consider a data structure $\mathcal{D}$ that stores a dynamic set or
multiset of elements. Assume that $\mathcal{D}$ uses a linear number
of words in addition to the elements stored.  In this paper several
data-structural transformations are described which can be used to
transform $\mathcal{D}$ into another data structure $\mathcal{D}'$
which has the same functionality as $\mathcal{D}$, has considerably
smaller memory overhead than $\mathcal{D}$, and performs the supported
operations by a small constant factor or a small additive term slower
than $\mathcal{D}$, depending on the data structure and operation in
question. The compaction technique has been successfully applied for
linked lists, ordered dictionaries, unordered dictionaries,
single-ended priority queues, and double-ended priority queues.

%Consider a data structure D that stores a dynamic set or multiset of
%elements. Assume that D uses a linear number of words in addition to
%the elements stored.  In this paper several data-structural
%transformations are described which can be used to transform D into
%another data structure D' which has the same functionality as D, has
%considerably smaller memory overhead than D, and performs the
%supported operations by a small constant factor or a small additive
%term slower than D, depending on the data structure and operation in
%question. The compaction technique has been successfully applied for
%linked lists, ordered dictionaries, unordered dictionaries,
%single-ended priority queues, and double-ended priority queues.
%
\end{abstract}

\begin{keywords}
Data structures, lists, dictionaries, priority queues, space efficiency
\end{keywords}

\section{Introduction}
\label{sec:intro}

In this paper we consider the space efficiency of data structures that
can be used for maintaining a dynamic set or multiset of
elements. Earlier research in this area has concentrated on the space
efficiency of some specific data structures or on the development of
implicit data structures. Our focus is on general data-structural
transformations that can be used to transform a data structure
$\mathcal{D}$ into another data structure $\mathcal{D}'$ that has the
same functionality as $\mathcal{D}$, has about the same efficiency as
$\mathcal{D}$, but uses significantly less space than $\mathcal{D}$.
In particular, we consider a single compaction technique that can be
applied with minor variations to several different data structures.

One particular aspect of concrete implementations of abstract data
types that has received much attention is the issue of memory
overhead.  \emph{Memory overhead} is any storage used by the data
structure beyond what is actually required to store the data elements.
The starting point for our research was the known implicit data
structures where the goal is to minimize the memory overhead.  A
classical example is the binary heap of Williams \cite{Wil64} which a
priority queue supporting the methods \Findmin{}, \Insert{}, and
\Extractmin{}. Another example is the searchable heap of Franceschini
and Grossi \cite{FG03} which is an ordered dictionary supporting the
methods \Member{}, \Predecessor{}, \Successor{}, \Insert{}, and
\Extract{}. A searchable heap is a complicated data structure and
because of the bit encoding techniques used it can only store a set,
not a multiset. One should observe that implicit data structures are
designed on the assumption that there is an infinite array available
to be used for storing the elements. In other words, it is assumed
that the whole memory is used for the data structure alone and no
other processes are run simultaneously. To relax from this assumption,
a resizable array could be used instead.  It is known that any
realization of a resizable array requires at least $\Omega(\sqrt{n})$
space for pointers and/or elements \cite{BCDMS99}, and realizations
exist that only require $O(\sqrt{n})$ extra space \cite{BCDMS99,KM01}.

We call a data structure \emph{elementary} if it only allows
element-based access (or key-based access). In particular, all
implicit data structures are elementary.  An important facility often
imposed by modern libraries (e.g.~C\texttt{++} standard library
\cite{ISO}) is to provide iterator-based access to elements. In
general, for efficiency reasons it might be advantageous to allow
application programs to maintain references, pointers, or iterators to
elements stored inside a data structure. For the sake of correctness,
it is important to keep these references valid at all times. We call
this issue \emph{referential integrity}. For general priority queues,
iterator-based access is essential since without it general \Extract{}
or fast \Decrease{} operations cannot be provided. For ordered
dictionaries, a finger can be seen as a special iterator, so to
support finger search iterator-based access must be possible.  In
\cite{CLRS01}, \emph{handles} were proposed as a general solution to
achieve referential integrity but, even if this technique works, it
alone would increase the memory overhead of the data structure in
question by $2n$ pointers.

For the fundamental data types, like the ordered-dictionary and
priority-queue data types studied in this paper, many implementations
are available, each having its own advantages and disadvantages.  In
earlier research, the issues considered include the repertoire of
supported operations, complexity of implementation, constants involved
in running times, worst-case versus average-case versus amortized
performance guarantees, memory overhead, and so on.  Many
implementations are also special because they provide certain
\emph{properties}, not provided by implicit data structures.  Examples
of such properties include the working-set property \cite{i01,st85b},
the queueish property \cite{il02}, the unified property
\cite{i01,st85b}, the ability to \Insert{} and \Extract{} in constant
time \cite{Fle96,hm82,LO88}, and the (static or dynamic) finger
property \cite{as96,bt80,c95,gmpr77,hm82}.  These special properties
are a large part of the reason that so many implementations of the
fundamental data types exist.  Many algorithmic applications require
data structures that have some of the special properties to ensure the
bounds on their running times or storage requirements.  Thus, it is
not always possible to substitute one implementation for another.

Several ordered-dictionary implementations, including variants of
height-balanced search trees \cite{Bro79}, self-adjusting search trees
\cite{And99,gr93,st85b}, and randomized search trees \cite{as96,p90},
have been proposed that reduce the extra storage to two pointers per
element. However, when an ordered dictionary has to support
iterator-based access to elements, additional pointers (like parent
pointers) are often needed. By using the child-sibling representation
of binary trees (see, e.g.~\cite[Section 4.1]{Tar83}) and by
compacting color bits in pointers (as proposed, for example, in
\cite{BK06}), red-black trees supporting iterator-based access are
obtained that only use about $2n$ words of memory in addition to the
$n$ elements stored.

Contrast to binary heaps, many priority-queue implementations rely on
pointers. Actually, pointer-based implementations are often necessary
to provide referential integrity, and to support general \Extract{} and
constant-time \Decrease{}. Of the known implementations, all need at
least $2n$ extra words \cite{DW93,MP05}, $3n$ extra words
\cite{EJK05}, or more \cite{DGST88,KST02}. In particular, observe that
in \cite{MP05} the data structure was claimed to be implicit, but it
actually uses $2n + O(1)$ words of extra storage. It was also claimed
that the space efficiency cannot be improved if \Decrease{} is to be
supported in constant time. This claim is in a direct contradiction to
the results presented in the present paper (cf.~Corollary
\ref{corollary:queue}).

Much research effort has gone into these attempts to reduce the memory
overhead of existing data structures.  For example, in a recent
paper, Blelloch \etal\ \cite{bmw03} describe a fairly intricate data
structure of size $O(\lg n)$ called a \emph{hand} that sits on top of
a degree-balanced search tree (e.g.~an $(a,b)$-tree \cite{hm82}) so
that it supports efficient finger searches.  However, it has long been
known that degree-balanced search trees easily support efficient
finger searches if we augment each node with three pointers
\cite{bt80}.  Thus, the main contribution of the hand is to eliminate
these three pointers per node. It is an encumbrance to a hand that its
construction takes logarithmic time since all iterator operations are
normally assumed to take constant time.

In this paper we elaborate upon a technique that can be used to make
\emph{any} pointer-based data structure that uses $cn$ words of extra
storage, for a positive constant $c$, into an equivalent data
structure that uses $(1 + \varepsilon) n$ words of extra storage, for
any $\varepsilon>0$ and sufficiently large $n > n(\varepsilon)$.  By
applying this simple technique, any application program can employ the
data structure (and its properties) without worrying about the memory
overhead. In some special cases, we can even make the amount of extra
space needed as small as $O(n/\lg n)$.

The compaction technique used by us is simple, and similar to the idea
used in degree-balanced search trees \cite{hm82} where each node
contains at least $a$ elements and at most $b$ elements for
appropriately chosen $a$ and $b$. That is, instead of operating on
elements themselves, we operate on groups of elements which we call
\emph{chunks}. Each chunk has a \emph{representative} and the
underlying data structure operates on these representatives. The
elements in a chunk determine an \emph{interval} of values. When the
compaction technique is applied for different data structures, there
are minor variations in the scheme depending on how chunks are
implemented (as a singly-linked list or an array), how elements are
ordered within chunks (in sorted or in unsorted or in some other
order), how the representative of a chuck is selected, and whether the
induced intervals are pairwise disjoint or not.

The technique itself is not new; it has explicitly been used, for
example, in context of AVL-trees in \cite{Mun86} and in context of
run-relaxed heaps in \cite{DGST88}. Unfortunately, in the latter paper
the technique is applied incorrectly because the issue of referential
integrity is ignored.  Our purpose in this paper is to advocate the
wide applicability and generality of the technique.  Also, the
technique seems to be of practical value so our plan is to verify this
experimentally.

The remainder of the paper is organized as follows.  In
\secref{elementary-dictionary}, we show how the compaction technique
can be used for reducing the memory overhead of any elementary ordered
dictionary. More specifically, we extend an old result of Munro
\cite{Mun86} by proving that any elementary dictionary can be
converted into an equivalent data structure that requires at most
$O(n/\lg n)$ extra space without affecting the asymptotic running
times of algorithms for searching, inserting, and deleting.  In
\secref{iterator-dictionaries}, we describe another application of the
technique to be able to transform any linear-size iterator-based
dictionary into a more space-efficient data structure. The resulting
data structure has the same functionality as the original, requires at
most $(1 + \varepsilon)n$ words of extra storage, and increases the
running times of the algorithms used in its manipulation by an
additive term of $O(1/\varepsilon)$, for any $\varepsilon > 0$ and
sufficiently large $n > n(\varepsilon)$.  In \secref{iterator-queues}
we prove a similar result for iterator-based priority queues.
Finally, in \secref{discussion} we mention some other applications of
the compaction technique and conclude with a few final remarks.

\section{Abstract Data Types Considered}
\seclabel{adts}

\subsection{Ordered Dicationaries}
\noindent\textbf{**Snipped from \secref{elementary-dictionaries}}

An \emph{ordered dictionary} is an abstract data type that stores a
set or multiset of elements drawn from an ordered universe
in sorted order.  In its elementary form, a realization should provide
(a subset of) the following methods:
\begin{description}
\item[\Member{}$(\mathcal{S}, x).$] Return a reference to an element in sequence
$\mathcal{S}$ that is equivalent to (compares equal to) element $x$.  The
function used in element comparisons is to be specified at the
construction time of $\mathcal{S}$. If $\mathcal{S}$ does not contain any matching member,
return a reference to a special \Null{} element.

\item[\Predecessor{}$(\mathcal{S}, x).$] Return a reference to the element in
sequence $\mathcal{S}$ which is greatest among all elements that are smaller than
element $x$. If no such element exists, return a reference to a
special \Null{} element.

\item[\Successor{}$(\mathcal{S}, x).$] Return a reference to the element in sequence
$\mathcal{S}$ which is smallest among all elements that are greater than element
$x$.  If no such element exists, return a reference to a special
\Null{} element.

\item[\Insert{}$(\mathcal{S}, x).$] Insert a copy of element $x$ into sequence $\mathcal{S}$.

\item[\Extract{}$(\mathcal{S}, x).$] Remove element $x$ from sequence $\mathcal{S}$. If no
matching element exists, do nothing.

\item[\Findmin$(\mathcal{S})$/\Findmax$(\mathcal{S}).$] Return a reference to an element
that, of all elements in sequence $\mathcal{S}$, has the minimum/maximum value. 
If $\mathcal{S}$ is empty, return a reference to a special
\Null{} element.

\end{description}
Invocations of \Member{}, \Predecessor{}, and \Successor{} are
colloquially called \emph{searches}, and invocations of \Insert{} and
\Extract{} \emph{updates}. The arguments for the methods can be
partially constructed; i.e.~if an element is a pair of a key and some
satellite data, only the key needs to be specified before a search.
In addition to the above-mentioned methods, there should be methods
for constructing and destructing elements and sets, and methods for
examining the cardinality of sets, but these are algorithmically less
interesting and therefore omitted in our discussion.

\noindent\textbf{**Snip}


This is where the definitions of ordered dictionary, unordered
dictionary, elementary dictionary, dictionary, and priority queue will
go?

\section{Elementary dictionaries}%
\seclabel{elementary-dictionaries}

Now our task is to transform a given ordered dictionary to one that is
more space-efficient.  Our approach is to partition the data into
chunks and store these chunks in the ordered dictionary being under
transformation (skiplist, 2-3 tree, red-black tree, etc).  To avoid
confusion, we will refer the data structure we are describing as ``the
dictionary'' and the data structure we are using as ``the data
structure.''  Similarly, we will use the terms chunks and elements to
refer to groups of elements and individual elements, respectively.  

We study two different schemes based on this idea.  Both schemes have
the following in common:  Each chunk has a header containing a
constant amount of data, the elements of each chunk are stored in an
array, the elements stored wi
We call the objects containing a pointer to the beginning of a chunk,
the \emph{head} of that chunk.  A head $f$ contains two pointers
$\Pred(f)$ and $\Succ(f)$ that point to other heads, a pointer to an
array $\Elem(f)$ that contains anywhere between $b$ and $2b$ elements,
and an integer $\Size(f)$ that tells the number of elements in
$\Elem(f)$.  The $\Pred$ and $\Succ$ fields are used to link all heads
together into a doubly-linked list.  The elements of $\Elem(f)$ are
always sorted in increasing order, and the elements of $\Elem(f)$ are
all less than the elements of $\Elem(g)$ if $f$ appears before $g$ in
the doubly-linked list.  In this way, the list of heads gives the
elements of the dictionary in sorted order.  Additionally, two special
pointers are maintained that point to the first and the last head,
respectively.  The data structure, denoted $\mathcal{S}$, stores
\emph{handles} to the heads.  Each head contains an iterator pointing
back to the corresponding node inside the data structure.  If the
iterators of the data structure were normal pointers, the extra
indirection could be avoided and the data structure could store the
heads directly.

thin each chunk are kept in sorted order,
and the chunks themselves are stored in a doubly-linked list.  At all
times we maintain the invariant that every element in chunk $A$ is
less than or equal to every element in chunk $B$ if $A$ appears before
$B$ in this list.  Thus, a traversal of this \emph{chunk list} is
sufficient to output all elements in sorted order.

In addition to being linked into a list, the chunks themselves are
stored in the the dictionary.  Notice that, because of the above
invariant, any two chunks can be compared in $O(1)$ time.  To test,
for example, if chunk $A$ is less than chunk $B$ we simply check if
the last element of $A$ is less than the first element of $B$.  It is
this comparison function that we use to store the chunks in the
dictionary.  Similarly, we can compare a chunk to an item $x$.  To
test if $x$ is less than $A$ we test if $x$ is less than the first
element of $A$.  To test if $x$ is greater than $A$ we test if $x$ is
greater than the last element of $A$.  If $x$ is neither greater than
nor less than $A$ then $x$ is equal to $A$.  

\noindent\textbf{Say something about multisets here.}

With this representation, it is easy to see how \Member{},
\Predecessor{}, and \Successor{} can each be implemented by performing
at most one search in the dictionary followed by one (linear or
binary) search in a chunk.  The functions \Findmin{} and \Findmax{}
are even easier;  we simply return the first element of the first
chunk or the last element of the last chunk, respectively.  Where our
two implementations differ is in how they implement the \Insert{} and
\Extract{} functions.

\subsection{Fixed-Sized Chunks}

The simplest realization of the above ideas is to store the chunks as
fixed-size arrays, each of size $b+1$.  With the exception of at most
one chunk, each chunk will store either $b-1$, $b$, or $b+1$ data
elements.  All that remains is to describe how we perform insertion
and extraction in the data structure.

To insert $x$ into the data structure we first search for $x$ in the
data structure.  The result of this is some chunk $A$ that is either
the first chunk, the last chunk, or a chunk in which $x$ is greater
than the first element and less than the last element.  Next, by
traversing the chunk list we examine up to $b$ consecutive chunks
including $A$.

If any of these chunks contains $b-1$ or $b$ elements then we can, in
$O(b^2)$ time, reassign the elements to chunks so as to place $x$ into
some chunk and restore our global invariant.  Otherwise, all $b$
chunks contain $b+1$ elements.  In this case we create a new chunk and
reassign these $b(b+1)$ elements across the $b+1$ chunks so that each
chunk contains exactly $b$ elements.  Finally, we place $x$ in the
appropriate chunk and insert the newly created chunk into the
dictionary.

Extraction is similar to insertion. To extract $x$, we first search for
the chunk $A$ containing $x$. Then, in the $b$ consecutive chunks
including $A$ either: (1)~there is some chunk containing $b$ or $b+1$
elements, in which case we remove $x$ and redistribute elements so as
to maintain the global invariant, or (2)~all $b$ chunks contain
exactly $b-1$ elements in which case we delete one chunk from the
dictionary and redistribute the $b(b-1)-1$ elements among the $b-1$
chunks so that each chunk stored $b$ elements, except one, which
contains $b-1$ elements.

It is clear that both the insertion and deletion procedures require
one search in the dictionary, $O(b^2)$ work, and at most one insertion
or deletion, respectively, in the dictionary.  Indeed a slightly more
careful amortized analysis shows that a sequence of $n$
\Insert/\Extract\ operations requires only $O(n/b)$ insertions or
extractions from the dictionary.  careful analysis shows that

\begin{theorem}
\label{theorem:elementary-a}
Given an ordered dictionary that requires $S(n)$, $I(n)$, and $D(n)$
time to search, insert, and delete (respectively) an element from a
set of $n$ elements and that has a memory overhead of $cn$ for a
positive constant $c$, we can construct an equivalent data structure
that requires 
\begin{enumerate}
\item $S'(n)=O(S(n/b)+\lg b)$, $I'(n)=O(S(n/b)+I(n/b)+b^2)$,
and $D'(n)=O(S(n/b)+D(n/b)+b^2)$ worst-case time to search, insert,
and delete (respectively), 
\item requires $S''(n)=O(S(n/b)+\lg b)$,
$I''(n)=O(S(n/b)+\frac{1}{b}I(n/b)+b^2)$, and
$D'(n)=O(S(n/b)+\frac{1}{b}D(n/b)+b^2)$ amortized time to search,
insert, and delete (respectively), and 
\item that has a memory overhead of
$O(n/b)$.
\end{enumerate}
\end{theorem}

One issue that several researchers have been concerned with is that of
\emph{fragmentation} \cite{X}, i.e., how many distinct memory
locations are required for a data structure.  Using the above scheme,
in combination with simple techniques for maintaining a pool of
buffers for the chunks \cite{X}, the entire dictionary described above
can be made to live in one resizeable array that contains $n$ data
elements and $O(n/b)$ additional information.  In some articles
(namely those dealing with implicit data structures \cite{X,X,X,X}),
an unbounded resizeable array is considered to be part of the model of
computation. Alternatively, Brodnik \etal\ \cite{bXX} show how to
implement resizeable arrays in a standard model of computation using
$O(\sqrt{n})$ additional space for an array of size $n$. 
\textbf{What about compactor zones?}

\subsection{Variable-Sized Chunks}

One drawback of the scheme described in the previous subsection is the
$O(b^2)$ term in the running time for the \Insert\ and \Extract\
operations.  In this subsection we describe an alternative that avoids
this problem but requires more extensive use of memory management and,
consequently, for which it is more complicated to avoid fragmentation.
In this scheme, all but one chunk contains somewhere between $b/2$ and
$2b$ elements, inclusive.\footnote{Throughout this section we assume
$b$ is even so that $b/2$ is an integer.}  However, unlike the
previous scheme, each chunk is allocated to the precise size required
to store the number of elements it currently contains.

Since chunks will be reallocated frequently and are addressed by the
nodes of the dictionary, each chunk also has an associated
\emph{handle}.  This is a memory location that stores a pointer to the
chunk and the header of each chunk stores a pointer to this handle.
The dictionary, rather than maintain a direct pointer to the chunk,
maintains a pointer to the handle.  In this way, when a chunk is
reallocated it only need update its handle.  From this point on,
updating of handles is implicit when we say that a chunk is resized.

An insertion of the element $x$ in the data structure proceeds as
follows: The chunk $A$ that should contain $x$ is located.  If $A$
contains fewer than $2b$ elements then $A$ is resized to make room for
$x$ and $x$ is added to $A$.  Otherwise, if $A$ already contains $2b$
elements then $A$ is split into two chunks, each containing $b$
elements, one of the new chunks is inserted into the dictionary, and
$x$ is added to the appropriate chunk.

The extraction procedure is similar to the insertion procedure.  If
the chunk $A$ containing $x$ contains more than $b/2$ elements then
$A$ is resized and $x$ is removed.  Otherwise, we check one of the
neighbouring chunks, $B$, of $A$. If $B$ contains more than $b/2$
elements then an element is removed from $B$ and used to replace $x$
in $A$ and then $B$ is resized. Otherwise, both $A$ and $B$ have size
$b/2$, in which case we merge them into a single chunk containing $b$
elements and delete one of the chunks from the data structure.

Again, it is clear that both \Insert\ and \Extract\ can be done using
at most one search, $O(b)$ time to operate on the at most 2 chunks
involved, and at most one insertion and extraction.  Again, a simple
amortization argument shows that a sequence of $n$ \Insert\ and
\Extract\ operations results in only $O(n/b)$ insertion and
extractions in the data structure.

\begin{theorem}
\label{theorem:elementary-b}
Given an ordered dictionary that requires $S(n)$, $I(n)$, and $D(n)$
time to search, insert, and delete (respectively) an element from a
set of $n$ elements and that has a memory overhead of $cn$ for a
positive constant $c$, we can construct an equivalent data structure
that requires 
\begin{enumerate}
\item $S'(n)=O(S(n/b)+\lg b)$, $I'(n)=O(S(n/b)+I(n/b)+b)$,
and $D'(n)=O(S(n/b)+D(n/b)+b^2)$ worst-case time to search, insert,
and delete (respectively), 
\item requires $S''(n)=O(S(n/b)+\lg b)$,
$I''(n)=O(S(n/b)+\frac{1}{b}I(n/b)+b)$, and
$D'(n)=O(S(n/b)+\frac{1}{b}D(n/b)+b)$ amortized time to search,
insert, and delete (respectively), and 
\item that has a memory overhead of
$O(n/b)$.
\end{enumerate}
\end{theorem}

\comment{
\comment{The components used by this compaction scheme are
illustrated in Figure~\ref{fig:array}.

\begin{figure}
\begin{center}
\input{array.pstex_t}
\end{center}
\caption{A chunk storing seven integers in an array.\label{fig:array}}
\end{figure}
}


We call the objects containing a pointer to the beginning of a chunk,
the \emph{head} of that chunk.  A head $f$ contains two pointers
$\Pred(f)$ and $\Succ(f)$ that point to other heads, a pointer to an
array $\Elem(f)$ that contains anywhere between $b$ and $2b$ elements,
and an integer $\Size(f)$ that tells the number of elements in
$\Elem(f)$.  The $\Pred$ and $\Succ$ fields are used to link all heads
together into a doubly-linked list.  The elements of $\Elem(f)$ are
always sorted in increasing order, and the elements of $\Elem(f)$ are
all less than the elements of $\Elem(g)$ if $f$ appears before $g$ in
the doubly-linked list.  In this way, the list of heads gives the
elements of the dictionary in sorted order.  Additionally, two special
pointers are maintained that point to the first and the last head,
respectively.  The data structure, denoted $\mathcal{S}$, stores
\emph{handles} to the heads.  Each head contains an iterator pointing
back to the corresponding node inside the data structure.  If the
iterators of the data structure were normal pointers, the extra
indirection could be avoided and the data structure could store the
heads directly.

Note that the comparison of two heads is a constant time operation,
since to test if $f<g$ we compare the last element of $\Elem(f)$ to
the first element of $\Elem(g)$.  Similarly, given an element $x$ we
can test if $x<f$, respectively $f<x$, by comparing $x$ with the
first, respectively last, element in $\Elem(f)$.  Therefore, to reduce
the storage requirements of a dictionary, we make a data structure on
a set of heads whose $\Elem$ arrays contain all the elements in our
dictionary.

\paragraph{Searching.}
If we search for some element $x$ in the data structure, we find the
head $f$ such that $\Elem(f)$ contains the smallest key greater
than or equal to $x$.  An additional search in $\Elem(f)$ finds the
actual element we are looking for at a cost of $O(b)$, or $O(\lg b)$
if we use binary search.  If the data structure takes $S(n)$ time to
perform a search on a set of $n$ elements, then the dictionary takes
$S'(n)=O(S(n/b)+\lg b)$ time to perform the search.

\Insert{}$(\mathcal{S}, x)$.
Assume we are given a pointer to the dictionary node containing the
head $f$ we would find if we searched for $x$.  To insert $x$, we
proceed as follows.  If $\Elem(f)$ contains fewer than $2b$ elements
then we simply reallocate $\Elem(f)$ to increase its size by 1 and add
$x$ to $\Elem(f)$.  Otherwise ($\Elem(f)$ contains $2b$ elements), we split
$\Elem(f)$ into two heads, one that contains $b$ elements and one
that contains $b+1$ elements and insert the newly created head
into our data structure.  Therefore, if the data structure takes
$I(n)$ time to perform an insertion on a set of $n$ elements then
insertion in the dictionary takes $I'(n)=O(I(n/b)+b)$ time.

\Extract{}$(\mathcal{S}, x)$.
To remove element $x$ from the dictionary, we assume we are given a
pointer to the dictionary node containing the head $f$ such that
$\Elem(f)$ contains $x$.  If $\Elem(f)$ has size greater than $b$, then
we simply reallocate $\Elem(f)$ to decrease its size by one and exclude
$x$.  Otherwise ($\Elem(f)$ has size $b$), we examine a head $g$
that is a neighbour of $f$ in the linked list.  If $\Elem(g)$ contains
more than $b$ elements then we take the first or last element in
$\Elem(g)$ (depending on whether $g=\Succ(f)$ or $g=\Pred(f)$) and use it
to replace $x$ in $g$.  Otherwise ($\Elem(f)$ and $\Elem(g)$ both have
size $b$), we merge $f$ and $g$ into a single chunk and remove one
of them from the data structure.  If $D(n)$ is the time it takes to
remove element $x$ from a data structure of size $n$, then deletion
in the modified data structure takes $D'(n)=O(D(n/b)+b)$ time.

A data structure containing $m$ heads contains at least $n=bm$
elements.  Each head has a constant amount of overhead, and the data
structure has an overhead of $cm$ for some constant $c$, so the
overhead is $O(n/b)$. The above discussion can be summarized as
follows.

\begin{theorem}
\label{theorem:elementary}
Given an ordered dictionary that requires
$S(n)$, $I(n)$, and $D(n)$ time to search, insert, and delete
(respectively) an element from a set of $n$ elements and that has a memory
overhead of $cn$ for a positive constant $c$, we can construct an equivalent
data structure that requires $S'(n)=O(S(n/b)+\lg b)$,
$I'(n)=O(I(n/b)+b)$, and $D'(n)=O(D(n/b)+b)$ time to
search, insert, and delete (respectively) and that has a memory
overhead of $O(n/b)$.
\end{theorem}
}


Knowing that element-based operations on an ordered dictionary take at
least logarithmic time, it would be tempting to choose $b = \lg n$.
The problem with this choice is that $n$, the number of elements
stored, is varying whereas $b$ must be kept fixed. To avoid this
problem, we use Frederickson's partitioning scheme \cite{Fre83} and
partition the data structure into $O(\lg\lg n)$ separate
\emph{portions} of exponentially increasing sizes $2^{2^{1}}$,
$2^{2^{2}}$, $2^{2^{3}}$, and so on, except that the last portion can
be smaller.  Inside each portion we can use a fixed chunk size
$b_i=2^i$ in the $i$th portion. Insertions are done in the last
portion, searches visit all structures but the overall cost is
dominated by the cost incurred by the last portion [all $O(\lg\lg n)$
searches still take $O(\lg n)$ time if the data structure supports
searches in $O(\lg n)$ time], and deletions are performed by borrowing
an element from the last portion. 

\begin{theorem}
\label{theorem:elementary-c}
Given an ordered dictionary that requires $S(n)$, $I(n)$, and $D(n)$
time to search, insert, and delete (respectively) an element from a
set of $n$ elements and that has a memory overhead of $cn$ for a
positive constant $c$, we can construct an equivalent data structure
that requires 
\begin{enumerate}
\item $S'(n)=O(S(n/\lg n)+\lg\lg n)$, $I'(n)=O(S(n/\lg n)+I(n/\lg n)+\lg n)$,
and $D'(n)=O(S(n/\lg n)+D(n/\lg n)+\lg n)$ worst-case time to search, insert,
and delete (respectively), 
\item requires $S''(n)=O(S(n/\lg n)+\lg\lg n)$,
$I''(n)=O(S(n/\lg n)+\frac{1}{\lg n}I(n/\lg n)+\lg n)$, and
$D'(n)=O(S(n/\lg n)+\frac{1}{\lg n}D(n/\lg n)+\lg n)$ amortized time to search,
insert, and delete (respectively), and 
\item that has a memory overhead of
$O(n/\lg n)$.
\end{enumerate}
\end{theorem}

\comment{
\begin{theorem}
\label{theorem:advanced}
Given an ordered dictionary that requires
$S(n)$, $I(n)$, and $D(n)$ time to search, insert, and delete
(respectively) an element from a set of $n$ elements and that has a memory
overhead of $cn$ for a positive constant $c$, we can construct an equivalent
data structure that requires $S'(n) = O(S(n/\lg n)+\lg\lg n)$,
$I'(n)=O(I(n/\lg n)+\lg n)$, and $D'(n)=O(D(n/\lg n)+\lg n)$ time to
search, insert, and delete (respectively) and that requires at most $n
+ O(\sqrt{n})$ locations for elements and 
$O(n/\lg n)$ words for pointers, counters, and flags.
\end{theorem}
}

As an example, let us apply Theorem \ref{theorem:elementary-c} to any
worst-case efficient balanced search tree, e.g. to a red-black
tree.

\begin{corollary}
\label{corollary:red-black}
There exists an ordered dictionary that supports searches and updates
in $O(\lg n)$ worst-case time, \Findmin{} and \Findmax{} in $O(1)$
worst-case time, and has a memory overhead of $O(n/\lg n)$.
\end{corollary}

\noindent\textbf{What about compactor zones?}

\section{Iterator-based dictionaries}\seclabel{iterator-dictionaries}

The data structural transformations described in the previous sections
can be used to reduce the memory overhead of dictionary data
structures.  However, they break down when the interface to those data
structures allows access to elements using references, pointers, or
fingers\ldots In this section we show two ways in which these data
structures can be extended to allow this type of access.  The first
method is preferable when the total number of iterators or references
to elements in the data structure is small, while the second is
preferable when there are many iterators.

\subsection{Reference Arrays}

In this section we present a method of maintaining iterators where the
memory overhead is proportional to the total number of iterators.




In this section we describe another space-reducing transformation for
ordered dictionaries which similar to our previous construction. The
primary goal of this new construction is to provide iterator-based access
to elements.  The secondary goal is to avoid extensive element copying
that is undesirable in a generic environment since the type of
elements may be unknown beforehand (e.g.~when developing a generic
data structure for a program library).

A \emph{locator} is a mechanism for maintaining the association
between an element and its location in a data structure \cite[Section
6.4]{GT98}. A locator follows its element even if the element changes
its location inside the data structure. In terms of the C\texttt{++}
programming language \cite{ISO}, a locator is an object that provides
default constructor, copy constructor, \Operator{}*,
\Operator{}\texttt{==}, and \Operator{}\texttt{!=}. That is, for
locator $p$, *$p$ denotes the element stored at the specified
location.  An \emph{iterator} is a generalization of a locator which
captures the concepts \textit{location} and \textit{iteration} in a
sequence of elements. In addition to the operations supported for a
locator, a \emph{bidirectional iterator} provides
\Operator{}\texttt{++} and \Operator{}\texttt{-}\texttt{-}. That is,
for iterator $p$, $p$\texttt{++} returns an iterator pointing to the
element following the current element according to some
specific order. An error occurs if $p$ points to the past-the-end
element of the corresponding sequence.

For an ordered dictionary providing iterator-based access to elements,
the methods \Member{}, \Predecessor{}, \Successor{}, \Insert{}, and
\Extract{} are overloaded as follows:
\begin{description}
\item[\Member{}$(\mathcal{S}, p, x).$] Search for element $x$ from sequence $\mathcal{S}$
starting from the element pointed to by iterator $p$.
If $\mathcal{S}$ does not contain any
matching member, return an iterator to the past-the-end element.

\item[\Predecessor{}$(\mathcal{S}, p, x)$/\Successor{}$(\mathcal{S}, p, x).$] Defined in a
similar manner as \Member{}.

\item[\Insert{}$(\mathcal{S}, p, x).$] Insert a copy of element $x$
\emph{immediately before} the element pointed to by iterator $p$ in
sequence $\mathcal{S}$, and return an iterator pointing to the newly inserted element.

\item[\Extract{}$(\mathcal{S}, p).$] Remove the element pointed to by iterator $p$
from sequence $\mathcal{S}$.
\end{description}
To iterate over a sequence, it is also possible to access some special
locations:
\begin{description}
\item[\Begin{}$(\mathcal{S}).$] Return an iterator pointing to the first element
in sequence $\mathcal{S}$.

\item[\End{}$(\mathcal{S}).$] Return an iterator pointing to the past-the-end
element for sequence $\mathcal{S}$.
\end{description}
Since \Begin$(\mathcal{S})$ points to the minimum element of $\mathcal{S}$ and
\End$(\mathcal{S})$\texttt{-}\texttt{-} to the maximum element (if any),
the methods \Findmin{} and \Findmax{} will be superfluous.

\begin{figure}
\begin{center}
\input{list.pstex_t}
\end{center}
\caption{A chunk storing eight integers in a singly-linked list.\label{fig:heap}}
\end{figure}

For a positive integer $b$, we use a ($b$,$2b$)-compaction scheme to
carry out the transformation. Instead of an array, each chunk stores
its elements in a singly-linked list in sorted order (for an
illustration, see Figure \ref{fig:list}).  The first element of each
list is the representative of the corresponding chunk.  The chunks are
pairwise disjoint when considered as intervals.  The heads of chunks
are stored in an ordered dictionary $\mathcal{S}$ which is organized
according to the representatives.  The heads of all the chunks are
manipulated as in our previous construction.

Iterators given to users are implemented as pointers to the list
nodes. This makes all locator operations to be provided trivial.  Let
us now consider how the bidirectional iterator operations are
realized. To guarantee that each of these operations takes constant
time, we have to assume that $b$ is a constant.

\Begin{}$(\mathcal{S})$. Go to the first chunk and return an iterator
pointing to its first element.

\End{}$(\mathcal{S})$. Go to the last chunk and return an iterator
pointing to the past-the-end element stored at that chunk.

\Operator{}\texttt{++}$(p)$. If the element pointed to by $p$ is not
the last element in its list, return an iterator to the next element
in the list. Otherwise, visit the next chunk and return an iterator
pointing to its first element.

\Operator{}\texttt{-}\texttt{-}$(p)$. Access the predecessor of the
element pointed to by $p$ by traversing the list, first going to the
head and then visiting the list nodes until the current node is
met again. Return an iterator pointing to the predecessor.

Next let us consider how different dictionary operations can be
realized.  For a data structure $\mathcal{D}$ of size $n$, we let
$S^{\mathcal{D}}(n)$, $I^{\mathcal{D}}(n)$, and $E^{\mathcal{D}}(n)$
denote the running time of searches, \Insert{}, and \Extract{},
respectively.

\Member{}$(\mathcal{S}, p, x)$.  Access the elements in the current
chunk that come after the element pointed to by $p$ until a member
matching $x$ or an element greater than $x$ or the head is met. If a
matching member is found, return an iterator pointing to that element
and stop. If a greater element is met, return an iterator pointing to
the past-the-end element. Otherwise, use $\mathcal{S}$ to find the
first matching element and, if no such element exists, the first
element that is greater than $x$. Finally, scan the predecessor chunk
to see whether that contains a matching member, and return an iterator
pointing to the first matching member. If no such member is found,
return an iterator pointing to the past-the-end element.  Clearly, the
running time this operation is $S^{\mathcal{D}}(n/b) + O(b)$.

\Predecessor{}$(\mathcal{S}, p, x)$/\Successor{}$(\mathcal{S}, p, x)$.
These are realized in a similar manner as \Member$(\mathcal{S}, p,
x)$.

\Insert{}$(\mathcal{S}, p, x)$. If the element pointed to by $p$ is a
representative, insert $x$ as the last element of the predecessor
chunk. If no such chunk exists, invoke \Extract{} for $\mathcal{S}$
to remove the representative of the present chunk from $\mathcal{S}$
and insert $x$ into the present chunk. Let $C$ denote the chunk into
which $x$ was inserted.
  If the size of $C$ is equal to $2b$ and  $C$ was
removed from $\mathcal{S}$, split $C$ into two pieces of equal size
and invoke \Insert{} for $\mathcal{S}$ to get both chunks back to the
data structure.
  If the size of $C$ is no greater than $2b$
and  $C$ was removed from $\mathcal{S}$, invoke \Insert{} for
$\mathcal{S}$ to get the chunk back into the data structure with a new
representative. 
  If the size of $C$ is equal to $2b$ and $C$ was removed from
$\mathcal{S}$, split $C$ into two halves and invoke \Insert{} for
$\mathcal{S}$ to get the representative of the second half into the
data structure.
  Finally, if the size of $C$ is no greater than $2b$ and $C$ was not removed
from $\mathcal{S}$, do nothing.
{}From this description, one can immediately conclude that the
running time of an insertion is 
$2\times I^{\mathcal{D}}(n/b) + E^{\mathcal{D}}(n/b) + O(b)$.

\Extract{}$(\mathcal{S}, p)$. Remove the element pointed to by $p$ by
accessing the predecessor element via the head. 
Let $C$ denote the current chunk.
  If the size of $C$ is no smaller than $b$ and the representative of
$C$ was not
removed, do nothing and stop. 
  If the size of $C$ is no smaller than $b$ and the representative of
$C$ was removed, invoke \Extract{} and thereafter
\Insert{} for $\mathcal{S}$ to get the chunk back to the data structure.
  If the size of $C$ is smaller than $b$, merge $C$ with one of its
neighbouring chunks. Let $J$ be the resulting chunk. If the neighbour
selected was the predecessor chunk of $C$, invoke \Extract{} for
$\mathcal{S}$ to remove the representative of $C$ from $\mathcal{S}$.
If the neighbour was the successor chunk of $C$, invoke \Extract{} for
$\mathcal{S}$ to remove the representative of the neighbour from
$\mathcal{S}$. If the size of $J$ is
no larger than $2b$, stop. Otherwise, split $J$ into two halves, and
invoke \Insert{} for $\mathcal{S}$ to get both chunks back to the data structure.
From this description, we can readily conclude that the running time
of a deletion is 
$2\times I^{\mathcal{D}}(n/b) + E^{\mathcal{D}}(n/b) + O(b)$.

The dictionary data structure $\mathcal{S}$ stores handles to the heads
and these store iterators back to the dictionary
nodes. There are at most $\ceils{n/b}$ heads. Since the size of
$\mathcal{S}$ is assumed to be linear, $\mathcal{S}$ requires $O(n/b)$
words of memory. Since iterators are constant-sized objects and a head
contains $O(1)$ pointers and counters, the size of a head is
$O(1)$. That is, their total space usage is $O(n/b)$. Each
node in a chunk list requires one pointer, a pointer to another list
node or a pointer to a head, and a bit indicating the type of the
pointer.   Hence, the memory overhead induced by the whole construction is
$O(n/b)$ words plus one pointer per element and one bit per element.
Because of word alignment the last two bits of a pointer are unused,
so in normal circumstances an indicator bit and a pointer can be
stored in one word. To sum up, the memory overhead is $n + O(n/b)$
words. By making $b$ large enough, but still keeping it a constant, we
get the following theorem.

\begin{theorem}
\label{theorem:dictionary}
An ordered dictionary $\mathcal{D}$ that uses $O(n)$ words of extra
storage can be transformed into an equivalent data structure
 that uses $(1+\varepsilon)n$ words of extra storage,
for an arbitrary $\varepsilon > 0$ and sufficiently large $n >
n(\varepsilon)$, and executes searches and updates in
$S^{\mathcal{D}}(\varepsilon n) + O(1/\varepsilon)$ and $2\times
I^{\mathcal{D}}(\varepsilon n) + E^{\mathcal{D}}(\varepsilon n) +
O(1/\varepsilon)$ time, respectively.
\end{theorem}

As a concrete example of the usage of Theorem
\ref{theorem:dictionary}, let us apply it to the ordered dictionary
presented in \cite{BLMTT03}. This gives us the following corollary.

\begin{corollary}
\label{corollary:dictionary}
There exists an ordered dictionary that supports iterator-based
updates in $O(1/\varepsilon)$ worst-case time, finger searches in
$O(\lg d + 1/\varepsilon)$ worst-case time, $d$ being the distance to
the target, and requires at most $(1 + \varepsilon)n$ words of
extra storage, for any $\varepsilon > 0$ and sufficiently large
$n > n(\varepsilon)$.
\end{corollary}

\section{Iterator-based priority queues}\seclabel{iterator-queues}

Any ordered dictionary can used as a priority queue as well. However,
two methods, \Insert{} and \Decrease{}, have turned out to be special
since there exist priority queues that support all of them in constant
time.  When our previous transformations are applied to such a
priority queue, the resulting data structure is not necessarily
optimal with respect to these operations. Therefore, in this section
we describe a transformation that is suitable for priority queues.

We want to point out that we do not consider elementary priority
queues since an implicit priority queue exists that can support
\Findmin{} and \Insert{} in $O(1)$ worst-case time, and \Extractmin{} in
$O(\lg n)$ worst-case time \cite{CMP88}, $n$ being the number of
elements stored. As mentioned in \secref{sec:intro}, in a realistic
model of computation, the amount of extra space used by such a data
structure is only $O(\sqrt{n})$.%
%In \cite{CMP88} it was shown that the
%priority queue could even be made double-ended, i.e.~made to support
%\Findmax{} and \Extractmax{} as well.

A \emph{priority queue} is an abstract data type that is supposed to
provide the methods \Findmin{} and iterator-based \Extract{} as well
the following methods:
\begin{description}
\item[\Insert{}$(\mathcal{Q}, x).$] Insert a copy of element $x$ into sequence $\mathcal{Q}$ and return an iterator pointing to the newly inserted
element.

\item[\Borrow{}$(\mathcal{Q}).$] Extract an \emph{arbitrary} element from
sequence $\mathcal{Q}$ and return an iterator pointing to that element. If $\mathcal{Q}$
is empty, return an iterator to the past-the-end element.

\item[\Decrease{}$(\mathcal{S}, p, x).$] Replace the element pointed to by
iterator $p$ with element $x$, which is assumed to be no greater than
the original element pointed to by $p$.
\end{description}
Of these operations, \Borrow{} is non-standard, but according to our
earlier research \cite{EJK04,EJK07} this operation is very useful, for
instance, for data-structural transformations.

Again we use ($b$,$2b$) compaction scheme in our transformation, $b$
being a positive integer. Now each chunk stores its elements in a
singly-linked list in an arbitrary order, except that the minimum
element which is the representative of that chunk is stored as the
first element (cf.~Figure~\ref{fig:list}).  Moreover, the chunks can
overlap when considered as intervals of values.  There is one special
chunk, called a \emph{buffer} $B$, which may contain less than $b$
elements. The other chunks contain between $b$ and $2b$ elements and
are stored in a priority queue $\mathcal{Q}$, which only operates on
the representatives. The chunks are chained in a doubly-linked list so
that the elements present can be accessed in arbitrary order. 
The heads of the chunks are maintained as in our previous constructions.

If $\mathcal{Q}$ supports
iterators, iterator operations can be realized as for ordered
dictionaries.  Next we consider how different priority-queue
operations can be realized.

\Findmin{}$(\mathcal{Q})$. The minimum is stored either in buffer $B$ or the
minimum representative stored in $\mathcal{Q}$. An iterator, which is a pointer to a list
node, pointing to the node storing the minimum of the two elements can
be returned. Since both the relevant elements are the first elements
is their respective lists, this operation can be accomplished in
$O(1)$ worst-case time.

\Insert{}$(\mathcal{Q}, x)$. Insert $x$ into buffer $B$. If $x$ is smaller than
the current minimum of $B$, put it as the first element; otherwise, put it as
the second element. If the size of $B$ is equal to $2b$, insert
$B$ into $\mathcal{Q}$ and create a new empty buffer. In addition to a single
invocation of \Insert for $\mathcal{Q}$, only a constant amount of
work is done (under the assumption that chunks know their size).

\Borrow{}$(\mathcal{Q})$. If $B$ is empty, extract one chunk from
$\mathcal{Q}$ and make that a new buffer $B$. If after this $B$ is
still empty, return an iterator to the past-the-end element and
stop. Now take one element from $B$, the second element if $B$ has
one, otherwise take the first element, and return an iterator pointing
to that element. Again one invocation of \Borrow{} for $\mathcal{Q}$
may be necessary, but otherwise only a constant amount of work is
done.

\Decrease{}$(\mathcal{Q}, p, x)$. Access the head of the chunk to
which $p$ points. Thereafter, carry out the element replacement and
make the replaced element the representative of the chunk if it is
smaller than the previous representative. If the chunk is not buffer
$B$ and if the representative of the chunk changes, invoke
\Decrease{} for $Q$. In addition to one invocation of \Decrease{}, it
may be necessary to traverse the
whole list once to make the required changes. That is, the additional
work done is proportional to $b$.

\Extract{}$(\mathcal{Q}, p)$. Access the head of the chunk to which
$p$ points. If the size of the chunk becomes smaller than $b$ or if
the representative of the chunk is removed, remove the chunk from
$Q$. Then merge the removed chunk with buffer $B$ to form a new
buffer. If the size of $B$ is larger than $2b$, cut off a piece of
size $2b$ from $B$, make the rest a buffer and insert the chunk cut
off into $Q$. To sum up, one invocation of \Extract{} and \Insert{} for
$Q$ may be necessary, but the other work done is proportional to $b$.

The above discussion is summarized in the following theorem.  Let
$\mathcal{D}$ be a data structure and let $n$ denote the number of
elements stored in $\mathcal{D}$ prior to each operation.  We use the
notation that $\mathcal{D}$ executes \Findmin{}, \Insert{},
\Borrow{}, \Decrease{}, and \Extract{} in 
$F^{\mathcal{D}}(n)$,
$I^{\mathcal{D}}(n)$, 
$B^{\mathcal{D}}(n)$, 
$D^{\mathcal{D}}(n)$, and
$E^{\mathcal{D}}(n)$ time, respectively. To get the claimed result, we
should just make $b$ large enough to get the amount of extra storage
used down from $O(n)$ to $(1+\varepsilon)n$.

\begin{theorem}
\label{theorem:queue}
A priority queue $\mathcal{D}$ that uses $O(n)$ words of extra storage
can be transformed into a equivalent data structure that
uses
$(1+\varepsilon)n$ words of extra storage, for an arbitrary
$\varepsilon > 0$ and sufficiently large $n > n(\varepsilon)$, and performs 
\Findmin{}, \Insert{},
\Borrow{}, \Decrease{}, and \Extract{} in
$F^{\mathcal{D}}(\varepsilon n) + O(1)$,
$I^{\mathcal{D}}(\varepsilon n) + O(1)$,
$B^{\mathcal{D}}(\varepsilon n) + O(1)$,
$D^{\mathcal{D}}(\varepsilon n) + O(1/\varepsilon)$,
$E^{\mathcal{D}}(\varepsilon n) + I^{\mathcal{D}}(\varepsilon n) + O(1/\varepsilon)$ time, respectively.
\end{theorem}

To illustrate the power of Theorem \ref{theorem:queue}, let us apply
it to the priority queue described in \cite{EJK06}. As an outcome we
get the following result:

\begin{corollary}
\label{corollary:queue}
There exists a priority queue that supports \Findmin{} and \Insert{}
in $O(1)$ worst-case time, \Decrease{} in $O(1/\varepsilon)$
worst-case time, \Extract{} in $O(\lg n + 1/\varepsilon)$ worst-case
time including at most $\lg n + 3\lg\lg n + O(1/\varepsilon)$ element
comparisons, and requires at most $(1 + \varepsilon)n$ words of extra
storage, for any $\varepsilon > 0$ and sufficiently large $n > n(\varepsilon)$.
\end{corollary}

\section{Discussion}\seclabel{discussion}

We showed that many data structures of size $O(n)$ can be put on a
diet, so that the memory overhead gets down to $O(n/\lg n)$ or to $(1
+ \varepsilon)n$ words for any $\varepsilon >0$ and sufficiently large
$n > n(\varepsilon)$. This diet does not change the running times of
the operations on the data structures except by a small constant
factor independent of $\varepsilon$ and/or an additive term of
$O(1/\varepsilon)$.

We considered ordered dictionaries and single-ended priority queues,
but the compaction technique can be used for slimming down other data
structures as well. Other applications, where the technique is known
to work, include positional sequences (doubly-linked lists), unordered
dictionaries (e.g.~hashtable relying on chaining), and doubly-ended
priority queues.  The details follow closely the guidelines given in
this paper, so we leave the proof of the following theorems for an
interested reader.

\begin{theorem}
A doubly-linked list, which provides bidirectional iterators and
iterator-based updates, can be transformed into an equivalent
positional sequence that uses $(1 + \varepsilon)n$ words of memory for
any $\varepsilon > 0$ and sufficiently large $n > n(\varepsilon)$,
excluding the space used by the $n$ elements stored. This
transformation only slows down the running times of the supported
operations by an additive term of $O(1/\varepsilon)$.
\end{theorem}

\begin{theorem}
A hashtable relying on chaining, which provides bidirectional
iterators and member searches and updates, can be transformed into an
equivalent unordered dictionary that uses $(1 + \varepsilon)n$ words
of extra storage, for any $\varepsilon > 0$ and sufficiently large $n
> n(\varepsilon)$.  This transformation only slows down the running
times of the supported operations by an additive term of
$O(1/\varepsilon)$.
\end{theorem}

\begin{theorem}
A doubly-ended priority queue, which uses $O(n)$ words of extra
storage, provides bidirectional iterators and methods \Findmin{},
\Findmax{}, \Insert{}, \Borrow{}, and \Extract{}, can be transformed
into an equivalent double-ended priority queue that uses $(1 +
\varepsilon)n$ words of extra storage, for any $\varepsilon > 0$ and
sufficiently large $n > n(\varepsilon)$.  This transformation only
slows down the running times of the supported operations by an
additive term of $O(1/\varepsilon)$.
\end{theorem}

The practical value of the compaction technique is that users of data
structures need not worry about the memory overhead associated with
the data structure they choose.  If it becomes a problem, they can
simply apply the technique and the problem is solved.  The theoretical
value of the compaction technique is that designers of data structures
need no longer make unnecessarily complicated algorithms and
representations for the sake of reducing the memory overhead.  They
can get on with the problem by designing data structures that have
fast running times and useful properties without worrying about
memory overhead.

%\section*{Acknowledgement}
%We thank Nicolai Esbensen for pointing out that the compaction technique
%can also be used for slimming down a doubly-linked list.

\bibliography{diet}
\bibliographystyle{DIKU}


\end{document}
