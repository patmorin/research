\documentclass{article}

\input{pat.tex}


\newcommand{\y}{\mathrm{y}}

\title{Skiplists as Priority Queues%
	\thanks{This research was partially funded NSERC.}}
\author{Pat Morin \\ 
	School of Computer Science \\ 
	Carleton University \\ 
	\texttt{morin@cs.carleton.ca}}
\date{}

\begin{document}
\maketitle
\begin{abstract}
We study the use of skiplists as priority queues.  Using skiplists we
obtain easy-to-implement priority queues that have the working set
property \cite{st85} or the queueish property \cite{il02}.  Previously
implementations of priority queues with the working set or queueish
property only satisfied these properties in an amortized sense.
Priority queues based on skiplists can satisfy these properties with
high probability.
\end{abstract}

\section{Introduction}\seclabel{introduction}

Consider the a priority queue $Q$ containing a set of $S$ of $n$
elements.  For an element $x$ contained in $Q$, let $\y_x$ denote the
set of elements currently in $Q$ that were inserted into $Q$ after the
insertion of $x$.  (These elements are \emph{younger} than $x$.)  We
say that $Q$ has the \emph{working set} property if the deletion of
$x$ can be performed in $O(\log \y_x)$ time.  We say that $Q$ has the
\emph{queueish} property if the deletion of $x$ can be done in $O(\log
(n-\y_x))$ time.

The working set and queueish properties are complementary.
Intuitively, if $Q$ is used as a stack then the working set property
implies that deletions are a constant time operation.  On the other
hand, if $Q$ is used as a queue then the queueish property implies
that deletion are a constant time operation.

There are several examples in which these properties are useful.  For
the working set property, examples include\ldots For the queueish
property, examples include sorting inputs that have few inversions and
running Dijkstra's algorithm on graphs with almost-uniform edge
weights.

Priority queue implementations with the working set and queueish
properties do exist.  In particular, pairing heaps \cite{fsst86} achieve
the working set property and queaps \cite{il02} achieve the queueish
property.  However, both these data structures achieve these
properties only in an amortized sense, and the cost of an individual
deletion operation can be $\Omega(n)$.

In this paper, we study the use of skiplists \cite{p90} in the
implementation of priority queues.  We show that, using skiplists, it
is possible to implement priority queues that have the working set or
queueish properties with high probability.  For the working set
property, this means that the probability that deleting $x$ takes more
than $c\log \y_x$ time is at most $\y_x{}^{-c}$.  Of course, this
implies that these data structures also have these properties in an
expected sense, so that the expected cost of deleting $x$ is $O(\log
\y_x)$ for the working set data structure.

The remainder of this paper is organized as follows. In
\secref{working-set} we describe a priority queue implementation with
the working set property.  In \secref{queueish} we give an
implementation with the queuish property.  In \secref{conclusion} we
summarize and make some closing remarks.

\section{A Working Set Priority Queue}\seclabel{working-set}

\section{A Queueish Priority Queue}\seclabel{queueish}

\section{Conclusion}\seclabel{conclusion}


\bibliography{skipq}
\bibliographystyle{plain}

\end{document}