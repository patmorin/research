\documentclass[charterfonts,lotsofwhite]{patmorin}
\usepackage[noend]{algorithmic}

 
%\usepackage{amsthm}

\newcommand{\centeripe}[1]{\begin{center}\Ipe{#1}\end{center}}
\newcommand{\comment}[1]{}

\newcommand{\centerpsfig}[1]{\centerline{\psfig{#1}}}

\newcommand{\seclabel}[1]{\label{sec:#1}}
\newcommand{\Secref}[1]{Section~\ref{sec:#1}}
\newcommand{\secref}[1]{\mbox{Section~\ref{sec:#1}}}

\newcommand{\alglabel}[1]{\label{alg:#1}}
\newcommand{\Algref}[1]{Algorithm~\ref{alg:#1}}
\newcommand{\algref}[1]{\mbox{Algorithm~\ref{alg:#1}}}

\newcommand{\applabel}[1]{\label{app:#1}}
\newcommand{\Appref}[1]{Appendix~\ref{app:#1}}
\newcommand{\appref}[1]{\mbox{Appendix~\ref{app:#1}}}

\newcommand{\tablabel}[1]{\label{tab:#1}}
\newcommand{\Tabref}[1]{Table~\ref{tab:#1}}
\newcommand{\tabref}[1]{Table~\ref{tab:#1}}

\newcommand{\figlabel}[1]{\label{fig:#1}}
\newcommand{\Figref}[1]{Figure~\ref{fig:#1}}
\newcommand{\figref}[1]{\mbox{Figure~\ref{fig:#1}}}

\newcommand{\eqlabel}[1]{\label{eq:#1}}
\newcommand{\eqref}[1]{(\ref{eq:#1})}

\newtheorem{thm}{Theorem}{\bfseries}{\itshape}
\newcommand{\thmlabel}[1]{\label{thm:#1}}
\newcommand{\thmref}[1]{Theorem~\ref{thm:#1}}

\newtheorem{lem}{Lemma}{\bfseries}{\itshape}
\newcommand{\lemlabel}[1]{\label{lem:#1}}
\newcommand{\lemref}[1]{Lemma~\ref{lem:#1}}

\newtheorem{cor}{Corollary}{\bfseries}{\itshape}
\newcommand{\corlabel}[1]{\label{cor:#1}}
\newcommand{\corref}[1]{Corollary~\ref{cor:#1}}

\newtheorem{obs}{Observation}{\bfseries}{\itshape}
\newcommand{\obslabel}[1]{\label{obs:#1}}
\newcommand{\obsref}[1]{Observation~\ref{obs:#1}}

\newtheorem{assumption}{Assumption}{\bfseries}{\rm}
\newenvironment{ass}{\begin{assumption}\rm}{\end{assumption}}
\newcommand{\asslabel}[1]{\label{ass:#1}}
\newcommand{\assref}[1]{Assumption~\ref{ass:#1}}

\newcommand{\proclabel}[1]{\label{alg:#1}}
\newcommand{\procref}[1]{Procedure~\ref{alg:#1}}

\newtheorem{rem}{Remark}
\newtheorem{op}{Open Problem}

\newcommand{\etal}{\emph{et al}}

\newcommand{\voronoi}{Vorono\u\i}
\newcommand{\ceil}[1]{\left\lceil #1 \right\rceil}
\newcommand{\floor}[1]{\left\lfloor #1 \right\rfloor}


\newcommand{\td}{\mathrm{depth}}
\newcommand{\lp}{\mathrm{LP}}

\title{\MakeUppercase{Output-Sensitive Algorithms for Tukey Depth}}
\author{David Bremner \and
	John Iacono \and
	Stefan Langerman \and
	Pat Morin}
\date{}
\begin{document}
\maketitle

\begin{abstract}
The \emph{Tukey depth} of a point $p$ with respect to a set $S$ of
points is the minimum number of elements of $S$ contained in any
closed halfspace that contains $p$.  Algorithms for computing the
Tukey depth of a point in various dimensions are considered.  The
running times of these algorithms depends on the value of the output,
possibly making them suited to situations such as outlier removal when
the value of the output is small.
\end{abstract}

\section{Introduction}

Let $S$ be a set of $n$ points in $\R^d$.
The \emph{Tukey depth}, or \emph{halfspace depth} of a point $p\in\R^d$ with
respect to $S$ can be defined in several equivalent ways \cite{t73}:
\begin{eqnarray}
\td(p,S) & = & \min\{ |h\cap S| :
	             \mbox{$h$ is a halfspace containing $p$} \} \\ 
                     \eqlabel{tuk-orig}
            & = & \min\{ |h\cap S| :
                      \mbox{$h$ is a halfspace with $p$ on its boundary} \} \\ 
                       \eqlabel{tuk-boundary}
            & = & \min\{ |S'| :
                      \mbox{$p$ is outside the convex hull of 
                           $S\setminus S'$} \}
                      \eqlabel{tuk-hull}
\end{eqnarray}

\Eqref{tuk-boundary} immediately yields an $O(n^{d})$ time algorithm
to compute $\td(p,S)$.

In this paper we consider algorithms for computing the Tukey depth of
a point $p$ with respect to a set $S$, whose running time is dependent
on $k$, the value of the output. In particular, we present:

\begin{enumerate}
\item A simple $O(n + k\log k)$ time algorithm for points in $\R^2$
(\secref{2d}).  The most complicated data structure used in this
algorithms is a binary heap.

\item An $O(n + (n-k)\log(n-k))$ time algorithm to find the largest
clique in an interval graph or circular-arc graph, where $k$ is the
size of the clique found (\secref{max-clique}).  This problem is
a generaliation of the 2-dimensional Tukey depth problem.

\item An $O(n\log n + k^2\log n)$ time algorithm for points in $\R^3$
and an $O(n + k^{11/4}n^{1/4})$ time algorithm for points in $\R^4$
(\secref{3-4-d}).  These algorithms rely on results of Chan on linear
programming with violated constraints \cite{X} which in turn rely on
sophisticated range searching data structures \cite{X} and/or dynamic
convex hull data structures \cite{X}.

\item A simple $O((d+1)^k \lp(n,d))$ time algorithm for points in
$\R^d$, where $\lp(n,d)$ denotes the time required to solve a linear
program having $n$ constraints and $d$ variables (\secref{d-d}).
Not surprisingly, this algorithm is also based on linear programming
with violated constraints and is obtained by presenting a solution to
the NP-hard \textsc{Maximum-Feasible-Subsystem}.
\end{enumerate}


\section{An Algorithm for Points in $\R^2$}
\seclabel{2d}

In this section we give a simple $O(n + k\log k)$ time algorithm to
compute the Tukey depth of a point $p\in\R^2$ with respect to
a set $S$ of $n$ points in $\R^2$.  The algorithm begins by
partitioning $\R^2$ into 4 quadrants $Q_0,\ldots,Q_3$ around $p$.
The algorithm then simultaneously begins computing the 4 quantities
$\td_0(p,S),\ldots,\td_3(p,S)$ where 
\begin{equation}
     \td_i(p,S) = \min\{|h\cap S| : \mbox{$h$ is a halfspace containing
$Q_i$} \} \enspace . \eqlabel{four-min}
\end{equation}
Clearly, $\td(p,S) = \min\{\td_i(p,S): 0\le i \le 3 \}$ since any
halfspace containing $p$ contains at least one of the four quadrants.
In the remainder of this section we will describe how to compute
$k_i=\td_i(p,S)$ in $O(n + k_i\log k_i)$ time.  Since the
computation can stop once $\td_i(p,S)$ has been computed for
the index $i$ that minimizes \eqref{four-min}, this yields an $O(n +
k\log k)$ time algorithm, where $k=\td(p,S)$.

Let $S_i=S\cap Q_i$. To compute $\td_i(p,S)$ we create two binary
heaps $H_{i-1}$ and $H_{i+1}$ that store the elements of $S_{i-1}$,
respectively $S_{i+1}$, in counterclockwise, respectively, clockwise,
order around $p$. Creating these two heaps takes $O(n)$ time using the
standard bottom-up algorithm to construct a binary heap \cite{clrXX}.
Next we extract elements one at a time from each of $H_{i-1}$ and
$H_{i+1}$ until either (a)~one of the heaps is empty or (b)~we extract
two elements $q$ from $H_{i-1}$ and $r$ from $H_{i+1}$ such that the
angle $\angle spq > \pi$.  Suppose we have extracted $\ell$ elements
each from $H_{i-1}$ and $H_{i+1}$ when this occurs.  Then it is easy
to verify that 
\[  
  |S_i| + \ell - 1 \le \td_i(p,S) \le |S_i| + 2\ell \enspace .
\]

Next, we continue to extract as many elements as possible from each of
$H_{i-1}$ and $H_{i+1}$ up to a maximum of $\ell$ elements each. The
total time required to extract these at most $4\ell$ from the two
heaps is $O(\ell\log n)$.  By sorting and scanning all the elements
extracted from the heap plus the elements of $S_i$ we can then
compute $\td_i(p,S)$ in an additional
\[
     O((|S_i|+\ell)\log n) = O(k_i\log n)
\] 
time.  This yields an a total running time of 
\[  
   O(n + k_i\log n) = O(n + k_i\log k_i) \enspace ,
\]
as required.  This completes the proof of:

\begin{thm}\thmlabel{2-d}
The Tukey depth of a point $p$ with respect to a set $S$ of $n$ points
in $\R^2$ can be computed in $O(n + k\log k)$ time, where
$k$ is the value of the output.
\end{thm}

\section{Algorithms for \textsc{Max-Clique} in Circular-Arc Graphs}
\seclabel{max-clique}

The problem of computing Tukey depth in $\R^2$ can be viewed as a
problem on a set of circular arcs.  By \eqref{tuk-boundary}, computing
the Tukey depth of $p$ is equivalent to finding a unit normal vector
$v$ such the halfspace with $p$ on its boundary and having inner
normal $v$ contains the fewest number of points of $S$ as possible.
Note that the space of unit normals in $\R^2$ is a circle and that
each point $q\in S$ defines a circular arc such that all choices of
$v$ in this circular arc yield a halfspace that does not contain $q$.
Thus, the Tukey depth problem reduces to the problem of finding a
vector $v$ that is contained in the largest number of circular arcs
The partitioning into 4 quadrants used in the proof of \thmref{2-d}
works because all the circular arcs are actually half circles.  In
this section we extend the results of the previous section to the case
of arbitrary circular arcs.

Let $C$ be a set of $n$ circular arcs of the unit circle $\Sp^1$.  We
describe an $O(n+(n-k)\log (n-k))$ time algorithm to find a point $v$
that is contained in the largest number of elements of $C$\ldots


\section{Algorithms for Points in $\R^3$ and $\R^4$}
\seclabel{3-4-d}

\section{An Algorithm for Points in $\R^d$}
\seclabel{d-d}

Finally, we consider the general case of point sets in $\R^d$.  We
formulate our solution by considering the third definition of Tukey
depth, i.e., Equation~\eqref{tuk-hull}. That is, we wish to remove the
smallest possible set $S'\subseteq S$ such that $p$ is outside the
convex hull of $S\setminus S'$.  We begin by considering the problem
of determining if $p$ is outside the convex hull of $S$, i.e., if
$\td(p,S)=0$.

Without loss of generality assume $p$ is the origin. Then testing if
$p$ is outside the convex hull of $S$ can be formulated as the problem
of finding coefficients $a_1,\ldots,a_d$ such that $\sum_{i=1}^da_i
q_i > 0$ for all $q\in S$.  \notice{is it easy to make this a
non-strict inequality?} Note that this is a set of $n$ linear
inequalities in $d$ variables. Any of the standard algorithms for
linear programming will either report that this set of inequalities is
feasible and give a vector $(a_1,\ldots,a_d)$ satisfying all
constraints or will report a set of $d+1$ constraints that not
feasible. These $d+1$ constraints correspond to $d+1$ points of $S$
whose convex hull (a simplex) contains $p$.  Denote by
$\textsc{Carath\'eodoryWitness}(p,S)$ a routine that takes as input
$p$ and $S$ and either returns a set of at most $d+1$ points in $S$
whose convex hull contains $p$ or, in the case where $p$ is outside
the convex hull of $S$, the empty set.  The following algorithm
determines whether $\td(p,S)\le k$:

\noindent$\textsc{TukeyDepthDecision}(p,S,k)$
\begin{algorithmic}
\STATE{\COMMENT{ return true if $\td(p,S)\le k$ and false otherwise }}
\STATE{$\Delta\gets \textsc{Carath\'eodoryWitness}(p,S)$}
\IF{$\Delta=\emptyset$} 
   \STATE{\textbf{return} true} 
\ENDIF
\IF{$k= 0$}
   \STATE{\textbf{return} false}
\ENDIF
\FOR{each $q\in \Delta$}
   \IF{$\textsc{TukeyDepthDecision}(p,S\setminus\{q\},k-1)=\mathrm{true}$}
     \STATE{\textbf{return} true}
   \ENDIF
\ENDFOR
\STATE{\textbf{return} false}
\end{algorithmic}

Correctness of the above algorithm follows easily by induction on
$\td(p,S)$.  The running time of hte algorithm is given by the
recurrence
\[
  T(n,d,k) \le \lp(n,d) + (d+1) T(n-1,d,k-1)
\]
which resolves easily to $O((d+1)^k\lp(n,d))$ where $\lp(n,d)$ is the
time to solve a linear program with $n$ constraints and $d$ variables.
Running the above decision algorithm repeatedly with $k=0,1,\ldots$
completes the proof of:

\begin{thm}\thmlabel{d-d}
The Tukey depth of a roint $p$ with respect to a set $S$ of $n$ points
in $\R^d$ can be computed in $O((d+1)^k\lp(n,d))$ time, where
$k$ is the value of the output and $\lp(n,d)$ is the time to solve a
linear program with $n$ constraints and $d$ variables.
\end{thm}



\end{document}

