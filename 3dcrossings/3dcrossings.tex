\documentclass{patmorin}
\usepackage{amsthm,amsmath,graphicx,stmaryrd,amsopn}
\usepackage{pat}

\newcommand{\eps}{\varepsilon}

\title{\MakeUppercase{A Crossing Number Inequality for 3D Grid Drawings}}
\author{Sheldon Cooper, Pat Morin,\thanks{School of Computer Science, Carleton University, \email{morin@scs.carleton.ca}}\ and Howard Wolowitz}


\DeclareMathOperator{\x}{x}
\DeclareMathOperator{\y}{y}
\DeclareMathOperator{\z}{z}
\DeclareMathOperator{\crs}{cr}

\begin{document}
\maketitle

\begin{abstract}
   We study crossing number inequalities for 3d-grid drawings.
\end{abstract}

\section{Introduction}

The \emph{$X\times Y\times Z$ grid} is the point set $\{(x,y,z):
x\in\{1,\ldots,X\},\, y\in\{1,\ldots,Y\},\, z\in\{1,\ldots,Z\}\}$.
The volume of the $X\times Y\times Z$ grid is $XYZ$. A \emph{(proper)
drawing}, $\varphi:V(G)\rightarrow R$, of a graph, $G$, on a grid,
$R$, is a one-to-one mapping such that, for every edge $uw\in E(G)$ and
every vertex $x\in V(G)$, the open line segment with endpoints $\varphi(u)$ and $\varphi(w)$
does not contain $x$.

Define $\crs(G,n)$ to be the minimum number of edge crossings
in a drawing of $G$ on a grid whose volume is $n$.  Define
$g(n,m)=\min\{\crs(G,n):\text{$G$ is a graph with $m$ edges}\}$.


\section{A Lower Bound}

\begin{thm}
  $g(n,m) \in \Omega((m^2/n)\log\log (m/n))$.
\end{thm}

\begin{proof}
Let $G$ be a drawing of a graph on the $X\times Y\times Z$ grid, with
$n=XYZ$, that has $m$ edges and for which $\crs(G,n)=g(n,m)$. ($G$
has the minimum number of crossings over all drawings of graphs with
$m$ edges.)  We may assume, without loss of generality, that no edge of
$G$ contains any point of the $X\times Y\times Z$ grid in its interior;
such an edge could be replaced with a shorter edge without introducing
any additional crossings.

For a prime number, $p$, define the $X\times Y\times Z$ \emph{$p$-grid} as the set of points
\[
  \{(x/p,y/p,z/p): x\in\{1,\ldots,pX\},\, y\in\{1,\ldots,pY\},\,
  z\in\{1,\ldots,pZ\}\} \enspace .
\]
Note that the number of points in the $X\times Y\times Z$ $p$-grid is
no more than $np^3$.  Note, furthermore, that each edge $uw$ of $G$
contains the $p$-grid points
\[
    P_{uw}^p = \{ (i/p)u + ((p-i)/p)w : i\in\{1,\ldots,p-1\} \} \enspace .
\]
Finally, observe that $P_{uw}^p$ does not contain any $q$-grid points
for any prime $q\neq p$.  (This follows from the assumption that no edge
contains a grid point in its interior, so $\gcd(\x(u)-\x(w), \y(u)-\y(w),
\z(u)-\z(w))=1$.  In particular, at least one of $\x(u)-\x(w)$,
$\y(u)-\y(w)$, $\z(u)-\z(w)$ does not have $p$ as a divisor.)

Therefore, there are at least $(p-1)m$ incidences between edges of $G$
and points of the $p$-grid that are not points of any $q$-grid for $q\neq p$.
Any point of the $p$ grid incident
on $k$ edges contributes $\binom{k}{2}$ to $\crs(G)$.  Therefore, the
total contribution of the $p$-grid to $\crs(G)$ is at least
\[
    np^3\binom{m(p-1)/np^3}{2} \ge cm^2/np
\]
for $c<blah$ and $m(p-1)/np^3 \ge 2$.  This latter condition can be rewritten
as $p \le c'(m/n)^{1/2}$ for a sufficiently small $c'>0$.
Finally, we finish by summing over primes $p$ smaller than $c'p^{1/2}$:
\[
   \crs(G) \ge \sum_{p} cm^2/np = (cm^2/n) \sum_{p} 1/p = ((cm^2/n)(\ln\ln(m/n)-O(1))
\]
where the last inequality is Euler's result on the sums of reciprocals
of primes. (This sum is also known as a harmonic series of primes).
\end{proof}

%\To see this, notice that 
%\begin{thm}
%  $g(n,m,q)=\Omega(m^3/q^2)$ 
%\end{thm}
%
%\begin{proof}
%The weaker inequality ,
%\begin{equation}
%  g(n,m,q)\ge m-7q \enspace ,\eqlabel{weak}
%\end{equation}  
%is easily shown by induction
%on $m$, using the result of Cyzowicz \etal\ \cite{X} which states that
%$g(n,m,q) > 0$ if $m> 7q$ (any drawing with more than $7q$ edges contains
%at least one crossing).
%
%Next we show that, using induction on $n$, that 
%\[
%   g(n,m,q) \ge Cm^3/q^2
%\]
%for any $m\ge 8q$.  Note that the case 
%
%
%First, note that the result is easily established
%using \eqref{weak} for any $m$ such that $8q\le m\le 9q$ for $C\le 1/81$.
%
%Next, consider the case $m> 9q$.  Consider some drawing, $\varphi$, of
%$G$ with $\crs(G,q)$ crossings.  Count the pairs $(c,w)$ where $c$ is
%a crossing and $w$ is a vertex not incident to either of the two edges
%involved in $c$.  Each crossing, $c$, appears in $n-4$ such pairs so
%the total number of those pairs is $(n-4)\crs(G,q)$.  On the other hand,
%each vertex, $w$, occurs in at least $\crs(G\setminus w,q)$ such pairs.
%Therefore,
%\[
%   (n-4)\crs(G,q) \ge \sum_{w\in V(G)} \crs(G\setminus w, q) \enspace .
%\]
%The number, $m_w$, of edges in $G\setminus w$ is at least $m-(n-1)
%\ge m-q\ge 8q$ so, by induction, $\crs(G\setminus w,q)\ge Cm_w^3/q^2$.
%Putting everything together, we have
%\begin{align*}
%  (n-4)\crs(G,q) 
%    & \ge  \sum_{w\in V(G)} \crs(G\setminus w, q) \\
%    & \ge  C\sum_{w\in V(G)} m_w^3/q^2  \\
%    & \ge  Cn((n-2)m/n)^3/q^2 & \left(\mbox{since $\sum_{w\in V(G)} m_w^3\ge n((n-2)m/n)^2$}\right) \\
%    & =  (Cm^3/q^2)\cdot(n(n-2)^3/n^3)  \\
%\end{align*}
%Damn! The second factor in the last line is less than 1 and I don't see any
%way to fix it.
%\end{proof}
%

\section{The First Section}

\section{The Second Section}

\section{Summary}
\seclabel{summary}

\section*{Acknowledgement}

The work of Pat Morin was partly funded by NSERC and CFI.

\bibliographystyle{plain}
\bibliography{template}


\end{document}


