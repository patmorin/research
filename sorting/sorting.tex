\documentclass{patmorin}
\usepackage{pat}

\title{An Improved Lower Bound for Comparison-Based Sorting}
\author{\ }


\begin{document}
\maketitle

\section{Introduction}

A \emph{comparison tree}, $T$, for sorting $n$ elements is a binary tree.
Each internal node, $u$, of $T$ is labelled with an ordered pair of
indices $(i_u,j_u)$.  Each leaf, $\ell$, of $T$ is labelled with a
permutation, $\pi_\ell$.

Every sequence, $a_1,\ldots,a_n$, of numbers defines a root-to-leaf path in the comparison tree as follows:  The path begins at the root.  When situated at an internal node $u$, the path continues to $u$'s left child if $i_u < j_u$ and continues to $u$'s right child otherwise.   The last (leaf) node, $\ell$, on this path is labelled with a permutation $\pi_\ell$ such that:
\[
    a_{\pi_\ell(1)} \le a_{\pi_\ell(2)} \le \cdots \le a_{\pi_\ell(n)}
\]

Note that, for any node, $u$, in a comparison tree there exists a
maximal set of permutations, $A(u)$, of $\{1,\ldots,n\}$ such that
every permutation in $A(u)$ leads to $u$.  Note that we may assume that
$A(u)\neq\emptyset$, for all $u$, and that $|A(\ell)|=1$ for every leaf
$\ell$.  We may also assume that each internal node of $T$ has 2 children.
This implies that the algorithm does no redundant comparisons: For every
internal node $u$, $A(u)$ contains inputs in which the $i_u$th
element is greater than the $j_u$th element as well as inputs in
which the $i_u$th element is less than the $j_u$th element.

Throughout the remainder of this paper, we assume that all comparison
trees sort correctly and that they are minimal in the sense of the
preceding paragraph.


\section{The Lower Bound}

Assume $n\equiv 0\pmod{3}$ and group the elements of $a$ into $n/3$
triples, where the $i$th triple, $T_i$, contains the three indices of $a$
containing $3i+1$, $3i+2$, $3i+3$.

For any triple, $T_i$, on input $a=a_1,\ldots,a_n$, the path for $a$
in $T$ has a first node, $u$, for which $i_u$ and $j_u$ both belong
to $T_i$.  There is also a second node, $w$, for which $i_w$ and $j_w$
both belong to $T_i$.  We say that the $T$ \emph{wins} on triple $T_i$
if the two comparisons done at $u$ and $w$ are sufficient to determine
the relative order of the elements in $T_i$.  This happens, for example,
when $j_u=i_w$ and
\[
     a_{i_u} < a_{j_u} = a_{i_w} < a_{j_w} \enspace .
\]
When this does not happen, $T$ is forced to perform a third comparison
involving a pair from the triple $T_i$.  In this case, we say that $T$
\emph{loses} on triple $T_i$.

\begin{lem}\lemlabel{triples}
  Let $a_1,\ldots,a_n$ be a random permutation of $\{1,\ldots,n\}$.
  Then the expected number of triples on which $T$ wins in $n/9$.
  Equivalently, the expected number of triples on which $T$ loses
  is $2n/9$.
\end{lem}

\begin{proof}
Easy.
\end{proof}

\begin{lem}\lemlabel{permutation}
  For any decision tree $T$, the exists an input $a=a_1,\ldots,a_n$
  such that the root-to-leaf path defined by $a$ has length at least
  $\log(n!)-4$ and the number of triples on which $T$ wins is at least
  $n/12$.
\end{lem}

\begin{proof}
Let $a$ be a random input.  Then, with probability at least $15/16$, the
length of the path defined by $a$ is at least $\log(n!)-4$.  (Any binary
tree contains at most $k/16$ leaves of depth at most $\log(k)-4$.)

On the other hand, the probability that $T$ wins on fewer than $n/12$
triples is equal to the probability that $T$ loses on more than $3n/12$
triples.  Letting $X$ denote the number of triples on which $T$ loses
we have, By Markov's Inequality and \lemref{triples},
\[
   \Pr\{X\ge (3/4)(n/3)\} = \Pr\{X \ge (9/8)\E[X]\} \le 8/9 \enspace . 
\]
Therefore, with probability at least $15/16$ the path defined by $a$
has length at least $\log(n!)-4$ and with probability at least $1/9$, $T$
wins on at least $n/12$ triples of $a$.  Since $15/16+1/9 > 1$ we conclude
that there exists some $a$ that satisfy the conditions of the lemma. (In
fact, a constant fraction of all permutations satisfy the lemma.)
\end{proof}

\subsection{Finding a Long Path in $T$}

Next, we show how the permutation, $a$, guaranteed by
\lemref{permutation}, can be used to find a root-to-leaf path of length
$\log(n!)+\Omega(n)$ in any comparison tree.  The construction works
by starting with the permutation $a$ and modifying the relative order
of pairs of elements within winning triples so that they become losing
triples.

For this strategy to work, we also modify the decision tree, $T$, in
ways that do not increase the length of the original search path root
to leaf path.  We do this with two kinds of operations.  The first such
operation is a \emph{lifting}.  The second is a \emph{switcheroo}.

\subsection{Lifting}

Let $a$ be an input on which a search tree loses on the triple $T_i$.
Let $u$ be the node at which this triple loses and, suppose that the
search path for $a$ proceeds to $u$'s left child.  Let $A\subset A(u)$
be subset of $A(u)$ that respects the triple structure of $a$ and
 that do not modify the triple $T_i$.  Then, every root-to-leaf path
 defined by elements of $A$ contains $u$ and, furthermore, contains a
 node in the subtree of $u$ that contains that compares the third pair
 in $T_i$.
\begin{proof}

\end{proof}



\end{document}
