%% -*- LaTeX -*- This is LaTeX2e code

\newcommand{\LA}{\ensuremath{<_{y}}}
\newcommand{\LB}{\ensuremath{<_{x}}}

\section{Selecting the $k$-th smallest element} \label{sec:detkmed}

In this appendix, we describe a space-efficient variant of the
well-known median-find algorithm by Blum \etal~\cite{blum:selection}.
Our algorithm assumes that the input is given in the form of an array
${A}[0\ldots n-1]$ which is sorted according to some total
order \LA. The goal of the algorithm is, given an integer $k\in
[0\ldots n-1]$, to select the $k$-th smallest element in {A}
\emph{according to some other total order} \LB. This algorithm will
run in linear time and will require only \Oh{1} extra space. 
Additionally, the algorithm returns the array ${A}[0\ldots n-1]$ 
in the same order as it was presented, namely sorted according to \LA.

The correctness of the algorithm described below will follow from the
following invariant:

\begin{description}
\item[Invariant:] Assume the algorithm is called to select the $k$-th
  smallest element from a \LA-sorted (sub-)array
  ${A}[b\ldots e-1]$, where $b,e$, and $k$, are three global
  variables. Then, upon returning from this call, $b,e$, and $k$ 
  have been restored to the values they had when the algorithm was 
  called. Additionally, ${A}[b\ldots e-1]$ is sorted according 
  to \LA. 
\end{description}


\newcounter{ALGBREAK}
\newcounter{SelectA}
\newcounter{SelectB}
\newcounter{SelectC}
\newcounter{StackA}
\newcounter{StackB}

\begin{algorithm}
  \caption{The algorithm
    $\textsc{RestoringSelect}({A},b,e,k,\text{mode})$ for
    selecting the $k$-th smallest element in ${A}[b\ldots
    e-1]$ (if $\text{mode}=\textsc{Select}$) or the median element (if
    $\text{mode}=\textsc{Median}$).}
  \begin{algorithmic}[1]
    \REQUIRE ${A}[b\ldots e-1]$ is sorted according to
    \LA. 
    \ENSURE ${A}[b\ldots e-1]$ is sorted according to
    \LA. The variables $b,e$, and $k$ are reset to their
    original values. 
    \STATE Subdivide ${A}[b\ldots e-1]$ into $\lfloor(e-b)/5
    \rfloor$ groups of $5$ elements and (possibly) one group of size 
    $\leq 4$.  
    \STATE Move the medians of the first $\lfloor(e-b)/5 \rfloor$ 
    groups to ${A}[b\ldots b+\lfloor(e-b)/5\rfloor-1]$ using 
    algorithm \textsc{SortedSubsetSelection}.  
    \STATE $i \leftarrow (e-b) - \lfloor(e-b)/5\rfloor\cdot 5$
    \STATE $\textsc{Push}(S,i)$ \setcounter{StackA}{\value{ALC@line}}
    \STATE
    $i_{\mathrm{med}} \leftarrow 
       \textsc{RestoringSelect}({A},b,b+\lfloor(e-b)/5\rfloor,k,\textsc{Median})$
    \setcounter{SelectA}{\value{ALC@line}}
    \STATE
    $\textsc{UndoSubsetSelection}({A},b,b+\lfloor(e-b)/5\rfloor\cdot 5, \lfloor(e-b)/5\rfloor)$
    \STATE $i \leftarrow \textsc{Pop}(S)$ \setcounter{StackB}{\value{ALC@line}}
    \STATE $e \leftarrow b + \lfloor(e-b)/5\rfloor \cdot 5 + i$
    \IF{$\text{mode} = \textsc{Median}$}
      \STATE $l \leftarrow \lfloor (e-b)/2 \rfloor$
    \ELSE
      \STATE $l \leftarrow k$
    \ENDIF
    \STATE $x \leftarrow {A}[i_{\mathrm{med}}]$
    \STATE 
     $k_< \leftarrow |\left\{a\in{A}[b\ldots e-1] \mid a<x\right\}|$
    \STATE 
     $k_= \leftarrow |\left\{a\in{A}[b\ldots e-1] \mid a=x\right\}|$ 
    \STATE 
     $k_> \leftarrow |\left\{a\in{A}[b\ldots e-1] \mid a>x\right\}|$
    \IF{$l\notin [k_< + 1, k_< + k_=]$}
      \IF{$l \leq k_<$}
        \IF{$l = k_<$}
          \STATE Set $i_{\mathrm{med}}$ to point to the largest element
          in ${A}[b\ldots e-1]$ less than $x$ (according to \LB).
        \ELSE
          \STATE Move $x$ to ${A}[b]$.
          \STATE Using \textsc{SortedSubsetSelection}, move all elements in
          ${A}[b+1\ldots e-1]$ less than $x$ to
          ${A}[b+1\ldots b+k_<]$.
          \STATE Move the largest element less than $x$ (according to
          \LB) to ${A}[e-1]$ .
          \STATE
          $i_{\mathrm{med}} \leftarrow 
            \textsc{RestoringSelect}({A},b+1,b+k_<,k,\textsc{Select})$
          \setcounter{SelectB}{\value{ALC@line}}
          \STATE Starting at ${A}[b+k_<]$, scan ${A}$ to find
          $e-1$ (the index of the first element $y$ for which 
          $y\LB x:=A[b]$). 
          \STATE Move ${A}[e-1]$ to its proper position in
          ${A}[b+1\ldots b+k_<]$.
          \STATE $\textsc{UndoSubsetSelection}({A},b+1,e,b+k_<+1)$.
          \STATE Reinsert (according to \LA) $x:={A}[b]$ into
          ${A}[b\ldots e-1]$ maintaining $i_{\mathrm{med}}$.
        \ENDIF
          \setcounter{ALGBREAK}{\value{ALC@line}}
       \ELSE
         \STATE (\ldots) \\[2ex]
         \setcounter{ALC@line}{44}
       \ENDIF
     \ENDIF
     \STATE Return $i_{\mathrm{med}}$.
 \end{algorithmic}
\end{algorithm}

\begin{algorithm}
  \addtocounter{algorithm}{-1}
  \caption{Algorithm
    $\textsc{RestoringSelect}({A},b,e,k,\text{mode})$ (contd.)}
  \begin{algorithmic}[1]
    \setcounter{ALC@line}{17}
    \IF{$l\notin [k_< + 1, k_< + k_=]$}
      \IF{$l \leq k_<$}
        \STATE (\ldots)\\[2ex]
        \setcounter{ALC@line}{\value{ALGBREAK}}
      \ELSE
        \IF{$l = b-e$}
          \STATE Set $i_{\mathrm{med}}$ to point to the largest element
          in ${A}[b\ldots e-1]$ larger than $x$ (according to \LB).
        \ELSE
          \STATE Using \textsc{SortedSubsetSelection}, move all elements in
          ${A}[b+1\ldots e-1]$ larger than $x$ to
          ${A}[b+1\ldots b+k_>]$.
          \STATE Move the smallest element larger than $x$ (according to
          \LB) to ${A}[e-1]$.
          \STATE
          $i_{\mathrm{med}} \leftarrow 
           \textsc{RestoringSelect}({A},b+1,b+k_>,k-(k_<+k_=),\textsc{Select})$.
          \setcounter{SelectC}{\value{ALC@line}}
          \STATE Starting at ${A}[b+k_>]$, scan ${A}$ to find
          $e-1$ (the index of the first element $y$ for which 
          ${A}[b] =: x <_{x} y$). 
          \STATE Move ${A}[e-1]$ to its proper position in
          ${A}[b+1\ldots b+k_>]$.
          \STATE $\textsc{UndoSubsetSelection}({A},b+1,e,b+k_>+1)$.
          \STATE Reinsert (according to \LA) $x:=A[b]$ into
          ${A}[b\ldots e-1]$ maintaining $i_{\mathrm{med}}$.
          \STATE Recompute $k_<$ and $k_=$ (as above) and set 
            $k \leftarrow (k-(k_<+k_=))+k_<+k_=$. 
        \ENDIF
       \ENDIF
    \ENDIF
    \STATE Return $i_{\mathrm{med}}$.
  \end{algorithmic}
\end{algorithm}

The invariant is to enforce trivially for any constant-sized input. 
Algorithm~13 is described in a recursive way to facilitate the 
analysis and the proof of correctness. We can convert this algorithm 
into a non-recursive variant by simply maintaining a stack of 
two-bit entries that indicate whether the current ``invocation'' took 
place from line \theSelectA, \theSelectB, or \theSelectC. This stack 
has a worst-case depth of \Oh{\log n} and thus will (in an asymptotic 
sense) not increase the extra space required by this algorithm. 
Also, the stack $S$ used in lines \theStackA\ and \theStackB\ 
contains only integers in the range $[0\ldots 4]$, so its overall 
size is bounded by \Oh{\log n} bits,
too. 

Assuming that the above invariant is fulfilled for any constant-size
input, we can inductively assume that the invariant holds after the
``recursive'' call in line~\theSelectA. This implies that for the
successive call to \textsc{UndoSubsetSelection} the parameters $b$,
$\lfloor(e-b)/5\rfloor$, and hence also $e_1:=b+\lfloor(e-b)/5\rfloor$
and $e_2:=b+\lfloor(e-b)/5\rfloor\cdot 5$ are known. The situation
prior to the call to \textsc{UndoSubsetSelection} is depicted in
Figure~\ref{fig:undomedian}: The array ${A}[b\ldots e_1-1]$
contains the medians in sorted \LA-order that had been
selected from ${A}[b\ldots e_2-1]$, and the remaining $i$
elements are still untouched, hence also sorted according to
\LA. 

\begin{figure}[ht]
  \renewcommand{\arraystretch}{1.75}
  \begin{center}
  \begin{tabular}{cccrrcrl}
    \hline
    \ldots & 
    \multicolumn{2}{|c|}{medians \LA-sorted} &
    \multicolumn{2}{c|}{unsorted} &
    \multicolumn{2}{c|}{rest \LA-sorted} & 
    \ldots \\
    \hline 
    &
    {\small $b$} &
    &
    {\small $e_1$} & 
    &
    {\small $e_2$} & 
    &
    {\small $e_2+i$} 
  \end{tabular}
  \end{center}
  \caption{Restoring the \LA-order after having computed
    the median of the $\lfloor(b-e)/5\rfloor$ medians. 
    Here $e_1:= b + \lfloor(b-e)/5\rfloor$ and  
    $e_2:= b + \lfloor(b-e)/5\rfloor \cdot 5$.}
  \label{fig:undomedian}
\end{figure}

As a consequence, we can first undo the effects of
\textsc{SortedSubsetSelection} on ${A}[b\ldots e_2 -1]$, hence
restoring it to sorted \LA-order and finally adjust $e$
to point to $e_2 + i$. This implies that $b$ and $e$ are known and
${A}[b\ldots e-1]$ is sorted according to \LA. As the original
value of $k$ had been passed to the ``recursive'' call to
\textsc{RestoringSelection}, by the invariant, we still know its value.

For the second and third situtation in which a ``recursive'' call to
\textsc{RestoringSelection} may happen (lines \theSelectB\ and
\theSelectC), we do not know how many elements are passed to
the ``recursive'' call. In order to recover the original value of
$e$ after the call, we move the median-of-medians $x$ to the front of
the array and use a distinguished element $y$ to denote the \emph{end}
of the subarray that is not passed to the recursive call. Let us 
consider the situation where the $k$-th element to be selected is 
larger (according to \LB) than the median-of-medians $x$ (the other 
situation is handled analogously). Prior to the ``recursive'' call 
in line \theSelectC\ we have moved all elements larger than $x$ to 
the front of the array, more specifically to the subarray 
${A}[b+1\ldots b+k_>]$. Then we find the largest element 
larger than $x$ (using a single scan) and move it to the end of the 
current array ${A}[b\ldots e-1]$. This element, being larger 
than $x$, is also larger than any element not passed to the 
``recursive'' call and will be the first element larger than $x$ 
encountered when scanning the array ${A}$ starting from 
${A}[b+k_>]$ (Figure~\ref{fig:undoselect}).

\begin{figure}[ht]
  \renewcommand{\arraystretch}{1.75}
  \begin{center}
  \begin{tabular}{ccccrrcl}
    \hline
    \ldots & 
    \multicolumn{1}{|c}{$x$} &
    \multicolumn{2}{|c|}{``$x$ \LB'' \LA-sorted} &
    \multicolumn{2}{c|}{``$x \not<_{x}$'' unsorted} &
    \multicolumn{1}{c|}{$y$} & 
    \ldots \\
    \hline 
    &
    {\small $b$} &
    {\small $b+1$} &
    &
    {\small $b+k_>$} & 
    & {\small ?}
    & {\small $e$}
  \end{tabular}
  \end{center}
  \caption{Restoring the \LA-order after having selected the
    $k-(k_<+k_=)$-th element from ${A}[b+1\ldots b+k_>-1]$.
    Here $y$ is the largest element in ${A}[b\ldots e-1]$ for
    which $x\LB y$.}
  \label{fig:undoselect}
\end{figure}

By the invariant, we know that after the ``recursive call'' to
\textsc{RestoringSubsetSelection}$({A},b+1,b+k_>,k-(k_<+k_=)$,
the subarray ${A}[b+1\ldots b+k_>-1]$ will be sorted according
to \LA, and we will know the values of $b+1$, $b+k_>$, and
$k-(k_<+k_=)$. This enables us to retrieve the median-of-medians $x$
from ${A}[b]$ and (starting from ${A}[b+k_>]$) to scan
for the first element larger than $x$. After we have found this
element at position $e-1$, we have restored to original value of $e$,
and a single scan over ${A}[b\ldots e-1]$ allows us to
compute the values $k_<$ and $k_=$, which are needed to restore $k$ to
its original value. 

Inductively, we see that the invariant holds for the initial call to
the algorithm, and this implies that the algorithm selects the $k$-th
smallest element according to \LB\ while maintaining the \LA-order in
which the elements had been given. The space requirement of this
algorithm is \Oh{\log n} bits, because besides a constant number of
indices, only two stacks of size \Oh{\log n} bits are needed. Using
the analysis of the original algorithm by Blum 
\etal~\cite{blum:selection}, the running time can be shown to be
\Oh{n}, and we conclude with the following theorem:

\begin{theorem}\label{lem:select}
 The $k$-th smallest element in an array of $n$ element can be 
 selected in linear time using \Oh{1} extra space. Furthermore, if 
 the set is given sorted according to some total order, this order 
 can be restored in the same time and space complexity.
\end{theorem}

%%% Local Variables:
%%% mode: latex
%%% TeX-master: "paper"
%%% End:
