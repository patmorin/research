\documentclass[12pt]{article}
\usepackage{amsopn,amsthm,amsmath}

\newtheorem{thm}{Theorem}
\DeclareMathOperator{\cn}{cr}

\begin{document}
\begin{thm}
Let $S$ be a set of $r$ points and let $H_0,\ldots,H_{t-1}$ be a
set of lines such that the orthogonal projection of $S$ onto $H_{j}$
gives exactly $i+j$ distinct values, for each $j\in\{0,\ldots,t-1\}$.
Then $t \le \tau i$.
\end{thm}

\begin{proof}
First we observe that the number of points, $r$, satisfies $r \ge i+t-1$.
Each projection direction $H_j$ defines a set of $i+j$ parallel lines,
each of which contains at least one point of $S$.  In total, this defines
a set $L$ of
\[
   \ell = \sum_{j=0}^{t-1} (i+j) = ti + t(t-1)^2/2 \le ti + t^2/2 \enspace .
\]
lines.  The total number of incidences $I$ between the these lines and
the points in $S$ is $tr$.

Define a geometric graph $G$ having $r+2$ vertices consisting
of the points in $S$ plus two points ``at infinity,'' as in
Figure~\ref{fig:G}. Notice that $G$ is a graph with no self-loops and no
parallel edges and can be drawn so that the only crossings occur where
lines in $L$ intersect each other.  Thus, the crossing number of $G$
is at most $\cn(G) \le {\ell}^{2}$.  Furthermore, the number of
edges in $G$ is exactly $tr + \ell$.

The Crossing Lemma \cite{X} then implies that
\[
    {\ell}^2 \ge \cn(G) \ge \gamma\cdot \frac{(tr+\ell)^3}{(r+2)^2} \enspace .
\]
We can rewrite this as
\begin{eqnarray*}
  t & \le & \frac{\ell^{2/3}(r+2)^{2/3} - \ell}{\gamma^{1/3}r} \\
  & \le & \frac{\ell^{2/3}}{r^{1/3}} + \frac{\gamma^{1/3}\ell^{2/3} - {\ell}}{r} \\
\end{eqnarray*}

\end{proof}



\end{document}
