\documentclass{patmorin}
\usepackage{amsthm,amsmath,graphicx,stmaryrd}
\usepackage{pat}

\newcommand{\eps}{\varepsilon}

\title{\MakeUppercase{A Crossing Number Inequality for 3D Grid Drawings}}
\author{Sheldon Cooper, Pat Morin,\thanks{School of Computer Science, Carleton University, \email{morin@scs.carleton.ca}}\ and Howard Wolowitz}



\begin{document}
\maketitle

\begin{abstract}
   We show that when a graph with $m\ge cq$ edges is drawn on a 3D-grid
   with volume $q$, the resulting drawing contains $\Omega(m^3/n^2)$
   crossings.
\end{abstract}

\section{Introduction}

The \emph{$X\times Y\times Z$ grid} is the point set $\{(x,y,z):
x\in\{1,\ldots,x\},\, y\in\{1,\ldots,Y\},\, z\in\{1,\ldots,Z\}\}$.
The volume the $X\times Y\times Z$ grid is $XYZ$. A \emph{(proper)
drawing}, $\varphi:V(G)\rightarrow R$, of a graph, $G$, on a grid,
$R$, is a one-to-one mapping such that, for every edge $uw\in E(G)$ and
every vertex $x\in V(G)$, the open line segment with endpoints $\varphi(u)$ and $\varphi(w)$
does not contain $x$.

Define $\chi(G,q)$ to be the minimum number of edge crossings in a drawing of $G$ on a grid whose volume is $q$.

Define $g(n,m,q)=\min\{\chi(G,q):\text{$G$ is a graph with
$n$ vertices and $m$ edges}\}$.

\begin{thm}
  $g(n,m,q)=\Omega(m^3/q^2)$ 
\end{thm}

\begin{proof}
The weaker inequality ,
\begin{equation}
  g(n,m,q)\ge m-7q \enspace ,\eqlabel{weak}
\end{equation}  
is easily shown by induction
on $m$, using the result of Cyzowicz \etal\ \cite{X} which states that
$g(n,m,q) > 0$ if $m> 7q$ (any drawing with more than $7q$ edges contains
at least one crossing).

Next we show that, using induction on $n$, that 
\[
   g(n,m,q) \ge Cm^3/q^2
\]
for any $m\ge 8q$.  Note that the case 


First, note that the result is easily established
using \eqref{weak} for any $m$ such that $8q\le m\le 9q$ for $C\le 1/81$.

Next, consider the case $m> 9q$.  Consider some drawing, $\varphi$, of
$G$ with $\chi(G,q)$ crossings.  Count the pairs $(c,w)$ where $c$ is
a crossing and $w$ is a vertex not incident to either of the two edges
involved in $c$.  Each crossing, $c$, appears in $n-4$ such pairs so
the total number of those pairs is $(n-4)\chi(G,q)$.  On the other hand,
each vertex, $w$, occurs in at least $\chi(G\setminus w,q)$ such pairs.
Therefore,
\[
   (n-4)\chi(G,q) \ge \sum_{w\in V(G)} \chi(G\setminus w, q) \enspace .
\]
The number, $m_w$, of edges in $G\setminus w$ is at least $m-(n-1)
\ge m-q\ge 8q$ so, by induction, $\chi(G\setminus w,q)\ge Cm_w^3/q^2$.
Putting everything together, we have
\begin{align*}
  (n-4)\chi(G,q) 
    & \ge  \sum_{w\in V(G)} \chi(G\setminus w, q) \\
    & \ge  C\sum_{w\in V(G)} m_w^3/q^2  \\
    & \ge  Cn((n-2)m/n)^3/q^2 & \left(\mbox{since $\sum_{w\in V(G)} m_w^3\ge n((n-2)m/n)^2$}\right) \\
    & =  (Cm^3/q^2)\cdot(n(n-2)^3/n^3)  \\
\end{align*}
Damn! The second factor in the last line is less than 1 and I don't see any
way to fix it.
\end{proof}


\section{The First Section}

\section{The Second Section}

\section{Summary}
\seclabel{summary}

\section*{Acknowledgement}

The work of Pat Morin was partly funded by NSERC and CFI.

\bibliographystyle{plain}
\bibliography{template}


\end{document}


