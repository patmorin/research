\documentclass{cccg10}
\usepackage{amsopn,amsmath,graphicx}
\listfiles
\usepackage{pat}
\DeclareMathOperator{\cn}{cr}

\newcommand{\const}{268.19}

\title{Ghost Chimneys}
\author{Brad Ballinger?
	\and David Charlton?
	\and Erik D. Demaine 
	\and Martin Demaine
	\and Vida Dujmovi\'c
	\and Pat Morin
	\and Ryuhei Uehara} 

\begin{document}
\maketitle

\begin{abstract}
A set $S$ of points in $\R^2$ is an $(i,t)$ set of ghost chimneys
if there exists lines $H_0,\ldots,H_{t-1}$ such that the orthogonal
projection of $S$ on $H_j$ consists of exactly $i+j$ distinct points.
In this paper we give upper and lower bounds on the maximum value of $t$
in an $(i,t)$ set of ghost chimneys.
\end{abstract}

\section{Introduction}

We consider the following problem:  Given an integer $i$, what
is the maximum value $t(i)$ such that there exists a set of points
$S\subset\R^2$ and a set $H_0,\ldots,H_{t-1}$ of lines where, for each
$j\in\{0,\ldots,t(i)-1\}$, the orthogonal projection of $S$ onto $H_j$
consists of exactly $i+j$ distinct points.  We prove the following result:

\begin{thm}\thmlabel{main}
For any integer $i\ge 1$,  $2i \le t(i) \le 123.33i$.
\end{thm}

In addition to \thmref{main}, we show that $t(1)=2$, $t(2)=5$, $t(3)=9$,
and $t(4)\ge 12$.  These results show that neither the lower bound
nor the upper bound of \thmref{main} is tight for all values of $i$.
\thmref{main} is an immediate consequence of \lemref{lower-bound} and
\lemref{upper-bound}, which are proven in the next two sections.

\section{The Lower Bound}

\begin{lem}\lemlabel{lower-bound}
For each integer $i\ge 1$, there exists a set $S=S(i)$ of $3i-1$
points and a set $H_0,\ldots,H_{2i-1}$ of lines such that, for each
$j\in\{0,\ldots,2i-1\}$, the orthogonal projection of $S$ onto $H_j$
has exactly $i+j$ distinct values.
\end{lem}

\begin{proof}
The point set $S$ consists of the points of a $i\times 3$ grid with the bottom-right corner removed (see \figref{lower-bound}).  For even $j$, $H_j$ is a line of slope $j/2$.  For odd $j$, $H_j$ is a line of slope $-(j+1)/2$.
\begin{figure*}
  \begin{center} 
    \begin{tabular}{cc}
      & \includegraphics{j0} \\ 
      & $j = 0$  \\
       \includegraphics{j1} & \includegraphics{j2} \\
        $j=1$ & $j=2$ \\
       \includegraphics{j3} & \includegraphics{j4} \\
        $j=3$ & $j=4$
    \end{tabular}
  \end{center}
  \caption{The set $S(i)$  for $i=9$ and the projection directions that yield $i,\ldots,i+4$ distinct points.}
  \figlabel{lower-bound}
\end{figure*}
\end{proof}



\section{The Upper Bound}

Our upper-bound proof is closely related to Sz\'ekely's proof of the
Szem\'eredi-Trotter Theorem \cite{s97}.  We make use of the following
version of the Crossing Lemma, which was proven by Pach, Radoici\'c,
Tardos, and T\'oth \cite{prtt04}:

\begin{lem}[Crossing Lemma]\label{lem:crossing}
Let
$\beta=103/6$, $\gamma = 1024/31827$, and let
$G$ be a graph with no self loops, no parallel edges, $v$ vertices, and
$e > \beta v$ edges.  Then
\[
  \cn(G) \ge \gamma \cdot \frac{e^3}{v^2} \enspace .
\]
\end{lem}


\begin{lem}\lemlabel{upper-bound-general}
Let $t=\alpha i$, let $S$ be a set of $r$ points, and let
$H_0,\ldots,H_{t-1}$ be a set of lines such that the orthogonal projection
of $S$ onto $H_{j}$ gives exactly $i+j$ distinct values.  Then, $t\le 34$
or $r\le \max\{4,2/\alpha + 2 + \alpha/2\}i/\gamma$.
\end{lem}

\begin{proof}
Each projection direction $H_j$ defines a set $L_j$ of $i+j$ parallel
lines, each of which contains at least one point of $S$.  Let $G$ be
the geometric graph that contains the points in $S$ plus $t$ additional
points $p_0,\ldots,p_{t-1}$.  Two vertices in $S$ are connected by an
edge in $G$ if and only if they occur consecutively on some line in
$\bigcup_{j=0}^{t-1}L_j$.  Additionally, each vertex $p_j$ is connected
to each of the $i+j$ lexicographically largest points on each of the
lines in $L_j$.  See \figref{graph}.

\begin{figure}
  \begin{center}
    \includegraphics{graph}
  \end{center}
  \caption{The graph $G$ for a set of points with $i=9$ and $t=3$.}
  \figlabel{graph}
\end{figure}

The graph $G$ has $t+r$ vertices
and $tr$ edges.  Observe that we have a drawing of $G$ so that the only
crossings between edges occur where lines in $L$ intersect each other.
The total number of intersecting pairs of lines in $L$ is
\[
  \begin{aligned}
    X 
      & = \sum_{j=1}^{t-1}(i+j)\cdot\sum_{k=0}^{j-1}(i+k) \\
      & \le \sum_{j=1}^{t-1}(i+j)(ij + j^2/2) \\
      & \le \sum_{j=1}^{t-1}(i^2j+3ij^2/2 + j^3/2) \\
      & \le i^2t^2/2 + it^3/2 + t^4/8 \enspace .
  \end{aligned}
\]

Applying \lemref{crossing}, we learn that either
\begin{equation}
   tr \le \beta (t+r) \enspace , \eqlabel{crossing-a}
\end{equation}
or
\begin{equation}
   X \ge \cn(G) \ge \gamma \frac{(tr)^3}{(t+r)^2} \eqlabel{crossing-b} \enspace .
\end{equation}

In the former case, we rewrite \eqref{crossing-a} to obtain
\[
   t \le \beta(t/r + 1) \le 2\beta \le 34 + 1/3 \enspace ,
\]
so $t\le 34$ (since $t$ is an integer).

In the latter case, we expand \eqref{crossing-b} to obtain
\[ i^2t^2/2 + it^3/2 + t^4/8 \ge \gamma\frac{(tr)^3}{(t+r)^2} .  \]
Substituting $t=\alpha i$ gives
\[ i\left(\frac{1}{2\alpha} 
    + \frac{1}{2}+\frac{\alpha}{8}\right) 
      \ge \gamma \frac{r^3}{(t+r)^2} \ge \gamma r/ 4 \enspace ,
\]
where the second inequality follows from the fact that $t\le i+t-1 \le r$.
Rewriting to isolate $r$ finally gives
\[
  r \le \left(\frac{2}{\alpha} + 2 +\frac{\alpha}{2}\right)i/\gamma \enspace .
\]
We finish the proof by observing that, for $\alpha > 2$, the inequality
$r\le 4i/\gamma$ obtained by setting $\alpha=2$ is stronger and still
applies.
\end{proof}

\begin{lem}\lemlabel{upper-bound}
For all integers $i\ge 1$, $t(i) \le 123.33i$.
\end{lem}

\begin{proof}
Observe that $i+t-1\le r$.  Therefore, for $i\ge 18$, the lemma
follows by applying \lemref{upper-bound-general} with $\alpha=2$.
For $i\in\{1,\ldots,17\}$, the lemma follows by setting $\alpha = 35/i$.
\end{proof}

\section{Small Values of $i$}

In this section we give some tighter bounds on $t(i)$ for $i\in\{1,2,3,4\}$.

\begin{lem}\lemlabel{i1i2}
$t(1) = 2$, and $t(2)=5$.
\end{lem}

\begin{proof}
Point sets achieving these bounds are the $1\times 2$ and the $2\times
3$ grid, respectively (see \figref{i1i2}).  That these point sets are
optimal follows from the fact that the existence of $H_0$ and $H_1$
implies that the points of $S$ lie on the intersection of $i$ parallel
lines with another set of $i+1$ parallel lines.  Thus, $|S|\le i(i+1)$,
so $t(i) \le |S|-i+1\le i^2+1$.
\end{proof}

\begin{figure}
  \begin{tabular}{cc}
    \includegraphics{i1} & \includegraphics{i2} \\
     (a) & (b)
  \end{tabular}
  \caption{Point sets showing that (a)~$t(1) \ge 2$ and (b)~$t(2) \ge 5$.}
  \figlabel{i1i2}
\end{figure}

Notice that the proof of \lemref{i1i2} implies that, for any $i$, $t(i)\le
i^2 + 1$.  The following lemma shows that, for $i\ge 3$, $t(i)\le i^2$.  Of course, this upper bound is tighter than \lemref{upper-bound} for $i\le 123$.

\begin{lem}\lemlabel{i3}
$t(3)=9$.
\end{lem}

\begin{proof}
The point set $S(4)$ described in the proof of \lemref{lower-bound}
results in 3 distinct points when projected onto a vertical line, therefore
$t(3)\ge 9$.

For the upper bound, refer to \figref{3opt}.  By an affine transformation,
we may assume that $H_0$ is vertical and $H_1$ is horizontal.  Thus, the
points of $S$ are contained in the intersection of 3 horizontal lines
with 4 vertical lines.  This establishes that $|S|\le 12$, so $t(3) \le 10$.
To see that $|S|< 12$, assume otherwise and consider any line $\ell$ that
is neither horizontal nor vertical. By a reflection through a horizontal
line, we may assume that $\ell$ has positive slope, so that every point
on the bottom row and right column of $S$ has a distinct projection
onto $\ell$, so $S$ projects onto at least 6 distinct points on $\ell$.
In particular, this implies that there is no line $H_2$ such that $S$
projects onto 5 distinct points on $H_2$.
\end{proof}
\begin{figure}
  \begin{center}
    \includegraphics{3opt}
  \end{center}
  \caption{The proof of \lemref{i3}.}
  \figlabel{3opt}
\end{figure}

\begin{lem}\lemlabel{lower-bound-4}
$12 \le t(4) \le 15$.
\end{lem}

\begin{proof}
The point set and lines $H_0,\ldots,H_{10}$ that show $t(4)\ge 12$
are shown in \figref{i4}.  ($H_{11}$ is omitted since any sufficiently
general line will do.)

To see that $t(4) \le 18$, we argue as in the proof of the second half
of \lemref{i3}.  This establishes that $|S|\le 20$.  If $|S|\in\{19,20\}$
then the number of distinct projections of $S$ onto $\ell$ is at least $7$,
but this contradicts the existence of $H_2$.  Thus, we must have $|S|\le
18$, to $t(4)\le 15$.
\end{proof}

\begin{figure*}
  \begin{center}
    \begin{tabular}{cc}
      \includegraphics{i4a} & \includegraphics{i4b}
    \end{tabular}
  \end{center}
  \caption{A $(4,12)$ set of ghost chimneys.}
  \figlabel{i4}
\end{figure*}


\section{Conclusions}

We have given upper and lower bound on the value of $t$, as a function of
$i$, in an $(i,t)$ set of ghost chimneys.  These bounds differ only by a
(admittedly large) constant factor.  Reducing this factor remains an open
problem.
For small values of $i$, we have shown that $t(1)=2$, $t(2)=5$,
$t(3)=9$, and $t(4)\ge 12$.

%Another open problem is the generalization of these results to 3,
%or higher, dimensions: Given an integer $i$, what is the maximum
%value $t(i)$ such that there exists a set of points $S\subset\R^d$
%and a set $H_0,\ldots,H_{t-1}$ of hyperplanes where, for each
%$j\in\{0,\ldots,t(i)-1\}$, the orthogonal projection of $S$ onto $H_j$
%consists of exactly $i+j$ distinct points?

\bibliographystyle{plain}
\bibliography{chimneys}



\end{document}
