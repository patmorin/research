\documentclass[lotsofwhite,charterfonts]{patmorin}
 
%\usepackage{amsthm}

\newcommand{\centeripe}[1]{\begin{center}\Ipe{#1}\end{center}}
\newcommand{\comment}[1]{}

\newcommand{\centerpsfig}[1]{\centerline{\psfig{#1}}}

\newcommand{\seclabel}[1]{\label{sec:#1}}
\newcommand{\Secref}[1]{Section~\ref{sec:#1}}
\newcommand{\secref}[1]{\mbox{Section~\ref{sec:#1}}}

\newcommand{\alglabel}[1]{\label{alg:#1}}
\newcommand{\Algref}[1]{Algorithm~\ref{alg:#1}}
\newcommand{\algref}[1]{\mbox{Algorithm~\ref{alg:#1}}}

\newcommand{\applabel}[1]{\label{app:#1}}
\newcommand{\Appref}[1]{Appendix~\ref{app:#1}}
\newcommand{\appref}[1]{\mbox{Appendix~\ref{app:#1}}}

\newcommand{\tablabel}[1]{\label{tab:#1}}
\newcommand{\Tabref}[1]{Table~\ref{tab:#1}}
\newcommand{\tabref}[1]{Table~\ref{tab:#1}}

\newcommand{\figlabel}[1]{\label{fig:#1}}
\newcommand{\Figref}[1]{Figure~\ref{fig:#1}}
\newcommand{\figref}[1]{\mbox{Figure~\ref{fig:#1}}}

\newcommand{\eqlabel}[1]{\label{eq:#1}}
\newcommand{\eqref}[1]{(\ref{eq:#1})}

\newtheorem{thm}{Theorem}{\bfseries}{\itshape}
\newcommand{\thmlabel}[1]{\label{thm:#1}}
\newcommand{\thmref}[1]{Theorem~\ref{thm:#1}}

\newtheorem{lem}{Lemma}{\bfseries}{\itshape}
\newcommand{\lemlabel}[1]{\label{lem:#1}}
\newcommand{\lemref}[1]{Lemma~\ref{lem:#1}}

\newtheorem{cor}{Corollary}{\bfseries}{\itshape}
\newcommand{\corlabel}[1]{\label{cor:#1}}
\newcommand{\corref}[1]{Corollary~\ref{cor:#1}}

\newtheorem{obs}{Observation}{\bfseries}{\itshape}
\newcommand{\obslabel}[1]{\label{obs:#1}}
\newcommand{\obsref}[1]{Observation~\ref{obs:#1}}

\newtheorem{assumption}{Assumption}{\bfseries}{\rm}
\newenvironment{ass}{\begin{assumption}\rm}{\end{assumption}}
\newcommand{\asslabel}[1]{\label{ass:#1}}
\newcommand{\assref}[1]{Assumption~\ref{ass:#1}}

\newcommand{\proclabel}[1]{\label{alg:#1}}
\newcommand{\procref}[1]{Procedure~\ref{alg:#1}}

\newtheorem{rem}{Remark}
\newtheorem{op}{Open Problem}

\newcommand{\etal}{\emph{et al}}

\newcommand{\voronoi}{Vorono\u\i}
\newcommand{\ceil}[1]{\left\lceil #1 \right\rceil}
\newcommand{\floor}[1]{\left\lfloor #1 \right\rfloor}



\title{\MakeUppercase{Biased Range Trees}}
\author{Vida Dujmovi\'c
	\and John Howat
	\and Pat Morin}

\begin{document}
\maketitle
\begin{abstract}
A data structure, called a \emph{biased range tree}, 
is presented that preprocesses a set $S$ of $n$ points in $\R^2$
and a query distribution $D$ for 2-sided orthogonal range counting
queries.  The expected query time for this data structure under the
distribution $D$ is competitive with the optimal decision tree for $S$
and $D$.   The memory and preprocessing requirements of the data
structure are $O(n\log n)$.
\end{abstract}

\section{Introduction}

Let $S$ be a set of $n$ points in $\R^2$ and let $D$ be a probability
measure over $\R^2$.  A \emph{2-sided range counting query} over $S$
asks, for a query point $q=(q_1,q_2)$, the number of points in
$(p_1,p_2)\in S$ such $p_1 \ge q_1$ and $p_2 \ge q_2$.  A 2-sided
range counting query \emph{has distribution $D$} if the query point
$q$ is chosen from the probability measure $D$.  If $T$ is a data
structure for answering 2-sided range queries over $S$ then we denote
by $\mu_D(T)$ the expected time, using $T$, to answering a range query
with distribution $D$.  The current paper is concerned with
preprocessing a point set $S$ and a distribution $D$ to build a data
structure $T$ that minimizes $\mu_D(T)$.

\subsection{Previous Work}

\begin{enumerate}

\item The topic of geometric range queries is a huge field
\cite{agarwal_erickson}.

\item Range trees \cite{bentley} use $O(n\log n)$ space and
preprocessing and answer range queries in $O(\log^2 n)$
worst-case time.  When combined with fractional cascading, the query
time can be reduced to $O(\log n)$.

\item $k$-d trees use $O(n)$ space and answer range queries in
$O(\sqrt{n})$ time.

\item Get the two SICOMP papers by Chazelle.
\end{enumerate}

\subsection{New Results}

In the current paper we present a data structure of size $O(n\log n)$,
that can be constructed in $O(n\log n)$ time, and that answers range
counting (or semigroup) queries in $O(\mu_D(T))$ expected time, where
$T$ is any comparison tree that answers range counting queries over $S$.
In particular, $T$ could be the comparison tree that minimizes
$\mu_D(T)$ implying that the expected query time of our data structure
is as fast as the fastest comparison tree.

Note that we do not place any restrictions on the comparison tree $T$.
Thus, our data structure, while requiring only $O(n\log n)$ space, is
competitive with any data structure that can be represented as a
comparison tree, without any limits of the space required by that data
structure.

\section{Preliminaries}

\begin{itemize}
\item Define a rectangle (possibly unbounded)
\item Definition of decision tree and comparison tree
\item Shannon's entropy lower bound
\item Lemma that says that entropy decreases when the support is
partitioned
\item Lower bound in terms of the entropy of the ray shooting
arrangement
\end{itemize}

\section{Biased Range Trees}

In this section we present our data structure, which, like range trees
is a two level structure consisting of a primary tree whose nodes
store secondary structures.

\subsection{The Primary Tree}

The primary tree is a 2-dimensional $k$-d tree \cite{X} $T$.  Each
node $v$ of $T$ is associated with an open rectangle $r(v)$.  The
rectangle associated with the root of $T$ is all of $\R^2$. The two
children of a node $v$ are associated with the two rectangles obtained
by splitting $r(v)$ with a horizontal or vertical line depending on
whether the distance from $v$ to the root of $T$ is even or odd,
respectively. 

The line used to split the rectangle $r(v)$ into two rectangles $r_1$
and $r_2$ is selected so that $\Pr\{r_1\}\le \Pr\{r(v)\}/2$ and
$\Pr\{r_2\}\le \Pr\{r(v)\}/2$.

  The root of
the tree stores a vertical line that paritions $\R^2$ into two open
halfspaces $h_1$ and $h_2$ such that $\Pr\{h_1\}\le 1/2$ and
$\Pr\{h2\} \le 1/2$


\subsection{The Catalogues}

\section{Optimality of Biased Range Trees}

\subsection{The Catalogue Location Step}


\subsection{The Tree Searching Step}

\section{Summary and Conclusions}

\end{document}
