\documentclass[lotsofwhite]{patmorin}
\usepackage{amsmath,graphicx}

\input{pat}

\title{Algorithms for Price/Quality Tradeoffs}
\author{Joachim Gudmundsson%
	\and Pat Morin%
	\and Michiel Smid}

\newcommand{\cost}{\operatorname{cost}}
\newcommand{\ppu}{\operatorname{ppu}}
\newcommand{\val}{\operatorname{profit}}

\begin{document}
\maketitle

\begin{abstract}
The abstract should go here.
\end{abstract}

\section{Introduction}

This paper considers a marketing problem in which a product
$P=(q_1,\ldots,q_d,p)$ is defined by a number of real-valued
\emph{qualities} $q_1,\ldots,q_d$ and its \emph{price} $p$.  The
\emph{market} for a product is a collection of customers
$C=\{C_1,\ldots,C_n\}$, where $C_i=(q_{i,1},\ldots,q_{i,d},p_i)$.  A
customer will purchase the least expensive product that meets all
their minimum quality requirements and whose price is below their
maximum price range.  That is, the customer $C_i$ will \emph{consider}
the product $P$ if $q_j \ge q_{i,j}$ for all $j\in\{1,\ldots,d\}$ and
$p \le p_i$.  The customer $C_i$ will \emph{purchase} the product if
it has minimum price among all products $C_i$ considers.

We consider markets that are \emph{saturated}.  That is, for every
customer there is an existing product that satisfies their
requirements. From the point of view of a manufacturer introducing one
or more new products, this means that all customers are \emph{pareto
optimal}, i.e., there are no two customers $C_i$ and $C_j$ such that
$q_{i,k} > q_{j,k}$ for all $k\in\{1,\ldots,d\}$ and $p_i < p_j$.  In
a saturated market, there is no need to consider existing products, as
these can be encoded into the customers' prices.

As an example of such a market, consider a market for computers in
which an example customer may be looking for a computer with a minimum
of 8 GB of RAM, a PC Benchmark(?) score of at least 3000, and 3D
Mark video score of at least 2000, and is willing to pay at most
\$1500.  In addition, there is already a computer on the market which
meets these requirements and retails for \$1200.  Thus, this customer
would be described the vector $(8,3000,2000,1200)$.  If a manufacturer
introduces a new product $(8,3500,2000,1199)$ then this customer would
buy the product.

With appropriate scaling, one can assume that the cost, $\cost(P)$, of
manufacturing $P$ is equal to the sum of its qualities
\[
   \cost(P) = \sum_{i=1}^d q_i \enspace .
\]
The \emph{profit per unit sold} of $P$ is therefore
\[
   \ppu(P) = p-\cost(P) \enspace .
\]
In this paper we consider algorithms that a manufacturer can use when
choosing new products to introduce into an existing saturated market.
The goal of a manufacturer is to introduce 1 or more new products into
the market in order to maximize their profit.  More precisely, given
an integer $k$ and customers $C_1,\ldots,C_n$, the goal is to find $k$
products $P_1,\ldots,P_k$ such that 
\[
  \val(P_1,\ldots,P_k) = \sum_{j=1}^k \ppu(P_j)
    \times |\{i:\mbox{$C_i$ purchases $P_j$}\}|
\]
is maximized.  Note that, when computing the quantity $|\{i:\mbox{$C_i$
purchases $P_j$}\}|$ one must consider all products $P_1,\ldots,P_k$
since a product $P_\ell$, $\ell\neq j$, may be preferable to a
customer who would otherwise buy $P_j$ if $P_\ell$ were unavailable.

\section{An $O(n\log n)$ algorithm for the case $k=d=1$}

In this section, we consider the simplest case, when the manufacturer
wishes to introduce one new product in which the quality of a product
has only one dimension.  Examples of such markets include, for
example, suppliers to the construction industry in which items (steel
I-beams, finished lumber, logs) must have a certain minimum length to
be used for a particular application.  An overly long piece can be cut
down to size, but gluing two short pieces together is not an option.


Throughout this section, since $d=1$ and $k=1$, we will use the
shorthand $q_i$ for $q_{i,1}$ and $P=(q,p)$ for $P_1$.
Our algorithm is an implementation of the \emph{plane-sweep} paradigm
\cite{bsXX}.  The correctness of the algorithm relies on the following
critical lemma, which is illustrated in \figref{lemma-monotone}:
\begin{figure}
  \begin{center}
    \includegraphics{lemma-monotone}
  \end{center}
  \caption{$\val(q,p) \le \val(q',p)$ implies that $\val(q,p') \le
           \val(q',p')$ for all $p' \le p$.}
  \figlabel{lemma-monotone}
\end{figure}

\begin{lem}\lemlabel{monotone}
Let $q' \le q$ and let $p$ be such that $0 < \val(q,p) \le \val(q',p)$.
Then, for any $p' \le p$, $\val(q',p) \le \val(q',p')$
\end{lem}

\begin{proof}
By definition, $\val(q,p) = a(p-q)$ and $\val(q',p) = a'(p-q')$, where
$a$ and $a'$ are the number of customers who would purchase $(q,p)$
and $(q',p)$, respectively.

These customers are all taken from the set $\{C_i: p_i \ge p\}$.  Now,
consider the customers in the set $C'=\{C_i: p' \le p_i < p\}$.  By
the assumption that customers are pareto optimal, any customer $C_i$
in $C'$ has $q_i \le q'$, so they will purchase either $(p',q')$ or
$(p',q)$ if it is offered.  Therefore, 
\[
  \begin{aligned}
    \val(q',p')
      &  =   (a'+|C'|)(p'-q') \\
      &  =   a'(p'-q') + |C'|(p'-q') \\
      & \ge  a'(p'-q') + |C'|(p'-q) 
        && \mbox{since $q > q'$} \\
      &  =   a'(p-q') + a'(p'-p) + |C'|(p'-q) \\
      & \ge  a'(p-q') + a(p'-p) + |C'|(p'-q) 
        && \mbox{since $a \ge a'$ and $(p'-p) < 0$} \\
      & \ge  a(p-q) + a(p'-p) + |C'|(p'-q) 
        && \mbox{by assumption} \\
      &  =  a(p'-q) + |C'|(p'-q) \\
      &  =  \val(q,p') \enspace , \\
  \end{aligned}
\]
as required.
\end{proof}

\Lemref{monotone} allows us to apply the plane sweep paradigm by
sweeping by decreasing price.  It tells us that a product $(q',p)$
that gives better profit than the higher-quality product $(q,p)$ at
the current price $p$, then it will always give a better profit for
the remainder of the sweep.  In particular, there will never be a
reason to offer a product with quality $q$ for the remainder of the
algorithm's execution.

At any point in the algorithm, there is a current price $p$, which
starts at $p=\infty$ and decreases throughout the execution of the
algorithm.  At all times, the algorithm maintains a list of qualities
$q_1^* > q_2^* > \cdots > q_m^*$ such that $\val(q_1^*,p) >
\val(q_2^*,p) >\cdots>\val(q_m^*,p)$.


\end{document}
