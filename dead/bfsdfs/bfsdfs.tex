\documentclass[lotsofwhite]{patmorin}
\usepackage{charter}
\usepackage{graphicx,ipe}
\usepackage[noend]{algorithmic}

\newcommand{\email}[1]{\texttt{#1}}
\newcommand{\dfst}{\textsc{DFS-Traversal}}
\newcommand{\bfst}{\textsc{BFS-Traversal}}
\newcommand{\layer}{d}
\newcommand{\fakefigure}{\begin{center}\rule{2in}{2in}\end{center}}
 
%\usepackage{amsthm}

\newcommand{\centeripe}[1]{\begin{center}\Ipe{#1}\end{center}}
\newcommand{\comment}[1]{}

\newcommand{\centerpsfig}[1]{\centerline{\psfig{#1}}}

\newcommand{\seclabel}[1]{\label{sec:#1}}
\newcommand{\Secref}[1]{Section~\ref{sec:#1}}
\newcommand{\secref}[1]{\mbox{Section~\ref{sec:#1}}}

\newcommand{\alglabel}[1]{\label{alg:#1}}
\newcommand{\Algref}[1]{Algorithm~\ref{alg:#1}}
\newcommand{\algref}[1]{\mbox{Algorithm~\ref{alg:#1}}}

\newcommand{\applabel}[1]{\label{app:#1}}
\newcommand{\Appref}[1]{Appendix~\ref{app:#1}}
\newcommand{\appref}[1]{\mbox{Appendix~\ref{app:#1}}}

\newcommand{\tablabel}[1]{\label{tab:#1}}
\newcommand{\Tabref}[1]{Table~\ref{tab:#1}}
\newcommand{\tabref}[1]{Table~\ref{tab:#1}}

\newcommand{\figlabel}[1]{\label{fig:#1}}
\newcommand{\Figref}[1]{Figure~\ref{fig:#1}}
\newcommand{\figref}[1]{\mbox{Figure~\ref{fig:#1}}}

\newcommand{\eqlabel}[1]{\label{eq:#1}}
\newcommand{\eqref}[1]{(\ref{eq:#1})}

\newtheorem{thm}{Theorem}{\bfseries}{\itshape}
\newcommand{\thmlabel}[1]{\label{thm:#1}}
\newcommand{\thmref}[1]{Theorem~\ref{thm:#1}}

\newtheorem{lem}{Lemma}{\bfseries}{\itshape}
\newcommand{\lemlabel}[1]{\label{lem:#1}}
\newcommand{\lemref}[1]{Lemma~\ref{lem:#1}}

\newtheorem{cor}{Corollary}{\bfseries}{\itshape}
\newcommand{\corlabel}[1]{\label{cor:#1}}
\newcommand{\corref}[1]{Corollary~\ref{cor:#1}}

\newtheorem{obs}{Observation}{\bfseries}{\itshape}
\newcommand{\obslabel}[1]{\label{obs:#1}}
\newcommand{\obsref}[1]{Observation~\ref{obs:#1}}

\newtheorem{assumption}{Assumption}{\bfseries}{\rm}
\newenvironment{ass}{\begin{assumption}\rm}{\end{assumption}}
\newcommand{\asslabel}[1]{\label{ass:#1}}
\newcommand{\assref}[1]{Assumption~\ref{ass:#1}}

\newcommand{\proclabel}[1]{\label{alg:#1}}
\newcommand{\procref}[1]{Procedure~\ref{alg:#1}}

\newtheorem{rem}{Remark}
\newtheorem{op}{Open Problem}

\newcommand{\etal}{\emph{et al}}

\newcommand{\voronoi}{Vorono\u\i}
\newcommand{\ceil}[1]{\left\lceil #1 \right\rceil}
\newcommand{\floor}[1]{\left\lfloor #1 \right\rfloor}



\title{\MakeUppercase{Recognizing Depth-First-Search and
	Breadth-First-Search Trees}}

\author{Jurek Czyzowicz\thanks{%
		Universit\'e du Qu\'ebec en Outaouais,
		\email{jurek@uqo.ca}} \and
	Wojciech Fraczak\thanks{%
		IDT Canada,
		\email{wojtek.fraczak@idt.com}} \and
	Pat Morin\thanks{%
		Carleton University,
		\email{morin@cs.carleton.ca}} \and
	Andrzej Pelc\thanks{%
		Universit\'e du Qu\'ebec en Outaouais,
		\email{pelc@uqo.ca}}}
\date{}

\begin{document}
\maketitle

\begin{abstract} 
Given a graph $G$ and a spanning tree $T$ of $G$, is $G$ a
depth-first-search or breadth-first-search tree of $G$?  We show that
this question can be answered in linear time.
\end{abstract}

%%%%%%%%%%%%%%%%%%%%%%%%%%%%%%%%%%%%%%%%%%%%%%%%%%%%%%%%%%%%%%%%%%%%%%%
\section{Introduction}

Two of the most important paradigms in the design of efficient graph
algorithms are \emph{depth-first-search} (DFS) and
\emph{breadth-first-search} (BFS).  Both of these graph traversal
algorithms have been in use for nearly thirty years and have many
applications including planarity testing \cite{X}, graph drawing
\cite{X}, separator finding (divide-and-conquer) \cite{X}, VLSI layout
\cite{X}, \ldots Both techniques have been studied extensively and
their complexity has been considered in sequential \cite{X}, parallel
\cite{X}, distributed \cite{X}, and external-memory \cite{X} computing
environments.

Let $G$ be a (directed or undirected) graph and denote by $V(G)$ the
vertex set of $G$ and by $E(G)$ the edge set of $G$.  We say that a
node $v$ is the \emph{neighbour} of a node $u$ if $E(G)$ contains the
edge $(u,v)$ (for directed graphs) or the edge $\{u,v\}$ (for
undirected graphs).  The DFS algorithm uses a stack and a mark bit at
each node of $G$ to keep track of which vertices have already been
visited.  The following algorithm, when called on input $r\in V(G)$
with all nodes of $G$ unmarked will build a \emph{DFS tree} $T$ of $G$
rooted at $r$:

\noindent
\begin{minipage}{\textwidth}
$\dfst(v)$
\begin{algorithmic}[1]
\STATE{mark $r$ and push it onto the stack}
\WHILE{the stack is not empty}
  \IF{vertex $u$ on the top of the stack has an unmarked neighbour $v$}
    \STATE{add edge $(u,v)$ to $T$}
    \STATE{mark $v$ and and push it onto the stack}
  \ELSE
    \STATE{pop $u$ from the stack}
  \ENDIF
\ENDWHILE
\end{algorithmic}
\end{minipage}

The BFS algorithm uses a queue and mark bits on the vertices of $G$ to
traverse $G$.  The following algorithm, when called on input $r$ with
all nodes of $G$ unmarked will build a \emph{BFS tree} $T$ of $G$
rooted at $r$:

\noindent
\begin{minipage}{\textwidth}
$\bfst(v)$
\begin{algorithmic}[1]
\STATE{mark $r$ and enqueue it}
\WHILE{the queue is not empty}
  \STATE{dequeue the next vertex $u$}
  \WHILE{$u$ has an unmarked neighbour $v$}
     \STATE{add edge $(u,v)$ to $T$}
     \STATE{mark $v$ and enqueue it}
  \ENDWHILE
\ENDWHILE
\end{algorithmic}
\end{minipage}

Note that both these algorithms are non-deterministic in the sense
that the tree $T$ depends on the order in which the neighbours of $u$
are processed in lines 3 and 4, respectively.  Thus, $G$ may have many
different DFS or BFS trees rooted at $r$.  A question that therefore
occurs naturally is: Given a graph $G$ and a tree $T$, is $T$ is a DFS
(respectively BFS) tree of $G$ rooted at $r$?  For an undirected tree
$T$, we may also ask: What is the set of vertices $r\in V(G)$ such
that $T$ is a DFS (respectively BFS) tree of $G$ rooted at $r$?

Although there are many algorithms for recognizing specific types of
graphs (e.g., chordal graphs \cite{X}, bounded tree-width or
path-width graphs \cite{X}, interval graphs \cite{X}).  There are also
several algorithms for finding specific subgraphs \cite{X} or minors
\cite{X} withing a graph.  However, we are unaware of any result on
the problem of testing whether a spanning tree $T$ of a graph $G$ is a
DFS or BFS tree of $G$.  The most similar such result is on testing
whether a spanning tree of $G$ is a minimum spanning tree of $G$
\cite{X}.

In this paper, we show that all these questions can be answered in
optimal $O(|V(G)|+|E(G)|)$ time.  The remainder of this paper is
organized as follows: In \secref{dfs} we desribe algorithms for
recognizing DFS trees.  In \secref{bfs} we present algorithms for
recognizing BFS trees.  In \secref{conclusions} we summarize and
conclude with open problems.

%%%%%%%%%%%%%%%%%%%%%%%%%%%%%%%%%%%%%%%%%%%%%%%%%%%%%%%%%%%%%%%%%%%%%%%
\section{Recognizing DFS Trees}\seclabel{dfs}

In this section, we present algorithms for recognizing DFS trees.  Our
exposition proceeds is as follows: First we give an algorithm that,
given an undirected graph $G$, a spanning tree $T$ of $G$ and a vertex
$r\in V(G)$ determines whether $T$ is a DFS tree of $G$ rooted at $r$.
Next we show how to extend this algorithm to the case where $G$ is
directed.  Finally, we describe an algorithm that, given an undirected
graph $G$ and a spanning tree $T$ of $G$ determines all vertices $r\in
V(G)$ such that $T$ is a DFS tree of $G$ rooted at $r$.

%======================================================================
\subsection{DFS: Rooted Spanning Trees of Undirected Graphs}

In this section, $G$ is a connected undirected graph, $T$ is a
spanning tree of $G$ and some distinguished node $r\in V(G)$ is called
the \emph{root} of $T$.  We denote this rooted tree by $T_r$.  We say
that an edge that is in $G$ but not in $T$ is a \emph{non-tree} edge
of $G$.  We say that a node $u$ is an \emph{ancestor} of a node $v$ in
$T_r$ if $u$ is on the path from $r$ to $v$ in $T$.  In this case we
also say that $v$ is a \emph{descendant} of $u$ in $T_r$. We begin
with the following easy lemma about the \dfst\ algorithm:

\begin{lem}\lemlabel{ancestor}
During a run of the \dfst\ algorithm that produces the tree $T$, $u$
is an ancestor of $v$ in $T_r$ if and only if $v$ is marked while $u$
is on the stack.
\end{lem}

\begin{proof}
The stack lists the vertices on the path from $r$ to the node
currently on the top of the stack.  Therefore, $u$ is an ancestor of
$v$ in $T_r$ if and only if $u$ is on the stack when $v$ is marked and
pushed onto the stack.
\end{proof}

The following characterization of DFS trees in undirected graphs forms
the basis of our algorithm.

\begin{lem}\lemlabel{back-edges}
Let $G$ be a connected undirected graph and let $T$ be a spanning tree
of $G$.  Then $T$ is a DFS tree of $G$ rooted at $r$ if and only if,
for each non-tree edge $\{u,v\}$, either $u$ is an ancestor or a
descendant of $v$ in $T_r$.
\end{lem}


\begin{proof} 

Suppose $T$ is a DFS tree of $G$ rooted at $r$.  Consider any non-tree
edge $\{u,v\}\in E(G)$ and note that in any execution of the \dfst\
algorithm, one of $u$ or $v$ must be pushed on the stack first and
this node does not leave the stack until all of its neighbours are
marked.  But then, by \lemref{ancestor}, this (first) node is an
ancestor of the other (second) node.

Suppose $T$ is such that, for every non-tree edge $\{u,v\}$ in $G$,
$u$ is an ancestor or descendant of $v$.  Imagine running the \dfst\
algorithm on the input graph $T$.  In this case, the algorithm will
certainly produce the DFS tree $T$.  Note that, while running the
algorithm on $T$, a node $u$ is not popped until all its ancestors and
descendants in $T$ have been marked.  Therefore, if the only non-tree
edges of $u$ in $G$ are all ancestors or descendants of $u$ in $T$,
then all of $u$'s neighbours in $G$ have also been marked so this
execution of the \dfst\ algorithm is also a valid execution of the
\dfst\ algorithm on $G$ that generates the tree $T$.  Therefore, $T$
is a DFS tree of $G$.
\end{proof}

Given a tree $T$, to test if it satisfies the conditions of the lemma,
we simply run the \dfst\ algorithm on $T$ and check that, whenever a
node $u$ is popped from the stack, all of $u$'s neighbours in $G$ are
already marked.  This algorithm is easily implemented to run in
$O(|V(G)|+|E(G)|)$ time and its correctness follows from the proof of
\lemref{back-edges}.

\begin{thm} 
Let $G$ be an undirected graph, $T$ be a spanning tree of $G$ and $r$
be vertex in $V(G)$.  There exists an $O(|V(G)|+|E(G)|)$ time
algorithm to test if $T$ is a DFS tree of $G$ rooted at $r$.
\end{thm}

%======================================================================
\subsection{DFS: Rooted Spanning Trees of Directed Graphs}

Next we consider the case in which $G$ is a directed graph. In this
section, we will talk about $T$ as if it is undirected so that, for
any two vertices $u,v\in V(G)$, there is a unique path in $T$ that joins
them, although in actuality, the tree $T$ is a directed tree with all
edges directed away from the root $r$.  The directed case is
considerably more complicated than the undirected case, since now
non-tree edges need not lead to ancestors or descendants.  The
following lemma captures exactly what types of non-tree edges are
allowed.

\begin{lem}\lemlabel{ancestor2} 
Let $G$ be a directed graph, let $T$ be a DFS tree of $G$ rooted
at $r$, and let $(u,v)$ be a non-tree edge of $G$.  Then, in any run
of the \dfst\ algorithm on $G$ that produces the DFS tree $T$, the
node $v$ is marked before $u$ is popped from the stack.  
\end{lem}

\begin{proof}
By definition of the \dfst\ algorithm, $u$ is not popped from the
stack until all its neighbours are marked.  Since $v$ is a neighbour
of $u$, $v$ must be marked before $u$ is popped from the stack.
\end{proof}

\begin{lem}\lemlabel{back-edges2}
The tree $T$ is a DFS tree of $G$ if and only if there is an execution
of the \dfst\ algorithm on $T$ such that the condition of
\lemref{ancestor2} holds for every non-tree edge $(u,v)\in E(G)$.  
\end{lem}

\begin{proof}
The proof is similar to the proof of \lemref{back-edges}.
\end{proof}

For two vertices $u,v\in V(G)$, the \emph{lowest common ancestor
(LCA)} of $u$ and $v$ in $T_r$ is the vertex on the path from $u$ to
$v$ in $T$ that is closest to $r$ in $T$ (see \figref{lca}).  Note
that if neither $u$ nor $v$ is a descendant of the other, then
$a\notin\{u,v\}$.  In this case, let $u'$ (respectively, $v'$) denote
the node that immediately precedes $a$ (respectively, follows $a$) on
the path from $u$ to $v$ in $T$.  Bender and Farach-Colton \cite{X}
give an algorithm for preprocessing a rooted tree $T_r$ in linear time
so that, for any pair of vertices $u$ and $v$, one can determine the
LCA $a$ of $u$ and $v$ in $T_r$ in constant time.  Furthermore, in the
case where $a\notin\{u,v\}$ their algorithm can also report the
vertices $u'$ and $v'$ in constant time.

\begin{figure}
\centeripe{lca}
\caption{The lowest common ancestor $a$ of $u$ and $v$ in
  $T_r$.}\figlabel{lca}
\end{figure}

Note that when $u$ is an ancestor of $v$ (or vice versa) then the
condition of \lemref{ancestor2} is met by any depth first traversal of
$T$ since $u$ is not popped from the stack until all its ancestor and
descendants have been marked.  However, when neither $u$ nor $v$ is an
ancestor of the other, the edge $(u,v)$ implies a constraint on the
order in which the \dfst\ algorithm can visit the descendants of $a$.
In particular, the traversal meets the conditions of
\lemref{ancestor2} if and only if it pushes $v'$ onto the stack before
pushing $u'$ (otherwise, $u$ will be popped before $v$ is marked).
Therefore, by \lemref{back-edges2}, $T$ is a DFS tree of $G$ rooted at
$r$ if and only if the \dfst\ algorithm can meet all such constraints.

By using the LCA data structure to the compute the LCA of $u$ and $v$
for each non-tree edge $(u,v)$ we can compute, for each vertex $a\in
V(G)$, the constraints on the order in which the \dfst\ algorithm must
push the children of $a$.  This gives us a directed graph on the
children of $a$, and all the constraints can be satisfied if and only
if this graph is acyclic (i.e., the constraints form a partial order).
Computing all of these directed graphs is easily done in $O(|E(G)|+|V(G)|)$
time and testing whether they are acyclic can be done in the same
time-bound using topological sort (see, e.g., \cite{X}).


\begin{thm} 
Let $G$ be a directed graph, $T$ be a spanning tree of $G$
and $r$ be vertex in $V(G)$.  There exists an $O(|V(G)|+|E(G)|)$ time algorithm
to test if $T$ is a DFS tree of $G$ rooted at $r$.
\end{thm}

%======================================================================
\subsection{DFS: Unrooted Spanning Trees of Undirected
Graphs}\seclabel{edge-marker}

Next we consider the case where we are given only an undirected graph
$G$ and a spanning tree $T$ of $G$ and our goal is to determine all
vertices $r$ such that $T$ is a spanning tree of $G$ rooted at $r$.
Note that, by \lemref{back-edges}, this is the problem of finding all
vertices $r$ such that all non-tree edges of $G$ are between descendants
and their ancestors in $T_r$.

Refer to \figref{dfs-locations}.  Consider a particular non-tree edge
$\{u,v\}$.  Let $e_u$ (respectively, $e_v$) denote the first
(respectively, last) edge on the path from $u$ to $v$ in $T$.  If we
remove the edges $e_u$ and $e_v$ from $T$ then the resulting forest
has 3 connected components, one that contains $u$, one that contains
$v$ and one that contains neither $u$ nor $v$.  Observe that $v$ is a
descendant of $u$ in $T_r$ if and only if $r$ is in the first of these
three components and that $u$ is a descendant of $v$ in $T_r$ if and
only if $r$ is in the second of these components.  Therefore, by
\lemref{back-edges} if $T$ is a DFS tree of $G$ rooted at $r$, then
the root $r$ must be in the first two of these components.  In this
way, every non-tree edge $\{u,v\}$ of $G$ generates a constraint on
the location of $r$.  If we can find a vertex $r$ that satisfies all
such constraints simultaneously then $T$ is a DFS tree of $G$ rooted
at $r$.

\begin{figure}
\centeripe{dfs-locations}
\caption{The root of $T$ must be in a component of
$T\setminus\{e_u,e_v\}$ that contains $u$ or
$v$.}\figlabel{dfs-locations}
\end{figure}

Note that we can represent the constraint induced by edge $\{u,v\}$ by
placing symbolic ``markers'' on the edges $e_u$ and $e_v$ indicating
on which side of these edges the root $r$ should be placed in order to
satisfy the constraint.  We can compute all these markers using a
depth first traversal of $T$ starting arbitrary vertex $r'$.  In this
traversal, the stack is represented as an array of length $n$ and when
we push a vertex $v$ onto the stack, we keep track of the index $i_v$
of $v$ in the stack as well as a pointer to the edge of $G$ that
brought us to $v$.  When the traversal is about to pop a vertex $v$
from the stack, we examine the neighbours of $v$ in $G$ (see
\figref{edge-mark}).

\begin{figure}
\begin{center}\begin{tabular}{c@{\hspace{2cm}}c}
\Ipe{edge-mark-a} & \Ipe{edge-mark-b} \\
(1) & (2) 
\end{tabular}\end{center}
\caption{Using DFS to mark place markers on the edges of $T$. Bold
edges indicate the path currently on the DFS stack.}
\figlabel{edge-mark}
\end{figure}

\begin{enumerate} 

\item If $u$ is a neighbour of $v$ and $u$ is currently on the stack
then the stack contains the path from $u$ to $v$ and we can find the
edges $e_u$ and $e_v$ by looking at the entries for $i_u+1$ and $i_v$
on the stack.  In this case we place markers on the edges $e_u$ and
$e_v$ and mark the edge $\{u,v\}$ as already having been processed.

\item If $u$ is a neighbour of $v$, $u$ is not currently on the stack,
and the edge $\{u,v\}$ has not already been processed then the edge
$e_v$ is given by $i_v$.  In this case we place a marker on $e_v$ but
do not mark the edge $\{u,v\}$ as already having been processed (since
we might still need to place a marker on the edge $e_u$).  

\end{enumerate}

Correctness of the above algorithm follows from the fact that, for
each edge $\{u,v\}$, the markers on $e_u$ and $e_v$ are correctly
computed by either Rule~1 (if one of $u$ and $v$ is a descendant of
the other in $T_{r'}$) or by Rule~2 (otherwise).  Using the indices
into the stack and the pointers to edges the algorithm is easily
implemented to run in $O(|V(G)|+|E(G)|)$ time.

To determine which vertices satisfy all constraints simultaneously, we
again pick an arbitrary vertex $r'$ and walk around $T$ to count how
many constraints are satisfied by $r'$.  We then do a DFS traversal of
$T$ starting at $r'$. When the traversal moves from vertex $u$ to
vertex $v$, we examine the markers on edge $\{u,v\}$ to update the
number of satisfied constraints. In this way we can compute the number
of constraints satisfied by each vertex in $V(G)$ using one DFS search
and examining each edge marker only twice. Since the number of
constraints (non-tree edges) is exactly $|E(G)|-|V(G)|+1$, this tells
us exactly which vertices of $G$ satisfy all constraints
simultaneously.  This algorithm is easily implemented to run in
$O(|E(G)|+|V(G)|)$ time so we have just proven:

\begin{thm} 
Let $G$ be an undirected graph $G$ and let $T$ be a spanning
tree of $G$.  There exists an $O(|V(G)|+|E(G)|)$ time algorithm to find the
set of all vertices $r\in V(G)$ such that $T$ is a DFS tree of $G$
rooted at $r$.  
\end{thm}


%%%%%%%%%%%%%%%%%%%%%%%%%%%%%%%%%%%%%%%%%%%%%%%%%%%%%%%%%%%%%%%%%%%%%%%
\section{Recognizing BFS Trees}\seclabel{bfs}

Next we consider the problem of recognizing BFS trees of graphs.  The
order of exposition is slightly different than what was used for DFS
trees.  In particular, we do not consider the special case of rooted
trees in undirected graphs since the algorithm for this case does not
seem to be much simpler than the algorithm for the (more general) case
of rooted trees in directed graphs.

%======================================================================
\subsection{BFS: Rooted Spanning Trees of Directed Graphs}

First we consider the problem of testing, given a directed graph $G$
and a rooted spanning tree $T_r$ of $G$, whether $T$ is a BFS tree of
$G$ rooted at $r$.  In studying problems on BFS trees, we naturally
think of the vertices of $T_r$ as being partitioned into \emph{layers}
where the vertices of one layer all have the same distance to the root
$r$.  Let $\layer_r(v)$ denote the distance from $v$ to $r$ in $T$.  We
begin with the following easy lemma.

\begin{lem}\lemlabel{layer} 
If $T$ is a BFS tree of $G$ rooted at $r$ then, for each non-tree edge
$(u,v)$ in $G$, $\layer_r(v)\le\layer_r(u)+1$.
\end{lem}

\begin{proof} 
This follows immediately from the well-known fact (easily proven by
induction) that when a node at layer $i$ is dequeued, the only marked
nodes are those at layers $0,\ldots,i+1$.
\end{proof}

Since testing the condition of \lemref{layer} is easily done in
$O(V(G)+E(G))$ time, we assume from this point onwards that this
condition has been checked and is true. However, this condition is not
sufficient. The following lemma shows that edges $(u,v)$ for which
$\layer_r(v)=\layer_r(u)+1$ impose additional constraints on the way
the \bfst\ algorithm must traverse $T$ (see \figref{forward}).

\begin{figure}
\centeripe{forward}
\caption{Illustration of \lemref{forward}.}\figlabel{forward}
\end{figure}


\begin{lem}\lemlabel{forward}
Let $G$ be a directed graph, let $T$ be a BFS tree of $G$ rooted at
$r$, let $(u,v)$ be a non-tree edge such that
$\layer_r(v)=\layer_r(u)+1$, and let $a$ be the LCA of $u$ and $v$ in
$T_r$.  Let $u'$ be the second node on the path in $T$ from $a$ to $u$
and let $v'$ be the second node on the path in $T$ from $a$ to $v$.
Then, in any run of the \bfst\ algorithm on $G$ that produces $T$, the
node $v'$ is enqueued before the node $u'$.
\end{lem}

\begin{proof}
Suppose that this were not the case and that $u'$ was enqueued before
$v'$.  Then $u$ would be dequeued before the parent of $v$ in $T$, so
when processing $u$, $v$ would be unmarked and would become a child of
$u$.  But this contradicts the assumption that $(u,v)$ is a non-tree
edge.
\end{proof}


\begin{lem}\lemlabel{bfs-condition}
The tree $T$ is a BFS tree of $G$ if and only if there is an execution
of the \bfst\ algorithm on $T$ such that the condition of
\lemref{forward} holds for every non-tree edge $(u,v)\in E(G)$.  
\end{lem}

\begin{proof}
Similar to the proof of \lemref{back-edges}.
\end{proof}


\lemref{bfs-condition} leads to an algorithm that is very similar to
the algorithm for testing if $T$ is a DFS tree of $G$.  For each
non-tree edge $(u,v)$ such that $\layer_r(v)=\layer_r(u)+1$ we find
the LCA of $u$ and $v$ as well as the first vertices $u'$ and $v'$ on
the paths in $T$ from $a$ to $u$ and $v$, respectively.  This edge
establishes the constraint that, when visiting $a$, the vertex $v'$
must be enqueued before the vertex $u'$.  As before, we use
topological sorting to determine if all these constraints can be
satisfied simultaneously.

\begin{thm}
Let $G$ be a directed graph, $T$ be a spanning tree of $G$, and $r$ be
some vertex in $V(G)$. There is an $O(|V(G)|+|E(G)|)$ to test if $T$
is a BFS tree of $G$ rooted at $r$.
\end{thm}


%======================================================================
\subsection{BFS: Unrooted Spanning Trees of Undirected Graphs}

In this section we consider the case where $G$ is an undirected graph,
$T$ is a spanning tree of $G$ and we want to compute the set of all
$r\in V(G)$ such that $T$ is a BFS tree of $G$ rooted at $r$.  We
proceed by first finding a subtree $T'$ of $T$ that contains all
possible locations of the root $r$.  We then do a test on $T'$ that
tells us if all or none of these locations are valid roots for a BFS
tree.

There are two structural properties of BFS trees that produce
constraints on the possible locations of $r$.  The first property that
we require for $T$ to be a BFS tree rooted at $r$ comes from
\lemref{layer}.  In the language of undirected graphs, \lemref{layer}
says that, for every non-tree every edge $\{u,v\}$, we must choose a
root $r$ such that
\begin{equation}
	\left|\layer_r(u)-\layer_r(v)\right|\le 1 \enspace . \eqlabel{layers}
\end{equation}

We classify each non-tree edge $\{u,v\}$ of $G$ as being \emph{even}
or \emph{odd} depending on whether the length of the path from $u$ to
$v$ in $T$ is even or odd, respectively.  See \figref{even-odd} for
what follows.  If $\{u,v\}$ is an even edge, then in order to satisfy
\eqref{layers} we must choose a root $r$ so that the LCA of $u$ and
$v$ in $T_r$ is the median vertex on the path from $u$ to $v$ in $T$.
Similarly, if $\{u,v\}$ is an odd edge then we must choose a root $r$
so that the LCA of $u$ and $v$ in $T_r$ is one of the two median
vertices on the path from $u$ to $v$ in $T$.  Note that in both cases,
the set of possible locations for $r$ induces a connected subtree of
$T$.  It follows that if we take the intersection of all such sets
then they also induce a connected subtree $T'$ of $T$.

\begin{figure}
\begin{center}\begin{tabular}{c@{\hspace{2cm}}c}
\Ipe{median-even} & \Ipe{median-odd} \\
(a) & (b)
\end{tabular}\end{center}
\caption{The set of possible locations for the root $r$ when
considering the edge $\{u,v\}$ (a)~when $\{u,v\}$ is even and (b)~when
$\{u,v\}$ is odd.}\figlabel{even-odd}

\end{figure}

To compute the tree $T'$ we first need to find, for each non-tree edge
$\{u,v\}$ the median vertex or vertices on the path from $u$ to $v$ in
$T$.  To find these median vertices, we fix an arbitrary root $r$ of
$T$ and compute, for each edge $\{u,v\}$ the lowest common ancestor of
$u$ and $v$ in $T_r$. At the same time we also compute $\layer(v)$ for
every vertex $v$.  Next we do a DFS traversal of $T$ beginning at $r$.
Just before the traversal pops an element $u$ off the stack, we check
the non-tree edges of $G$ incident to $u$.  When examining an edge
$\{u,v\}$ such that the LCA of $u$ and $v$ in $T_r$ is $a$ we do the
following (see \figref{median-find}):

\begin{figure}
\begin{center}\begin{tabular}{c@{\hspace{2cm}}c@{\hspace{2cm}}c}
\Ipe{median-find-a} & \Ipe{median-find-b} & \Ipe{median-find-c} \\
(1) & (2) & (3)
\end{tabular}\end{center}
\caption{Finding the median element on the path from $u$ to $v$ in
$T$. Bold edges indicate the path of vertices currently on the DFS
stack. The box indicates the position of the median vertex or
vertices.}
\figlabel{median-find}
\end{figure}

\begin{enumerate}
\item If $\layer_r(u)=\layer_r(v)$ then the median vertex is $a$.

\item If $\layer_r(u)>\layer_r(v)$ then the median vertex or vertices
are stored on the stack and their location can be computed by
examining $\layer_r(u)$, $\layer_r(v)$ and $\layer_r(a)$.

\item Otherwise, the median vertex or vertices will be computed when
$v$ is popped from the stack.
\end{enumerate}

Once these median vertices are known for each edge we can apply a
simple variation on the algorithm of \secref{edge-marker} that uses
edge markers to compute the intersection of all these subtrees in
$O(|V(G)|+|E(G)|)$ time.  We leave the details to interested reader.

Next we consider the second step of our algorithm, which is testing to
see if the vertices of $T'$ are indeed valid roots of $T$.  Let
$\{x,y\}$ be an edge of $T'$.  We say that $\{x,y\}$ \emph{supports} a
non-tree edge $\{u,v\}$ if the path in $T$ from $u$ to $v$ uses the
edge $\{x,y\}$.  If $\{x,y\}$ supports $\{u,v\}$ then it follows that
$\{u,v\}$ is an odd edge and that $x$ and $y$ are on the path from $u$
to $v$ in $T$.  Note that, for different vertices $r\in V(T')$ the LCA
of $u$ and $v$ in $T_r$ may be either $x$ or $y$, and this is the
primary obstacle in findin an efficient algorithm.  The following
surprising result says that after computing the tree $T'$, we can
forget about the edge $\{u,v\}$.

\begin{lem}\lemlabel{shifty}
Let $r$ be a vertex in $T'$.  Then $T$ is a BFS tree of $G$ rooted at
$r$ if and only if $T$ is a BFS tree of $G'$ rooted at $r$, where
$V(G')=V$ and $E(G')=E(G)\setminus E'$ and $E'$ is the set of non-tree
edges of $G$ supported by edges in $T'$.
\end{lem}

\begin{proof}
If $T$ is a BFS tree of $G$ rooted at $r$ then $T$ is certainly a BFS
tree of $G'$ since the non-tree edges of $G'$ are a subset of the
non-tree edges of $G$.

Suppose then that $T$ is not a BFS tree of $G$ rooted at $r$.  Since
$r$ is in $T'$, all non-tree edges $\{u,v\}$ in $G$ satisfy
$|\layer_r(u)-\layer_r(v)|\le 1$.  Therefore, by
\lemref{bfs-condition} there is some vertex $a\in V(G)$ such that the
constraints imposed on the children of $a$ by \lemref{forward} can not
be satisfied.  The children of $a$ in $T_r$ can be partitioned (see
\figref{constrain-a}) into the set $S_1$ of children that are in $T'$
and the remaining children $S_2$ (that are in $T$ but not $T'$).
There are four types of constraints on the children of $a$ (see
\figref{no-support}):

\begin{figure}
\centeripe{constrain-a}
\caption{Partitioning the children of $a$ into the sets $S_1$ and
$S_2$.}\figlabel{constrain-a}
\end{figure}

\begin{figure}
\begin{center}\begin{tabular}{cc}
\Ipe{no-support-a} &
\Ipe{no-support-b} \\
(1) & (2) \\[1cm]
\Ipe{no-support-c} &
\Ipe{no-support-d} \\
(3) & (4)
\end{tabular}\end{center}
\caption{Four different types of constraints on the children of $a$.
Bold edges represent edges of $T'$.}
\figlabel{no-support}
\end{figure}

\begin{enumerate}
\item Constraints that force a node of $S_2$ to be enqueued before
another node of $S_2$,

\item constraints that force a node of $S_1$ to be enqueued before a
node of $S_2$,

\item constraints that force a node of $S_1$ to be enqueued before
another node of $S_1$, and

\item Constraints that force a node of $S_2$ to be enqueued before a
node of $S_1$.
\end{enumerate}

Since these constraints are not satisfiable they must form a cycle.
Notice that constraints of type 2, 3 and 4 are caused by non-tree
edges that are supported by an edge of $T'$ and constraints of type~1
are caused by non-tree edges that are not supported by an edge of
$T'$. 

We will show that constraints of types~3 and 4 do not exist.  It then
follows that the elements of $S_1$ can not take part in a cycle of
constraints.  Since constraints of type 2 always involve elements of
$S_1$, it follows that these constraints are not involved in a cycle
of constraints.  Therefore, removing these constraints (by removing
the non-tree edges that generated them) does not remove the cycle of
constraints in the children of $a$, so $T$ is also not a BFS tree of
$G'$ rooted at $r$, as required.

To see that there are no constraints of type~3, consider an edge
$\{u,v\}$ that would generate such a constraint.  Then, $T'$ contains
two edges $\{u',a\}$ and $\{a,v'\}$ on the path from $u$ to $v$ in
$T$.  But this can't happen since, by construction, at most one edge
on the path from $u$ to $v$ in $T$ can be in $T'$.

To see that there are no constraints of type~4, we again consider an
edge $\{u,v\}$ that would generate such a constraint.  Without loss of
generality, suppose $u'\in S_1$ is the second node on the path from
$a$ to $u$ in $T$.  Then $T'$ contains an edge $\{u',a\}$ on the path
from $u$ to $v$ in $T$.  But then this must be the median edge on the
path from $u$ to $v$ in $T$, in which case $\layer(u)=\layer(v)+1$ so
the constraint generated by such an edge is of type~2, not type~4.
\end{proof}

\lemref{shifty} is quite powerful.  For example, it has the following
all-or-nothing consequence.

\begin{lem}
If $T$ is a BFS tree of $G$ rooted at some vertex $r\in T'$ then $T$
is a BFS tree of $G$ rooted at \emph{any} vertex $r\in T'$.
\end{lem}

\begin{proof} 
By \lemref{bfs-condition}, whether or not $T$ is a BFS tree rooted at
$r\in T'$ depends only the following information about each non-tree
edge $\{u,v\}$: (1)~$\layer_r(u)$, (2)~$\layer_r(v)$ and (3)~LCA of
$u$ and $v$ in $T_r$.  Now observe that, unless $\{u,v\}$ is supported
by an edge of $T'$, all of this information is the same for any $r\in
T'$.  But, by \lemref{shifty}, the non-tree edges supported by edges
of $T'$ have no effect on whether $T$ is a BFS tree of $G$ rooted at $r$.
\end{proof}

Therefore, to find all vertices $r$ such that $T$ is a BFS tree rooted
at $r$, we first compute $T'$ and, if $T'$ is not empty choose a
representative $r\in T'$ and use the algorithm of the previous section
to test if $T$ is a BFS tree of $G$ rooted at $r$. If so, then $T$ is
a BFS tree of $G$ for every vertex $r\in V(T')$.  If not, then $T$ is
no a BFS tree of $G$ for any root $r\in V(G)$.  Since each of these
steps can be accomplished in $O(V(G)+E(G))$ time, we obtain the
following result:

\begin{thm}
Let $G$ be an undirected graph $G$ and let $T$ be a spanning
tree of $G$.  There exists an $O(|V(G)|+|E(G)|)$ time algorithm to find the
set of all vertices $r\in V(G)$ such that $T$ is a BFS tree of $G$
rooted at $r$.  
\end{thm}



%%%%%%%%%%%%%%%%%%%%%%%%%%%%%%%%%%%%%%%%%%%%%%%%%%%%%%%%%%%%%%%%%%%%%%%
\section{Conclusions}\seclabel{conclusions}

We have considered algorithms that take as input a graph $G$ and a
spanning tree $T$ of $G$ and decide if $T$ is a BFS or DFS tree of
$G$.  These algorithms all run in optimal $O(|V(G)|+|E(G)|)$ time.  In
deriving these algorithms, we found some simple but non-obvious
properties properties of DFS and BFS trees.  For example, it follows
from the algorithm of \secref{edge-marker} that if $T$ is a DFS tree of an
undirected graph $G$ then it is always possible to find a root of $T$
that is a leaf.  Another example of such a property is that the set of
possible roots for a BFS tree form a connected subtree.

One case we have not considered is when $G$ is a directed graph, $T$
is an undirected tree and the goal is to find all vertices $r\in V(G)$
such that the directed tree $T_r$ is a BFS or DFS tree rooted at $r$.
As is the case with rooted DFS trees, it seems that directed edges
complicate matters considerably.

The two graph traversal algorithms we have considered lie at opposite
extremes of the spectrum in the sense that DFS attempts to fully
examine a subtree before moving onto another subtree while BFS
examines only the next level in a subtree before switching to a
different subtree.  There are, of course, other graph traversal
algorithms that lie somewhere between these two extremes.  Given any
graph traversal algorithm $\mathcal{X}$ that generates a spanning
tree, we can always ask the question: Is $T$ a spanning tree that could have
been generated by $\mathcal{X}$?

\bibliographystyle{plain}
\bibliography{bfsdfs}

\end{document}
