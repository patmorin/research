\documentclass{patmorin}
\usepackage{amsmath}
\usepackage{fancyvrb}

\title{Responses to Reviewers}
\author{Prosenjit Bose, Vida Dujmovi\'c, Pat Morin, and Michiel Smid}

\begin{document}
\maketitle

This document contains a summary of changes made to our paper in response
to referee comments.  We are grateful to the referees for the careful
treatment they have given our paper and their many suggestions that have
undoubtedly improved the paper.

\section{Comments from Editor}

The following editorial remarks, provided by the editor, have all been
addressed:

\begin{Verbatim}[frame=single]
paragraph before section 1.3: You mean that the number of faults need
not be known in advance, it's f(k) for all k simultaneously

page 4 line -1: x_i are chosen arbitrarily?

page 5 line 1: do you assume f(k) section 1.4 line 7: what is the value
of t here?

page 5 line -4: spanner of what stretch?

middle of page 15: correct -> closer to the truth/being correct (since
it could be that neither is correct)

page 15 line -9: the total lengths -> their total length
\end{Verbatim}

\section{Referee \#1}

The following editorial remarks, provided by the referee, have all been
addressed:

\begin{Verbatim}[frame=single]
Page 2:
-"spanning ratio" is usually called the stretch or the distortion,
I believe.

-You should probably cite some of the papers on fault-tolerant spanners
when saying that they are the closest work

-Don't capitalize after a colon: should be "...same shortcoming: any
$r$-fault-tolerant..."

Page 3:
-In the figure caption, while it does seem true that G \ S^+ is a
3-spanner, since robustness just require that G \ S be a 3-spanner on 
G \ S^+ it's a little confusing to talk about G \ S^+ being a spanner.

Page 4:
-"..leave a subgraph of size n - o(n) that is a t-spanner" 
-- shouldn't it be a 1-spanner, not a t-spanner?
\end{Verbatim}

With respect to the following suggestions, we agree with the referee
that the purpose of Section~1.3 was initially unclear.  The main point
we wanted to make was that $f(k)$-robust spanners have non-trivial
magnification functions (size of neighbourhood of a set), which implies
non-trivial isoperimetric inequalities (number of edges leaving a set),
and, as a consequence, they do not have small separators.
We have rewritten this section to make it clearer.

With regard to the definition of magnification function, we have left it
as is.  There is nothing in the definition that precludes a magnification
function such as $h(x) = \min\{x^2, n-x\}$.

\begin{Verbatim}[frame=single]
Section 1.3:
-At a high level, I'm not convinced by this section that robustness is
really anything at all like expansion. I don't really even see why you're
trying to make this argument, but it's certainly not convincing to me.

-First sentence of section: this definition of magnification cannot be
right, since if S = V then |N(S)| = 0. Maybe only require it for S with
|S| \leq n/2 (like in vertex expanders)?

-"If not..." -- you should make it clear that this is negating the
existence of such a magnification function (for a proof by
contradiction) rather than negating the "If G is f(k) robust"
=The last paragraph of the section makes me wonder what the point of
even writing the section was.
\end{Verbatim}

The following editorial remarks, provided by the referee, have all been
addressed:

\begin{Verbatim}[frame=single]
Page 5:
-Next to last paragraph: the size of the O(k f(k))-robust spanner is
O(n f*(n)), not O(n f*(k)).

Page 6:
-In Figure 4, why are there no edges {x_i, x_{i+1}} for i \geq r?
Shouldn't they still exist in G'?

-"O(k log k)-robust spanner" should be "O(k log k)-robust 1-spanner".

- "r \in \{0,...,\}" -- it's not at all obvious that this set consists

of all integers in the range. Maybe instead something like "r \in
\{0,1,2,3,...,2^{\log n} - 1\}".

-You should make it clear that "integer multiple" includes negatives
-- I at first thought that this only held for i \geq r.

-Somehow the first two paragraphs of the proof of Theorem 1 managed to
make a very simple construction seem very complicated. Using \ell
instead of 2^j is quite confusing, as is the definition that i-r is a
multiple of \ell. Maybe instead talk about a hierarchy of edges, each
of the appropriate length? I think it would make things much clearer.

Page 7:
-"G_{2X} \setminus S is a 1-spanner of S^+" -- should be 1-spanner of
G \setminus S^+.

-Why switch from i-r to r-1? Obviously it's the same with respect to
being equal to 0 mod 2^{log(4k)}, but it's unnecessarily confusing.
\end{Verbatim}

The following three comments are related.  The reason edges are forced to be powers of 2 is so that, after removing the vertices in $S^+$, there remains a path that visits every every vertex in $V\setminus S^+$ in order, so the remaining graph is a 1-spanner.  This explanation was missing and has now been added.

\begin{Verbatim}[frame=single]
-Why force the length of edges to be powers of 2? It works, but it
seems like a strange requirement and should probably be justified
somewhere.

-Theorem 2: should mention that it's a 1-spanner.

-Proof of Theorem 2: stretch needs to be analyzed somewhere
\end{Verbatim}

The following comments from the referee have all been addressed:

\begin{Verbatim}[frame=single]
Page 8:
-Next to last line of page: obvious typo in "i - r \equiv..."

Page 9:
-In figure 6, need the "round up to power of 2" braces around the
lengths of edges

-S^+ needs to be defined as S \cup killed vertices -- right now it's
undefined.

Page 10:
-Corollary 1: it should be specified that \vareps > 0, and the
dependence on \vareps should be shown (or it should be stated that
\vareps is a constant)

-Bottom of the page: k has been fixed to an arbitrary natural number,
so what does it mean to be a ck-robust k-spanner? Constant robustness
is meaningless -- the robustness function must be at least linear. I
assume you don't want to fix k until later.
\end{Verbatim}

To address the following comments, the proof of Theorem~3 has been
rewritten and a figure has been added.  The proof of Theorem~4 has also
been expanded.
\begin{Verbatim}[frame=single]
Page 11:
-The proof of Theorem 3 is terribly written, and thus very confusing.
I suggest cleaning it up, and maybe adding a figure.

-Theorem 4: for a journal submission I'd expect a proof, not just
a sketch.
\end{Verbatim}

The following comments from the referee have all been addressed:

\begin{Verbatim}[frame=single]
Page 12:
-The first paragraph break doesn't make any sense, since you're still
describing the spanner.

-It should be mentioned that the size of the dumbbell tree spanner
is O(pn).

-Theorem 5: what is the dependence on d and t?
\end{Verbatim}

\section{Referee \#2}

We have addressed all of the following comments provided by the referee:

\begin{Verbatim}[frame=single]
-P.2, l.-7: Not sure 2007 should be considered recent (even though it
feels as the book came out yesterday).

-The crosses in Figure 1 are hard to make out. Please improve it.

-I found the "magnification function" slightly confusing (probably
just the wording). As I understand it every graph has an infinite
number of magnification functions. Then you say a graph "has a
magnification function". Should it be "there exists a magnification
function h(x)=cx for G"?

-P.8, l.7: It's not enough for the function f(k) to be increasing and
convex for f(x)/x to be non-decreasing. You have to add the property
from page 7 "f(k_0+1)-f(k_0)>1".

-P.10, footnote: Is the footnote really needed?

-References 24 and 28: Euclidean and Delaunay

\end{Verbatim}

We have not implemented the following request, because we are unable to
come up with figures that are simultaneously non-trivial, accurate, and
readable.  Dumbbell trees are messy; the dumbbells defining a dumbbell
tree are not strictly nested, the trees have alternating ``dumbbell
nodes'' and ``head nodes''.  Even an accurate representation of a dumbbell
tree for 6 points is quite messy.  We felt it better to give the reader
only an accurate description of the properties of dumbbell trees that
we use in our proof.

\begin{Verbatim}[frame=single]
-Section 3: Would it be possible to add a figure of a dumbbell tree?
\end{Verbatim}

\section{Referee \#4}

The following comments from the referee have all been addressed:

\begin{Verbatim}[frame=single]
p.5, Sec. 1.4: Provide bounds for explicit functions f() too. Also,
for dimensions d > 1, say what is the stretch, and what is the running
time.

p.6, Fig. 4: The graph should have consecutive edges (x_i,x_{i+1}) for
I \ge r. This mistake in the figure is very confusing.

p.6, the beginning of the proof of Theorem 1: multiple => integer
multiple.

p.6 Also, explain in more details how does the graph look like. Provide
more details as to why (x_{i-2^j},x_{i+2^j}) is an edge. Specify that
f(k) = 2k in this proof.

p.8, l.-2: typo in the "mod" expression.

Sec. 3: explicate the dependencies on the stretch and the dimension.
How many dumbbell trees are there?

p.12, the first paragraph of the proof of Thm 5: "a set of O(n/k')
vertices" - prove this fact.

p.14, Sec. 3.1, proof of theorem 6: Why will this spanner contain O(n)
edges? I could not follow this argument. What is s(n)? Section 3.1
should be significantly expanded.
\end{Verbatim}

We have not addressed the following comment because we are not sure what the referee means by the term ``induced spanners'':

\begin{Verbatim}[frame=single]
P.2 Explain the notion of induced spanners already in the
introduction. This would facilitate understanding of the notion of
robust spanners, and of your results.
\end{Verbatim}



\end{document}
