\documentclass{patmorin}
\usepackage{amsthm,amsmath,graphicx}
\usepackage{pat}

\title{\MakeUppercase{Notes on Robust Spanners}}
\author{Prosenjit Bose, Vida Dujmovi\'c, Pat Morin, and Michiel Smid}



\begin{document}
\maketitle

\begin{abstract}
We consider a new definition of fault-tolerance for $t$-spanners that
we dub robustness.  A $t$-spanner is robust if the removal of any $k$
vertices leaves $n-f(k)\cdot o(n)$ vertices whose induced subgraph is
a $t$-spanner.
\end{abstract}

\section{Introduction}

For a graph $G=(V,E)$ and a subset $S\subseteq V$ of $G$'s vertices,
$G[S]$ denotes the subraph of $G$ induced by $S$ and $G\setminus
S=G[V\setminus S]$ is the graph obtained by removing the elements of $S$
from $V$.

Let $V\subset \R^d$ be a set of $n$ points in $\R^d$.  A connected graph
$G=(V,E)$ is a $t$-spanner of $V$ if, for every pair $x,y\in V$,
\[
   \frac{\|xy\|_G}{\|xy\|} \le t \enspace ,
\]
where $\|xy\|$ denote the Euclidean distance between $x$ and $y$ and
$\|xy\|_G$ denotes the length of the Euclidean shortest path from $x$ to
$y$ that uses only edges in $G$.

The graph $G$ is an \emph{$(f(k),g(n))$-robust $t$-spanner} of $V$ if,
for every subset $S\subseteq V$, there exists a superset $S'\supseteq
S$, $|S'|\le f(|S|)$, such that $G\setminus S'$ is a $t$-spanner.
Intuitively, killing the vertices in $S$ leaves a small number, $f(|S|)$,
of vertices stranded but most of the graph remains a $t$-spanner.

Robustness is related to, but different from, $k$-fault tolerance.
A \emph{$k$-fault-tolerant} $t$-spanner $G=(V,E)$ has the property that
$G\setminus S$ is a $t$-spanner for any subset $S\subseteq V$ of size
at most $k$.  At a minimum, a $k$-fault-tolerant spanner must remain
connected after the removal of any $k$ vertices.  This immediately implies
that any $k$-fault-tolerant spanner with $n$ vertices has $\Omega(kn)$
edges.  Several constructions of $k$-fault-tolerant spanners having
$O(kn)$ edges exist \cite{A,B,C}.
 

\section{One Dimensional Point Sets}

%This section gives upper and lower-bounds on the number of edges
%in $(f(k),1)$ robust $t$-spanners for 1-dimensional point sets.
%%We first give a simple construction that is $(O(k\log k),1)$-robust
%%and a lower-bound on the number of edges in any $t$-spanner that is
%%$(O(k),1)$-robust.
%
%The arguments used to prove these simple bounds contain the main ideas
%needed to prove bounds for a more general class of functions, $f(k)$,
%that are subsequently discussed.

\subsection{An $O(k\log k)$-robust spanner with $O(n\log n)$ edges}

In this section, we consider the following graph, $G=(V,E)$ which is
closely related to the hypercube.  Let $x_1<x_2<\cdots<x_n$ be the
elements of $V$. The edge set, $E$, of $G$ consists of
\[
    E = \{x_ix_{i+2^j} : j\in\{0,\ldots,\floor{\log n}\},\, i\in\{1,\ldots,n-2^j\} \} \enspace .
\] 
Notice that $G$ is a 1-spanner since it contains every every edge of the
form $x_i,x_{i+1}$, for $i\in\{1,\ldots,n-1\}$. Furthermore, $G$ has size
$O(n\log n)$ since every vertex has degree at most $2\floor{\log n}+2$.
We now prove an upper-bound on the robustness of $G$ by using the
probabilistic method.

\begin{thm}\thmlabel{better-upper-bound-1d}\thmlabel{klogk-1d}
Let $V\subset \R$ be any set of $n$ real numbers.  Then there exists an
$(O(k\log k),1)$-robust $1$-spanner of $V$ of size $O(n\log n)$.
\end{thm}

\begin{proof}
Consider any set $S\in\binom{V}{k}$.  Select a random integer
$r\in\{0,\ldots,2^{\floor{\log n}}\}$ and consider the subgraph, $G'$,
of $G$ consisting only of the edges of the form $(x_i,x_{i+\ell})$
where $i-r$ is a multiple of $\ell$.  (For readers with a background
in data structures, this subgraph looks a lot like a skiplist; see
\figref{g-prime}.)

\begin{figure}
  \begin{center}
  \includegraphics{robust-1}
  \end{center}
  \caption{The graph $G'$}
  \figlabel{g-prime}
\end{figure}

For a vertex $x_i\in S$, let $j$ be the largest integer such that $i-r$
is a multiple of $2^j$.  Then we say that $x_i$ \emph{kills} the vertices
$x_{i-2^{j}+1},\ldots,x_{i+2^{j}-1}$ in $G'$; see \figref{killing}.
When this happens, the \emph{cost} of $x_i$ is $c(x_i)=2^{j+1}-1$.
Observe that, if we define $S'$ to be the set of all vertices killed by
vertices in $S$, then $G'\setminus S'$ (and hence also $G\setminus S'$)
is a 1-spanner.

\begin{figure}
  \begin{center}
  \includegraphics{robust-2}
  \end{center}
  \caption{Constructing the set $S'$ (whose elements are denoted by $\times$)
from the set $S$ (whose elements are denoted by \textbullet).}
  \figlabel{killing}
\end{figure}

We say that a vertex $x\in S$ is
\emph{cheap} if $c(x) < 4k$ and \emph{expensive} otherwise.
We call our experiment (of choosing $r$) a \emph{failure} if
\begin{enumerate}
  \item $\mathcal{A}$: any vertex of $S$ is expensive; or
  \item $\mathcal{B}$: the total cost of all cheap vertices exceeds $16k\log k$
\end{enumerate}
The experiment is a \emph{success} otherwise.
Observe that, if our experiment is successful, then we obtain a set $S'$,
$|S'|\le ck\log k$, such that $G\setminus S'$ is a 1-spanner.  Therefore,
all that remains is to show that the probability of success is greater than 0.

We first note that 
\[
   \Pr\{\mathcal{A}\} \le k/4k = 1/4 \enspace ,
\]
since the number of vertices of $G'$ incident on edges of length greater
than $4k$ is at most $n/4k$.  
Next, we observe that the total expected cost of all cheap vertices is at most
\begin{eqnarray*}
   \E\left[\sum_{x_i\in S}c(x_i) \right] 
  &=& k\sum_{x_i\in S}\E[c(x_i)]  \\
  &\le&  k\sum_{j=0}^{\floor{\log (4k)}} (2^{j+1}-1)/2^j \\
   &=& k\sum_{j=0}^{\floor{\log (4k)}} 2^j \\
   &<& 8k\log k \enspace .
\end{eqnarray*}
Therefore, by Markov's Inequality, $\Pr\{\mathcal{B}\}\le 1/2$.
By the union bound
\[
   \Pr\{\mbox{$\mathcal{A}$ or $\mathcal{B}$}\} \le 1/4+ 1/2 < 1
\enspace .
\]
for sufficiently large $k$.
\end{proof}


\newcommand{\eps}{\varepsilon}

\subsection{An $O(k^{2})$-robust spanner with $O(n\log\log n)$ edges}

Next, we use a construction similar to the one in the previous section to
obtain an $O(k^2)$-robust 1-spanner for 1-dimensional point sets.
Again, let $x_1<x_2<\cdots<x_n$ be the elements of $V$.  Let $P$ denote the
set of edges in the path that visit $x_1,\ldots,x_n$ in order.
The edge set,
$E$, of $G$ consists of
\[
  E =  P 
      \cup \{x_ix_{i+2^{2^j}} :
               j\in\{0,\ldots,\floor{\log\log n}\},\, i\in\{1,\ldots,n-2^{2^j}\} \} \\
\enspace .
\]
The maximum degree of any vertex in $G$ is no more than $2\log\log
n+2$. Therefore $G$ has size $O(n\log\log n)$.

\begin{thm}\thmlabel{quadratic-1d}
Let $V\subset \R$ be any set of $n$ real numbers.  Then there exists an
$(O(k^2),1)$-robust $1$-spanner of $V$ of size $O(\eps^{-1}n\log\log n)$.
\end{thm}

\begin{proof}
The proof is very similar to the proof of \thmref{better-upper-bound-1d}.
Again, we select a random integer $r\in\{0,\ldots,2^{2^{\ceil{\log\log
n}}}-1\}$ and consider the graph $G'$ consisting only of edges of the
form $x_i,x_{i+\ell}$, where $i-r$ is a multiple of $\ell$.  Note that
$G'$ is similar, again, to a skiplist.  In particular, it is a 1-page
graph and $x_r$ is a cut vertex.

For an edge $x_i\in S$, consider the smallest integer $j$ such that
$i-r$ is not a multiple of $2^{2^{j}}$.  (Informally, on can think
of $2^{2^{j}}$ as the length of the shortest edge of $G'$ that passes
over $x_i$.)  Then $x_i$ kills $x_{i-p},\ldots,x_{i+q}$ where
\[
   p= i-r \bmod 2^{2^j}+1 \text{ and } 
   q= r-i \bmod 2^{2^j}-1 \enspace .
\]
The set $S'$ contains all the vertices killed by all vertices in $S$.
As before, there is a monotone path in $G'$ (and hence also in $G$)
that contains all vertices not in $S'$, so $G\setminus S'$ is a 1-spanner.

The cost, $c(x_i)$, of a vertex $x_i\in S$ is equal to the number
of vertices it kills.  A vertex in $S$ is expensive if it kills at
least $m=16k^{2}$ vertices and cheap otherwise.  If the shortest edge
that passes over a vertex $x$ has length at least $16k^2$, then $x$
is incident on an edge whose length is at least $4k$ (recall that
$2^{2^j}=\sqrt{2^{2^{j+1}}}$).  Therefore, the probability that there
exists any expensive vertex is at most $1/4$.  What remains is to bound
the expected cost of the cheap vertices.  This is bounded by the sum
\begin{eqnarray*}
  \E\left[\sum_{x\in S} c(x)\right] 
    & \le & k\sum_{i=0}^{\lfloor\log\log m\rfloor}\frac{2^{2^{j+1}}}{2^{2^{j}}} \\  
    & = & \sum_{i=0}^{\lfloor \log\log m \rfloor}2^{2^j} \\
    & \le & 16k^2 + O(k) \in O(k^2) \enspace ,
\end{eqnarray*}
as required.
\end{proof}

\subsection{An $O(k^{(1+\eps)})$-robust spanner with $O(n\log\log n)$ edges}

The construction in \thmref{quadratic-1d} generalizes to yield an
$O(k^{1+\eps})$-robust spanner for any $\eps>0$.  Let $\eps$, $0<\eps<1$
be a parameter.  The edge set, $E$, of $G$ consists of
\[
  E =  \{x_ix_{i+1} : i\in\{1,\ldots,n-1\}\} 
      \cup \{x_ix_{i+2^{\floor{(1+\eps)^j}}} : 
               i\in\{1,\ldots,n-2^{\floor{(1+\eps)^j}}\} \} \\
    \enspace .
\]
The maximum degree of any vertex in $G$ is $2j$, where $i$ is the minimum
integer such that $2^{\floor{(1+\eps)^j}} > n$.  Taking logarithms twice
and using the fact that $1+1/t = \Theta(e^t)$ implies that
\[
    j \in O(\epsilon^{-1}\log\log n) \enspace .
\]

The same proof technique used to show \thmref{quadratic-1d} yields the
following result:

\begin{thm}\thmlabel{better-upper-bound-1d}\thmlabel{doubleexp-1d}
Let $S\subset \R$ be any set of $n$ real numbers.  Then, for any $0<\eps
< 1$, there exists an $(O(k^{1+\eps}),1)$-robust $1$-spanner of $S$
of size $O(\eps^{-1}n\log\log n)$.
\end{thm}


\section{Lower Bounds}


\begin{thm}\thmlabel{simple-lower-bound-1d}
Let $V=\{1,\ldots,n\}$ and let $t\ge 1$ be a constant.  Then any
$(O(k),1)$-robust $t$-spanner of $S$ has $\Omega(n\log n)$ edges.
\end{thm}

\begin{proof}
Fix an even natural number $k$ and consider some $i\in\{ck,\ldots,n-ck\}$.
We claim that if $G=(V,E)$ is a $(ck,1)$-robust\footnote{Note that we
have gone from $(O(k),1)$-robust in the statement of the theorem to
$(ck,1)$-robust in the proof.  This does not cause a problem so long
as we only consider values of $k$ greater than some constant $k_0$
hidden in the $O$ notation.} $t$-spanner, then $G$ has at least $k/2$
edges, $xy$, such that $x < i < i+k/2 < y$ and such that $y-x < ctk$.
If this is not the case, then there exists a set $S$, $|S|\le k$,
such that any edges of $G$ with one endpoint to the left of $S$
and one endpoint to the right of $S$ have length at least $ctk$.
A straightforward analysis shows that the existence of such an $S$
implies that $G$ is not $(ck,1)$-robust.

Applying the above argument to $i=ck+2jtck$, for
$j\in\{0,\ldots,\floor{(n-ck)/2tck}\}$ implies that $G$ contains
$\Omega(n/tc)$ edges whose length is in the range $[k/2,ctk]$.  Applying
this argument for $k\in\{\ceil{(2tck_0)^{j}} : j\in\{0,\ldots,\floor{(\log
n)/\log(2tck_0)}\}$ proves that, for any constants $c,k_0,t>1$, $G$
has $\Omega(n\log n)$ edges.
\end{proof}

Notice that the upper bound in \thmref{klogk-1d} and the lower bound
in \thmref{simple-lower-bound-1d} differ by a factor of $\log k$ in the
function $f(k)$.  We conjecture that the lower bound is the true answer:

\begin{conj}
Let $S\subset \R$ be any set of $n$ real numbers.  Then there exists an
$(O(k),1)$-robust $1$-spanner of $S$ of size $O(n\log n)$.
\end{conj}

\subsection{A General Lower Bound}\seclabel{generalized}

Let $k_0\ge 1$ be a constant and let $f(\cdot)$ be any function that is
weakly convex and increasing over the interval $[k_0,\infty)$, and such
that $f(k_0+1)-f(k_0) > 1$.  Let $f^{i}(k)$ be function $f$ iterated
$i$ times on the initial value $k$, i.e.,
\[
   f^{i}(k) = \underbrace{f(f(f(\cdots f}_{i}(k)\cdots))) \enspace .
\]
Then we define the \emph{iterated
$f$-inverse function}
\[
   f^*(n) = \min\{i : f^{i}(k_0) \ge n\} \enspace .
\] 
Notice that, for any $k> k_0$, there exists $i$ such that
\[
   f^i(k_0) < k \le f(f^i(k_0)) \enspace .
\]
In particular, the sequence $k_0,f(k_0),f^2(k_0),\ldots,$ contains
a value $f^{i+1}(k_0)$ such that
\[
      k  \le  f^{i+1}(k_0) < f(k) \enspace .
\]
%\begin{thm}[Generalized Upper Bound]\thmlabel{general-upper-bound-1d}
%Let $k_0$, $f$, and $f^*$ be defined as above, and let $S\subset \R$ be
%any set of $n$ real numbers.  Then there exists an $(O(kf(k+1)),1)$-robust
%$1$-spanner of $S$ of size $O(nf^*(n))$.
%\end{thm}
%
%\begin{proof}
%The upper bound construction is similar to the graph constructed in the
%proof of \thmref{upper-bound-1d}.  It contains the edge set
%\[
%    E = \{x_ix_{i+f^{j}(k_0)} : j\in\{0,\ldots,f^*(n)-1\} \enspace .
%\] 
%Given a set $S'\in \binom{S}{k}$, we define $d$ as
%\[
%   d=\min\{f^i(k_0) : f^i(k_0)> k+1,\, i\in\Z\} \enspace .
%\]
%Thus, we have the inequalities
%\[
%     f^{-1}(k+1) < d \le f(k+1) \enspace .
%\]
%As before, we define a good block as a sequence of $r>2d-1$ consecutive
%vertices that do not belong to the same block.  The remainder of the
%proof is as before.   There are at most $k+1$ maximal good blocks that
%contain a total of at least $n-k(2d-1)$ vertices.  A path can be found
%from each block to the next block that requires the elimination of at
%most $d=f(k+1)$ vertices from the end of one block and the beginning of
%the next.  In total, this leaves a set $\bar{S'}$ of size at least
%$n-O(kd) = n - O(kf(k+1))$, as required.
%\end{proof}
%
%Taking different functions $f(k)$ in \thmref{general-upper-bound-1d} gives several tradeoffs:
%
%\begin{cor}
%Let $S\subset \R$ be any set of $n$ real numbers.  Then there exists an
%\begin{itemize}
%  \item $(O(k^2),1)$-robust $1$-spanner of $S$ of size $O(n\log n)$
%  [$f(k)=2k$];
%  \item $(O(k^{2+\varepsilon},1)$-robust $1$-spanner of $S$ of size
%  $O(n\log\log n)$ [$f(k)=k^{1+\varepsilon}$];
%  \item $(O(k(1+\varepsilon)^k),1)$-robust $1$-spanner of $S$ of size
%  $O(n\log^* n)$ [$f(k)=(1+\varepsilon)^k$].
%\end{itemize}
%\end{cor}
%

\begin{thm}\thmlabel{general-lower-bound-1d}
Let $k_0$, $f$, and $f^*$ be defined as above, and let $S\subset \R$
be any set of $n$ real numbers.  Let $S=\{1,\ldots,n\}$ and let $t\ge
1$ be a constant.  Then any $(f(k),1)$-robust $t$-spanner of $S$ has
$\Omega(nf^*(n)$ edges.
\end{thm}

\begin{proof}[Proof Sketch]
The proof is similar to the proof of \thmref{simple-lower-bound-1d}.
We need only consider $f(k)=\omega(n)$ since, otherwise we can apply
\thmref{simple-lower-bound-1d}.  We group the edges of the graph $G$ into
$\Omega(f^*(n))$ classes and show that each class contains $\Omega(n)$
vertices.  In this case, the edges in each class have lengths in a
range of the form $[f^i(k_0),tcf^{i}(k_0)]$.  For large enough $i$,
these classes do not overlap since $f(k)=\omega(n)$.
\end{proof}

\begin{rem}
Observe that if we take $f(k)=k^{1+\eps}$ for any constant $\eps >0$, then
$f^*(k) \in \Theta(\log\log n)$.  Applying \thmref{general-lower-bound-1d}
then shows that both \thmref{quadratic-1d} and \thmref{doubleexp-1d}
are optimal with respect to the number of edges in $G$.
\end{rem}

\begin{rem}
We note that using the iterated functions defined in this section, other
``optimal'' upper-bounds can be obtained.  However, taking $f(k)$ to
be any polynomial in $k$ still yields a graph with $\Theta(\log\log n)$
edges.  Thus, one needs to choose superpolynomial functions $f(k)$, which
have quickly diminishing returns.  For instance, one can redo the proof
of \thmref{quadratic-1d} to show that there is an $(O(2^k),1)$-robust
spanner having $O(n\log^* n)$ edges, and \thmref{general-lower-bound-1d}
shows that this is optimal.
\end{rem}



\section{Higher Dimensions}

In this section, we give a family of constructions for point sets
$V\subset\R^d$, $d\ge 1$.  These constructions make use of dumbbell
tree spanners \cite{X}.  A \emph{dumbbell tree spanner} of $V$ is a set
of $O(1)$ binary trees $\mathcal{T}=\{T_1,\ldots,T_p\}$.  Each binary
tree $T_i$ has $n$ leaves, one for each vertex of $S$.  Each of $T_i$'s
internal nodes also stores a point of $S$, and no point of $S$ is stored
in more than one internal node of $T_i$.  These trees have the property
that, for any two points $x,y\in S$, there exists some tree $T_i$ such
that the path from $x$ to $y$ in $T_i$ has length at most $t'\|xy\|$,
where $t'>1$ is a parameter.  Thus, the graph $G=(S,E)$ obtained by
taking the union of all edges in all trees is a $t'$-spanner.

\begin{thm}\thmlabel{dd}
Let $k_0$, $f$, and $f^*$ be defined as in \secref{generalized}.
Let $S\subset \R^d$ be any set of $n$ points in $\R^d$.  Then, for any
constant $t>1$,  there exists a $(O(kf(k)),1)$-robust $t$-spanner of $S$
with $O(nf^*(n))$ edges.
\end{thm}

\begin{proof} 

Fix a value $k$ and recall that, in any binary tree, $T$, with $n$
nodes, there exists a vertex whose removal disconnects $T$ into at most 3
components each of size at most $n/2$.  Repeatedly applying this
fact yields a set of $O(n/k)$ vertices whose removal disconnects $T$ into
components of size at most $k$; see \figref{dumbbell-chop}.  Perform this decomposition for each of
the trees $T_1,\ldots,T_p$ in a dumbbell tree spanner to obtain a set of
$X$ of $O(n/k)$ vertices whose removal disconnects every tree in $T$ into
components each of size at most $k$.  Using any of the $k$-fault tolerant
spanner constructions discussed in the introduction, we can construct
a $k$-fault tolerant spanner for $X$ having $O(k|X|)=O(n)$ edges.
Let $G_k=(V,E_k)$ denote the graph whose edge set contains all edges of the
dumbbell spanner and all edges of the $k$-fault tolerant spanner on $X$.

\begin{figure}
\begin{center}
  \includegraphics{dumb-1}
\end{center}
\caption{A dumbbell tree decomposed in components of size $O(k)$ by the
removal of a set $X$ of $O(n/k)$ vertices (each denoted by $\circ$).}
\figlabel{dumbbell-chop}
\end{figure}

Suppose, now that we are given a set $S\subseteq V$, $|S|\le k$.  A vertex
$x\in S$ appears at most twice in each tree $T_i$.  We say that $x$ kills
all the vertices in any component of $T_i\setminus X$ that contains $x$.
Furthermore, if $x\in X$, the $x$ kills the (at most 2) components whose
roots have $x$ as a parent.  The total number of vertices killed by $x$
is therefore $O(k)$; see \figref{dumbbell-kill}.

\begin{figure}
\begin{center}
  \includegraphics{dumb-2}
\end{center}
\caption{The set $S$ (whose elements are denoted by \textbullet) kill
$O(|S|k)$ vertices in each dumbbell tree.}
\figlabel{dumbbell-kill}
\end{figure}

Let $S'$ be the set of all vertices killed by all vertices in $S$ and
consider some pair of vertices $x,y\in V\setminus S'$.  There exists a
tree $T_i$ such that the path in $T_i$ from the leaf containing $x$ to the
leaf containing $y$ has length at most $t'\|xy\|$.  If the leaves of $T_i$
containing $x$ and $y$ are in the same component of $T_i\setminus X$ then
the path in $T_i$ from $x$ to $y$ is also a path in $G_k\setminus S'$ and
this path has length $t'\|xy\|$.

If $x$ and $y$ are in different components of $T_i\setminus X$ then
$G_k\setminus S'$ contains a path, from $x$ up to the lowest ancestor,
$x'$, of $x$ that appears in $X$.  Similarly, $G_k\setminus S'$ contains
a path from $y$ up to the lowest ancestor, $y'$, of $y$ that appears
in $X$.  Finally, since $G_k$ contains a $k$-fault tolerant $t'$-spanner
on the vertices of $X$, and $|S|\le k$, there is a path from $x'$ to
$y'$ of length at most $t'\|x'y'\|$.  Therefore, $G_k\setminus S'$,
contains a path from $x$ to $y$ of length at most $(t')^2\|xy\|$.
Taking $t'=\sqrt{t}$ shows that $G_k\setminus S'$ is a $t$-spanner.

We have just shown how to construct a graph $G_k$ that has $O(n)$ edges
and is $O(k^2)$ robust provided that $|S|\le k$.  To obtain a graph that
is $kf(k)$-robust for any value of $k$, we take the graph $G$ containing
the edges of each $G_k$ for $k\in\{f^i(k_0) : i\in\{0,\ldots,f^*(k)\}\}$.
The graph $G$ has $O(nf^*(n))$ edges.  For any set $S\in \binom{V}{k}$,
we can apply the above argument on the subgraph $G_{k'}$ with $k \le k'
< f(k)$, to show that $G$ is $(O(kf(k)),1)$-robust.
\end{proof}

\subsection{Linear Size Robust Spanners}

Finally, we show that linear size robust spanners exist, if one is willing
to pay for this with robustness.

\begin{thm}\thmlabel{linear-size}
If $(f(k),g(n))$-robust $t$-spanners with $O(n s(n))$ edges exist for all
$V\subset\R^d$, then $(f(k),s(n)g(n))$-robust $t$-spanners with $O(n)$
edges exist for all $V\subset\R^d$.
\end{thm}

\begin{proof}
Perform the same dumbbell tree decomposition used in the proof of
\thmref{dd} to obtain a set $X$ of $O(n/s(n))$ nodes whose removal
partitions each dumbbell tree into components of size at most $s(n)$.
Construct an $(f(k),g(n))$-robust $t$-spanner on the elements of $X$.  The
same argument used to prove \thmref{dd} shows that the resulting
construction is $O(f(k),s(n)g(n))$ robust.  (Each vertex of $X$ that
belongs to $S$ or $S'$ results in the loss of at most 2 components in each
dumbbell tree, each of size at most $s(n)$.
\end{proof}

The following corollary is obtained by combining \thmref{linear-size}
with some of our upper-bound constructions:
\begin{cor}
There exist linear size
\begin{itemize}
  \item $(O(k\log k),O(\log n))$-robust $1$-spanners of any
$V\subset \R$;
  \item $(O(k^{1+\eps}),O(\log\log n))$-robust $1$-spanners of any
$V\subset \R$;
  \item $(O(k^2\log k), O(\log n))$-robust $t$-spanners of any
$V\subset \R^d$; and
  \item $(O(k^{2+\eps}), O(\log\log n))$-robust $t$-spanners of any
$V\subset \R^d$.
\end{itemize}
\end{cor}

\section{Summary}

We have introduced the notion of $(f(k),g(n))$-robust $t$-spanners and
given upper and lower-bounds on the number of edges in such spanners.
Our lower bounds show that, if $g(n)\in O(1)$, then any such spanner
must have a super-linear number of edges, even in 1 dimension.  Our
1-dimensional constructions nearly match this lower-bound except when the
function $f$ is nearly linear 

[Note: When does the switch-over happen?  $f(k)=k\log k$ gives ].

%This weaker result is no longer used
%The following theorem shows that $G$ is also $(O(k^2),1)$-robust.
%This theorem is not tight; later we will show that $G$ is in fact
%$(O(k\log k),1)$-robust. Nevertheless, the proof of this theorem
%can be considered as a warm-up to the proof of the more abstract
%\thmref{general-upper-bound-1d}.
%
%\begin{thm}[Simple 1-d Upper Bound]\thmlabel{upper-bound-1d}
%Let $V\subset \R$ be any set of $n$ real numbers.  Then there exists an
%$(O(k^2),1)$-robust $1$-spanner of $S$ of size $O(n\log n)$.
%\end{thm}
%
%\begin{proof}
%We will prove that $G$ is a robust $1$-spanner. Consider any
%$S\in\binom{V}{k}$, with $k\le \sqrt{n}$.  Let $d=2^{\ceil{\log (k+1)}}$
%and observe that $k < d < 2k+2$ and that the edge $x_ix_{i+d}$ is in $G$
%for every $i\in\{1,\ldots,n-d\}$.
%
%We say that $x_i,\ldots,x_{i+r}$ form a \emph{good block} if $r\ge
%2d-1$ and $S$ contains no element of $x_i,\ldots,x_{i+r}$.  We focus
%our attention on the set of maximal good blocks.  Note that the number
%of maximal good blocks is at most $k+1$ and that the total number of
%vertices contained in these blocks is at least $n-k(2d-1)=n-O(k^2)$.
%We will show that there exists a set $S'\supseteq S$ that avoids all but
%$2(d-1)$ vertices from each good block, and such that $G\setminus S'$
%is a 1-spanner.
%Therefore, the size of $S'$ is at most $O(k^2)-2(d-1)(k+1)=O(k^2)$.
%
%Initially our set $S'$ contains all vertices in $S$, and
%we will add at most $d-1$ vertices from the beginning and
%end of each block.  Between any two consecutive good blocks,
%$x_i,\ldots,x_{i+r}$ and $x_j,\ldots,x_{j+s}$, $G$ contains $d$
%vertex disjoint paths, $P_1,\ldots,P_{d-1}$ each of which begins
%at a vertex in $x_{i+r-d+1},\ldots,x_{i+r}$ and ends at a vertex in
%$x_{j},\ldots,x_{j+d-1}$.  (For example, one of these paths consists of
%$x_i,x_{i+d},x_{i+2d},\ldots,x_{i+cd}$, for some integer $c$.)
%
%Since $d > k$, at least one path, say $P_1$, does not use any vertices
%of $S$.  If $P_1$ begins at $x_k$ and ends at $x_\ell$ then we add
%to $S'$ all vertices in 
%$x_{k+1},\ldots,x_{\ell-1}$ that are not included in $P_1$.  Observe that adds,
%to $S'$, at most $d-1$ of the first $d$ vertices of each block
%and at most $d-1$ of the last $d$ vertices of each block.  Therefore,
%the number of vertices added to $S'$ from all good blocks is at most
%$2(d-1)(k+1)$, as claimed.
%
%Let $x_{i_1},\ldots,x_{i_{n'}}$ denote the vertices in $V\setminus
%S'$, ordered from left to right.  None of these vertices are in $S'$,
%and for each $j\in\{1,\ldots,n'-1\}$, $i_{j+1}-i_{j}\in\{1,d\}$, so the
%edge $x_{i_j}x_{i_{j+1}}$ is present in $G$.  Therefore $G\setminus S'$
%is a 1-spanner.
%\end{proof}



\end{document}
