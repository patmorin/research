\documentclass[lotsofwhite]{patmorin}
\usepackage{amsthm}

\title{\MakeUppercase{A Centerpoint Theorem for Wedges}}
\author{Jeff Erickson \and
	Ferran Hurtado \and
	Pat Morin \and
	Perouz Taslakian \and
	David R. Wood}
\date{}
\input{pat.tex}

\begin{document}
\maketitle

\begin{abstract}
The \emph{Centerpoint Theorem} states that, for any set $S$ of $n$
points in $\mathbb{R}^2$, there exists a point $p$ in $\mathbb{R}^2$
such that every halfplane containing $p$ contains at least $\ceil{n/3}$
points of $S$ and such a point can be found in $O(n)$ time using an
algorithm of Jadhav and Mukhopadhyay (1994).  We consider a
generalization of the Centerpoint Theorem in which halfplanes are
replaced with wedges (cones) of angle $\alpha$.
\end{abstract}

\section{Introduction}

Let $S$ be a set of $n$ points in $\mathbb{R}^d$.  The \emph{halfspace
depth} \cite{t75} of a point $p$ with respect to $S$ is defined as
\[
D_\pi(p,S) = 
   \min\{|h\cap S| : \mbox{$h$ is a closed halfspace that contains $p$} \}
    \enspace .
\]
The \emph{Centerpoint Theorem}, which is a simple consequence of
Helly's Theorem \cite{e93} states that for any point set $S$, there
exists a point whose halfspace depth is at least $\ceil{n/(d+1)}$.
Furthermore, for every $n$, there exists a point set $S$ in
$\mathbb{R}^d$ of size $n$ for which no point in $\mathbb{R}^d$ has
halfspace depth greater than $\ceil{n/(d+1)}$. 

In this paper we consider a generalization of halfspace depth called
\emph{$\alpha$-wedge depth}. Let $r$ be a ray with endpoint $q$.  An
$\alpha$-wedge with apex $q$ and axis $r$ is the set of all points $x$
such that the angle\footnote{We use the convention that the angle
between two line segments with an endpoint in common is the smaller of
the two angles occuring at the common point.} between $pq$ and $r$ is
at most $\alpha/2$.  The $\alpha$ wedge depth of a point $p$ with
respect to a point set $S$ is defined as
\[
D_\alpha(p,S) =
   \min\{|h\cap S| : \mbox{$h$ is an $\alpha$-wedge that contains $p$} \} 
   \enspace .
\]
Define the function $f^d_\alpha(n)$ as follows:
\[
   f^d_\alpha(n) = \min\left\{\max\{D_\alpha(p,S):p\in
\mathbb{R}^d\}: S\subseteq\mathbb{R}^d\, ,\, |S|=n\right\}
\]
That is, $f^d_\alpha$ defines, for each $n$, the maximum value $k$ for
which every point set $S$ of size $n$ is guaranteed to define a point
whose $\alpha$-wedge depth with respect to $S$ is at least $k$.  The
Centerpoint Theorem states that $f^d_\pi(n) = \ceil{n/(d+1)}$.  In this paper
we prove the following Theorem about 2-dimensional point sets:

\begin{thm}\thmlabel{twod-result}
\[
   f^2_\alpha(n) = \left\{\begin{array}{ll}
                        1 & \mbox{if $\alpha < \pi$} \\
                        \ceil{n/3} & \mbox{if $\pi\le \alpha < 4\pi/3$} \\
                        \ceil{n/2} & \mbox{if $4\pi/3\le \alpha < 2\pi$} \\
                        n & \mbox{if $\alpha = 2\pi$\enspace .} 
                       \end{array}\right.
\]
Furthermore, for any $\alpha$ and any point set $S$ of size $n$, a
point $p$ such that $D_\alpha(p) \ge
f^2_\alpha(n)$ can be found in $O(n)$ expected time.
\end{thm}

\section{Proof of \thmref{twod-result}}

In this section we prove a sequence of lemmata that immediately imply
\thmref{twod-result}.

\begin{lem}\lemlabel{a}
If $\alpha < \pi$ then $f^2_\alpha(n) = 1$ and a point $p$ such that
$D_\alpha(p)\ge 1$ can be found in $O(1)$ time.
\end{lem}

\begin{proof}
To prove the lower bound, we simply observe that for any non-empty
point set $S$, every point $p\in S$ satisfies $D_\alpha(p,S)\ge 1$, so
$f^2_\alpha(n) \ge 1$. This proves the lower bound and gives an $O(1)$
time algorithm for finding $p$. 

For the upper bound, simply consider a set $S$ of points that are all
on the $x$-axis.  For any point $p$ on or above the $x$ axis, the
$\alpha$-wedge whose axis is vertical and upwards intersects the $x$
axis in at most one point, therefore $D_\alpha(p) \le 1$.  For any
point $p$ below the $x$ axis, the $\alpha$-wedge whose axis is
vertical and downwards does not intersect the $x$ axis at all, so
$D_\alpha(p)=0$.  In either case, $D_\alpha(p)\le 1$ so $f^2_\alpha(n)
\le 1$.
\end{proof}

\begin{lem}\lemlabel{b}
If $\pi\le \alpha < 4\pi/3$ then $f^2_\alpha(n) = \ceil{n/3}$ and a
point $p$ such that $D_\alpha(p)\ge \ceil{n/3}$ can be found in $O(n)$ time.
\end{lem}

\begin{proof} 
For the lower bound, we observe that every $\alpha$-wedge containing
$p$ also contains a halfspace containing $p$.  Therefore, the
Centerpoint Theorem implies that $f^2_\alpha(n) \ge \ceil{n/3}$.  This
proves the lower bound and the algorithm of Jadhav and Mukhopadhyay
\cite{jm94} gives an $O(n)$ time algorithm for finding $p$.

For the upper bound, we consider a point set in which $\ceil{n/3}$ or
$\floor{n/3}$ points occur on each of 3 rays originating from the
origin and such that the angle between any two rays is $2\pi/3$.  For
any point $p\in\mathbb{R}^2$, there exists a $4\pi/3$-wedge whose apex
is at $p$ and whose interior intersects only one of the three rays
(the axis of this wedge is parallel to this ray). This $4\pi/3$ wedge
contains an $\alpha$-wedge that contains $p$ and intersects only one
of the three rays, therefore $D_\alpha(p,S)\le \ceil{n/3}$.  Since the choice
of $p$ is arbitrary, this implies that $f^2_\alpha(n) \le \ceil{n/3}$.
\end{proof}

The next part of the proof uses the notion of halving lines.  A line
$\ell$ is a \emph{halving line} of a (2-dimensional) point set $S$ of
size $n$ if $\ell$ contains at least one point of $S$ and each of the
two closed halfplanes bounded by $\ell$ contains at least $\ceil{n/2}$
points of $S$.  The following lemma was proven by Fekete and Meijer
\cite[Lemma~2]{fm00}.  However, we include a slightly different proof
because ideas from the proof will be used in the algorithm we present
later on.

\begin{lem}\lemlabel{construction}
For any point set $S$ there exists three concurrent halving lines 
of $S$ such that the angle\footnote{We use the convention that
the angle between a pair of lines is the smaller of the two angles
defined by the two lines.} between any pair of lines is $\pi/3$.
\end{lem}

\begin{proof}
To prove the existence of these three halving lines we start with one
vertical halving line and the other two halving lines forming angles
of $\pi/3$ with it, one having positive slope and the other
having negative slope.  If these three halving lines are concurrent
then the construction is complete. 

If the three halving lines are not concurrent then they bound an
equilateral triangle.  Suppose that initially the lines are directed
so that the edges of this triangle are directed counterclockwise.
Define the \emph{signed area} of a directed triangle to be the area of
the triangle if its edges are directed counterclockwise and the
negation of the area of the triangle if its edges are directed
clockwise.  Imagine continuously rotating the three directed lines while
maintaining the invariant that they are all halving lines and that the
angle between any two is $\pi/3$.  Note that, during this process, the
signed area of the triangle bounded by the three lines changes
continuously.  However, after having rotated the lines by an angle of
$\pi$, the three halving lines are identical to their initial
configuration but their directions are reversed and therefore the
signed area of the triangle is negated.  We conclude that at some
point during this process the signed area of the triangle must have
been equal to 0, at which point the three lines were concurrent.  This
completes proof.
\end{proof}

\begin{lem}\lemlabel{algorithm}
Three halving lines satisfying the conditions of \lemref{construction}
can be found in $O(n)$ time.
\end{lem}

\begin{proof}

To find the three halving lines we apply the prune-and-search paradigm
in much the same way as the algorithm of Lo, Matou\v{s}ek and Steiger
\cite{lms94} for finding planar ham-sandwich cuts.  In the standard
``computational geometry dual'' \cite[Chapter~X]{dbeXX}, our problem
is to find three points on the median level of $n$ lines such that
these points are collinear and their $x$-coordinates satisfy a certain
equation.  

More precisely, given a set $S^*$ of $n$ lines (that are dual to the
points of $S$), let
\[
    h_k(x) = \min\{ y : 
        \mbox{$(x,y)$ is on or above at least $k$ lines of
$S^*$}\} 
\]
and let $h=h_{\ceil{n_2}}$.
Then our problem is to find a value $x$ such that the three points
$(x,h(x))$, $(g_1(x),h(g_1(x)))$ and $(g_2(x),h(g_2(x)))$ are collinear.
Here $g_1(x)=blah$ and
$g_2(x)=blah$ which captures the condition that each pair of halving
lines form an angle of $\pi/3$.  [Informally, the continuity argument in
the proof of \lemref{construction} is equivalent to the observation
that, if the sequence of points $\langle(-\infty,h(-\infty)),
(g_1(-\infty),h(g_1(-\infty)), (g_2(-\infty),h(g_2(-\infty))\rangle$ form
a right (respectively left) turn then the points $\langle(\infty,h(\infty)),
(g_1(\infty),h(g_1(\infty)), (g_2(\infty),h(g_2(\infty))\rangle$ form a
left (respectively right) turn, so there must be some
$x\in(-\infty,\infty)$ such that 
$(x,h(x))$, $(g_1(x),h(g_1(x)))$ and $(g_2(x),h(g_2(x)))$ are
collinear.]

The algorithm we describe is recursive, and the recursive subproblems
it solves are slightly more general.  Given a set $S^*$ of $m$ lines,
two $x$-coordinates $x_1$ and $x_2$ and three integers $k$, $k_1$ and
$k_2$, the algorithm finds an $x$-coordinate $x\in[x_1,x_2]$ such that
the three points $(x,h_k(x))$, $(g_1(x),h_{k_1}(g_1(x)))$ and
$(g_2(x),h_{k_2}(g_2(x)))$ are collinear.  Such a value $x$ is
guaranteed \emph{a priori} to exist.  Note that, for our initial
recursive call we set $x_1=-\infty$, $x_2=\infty$, and
$k=k_1=k_2=\ceil{n/2}$.
 
To implement a single step of our algorithm, we create a set $X$ of
$x$-coordinates that initially contains the values $x_1$ and $x_2$.
Next we add to $X$ an additional $O(1)$ values so that between any
consecutive pair of $x$-coordinates in $X$, the arrangement of our $m$
lines contains at most $\delta m^2/40$ vertices, for some constant
$\delta$ to be specified later.  These points can be found in $O(m)$
time using the algorithm of Chazelle \cite{X}.  Finally, for each
value $x\in X$ we add the values $g_1^{-1}(x)$ and $g_2^{-1}(x)$ to
$X$.  

Now, in $O(m)$ time we can determine two consecutive elements $x_1'$
and $x_2'$ of $X$ such that $x_1',x_2'\in[x_1,x_2]$ and a solution to our
problem lies in the interval $[x_1',x_2']$.  Consider the trapezoid
$T$ whose four corners are given by $(x_1',h_{k\pm\ceil{\delta
m/2}}(x_1'))$ and $(x_2',h_{k\pm\ceil{\delta m/2}}(x_2'))$.  A simple
argument \cite[Lemma~X]{lms94} shows that this trapezoid intersects at
most $\delta m$ lines of $S^*$ and that the $k$ level of $S^*$ does
not intersect the top or bottom sides of this trapezoid.  Similarly,
there are trapezoids $T_1$ and $T_2$ defined by the four points 
$(g_1(x_1'),h_{k_1\pm\ceil{\delta m/2}}(g_1(x_1')))$ and
$(g_1(x_2'),h_{k_1\pm\ceil{\delta m/2}}(g_1(x_2')))$ and the four
points
$(g_2(x_1'),h_{k_1\pm\ceil{\delta m/2}}(g_2(x_1')))$ and
$(g_2(x_2'),h_{k_1\pm\ceil{\delta m/2}}(g_2(x_2')))$, respectively.
Neither $T_1$ nor $T_2$ intersect more than $\delta m$ lines in $S^*$
and the $k_1$-level (respectively $k_2$-level) of $S^*$ does not
intersect the top or bottom sides of $T_1$ (respectively $T_2$).

Altogether, this means that there are at least $m-3\delta m$ lines in
$S^*$ that do not intersect any of the trapezoids $T$, $T_1$ or $T_2$.
When we recurse, we discard these lines, set $x_1=x_1'$, $x_2=x_2'$,
and subtract from $k$ (respectively $k_1$ and $k_2$) the number of
discarded lines that pass below $T$ (respectively $T_1$ and $T_2$).
Each iteration of the algorithm takes time linear in the size of $S^*$
and decreases the size of $S^*$ by a factor of $3\delta$.  Thus, the
running time of this algorithm is $O(n)$, provided that we select
$\delta < 1/3$.
\end{proof}

\begin{lem}\lemlabel{c}
If $4\pi/3\le \alpha < 2\pi$ then $f^2_\alpha(n) = n/2$ and a point
$p$ such that $D_\alpha(p)\ge n/2$ can be found in $O(n)$ time.
\end{lem}

\begin{proof}
For the lower bound, consider the three halving lines whose existence
is given by \lemref{construction}.  These three halving lines
naturally define six $\pi/3$-wedges.  Observe that if we take $p$ to
be the common intersection point of the three halving lines then any
$\alpha$-wedge that contains $p$ contains at least 3 consecutive
wedges and therefore contains at least $\ceil{n/2}$ points of $S$
(because it contains a closed halfplane bounded by one of the
halving lines).  Therefore, $f^2_\alpha(n)\ge \ceil{n/2}$, and the
point $p$ such that $D_\alpha(p)\ge \ceil{n/2}$ can be found in $O(n)$
time using \lemref{algorithm}.

 
For the upper bound, we consider a point set in which the points have
been clustered into two groups of size $\floor{n/2}$ and $\ceil{n/2}$.
Each of the two groups is contained in a unit ball and the distance
between the two groups is very large, say $r$.  Now, observe that
any point $p$ must be at distance $r/2$ from at least one of the
two groups.  This means that, if $r$ is sufficiently large, then
there exists a $(2\pi-\alpha)$-wedge whose apex is $p$ and that
contains this group in its interior.  The complementary $\alpha$-wedge
contains $p$ and does not contain any points of this group.
Therefore, $D_\alpha(p)\le \ceil{n/2}$.  Since $p$ was chosen
arbitrarily, we conclude that $f^2_\alpha(n)\le \ceil{n/2}$.
\end{proof}

Lemmata \ref{lem:a}, \ref{lem:b} and \ref{lem:c} immediately imply
\thmref{twod-result}.

\section{Conclusions}

\begin{enumerate}
\item Given a point set $S$ in $\mathbb{R}^2$, what is the complexity
of finding a point $p\in\mathbb{R}^2$ that maximizes $D_\alpha(p,S)$?
For $\alpha=\pi$, i.e., halfplane depth, Chan has given an $O(n\log
n)$ time algorithm.

\item Prove a theorem in $d$ dimensions.

\end{enumerate}



\bibliographystyle{plain}
\bibliography{wedge}





\end{document}
