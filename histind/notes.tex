\documentclass{article}

\title{Notes on History-Independent Search Trees}
\author{Stefan Langerman \and Pat Morin}

\begin{document}
\maketitle

Here's what's known.

\begin{enumerate}
\item Snyder shows that any history-independent deterministic search
tree requires $\Omega(\sqrt{n})$ for some operations.  

\item An italian guy gave a weakly history-independent implementation
of 2-3 trees.  

\item The ISAAC guys characterize strongly history-independent
structures and show that the representation-transition graphs of SHI
data structures for reversible abstract data types have a very special
form; their representation is unique once an initial random choice
(e.g., a hash function) is made.

\end{enumerate}

Here's what we found out:
\begin{enumerate}

\item No SHI dictionary has worst-case $o(\sqrt{n})$ time for all
operations.  This follows from 1 and 3 above.  (What can we say about
WHI data structures?)

\item There is a SHI search tree with $O(\log n)$ expected time per
operation.  This data structure is a treap or skiplist where
priorities, respectively heights, are obtained by hashing the data
values.  Interestingly, these data structures also have the dynamic
finger property.

\item There is a WHI search tree that has $O(\log n)$ worst-case
search time and $O(\log n)$ expected amortized update time.  (Can we
deamortize?  Can we make it SHI? Is there a data structure with
$O(\log n)$ [or $O(1)$] worst-case insertion time given a pointer and
$O(\log n)$ expected search time?)

\item There is a SHI randomized B-Tree that uses $O(\log_B n)$
expected I/Os per operation.  This is a B-treap.  (Is there also a
B-skiplist?  What about deterministic data structures -- Can we
achieve $O(\sqrt{n}/B)$ I/Os per operation?)

\end{enumerate}

\end{document}
