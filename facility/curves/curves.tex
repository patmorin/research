\documentclass[charterfonts,lotsofwhite]{patmorin}
\usepackage{graphicx,ipe}
\usepackage{amsfonts}
\usepackage{amsthm}

 
%\usepackage{amsthm}

\newcommand{\centeripe}[1]{\begin{center}\Ipe{#1}\end{center}}
\newcommand{\comment}[1]{}

\newcommand{\centerpsfig}[1]{\centerline{\psfig{#1}}}

\newcommand{\seclabel}[1]{\label{sec:#1}}
\newcommand{\Secref}[1]{Section~\ref{sec:#1}}
\newcommand{\secref}[1]{\mbox{Section~\ref{sec:#1}}}

\newcommand{\alglabel}[1]{\label{alg:#1}}
\newcommand{\Algref}[1]{Algorithm~\ref{alg:#1}}
\newcommand{\algref}[1]{\mbox{Algorithm~\ref{alg:#1}}}

\newcommand{\applabel}[1]{\label{app:#1}}
\newcommand{\Appref}[1]{Appendix~\ref{app:#1}}
\newcommand{\appref}[1]{\mbox{Appendix~\ref{app:#1}}}

\newcommand{\tablabel}[1]{\label{tab:#1}}
\newcommand{\Tabref}[1]{Table~\ref{tab:#1}}
\newcommand{\tabref}[1]{Table~\ref{tab:#1}}

\newcommand{\figlabel}[1]{\label{fig:#1}}
\newcommand{\Figref}[1]{Figure~\ref{fig:#1}}
\newcommand{\figref}[1]{\mbox{Figure~\ref{fig:#1}}}

\newcommand{\eqlabel}[1]{\label{eq:#1}}
\newcommand{\eqref}[1]{(\ref{eq:#1})}

\newtheorem{thm}{Theorem}{\bfseries}{\itshape}
\newcommand{\thmlabel}[1]{\label{thm:#1}}
\newcommand{\thmref}[1]{Theorem~\ref{thm:#1}}

\newtheorem{lem}{Lemma}{\bfseries}{\itshape}
\newcommand{\lemlabel}[1]{\label{lem:#1}}
\newcommand{\lemref}[1]{Lemma~\ref{lem:#1}}

\newtheorem{cor}{Corollary}{\bfseries}{\itshape}
\newcommand{\corlabel}[1]{\label{cor:#1}}
\newcommand{\corref}[1]{Corollary~\ref{cor:#1}}

\newtheorem{obs}{Observation}{\bfseries}{\itshape}
\newcommand{\obslabel}[1]{\label{obs:#1}}
\newcommand{\obsref}[1]{Observation~\ref{obs:#1}}

\newtheorem{assumption}{Assumption}{\bfseries}{\rm}
\newenvironment{ass}{\begin{assumption}\rm}{\end{assumption}}
\newcommand{\asslabel}[1]{\label{ass:#1}}
\newcommand{\assref}[1]{Assumption~\ref{ass:#1}}

\newcommand{\proclabel}[1]{\label{alg:#1}}
\newcommand{\procref}[1]{Procedure~\ref{alg:#1}}

\newtheorem{rem}{Remark}
\newtheorem{op}{Open Problem}

\newcommand{\etal}{\emph{et al}}

\newcommand{\voronoi}{Vorono\u\i}
\newcommand{\ceil}[1]{\left\lceil #1 \right\rceil}
\newcommand{\floor}[1]{\left\lfloor #1 \right\rfloor}



\title{\MakeUppercase{Succinct Data Structures for Approximating
Convex Functions} \\ \MakeUppercase{with Applications}%
     \thanks{This research was partly supported by NSERC.}}
\author{Prosenjit Bose\thanks{School of Computer Science, Carleton
	University, \texttt{\{jit,morin\}@scs.carleton.ca}} \and
  Luc Devroye\thanks{School of Computer Science, McGill University,
  	\texttt{luc@cgm.cs.mcgill.ca}} \and 
  Pat Morin\footnotemark[2]}
\date{}

\newcommand{\fw}{\mathrm{FW}}
\newcommand{\eps}{\varepsilon}
\newcommand{\Real}{\mathbb{R}}
\newcommand{\E}{\mathbf{E}}
\newcommand{\dlt}{2\eps}

\begin{document}
\maketitle

\begin{abstract}
We study data structures for providing $\epsilon$-approximations 
of convex functions whose
slopes are bounded. Since the queries are efficient in these 
structures requiring
only $O(\log(1/\eps)+\log\log n)$ time, 
we explore different applications 
of such data structures to efficiently solve
problems in clustering and facility location. Our data structures are succinct
using only $O((1/\eps)\log^2(n))$ bits of storage. 
We show that this is optimal by
providing a matching lower bound showing that any data structure 
providing such an
$\epsilon$-approximation requires at least $\Omega((1/\eps)\log^2(n))$ bits of storage.
\end{abstract}

\section{Introduction}

We consider the problem of approximating convex functions of 
one variable whose slopes are bounded.  
We say that a non-negative number $y$ is an
$\eps$-approximation to a non-negative number $x$ if $(1-\eps)x\le y
\le x$.\footnote{This definition is a bit more one-sided that the
usual definition, which allows any $y$ such that $|x-y|\le \eps
x$.}  We say that a function $g$ is an $\eps$-approximation to a
function $f$ if $g(x)$ is an $\eps$-approximation to $f(x)$ for all
$x$ in the domain of $f$.

Let $f:\Real\rightarrow \Real^+$ be a convex function that is
non-negative everywhere.  In this paper we show that, if the absolute
value of the slope of $f$ is bounded above by $n$, then there exists a
piecewise-linear function $g$ that $\eps$ approximates $f$ at all
points $x$ except where the slope of $f$ is small (less than 1) and
that consists of $O(\log_E n)$ pieces, where $E=1/(1-\eps)$.  The
function $g$ can be computed in $O(K\log_E n)$ time, where $K$ is the
time it takes to evaluate expressions of the form $\sup\{x: f'(x)\le
t\}$ and $f'$ is the first derivative of $f$.  Once we have computed
the function $g$, we can store the pieces of $g$ in an array sorted by
$x$ values so that we can evaluate $g(x)$ for any query value $x$ in
$O(\log\log_E n)$ time.  Since we are interested in the joint
complexity as a function of $\eps < 1/2$ and $n \ge 10$, it is worth
noting that $log_E n = \Theta ( (1/\eps) \log n)$ and thus that
$\log\log_E n = \Theta (\log (1/\eps) + \log \log n)$.

As an application of these results, we consider functions defined by
sums of Euclidean distances in $d$ dimensions and show that that they
can be approximated using the above results.  To achieve this, we use
a random rotation technique similar to the method of random
projections \cite{i01}.  We show that the sum of Euclidean distances
from a point to a set of $n$ points can be closely approximated by
many sums of Manhattan distances from the point to the set.  This
technique is very simple and of independent interest.

The remainder of the paper is organized as follows.  \secref{approx}
presents our result on approximating convex functions using few linear
pieces.  \secref{ds} discusses how these results can be interpreted in
terms of data structures for approximating convex functions.
\secref{lower-bound} gives lower bounds on the space complexity of
approximating convex function.  \secref{applications} describes
applications of this work to facility location and clustering
problems.  Finally, \secref{conclusions} summarizes and concludes the
paper.

\section{Approximating Convex Functions}\seclabel{approx}

Let $h(x)=c+|nx|$, for some $c,n\ge 0$.  Then, it is clear that the
function $g$ such that $g(x)=c+(1-\eps)|nx|$ is an
$\eps$-approximation of $h$.  Furthermore, $g$ is an
$\eps$-approximation for any function $h_2$ such that $g(x)\le
h_2(x)\le h(x)$ for all $x\in\Real$. (see \figref{trivial}).  This
trivial observation is the basis of our data structure for
approximating convex functions.

\begin{figure}
\centeripe{trivial}
\caption{The function $g$ is an approximation of $h$ and
of $h_2$.}\figlabel{trivial}
\end{figure}

Let $f$ be a non-negative convex function and let $f'$ be the first
derivative of $f$.  Assume that $f'(x)$ is defined for all but a
finite number of values of $x$ and that $|f'(x)|\le n$ for all $x$ in
the domain of $f'$.  For convenience, we define the \emph{right
derivative} $f^*(x)$ as follows: If $f'(x)$ is defined, then
$f^*(x)=f'(x)$.  Otherwise, $f^*(x)=\lim_{\delta\rightarrow 0+}
f'(x+\delta)$.

Let $a$ be the largest value at which the slope of $f$ is at most
$-(1-\eps)n$, i.e., 
\[
a=\max\{x:f^*(x)\le -(1-\eps)n \} \enspace .
\]
(Here, and throughout, we use the convention that
$\max\emptyset=-\infty$ and $\min\emptyset=\infty$.)  Similarly, let
$b=\min\{x:f^*(x)\ge (1-\eps) n\}$.  Then, from the above discussion,
it is clear that the function
\begin{equation}
 g(x) = \left\{\begin{array}{ll}
                 f(a) + (1-\eps)(x-a)n & \mbox{if $x\le a$} \\
                 f(b) + (1-\eps)(b-x)n & \mbox{if $x\ge b$} \\
                 f(x)   & \mbox{otherwise} \eqlabel{approx}
 \end{array}\right.
\end{equation}
is an $\eps$-approximation of $f$ (see \figref{approx}).

\begin{figure}
\centeripe{approx}
\caption{The function $g$ is a $(1-\eps)$ approximation of
$f$.}\figlabel{approx}
\end{figure}

\Eqref{approx} tells us that we can approximate $f$ by using two
linear pieces and then recursively approximating $f$ in the range
$(a,b)$.  However, in the range $(a,b)$, $f^*$ is in the range
$(-(1-\eps)n, (1-\eps)n)$.  Therefore, if we recurse $\lceil \log_E
n\rceil$ times, we obtain a function $g$ with $O(\log_E n) =
O((1/\eps) \log n)$ linear pieces that approximates $f$ at all points
except possibly where $f^*$ is less than one.

\begin{thm}\thmlabel{approx}
Let $f$ and $f^*$ be defined as above.  Then there exists a piecewise-linear
function $g$ with $O(\log_E n)$ pieces that is an
$\eps$-approximation to $f$ at all values except where $|f^*(x)|\le 1$.
\end{thm}

\section{Data Structures}\seclabel{ds}

In this section, we consider the consequences of \thmref{approx} in
terms of data structures for approximating convex functions.  By
storing the pieces of $g$ in an array sorted by $x$ values, we obtain
the following.

\begin{thm}
Let $f$ and $f^*$ be defined as in \secref{approx}.  Then there exists
a data structure of size $O((1/\eps) \log n)$ that can compute an
$\eps$-approximation to $f(x)$ in $O(\log (1/\eps) + \log \log n)$
time for any query value $x$ where $|f^*(x)|\ge 1$.
\end{thm}

Next, we consider a more dynamic model, in which the function $f$ is
updated over time.  In particular, we consider the following
operations that are applied to the initial function $f(x)=0$, for all
$x\in\Real$.

\begin{enumerate}
\item $\textsc{Query}(x)$: Return an $\eps$-approximation to
  $f(x)$.

\item $\textsc{Insert}(a)$: Increase the slope of $f$ by 1 in the
  range $(a,\infty)$, i.e., set $f(x)\gets f(x)+x-a$ for all $x\in
  [a,\infty)$.

\item $\textsc{Delete}(x)$: Decrease the slope of $f$ by 1 in the
  range $(x,\infty)$.  In order to maintain convexity, the number of
  calls to $\textsc{Delete}(x)$ may not exceed the number of calls to
  $\textsc{Insert}(x)$ for any value of $x$.
\end{enumerate}

Note that a sequence of \textsc{Insert} and \textsc{Delete} operations
can only produce a monotonically increasing function $f$ whose slopes
are all integers.  This is done to simplify the exposition of the data
structure.  If an application requires that $f$ be allowed to decrease
and increase then two data structures can be used and their results
summed.  \comment{Also, if non-integer slopes are required then the data
structure and algorithms are easily modified to support an additional
parameter $\delta$, $0<\delta\le 1$ to the \textsc{Insert} and
\textsc{Delete} procedures that specifies the amount by which the
slope is to increased or decreased, respectively.}

The function $f$ has some number $m$ of breakpoints, where the slope
of $f$ changes.  We store these breakpoints in a balanced search tree
$T$, sorted by $x$-coordinate.  With each breakpoint $x$, we also
maintain the value $\Delta(x)$ by which the slope of $f$ increases at
$x$.  In addition, we link the nodes of $T$ in a doubly-linked list,
so that the immediate successor and predecessor of a node can be found
in constant time.  It is clear that $T$ can be maintained in $O(\log
n)$ time per operation using any balanced search tree data structure.

In addition to the search tree $T$, we also maintain an array $A$ of
size $O((1/\eps) \log n)$ that contains the piecewise linear
approximation of $f$.  The $i$th element in this array contains the
value $x_i$ such that $x_i=\min\{x:f^*(x) \ge E^i \}$, a pointer to
the node in $T$ that contains $x_i$, and the values of $f(x_i)$ and
$f^*(x_i)$, i.e., the value of $f$ at $x_i$ and slope of $f$ at $x_i$.
To update this array during an \textsc{Insert} or \textsc{Delete}
operation, we first update the values of $f(x_i)$ and $f^*(x_i)$ for
each $i$.  Since there are only $O((1/\eps) \log n)$ array entries,
this can be done in $O((1/\eps) \log n)$ time.

Next, we go through the array again and check which values of $x_i$
need to be changed (recall that $x_i=\min\{x:f^*(x) \ge E^i \}$).
Note that, since \textsc{Insert} or \textsc{Delete} can only change
the value of $f^*(x)$ by 1, if the value of $x_i$ changes then it
changes only to its successor or predecessor in $T$.  Since the nodes
of $T$ are linked in a doubly-linked list, and we store the values of
$f(x_i)$ and $f^*(x_i)$ we can detect this and update the value of
$x_i$, $f(x_i)$ and $f^*(x_i)$ in constant time.  Therefore, over all
array entries, this takes $O((1/\eps) \log n)$ time.

To evaluate an approximation to $f(x)$, we do a binary search on $A$
to find the index $i$ such that $[x_i,x_{i+1})$ contains $x$ and then
output $f(x_i) + (x-x_i)f^*(x_i)$.  By the choice of $x_i$, this is a
$\eps$-approximation to $f(x)$.  We have just proven the
following:

\begin{thm}
There exists a data structure of size $O(n)$ that supports the
operations \textsc{Insert}, \textsc{Delete} in $O((1/\eps) \log n)$
time and \textsc{Query} in $O(\log (1/\eps) + \log \log n)$ time,
where $n$ is the maximum slope of the function $f$ being maintained.
\end{thm}

\section{A Lower Bound on Storage}\seclabel{lower-bound}

In this section we prove an $\Omega((1/\eps)\log^2 n)$
lower bound on the number of bits required by any data structure that
provides an $\eps$-approximation for convex functions.  The idea
behind our proof is to make $m=\Theta((1/\eps) \log n)$ choices from a
set of $n$ elements.  We then encode these choices in the form of a
convex function $f$ whose slopes are in $[0,n]$.  We then show that
given a function $g$ that is an $\eps$-approximation to $f$ we can
recover the $m=\Theta((1/\eps) \log n)$ choices.  Therefore, any data
structure that can store an $\eps$-approximation to convex functions
whose slopes lie in $[0,n]$ must be able to encode $n\choose m$
different possibilities and must therefore store $\Omega((1/\eps)
\log^2 n)$ bits in the worst case.

Let $x_1,\ldots,x_n$ be an increasing sequence where $x_1=0$ and each
$x_i$, $2\le i\le n$ satisfies
\begin{equation}
x_{i-1}+(1-\eps)\left(\frac{x_i-x_{i-1}}{1-\dlt}\right) 
 > x_{i-1}+\frac{x_i-x_{i-1}}{1-\dlt} \enspace , \eqlabel{increasing}
\end{equation}
which is always possible since $(1-\epsilon)/(1-\dlt)>1$.

Let $p_1,\ldots,p_m$ be any increasing sequence of $m=\lfloor\log_{E'}
n\rfloor$ integers in the range $[1,n]$, where $E'=1/(1-2\eps)$.  We
construct the function $f$ as follows:
\begin{enumerate}
\item For $x\in [-\infty,0)$, $f(x)=0$.
\item For $x\in (x_{p_i},x_{p_{i+1}})$, $f(x)$ has slope $1/(1-\dlt)^i$.
\item For $x>p_m$, $f(x)$ has slope $n$.
\end{enumerate}

The following lemma, illustrated in \figref{encoding} allows us to
decode the values of $p_1,\ldots,p_m$ given an
$\eps$-approximation to $f$.

\begin{figure}
\centeripe{encoding}
\caption{An illustration of \lemref{encoding}.}\figlabel{encoding}
\end{figure}

\begin{lem}\lemlabel{encoding}
  For the function $f$ defined above and for any $i$ such that $i=p_j$
  for some $1\le j\le m$, 
  \[  (1-\eps)f(x_{p_{j+1}}) > f(x_{p_j}) + \frac{x_{p_{j+1}}-x_{p_j}}{(1-\dlt)^{j-1}} 
  \enspace .\]
\end{lem}

\begin{proof}
  The lemma follows (with some algebra) from \eqref{increasing}.
\end{proof}

Suppose that $g$ is an $\eps$-approximation to $f$, i.e, for all
$x\in\Real$, $g(x)$ satisfies $(1-\eps)f(x)\le g(x)\le f(x)$.  Then
\lemref{encoding} can be used to recover the values of
$p_1,\ldots,p_m$ from $g$.  Suppose, that we have already recovered
$p_1,\ldots,p_{j}$ and that we now wish to recover $p_{j+1}$.  Note
that, since we have recovered $p_1,\ldots,p_{j}$ we can compute the
exact value of $f(x_{p_j})$.  We then evaluate $g(x_{p_j}+1)$,
$g(x_{p_j}+2)$, and so on until encountering a value $k$ such that
\[
   g(x_{p_j}+k)>f(x_{p_j})+\frac{x_{p_j}+k-x_{p_j}}{(1-\dlt)^j}
\]
\lemref{encoding} then guarantees that $p_{j+1}=p_{j}+k-1$.  In this
way, we can reconstruct the entire function $f$ and recover the values
of $p_1,\ldots,p_m$.

Although in the above discussion the slopes used in the construction
of $f$ are not always integral it is clear that carefully rounding
values appropriately would yield the same results using only integer
valued slopes.  Since we can encode $n\choose m$ different choices of
$p_1,\ldots,p_m$ in this manner and 
$\log {{n}\choose{m}} =\Omega((1/\eps)\log^2n)$, 
we conclude the following:

\begin{thm}
Any data structure that can represent an $\eps$-approximation to any
convex function whose slopes are integers in the range $[0,n]$ must
use $\Omega((1/\eps)\log^2 n)$ bits of storage in the worst
case.
\end{thm}

\begin{rem}
Some readers may complain that the function used in our lower bound
construction uses linear pieces whose lengths are exponential in $n$.
However, one should take into account that the endpoints of these
pieces have $x$-coordinates that are integral powers of 2 and they can
therefore be encoded in $O(\log n)$ bits each using, e.g., a floating
point representation.
\end{rem}

\begin{rem}
Another easy consequence of \lemref{encoding} is that any piecewise
linear function that $\eps$-approximates $f$ has $\Omega((1/\eps) \log
n)$ pieces.
\end{rem}

\section{Applications}\seclabel{applications}

Next, we consider applications of our approximation technique for
convex functions to the problem of approximating sums of distances in
$d$ dimensions.  Let $S$ be a set of $n$ points in $d$ dimensions.
The \emph{Fermat-Weber weight} of a point $q\in\Real^d$ is
\[
   \fw(p) = \sum_{p\in S} \|pq\| \enspace ,
\]
where $\|pq\|$ denotes the distance between points $p$ and $q$. Of
course, different definitions of distance (e.g., Euclidean distance,
Manhattan distance) yield different Fermat-Weber weights.

\subsection{The 1-dimensional case}

One setting in which distance is certainly well defined is in one
dimension.  In this case,
\[ 
\|pq\| = |p-q| \enspace ,
\]
so the Fermat-Weber weight of $x$ is given by
\[
   \fw(x) = f(x) = \sum_{y\in S} |x-y|  \enspace .
\]
Note that the function $f$ is convex (it is the sum of $n$ convex
functions) and has slopes bounded below by $-n$ and above by $n$, so
it can be approximated using the techniques \secref{ds}.  Furthermore,
adding or removing a point $p$ to/from $S$ decreases the slope of $f$
by 1 in the range $(-\infty,p)$ and increases the slope of $f$ by 1 in
the range $(p,\infty)$, so the dynamic data structure of the previous
section can be used to maintain an $\eps$-approximation of $f$ in
$O(\log_E n)= O((1/\eps) \log n)$ time per update.

Given the set $S$, constructing the $\eps$-approximation for $f$
can be done in $O(n/\eps)$ time by a fairly straightforward
algorithm.  Using a linear-time selection algorithm, one finds the
elements of $S$ with ranks $\lfloor\eps n/2\rfloor$ and
$\lceil(1-\eps/2) n\rceil$.  These are the values $a$ and $b$ in
\eqref{approx}.  Once this is done, the remaining problem has size
$(1-\eps) n$ and is solved recursively.  Although some care is
required to compute the values $f(a)$ and $f(b)$ at each stage, the
deatails are not difficult and are left to the interested reader.

\begin{rem} 
A general result of Agarwal and Har-Peled \cite{ah01} implies that the
Fermat-Weber weight of points in one dimension can actually be
$\eps$-approximated by a piecewise-linear function with $O(1/\eps)$
pieces.  However, it is not clear how easily this approach can be made
dynamic to handle insertion and deletions of points.
\end{rem}

\subsection{The Manhattan case}

The Manhattan distance between two points $p$ and $q$ in $\Real^d$ is
\[
\|pq\|_1 = \sum_{i=1}^d |p_i-q_i| \enspace ,
\]
where $p_i$ denotes the $i$th coordinate of point $p$.  We simply
observe that Manhattan distance is the sum of $d$ one-dimensional
distances, so the Fermat-Weber weight under the Manhattan distance can
be approximated using $d$ one-dimensional data structures.

\subsection{The Euclidean case}

The Euclidean distance between two points $p$ and $q$ in $\Real^d$ is
\[
\|pq\|_2 = \left(\sum_{i=1}^d (p_i-q_i)^2\right)^{1/2} \enspace .
\]

A general technique used to approximate Euclidean distance is to use a
polyhedral distance function, in which the unit sphere is replaced
with a polyhedron that closely resembles a sphere.  For example, the
Manhattan distance function is a polyhedral distance function in which
the unit sphere is replaced with a unit hypercube.  Although this
technique works well when $d$ is small, such metrics generally require
a polyhedron with a number of vertices that is exponential in $d$,
making them poorly suited for high dimensional applications.

Another technique, that works well when $d$ is very large (greater
than $\log n$), and for most distance functions, is that of random
projections \cite{i01}. Here, a random $O(\log n)$-flat is chosen and
the points of $S$ are projected orthogonally onto this flat.  With
high probability, all interpoint distances are faithfully preserved
after the projection, so the problem is reduced to one in which the
dimension of the point set is $O(\log n)$.  The difficulty with this
approach, when using Euclidean distances, is that sums of Euclidean
distances are difficult to deal with even when $d=2$ \cite{b88}, thus
the reduction in dimension does not help significantly.

Here we present a third approach that combines both of these
techniques and adds two new twists: (1)~we obtain a polyhedral metric
as the sum of several Manhattan metrics and (2)~our polyhedron is
random.  The first twist allows us to apply approximate data
structures for one-dimensional convex functions while the second
allows us to achieve approximation guarantees using an a number of
vertices that increases only linearly with $d$.

Let $f(p)$ denote the Fermat-Weber weight of $p$ under the Euclidean
distance function.  Choose $k$ independent random orientations of the
coordinate axes.  Let $f_i(p)$ denote the Fermat-Weber weight of $p$
under the Manhattan distance function after rotating the axes
according to the $i$th random orientation.  Then $f_i(p)$ may take on
any value in the range $[f(p),\sqrt df(p)]$.  In particular, $f_i(p)$
has an expected value 
\[
   \E[f_i(p)]=c_df(p) \enspace ,
\]
where $c_d$ is a constant, dependent only on $d$, whose value is
derived in \appref{luc}.  Consider the function
\[
  g(p) = \frac{1}{kc_d}\times\sum_{i=1}^k f_i(p) 
\]
that approximates the Fermat-Weber weight under Euclidean distance.

\begin{lem}
$\Pr\{|g(p)-f(p)| \ge \eps f(p)\} = \exp(-\Omega(\eps^2k))$
\end{lem}

\begin{proof}
The value of $g(p)$ is a random variable whose expected value is
$f(p)$ and it is the sum of $k$ independent random variables, all of
which are in the range $[f(p),\sqrt d f(p)]$.  Applying Hoeffding's
inequality \cite{h63} immediately yields the desired result.
\end{proof}

In summary, $g(p)$ is an $\eps$-approximation of $f(p)$ with
probability $1-e^{-\Omega(\eps^2k)}$.  Furthermore, $g(p)$ is the sum
of $k$ Fermat-Weber weights of points under the Manhattan distance
function.  Each of these Manhattan distance functions is itself a sum
of $d$ Fermat-Weber weights in 1 dimension.  These 1-dimensional
Fermat-Weber weights can be approximated using the results of
\secref{ds} or the results of Agarwal and Har-Peled \cite{ah01}.

\subsection{Clustering and Facility Location}

Bose \etal\ \cite{bmm02} describe data structures for approximating
sums of distances.  They show how to build a data structure in
$O(n\log n)$ time that can $(1-\eps)$-approximate the Fermat-Weber
weight of any point in $O(\log n)$ time.  However, the constants in
their algorithms depend exponentially on the dimension $d$.

The same authors also give applications of this data structure to a
number of facility-location and clustering problems, including
evaluation of the Medoid and AverageDistance clustering measures, the
Fermat-Weber problem, the constrained Fermat-Weber problem, and the
constrained obnoxious facility-location problem.  All of these
applications also work with the data structure of \secref{ds}, many
with improved running times.  A summary of these results is given in
\tabref{applications}.


\begin{table}
\begin{minipage}{\textwidth}
\begin{center}\begin{tabular}{l|cccc}
Problem      & Exact solution & Previous $\eps$-approx. & Ref. & New $\eps$-approx. \\ \hline
Average distance  & $O(n^2)$ & $O(n)^a$ $O(n\log n)$ & 
    \cite{i99,bmm02} & $O(n\log\log n)$ \\
Medoid (1-Median) & $O(n^2)$ & $O(n)^a$ $O(n\log n)$ & 
    \cite{i99,bmm02} & $O(n\log\log n)$ \\
Discrete Fermat-Weber & $O(n^2)$ & $O(n)^a$ $O(n\log n)$ & 
    \cite{i99,bmm02} & $O(n\log\log n)$ \\
Fermat-Weber & -- & $O(n)^b$ $O(n\log n)$ & 
    \cite{i99,bmm02} & $O(n)$ \\
Constrained Fermat-Weber & $O(n^2)$ & $O(n)^b$ $O(n\log n)$ & 
    \cite{i99,bmm02} & $O(n)$ \\
Constrained OFL & $O(n^2)$ & $O(n)^a$ $O(n\log n)$ & 
    \cite{i99,k97,bmm02} & $O(n\log\log n)$ 
\end{tabular}\end{center}
\footnotetext[1]{Refers to a randomized algorithm that
  outputs a  $(1-\epsilon)$-approximation with constant probability.}
\footnotetext[2]{Refers to a randomized algorithm that outputs a
  $(1-\epsilon)$-approximation with high probability, i.e., with probability
  $1-n^{-c}$, for some $c>0$.}
\end{minipage}
\caption{Applications of the data structure for evaluating the Fermat-Weber
  weights of points under the Euclidean distance function.}
\tablabel{applications}
\end{table}



\section{Summary and Conclusions}\seclabel{conclusions}

We have given static and dynamic data structures for approximating
convex functions of one variable whose slopes are bounded.  These data
structures have applications to problems involving sums of distances
in $d$ dimensions under both the Manhattan and Euclidean distance
functions.  In developing these applications we have arrived at a
technique of independent interest, namely that of approximating
Euclidean distance as the sum of several Manhattan distances under
several different orientations of the coordinate system.

\bibliographystyle{plain}
\bibliography{curves}


\appendix
\section{The value of $\mathbf{c_d}$}\applabel{luc}

The value of $c_d$ is given by
\[
  c_d = \E \left[\sum_{i=1}^d|X_i| \right] \enspace ,
\]
where $(X_1,\ldots,X_d)$ is a point taken from the uniform
distribution on the surface of the unit ball in $\mathbb{R}^d$.
We observe that $(X_1^2,\ldots,X_d^2)$ is distributed as 
\[
\left(\frac{N_1^2}{N^2},\ldots,\frac{N_d^2}{N^2}\right) \enspace ,
\]
where $N^2=\sum_{i=1}^d N_i^2$ and $(N_1,\ldots,N_d)$ are i.i.d.\
$\mathrm{normal}(0,1)$.  Clearly,
\[
\frac{N_1^2}{N^2} = \frac{N_1^2}{N_1^2+\sum_{i=2}^dN_i^2}
  \stackrel{\mathcal{L}}{=} \frac{G(\frac{1}{2})}{G(\frac{1}{2})+G(\frac{d-1}{2})}
\]
where $G(\frac{1}{2})$, and $G(\frac{d-1}{2})$ are independent
$\mathrm{gamma}(\frac{1}{2})$ and $\mathrm{gamma}(\frac{d-1}{2})$
random variables, respectively.  Thus, $N_1^2/N$ is distributed as a
$\mathrm{beta}(\frac{1}{2},\frac{d-1}{2})$ random variable,
$\beta(\frac{1}{2},\frac{d-1}{2})$.  We have:
\begin{eqnarray*}
\E\left[\sum_{i=1}^d |X_i|\right]
 & = & d\,\E\left[\sqrt{\beta\left(\frac{1}{2},\frac{d-1}{2}\right)}\right] \\
 & = & d\int_0^1\frac{x^{\frac{1}{2}-1}(1-x)^{\frac{d-1}{2}-1}}
                    {B(\frac{1}{2},\frac{d-1}{2})}\cdot\sqrt{x} \quad dx\\
 & = & d\cdot\frac{B(1,\frac{d-1}{2})}{B(\frac{1}{2},\frac{d-1}{2})}\\
\comment{ & = & d\cdot 
       \frac{\Gamma(\frac{d-1}{2})}
            {\Gamma(\frac{d+1}{2})} \cdot
       \frac{\Gamma(\frac{d}{2})}
            {\Gamma(\frac{1}{2})\cdot\Gamma(\frac{d-1}{2})} \\
 & = & \frac{d\Gamma(\frac{d}{2})}
            {\Gamma(\frac{1}{2})\cdot\Gamma(\frac{d+1}{2})} \\ 
 & = & \frac{2\Gamma(\frac{d}{2}+1)}
            {\Gamma(\frac{1}{2})\cdot\Gamma(\frac{d+1}{2})} \\
}
 & = & \frac{2}{B(\frac{1}{2},\frac{d+1}{2})} \enspace , 
\end{eqnarray*}
where $B(a,b)$ is the beta function.

>From Mitrinovic \cite[p.~286]{m70}, we note:
\begin{eqnarray}
\frac{2}{B(\frac{1}{2},\frac{d+1}{2})} & \ge &
  \sqrt{\frac{d}{2}+\frac{1}{4}+\frac{1}{16d+32}}\cdot 
     \frac{1}{\Gamma(\frac{1}{2})} \\
 & = & 2\sqrt{\frac{d}{2}+\frac{1}{4}+\frac{1}{16d+32}}\cdot
      \frac{1}{\sqrt{\pi}} \\
 & \ge & \sqrt{\frac{2d+1}{\pi}} \enspace .
\end{eqnarray}

Furthermore,
\begin{eqnarray*}
\E\left[\sum_{i=1}^d|X_i|\right] & = & d\,\E[|X_1|] \\
  & \le & \frac{d+1}{\sqrt{\pi}\cdot\sqrt{\frac{d}{2}+
                       \frac{3}{4}+\frac{1}{16d+48}}} \\
  & \le & \frac{2(d+1)}{\sqrt{\pi}\cdot\sqrt{2d+3}} \\ 
  & \le & \sqrt\frac{2(d+1)}{\pi} \enspace . 
\end{eqnarray*}

In summary,
\[
\sqrt\frac{2d+1}{\pi} \le c_d = \frac{2\Gamma(\frac{d}{2}+1)}
                            {\sqrt{\pi}\cdot\Gamma(\frac{d+1}{2})}
 \le \sqrt{\frac{2(d+1)}{\pi}} \enspace .
\]


\end{document}
