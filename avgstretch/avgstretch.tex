\documentclass{patmorin}
\usepackage{amsthm,amsmath,graphicx}
\usepackage{pat}
%\usepackage{coffee4}

\usepackage{tikz,hyphenat}
\let\oldmarginpar\marginpar
% renew the \marginpar command to draw 
% a node; it has a default setting which 
% can be overwritten
\renewcommand{\marginpar}[2][rectangle,draw,fill=yellow,rounded corners,text width=2.21cm]{%
        \oldmarginpar{%
        \tikz \node at (0,0) [#1]{#2};}%
        }
\newcommand{\note}[1]{\marginpar{\raggedright\footnotesize\nohyphens{#1}}}

\DeclareMathOperator{\asf}{asf}
\DeclareMathOperator{\strf}{sf}
\DeclareMathOperator{\depth}{depth}
\DeclareMathOperator{\radius}{radius}
\DeclareMathOperator{\msst}{msst}
\newcommand{\mand}{\mathrm{,\,\,and\,}}
\newcommand{\oand}{\mathrm{,\,\,}}
\newcommand{\eps}{\varepsilon}

\title{\MakeUppercase{Average Stretch Factor: How Low Does It Go?}}
\author{Vida Dujmovi\'c, Pat Morin, and Michiel Smid}


\begin{document}
\begin{titlepage}
\maketitle
%\cofeAm{0.7}{0.38}{0}{5.5cm}{3.5in}

\begin{abstract}
  In a geometric graph, $G$, the \emph{stretch factor} between two
  vertices, $u$ and $w$, is the ratio of the shortest path from $u$ to $w$
  in $G$ to the Euclidean distance between $u$ and $w$.  The \emph{average
  stretch factor} of $G$ is the average stretch factor taken over all
  pairs of vertices in $G$.  We show that for any set, $V$, of $n$
  points in $\R^d$, there exists a geometric graph with vertex set $V$,
  that has $O(n)$ edges, and that has average stretch factor $1+ o_n(1)$.
  More precisely, the average stretch factor of this graph is $1+O((\log
  n/n)^{4d-1})$.  We complement this upper-bound with a lower bound
  which shows that, even in $\R^2$, there are point sets for which
  any graph with a linear number of edges has average stretch factor
  $1+\Omega(1/\sqrt{n})$.
\end{abstract}

\end{titlepage}

\section{Introduction}

A \emph{geometric graph} is a simple undirected graph whose vertex set
is made up of  points in $\R^d$.  The \emph{average stretch factor}
of a geometric graph, $G=(V,E)$, with vertex set $V\subset \R^d$ and
edge set $E$ is
\begin{equation}
    \asf(G) = \binom{|V|}{2}^{-1}\sum_{\{u,w\}\in\binom{V}{2}}\frac{\|uw\|_G}{\|uw\|} \eqlabel{asf}
\end{equation}
where $\|uw\|$ denotes the Euclidean distance between $u$ and $w$
and $\|uw\|_G$ denotes the shortest Euclidean path from $u$ to $w$
that uses only edges of $G$.

In this paper we are interested in the following question: Given
$V\subset\R^d$, with $|V|=n<\infty$, how does one construct a graph,
$G=(V,E)$, with $O(n)$ edges, such that $\asf(G)$ is small, and how small
can we make $\asf(G)$?  Clearly, $\asf(G)\ge 1$, but how close to 1 can
we get with a graph $G$ that has only $Cn$ edges?

The related question of finding graphs with small \emph{(worst-case)
stretch factor}
\[
    \strf(G) = \max_{\{u,w\}\in\binom{V}{2}}\frac{\|uw\|_G}{\|uw\|} 
\]
has been studied extensively.  It is known that, for any $\eps >0$,
it is possible to construct a graph with $O(n/\eps^{d-1})$ edges such
that $\strf(G)\le 1+\eps$ \cite{x} and this dependence on $\epsilon$
and $d$ is optimal \cite{Y}.\note{what are the correct references?}
This upper-bound on (worst-case) stretch factor certainly implies the
same upper-bound for average stretch factor.  Intuitively, though, one
might think that it is possible to obtain improved results for average
stretch factor that, say, reduce the dependence on $\epsilon$ and/or $d$.

\subsection{New Results}

In this paper, we show that much stronger results are possible for
average stretch factor than for (worst-case) stretch factor by proving
the following result: For any constant dimension, $d$, and any set
$V\subset \R^d$, with $|V|=n<\infty$, there exists a geometric graph
$G=(V,E)$ having $|E|\in O(n)$ edges and such that $\asf(G)=1+o_n(1)$.
More precisely,
\[
   \asf(G)=1+O((\log n/n)^{4d-1}) \enspace .
\]
The proof of this result is in \secref{upper-bound} and constitutes the
bulk of the paper.

In \secref{lower-bound} we prove a simple lower-bound that shows our
upper-bound is at least of the right flavour:  For every positive integer
$n$, there exists an $n$ point set in $\R^2$ for which any graph with
$O(n)$ edges has average stretch factor $1+\Omega(1/\sqrt{n})$.

\subsection{Related Work}

General results on embedding metric spaces into ultrametrics
\cite{bartal:graph,fakcharoenphol.rao.ea:tight} implies that, for
every point set $V$ in $\R^d$, there exists a tree $T=(V,E)$ such that
$\asf(T)\in O(\log n)$.  More recently, this was improved by Abraham
\etal\ \cite{abraham.bartal.ea:metric,abraham.bartal.ea:embedding}
to show that there exists a spanning tree, $T$, with $\asf(T)\in O(1)$.

Aldous and Kendall \cite{aldous.kendall:short} show that, for any
set, $V\subset[0,\sqrt{n}]^2$, and any $\epsilon > 0$, there exists a
Steiner network, $N=(V',E)$, with $V'\supseteq V$, of total edge length
$\sum_{uw\in E}\|uw\| \le \msst(V) + \epsilon n$ and in which,
\[
    \sum_{\{u,w\}\in \binom{V}{2}}\frac{\|uw\|_N}{\|uw\|} = 1 + O(\log n/n) .
\]
Here $\msst(V)$ denotes the length of the minimum Steiner spanning
tree of the points in $V$.  For most well distributed\footnote{For
example, any family of point sets that satisfies the quantitative
equidistribution condition \cite{aldous.kendall:short}.} point sets,
$V$, $\msst(V)\in\Omega(n)$, so $N$ is a graph that is only slightly
longer than the minimum Steiner spanning tree of $V$ and for which the
average stretch factor (taken over pairs in $V$) tends to 1.

We call the points in $V'\setminus V$ \emph{Steiner points}.  The work
of Aldous and Kendall, which was the starting point for the current work,
immediately raises two questions: (1)~Is there always a graph, $G=(V,E)$,
that does not use Steiner points and has $\asf(G)=1+o_n(1)$? (2)~Does
this result generalize to point sets in $\R^d$?  Our results answer both
of these questions in the affirmative.

We note that it does not seem easy to answer either of the preceding two
questions using a modification of Aldous and Kendall's construction, which
consists of a minimum Steiner spanning tree of $V$, some additional random
line segments, and some additional segments that form a grid.  Anywhere
two segments cross, a vertex is added to $V'$, so this construction makes
essential use of Steiner points.  Furthermore, the main technical tool
used by Aldous and Kendall is a new result on the lengths of boundaries
of certain cells in arrangements of random lines \cite[Theorems~3 and 4]{aldous.kendall:short}.  In dimensions greater
than 2, arrangements of lines do not decompose space into cells, so it
seems difficult to generalize this result to higher dimensions.

\section{The Construction}
\seclabel{upper-bound}

The construction of a good average stretch factor graph, $G=(V,E)$,
makes use of a clustering of the points of $V$ into $O(n/k)$ clusters,
each of size at most $k$, that we call a $k$-partition.  In the next
subsection, we define $k$-partitions and show how to compute them.
In the following subsection we show how to construct the graph $G$.

\subsection{$k$-Partitions}

We make use of the following construct:  A \emph{$k$-partition} of a
set $V$ of $n$ points in $\R^d$ consists of a set, $D$, of balls and an
assignment $f:V\to D$ such that
\begin{enumerate}
  \item $|D|\in O(n/k)$;
  \item for each $u\in V$, $u\in f(u)$ (i.e., $u$ is a assigned to a
    ball that contains $u$);
  \item for each $\Delta\in D$, $|\{u\in V: f(u)=\Delta\}|\le k$ (i.e.,
   at most $k$ points are assigned to each ball);
  \item for every $r> 0$ and $p\in\R^d$, the number of balls
   in $D$ whose radius is in the range $[r,2r)$ and that contain $p$
   is $O(1)$; and
  \item for every $r\ge 0$ and every ball, $B$, of radius $r$, 
   \[
      |\{ u\in V : u\in B \oand \Delta\in D
        \oand \radius(\Delta)\ge r \mand f(u)=\Delta\}| \in O(k)
   \] 
   (i.e., there are only $O(k)$ points
   of $V$ that are in $B$ and that are assigned to balls of radius at
   least $r$).
\end{enumerate}

Note that, aside from Properties~4 and 5, there is very little structure
to the balls in $D$. In particular, balls in $D$ may overlap and may
even contain each other.

\begin{lem}
  For any set $V$ of $n$ points, a $k$-partition of $V$ exists and can
  be found in $O(n\log n)$ time.
\end{lem}

\begin{proof}
  We construct a $k$-partition using the binary \emph{fair-split
  tree}, $T=T(V)$, which is defined recursively as follows
  \cite{callahan.kosaraju:decomposition}: If $V$ consists of a single
  point, $u$, then $T$ contains a single node corresponding to $u$.
  Otherwise, consider the minimal axis-aligned bounding box, $B(V)$,
  that contains $V$.  The root of $T$ corresponds to $B(V)$ and this
  box is split into two boxes $B_1(V)$ and $B_2(V)$ by cutting $B(V)$
  with a hyperplane in the middle of its longest side.  The left and
  right subtrees of the root are defined recursively by constructing
  fair-split trees for $B_1(V)\cap V$ and $B_2(V)\cap V$. See \figref{fst}.

  \begin{figure}
    \begin{center}
      \includegraphics{whole-thing}
    \end{center}
    \caption{A fair split tree for $V$ repeatedly by splits the bounding
      box $B(V)$ in the middle of its longest side.}
    \figlabel{fst}
  \end{figure}

  \paragraph{The $k$-Partition.}
  For each node, $u$, of $T$ there is a naturally defined subset
  $V(u)\subseteq V$ of points associated with $u$ as well as a bounding
  box $B(u)=B(V(u))$.  Since $T$ is a binary tree with $2n+1$ nodes, it
  has a set of $t-1$ edges whose removal partitions the vertices of $T$
  into $t\in O(n/k)$ maximally-connected components $C_1,\ldots,C_t$,
  each having at most $k$ vertices.

  For each $i\in\{1,\ldots,t\}$, let $u_i$ denote the root of the
  subtree $C_i$.  To obtain the balls, $\Delta_1,\ldots,\Delta_t$, of
  the $k$-partition we take, for each $i\in\{1,\ldots,t\}$ the smallest
  ball, $\Delta_i$ that contains $B(u_i)$.  For the mapping $f$, we
  map the point associated with each leaf, $w$, of $T$ to the unique
  ball $\Delta_i$, where $C_i$ contains $w$.  (Note that some balls,
  $\Delta_i$, may have no points mapped to them if $C_i$ contains no
  leaves of $T$; see \figref{fst-2}.)

  \begin{figure}
    \begin{center}
      \includegraphics{whole-thing2}
    \end{center}
    \caption{The fair split tree is partitioned into subtrees of size
    $k$ (${}=3$) by removing $O(n/k)$ edges.  The root, $u_i$, of each subtree
    defines a ball, $\Delta_i$, in the $k$-partition. (The ball $\Delta_1$
    is omitted from this figure.)}
    \figlabel{fst-2}
  \end{figure}


  The fair-split tree, $T$, and the boxes, $B(u)$, associated
  with each node, $u$, of $T$ can be computed in $O(n\log n)$ time
  \cite{callahan.kosaraju:decomposition}.  The partition of the vertices
  of $T$ into components $C_1,\ldots,C_t$ can easily be done in $O(n\log
  n)$ time by repeatedly finding an edge of a component of size $k'>k$
  whose removal partitions that component into two pieces each of size
  at most $\lceil 2k'/3\rceil$.  Thus, the construction of $D$ and $f$
  can be accomplished in $O(n\log n)$ time.

  The set of balls $D=\{\Delta_1,\ldots,\Delta_t\}$ and the mapping
  $f:V\to D$ described in the preceding paragraphs clearly satisfy
  Properties~1--3 in the definition of a $k$-partition.  What remains
  is to show that they also satisfy Properties~4 and 5. 

  For a node $u$ in $T$, with $B(u)=[a_1,b_1]\times\cdots\times[a_d,b_d]$,
  define $L_i(u)=b_i-a_i$ and let $L(u)=\max\{L_i(i):i\in\{1,\ldots,d\}\}$
  denote the length of $B(u)$'s longest side.  We call $\sum_{i=1}^d
  L_i(u)$ the \emph{total side length} of $B(u)$.  To establish
  Properties~4 and 5, we make use of the following result on fair split
  trees \cite[Lemma~9.4.3]{narasimhan.smid:geometric}:

  \begin{lem}\lemlabel{box-packing}
     Let $C$ be a box whose longest side has length $\ell$ and let
     $\alpha >0$ be any positive real number.  Let $w_1,\ldots,w_s$
     be some nodes of a fair-split tree, $T$, such that
     \begin{enumerate}
       %\item $w_i$ is not the root of $T$, for all $i\in\{1,\ldots,s\}$;
       \item the sets $V(w_i)$ are disjoint, for all $i\in\{1,\ldots,s\}$;
       \item $L(w_i)\ge \ell/\alpha$, for all
          $i\in\{1,\ldots,s\}$;\footnote{The original lemma
          \cite[Lemma~9.4.3]{narasimhan.smid:geometric} is slightly
          stronger in that it only requires that $L(w_i')\ge \ell/\alpha$,
          where $w_i'$ is the parent of $w_i$.} and
       \item $B(w_i)$ intersects $C$, for all $i\in\{1,\ldots,s\}$.
     \end{enumerate}
     Then $s\le (2\alpha + 2)^d$.
  \end{lem}
  %Note that the first condition of \lemref{box-packing} is equivalent
  %to the statement that no $u_i$ is an ancestor of $u_j$ for any
  %$\{i,j\}\subseteq\{1,\ldots,k\}$, which also implies that the interiors of
  %$B(u_i)$ and $B(u_j)$ are disjoint.

  \paragraph{Property 4.}
  To prove that the balls in $D$ satisfy Property~4, let
  $\{\Delta_{i_1},\ldots,\Delta_{i_q}\}\subseteq D$ be the subset of
  balls in $D$ having radii in the interval $[r,2r)$ and that all contain
  some common point, $p\in\R^d$.   Then each such ball, $\Delta_{i_j}$
  corresponds to a node $u_{i_j}$ of $T$ such that
  \begin{equation}
        \frac{r}{\sqrt{d}} \le L(u_{i_j}) \le 4r \enspace . \eqlabel{bounds}
  \end{equation}
  Therefore, each box $B(u_{i_j})$ intersects a ball of radius $2r$
  centered at $p$.  (Indeed, the center of $B(u_{i_j})$ is contained in
  this ball.)  Therefore, each box $B(u_{i_j})$ intersects a box, $C$, of
  side-length $4r$ centered at $p$.  

  We are almost ready to apply \lemref{box-packing} to
  $C$---whose side length is $\ell = 4r$---and the vertex set
  $w_1,\ldots,w_q=u_{i_1},\ldots,u_{i_q}$. For each $j\in\{1,\ldots,q\}$,
  $B(u_{i_j})$ intersects $C$, so Condition~3 of \lemref{box-packing}
  is satisfied.  Furthermore, \eqref{bounds} states that $L(u_{i_j})\ge
  r/\sqrt{d} = \ell/(4\sqrt{d})$, so Condition~2 of \lemref{box-packing}
  is satisfied with $\alpha=4\sqrt{d}$.  Unfortunately, there is still
  a little more work to do since the nodes $u_{i_1},\ldots,u_{i_t}$
  do not necessarily satisfy Condition~1 of \lemref{box-packing}.

  To proceed, we partition $u_{i_1},\ldots,u_{i_q}$ into a small number of
  subsets, each of which satisfies Condition~1 of \lemref{box-packing}.
  Observe that Condition~1 of \lemref{box-packing} is equivalent
  to the statement that no $w_i$ is an ancestor of $w_j$ for any
  $\{i,j\}\subseteq\{1,\ldots,s\}$.  A key observation is that, if $u$
  is an ancestor of $w$ in a fair-split tree, $T$, and the difference
  in depth between $u$ and $w$ is at least $d$, then
  \[
      L(u) \ge 2L(w) \enspace .
  \]
  This, and \eqref{bounds}, implies that, if $u_{i_j}$ is an ancestor
  of $u_{i_{j'}}$ then
  \[
     \depth(u_{i_{j'}})-\depth(u_{i_{j}}) \le d\log(4\sqrt{d}) \enspace .
  \]
  Thus, we can partition $u_{i_1},\ldots,u_{i_q}$ into $z=\lceil
  d\log(4\sqrt{d})\rceil$ subsets, $S_0,\ldots,S_{z-1}$, each of which
  satisfies Condition~1 of \lemref{box-packing}, by assigning $u_{i_j}$
  to the subset $S_{\depth(u_{i_j})\bmod z}$.  

  Now, \lemref{box-packing} implies that, for each $i\in\{0,\ldots,z-1\}$, 
  \[
     |S_i|\le (8\sqrt{d}+2)^d
  \]
  so that
  \[
     q = \sum_{i=0}^{z-1}|S_i|\le (8\sqrt{d}+2)^dz = (8\sqrt{d}+2)^d\lceil d\log(4\sqrt{d})\rceil  \in O(1) \enspace .
  \]
  Thus, for any point $p\in\R^d$, the set of balls in $D$ whose radius
  is in the interval $[r,2r)$ and that contain $p$ has size $O(1)$.
  Therefore the balls in $D=\Delta_1,\ldots,\Delta_t$ satisfy Property~4
  in the definition of a $k$-partition.

  \paragraph{Property 5.}
  To study Property~5, it is easier to work with the bounding boxes,
  $B(u)$, associated with each node, $u$, in the fair split tree as
  well as the box, $C$, of side length $2r$ that contains the ball $B$.
  See \figref{property-5}.  Observe that, if some ball $\Delta_i$
  is assigned a point in $B$, then the box $B(u_i)$ intersects $C$.
  Thus, we need only consider the set $U\subseteq\{u_1,\ldots,u_t\}$
  that contains only those nodes $u_i$ such that $\radius(\Delta_i)\ge r$
  and $B(u_i)$ intersects $C$.

  \begin{figure}
    \begin{center}
      \includegraphics{property-5}
    \end{center}
    \caption{Proving Property~5 of $k$-partitions.}
    \figlabel{property-5}
  \end{figure}

  For each $u\in U$, \eqref{bounds} implies that $L(u)\ge r/\sqrt{d}$.
  Therefore, by \lemref{box-packing}, $U$ contains a subset, $U'$,
  of size at most $(4\sqrt{d}+2)^d\in O(1)$ such that every node in
  $U$ is an ancestor of some node in $U'$.  Thus, the elements of $U$
  can be partitioned into $O(1)$ sets $U_1,\ldots,U_p$ such that, for
  every $i\in\{1,\ldots,p\}$ and every $u,w\in U_i$, $u$ is an ancestor
  of $w$ or $w$ is an ancestor of $u$.  Therefore, it is sufficient to
  consider a single one of the sets $U_i$, which is a sequence of vertices
  $w_1,\ldots,w_\ell$ where $w_j$ is an ancestor of $w_i$ for all $j>i$.

  For each $i\in\{1,\ldots,\ell\}$, let $C_{i}$, denote the box $C\cap
  B(w_{i})$.  Since $B(w_1)\subset\cdots\subset B(w_\ell)$, we have that
  $C_1\subseteq\cdots\subseteq C_\ell$. Observe that, for each $w_i$, the
  ball associated with $w_i$ is not assigned any points in $B(w_{i-1})$.
  Thus, it is sufficient to show that there are $O(1)$ values of $i$
  for which $V\cap C_i\neq V\cap C_{i-1}$; for each such $i$, the number
  of elements assigned to the corresponding ball of the $k$-partition
  is at most $k$.

  We will show that, for each $i\in\{2,\ldots,\ell\}$, at least one of
  the following statements is true
  \begin{enumerate}
    \item $V\cap C_i = V\cap C_{i-1}$; 
    \item the total side length of $C_i$ exceeds that of $C_{i-1}$ by at least
      $L(w_1)/2$; or
    \item $C_i$ intersects a side of $C$ that is not intersected by $C_{i-1}$.
  \end{enumerate}
  This is sufficient to prove the result since Case~1 does not result in
  any new points included in $B$, Case~2 can occur at most $4rd/L(w_1)\in
  O(1)$ times, and Case~3 can occur at most $2d\in O(1)$ times.

  To see why one of the preceding cases must occur, suppose that neither
  Case~1 nor Case~3 applies.  Since Case~1 does not apply, there is
  some point $q\in V\cap C_i$ that is not in $C_{i-1}$.  Let $w'$ be the
  first node on the path from $w_{i-1}$ to $w_i$ such that $q\in B(w')$.
  Without loss of generality, assume that the fair split tree cuts
  $B(w')$ with a plane, $\Pi$, that is perpendicular to the $x_0$-axis.
  Let $\Pi^+$ and $\Pi^-$ denote the closed halfspaces bounded by $\Pi$
  that contain $q$ and $B(w_{i-1})$, respectively.  Then we have that
  \[
      L_0(B(w')\cap \Pi^+) = L_0(w')/2 \ge L(w_{i-1})/2 \ge L(w_{1})/2  \enspace .
  \]
  Observe that $B(w_{i-1})$ does not intersect the side of $C$ that is
  parallel to $\Pi$ and contained in $\Pi^+$ and neither does $B(w')$
  (since, otherwise, Case~3 would apply).  This implies that
  \[
      L_0(C_i) \ge L_0(C_{i-1}) + L(w_1)/2 \enspace .
  \] 
  Thus, if neither Case~1 nor Case~3 applies to $u_i$, then Case~2
  applies.  This completes the proof.
\end{proof}

\subsection{The Graph $G$}

With the availability of $k$-partitions, we are now ready to construct
a graph $G$ with low average stretch.  In the following construction,
positive valued variables $c,k\in\omega_n(1)$ and $\epsilon\in o_n(1)$
are used without being specified.  Values of these variables that
optimize the average spanning ratio of $G$ will be given in the proof
of \thmref{upper-bound}.  In the meantime, the reader can mentally assign the
values $c=k=\log n$ and $\epsilon = 1/\log n$, which are sufficient to
prove that $\asf(G)=1+o_n(1)$.

\paragraph{A Net of Stars.}
We begin with a $k$-partition $(\{\Delta_1,\ldots,\Delta_{n'}\},f)$
of $V$, and we use the convention that $\Delta_1,\ldots,\Delta_{n'}$
are ordered by decreasing size.  For each $i\in \{1,\ldots,n'\}$,
let $V_i=\{u\in V : f(u)=\Delta_i\}$; that is, $V_i$ is the set of
points assigned to the ball $\Delta_i$.  For each set $V_i$, we choose a
representative vertex, $u_i$, and add edges joining $u_i$ to each other
vertex in $V_i$ (see \figref{overview}). Let $N=\{u_1,\ldots,u_{n'}\}$
denote the set of representative vertices and recall that $|N|=n'\in
O(n/k)$.

\begin{figure}
  \begin{center} 
    \includegraphics{overview}
  \end{center} 
  \caption{Using a $k$-partition, $G$ contains $O(n/k)$ stars whose
    centers are interconnected by a $(1+1/k^{1/(d-1)})$-spanner}
  \figlabel{overview}
\end{figure}

\paragraph{The Two Spanners.}
Next, we add two spanner constructions to $G$.  The first is a
$(1+1/k^{1/(d-1)})$-spanner of $N$ and the second is a $2$-spanner
of $V$.  The first construction adds $O(k|N|)=O(n)$ edges to $G$
\cite[Section~5.5]{narasimhan.smid:geometric}, while the second adds
$O(n)$ edges \cite{x,ys,ss}.  With the addition of these extra edges we
have, for any $u,w\in V$
\[
   \frac{\|uw\|_G}{\|uw\|} \le \begin{cases}
         1+1/k^{1/(d-1)} & \text{if $u,w\in N$} \\
         2 & \text{in any case.}
       \end{cases}
\]

Let $r_i$ denote the radius of $\Delta_i$ and let $D_i$ be the ball centered
at the center of $\Delta_i$ and having radius $cr_i$.  Points of $V$
that are in $D_i\setminus \Delta_i$ can be problematic for $V_i$; there
is no guarantee that such points have a $1+o(1)$ spanning path to the
points in $V_i$.  The final step in the spanner construction attempts
to deal with most of these problematic points.

\paragraph{Covering with Pennies.}
For each $i\in\{1,\ldots,n'\}$, we find a ball, $E_i$, of radius $r_i/c$,
that intersects $D_i$, and that contains the maximum number of points
of $V$.  (Note that this may include points of $V$ in $V_i$ or outside
of $D_i$.)  We then add edges joining each of the points in $V_i$ to
a carefully chosen point $w_i\in E_i$.  See \figref{hitter}.

The point $w_i$ is chosen as follows: For each point $w\in V$, let
$i(w)\in \{1,\ldots,n'\}$ denote the smallest index such that $w\in
E_{i(w)}$ and $|E_{i(w)}\cap V| \ge \epsilon n$; if no such index
exists, let $i(w)=\infty$.  The point $w_i\in E_i$ is selected to be
any of the points in $E_i$ that minimizes $i(w)$.  This concludes the
description of the graph, $G$.  


\begin{figure}
  \begin{center}
    \includegraphics{hitter}
  \end{center}
  \caption{The ball $E_i$ captures as many points of $V$ as possible
   while still having center in $D_i$.}
  \figlabel{hitter}
\end{figure}

\paragraph{Two Illustrative Examples.}

Before delving into the proof that $G$ has low average stretch factor,
it may be helpful to study two examples that illustrate why the disks
$E_1,\ldots,E_n$ are important and why the choice of the representative
vertices, $w_i\in E_i$, is  important.

The first example is a set of points arranged as a sequence of
$\sqrt{k}\times\sqrt{k}$ grids, $G_0,\ldots,G_{n/k-1}$.  The grid $G_i$
has its center on the x-axis at x-coordinate $2^{i+1}-1$, and has side
length $2^i$ (see \figref{grid}). The natural $k$-partition of this
grid is the one that assigns all points in each $G_i$ to a single ball,
$\Delta_i$.  In this grid, if we consider $G_i$, for some large value
of $i$, we see that all the points in $G_0,\ldots,G_{i-1}$ are within
distance $O(2^{i})$ of all the points in $G_i$.
Forcing every path from any $u\in G_i$ to any $w\in G_{j}$, $j<i$,
to go through a central vertex $u_i$ in $G_i$ would be too costly; on
average the detour through $u_i$ would increase the length of this path
by $\Omega(2^{i})$.

\begin{figure}
  \begin{center}
    \includegraphics{grids}
  \end{center}
  \caption{A sequence of exponentially increasing grids illustrates the
   need for connecting all points in $V_i$ to some point in $E_i$.}
  \figlabel{grid}
\end{figure}

The disk $E_i$ solves the preceding problem; $E_i$ is large enough to
cover all points in $G_0,\ldots,G_{i-\Theta(\log c)}$.  The path from
$u\in G_i$ directly to $w_i\in E_i$ and onto any $w\in E_i$ has length
at most
\[
    \|uw\| + O(2^{i}/c) \enspace ,
\]
Furthermore, all of the points in $E_i$ are at distance $\Omega(2^i)$ from
all the points in $G_i$, so $\|uw\|_G/\|uw\| = 1+O(1/c)$.  Part~3 of the
proof of \thmref{upper-bound} is dedicated to showing that, in general,
the disks $E_1,\ldots,E_{n'}$ cover many pairs of points that might
otherwise be problematic.

Our second example is intended to illustrate the importance of carefully
choosing the representative vertex $w_i\in E_i$.   In this example,
there is a dense cluster of $n/2$ points that is small enough that it
is just barely contained in some disk $E_j$ (see \figref{example2}).
Consider now some $i>j$ such that $r_i$ is much greater than $r_j$.
It is easy to make a configuration of points so that, for some cluster
$V_i$, the corresponding disk $E_i$ contains $E_j$ as well as one (or a
small number of) other points that are far from $E_j$. If one of these
points as the representative vertex $w_i$ then all $kn/2$ paths from
$u$ in $V_i$ to $w\in E_j$ will have to make a detour through $w_i$.
By repeating this for many different values of $i$, this is enough to
produce an average stretch factor significantly larger than 1.

\begin{figure}
  \begin{center}
    \includegraphics{example2}
  \end{center}
  \caption{An illustration of why it is important to choose $w_i$ carefully.
    A bad choice (like the one illustrated) leads to a significant detour
    on the paths from every $u\in V_i$ to every $w\in E_j$.}
  \figlabel{example2}
\end{figure}




The choice of $w_i$ is designed to avoid the preceding problem.  In this
example, $w_i$ would be chosen from the points in the $E_j$ such $r_j <
r_i$ and $E_i$ contains $n/2\ge \epsilon n$ points of $V$.  Part~4 of
the proof of \thmref{upper-bound} is dedicated to showing this careful
choice of $w_i$ works.  In particular, it ensures that, for most pairs
$u\in V_i$, $w\in E_i$,
\[
    \|uw_i\| + O(\|w_iw\|) = \|uw\|(1+O(1/c)) \enspace .
\]

Without further ado, we now prove that $G$ has low average stretch factor.

\begin{thm}\thmlabel{upper-bound}
  For every constant dimension, $d$, and every set, $V$, of
  $n<\infty$ points in $\R^d$, the graph $G=(V,E)$ described above
  has $O(n)$ edges and $\asf(G)=1+o_n(1)$.  More precisely,
  $\asf(G)=1+O((\log n)/n^{1/(4d-1)})$.
  % there exists a geometric graph $G$ with
  % $V(G)=V$, $|E(G)|\in O(n)$ and $\asf(G)=1+o_n(1)$. 
  % More precisely, $\asf(G)=1+O((\log n/n)^{1/2d})$.
\end{thm}

\begin{proof}
  That $G$ has $O(n)$ edges follows immediately from its definition.

  To upper-bound the average stretch factor of $G$, there are four types
  of pairs of points, $u\in V_i$, $w\in V_j$, $j\ge i$, to consider
  (recall that $r_i \ge r_j)$:
  \begin{enumerate}
    \item pairs that are both from the same set; i.e., where $i=j$;
    \item pairs for which $w$ is outside of $D_i$;
    \item pairs for which $w$ is contained in $E_i\cap D_i$; and
    \item pairs for which $w$ is contained in $D_i\setminus E_i$.
  \end{enumerate}
  In the next few section we consider each of these types of pairs in
  turn.  Our strategy is to study the $\binom{n}{2}$ terms that define
  $\asf(G)$ in \eqref{asf}.  We will show that $o(n^2)$ of these terms
  are at most 2 while the remaining terms are at most $1+o_n(1)$.  Thus,
  \[
     \asf(G)\le \binom{n}{2}^{-1}\left(2\cdot o(n^2)
                                       +\binom{n}{2}(1+o_n(1))\right)
     = 1+o_n(1) \enspace .
  \]

  \paragraph{Type~1 Pairs.}
  Each $V_i$, for $i\in\{1,\ldots,n'\}$ defines at most $\binom{k}{2}$
  Type~1 pairs, so the total number of Type~1 pairs that contribute a
  term to the sum in \eqref{asf} is at most
  \[
    \binom{k}{2}\cdot O(n/k) \in O(nk)
      \enspace .
  \]

  \paragraph{Type~2 Pairs.}
  For each Type~2 pair $u\in V_i$, $w\in V\setminus D_i$, there is a path
  from $u$ to $u_i$ and then onto $u_j$ and finally to $w$ whose length
  is at most
  \[
     2r_i + \|u_iu_j\|_G + 2r_j
      \le (1+1/k^{1/(d-1)})(\|uw\| + 8r_i) \enspace .
  \]
  Furthermore, $\|uw\|\ge (c-1)r_i$ since $w$ is outside of $D_i$.
  Therefore, for each Type~2 pair, the term that appears in \eqref{asf}
  is of the form
  \[
    \frac{\|uw\|_G}{\|uw\|}\le (1+1/k^{1/(d-1)})(1+O(1/c)) 
       = 1+O(1/k^{1/(d-1)}+1/c) \enspace .
  \]

  \paragraph{Type~3 Pairs.}
  The number of Type~3 pairs is no more than 
  \[  
     k\cdot\sum_{i=1}^{n'}|V\cap D_i\setminus E_i| \enspace .
  \]
  We will prove, by contradiction, that this quantity is $o(n^2)$.
  Suppose, for the sake of contradiction, that this is not the case
  and that
  \begin{equation}
    \sum_{i=1}^{n'} |V\cap D_i\setminus E_i| \ge \frac{\delta n^2}{k}
         \enspace , \eqlabel{kicker}
  \end{equation}
  where $\delta>0$ will be determined later.
  Each term on the left hand side of \eqref{kicker} is at most $n$
  and there are $n'\le \alpha n/k$ terms, for some constant $\alpha
  >0$.  We say that a term on the left hand side of \eqref{kicker} is
  \emph{small} if it is less than $\delta n/2\alpha$ and \emph{large}
  otherwise.  The sum of the small terms is at most $\delta n^2/2k$
  and therefore the sum of the large terms is at least $\delta n^2/2k$.
  Let $I$ be the index set of these large terms.  Then
  \[
    \sum_{i\in I} |V\cap D_i\setminus E_i| \ge \frac{\delta n^2}{2k} \enspace .
  \]
  By the pigeonhole principle, there must exist some point $w^*\in V$
  such that there are at least $\delta n/2k$ indices $i\in I$ such that
  $w^*\in V\cap D_i\setminus E_i$.  To summarize the discussion so far:
  There exists a point $w^*\in V$ and index set $I$
  such that:
  \begin{enumerate}
     \item $w^*\in V\cap D_{i}\setminus E_{i}$, for all
        $i\in I$
     \item $|V\cap D_{i}\setminus E_{i}|\in \Omega(\delta n)$,
       for all $i\in I$; and
     \item $|I|\in\Omega(\delta n/k)$.
  \end{enumerate}

  Suppose, without loss of generality, that the smallest ball
  $\Delta_i$ with $i\in I$ has unit radius.  Partition $I$ into groups
  $G_0,G_1,\ldots$ such that $G_t$ contains all indices $i\in I$ such
  that $\Delta_{i}$ has radius in the interval $[2^t,2^{t+1})$.  We claim
  that each such group, $G_t$, has size $O(c^d)$.  To see why this is
  so, observe that, for each group $G_t$, there exists a point---namely
  $w^*$---that is contained in $|G_t|$ balls $D_{i}$ where $i\in G_t$.
  $D_{i}$ has radius at most $c2^{t+1}$.  This means that the set of balls
  \[
     \{ \Delta_i : i\in G_t\}
  \]
  is contained in a ball, centered at $w^*$, of radius at most
  $(c+2)2^{t+1}$.  Since each ball in this set has radius in
  $[2^t,2^{t+1})$, a simple packing argument that uses Property~4 of
  $k$-partitions implies that the size of $G_t$ is $O(c^d)$.

  Thus far, we have shown that each group, $G_t$, has size $O(c^d)$
  and the total size of all groups is $|I|\in\Omega(\delta n/k)$.
  Therefore, there must be at least $\Omega(\delta n/(kc^d))$ groups.
  In particular, we can find $h\in\Omega(\delta n/(kc^d\log c))$ groups,
  $G_{t_1},\ldots,G_{t_h}$, such that $t_{i+1} \ge t_{i}+2\log c+2$ for
  each $i\in\{1,\ldots,h-1\}$.  By selecting a representative element from
  each of these groups, we obtain a sequence of indices $i_1,\ldots,i_h$
  such that the radius of $\Delta_{i_{j+1}}$ is at least $4c^d$ times
  the radius of $\Delta_{i_j}$ for each $j\in\{1,\ldots,h-1\}$.

  By choice, $D_{i_1}$ contains at least $ a \delta n$ elements of
  $V$, for the constant $a=1/2\alpha$.  Also by choice, $D_{i_2}\setminus
  E_{i_2}$ contains at least $ a \delta n$ elements of $V$.
  Both $D_{i_1}$ and $D_{i_2}$ contain the point $w^*$.  We claim
  that $E_{i_2}$ contains at least $ a \delta n$ elements of $V$
  as well since there exists a ball, $E_{i_2}'$, centered at $w^*$,
  of radius $r_{i_2}/c > 4cr_{i_1}$, that contains $D_{i_1}$ and
  therefore contains all the (at least $ a\delta n$) points in $D_{i_1}$
  (see \figref{containment}).  The ball $E_{i_2}$ was chosen to contain
  as many elements of $V$ as possible, so it contains at least as many
  elements as $E_{i_2}'$.  Therefore, $D_{i_2}\cup E_{i_2}$ contains at
  least $2 a  \delta n$ points of $V$.

  \begin{figure}
     \begin{center}
       \includegraphics{containment}
     \end{center}
     \caption{$D_{i_2}\cup E_{i_2}$ contains at least $2 a  n$ 
              points of $V$.}
     \figlabel{containment}
   \end{figure}

  We can now argue similarly to show that $E_{i_3}$ contains at
  least $2 a \delta n$ points of $V$ so $D_{i_3}\cup E_{i_3}$
  contains at least $3 a \delta n$ points of $V$.  In general,
  this argument shows that $D_{i_h}\cup E_{i_h}$ contains at least
  $h a \delta n$ points of $V$.  But this yields a contradiction for
  $h> 1/ a \delta$, since $V$ contains only $n$ points.  To obtain
  this contradiction, our choice of $\delta$, $c$, and $k$ must satisfy
  \[
       h\in\Omega\left(\frac{\delta n}{kc^d\log c}\right) \ge
          \frac{1}{ a  \delta}
  \]
  which is satisfied by any choice of $\delta$, $c$, and $k$ such that
  \[
       \frac{\delta^2 n}{kc^d\log c} \ge \frac{C}{a}
  \]
  for some sufficiently large constant $C$.  In particular, the value
  \[
       \delta = \frac{C}{a}\sqrt{\frac{kc^d\log c}{n}}
  \]
  works.  So the total number of terms of the sum in \eqref{asf}
  contributed by Type~3 pairs is at most
  \[
    k\delta n^2 \in O(k^{3/2}n^{3/2}c^{d/2}\log c) \enspace .
  \]

  \paragraph{Type~4 Pairs.}  
  Let $\beta > 0$ be a constant whose value will be discussed later.
  We say that a Type~4 pair of points $u\in V_i$, $w\in E_i$ is
  \emph{bad} if
  \[
      \|uw_i\|+2\|w_iw\| \ge (1+\beta/c)\|uw\| \enspace ,
  \]
  and otherwise the pair is \emph{good}.
  For any good pair $(u,w)$,
  \[
    \frac{\|uw\|_G}{\|uw\|} = 1+O(1/c) \enspace ,
  \]
  so we can focus our effort on upper-bounding the number of bad pairs.
  The remainder of this will assume, for the sake of contradiction,
  that the number of bad Type~4 pairs is at least $\epsilon n^2$.

  Let $b_i$ denote the number of number of bad pairs $(u,w)$ with
  $u\in V_i$ and $w\in E_i$.  Then, by assumption,
  \[
    \sum_{i=1}^{n'} b_i \ge \epsilon n^2 \enspace .
  \]
  Applying the same reasoning used to study Type~3 pairs, we can find a
  point $w^*\in V$ and a set of indices $i_1,\ldots,i_{\ell}$ such that
  \begin{enumerate}
    \item $w^*\in E_{i_j}$, for all $j\in\{1,\ldots,\ell\}$;
    \item $b_{i_j} \in \Omega(\epsilon kn)$, for all $j\in\{1,\ldots,\ell\}$;
    \item $\ell\in \Omega(\epsilon n/k)$.
  \end{enumerate}

  Assume that the indices $i_1,\ldots,i_\ell$ are ordered so that
  $r_{i_j}\le r_{i_{j+1}}$ for all $j\in\{1,\ldots,\ell-1\}$.  Consider
  the sequences of balls $E'_{i_1},\ldots,E'_{i_\ell}$, where each
  $E'_{i_j}$ is a disk of radius $2r_{i_j}/c$ centered at $w^*$. Recall
  that the radius of $E_{i_j}$ is $r_{i_j}/c$, so that $E'_{i_j}\supset
  E_{i_j}$ and, in particular, $|E'_{i_j}\cap V|\ge |E_{i_j}\cap V|$,
  for each $j\in\{1,\ldots,\ell\}$.

  The plan for the rest of the proof is as follows:  We will find an
  annulus $A=E'_{i_{j^*+t+C}}\setminus E'_{i_{j^*}}$ that does not contain
  very many points of $V$.  We will then use the fact that $A$ does not
  contain many of points of $V$ and Property~5 of $k$-partitions to prove
  that, for some index $j\in\{j^*,\ldots,j^*+t\}$, $b_{i_j}< D\epsilon kn$
  for any constant $D>0$.  This yields the desired contradiction, since
  $i_1,\ldots,i_\ell$ were chosen so that $b_{i_j}\in\Omega(\epsilon k n)$
  for every $j\in\{1,\ldots,\ell\}$.

  To begin, we fix some positive integers $C\in O(1)$ and $t< \ell
  - C$ to be specified later.  For each $j\in\{2,\ldots,\ell\}$,
  let $n_{i_j}=|E'_{i_j}\cap V\setminus E'_{i_{j-1}}|$. We have that
  $\sum_{i=2}^{\ell} n_{i_j} \le n$ and $\ell \in\Omega(\epsilon n/k)$.
  Using these two bounds, a simple averaging argument is sufficient
  to show that there must exist an index $j^*\in\{1,\ldots,\ell-t-C\}$
  such that
  \begin{equation}
     \sum_{j=j^*+1}^{j^*+t+C} n_{i_j}
        = |V\cap E'_{i_{j^*+t+C}}\setminus \E'_{i_{j^*}}| 
          \in O((t+C)k/\epsilon) \enspace . \eqlabel{sparse}
  \end{equation}

  The careful choice of $w_i$'s implies the following claim, whose proof
  is deferred until later.
  \begin{clm}\clmlabel{zuper}
    For every $j\in\{j^*+1,\ldots,j^*+t\}$, every $u\in V_{i_j}$, and
    every $w\in E_{i_{j^*}}\cap V$, $G$ contains a path of length at most
    $\|uw\|+O(r_{i_{j^*}}/c)$.
  \end{clm}

  
  \begin{figure}
    \begin{center}
      \includegraphics{type4}
    \end{center}
    \caption{The number of points in $E$ that are assigned to
      $\Delta_{i_{j^*}},\ldots,\Delta_{i_{j^*+t}}$
      is only $O(k)$.}
    \figlabel{type4}
  \end{figure}

  Refer to \figref{type4}.  Let $E$ denote the disk centered at $w^*$
  and having radius $r_{i_{j^*}}=c\cdot\radius(E_{i_{j^*}})$.  Note that
  any point $u\not\in E$ is at distance at least $(1-2/c)r_{i_{j^*}}$ from
  any point $w\in E_{i_{j^*}}$.  Therefore, by \clmref{zuper}, for any
  $w\in E_{i_{j^*}}\cap V$, any $j\in\{j^*,\ldots,j^*+t\}$ and any $u\in
  V_{i_{j}}\setminus E$,
  \[  
     \frac{\|uw\|_G}{\|uw\|} = 1+O(1/c) \enspace . 
  \]
  In other words, by choosing an appropriate constant $\beta$ in the
  definition of bad pairs, $u$ can not form a bad pair with a point
  $w\in E_{i_{j^*}}$ unless $u$ is contained in $E$.

%  Next we show that the sequence of radii $r_{i_1},\ldots,r_{i_\ell}$
%  is, in a weak sense, exponentially increasing. Specifically,
%  there exists an integer constant $\mu$ such that $r_{i_{j+\mu}}
%  \ge 2r_{i_j}$ for all $j\in\{1,\ldots,\ell-\mu\}$.  To show
%  this, we first observe that, if $u\in V_i$ and $w\in E_i$ form
%  a bad pair, then the distance from $\Delta_i$ to $E_i$ is less
%  than $r_i$.  Next, suppose that $r_{i_{j+\mu}} < 2r_{i_j}$.  Then
%  $\Delta_{i_j},\ldots,\Delta_{i_{j+\mu}}$ is a set of $\mu+1$ disks
%  all having radii in $[r_{i_j},2r_{i_j})$ and that are all contained
%  in a ball of radius $6r_{i_j}$ centered at $w^*$.  Therefore, some
%  point, $p$, in this ball is contained in $\Omega(\mu)$ of these disks.
%  But then Property~4 of $k$-partitions implies that $\mu\in O(1)$.

  Now consider the disks
  $\Delta_{i_{j^*+1}},\ldots,\Delta_{i_{j^*+t}}$. Each of these disks
  has radius at least $r_{i_{j^*}}=\radius(E)$.  By Property~5 of
  $k$-partitions,
  \[
    \left|\bigcup_{j=j^*+1}^{j^*+t} V_{i_{j}}\cap E\right|
      \in O(k)
  \]
  Combining this with \eqref{sparse}, we find that the total number
  of bad pairs of the form $u\in V_{i_{j}}$, $w\in E_{i_{j}}$,
  and $j\in\{j^*,\ldots,j^*+t\}$ is upper bounded by
  \begin{equation}
     \sum_{j=j^*+1}^{j^*+t} b_{i_{j}} 
        \in O((t+C)k^2/\epsilon) \enspace . \eqlabel{ub-4}
  \end{equation}
  On the other hand, the indices $i_1,\ldots,i_\ell$ are chosen
  so that $b_{i_j}\in\Omega(\epsilon kn)$, for \emph{every}
  $j\in\{1,\ldots,\ell\}$.  Therefore
  \begin{equation}
    \sum_{j=j^*+1}^{j^*+t} b_{i_{j}} 
        \in \Omega(t\epsilon kn) \enspace .
        \eqlabel{lb-4}
  \end{equation}
 
  Equating the right hand sides of \eqref{ub-4} and \eqref{lb-4}, we
  obtain a contradiction when $t>2C$ and
  \[
      \epsilon = D\sqrt{k/n}
  \] 
  for a sufficiently large constant $D$.
  Thus, the total number of bad Type~4 pairs is at most
  \[
    \epsilon n^2 \in O(n^{3/2}k^{1/2}) \enspace .
  \]

  All that remains in handling Type~4 pairs is to prove \clmref{zuper}.

  \begin{proof}[Proof of \clmref{zuper}]
  Let $u$ by any point $V_{i_{j}}$, for any $j\in\{j^*+1,\ldots,j^*+t\}$.
  It is sufficient to prove that $\|w^*w_{i_{j}}\|\in O(r_{i_{j^*}}/c)$.
  If $w_{i_{j}}=w^*$, then we are done, so assume $w_{i_{j}}\neq
  w^*$.  Since $w^*\in E_{i_{j}}$ but was not chosen to act as
  $w_{i_{j}}$, there must exist some index $i'$ such that $E_{i'}\cap
  E_{i_{j}}\neq\emptyset$, $|E_{i'}\cap V|\ge\epsilon n$, $r_{i'}\le
  r_{i_{j^*}}$, and $w_{i_{j}}\in E_{i'}$.

  There are two cases to consider:
  \begin{enumerate}
    \item 
    If $E'_{i_{j^*}}$ intersects $E_{i'}$, then (see \figref{gummo-a})
    \[
       \|w^*w_{i_j}\|\le 2r_{i_{j^*}}/c + 2r_{i'}/c \le 4r_{i_{j^*}}/c \in O(r_{i_{j^*}}) \enspace ,
    \]
    and we are done. 
    \begin{figure}
      \begin{center}
        \includegraphics{gummo-a}
        \caption{If $E_{i'}$ intersects $E'_{i_j^*}$ then $\|w^*w_{i_j}\|\le 4r_{i_{j^*}}/c$.}
        \figlabel{gummo-a}
      \end{center}
    \end{figure}
  
    \item If $E_{i'}$ and $E'_{i_{j^*}}$ are disjoint, then we claim that
    $E_{i'}\subset E'_{i_{j^*+t+C}}$ (see \figref{gummo-b}).  To see why
    this is so, we argue that, when $C$ is a sufficiently large constant,
    \[
      r_{i_{j^*+t+C}} > 2r_{i_{j^*+t}} 
          \ge 2r_{i_j} \ge r_{i_j} + r_{i'} \enspace .
    \]  
    To see why this first inequality holds, we first observe
    that, for any $i\in\{1,\ldots,n'\}$, if $u\in V_i$ and $w\in E_i$
    form a bad pair, then the distance from $\Delta_i$ to $E_i$ is less
    than $r_i$.  Now, if $r_{i_{j^*+t+C}} < 2r_{i_{j^*+t}}$.
    Then $\Delta_{i_{j^*+t}},\ldots,\Delta_{i_{i_{j^*+t+C}}}$ is a set
    of $C+1$ disks all having radii in $[r_{i_{j^*+t}},2r_{i_{j^*+t}})$ and
    that are all contained in a ball of radius $6r_{i_{j^*+t}}$ centered
    at $w^*$.  Therefore, some point, $p$, in this ball is contained in
    $\Omega(C)$ of these disks.  But then Property~4 of $k$-partitions
    implies that $C\in O(1)$.  Thus, for a sufficiently large constant, $C$,
    $r_{i_{j^*+t+C}} > 2r_{i_{j^*+t}}$ and $E'\subset E_{i_{j^*+t+C}}$.

    Since $E_{i'}$ and $E_{i_{j^*}}$ are disjoint and $E_{i'}\subset
    E_{i_{j^*+t+C}}$,
    \[
      |V\cap E_{i_{j^*+t+C}}\setminus E_{i_{j^*}}| 
         \ge |V\cap\E_{i'}| \ge \epsilon n \enspace .
    \]
    \begin{figure}
      \begin{center}
        \includegraphics{gummo-b}
        \caption{If $E_{i'}$ does not intersect $E'_{i_j^*}$ then
          $E_{i_{j^*+t+C}}\setminus E_{i_{j^*}}$ contains at least
          $\epsilon kn$ points.}
        \figlabel{gummo-b}
      \end{center}
    \end{figure}
    But, by definition, 
    \[
      |V\cap E_{i_{j^*+t+2\mu}}\setminus E_i| 
         \in O(tk/\epsilon) \enspace .
    \]
    This yields a contradiction when $t<D \epsilon^2 n/k$, for a
    sufficiently small constant $D>0$.
  \end{enumerate}
  This completes the proof of \clmref{zuper}.
  \end{proof}


  \paragraph{Finishing Up.}
  We can now pull everything together to summarize and complete the proof
  of \thmref{upper-bound}.
  \begin{enumerate}
    \item The number of Type~1 pairs is at most $O(kn)$.
    \item For each Type~2 pair, $(u,w)$, 
      $\|uw\|_G/\|uw\|\le 1+ O(1/k^{1/(d-1)}+1/c)$.
    \item The number of Type~3 pairs is at most
      $O(k^{3/2}n^{3/2}c^{d/2}\log c)$.
    \item For each good Type~4 pair, $(u,w)$, 
      $\|uw\|_G/\|uw\|\le 1+ O(1/c)$.
    \item The number of bad Type~4 pairs is at most 
       $O(k^{1/2}n^{3/2})$.
  \end{enumerate}
  The contributions of Type~2 and Type~3 pairs dominate the others, and
  we obtain
  \[
     \asf(G) = 1 + O\left(1/k^{1/(d-1)} + 1/c 
       + \binom{n}{2}^{-1}\left(k^{3/2}n^{3/2}c^{d/2}\log c\right)\right) \enspace .
  \]
  Taking 
  \[ 
       c = (n/\log n)^{1/(4d-1)}
  \]
  and
  \[
       k = (n/\log n)^{(d-1)/(4d-1)} 
  \]
  yields
  \[
     \asf(G) = 1+ O\left((\log n/n)^{1/(4d-1)}\right) \enspace . \qedhere
  \]
\end{proof}


\section{Lower Bounds}
\seclabel{lower-bound}

\begin{thm}\thmlabel{lower-bound}
  For every positive integer, $n$, there exists a point set, $V$, of $n$
  points in $\R^2$, such that every geometric graph, $G$, with vertex
  set $V$ and having $O(n)$ edges has $\asf(G)\ge 1 + \Omega(1/\sqrt{n})$.
\end{thm}

\begin{proof}
  For simplicity, we assume $n$ is even.  The point set, $V$, has
  its points evenly distributed on two opposite sides of a square.
  The point set $V=A\cup B$, where
  \[  
      A = \{(1,i): i\in\{1,\ldots,n/2\}\} 
  \]
  and
  \[  
      B = \{(n/2,i): i\in\{1,\ldots,n/2\}\} 
  \]
  is an example (see \figref{lower-bound}.a).

  \begin{figure}
    \begin{center}
      \begin{tabular}{c@{\hspace{2cm}}c}
      \includegraphics{lower-bound} & \includegraphics{lower-bound-b} \\
      (a) &\hspace{1cm} (b) 
      \end{tabular}
    \end{center}
    \caption{The lower-bound (a)~point set for \thmref{lower-bound}, and
      (b)~the best-case ratio $\|uw\|_G/\|uw\|$ for a pair $(u,w)$ that
      is not covered by any edge.}
    \figlabel{lower-bound}
  \end{figure}

  Let $G$ be any graph with vertex set $V$.  We say that edge a $uw$
  with $u\in A$ and $w\in B$ \emph{covers} the set of pairs
  \[
     \{ \left(u+(0,i), w+(n,j)\right) : 
          i,j\in\{-\sqrt{\alpha n},\ldots,\sqrt{\alpha n}\}\}
  \]
  for some constant $\alpha$ to be discussed later.  Thus, any edge of
  $G$ covers at most $4\alpha n$ pairs in $A\times B$.

  Next, observe that if some pair of points $u\in A$ and $w\in B$ is
  not covered by any edge of $G$, then a straightforward minimization
  argument shows that
  \[
     \frac{\|uw\|_G}{\|uw\|}
       \ge \frac{\sqrt{\alpha n}+\sqrt{(n/2)^2+(n/2-\sqrt{\alpha n})^2}}
               {n/\sqrt{2}}
       \ge \frac{\sqrt{\alpha n}+n/\sqrt{2}-O(n^{1/4})}
               {n/\sqrt{2}}
       \ge 1+\Omega(1/\sqrt{n})
  \]
  (see \figref{lower-bound}.b).
  If $G$ has $m\in O(n)$ edges, then we select $\alpha \le
  \binom{n}{2}/(8mn)$ so that
  \[  
     m4\alpha n \le \frac{\binom{n}{2}}{2} 
  \]
  In this way, at least half of the $\binom{n}{2}$ pairs of points in $V$
  are not covered by any edge and therefore,
  \[
     \asf(G) \ge 1 + \binom{n}{2}^{-1}\cdot\frac{\binom{n}{2}}{2}\cdot
          \Omega(1/\sqrt{n}) = 1 + \Omega(1/\sqrt{n}) \enspace . \qedhere
  \]
\end{proof}

We remark that the proof of \thmref{lower-bound} is easily modified to
provide a tradeoff between the number of edges of $G$ and the average
spanning ratio.  In particular, if $G$ has $m\in o(n^2)$ edges, then
\[
   \asf(G) \ge 1 + \Omega(1/\sqrt{m}) \enspace .
\]

\section{Discussion}
\seclabel{discussion}

Our results leave many areas open for further research.  We say that
a graph, $G=(V,E)$, with $|V|=n$, has \emph{good average stretch} if
$\asf(G)=1+o_n(1)$ and $|E|=O(n)$.

We have proven that a good average stretch graph $G=(V,E)$ exists for
any point set $V\subset\R^d$.  Without much effort, the proof can be
made into an $O(n^2\log n)$ time algorithm to construct the graph, $G$
from the point set $V$.

\begin{op}
  Given a point set, $V$, how quickly can one construct a good average
  stretch graph $G=(V,E)$? \note{should we solve this ourselves? at
  least in $\R^2$?}
\end{op}

The following open problems have to do with strengthenings of
\thmref{upper-bound} in which $G$ has additional properties.

\begin{op}
  Given a point set, $V$, does there always exist a good average stretch
  graph $G=(V,E)$ whose total edge length is close to that of the minimum
  spanning tree?
\end{op}

\begin{op}
  Given a point set, $V$, does there always exist a good average stretch
  graph $G=(V,E)$ whose maximum degree is bounded by a constant?
\end{op}

\begin{op}
  Given a point set, $V$, does there always exist a good average stretch
  graph $G=(V,E)$ that is $k$-fault tolerant? That is, for any set
  $F\subset V$, $|F|\le k$, $\asf(G\setminus K)=1+o_n(1)$.
\end{op}

\begin{op}
  Bose \etal\ \cite{S} define $f(k)$-robust spanners in terms of the
  (worst-case) spanning ratio and their definition extends naturally
  to average spanning ratio.  Given a point set, $V$, does there always
  exist a good average stretch graph $G=(V,E)$ that is $f(k)$-robust?
\end{op}


The following questions ask in how general of a setting
\thmref{upper-bound} can be proved:

\begin{op}
  Given a weighted graph, $H=(V,E)$ with $|V|=n$ vertices, does $H$
  always have a subgraph
  $G=(V,E')$ with $|E'|\in O(n)$ and such that
  \[
     \asf_H(G) = \binom{n}{2}^{-1}\sum_{\{u,w\}\in V}\frac{\|uw\|_G}{\|uw\|_H}
         = 1+o_n(1) \enspace ?
  \]
\end{op}

\begin{op}\oplabel{metric-space}
  Given a metric space $(V,d)$, does there always exist a graph $G=(V,E)$,
  $|E|\in O(n)$ with $\asf(G)=1+o_n(1)$?  (Here shortest paths in $G$
  are measured in terms of the cost of their edges in the metric space.)
\end{op}

The techniques used to prove \thmref{upper-bound} seem applicable to
metric spaces of bounded doubling dimension \cite{x}.\note{reference?}
Is this extra restriction on the metric spaces necessary?   The results of
Abraham \etal\ \cite{abraham.bartal.ea:metric,abraham.bartal.ea:embedding}
show that, for arbitrary metric spaces, there exist spanning trees with
average stretch factor $O(1)$.  It seems conceivable that, even for
arbitrary metric spaces, there may exist good average stretch graphs.



\section*{Acknowledgement}

The authors of this paper are partly funded by NSERC and CFI.

\section*{Authors}

\noindent\emph{Vida Dujmovi\'c.}
School of Mathematics and Statistics and Department of Systems and Computer Engineering, Carleton University
%, \texttt{vida@cs.mcgill.ca}

\noindent\emph{Pat Morin} and \emph{Michiel Smid.}
School of Computer Scence, Carleton University
%, \texttt{\{morin,smid\}@scs.carleton.ca}


\bibliographystyle{abbrv}
\bibliography{avgstretch}





\end{document}


