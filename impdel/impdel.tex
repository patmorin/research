\documentclass[lotsofwhite]{patmorin}
\usepackage{graphicx}
\input{pat}

\title{\MakeUppercase{Simplified Delaunay Preprocessing of Imprecise Points}}
\author{NICTA Workshop on Computational Geometry for Imprecise Data}

\begin{document}
\maketitle

\begin{abstract}
We present a simple algorithm for preprocessing a set of $n$ imprecise
points modeled as disjoint unit disks so that the Delaunay triangulation of
any precise realization of these imprecise points can be computed in
$O(n)$ time.  This simplifies a previous proof of this result by
L\"offler and Snoeyink.
\end{abstract}

\section{Introduction}

Let $D=\langle D_1,\ldots,D_n\rangle$ be a sequence of disjoint unit
disks in the plane.  L\"offler and Snoeyink \cite{ls08} show how to
preprocess $D$ in $O(n\log n)$ time and using $O(n)$ space so that,
for any sequence of points $P=\langle p_1,\ldots,p_n\rangle$, with
$p_i\in D_i$, for each $i\in\{1,\ldots,n\}$, the Delaunay triangulation
of $P$ can be constructed in $O(n)$ time.  In this brief note we offer
an alternative proof of this result.

The preprocessing algorithm we present is (arguably) simpler,
requiring only the computation of a Delaunay triangulation of $O(n)$
points.  The reconstruction method is certainly simpler, requiring
only a result on splitting a Delaunay triangulations in linear time
\cite{cdhmst02}, whereas the method of L\"offler and Snoeyink requires
an application of Chazelle's famously complex linear time polygon
triangulation algorithm \cite{c90}.  The price we pay for this
simplicity is that the realization algorithm is randomized and its
running time is therefore only $O(n)$ in expectation, whereas
L\"offler and Snoeyink's original algorithm is deterministic.

\section{The Preprocessing}

Let $c_i$ denote the center of disk $D_i$ and let $D_i^r$, for $r >0$,
denote the disk centered at $c_i$ and having radius $r$.  The
preprocessing algorithm creates a point set $P'$ as follows:  For each
disk $D_i$, $P'$ contains one point at $c_i$ and 7 points equally
spaced on the boundary of $D_i^2$. The Delaunay triangulation $DT(P')$
is then computed and stored.  Since there are only $8n$ points in
$P'$, this preprocessing is easily accomplished in $O(n\log n)$ time
using any existing $O(n\log n)$ algorithm for Delaunay triangulation
(see, for example, the textbook by de~Berg \etal\
\cite[Section~9]{bcko08}).

The following lemma will be used several times in proving that the
Delaunay reconstruction algorithm runs in linear time.

\begin{lem}\lemlabel{fewpoints}
Any disk $K$ of radius $r$ contains at most $8(r+3)^2$ points of $P'$.
\end{lem}

\begin{proof}
Let $c$ denote the center of $K$.  Any point of $P'$ in $K$ is
generated by some disk $D_i$ with $\|c -
c_i\| \le r+2$.  The number of such $D_i$ is no greater than the
number of disjoint unit disks that can be packed into a disk of radius
$r+3$.  Since each unit disk has area $\pi$ and a disk of radius $r+3$
has area $\pi (r+3)^2$, this quantity is certainly no greater than
$(r+3)^2$.  Therefore, at most $(r+3)^2$ disks each contribute at most
$8$ points to $P'$ that are contained in $K$, as required.
\end{proof}


\section{The Reconstruction}

When given the sequence $P=\langle p_1,\ldots,p_n\rangle$ of points we
construct the Delaunay triangulation of these points by inserting them
into the Delaunay triangulation $DT(P')$ to obtain the Delaunay
triangulation $DT(P'\cup P)$. We then apply the algorithm of Chazelle
\etal\ \cite{cdhmst02} for splitting a Delaunay triangulation to remove the
elements of $P'$ and obtain the Delaunay triangulation of $P$.

The insertion of the points in $P$ proceeds as follows: For each point
$p_i$ we perform a (breadth-first or depth-first) search among the
triangles of $DT(P')$ starting at any triangle incident to $c_i$ and
never visiting a triangle that does not intersect $D_i$.  This search
terminates when we find a triangle $t_i$ that contains $p_i$.  At
this point we insert $p_i$ into $DT(P')$ by making it adjacent to the
three vertices of $t_i$ and then performing \emph{Delaunay flipping}.
See, for example, de Berg \etal\ \cite[Section~9.3]{bcko08} for
details.  The time taken to insert $p_i$ is proportional to the number
of triangles visited before finding $t_i$ plus the degree of $p_i$ in
the Delaunay triangulation of $P'\cup\{p_1,\ldots,p_i\}$.

The next lemma allows us to bound the various quantities
associated with the running time of the reconstruction algorithm.

\begin{lem}\lemlabel{noedge}
Let $P''=P'\cup X$ where $X$ is any point set and consider the
Delaunay triangulation $DT(P'')$.  Then, for any point $p\in P''\cap
D_i$, all neighbours of $p$ in $DT(P'')$ are contained $D_i^3$.
\end{lem}

\begin{proof}
Suppose, for the sake of contradiction, that this is not the case and
that there is an edge $pq$ with $p\in D_i^1$ and $q\not\in D_i^3$.
Refer to \figref{noedge}.  Then there is a disk $C$ have $p$ and $q$
on its boundary and having no point of $P''$ in its interior.  The
disk $C$ contains a (generally smaller) disk $C'$ that is tangent to
$D_i^1$ and to the boundary of $D_i^3$.  Consider the intersection of
$C'$ with the boundary of $D_i^2$.  This intersection is a circular
arc of length $8\arcsin(1/4) > 4\pi/7$.  But this is a contradiction
since one of the $7$ points on the boundary of $D_i^2$ must be
contained in $C'$ and also in $C$, thereby contradicting the
assumption that $C$ is empty.
\end{proof}

\begin{figure}
  \begin{center}
    \begin{tabular}{cc}
      \includegraphics{noedge-a} & \includegraphics{noedge-b}
    \end{tabular}
  \end{center}
  \caption{The proof of \lemref{noedge}.}
  \figlabel{noedge}
\end{figure}

The next 2 lemmata bound the number of triangles visited while
inserting $p_i$.

\begin{lem}\lemlabel{notriangle}
Any triangle of $DT(P'\cup\{p_1,\ldots,p_i\})$ that intersects $D_i$
has all three vertices in $D_i^3$.
\end{lem}

\begin{proof}
Let $t$ be a triangle that intersects $D_i$ and contains a vertex
$q$ outside of $D_i^3$.  Then $t$ is contained in a disk $C$ that
intersects $D_i$, has $q$ on its boundary, and contains no point of
$P'\cup\{p_1,\ldots,p_i\}$ in its interior.  By arguing
exactly as in \lemref{noedge} one can show that $C$ contains one of
the 7 points on the boundary of $D_i^2$, thereby contradicting the
assumption that $t$ is a Delaunay triangle.
\end{proof}

\begin{lem}\lemlabel{triangles}
The number of triangles of $DT(P'\cup\{p_1,\ldots,p_i\})$ that
intersect $D_i$ is at most 644.
\end{lem}

\begin{proof}
The set of triangles that intersect $D_i$ form the set of faces of a
planar graph.  By \lemref{notriangle} every vertex of this graph is in
$D_i^3$, so by \lemref{fewpoints}, the number of vertices 
is at most $v=9(3+3)^2=324$, so the number of faces is at most
$2v-4=644$. 
\end{proof}

The final lemma bounds the degree of $p_i$ at the time it is inserted.

\begin{lem}\lemlabel{degree}
The degree of $p_i$ in $DT(P'\cup\{p_1,\ldots,p_i\})$ is at most 324.
\end{lem}

\begin{proof}
By \lemref{noedge}, all the neighbours of $p_i$ are points in $D_i^3$
and by \lemref{fewpoints} there are at most $9(3+3)^2=324$ points
of $P'\cup\{p_1,\ldots,p_i\}$ in $D_i^3$.
\end{proof}

Thus, by \lemref{triangles} and \lemref{degree} the time to insert
each point of $P$ is $O(1)$ so the reconstruction algorithm requires
$O(n)$ time to construct $DT(P'\cup P)$.  A further $O(n)$ time is
then used to split this to obtain $DT(P)$ using the algorithm of
Chazelle \etal.  This completes the proof of the main theorem.

\begin{thm}
Let $D=\langle D_1,\ldots,D_n\rangle$ be a sequence of disjoint unit
disks in the plane.  After preprocessing requiring $O(n\log n)$ time
and using $O(n)$ space, the Delaunay triangulation of $P=\langle
p_1,\ldots,p_n\rangle$ can be constructed in $O(n)$ expected time, for any
sequence of points $P$ such that $p_i\in D_i$ for all
$i\in\{1,\ldots,n\}$.
\end{thm}

\section*{Acknowledgements}

This research in this paper took place at the NICTA Workshop on
Computational Geometry for Imprecise Data, December 18--22, 2008 at
the NICTA University of Sydney Campus.  We would like to thank the
other workshop participants, namely
Sang~Won~Bae,
Joachim~Gudmundsson,
Allan~J\o rgensen,
\ldots,
Maarten L\"offler,
Marc~Scherfenberg,
and
Thomas~Wolle
for helpful discussions and for providing a stimulating working
environment.

\bibliographystyle{plain}
\bibliography{impdel}


\end{document}
