\documentclass{elsart}
\usepackage{graphicx,ipe}
%\usepackage{amsthm}

\newcommand{\comment}[1]{}

% new "theorem" styles 
\newtheorem{rul}{Rule}
\newtheorem{theorem}{Theorem}
\newtheorem{lemma}{Lemma}
\newenvironment{proof}{\emph{Proof.}}{\qed}
% Title and reasearch grant information

\comment{
\title{\MakeUppercase{The Geometry of Carpentry and Joinery}%
  \thanks{This research was funded
    by Natural Sciences and Engineering Research Council of Canada, NCE-GEOIDE
    and the Government of Ontario.}}
\author{Pat Morin and Jason Morrison \\[1ex] 
	School of Computer Science, Carleton University \\
	1125 Colonel By Drive, Ottawa, CANADA, K1S 5B6 \\
	\texttt{\{morin,morrison\}@cs.carleton.ca}}
\date{}
}

\begin{document}
\begin{frontmatter}
\journal{Discrete Applied Mathematics}
\title{The Geometry of Carpentry and Joinery\thanksref{nserc}}
\thanks[nserc]{This research was funded by the Natural Sciences and
Engineering Research Council of Canada, NCE-GEOIDE and the Government
of Ontario.}
\author{Pat Morin} and
\author{Jason Morrison}
\address{School of Computer Science, Carleton University, 1125 Colonel
By Drive \\ Ottawa, Ontario, CANADA K1S~5B6}
\begin{abstract}
  In this paper we propose to model a simplified wood shop.  Following
  the work of {\mbox{Demaine}, \mbox{Demaine} and \mbox{Kaplan}}\ in
  \cite{ddk02} we limit the cutting tools of our carpenter to a
  circular saw.  We extend that previous work to include a model of
  basic rules of carpentry and joinery.  This model is then applied to
  the problem of building a polygon $P$ by joining together strips of
  wood and cutting them with a circular saw.  We describe a linear
  time algorithm to decide if a blueprint can be constructed in such a
  workshop.
\end{abstract}

\end{frontmatter}

%%%%%%%%%%%%%%%%%%%%%%%%%%%%%%%%%%%%%%%%%%%%%%%%%%%%%%%%%%%%%%%%%%%%%%%%
\section{Introduction}

{\mbox{Demaine}, \mbox{Demaine} and \mbox{Kaplan}} \cite{ddk02} study
the problem of cutting a polygon $P$ from a convex polygon $Q$ that
contains $P$ using a \emph{circular saw}.  In their model, a circular
saw is represented by a line segment of positive length $r$, called
the \emph{radius} of the saw.  A \emph{cut} is a line segment $s$,
disjoint from the interior of $P$ such that $s\setminus Q$ contains a
line segment whose length is at least $r$.  When one or more cuts
disconnects $Q$, the component(s) not containing $P$ are removed to
obtain a new polygon $Q'$, on which further cuts can be made.

Figure~\ref{fig:circular-cutting} illustrates how all this is
analogous to cutting a shape from a piece of plywood with a circular
saw by making successive cuts and removing the parts of the plywood
that become disconnected from the main form.  This model is a
reasonable mathematical abstraction of several types of hand-held and
tabletop saws whose blade is circular and must be spinning before it
enters the material to be cut.

\begin{figure}
  \begin{center}
  \IpeScale{90}
  \begin{tabular}{c@{\hspace{1cm}}c}
  \Ipe{circ1} & \Ipe{circ2} \\[1cm]
  \Ipe{circ3} & \Ipe{circ4} 
  \end{tabular}
  \end{center}
  \caption{Cutting a form with a circular saw.}
  \label{fig:circular-cutting}
\end{figure}

We say that a polygon $P$ is \emph{cuttable with a circular saw of
radius $r$}, if for any convex polygon $Q$ containing $P$, there
exists a finite sequence of cuts $c_1,\ldots,c_k$ resulting in a
sequence of polygons $Q=Q_0,Q_1,\ldots,Q_k=P$ where $Q_{i}$ is
obtained from $Q_{i-1}$ via cut $c_i$.  More simply, we say that $P$
is \emph{cuttable by a circular saw} if there exists an $r>0$ such
that $P$ is cuttable with a circular saw of radius $r$.

A \emph{reflex vertex} of $P$ is any vertex whose internal angle is
greater than $\pi$.  The main result of {\mbox{Demaine},
\mbox{Demaine} and \mbox{Kaplan}} is
\begin{theorem}[{\mbox{Demaine}, \mbox{Demaine} and \mbox{Kaplan}}]
  \label{thm:ddk}
  A polygon $P$ is cuttable by a circular saw if and only if $P$ does not have
  two consecutive reflex vertices on its boundary.
\end{theorem}

In this paper we study what happens to Theorem~\ref{thm:ddk} when the
model is extended using some basic knowledge of carpentry.  The basics
we speak of encompass two fundamental areas, aesthetics and robust
design.  The aesthetic qualities state that in making something from
wood it must be made to look as good as possible.  Robust design
implies that a design should eliminate as many possible sources of
error as is feasible.

Aesthetics criteria imply that only quality wood can be used and thus
plywood and particle board are out.  Thus the large convex sheet $Q$,
mentioned by {\mbox{Demaine}, \mbox{Demaine} and \mbox{Kaplan}}, must
be made by joining smaller pieces.  Also, wood pieces joined together
whose grains have different orientations become a single piece which
cannot be sanded.  Thus the wood grain must have a specific
orientation and desired polygons cannot be joined together with
arbitrary pieces.

So that our assembly process is robust, we require that all of the
pieces that form an edge of the polygon $P$ must be joined together
before that edge is cut.  The rationale for this is to suppose that
the pieces are cut before joining, then any portion of the cutting or
joining process can cause the pieces not to be flush in the final
product.  This is in direct contradiction to the definition of robust
design and therefore shows the need for the restriction.

With these ideas in mind it is clear that we study the case in which
$P$ is created by joining together regular strips (rectangles) at
their edges and cutting the joined pieces with a circular saw.  In
this process, there are two rules that must be obeyed.  The first is
that once two strips are joined together they cannot be unjoined.  The
second is that, before cutting an edge $e$ of $P$, all strips incident
on $e$ must be joined together.  This model appears to be a reasonable
facsimile of the process that creates many tabletops, desktops and
wood floors.

Any polygon that can be cut with a circular saw can be fabricated by
cutting and joining.  This is because all of the wood strips can be
joined into one large sheet and then cut.  However, the converse is
not true.  Many polygons exist that cannot be cut from a large sheet
using a circular saw that can be built by cutting and joining. An
example is given in Figure~\ref{fig:cut-and-join}.  This is due to the
fact that we can cut parts of the polygon individually and then join
them together.

\begin{figure}
  \begin{center}
  \begin{tabular}{c@{\hspace{1cm}}c}
  \raisebox{.18cm}{\Ipe{desktop1}} & \Ipe{desktop2} \\
  1 & 2 \\ \\[0.5cm]
  \Ipe{desktop3} & \Ipe{desktop4} \\
  3 & 4
  \end{tabular}
  \end{center}
  \caption{Building a polygon by cutting and joining strips of wood.}
  \label{fig:cut-and-join}
\end{figure}

A \emph{blueprint} $B=(P,C)$ is a polygon $P$, with $n$ edges, and a
set of $m$ vertical line segments $C$, each of which is contained in
$P$ and has both endpoints on the boundary of $P$. The elements of $C$
are called \emph{chords} of $P$. In computational terms, a blueprint
is represented as a subdivision of $P$ induced by the chords in
$C$. The chords in $C$ partition and the edges of $P$ partition the
interior of $P$ into maximally connected regions called \emph{faces}.

A \emph{join} is the process of removing a chord $c$ from $C$, thereby
merging the two faces incident on $c$.  A join models the joining of
two pieces of wood to form another, larger, piece of wood.

A \emph{cut} is simply an edge $e$ of $P$.  Since a cut is intended to
model the process of cutting an edge $e$ of $P$, it must satisfy the
following two rules:

\begin{rul}\label{rule:straight-cut}
  The edge $e$ must be on the boundary of only one face $f$.
\end{rul}

\begin{rul}\label{rule:ddk}
  At least one endpoint of $e$ must be a non-reflex vertex in $f$.
\end{rul}

Rule~\ref{rule:straight-cut} models the constraint that all strips
incident on $e$ must be merged together through a sequence of joins
before cutting $e$. Rule~\ref{rule:ddk} comes from
Theorem~\ref{thm:ddk} and the assumption that our cutting tool is a
circular saw.

A \emph{construction} $\mathcal{C}=v_1,\ldots,v_{n+m}$ of $B$ is a
sequence of joins and cuts in which each edge of $P$ and each chord of
$C$ appears exactly once.  We say that $B$ is \emph{feasible} if it
has a construction. Figure~\ref{fig:cut-and-join-two} shows a
construction of the desktop from Figure~\ref{fig:cut-and-join}.  Note
that a construction of $B$ only describes the order in which chords
are joined and edges are cut.  It does not actually provide a plan for
cutting the non-reflex edges of a face preceding a join.  It is
possible to compute such a plan, in linear time \cite{ddk02}.

\begin{figure}
  \begin{center}\begin{tabular}{c@{\hspace{1cm}}c}
  \Ipe{desktop1a} & \Ipe{desktop2a} \\
  1 & 2 \\ \\[0.5cm]
  \Ipe{desktop3a} & \Ipe{desktop4a} \\
  3 & 4 
  \end{tabular}\end{center}
  \caption{Illustrating the construction
    $c_3,e_1,e_2,e_4,e_5,e_6,e_8,c_2,e_3,c_1,e_7$ of the desktop from
    Figure~\ref{fig:cut-and-join}.}
  \label{fig:cut-and-join-two}
\end{figure}

In this paper we give a linear time algorithm to determine whether a
blueprint is feasible.  Section~\ref{sect:testing} describes our
algorithm for determining whether a blueprint is
feasible. Section~\ref{sect:design} considers the problem of designing
blueprints for a given polygon.  Section~\ref{sect:conclusions}
summarizes and concludes with directions for future research.

%%%%%%%%%%%%%%%%%%%%%%%%%%%%%%%%%%%%%%%%%%%%%%%%%%%%%%%%%%%%%%%%%%%%%%%%
\section{Testing Feasibility}
\label{sect:testing}

In this section, we study the problem of determining whether a
blueprint $B=(P,C)$ is feasible and give an algorithm for finding a
construction of $B$ when it is feasible.  Because of
Rule~\ref{rule:ddk}, it is intuitively clear that consecutive reflex
vertices will be the primary obstacle in finding a construction of
$B$.  Therefore, we call an edge $e$ of $P$ a \emph{reflex edge} if
both endpoints of $e$ are reflex vertices.

Our algorithm is divided into two steps which are discussed in the
next two subsections. In the first step we attempt to determine, for
each reflex edge $e$, the direction the circular saw will travel when
$e$ is cut.  In the second step, we find an ordering of the joining
and cutting operations that gives us a construction of $B$.

We first observe that chords in $C$ that have both endpoints strictly
in the interior of edges in $P$ are redundant, since nothing is lost
by removing those chords immediately, and they must be removed
(joined) before either of their incident edges are cut.  Therefore, we
assume $C$ does not contain any chords with both endpoints in the
interior of edges of $P$.

We begin by adding \emph{Steiner} chords to our blueprint so that each
face of the blueprint becomes a trapezoid.  These Steiner chords are
obtained by shooting vertical rays up and down from every vertex $v$
in polygon $P$ (see Figure~\ref{fig:steiner-chords}).  We denote by
$C'$ the set of all Steiner chords and observe that, by the assumption
of the previous paragraph, $C\subseteq C'$.  For clarity we say a
chord is a \emph{real chord} if it is in both $C$ and $C'$ and all
other chords in $C'$ are \emph{false chords}.  

We will show how to find a construction of $B'=(P,C')$ with the
additional restriction that each false chord in $C'$ must appear
before the each of the edges incident on it.  Once this is done, we
can easily obtain the construction of $B$ from the construction of
$B'$.

\begin{figure}
  \begin{center}
%  \IpeScale{75}
  \begin{tabular}{c@{\hspace{1cm}}c}
    \raisebox{.18cm}{\Ipe{steiner-psc}} & \Ipe{steiner-ps} \\
    a) Chords $C$ & b) Steiner Chords \\
  \end{tabular}
  \end{center}
  \caption{Two different chord sets in a polygon.}
  \label{fig:steiner-chords}
\end{figure}


%=====================================================================
\subsection{Directing Reflex Edges}

Let $e=(u,v)$ be an edge of $P$.  Then, during a construction
$\mathcal{C}$ we say that $e$ is cut in direction $\overleftarrow{uv}$
(equivalently, $\overrightarrow{vu}$) if the joining of a chord
incident on $u$ is performed before cutting $e$.  The following lemma
states that we can join the chords of at most one endpoint of a reflex
edge before cutting that edge.

\begin{lemma}\label{lem:bidirected}
  There is no construction of $B'$ in which a reflex edge $e=(u,v)$ is cut in
  direction $\overrightarrow{uv}$ and in direction $\overleftarrow{uv}$.
\end{lemma}

\begin{proof}
  Saying that $e$ is cut in both directions is equivalent to saying that the
  chords incident on both endpoints of $e$ are joined before $e$ is
  cut. However, once these two chords are joined $e$ is a reflex edge on the
  face containing $e$ and, by Rule~\ref{rule:ddk}, cannot be cut.
\end{proof}

An \emph{overlap} consists of two edges $e_1=(u,v)$ and $e_2=(w,x)$ and two
chords $c_v$ and $c_w$ such that $c_v$ is incident on $v$ and on the interior
of $e_2$ and $c_w$ is incident on $w$ and on the interior of $e_1$ (see
Figure~\ref{fig:overlap}).  The following lemma shows that reflex edges which
overlap have constraints on the directions in which they can be cut.

\begin{figure}
  \begin{center}
  \IpeScale{80}
  \Ipe{overlap}
  \end{center}
  \caption{An overlap.}
\label{fig:overlap}
\end{figure}

\begin{lemma}\label{lem:direction}
  Let $e_1=(u,v)$, $e_2=(w,x)$, $c_v$ and $c_w$ form an overlap.
  Then, in any construction of $B'$ in which $e_1$ is cut in direction
  $\overleftarrow{uv}$, $e_2$ must be cut in direction
  $\overleftarrow{wx}$.
\end{lemma}

\begin{proof}
  Suppose that this were not the case, and that there is a construction
  $\mathcal{C}$ of $B'$ in which $e_1$ is cut in direction $\overleftarrow{uv}$
  and $e_2$ is cut in direction $\overrightarrow{wx}$.  Then, by
  Rule~\ref{rule:straight-cut}, $c_w$ must be joined before $e_1$ is cut. By
  Rule~\ref{rule:ddk}, $e_2$ must be cut before $c_w$ is joined. By
  Rule~\ref{rule:straight-cut}, $c_v$ must be joined before $e_2$ is
  cut. Finally, by Rule~\ref{rule:ddk}, $e_1$ must be cut before $c_v$ is
  joined.  If we use the notation $a\prec b$ to denote that $a$ must occur
before $b$ in the construction then we have
\[ e_1 \prec c_v \prec e_2 \prec c_w \prec e_1 \enspace ,\] 
an impossibility.
\end{proof}

Lemma~\ref{lem:direction} provides a method for assigning directions
to the reflex edges of $P$. We assign directions to edges of $P$ using
the following algorithm.

\begin{enumerate}
\item If $e=(u,v)$ is a reflex edge with both endpoints on false
  chords then, by Lemma~\ref{lem:bidirected}, there is no construction
  of $B'$.

\item If $e=(u,v)$ is a reflex edge having only the endpoint $u$ on a
  real chord then, by Lemma~\ref{lem:bidirected}, any construction
  of $B'$ cuts $e$ in direction $\overrightarrow{uv}$.

\item Finally, we iterate the following procedure until no more
  directions are assigned: For every reflex edge $e_2=(w,x)$. If $e_2$
  overlaps an edge $e_1=(u,v)$ which has already been assigned the
  direction $\overleftarrow{uv}$ and which satisfies the conditions of
  Lemma~\ref{lem:direction} then we set the direction of $e_2$ to be
  $\overleftarrow{wx}$.  If at any time this procedure attempts to
  reassign a different direction to an edge whose direction has
  already been assigned then we can terminate since, by repeated
  applications of Lemmas~\ref{lem:bidirected} and
  \ref{lem:direction}, $B'$ is not feasible.

\item Once Step~3 is complete, any reflex edge $e=(u,v)$ that has not
  yet had a direction assigned to it is assigned the direction
  $\overrightarrow{uv}$ if the $x$-coordinate of $u$ is less than the
  $x$-coordinate of $v$ and $\overleftarrow{uv}$ otherwise, so that
  all such edges are directed ``left-to-right.''
\end{enumerate}

If the above algorithm succeeds in assigning directions to all edges
of $P$ then we say that the assignment of directions is
\emph{consistent}. The ability to consistently assign directions of
reflex edges is a necessary condition for $B'$ to be feasible, since
the only points at which the above algorithm fails to assign
consistent directions (Steps~1 and 3) provide a proof that $B'$ is not
feasible.  However, we have not yet proven that it is a sufficient
condition because we must also show that there exists a consistent
ordering among the cut and join operations.  This is the topic of the
next section.


%=====================================================================
\subsection{Ordering Joins and Cuts}

Define the directed graph $G(B')=(V,E)$ as follows:

\begin{enumerate}
\item The vertex set $V$ consists of each edge of $P$ and each chord
  of $C'$.
\item The edge $(c,e)$ is present in $E$ if chord $c\in C'$ is
  false and has an endpoint on $e\in P$.
\item The edge $(c,e)$ is present in $E$ if chord $c\in C'$
  has an endpoint strictly in the interior of $e\in P$.
\item The edge $(e,c)$ is present in $E$ if $e=(u,v)$, $c$ has an
  endpoint on $u$ and the direction of $e$ is $\overleftarrow{uv}$.
\end{enumerate}

An example of a blueprint $B'$ for a polygon with one reflex edge
along with the corresponding graph $G(B')$ is shown in
Figure~\ref{fig:graph}.

\begin{figure}
  \begin{center}
  \IpeScale{80}
  \Ipe{graph}
  \end{center}
  \caption{A blueprint $B'$ and the corresponding graph $G(B')$.  Real chords are drawn as solid lines and false chords are drawn as dotted lines.  Each edge is labelled with rule(s) (1--4) that generated it.}
  \label{fig:graph}
\end{figure}

The following is the main result of this section.

\begin{theorem}\label{thm:testing}
  $B'$ is feasible if and only if the reflex edges of $P$ can be
  assigned consistent directions and $G(B')$ has no cycle.
\end{theorem}

\begin{proof}
  We have already shown that if we cannot assign consistent directions
  to the reflex edges of $P$ then $B'$ is not feasible.  Therefore,
  suppose we can assign consistent directions to the edges of $P$ but
  that $G(B')$ has a cycle. First note that, since $P$ is a polygon
  and $C'$ forms a trapezoidation of $P$, any cycle of length greater
  than 4 must have repeated vertices.  Therefore, $G(B')$ contains a
  cycle of length exactly 4.  This cycle includes 2 edges $e_1$ and
  $e_2$ of $P$ and two chords $c_v$ and $c_w$ in $C'$.  The edges
  $e_1$ and $e_2$ and chords $c_v$ and $c_w$ form an overlap, thereby
  contradicting Lemma~\ref{lem:direction} (see
  Figure~\ref{fig:cycle}).
  
  \begin{figure}
    \begin{center}
    \IpeScale{80}
    \Ipe{cycle}
    \end{center}
    \caption{A simple cycle in $G(B')$ corresponds to an overlap that
      violates the conditions of Lemma~\ref{lem:direction}.}
    \label{fig:cycle}
  \end{figure}
  
  We claim that the directions of $e_1$ and $e_2$ were not assigned
  during Step~4 of the algorithm for assigning directions.  Clearly,
  one of them, say $e_1$, was not assigned during Step~4 otherwise
  they would both be directed ``left-to-right'' and would not violate
  Lemma~\ref{lem:direction}.  But then $e_2$ would also have had its
  direction assigned in Step~3.  Therefore, in any construction of
  $B$, $e_1$ and $e_2$ must be assigned those directions. But then, by
  Lemma~\ref{lem:direction}, there can be no construction of $B'$.
  
  It remains to show the converse, i.e., if directions can be
  consistently assigned to reflex edges of $P$ and $G(B')$ is acyclic
  then there exists a construction of $B'$.  Suppose therefore that
  there is a consistent assignment of directions and $G(B')$ does not
  contain a cycle. Then we topologically sort $G(B')$ to obtain a
  total ordering $v_1,\ldots,v_m$ on $V$, i.e., the edges of $P$
  and the chords in $C'$.  We claim that there exists a construction
  of $B$ in which the cuts and joins occur in the order in which they
  appear in $v_1,\ldots,v_m$.
  
  We prove this by showing that, by the time an edge $e$ of $P$ appears in the
  sequence $v_1,\ldots,v_{O(n)}$, $e$ it is entirely contained on the boundary
  of a single face $f$ and is not a reflex edge on $f$.  Therefore, the
  construction satisfies Rules~1 and 2.
  
  Because of the edges added during Step~1 in the construction of $G(B)$, all
  chords incident on $e$ must appear before $e$ in the total order. Therefore,
  Rule~\ref{rule:straight-cut} is satisfied.
  
  Next we need to show that the edge $e$ is not reflex in $f$.  If $e$ is not a
  reflex edge in $P$ then $e$ is still not a reflex edge in $f$. If $e$ is a
  reflex edge in $P$ then it has been assigned a direction, and during Step~4
  in the construction of $G(B')$ an edge was added that guarantees $e$ appears
  in the total order before one of the chords, say $c_v$, incident on one of
  the endpoints, say $v$, of $e$.  Therefore, on the face $f$, $v$ is not a
  reflex vertex and $e$ is not a reflex edge.  Thus, Rule~\ref{rule:ddk} is
  also satisfied.
\end{proof}

If $P$ is a polygon with $n$ edges then $C'$ contains $O(n)$ chords
and there is an algorithmic version of Theorem~\ref{thm:testing} that
runs in $O(n)$ time provided that the blueprint $B'$ is given in a
topological data structure (e.g., a doubly-connected edge list
\cite{ps85}).  If the input is not given in this form, then such a
data structure can be computed in $O(n\log n)$ time using plane-sweep
\cite{bo79}.

Step~1, 2, and 4 of the algorithm for directing reflex edges are easily
implemented in $O(n)$ time.  Step~3 can be implemented as a limited
breadth-first traversal of the graph having reflex edges of $P$ as vertices and
an edge between two vertices if the corresponding edges of $P$ are part of an
overlap. This graph has size $O(n)$ since each reflex edge overlaps at most 2
other reflex edges and hence this step of the algorithm can be completed in
$O(n)$ time.

Once the reflex edges of $P$ have been assigned directions, the graph $G(B')$
can easily be constructed in time linear in the size of $G(B')$.  Since $G(B')$
has $O(n)$ vertices and is planar, the construction of $G(B')$ can be completed
in $O(n)$ time.  Topologically sorting the vertices of $G(B')$ again takes
$O(n)$ time (c.f., \cite{clr90}).

%=====================================================================
\subsection{Summary Notes}

The previous sections provide an algorithm for testing feasibility of
the blueprint $B'=(P,C')$ in $O(n)$ time.  To obtain a construction
for the original blueprint $B=(P,C)$, we observe that we can use the
construction of $B'$ and simply ignore the false chords in the
construction.  This works because Step~1 of the algorithm for computing
$G'$ guarantees that in the construction of $B'$, all false chords appear
before any of their incident edges.


%%%%%%%%%%%%%%%%%%%%%%%%%%%%%%%%%%%%%%%%%%%%%%%%%%%%%%%%%%%%%%%%%%%%%%%%
\section{Designing Blueprints}
\label{sect:design}

The algorithm in Section~\ref{sect:testing} provides a means of
testing whether a given blueprint is feasible.  An obvious question
that arises is that of finding a blueprint for a given polygon $P$.
If we do not place any constraints on our wood strips then designing a
feasible blueprint is trivial.  By adding all of the Steiner chords to
the set $C$ we obtain a feasible blueprint.  To see this, observe that
every reflex edge of $P$ is incident on two chords in $C$ and hence
has its direction assigned in Step~4 of the algorithm for assigning
directions to reflex edges.  Then all reflex edges are directed
``left-to-right,'' so it is clear that $G$ does not contain any
cycles, hence $(P,C)$ is a feasible blueprint.

A more interesting problem arises when we require that the strips of
wood be of a fixed width.  This is a setting that models the
construction of a piece using standard-sized (store-bought) pieces of
wood.  In order to express the constraint of fixed width strips we
assume that the polygon $P$ has been scaled relative to the size of
the strips and that we must design a blueprint in which all chords
have integer $x$-coordinates.  In this way, we can compactly represent
the blueprint $B=(P,C)$ by a translation and rotation of $P$.  This
representation avoids redundant, possibly large, space use.

If only translations of $P$ are allowed (e.g., the grain of the wood
must run in a certain direction) and all chords must lie on integer
$x$ coordinates then there is an optimal algorithm for designing and
testing a blueprint.  This algorithm runs in time $O(n)$ where $n$ is
the number of edges in the polygon $P$.  Note that this is independent
of the number of chords in the final blueprint (which may be much
larger than $n$).

The first problem we encounter is that of computing a blueprint for
$P$ given a particular translation of $P$.  To do this, we first
partition $P$ into trapezoids by finding all Steiner chords.  This can
be done in $O(n)$ time \cite{c91} and gives us a \emph{trapezoidal
decomposition} of $P$.  Each Steiner chord can then be classified as
false or real in constant time, so we obtain the blueprint in $O(n)$
time and test it for feasibility in an additional $O(n)$ time.

To find a feasible blueprint for $P$, we proceed as follows.  If $P$
has no reflex edges, then $P$ is cuttable by a circular saw, so any
blueprint is valid.  If $P$ has at least one reflex edge $e$ then by
Rule~\ref{rule:ddk} in any feasible blueprint $e$ must have at least
one endpoint on a real chord.  Due to the regular spacing of chords we
can select either endpoint of the edge $e$ to lie on a chord and the
remainder of the chords are specified.  Thus there are only two
blueprints that need to be tested.  Each blueprint can be created and
tested for feasibility in $O(n)$ time, yielding an overall running
time of $O(n)$.



%%%%%%%%%%%%%%%%%%%%%%%%%%%%%%%%%%%%%%%%%%%%%%%%%%%%%%%%%%%%%%%%%%%%%%%%%

\section{Conclusions}\label{sect:conclusions}

We have studied the problems related to cutting strips of wood and
joining them together to form a polygon $P$ with $n$ vertices.  We
show that given a blueprint for a polygon with $n$ vertices using $m$
strips of wood, we can test if the blueprint is feasible and, if so,
give a construction in optimal $O(n)$ time. We have also shown that,
if the orientation of the polygon is given, it is possible to decide
if a blueprint exists in which all chords are on integer
$x$-coordinates in $O(n)$ time.

An open problem that remains is that of determining if a blueprint
exists when the orientation of the polygon is not fixed.  In the
preliminary version of this paper, we considered this problem but were
not able to obtain algorithms whose running time was bounded by a
function of $n$ \cite{mm01}.

\bibliographystyle{elsart-num}
\bibliography{carpentry}
\end{document}
