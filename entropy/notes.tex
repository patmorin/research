\documentclass{article}


\title{Notes on Entropy-Sensitive Range Searching}
\author{John Iacono \and Stefan Langerman \and Pat Morin}

\begin{document}
\maketitle

Here are some things we know (and think we know) about
entropy-sensitive range searching.

\section{Interval Searching}

Suppose we are given $n$ real number $x_1,\ldots,x_n$ and a
probability distribution $D$ over $Q=\{(i,j): 1\le i\le j\le n\}$.
Then we obtain a data structure for interval range searching whose
expected running time is $O(H(Q))$ in the following way:  For each
$1\le i\le n$, compute the quantity
\[
	p_i = \frac{1}{2}\left(\sum_{j=1}^i D(j,i) + \sum_{j=i+1}^n D(i,j)\right)
\]
and then build a biased binary search tree on $x_i$ to $x_n$ using the
weights $p_i$.


\section{Alternative Model}

With each point we are given a probability that the point is in the
range.

\section{Halfplane Range Searching}

In this model, there is a set $Q$ of ${n \choose 2}$ possible query
halfspaces and we are given a probability distribution $D$ over $Q$.

\subsection{Range Counting}

For range counting queries, we obtain a data structure with query time
$O(\min\{2^{cH},\sqrt{n}\})$ in the following way:  Dualize our input
points to get a line arrangement $A$ and triangulate $A$.  Each
triangle $t$ in the resulting triangulation is associated with three
vertices and from these we derive the probability of querying $t$.
Take the $n$ highest probability triangles and store them in an
entropy-sensitive point-location structure (Iacono, SODA 200X).  Also
store everything in a partition tree.  The query algorithm first looks
for the result in the point-location structure and then looks in the
partition tree. If $p$ is the probability of finding our answer in the
first structure and $D'$ is the probability distribution $D$
conditioned on finding our answer in the first structure, then a careful analysis of
\[
     H(D') + (1-p)\sqrt{n} = 2^{cH(D)}
\]
will determine the constant $c$.

\subsection{Range Reporting}

Can the convex layers structure be used to obtain an $O(H+k)$ time
data structure for range searching in $2d$?

\subsection{Lower Bounds}

Can Chazelle's lower bounds be carried over to work in our setting?


\end{document}
