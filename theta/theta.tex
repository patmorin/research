\documentclass{elsart}
%\usepackage{charter}
\usepackage{graphicx}
\usepackage{amsfonts}
\journal{CGTA}

\newcommand{\spanratio}[1]{1/(\cos(#1)-\sin(#1))}
\newcommand{\goob}[1]{(1+\epsilon)\ln #1}
\newcommand{\rank}{\mathrm{rank}}
\newcommand{\E}{\mathbf{E}}

\input{pat.tex}

\begin{document}
\begin{frontmatter}
\title{Ordered Theta Graphs\thanksref{nserc}}
\thanks[nserc]{This research was partially supported by NSERC.}

\author[carleton]{Prosenjit Bose},
\author[eindhoven]{Joachim Gudmundsson} and 
\author[carleton]{Pat Morin}
\address[carleton]{School of Computer Science, Carleton University,
	1125 Colonel By Drive, \\ Ottawa, Ontario, Canada, K1S~5B6}
\address[eindhoven]{Faculty of Mathematics and Computer Science,
	TU Eindhoven, \\ 5600 MB Eindhoven, The Netherlands}

\begin{abstract}
Let $V$ be a set of $n$ points in $\mathbb{R}^2$.  The $\theta$-graph
of $V$ is a geometric graph with vertex set $V$ that has been studied
extensively and which has several nice properties.  We introduce a new
variant of $\theta$-graphs which we call \emph{ordered
$\theta$-graphs}.  These are graphs that are built incrementally by
inserting the vertices one by one so that the resulting graph depends
on the insertion order. We show that careful insertion orders can
produce graphs with desirable properties including low spanning ratio,
logarithmic maximum degree and logarithmic diameter.
\end{abstract}
\end{frontmatter}

\section{Introduction}

Let $V$ be a set of $n$ points in the plane.  A \emph{$t$-spanner} of
$V$ is a geometric graph $G=(V,E)$ whose edges are weighted by the
distance between their endpoints and which has the property
\[
  \max\left\{\frac{\|uv\|_G}{\|uv\|} : u,v\in V, u\neq v\right\} 
	\le t \enspace,
\]
where $\|uv\|$, respectively $\|uv\|_G$, denotes the Euclidean
distance, respectively the length of the shortest path in $G$, between
$u$ and $v$.  We call a path $P$ from $u$ to $v$ a \emph{$t$-path} if
$\|uv\|_P/\|uv\|\le t$.  Thus, $G$ is a $t$-spanner if and only if
every pair of vertices in $G$ has a $t$-path between them.

It has long been known that, for any constant $t>1$, every point set
$V$ has a $t$-spanner with $O(n)$ edges.  One such construction is the
$\theta$-graph of $V$ \cite{k88,kg92}.  Let $\theta<\pi/4$ be a value
such that $k_\theta=2\pi/\theta$ is a positive integer.  The
$\theta$-graph of $V$ is obtained by drawing $k_\theta=2\pi/\theta$
non-overlapping cones around each $p\in S$, each spanning an angle of
$\theta$, and connecting $p$ to the point in each cone whose
orthogonal projection onto one of the walls of the cone is closest to
$p$.  The result is a $t_\theta$-spanner with at most $nk_\theta$
edges.  Here, and throughout the rest of the paper,
\[
  t_\theta = \spanratio{\theta} \enspace .
\]

This paper studies a variant of $\theta$-graphs that we call
\emph{ordered $\theta$-graphs}.  An ordered $\theta$-graph of $V$ is
obtained by inserting the points of $V$ in some order.  When a point
$p$ is inserted, we draw the same cones around $p$ and connect $p$ to
its closest previously-inserted neighbour in each cone.  An ordered
$\theta$-graph is dependent on the order imposed on $V$; different
orderings of $V$ can produce different graphs (see \figref{order}).
Nevertheless, in \secref{ographs} we show that ordered $\theta$-graphs
are also $t_\theta$-spanners, regardless of the ordering used.

\begin{figure}
\begin{center}\begin{tabular}{c@{\hspace{1cm}}c}
\includegraphics{theta-a} & \includegraphics{theta-b} 
\end{tabular}\end{center}
\caption{Different orderings lead to different ordered $\theta$-graphs.}
\figlabel{order}
\end{figure}

We also study different properties that can be obtained by carefully
choosing orderings of $V$.  In \secref{small-degree} we show that
every point set has an ordering such that the ordered $\theta$-graph
has maximum degree $O(k_\theta\log n)$.  In \secref{small-diameter} we
show that for every point set there exists an ordering such that, in
the resulting ordered $\theta$-graph, there is a $t_\theta$-path with
$O(\log n)$ edges between every pair of vertices.  We say that such a
graph has $O(\log n)$ \emph{spanner diameter}.

The two results described above are not new.  Sink spanners
\cite{admss95} are a transformation of $\theta$-graphs that achieve a
somewhat stronger result, namely a $t_\theta$-spanner with degree at
most $k_\theta{}^2+k_\theta$.  Skiplist spanners \cite{ams99} use
random-sampling of vertices to obtain a graph with spanner diameter
$O(\log n)$, and $O(nk_\theta)$ edges.  However, ordered
$\theta$-graphs show the existence of spanners with small degree and
spanners with small diameter using a unified approach.  Also, in the
case of small diameter, the proof improves the constants in skiplist
spanners since an ordered $\theta$-graph of $n$ points contains at
most $nk_\theta$ edges.

In \secref{generalizations} we show that two generalizations of
$\theta$-graphs also apply to ordered $\theta$-graphs.  Finally, in
\secref{conclusions} we summarize and conclude with open problems.

\section{Ordered $\theta$-Graphs}\seclabel{ographs}

In this section, we give a formal definition of ordered
$\theta$-graphs and prove some basic properties about them.  Let $V$
be a set of $n$ points in the plane and let $\theta<\pi/4$ be an angle
such that $k_\theta=2\pi/\theta$ is a positive integer.  Define any
total order $\pi$ on $V$ so that $\pi_v$ is the rank of $v$ in this
order.  Let $P_v$ denote the predecessors of $v$ in $\pi$, i.e.,
$P_v=\{u\in V:\pi_u<\pi_v\}$ and let $S_v$ denote the successors of
$v$ in $\pi$, i.e., $S_v=V\setminus(\{v\}\cup P_v)$.

The $\pi$-ordered $\theta$-graph of $V$ is obtained by repeating the
following for each point $v\in V$ (see \figref{theta-cones}).
Partition the plane into $k_\theta$ cones each spanning an angle of
$\theta$ and having their apexes on $v$.  Next, project each point of
$P_v$ orthogonally onto the counterclockwise wall of the cone that
contains it.  Finally, make an edge joining $v$ to the point in each
cone whose projection is closest to $v$.  We call the vertices
connected to $v$ in this way the \emph{$\theta$-neighbours} of $v$. 

\begin{figure}
\begin{center}\includegraphics{theta-fig}\end{center}
\caption{The $\theta$-cones of $v$.}\figlabel{theta-cones}
\end{figure}

\begin{lem}\lemlabel{spanning}
For any point set $V$ and any ordering $\pi$, the $\pi$-ordered
$\theta$-graph $G=(V,E)$ of $V$ is a $t_\theta$-spanner with at most
$k_\theta n$ edges.
\end{lem}

\begin{proof}
It follows immediately from the definition that $G$ has at most
$k_\theta n$ edges.

To prove that $G$ is a $t_\theta$-spanner, consider any pair of points
$u,v\in V$.  We use induction on $\max\{\pi_u,\pi_v\}$.  Without loss
of generality, assume $\pi_u>\pi_v$.  If $\pi_u=2$ then $\pi_v=1$ and
there is a direct edge from $u$ to $v$ so the claim is trivial.
Otherwise, consider the $\theta$-cone $c$ of $u$ that contains $v$ and
let $w$ be the $\theta$-neighbour of $u$ in $c$.  If $w=v$ then we are done.
Otherwise, the projection of $w$ onto the counterclockwise wall of $c$
is closer than the projection of $v$ onto the counterclockwise wall of
$c$.  Simple trigonometry shows that
\begin{equation}
  \|uw\| + t_\theta\|wv\| \le t_\theta\|uv\| \enspace .
  \eqlabel{triangle}
\end{equation}
Since $\pi_w<\pi_u$ we have 
\[
   \|uv\|_G\le \|uw\| + \|wv\|_G \le \|uw\| + t_\theta\|wv\| \le t_\theta\|wv\| \enspace ,
\]
where the second inequality follows from the inductive hypothesis and
the third follows from \eqref{triangle}.  This completes the proof.
\end{proof}

Note that the above proof gives an algorithm for finding a path
between $u$ and $v$ that works by constructing the path from both
ends.  If $\pi_u>\pi_v$ then the second vertex in the path from $u$ to
$v$ is the neighbour $w$ of $u$ that is contained in the same
$\theta$-cone of $u$ as $v$.  Otherwise, the second last vertex in the
path from $u$ to $v$ is the neighbour $w$ of $v$ that is contained in
the same $\theta$-cone of $v$ as $u$.  We call the path produced by
this algorithm the \emph{$\theta$-path} from $u$ to $v$.

While it is nice to know that ordered $\theta$-graphs have good
spanning properties, it is still important to be able to compute them
efficiently.

\begin{lem}\lemlabel{construction}
For any set $V$ of $n$ points in $\mathbb{R}^2$ and any ordering $\pi$
on $V$, the $\pi$-ordered $\theta$-graph of $V$ can be computed in
$O(k_\theta n\log n)$ time.
\end{lem}

\begin{proof}
The crucial part of the construction algorithm is finding the
neighbours of each point $v\in S$.  For this, we use $k_\theta$ range
trees \cite{b78}, one for each cone direction.  In each tree we store
the points of $V$ with their coordinates represented in terms of the
walls of one of the cones, i.e., in a coordinate system in which the
$x$ and $y$ axes meet at an angle of $\theta$ (see \figref{cones}).

Once this coordinate transformation is done, finding the neighbour of
$u$ in a cone $c$ is equivalent to finding the point with minimum $y$
coordinate that has both $x$ and $y$ coordinates larger than the $x$
and $y$ coordinates of $u$.  These \emph{dominance queries} are
exactly the types of queries that are answered by range trees.

\begin{figure}
\begin{center}\includegraphics{cones}\end{center}
\caption{Finding the neighbour of $u$ in $c$ is equivalent to finding
	the point in the upper right quadrant of $u$ with minimum $y$
	coordinate.}
	\figlabel{cones}
\end{figure}

To construct the $\pi$-ordered $\theta$-graph we work backwards
through the sequence $\pi$.  We first choose the point $v\in V$ such
that $\pi_v=n$, delete $v$ from all range trees and then use the range
trees to find the neighbours of $v$.  We continue in this manner for
the point $u\in V$ such that $\pi_u=n-1$ and so on until we have
computed the entire $\pi$-ordered $\theta$-graph of $V$.

Thus, computing the neighbours of each point $v$ involves $k_\theta$
deletions and searches in range trees.  A version of range trees that
supports construction in $O(n\log n)$ time, and queries and deletions
in $O(\log n)$ amortized time is given by Mehlhorn and N\"aher
\cite{mn90}.  Using this implementation of range trees, the above
algorithm constructs the $\pi$-ordered $\theta$-graph in $O(k_\theta
n\log n)$ time.
\end{proof}

%%%%%%%%%%%%%%%%%%%%%%%%%%%%%%%%%%%%%%%%%%%%%%%%%%%%%%%%%%%%%%%%%%%%%%%%
\section{Ordered $\theta$-Graphs with Low Degree}\seclabel{small-degree}

So far, we have established that ordered $\theta$-graphs have the same
spanning properties as $\theta$-graphs and can be constructed
efficiently.  In this section, we show that a carefully chosen
ordering $\pi$ yields a $\pi$-ordered $\theta$-graph in which each
vertex has small degree.


\begin{thm}\thmlabel{small-degree}
Every point set $V$ of size $n$ has an ordered $\theta$-graph in which
each vertex has degree at most $k_\theta(H_{n-1}+1)$.\footnote{Here,
and throughout, $H_m=\sum_{i=1}^m 1/i$ denotes the $m$th harmonic
number.  It is well known that $\ln m \le H_m \le \ln m + 1$
\cite{gkp94}.}  Furthermore, this ordered $\theta$-graph can be
constructed in $O(k_\theta n\log n)$ time.
\end{thm}

\begin{proof} 
We construct the ordering incrementally.  Initally we choose an
arbitrary vertex $v_n\in V$, mark $v_n$ and set $\pi_{v_n}\gets n$.
This determines up to $k_\theta$ edges of the $\pi$-ordered
$\theta$-graph.  In the second step we choose some unmarked vertex
$v_{n-1}\in V$ of degree 1 (this will be a neighbour of $v_n$), mark
$v_{n-1}$ and set $\pi_{v_{n-1}}\gets n-1$.  In general, in the $i$th
step (beginning at $i=0$) we choose an unmarked vertex $v$ of maximum
degree, mark $v$ and set $\pi_v\gets n-i$, thereby fixing up to
$k_\theta$ more edges of $\pi$-ordered $\theta$-graph.

To prove that the above algorithm gives the desired degree bound, we
relate it to the following game: Imagine we have a set of $n$ buckets
and two players $P_1$ and $P_2$.  In one round, $P_1$ removes a bucket
containing the maximum number of balls and $P_2$ adds a total of at
most $k_\theta$ balls to some subset of buckets.  The game ends after
$n$ rounds.  Dietz and Sleator \cite{ds87} show that, no matter what
$P_2$'s strategy is, the maximum number of balls contained in any
bucket at any point during the execution of the game does not exceed
$k_\theta(H_{n-1}+1)$.  This game and the above algorithm for
constructing $\pi$ are completely analagous if we think of the buckets
as $V$, $P_1$ as our algorithm and $P_2$ as the geometry of $V$ that
determines which edges that are fixed each time we fix the rank of a
vertex in $\pi$.  Thus, the result of Dietz and Sleator implies that
the degree of a vertex in the resulting $\pi$-ordered $\theta$-graph
does not exceed $k_\theta(H_{n-1}+1)$, as required.

As in the proof of \lemref{construction}, the above algorithm is
easily implemented to run in $O(k_\theta n\log n)$ time using the
deletion only range tree data structure of Mehlhorn and N\"aher
\cite{mn90}.
\end{proof}

While the bound in the proof of \thmref{small-degree} is optimal for
the pebble game used in the proof, we have no example of a point set
for which every ordering produces an ordered $\theta$-graph with
$\omega(k_\theta)$ degree at some vertex.

Before continuing, we remark that the algorithm in the proof of
\thmref{small-degree} can produce a graph with diameter $n-1$.  This
happens, for example if the point set $V$ lies on a line and the
algorithm begins by choosing one of the extreme points of $V$.

%%%%%%%%%%%%%%%%%%%%%%%%%%%%%%%%%%%%%%%%%%%%%%%%%%%%%%%%%%%%%%%%%%%%%%%%
\section{Ordered $\theta$-Graphs with Logarithmic Diameter}
\seclabel{small-diameter}

In this section, we show that an ordered $\theta$-graph constructed by
choosing a random ordering has $\theta$-paths that use only $O(\log
n)$ edges.

\comment{
\begin{thm}\thmlabel{expected}
Let $V$ be any set of $n$ points and let $\pi$ be a random ordering of
the points in $V$.  Then, for any pair of points $u,v\in S$, the
expected number edges in the $\theta$-path from $u$ to $v$ is no more
than $2+\ln n$, for any $\epsilon>0$.
\end{thm}

\begin{proof}
Let $T_{n,m}$ denote the expected number of edges in the $\theta$-path
from the vertex $u$ such that $\pi_u=n$ to the vertex $v$ such that
$\pi_v=m$, $n>m$.  The $\theta$-path from $u$ to $v$ begins by moving
to the neighbour $w$ of $u$ that is contained in the same
$\theta$-cone of $u$ as $v$.  If $w=v$ then the statement of the
theorem is clearly true.  Otherwise, observe that $\pi_w$ is equally
likely to be any of the $n-2$ values $\{0,\ldots,
m-1,m+1,\ldots,n-1\}$.  Therefore, the expected length of the
$\theta$-path from $u$ to $v$ is bounded by the recurrence
\[
  T_{n,m} \le 1+ \frac{1}{n-2}\left(\sum_{i=1}^{m-1} T_{m,i} + 
	\sum_{i=m+1}^{n-1} T_{i,m} \right) \enspace ,
\]
which we solve as follows:
\begin{eqnarray*}
T_{n,m} & \le & 1 + \frac{1}{n-2}\left(\sum_{i=1}^{m-1} T_{m,i}
		+ \sum_{i=m+1}^{n-1} T_{i,m}\right) \\
   & \le & 1 + \frac{c}{n-2}\left((m-1)(2+\ln m)+
	\sum_{i=m+1}^{n-1}(2+\ln i)\right) \\
   & \le & 1 + \frac{c}{n-2}\left((m-1)(2+\ln m)+
	\sum_{i=m+1}^{n-1}(2+\ln i)\right) \\
   & = & 1 + \frac{c}{n-2}\left((m-1)\ln m+
	\sum_{i=1}^{n-1}\ln i - \sum_{i=1}^m\ln i\right) \\
   & \le & 1 + \frac{c}{n-2}\left((m-1)\ln m+
	\int_{1}^{n}\ln x\, dx - \int_{1}^m\ln x\,dx\right) \\
   & \le & 1 + \frac{c}{n-2}\left((m-1)\ln m+
	n\ln n - n - m\ln m + m \right) \\
   & = & 1 + \frac{c}{n-2}\left(n\ln n - \ln m - n + m \right) \\
   & \le & 1 + \frac{c}{n-2}\left(n\ln n - \ln m - 1 \right) \\
\end{eqnarray*}
which solves to $\goob{n}$, as required.
\end{proof}
}

\begin{thm}
Let $G=(V,E)$ be an ordered $\theta$-graph of $V$ obtained by taking
the points of $V$ in random order.  Then the probability that there
exists a pair $u,v\in V$ such that the $\theta$-path from $u$ to $v$
has more than $c\log n$ edges is $n^{-\Omega(c)}$.
\end{thm}

\begin{proof}
Consider two consecutive steps of the algorithm for finding a
$\theta$-path from $u$ to $v$.  These steps either complete the path,
or reduce the problem of finding a path between $u,v\in V$ to a
problem of finding a path between $w,x\in V$.  We say that two
consecutive steps are \emph{successful} if they complete the path, or
if $\max\{\pi_w,\pi_x\}<\alpha\max\{\pi_u,\pi_v\}$, for some constant
$0<\alpha<1$.  A simple cases analysis shows that the probability that
two consecutive steps are successful is at least $\alpha^2$, and this
statement is true regardless of any previous steps taken by the
algorithm.

Observe that, since the length of a path is bounded by $n$, any run of
the algorithm for finding a $\theta$-path has at most
$\log_{1/\alpha^2}n$ successes.  Therefore, if we let $X$ denote the
number of edges in the $\theta$-path from $u$ to $v$ and let $B$
denote a $\mathrm{binomial}(2c\log_{1/\alpha^2} n, \alpha^2)$ random
variable then
\begin{eqnarray*}
  \Pr\left\{X\ge2c\log_{1/\alpha^2} n\right\} 
  & \le & \Pr\left\{B\le 2\log_{1/\alpha^2} n\right\} \\
  & = & \Pr\left\{B\le \frac{1}{\alpha^2c}\E B\right\} \\
  & \le & \comment{\exp\left(-\left(1-\frac{1}{\alpha^2c}\right)^2\alpha^2c\log_{1/\alpha^2} n\right) \\
  & = &} n^{-\left(1-\frac{1}{\alpha^2c}\right)^2\alpha^2c/\ln(1/\alpha^2)} \\
  & = &  n^{-\Omega(c)} \enspace ,
\end{eqnarray*}
where the second inequality follows from Chernoff's bound on the head
of the binomial distribution.  Thus, the probability that there exists
any pair of vertices $u,v\in V$ such that the $\theta$-path from $u$
to $v$ has more than $2c\log_{1/\alpha^2} n$ edges is at most
${n\choose 2}n^{-\Omega(c)}=n^{-\Omega(c)}$, as required.
\end{proof}

We remark that, unfortunately, a random ordering does not necessarily
produce an ordered $\theta$-graph in which every vertex has low
degree.  For example, consider a point set $V$ such that the
(unordered) $\theta$-graph of $V$ has a vertex $v$ of degree $n-1$.
In this case, the expected degree of $v$ in the randomly ordered
$\theta$-graph is
\[
  \frac{1}{n}\sum_{i=1}^n (n-i) = (n-1)/2 \enspace . 
\]


%%%%%%%%%%%%%%%%%%%%%%%%%%%%%%%%%%%%%%%%%%%%%%%%%%%%%%%%%%%%%%%%%%%%%%%%
\section{Generalizations}\seclabel{generalizations}

In this section we discuss two generalizations of ordered
$\theta$-graphs that follow from the corresponding generalizations of
(unordered) $\theta$-graphs.

%=======================================================================
\subsection{Higher Dimensions}

Ruppert and Seidel \cite{rs91} give a natural generalization of
$\theta$-graphs to $d$-dimensions that can be constructed in
$O(n\log^{d-1} n)$ time and yield a $t_\theta$-spanner with
$O(k_{d,\theta}n)$ edges, where $k_{d,\theta}=(d/\theta)^{O(d)}$.  A
close inspection of the proofs in Sections~\ref{sec:small-degree} and
\ref{sec:small-diameter} will reveal that they make no use of the
dimension of the point set $V$.  Thus, Theorems~{1} and {2} hold also
in $d$ dimensions, with $k_\theta$ replaced by $k_{d,\theta}$.

\begin{thm}
Let $V$ be any set of $n$ points in $\mathbb{R}^d$.  Then $V$ has an
ordered $\theta$-graph in which each vertex has degree at most
$k_{d,\theta}(H_{n-1}+1)$.  Furthermore, this ordered $\theta$-graph
can be constructed in $O(k_{d,\theta} n\log^{d-1} n)$ time.
\end{thm}

\begin{thm}
Let $V$ be a set of $n$ points in $\mathbb{R}^d$ and let $G=(V,E)$ be
an ordered $\theta$-graph of $V$ obtained by taking the points of $V$
in random order.  Then the probability that there exists a pair
$u,v\in V$ such that the $\theta$-path from $u$ to $v$ has more than
$c\log n$ edges is $n^{-\Omega(c)}$.
\end{thm}


\comment{
%=======================================================================
\subsection{Low Weight}

The weight of a geometric graph $G=(V,E)$ is the sum of its edge
weights, i.e., $w(G)=\sum_{(u,v)\in E}\|uv\|$.  A desirable property
of a geometric graph is that it have low weight.  Normally, this means
that the weight of $G$ is a constant multiple of the weight of the
minimum spanning tree of $V$.  

Unfortunately, with ordered $\theta$-graphs, there are some point sets
for which no ordering produces a graph with this property.  Such a
point set is obtained when $V$ consists of $n/2$ pairs of points where
the $i$th pair are at positions $(0,i\epsilon)$ and $(1,i\epsilon)$.
In any ordering of $V$ there is an edge joining each pair of points,
so the total weight of any $\theta$-graph is at least $n/2$.  On the
other hand, by making $\epsilon$ sufficient small, the weight of the
minimum spanning tree of $V$ can be made arbitrarily close to 1.

On the positive side, a general greedy technique can be used to find a
subset of any graph $G$ that is a $t'$-spanner and has weight
proportional to the minimum spanning tree of $G$.
}

\subsection{Fault-Tolerant Ordered $\theta$-Graphs}

We say that a graph $G=(V,E)$ is an \emph{$f$ fault-tolerant
$t$-spanner} if $G\setminus V'$ is a $t$-spanner for any subset
$V'\subseteq V$ of size at most $f$.  Lukovszki \cite{l99} shows that,
if we modify the construction of $\theta$-graphs so that each vertex
connects to $f+1$ vertices in each cone, then we obtain an $f$
fault-tolerant $t_\theta$-spanner with at most $fk_\theta n$ edges.

Applying the same modification to ordered $\theta$-graphs, i.e.,
connecting the vertex $v$ to the nearest $f+1$ elements of $P_v$ in
each cone, yields the same result for ordered $\theta$-graphs.  We
call such graphs $f$-fault tolerant ordered $\theta$-graphs.  The
results of Sections~\ref{sec:small-degree} and
\ref{sec:small-diameter} generalize immediately:

\begin{thm}
Let $V$ be any set of $n$ points in $\mathbb{R}^d$.  Then $V$ has an
$f$-fault tolerant ordered $\theta$-graph in which each vertex has
degree at most $fk_{d,\theta}(H_{n-1}+1)$.  Furthermore, this ordered
$\theta$-graph can be constructed in $O(fk_{d,\theta} n\log^{d-1} n)$
time.
\end{thm}

\begin{thm}
Let $V$ be a set of $n$ points in $\mathbb{R}^d$ and let $G=(V,E)$ be
an ordered $\theta$-graph of $V$ obtained by taking the points of $V$
in random order.  For any $V'\subset V$ of at most $f$ vertices in
$V$, the probability that there exists a pair $u,v\in V\setminus V'$
such that the $\theta$-path from $u$ to $v$ in $G\setminus V'$ has
more than $c\log n$ edges is $n^{-\Omega(c)}$.  It follows that the
probability that there exists any subset $V'\subseteq V$ such that the
$\theta$-path from $u$ to $v$ in $G\setminus V'$ has more than $c\log
n$ edges is $n^{f-\Omega(c)}$.
\end{thm}

\comment{
Combining this modification with the ordering algorithm in
\secref{small-degree} yields an $f$ fault-tolerant $t_\theta$-spanner
in which each vertex has degree $O(fk\log n)$.  Using a random
ordering as in \secref{small-diameter} gives an ordered $\theta$-graph
$G$ with $O(\log n)$ spanner diameter, with high probability.
Furthermore, $G$ continues to have $O(\log n)$ spanner diameter (with
high probability) even after the removal of up to $f$ vertices.
}


\section{Summary and Conclusions}\seclabel{conclusions}

We have defined ordered $\theta$-graphs, a variant of $\theta$-graphs
that allow some flexibility in their construction.  This flexibility
allows us to construct spanners with low degree and spanners with low
spanner diameter, but is close enough to the original definition
$\theta$-graphs that existing generalizations of $\theta$-graphs also
hold for ordered $\theta$-graphs.

We construct ordered $\theta$-graphs by projecting points onto the
walls of cones.  A better spanning ratio of
$(1/(1-2\sin(\theta/2))<t_\theta$ can be obtained if, instead, we
project points onto the central axes of cones.  While it is possible
to do this, the deletion only range tree data structure of Mehlhorn
and N\"aher does not support the types of queries we need to perform.
Using standard range trees increases the runnning time of the
construction algorithm to $O(k_\theta n\log^2 n)$.

\begin{op}
Give an $O(k_\theta n\log n)$ time algorithm for constructing the
ordered $\theta$-graph obtained by projecting points onto central axes
of cones.
\end{op}

Although we have shown that every point set $V$ of size $n$ has an
ordering in which the maximum degree of a vertex in the ordered
$\theta$-graph is $O(k_\theta\log n)$ we do not know if this result is
tight.
\begin{op}
Does every vertex set $V$ have an ordering $\pi$ such that the
$\pi$-ordered $\theta$-graph of $V$ has degree bounded by some
function of $k_\theta$?
\end{op}

There are constructions of spanner that simultaneously have small
spanner diameter and small degree \cite{admss95}.  Is it possible to
obtain similar results using only ordered $\theta$-graphs?


\begin{op}
Does every vertex set $V$ have an ordering $\pi$ such that the
$\pi$-ordered $\theta$-graph has small degree and $O(\log n)$ spanner
diameter?
\end{op}

\section*{Acknowledgements}

This research was started during the Carleton-Utrecht Workshop on
Geometric Networks held in Hilversum, The Netherlands, March 2--7,
2002.  We would like to thank the other workshop participants, namely
Pankaj~Agarwal, Tetsuo~Asano, Mark~de~Berg, Sergio~Cabello,
Otfried~Cheong, Hazel~Everett, Danny~Halperin, Herman~Haverkort,
Marc~van~Kreveld, Giri~Narasimhan, Mark~Overmars, Bettina~Speckmann,
Alexander~Wolff, and David~Wood for providing a stimulating work
environment.  We would also like to thank Michiel~Smid for helpful
discussions and for leading us to the paper of Dietz and Sleator
\cite{ds87}.

\bibliography{theta}
\bibliographystyle{plain}

\end{document}
