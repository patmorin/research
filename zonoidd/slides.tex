\documentclass{beamer}
\usepackage{amsopn}

\DeclareMathOperator{\conv}{conv}
\DeclareMathOperator{\average}{average}

\usepackage{beamerthemesplit}
\input{pat}

\title{An Optimal Randomized Algorithm \\ 
	for $d$-Variate Zonoid Depth}
\author{Pat Morin}
\institute{Carleton University}
\date{October 12, 2006}

\begin{document}

\frame{\titlepage}

\section[Outline]{}
\frame{\tableofcontents}

\section{Introduction}
\subsection{Depth Functions}
\frame
{
  \frametitle{Depth Functions}
  \begin{itemize}
  \item $S$: A set of $n$ points in $\R^d$
  \item A \emph{depth function} assigns a real value $D(p,S)$ to 
            each point in $\R^d$
  \item Generalizes 1-dimensional \emph{order statistics} 
  \item Produces robust multivariate statistics
  \item Many depth functions have been defined, including Tukey depth,
projection depth, Delaunay depth, Fermat-Weber depth, simplicial
depth, convex-hull peeling depth,\ldots,
  \item<1->zonoid depth
  \end{itemize}
}
    
\subsection{Zonoid Depth}
\frame
{
   \frametitle{Zonoid Depth}
   \begin{itemize}
   \item<1-> Recall the definition of the \emph{convex hull}
    \[ \conv(S) = \left\{\sum_{p\in S} p\lambda_p : 
         \mbox{$0\le\lambda_p\le 1$ and $\sum_{p\in S}\lambda_p = 1$} 
        \right\} 
    \]
   \item<2-> Change it slightly to get the \emph{$k$-zonoid}
    \[ Z_k(S) = \left\{\sum_{p\in S} p\lambda_p : 
         \mbox{$0\le\lambda_p\le 1/k$ and $\sum_{p\in S}\lambda_p = 1$} 
        \right\} 
    \]
   \item<3-> Note that $Z_1(S)=\conv(S)$ and $Z_n(S)=\{\average(S)\}$ 
   \item<4-> The \emph{zonoid depth} of $p$ is 
     \[ Z(p,S)=\max\{k : p\in Z_k(S)\} \]
   \end{itemize}
}

\frame
{
    \frametitle{Zonoid Depth}
     \only<1>{\centerline{\includegraphics{zonoid-eg-ps}} 
	$S$} 
     \only<2>{\centerline{\includegraphics{zonoid-eg-1}}
	$S,Z_1(S)$}
     \only<3>{\centerline{\includegraphics{zonoid-eg-2}}
	$S,Z_1(S),Z_2(S)$}
     \only<4>{\centerline{\includegraphics{zonoid-eg-3}}
	$S,Z_1(S),Z_2(S),Z_3(S)$}
     \only<5>{\centerline{\includegraphics{zonoid-eg-4}}
	$S,Z_1(S),Z_2(S),Z_3(S),Z_4(S)$}
     \only<6>{\centerline{\includegraphics{zonoid-eg-5}}
	$S,Z_1(S),Z_2(S),Z_3(S),Z_4(S),Z_5(S)$}
     \only<7>{\centerline{\includegraphics{zonoid-eg-6}}
	$S,Z_1(S),Z_2(S),Z_3(S),Z_4(S),Z_5(S),Z_6(S)$}
}


\subsection{Zonoid Depth Computation}
\frame
{
  \frametitle{Zonoid Depth Computation}
  \begin{itemize}
  \item<1-> How quickly can we compute the zonoid depth of $p$?
     \begin{itemize}  
       \item<2-> Bern and Eppstein 2001:  $O(n(Ld\log n)^c)$ time where $L$
            is the bit precision of the input  (uses ellipsoid method)
       \item<3-> Gopala and Morin 2004: $O(n)$ expected time, but only
            for $d=2$
       \item<4-> Here: $O(n)$ expected time for any constant $d$ (but
            with a superpolynomial dependence on $d$)
     \end{itemize}
     \item<5-> In this talk we focus mainly on the \emph{decision 
          problem}:  \\ ``Is $p\in Z_k(S)$?''
  \end{itemize}
}


\section{The Algorithm}

\subsection{Finding a Zonoid Vertex}

\frame{
   \frametitle{Finding a Zonoid Vertex}
   To Find the extreme vertex of $Z_k(S)$ in direction $\overrightarrow{w}$:
   \only<1>{\includegraphics{project-1}}
   \only<2>{\includegraphics{project-2}}
   \only<3>{\includegraphics{project-3}}
   \only<4>{\includegraphics{project-4}}
   \begin{enumerate}
     \item<2->Project points of $S$ onto $\overrightarrow{w}$
     \item<3->In $O(n)$ time, find the largest $k$ points in the projection 
     \item<4->Take the average of these $k$ points ($\lambda_p=1/k$)
   \end{enumerate}
}

\frame
{
   \frametitle{The Example Again}
   \begin{center}
    \includegraphics{zonoid-eg-6}
   \end{center}
}



\frame
{
   \frametitle{Going from Optimization to Decision}
 \begin{itemize}
   \item<1->Finding the extreme vertex of $Z_k(S)$ in any direction
	is easy
   \item<2->Bern and Eppstein use this fact in combination with the 
	ellipsoid method to test if 
	$p\in Z_k(S)$ by trapping $p$.
   \item<3-> We use this fact to apply a very powerful
	geometric optimization technique\ldots
  \end{itemize}
}

\subsection{Chan's Optimization Theorem}
\frame
{
  \frametitle{Chan's Optimization Theorem (2004)}
  Suppose:  
  \begin{enumerate}
   \item<1-> $\mathcal{P}$ is a computational problem
   \item<2-> $f:\mathcal{P}\mapsto C$ maps problem instances onto closed 
         convex subsets of $\R^d$
   \item<3-> $w:\R^d\mapsto \R$ is a linear objective function
   \item<5-> For any $p\in\R^d$ and any $P\in\mathcal{P}$ of size $n$ we
         can test if $p\in f(P)$ in $D(n)$ time
   \item<6-> For any $P\in\mathcal{P}$ of size $n$ we can find
         $P_1,\ldots, P_r\in\mathcal{P}$ such that $|P_i|\le\alpha n$ 
         and $f(P)=\bigcap_{i=1}^r f(P_i)$ \hfill{[$r=O(1)$ and $\alpha<1$]}
  \end{enumerate}
  Then:
  \begin{itemize}
    \item<4-> For any $P\in\mathcal{P}$ of size $n$ we can find the point
    $q\in f(P)$ that maximizes $w(q)$ in $O(D(n))$ time
  \end{itemize}
} 

\frame
{
   \frametitle{Chan's Technique}

   \begin{center}
   \begin{tabular}{ccc}
    $P$ & $\stackrel{f}{\Rightarrow}$ & \includegraphics[scale=.4]{chan-1} \\
    $\Downarrow$ \\
    $P_1$,$P_2$,$P_3$,$P_4$ & $\stackrel{f}{\Rightarrow}$ 
	& \includegraphics[scale=.2]{chan-2} \\
   \end{tabular}
   \end{center}
}

\subsection{Geometric Duality}
\frame
{ \frametitle{Geometric Duality}
   \only<1>{\centerline{\includegraphics{dual-7}}
	Points become\ldots}
   \only<2>{\centerline{\includegraphics{dual-8}}
	Points become hyperplanes}
    \only<3>{\centerline{\includegraphics{dual-9}} 
	A polytope becomes\ldots}
    \only<4>{\centerline{\includegraphics{dual-0}} 
	A polytope becomes a set of hyperplanes}
}

\subsection{Dual Zonoids}
\frame
{
   \frametitle{Dual Zonoids}
   \begin{itemize}
     \item<1-> We want to know if $p\in Z_k(S)$
     \item<2->
      \begin{center}
	\includegraphics{dual-zonoid}
      \end{center}
     Equivalent to asking if $\mathrm{dual}(p)$ separates $A_k(S)$ and $B_k(S)$.
     \item<3-> Assume $p=(0,0,\ldots,0)$ implies $\mathrm{dual}(p)$ is the flat
		hyperplane through the origin
     \item<4-> Find the lowest point in $A_k(S)$ and the highest point
in $B_k(S)$ 
   \end{itemize}
}

\subsection{Overview of Algorithm}
\frame
{
     \frametitle{Review}
     \begin{itemize}
     \item<1-> To test if $p\in Z_k(S)$:
     \begin{enumerate}
      \item<2-> Translate input so that $p=(0,0,\ldots,0)$
      \item<3-> Find the lowest point in $A_k(S)$
      \item<4-> Find the highest point in $B_k(S)$
      \item<5-> If either is on the wrong side of the hyperplane
	$x_d=0$ then return false, else return true
     \end{enumerate}
     \item<6->{Our problem reduces to two optimization problems over
	implicit polytopes \\}
     \item<7->{We handle these problems using Chan's Optimization Theorem}
    \end{itemize} 
}

\subsection{Applying Chan's Optimization Theorem}

\frame{
     \frametitle{Applying Chan's Method}
     To find the lowest point in $A_k(S)$:
     \begin{itemize}
        \item<1-> Problem instance: $P=(S,k)$
        \item<2-> Objective function: $w(x_1,\ldots,x_d)=-x_d$
        \item<3-> Constraint function: $f(S,k) = A_k(S)$
     \end{itemize}
}

\end{document}
