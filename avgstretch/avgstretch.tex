\documentclass{patmorin}
\usepackage{amsthm,amsmath,graphicx,stmaryrd}
\usepackage{pat}

\DeclareMathOperator{\asf}{asf}

\title{\MakeUppercase{Good Average Stretch Geometric Spanners}}
\author{Vida Dujmovi\'c, Pat Morin and Michiel Smid}


\begin{document}
\begin{titlepage}
\maketitle

\begin{abstract}
  The abstract goes here.
\end{abstract}

\end{titlepage}

\section{Introduction}

The \emph{average stretch factor} of a geometric graph, $G$, with vertex
set $V(G)\subset \R^d$ and edge set $E(G)$ is
\[
    \asf(G) = \binom{|V(G)|}{2}^{-1}\sum_{\{u,w\}\in\binom{V(G)}{2}}\frac{\|uw\|_G}{\|uw\|}
\]
\ldots

\section{The Construction}

We make use of the following construct:  A \emph{$k$-partition} of a
set $V$ of $n$ points consists of a set, $D$,
of disks and an assignment $f:V\to D$ such that
\begin{enumerate}
  \item $|D|\in O(n/k)$,
  \item for each $u\in V$, $u\in f(u)$,
  \item for each $\Delta\in D$, $|\{u\in V: f(u)=\Delta\}|\le k$,
  \item for every $r> 0$, $c\ge 2$, and $p\in\R^2$, the number of disks
   whose radius is in the range $[r,cr]$ that contain $p$ is $O(c)$.
\end{enumerate}
For any set $V$ of $n$ points, a $k$-partition exists and can be found
in $O(n\log n)$ time. [Prove this.]

\begin{thm}
  For every set, $V$, of $n<\infty$ points in $\R^2$, there exists a
  geometric graph $G$ with $V(G)$, $|E(G)|\in O(n)$ and $\asf(G)=1+o(1)$.
  Given $V$, the graph $G$ can be constructed in \ldots time.
\end{thm}

\begin{proof}
  We begin with a $k$-partition $(\{\Delta_1,\ldots,\Delta_{n'}\},f)$
  of $V$, $\Delta_1,\ldots,\Delta_{n'}$ are ordered by decreasing size.
  For each $i\in \{1,\ldots,n'\}$, let $V_i=\{u\in V :
  f(u)=\Delta_i\}$; that is, $V_i$ is the set of points assigned to
  disk $\Delta_i$.  For each set $V_i$, we choose a representative vertex,
  $u_i$ and add edge joining $u_i$ to each other vertex in $V_i$. Let
  $N=\{u_1,\ldots,u_{n'}\}$ denote the set of representative vertices
  and recall that $|N|=n'\in O(n/k)$

  Next, we add two spanner constructions to $G$.  The first is a
  $(1+1/k)$-spanner of $N$ and the second is a $2$-spanner of $V$.
  Thus, for any $u,w\in V$ we have
  \[
     \frac{\|uw\|_G}{\|uw\|} \le \begin{cases}
           1+1/k & \text{if $u,w\in N$} \\
           2 & \text{otherwise .}
         \end{cases}
  \]
  
  Let $r_i$ denote the radius of $\Delta_i$ and let $D_i$ be a disk
  centered at the center of $\Delta_i$ and having radius $cr_i$.
  Points of $V$ that are in $D_i\setminus \Delta_i$ can be problematic;
  there is no guarantee that such points have a $1+o(1)$ spanning path
  to the points in $V_i$.

  The final step in our spanner construction is to find a set of $d\in
  O(1)$ disks $D_{i,1},\ldots,D_{i,d}$, each having radius $r_i/c$ and
  whose union contains the maximum number of points in $V\cap D_i\setminus
  \Delta_i$.  [FIXME: an approximation will do here.]  We then connect
  each of the points in $V_i$ to some point in $D_{i,\ell}$, for each
  $\ell\in\{1,\ldots,d\}$.

  As we will see, the only problematic points that remain for $V_i$
  are those contained in $D_i'=D_i\setminus\bigcup_{i=1}^d D_{i,d}$.
  ($D_i'$ looks like a round piece of swiss cheese.)  The average stretch
  factor of $G$ can be expressed as
  \begin{align*}
    \asf(G) 
      & = 
      \binom n2^{-1}\left(
        \sum_{i=1}^{n'}\sum_{u,w\in V_i}\frac{\|uw\|_G}{\|uw\|}  
         + \sum_{i=1}^{n'-1}\sum_{j={i+1}}^{n'}
            \sum_{u\in V_i}\sum_{w\in V_j}\frac{\|uw\|_G}{\|uw\|}
      \right)  \\
      & \le 
         \binom n2^{-1}\left(n'k^2 
          + \sum_{i=1}^{n'-1}\sum_{j={i+1}}^{n'}
           \sum_{u\in V_i}\sum_{w\in V_j}\frac{\|uw\|_G}{\|uw\|}
      \right)  \\
      & = 
         O(k/n) + \binom n2^{-1}\left(  
          \sum_{i=1}^{n'-1}\sum_{j={i+1}}^{n'}
           \sum_{u\in V_i}\sum_{w\in V_j}\frac{\|uw\|_G}{\|uw\|}
      \right)  \\
   \end{align*}
   Next, we fix a particular value of $i$ and focus on the sum
   \begin{equation}
       \sum_{j={i+1}}^{n'}
          \sum_{u\in V_i}\sum_{w\in V_j}\frac{\|uw\|_G}{\|uw\|}
          \eqlabel{blah}
   \end{equation}
   that appears above.
   Partition the index set $\{i+1,\ldots,n'\}$ into two sets $I_i$
   and $I_i'$, where $I_i$ contains exactly the indices $j$ such that
   $\Delta_j$ intersects $D_i$.
   Then we have
   \begin{align*}
     \eqref{blah} 
      & = \sum_{j\in I_i}
           \sum_{u\in V_i}\sum_{w\in V_j}\frac{\|uw\|_G}{\|uw\|}
         + \sum_{j\in I_i'}
           \sum_{u\in V_i}\sum_{w\in V_j}\frac{\|uw\|_G}{\|uw\|} \\
      & = \sum_{j\in I_i}
          \sum_{u\in V_i}\sum_{w\in V_j}\frac{\|uw\|_G}{\|uw\|} 
         + \sum_{j\in I_i'} |V_i||V_j|(1+1/c)(1+1/k) \enspace .
   \end{align*}
   Putting these pieces back together, we have
   \begin{align*}
     \asf(G) & \le O(k/n) + \binom{n}{2}^{-1}\left(
       \sum_{i=1}^{n'}\left(\sum_{j\in I_i}
          \sum_{u\in V_i}\sum_{w\in V_j}\frac{\|uw\|_G}{\|uw\|} 
         + \sum_{j\in I_i'} |V_i||V_j|(1+1/c)(1+1/k)\right)\right) \\
      & = O(k/n) +\binom{n}{2}^{-1}
       \sum_{i=1}^{n'}\left(\sum_{j\in I_i}
          \sum_{u\in V_i}\left(
             \sum_{w\in V_j\cap D_i'}
               \frac{\|uw\|_G}{\|uw\|}
             +\sum_{w\in V_j\cap D_i\setminus D_i'}
               \frac{\|uw\|_G}{\|uw\|}
           \right)
          + \sum_{j\in I_i'} |V_i||V_j|(1+1/c)(1+1/k) \right) \\
      & = O(k/n) +\binom{n}{2}^{-1}
       \sum_{i=1}^{n'}\left(\sum_{j\in I_i}
          \sum_{u\in V_i}\left(
             \sum_{w\in V_j\cap D_i'}
               \frac{\|uw\|_G}{\|uw\|}
             +\sum_{w\in V_j\cap D_i\setminus D_i'}
               (1+1/c)
           \right)
          + \sum_{j\in I_i'} |V_i||V_j|(1+1/c)(1+1/k) \right) \\
     & \le O(k/n) + (1+1/c)(1+1/k) +\binom{n}{2}^{-1}
       \sum_{i=1}^{n'}\sum_{j\in I_i}
          \sum_{u\in V_i}
             \sum_{w\in V_j\cap D_i'}
               \frac{\|uw\|_G}{\|uw\|} \\
     & \le O(k/n) + (1+1/c)(1+1/k) +\binom{n}{2}^{-1}
       \sum_{i=1}^{n'}\sum_{j\in I_i} 2 |V_i||V_j\cap D_i'| \\
     & \le O(k/n) + (1+1/c)(1+1/k) +\binom{n}{2}^{-1}
       2k\sum_{i=1}^{n'}\sum_{j\in I_i} |V_j\cap D_i'| \\
     \enspace .
   \end{align*}
   Thus, all that remains is to show that
   $\sum_{i=1}^{n'}\sum_{j\in I_i} |V_j\cap D_i'| \in o(n^2/k)$.
   Suppose, for the same of contradiction that this is not the case
   and that
   \begin{equation}
      \epsilon n^2/k \le  \sum_{i=1}^{n'}\sum_{j\in I_i} |V_j\cap D_i'| 
         \le \sum_{i=1}^{n'} |V\cap D_i'| \enspace . \eqlabel{kicker}
   \end{equation}
   The right hand side of \eqref{kicker} has $n'\le \alpha n/k$ terms,
   for some constant $\alpha >0$ and each of these terms is which is at
   most $n$.  Therefore, there must exist at least $(\epsilon/\alpha)n/k$ 
   values of $i$ such that $|V\cap D_i'|\ge (\epsilon/\alpha)n$.  This
   implies that there exists $i_1,\ldots,i_t$, with $t\in\Omega(\log n)$
   such that 
   \[  
      \left|V\cap\bigcap_{\ell=1}^t D_{i_{\ell}}\right| \in \Omega(n^{\delta})
   \]
   for some $\delta > 0$. [FIXME: Can we make this stronger?  These aren't arbitrary sets.  For one, they have finite VC-dimension.]
\end{proof}



\section*{Acknowledgement}

The authors of this paper are partly funded by NSERC and CFI.

\section*{Authors}

\paragraph{Vida Dujmovi\'c.}
School of Mathematics and Statistics and Department of Systems and Computer Engineering, Carleton University
%, \texttt{vida@cs.mcgill.ca}

\paragraph{Pat Morin and Michiel Smid.}
School of Computer Scence, Carleton University
%, \texttt{\{morin,smid\}@scs.carleton.ca}


\bibliographystyle{plain}
\bibliography{template}





\end{document}


