\documentclass[lotsofwhite]{patmorin}

 
%\usepackage{amsthm}

\newcommand{\centeripe}[1]{\begin{center}\Ipe{#1}\end{center}}
\newcommand{\comment}[1]{}

\newcommand{\centerpsfig}[1]{\centerline{\psfig{#1}}}

\newcommand{\seclabel}[1]{\label{sec:#1}}
\newcommand{\Secref}[1]{Section~\ref{sec:#1}}
\newcommand{\secref}[1]{\mbox{Section~\ref{sec:#1}}}

\newcommand{\alglabel}[1]{\label{alg:#1}}
\newcommand{\Algref}[1]{Algorithm~\ref{alg:#1}}
\newcommand{\algref}[1]{\mbox{Algorithm~\ref{alg:#1}}}

\newcommand{\applabel}[1]{\label{app:#1}}
\newcommand{\Appref}[1]{Appendix~\ref{app:#1}}
\newcommand{\appref}[1]{\mbox{Appendix~\ref{app:#1}}}

\newcommand{\tablabel}[1]{\label{tab:#1}}
\newcommand{\Tabref}[1]{Table~\ref{tab:#1}}
\newcommand{\tabref}[1]{Table~\ref{tab:#1}}

\newcommand{\figlabel}[1]{\label{fig:#1}}
\newcommand{\Figref}[1]{Figure~\ref{fig:#1}}
\newcommand{\figref}[1]{\mbox{Figure~\ref{fig:#1}}}

\newcommand{\eqlabel}[1]{\label{eq:#1}}
\newcommand{\eqref}[1]{(\ref{eq:#1})}

\newtheorem{thm}{Theorem}{\bfseries}{\itshape}
\newcommand{\thmlabel}[1]{\label{thm:#1}}
\newcommand{\thmref}[1]{Theorem~\ref{thm:#1}}

\newtheorem{lem}{Lemma}{\bfseries}{\itshape}
\newcommand{\lemlabel}[1]{\label{lem:#1}}
\newcommand{\lemref}[1]{Lemma~\ref{lem:#1}}

\newtheorem{cor}{Corollary}{\bfseries}{\itshape}
\newcommand{\corlabel}[1]{\label{cor:#1}}
\newcommand{\corref}[1]{Corollary~\ref{cor:#1}}

\newtheorem{obs}{Observation}{\bfseries}{\itshape}
\newcommand{\obslabel}[1]{\label{obs:#1}}
\newcommand{\obsref}[1]{Observation~\ref{obs:#1}}

\newtheorem{assumption}{Assumption}{\bfseries}{\rm}
\newenvironment{ass}{\begin{assumption}\rm}{\end{assumption}}
\newcommand{\asslabel}[1]{\label{ass:#1}}
\newcommand{\assref}[1]{Assumption~\ref{ass:#1}}

\newcommand{\proclabel}[1]{\label{alg:#1}}
\newcommand{\procref}[1]{Procedure~\ref{alg:#1}}

\newtheorem{rem}{Remark}
\newtheorem{op}{Open Problem}

\newcommand{\etal}{\emph{et al}}

\newcommand{\voronoi}{Vorono\u\i}
\newcommand{\ceil}[1]{\left\lceil #1 \right\rceil}
\newcommand{\floor}[1]{\left\lfloor #1 \right\rfloor}



\newcommand{\eps}{\varepsilon}
\newcommand{\X}{1-1/\floor{d/2}+\delta}

\title{\MakeUppercase{``Output-Sensitive'' Halfspace Range Counting}
           \\ \MakeUppercase{in 2 and 3 Dimensions}}
\author{Pat Morin}
\date{}

\begin{document}
\maketitle
\begin{abstract}
The problem of preprocessing a set of $n$ points in $\mathbb{R}^d$ to
quickly report the number of points in a query halfspace is
considered.  Algorithms are presented whose running time depends on
the number of points in the query halfspace and which are better than
existing algorithms when this quantity is small. 
\end{abstract}

\section{Introduction}

Let $S$ be a set of $n$ points in $\mathbb{R}^d$.  A (static) data
structure for \emph{halfspace counting queries} preprocesses $S$ so
that, for any query halfspace $h$, the cardinality of $h\cap S$ can be
computed in quickly. Halfspace range counting has a long history
\cite{aeXX} and the state of the art consists of data structures
having size $O(n)$, that can be constructed in $O(n\log n)$ time, and
that can answer queries in time $O(n^{1-1/d}(\log n)^{O(1)})$ \cite{X}.  

A special case of halfspace counting queries called \emph{halfspace
emptiness queries} occurs when the goal is simply to test if $h\cap
S=\emptyset$.  This special-case turn out to be considerably easier
than the general problem.  For $d=2,3$, there exist data structures of
size $O(n)$ that can answer halfspace emptiness queries in $O(\log n)$
time \cite{preparata-shamos,dobkin-kirkpatrick,dhks}. For $d\ge 4$,
there exist data structures of size $O(n)$ that can answer halfspace
emptiness queries in time $O(n^{1-1/\floor{d/2}}(\log n)^{O(1)})$
\cite{matousek-shallow}.

What stands out about these two types of results is the large gap
between counting and emptiness.  This is particularly evident for the
cases $d=2$ and $d=3$ where counting queries take $\Omega(n^{1/2})$
and $\Omega(n^{2/3})$ time, respectively, while emptiness queries take
only $O(\log n)$ time.  In this paper we attempt to find a tradeoff
between counting and emptiness queries by studying data structures for
\emph{$k$-shallow halfspace counting queries}:  Preprocess a set $S$
and an integer $k$ so that, given a query halfspace $h$, one can test
whether $|h\cap S|\le k$ and, if so, report $|h\cap S|$. 
 
We describe a data structure of size $O(n)$ that, for any constant
dimension $d\ge 2$ answers shallow halfspace counting queries in time
$O(k^{1-1/d}n^{\X})$.  For the special case $d=2$, we describe a
different data structure that answers queries in time $O(k^{1-1/d}\log
n + log^3 n)$. 

Not surprisingly, our data structures use the machinery of
\emph{partition trees} \cite{many-matouseks,welzl}.  In particular, we
obtain our results by several applications of Matou\v{s}ek's partition
theorem for shallow hyperplanes \cite{matousek-shallow}, which was
originally used in data structures for \emph{halfspace reporting
queries}.   However, our 2-dimensional data structure returns to an
old idea of Welzl \cite{welzl-partition-trees}, namely spanning trees
of low stabbing number.  In proving our result we show that, for any
set set $S$ of $n$ points in $\mathbb{R}^d$, there exists a spanning
tree $T$ of size $S$ such that no $k$-shallow hyperplane for $S$
intersects $O(k^{1-1/d} + \log^2 n)$ edges of $T$.


\section{The Data Structure}


Let $S$ be a set of $n$ points in $\mathbb{R}^d$.  We say that a
hyperplane $h$ is \emph{$k$-shallow for $S$} if one of the two closed
halfspaces bounded by $h$ contains at most $k$ points of $S$.   A
\emph{simplicial partition} $\Pi$ of $S$ is a collection of pairs
$(\Delta_i,S_i)$ where the $S_i$'s form a partition of $S$ and each
$\Delta_i$ is a $d$-simplex that contains all the points in $S_i$.
Our data structure is partition tree based on Matou\v{s}ek's Partition
Theorem for Shallow Hyperplanes:

\begin{thm}[Matou\v{s}ek 1992]\thmlabel{shallow-partition}
Let $S$ be an $n$ point set in $\mathbb{R}^d$, $d\ge 2$, and let $r$
be a parameter, $1<r<n$.  Then there exists a simplicical partition
$\Pi=\{(\Delta_i,S_i):1\le i\le m\}$ for $P$ with the following
properties:
\begin{enumerate}
\item $\ceil{n/r}\le |S_i|\le 2\ceil{n/r}$ for all $1\le i\le m$ and 

\item any $(n/r)$-shallow hyperplane for $S$ intersects at most
$cr^{1-1/\floor{d/2}}$ (for $d\ge 4$) or $c\log r$ (for
$d=2,3$) simplices in $\{\Delta_1,\ldots,\Delta_m\}$.
\end{enumerate}
Such a simplicial partition can be computed in time $O(n^{1+\delta})$.
For $r\le n^\alpha$ (with a suitable constant $\alpha > 0$), it can be
computed in $O(n\log r)$ time.
\end{thm}

We are now ready to prove our main theorem:

\begin{thm}\thmlabel{main}
Let $S$ be a set of $n$ points in $\mathbb{R}^d$, $d\ge 2$, and let
$k$ be a parameter with $0\le k\le n$. There exists a data structure
of size $O(n)$ that can be constructed in $O(n\log n)$ time and
answers $k$-shallow halfspace counting queries for $S$ in
$O(k^{1-1/d}(n/k)^{\X})$.
\end{thm}
\noindent
\textbf{Note to self:} Check that this recurrence doesn't actually have
a better solution.
\begin{proof}

Let $r\ge 4$ be a constant whose choice will be discussed later.  We
construct a \emph{partition tree} \cite{X} as follows.  If $n\le kr$
then we construct a standard partition tree in $O(n\log n)$ time that
can answer halfspace counting queries in
$O(n^{1-1/d+\delta})=O(k^{1-1/d+\delta})$ time, for some (arbitrarily
small) constant $\delta>0$.  Otherwise, we construct the simplicial
partition of $S$ described in \thmref{shallow-partition}.  We store
the simplices $\Delta_1,\ldots,\Delta_m$ of this partition as well as
the sizes $|S_1|,\ldots,|S_m|$ at the root of our tree and store each
of the $S_i$ recursively as children of the root in the same type of
data structure. This data structure has size $O(n)$ since it is a tree
with $O(n/k)$ nodes, each leaf of which has size $O(k)$ and whose
internal nodes have size $O(r)=O(1)$.  The construction algorithm runs
in $O(n\log n)$ time since the work done at each of the $O(\log n)$
levels is $O(n\log r)=O(n)$.

Queries in this tree proceed almost in the standard fashion. Let $h$
be our query halfspace and let $\partial h$ denote $h$'s bounding
hyperplane.  Beginning at the root of the tree, we first determine how
many of the simplices stored at the root at intersected by $\partial
h$.  If this number exceeds $cr^{1-1/\floor{d/2}}$ (for $d\ge 4$) or
$c\log r$ (for $d=2,3$) then we conclude that $\partial h$ is not
$k$-shallow and therefore $|h\cap S|>k$.  If this is the case, our
query can terminate immediately with the answer ``false.''  Otherwise,
the query algorithm recurses on each of the children corresponding to
the simplices that intersect $\partial h$.  The algorithm then
computes the sum of the outputs of all recursive calls plus the sum of
the sizes of the subsets $S_i$ that are contained in simplices
$\Delta_i$ completely contained in $h$.  If this sum exceeds $k$, the
algorithm again outputs ``false,'' otherwise it outputs the value of
this sum.

The correctness of the above algorithm is clear. To analyze its
running time, we observe that the running time recurrence is bounded
by
\[
   T(n) \le \left\{ \begin{array}{ll}
     O(r) + (cr^{1-1/\floor{d/2}})T(2\ceil{n/r}) & \mbox{if $n > kr$} \\
     k^{1-1/d+\eps} & \mbox{otherwise} 
   \end{array}\right.
\]
for $d\ge 4$ and by
\[
   T(n) \le \left\{ \begin{array}{ll}
     O(r) + (c\log r)T(2\ceil{n/r}) & \mbox{if $n > kr$} \\
     k^{1-1/d+\eps} & \mbox{otherwise} 
   \end{array}\right.
\]
for $d=2,3$.  
In either case, choosing $r$ sufficiently large gives a query time of
$O(k^{1-1/d}(n/k)^{\X})$, as required.
\end{proof}

\thmref{main} also has implications for (standard) halfspace counting
queries. Given only the set $S$, we can create a collection of $O(\log
n)$ shallow halfspace emptiness query data structures
$D_0,\ldots,D_{\ceil{\log n}}$ with values of
$k=2^0,2^1,\ldots,2^{\ceil{\log n}}$, respectively.  Then, given a
query halfspace we can query $D_1,D_2,\ldots$ in succession until the
first data structure returns an answer other than ``false.''  At this
point we have the solution to our halfspace counting query.  Since the
running time of each query in the sequence increases by a constant
factor the running time of the entire sequence of queries is dominated
by the last query. This leads to the following corrollary of
\thmref{main}.

\begin{cor}
There exists a data structure of size $O(n\log n)$ that can be
constructed in $O(n\log^2 n)$ time and can answer
halfspace counting queries in $O(k^{1-1/d}n^{\X})$ time.
\end{cor}

\section{Spanning trees of Low Shallow Stabbing Number}

\begin{lem}
Let $S$ be a set of $n$ points and let $H$ be a set of $m$ lines that
are $k$-shallow for $S$ ($k\le n$) in $\mathbb{R}^2$.  Then, there
exists a pair of points $x,y\in S$ such that the segment $xy$
intersects at most
\[
   O\left(\frac{m\log n}{n} + \frac{m}{(n-k)^{1/2}}\right)
\]
lines in $H$.
\end{lem}

\begin{proof}
By applying \thmref{partition} with $r=n/k$, we obtain a partition of
$\mathbb{R}^2$ into at most $n/k$ triangles such that every line in
$H$ intersects at most $c\log r$ triangles and each triangle contains
at least $k$ points.  The total number of intersections between lines in
$H$ and triangles in the partition is $cm\log r$.  Therefore, some
triangle must intersect at most
\[
    \frac{cm\log n}{n/k} = \frac{kcm\log n}{n} 
\]
lines in $H$, as required.
\end{proof}


\section{An Incorrect Proof}

A (more general but less tight) version of the following lemma
originally appeared in the paper by Welzl \cite[Lemma~4.1]{welzl}.

\begin{lem}\lemlabel{pair}
Let $S$ be a set of $n$ points and $H$ be a set of $m$ hyperplanes
in $\mathbb{R}^d$. Then there exists a pair of points $x,y\in S$ such
that the segment $xy$ is intersected by $O(m/n^{1-1/d})$
hyperplanes in $L$.
\end{lem}

\begin{proof}
By the Cutting Lemma \cite[Lemma~X.X]{matousek-cutting-lemma} there
exists a partition of $\mathbb{R}^d$ into $n-1$ simplices such that
each simplex intersects $O(m/n^{1-1/d})$ of the lines in $L$.  Since
there are more points in $S$ than simplices in the partition, one of
the simplices must contain 2 points in $x,y\in S$.  The points $x$ and
$y$ satisfy the conditions of the lemma.
\end{proof}


\begin{thm}
Let $S$ be a set of $n$ points in $\mathbb{R}^2$.  There exists a
spanning tree $T$ of $S$ such that, for all $1\le k\le n/2$ any
$k$-shallow line for $S$ intersects at most $O(k^{1/2})$ edges of $T$.
\end{thm}

\begin{proof}

Let $Q$ be the set of all $O(n^2)$ combinatorially distinct lines with
respect to $S$.  We define the set $S_0=S$ and the multiset $Q_0$
whose elements are the lines in $Q$ and the multiplicity of a line $l$
of depth $k$ in $S$ is $e^{c_1(\sqrt{n}-\sqrt{k})}$.

We construct the tree as follows: For $i=1,\ldots,n-1$ we find a pair
of points $x_i,y_i\in S_{i-1}$ such that the segment $x_iy_i$
intersects at most $c|Q_{i-1}|/|S_{i-1}|^{1/2}$ of the lines in
$Q_{i-1}$. (\lemref{pair} guarantees the existence of this pair.)  The
pair $(x_i,y_i)$ becomes an edge of the tree.  The set
$S_i=S_{i-1}\setminus\{y_i\}$.  We obtain $Q_i$ from $Q_{i-1}$ by
doubling the multiplicity of each line $l$ that intersects $x_iy_i$.

It remains to show that the resulting tree has the desired property.
We begin by giving an upper bound on $|Q_{n-1}|$. For $i=0,\ldots,n$,
let $k_i=|Q_i|$.  Then, $k_0\le n^{O(1)}2^{n^{1/2}}$ and $k_i$, $i\ge
1$ satisfies the recurrence
\[
     k_i \le k_{i-1}(1+c/(n-i+1)^{1/2}) \enspace ,
\]
so $k_{n-1}$ obeys
\begin{eqnarray*}
     k_{n-1} 
      & \le & k_0\times\prod_{j=1}^{n-2}\left(1+c/j^{1/2}\right) \\ 
      & \le & k_0\times\prod_{j=1}^{n-2}\exp\left(c/j^{1/2}\right) \\
      &  =  & k_0\times\exp\left(\sum_{j=1}^{n-2} c/j^{1/2}\right) \\
      & \le & k_0\times\exp\left(2cn^{1/2}\right) \\
      & \le & n^{O(1)}\times\exp\left(c_1n^{1/2}\right)
                  \times\exp\left(2cn^{1/2}\right) \\
      & \le & n^{O(1)}\times\exp\left(3cn^{1/2}\right) \enspace ,
\end{eqnarray*}
provided that $c_1< c$.  On the other hand, consider some line $\ell\in
Q$ with depth $k$ in $S$.  Suppose $\ell$ crosses $m$ edges of $T$.  Then the cardinality of $\ell$ in $Q_{n-1}$ is 
\[
   |Q_{n-1}\cap \ell| = \exp\left(c_1\left(n^{1/2}-k^{1/2}\right)\right)
     \times 2^{m}
\]
Fuck!
\end{proof}

\end{document}
