\documentclass[lotsofwhite]{patmorin}

\input{pat}

\title{\MakeUppercase{Manhattan Graph Drawing}}
\author{Stefan Langerman \and Pat Morin \and Alexander Wolff}
\date{}

\begin{document}
\maketitle

Let $G=(V,E)$ be a finite undirected graph with vertex set
$V\subset\R^2$.  We consider the problem of embedding $G$ in the plane
so that
\begin{enumerate}
\item $v_i$ is drawn as a point at location $v_i$,
\item $e_i$ is drawn as an $x$ and $y$ monotone curve, and
\item no two edges intersect except possibly at their common endpoints.
\end{enumerate}
We call such an embedding a \emph{Manhattan embedding} of $G$.

\section{The Algorithm}

A \emph{geodesic segment} is a curve whose intersection with any
vertical or horizontal line has at most one connected component.  A
\emph{Manhattan embedding} of an edge in $(u,v)\in E$ is a geodesic
segment whose endpoints are $u$ and $v$.  A \emph{Manhattan embedding}
of a subset $\{e_1,\ldots,e_k\}\subseteq E$ is a set of geodesic
segments $\{s_1,\ldots,s_k\}$ where, for each $1\le i\le k$, $s_i$ is
a drawing of $e_i$ and no pair of segments intersect, except possibly
at their common endpoints.  A Manhattan embedding of $G$ is therefore
Manhattan embedding of $E$.

Assume for now that, for each edge $e\in E$, the $x$-coordinates of
both endpoints of $e$ are distinct as are the $y$-coordinates.  The
left endpoint of $e$ is the vertex of $e$ that has smaller
$y$-coordinate and the right endpoint of $e$ is the vertex that has
larger $x$-coordinate.  The edge $e$ is \emph{upward} if its left
endpoint has smaller $y$-coordinate than its right endpoint, otherwise
$e$ is \emph{downard}.

The \emph{precedence graph} $P(G)=(E,X)$ is a directed graph whose
vertices are the edges of $G$ and whose edges represent above/below
relationships.  An edge $(e_1,e_2)$ is present in $G'$ if and only if,
in every Manhattan embedding of $\{e_1,e_2\}$ by two segments
$\{s_1,s_2\}$, there exists a vertical line $\ell$ such that
$p_1=\ell\cap s_1$ and $p_2=\ell\cap s_2$ are each non-empty and $p_1$
is above $p_2$.  A special case occurs when there exists no drawing of
$\{e_1,e_2\}$, in which case we use the convention that the precedence
graph contains both the directed edges $(e_1,e_2)$ and $(e_2,e_1)$.

\begin{lem}
The graph $G$ has a Manhattan embedding if and only if $P(G)$ is
acyclic.
\end{lem}

\begin{proof}
If $P(G)$ contains a cycle then it is clear that $G$ does not have a
Manhattan embedding.

If $P(G)$ is acyclic then topologically sort the edges of $G$ to
obtain a total order $(E,\prec)$. We show how to obtain a Manhattan
embedding of $G$ using a \emph{plane sweep} \cite{boXX}.  We sweep a
vertical line from left to right over the vertices of $G$.  During the
sweep, there are three kinds of edges of $G$:
\begin{enumerate}

\item The \emph{complete edges} have both endpoints to the left of the
sweep line.  These edges are already drawn as geodesic segments.

\item The \emph{partial edges} have one endpoint on each side of the
sweep line.  These edges are partially drawn as a geodesic arc that
begins at the vertex to the left of the line and ends on the sweep
line.

\item The \emph{untouched edges} have both endpoints to the right of
the sweep line.  These edges are not drawn at all.
\end{enumerate}
At all times, we maintain the following two invariants:
\begin{enumerate}

\item The complete and partial edges are embedded as geodesic
segments.

\item For every upward (respecively, downward) partial edge $e$, the
partial embedding of $e$ is not a downward (respectively, upward)
geodesic segment.

\item No two segments intersect, except possibly at their common
endpoint.
\end{enumerate}
It is clear that if we maintain these three invariants throughout the
plane sweep then once the sweep line has passed over the rightmost
vertex of $G$ we have obtained a Manhattan embedding of $G$ 

Sweep line events occur at the vertices of $G$.  Between the vertices
of $G$, the partial edges are extended horizontally as the sweep line
moves.  When the sweep line reaches a vertex of $u$ zero or more new
partial edges are created at $u$, zero or more partial edges are
terminated at $u$ and become complete edges, and a number of
prexisting partial edges have a vertical edge inserted so that they
pass above or below $u$, as appropriate. 

More precisely, let $e_1,\ldots,e_k$ be the partial edges of $G$,
ordered from top to bottom, before the sweep line encounters $u$ and
let $f_1,\ldots,f_\ell$ be the edges whose left endpoint is $u$.  Note
that $e_1\prec e_2\prec\cdots\prec e_k$ and that there is some value
$1<i\le k$ such that $e_i\prec f_1,\ldots,f_\ell\prec e_{i+1}$.
Without loss of generality, assume that there is some index $j \le i$
such that all the partial edges $e_{j},\ldots,e_i$ are below $u$
before the sweep line reaches $u$.  Then $e_j,\ldots,e_i$ all have
near vertical segments inserted just before $u$ so that they pass
either just above $u$ if $u$ is not their right endpoint or they
terminate at $u$ if $u$ is their right endpoint.

Next we show that, after processing $u$, the three invariants are
maintained.

\begin{enumerate}
\item We have just created near-vertical segments that cause the embeddings of edges
$e_j\ldots,e_i$ be bend upwards.  We must therefore ensure that
$e_j,\ldots,e_i$ are upwards edges.

\end{enumerate}
\end{proof}

\end{document}

