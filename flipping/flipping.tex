\documentclass{article}
\usepackage{amsthm}
\usepackage{amsfonts}

\input{pat.tex}

\title{On Parallel Flips in Triangulations}
\author{Prosenjit Bose
	\and Zhicheng Gao 
	\and Pat Morin}
\date{}

\begin{document}
\maketitle

\begin{abstract}
We show that any triangulation of $n$ vertices can be reduced to a
canonical triangulation using $O(\log n)$ parallel flip operations.
\end{abstract}

\section{Introduction}

Let $G=(V,E,F)$ be a simple undirected loop-free graph with vertex set
$V$, $|V|=n$, edge set $E$ and a combinatorial embedding that defines
the face set $F$.  A \emph{flip}, or \emph{edge swap}, in $G$ involves
removing an edge $e$ incident on two 3 faces (producing a 4-face) and
replacing it with the other diagonal of the resulting 4-face (see
\figref{flip}).  A flip is \emph{valid} if it does not introduce a
parallel edge into $G$.  In the remainder of this paper, we only
consider valid edge flips.

Sleator, Tarjan and Thurston \cite{sttXX} showed that $O(n)$ valid
edge flips are sufficient to transform any triangulation of a convex
polygon to any other triangulation of the same polygon.  Cai and
Hirsch \cite{chXX} generalized some of this.

If $G$ is a geometric graph whose vertices are points in
$\mathbb{R}^2$ and whose edges are line segments then the situation is
somewhat different.  In this case, the condition for validity is
stronger.  Namely, $e$ is only valid if the quadrilateral obtained by
removing $e$ is convex.  In this geometric model of flipping, Hurtado
\etal\ \cite{hnuXX} and Hanke \etal\ show that $\Theta(n^2)$ flips are
always sufficient and sometimes necessary to convert any triangulation
of a point set into any another triangulation of the same point set.

In this paper, we consider parallel flips, as introduced by Galtier
\etal\ \cite{ghnpXX}.  A \emph{parallel flip} performs several
independent flipping operations simultaneously.  More precisely, a
parallel flip flips a subset $E'\subseteq E$ where the edges of $E'$
are flippable and no two edges in $E'$ are on the boundary of a common
face.  Galtier \etal\ showed that, under the geometric model of
flipping, $O(n)$ parallel flips are always sufficient and sometimes
necessary to convert one triangulation into another.  They also show
that $O(\log n)$ parallel flips are sufficient to transform any
triangulation of a convex polygon into any other triangulation of the
same polygon.

In this paper our main theorem is that, under the non-geometric model
of flipping, $O(\log n)$ parallel flips are always sufficient to
convert any triangulation (i.e., embedded maximal planar graph) into
any other triangulation.  We prove the theorem in the following way:
In \secref{separating} we show that $O(\log n)$ parallel flips are
sufficient to make any triangulation into a Hamiltonian graph.  In
\secref{interior} we show that, given a Hamiltonian cycle in a
triangulation, it is possible to apply the method of Galtier \etal\
\cite{gxxXX} to reduce the triangulation to a canonical form.  In
\secref{conclusions} we summarize and conclude with some open
problems.

\section{Removing Separating Triangles}\seclabel{separating}

A \emph{separating triangle} in a triangulation $T$ is a set of 3
vertices whose removal disconnect $T$.  In this section, we show that
$O(\log n)$ parallel flips are sufficient to remove all separating
triangles from any triangulation.  By a famous result of So and So
\cite{XX}, this implies that $O(\log n)$ parallel flips are sufficient
to make any triangulation Hamiltonian.  Furthermore, a Hamiltonian
cycle of this triangulation can be found using an $O(n)$ time
algorithm \cite{XX}.

We begin with the following lemma about single flips of separating
triangle edges.

\begin{lem}
Let $\{u,v\}$ be an edge that belongs to $k\ge 2$ separating triangles
of $T$.  Then $\{u,v\}$ is a valid flip that reduces the number of
separating triangles by $k$.
\end{lem}

\begin{lem}\lemlabel{oneflip}
Let $\{u,v,w\}$ be a separating triangle of $T$.  Then $(u,v)$,
$(v,w)$ and $(w,u)$ are valid flips and one of these flips reduces the
number of separating triangles in $T$.
\end{lem}

\begin{proof}
Consider the edge $(u,v)$.  It is incident on two 3-faces that have as
their third vertices $x$ and $y$, with $x,y\notin\{u,v,w\}$.
Furthermore, $x$ (say) is in the interior of the cycle defined by
$(u,v,w)$ and $y$ (say) is outside the cycle. Therefore, by the Jordan
curve theorem, the edge $(x,y)$ is not in $T$ and the flip $(u,v)$ is
valid.  Switching notation proves the same is true for $(v,w)$ and
$(w,u)$.

To prove the second part of the lemma, imagine that flipping $(u,v)$
does not decrease the number of separating triangles in $T$.  Then,
since flipping $(u,v)$ eliminates the separating triangle $\{u,v,w\}$,
it must introduce a new separating triangle containing $x$ and $y$.
Therefore, there must be a two-edge path in $T$ that joins $x$ and $y$
but does not go through $u$ or $v$.  The only such path that is
possible is $xwy$ so that $(x,w)$ and $(w,y)$ are edges of $T$.

It follows that one of $\{y,v,w\}$ or $\{y,u,w\}$ is a separating
triangle of $T$, so suppose wlog that it is $\{y,v,w\}$.  Next
consider what happens if we flip edge $(u,w)$ and let $z$ denote the
third vertex of the face $(u,w,z)$ that is outside of the cycle
$\{u,v,w\}$.  Then the same argument used above shows that this flip
either reduces the number of separating triangles in $T$, or
$\{u,y,w\}$ is a face of $T$ and there is a path from $y$ to $z$ of
length 2 that avoids $u$ and $w$.  Since this path must pass through
$v$, it follows that $z$ and $x$ must be the same vertex.  This is a
contradiction, since then the edge $(w,x)$ must appear twice in $T$.
\end{proof}


\section{Reducing a Hamiltonian Triangulation to a Canonical Form}

\section{Conclusion}

\end{document}
