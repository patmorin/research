\documentclass[charterfonts,lotsofwhite]{patmorin}
\usepackage{graphicx}
\usepackage{amsopn}
 
%\usepackage{amsthm}

\newcommand{\centeripe}[1]{\begin{center}\Ipe{#1}\end{center}}
\newcommand{\comment}[1]{}

\newcommand{\centerpsfig}[1]{\centerline{\psfig{#1}}}

\newcommand{\seclabel}[1]{\label{sec:#1}}
\newcommand{\Secref}[1]{Section~\ref{sec:#1}}
\newcommand{\secref}[1]{\mbox{Section~\ref{sec:#1}}}

\newcommand{\alglabel}[1]{\label{alg:#1}}
\newcommand{\Algref}[1]{Algorithm~\ref{alg:#1}}
\newcommand{\algref}[1]{\mbox{Algorithm~\ref{alg:#1}}}

\newcommand{\applabel}[1]{\label{app:#1}}
\newcommand{\Appref}[1]{Appendix~\ref{app:#1}}
\newcommand{\appref}[1]{\mbox{Appendix~\ref{app:#1}}}

\newcommand{\tablabel}[1]{\label{tab:#1}}
\newcommand{\Tabref}[1]{Table~\ref{tab:#1}}
\newcommand{\tabref}[1]{Table~\ref{tab:#1}}

\newcommand{\figlabel}[1]{\label{fig:#1}}
\newcommand{\Figref}[1]{Figure~\ref{fig:#1}}
\newcommand{\figref}[1]{\mbox{Figure~\ref{fig:#1}}}

\newcommand{\eqlabel}[1]{\label{eq:#1}}
\newcommand{\eqref}[1]{(\ref{eq:#1})}

\newtheorem{thm}{Theorem}{\bfseries}{\itshape}
\newcommand{\thmlabel}[1]{\label{thm:#1}}
\newcommand{\thmref}[1]{Theorem~\ref{thm:#1}}

\newtheorem{lem}{Lemma}{\bfseries}{\itshape}
\newcommand{\lemlabel}[1]{\label{lem:#1}}
\newcommand{\lemref}[1]{Lemma~\ref{lem:#1}}

\newtheorem{cor}{Corollary}{\bfseries}{\itshape}
\newcommand{\corlabel}[1]{\label{cor:#1}}
\newcommand{\corref}[1]{Corollary~\ref{cor:#1}}

\newtheorem{obs}{Observation}{\bfseries}{\itshape}
\newcommand{\obslabel}[1]{\label{obs:#1}}
\newcommand{\obsref}[1]{Observation~\ref{obs:#1}}

\newtheorem{assumption}{Assumption}{\bfseries}{\rm}
\newenvironment{ass}{\begin{assumption}\rm}{\end{assumption}}
\newcommand{\asslabel}[1]{\label{ass:#1}}
\newcommand{\assref}[1]{Assumption~\ref{ass:#1}}

\newcommand{\proclabel}[1]{\label{alg:#1}}
\newcommand{\procref}[1]{Procedure~\ref{alg:#1}}

\newtheorem{rem}{Remark}
\newtheorem{op}{Open Problem}

\newcommand{\etal}{\emph{et al}}

\newcommand{\voronoi}{Vorono\u\i}
\newcommand{\ceil}[1]{\left\lceil #1 \right\rceil}
\newcommand{\floor}[1]{\left\lfloor #1 \right\rfloor}



\newcommand{\defequals}{\stackrel{\mathrm{def}}{=}}
\newcommand{\boundary}{\partial}
\DeclareMathOperator{\depth}{depth}
\DeclareMathOperator{\lft}{left}
\DeclareMathOperator{\rght}{right}
\DeclareMathOperator{\prnt}{parent}
\newcommand{\blah}{O(1/(1-\alpha)^2)}


\title{\MakeUppercase{Distribution-Sensitive Point 
	Location in Convex Subdivisions}%
	\thanks{The research presented in this article took place
while the fourth author was a visiting researcher at the Universit\'e Libre de
Bruxelles, supported by a grant from FNRS.  The researchers are
partially supported by NSERC and FNRS.}}
\author{S\'ebastien Collette \\ \textit{Universit\'e Libre de Bruxelles}
  \and Vida Dujmovi\'c \\ \textit{McGill University}
  \and John Iacono \\ \textit{Polytechnic University}
  \and Stefan Langerman \\ \textit{Universit\'e Libre de Bruxelles}
  \and Pat Morin \\ \textit{Carleton University}}
\date{}

\begin{document}
\maketitle

\begin{abstract}
A data structure is presented for point location in convex planar
subdivisions when the distribution of queries is known in advance.
The data structure has an expected query time that is within a
constant factor of optimal.
\end{abstract}

\keywords{Planar point location, Entropy}

\section{Introduction}
\seclabel{intro}

The planar point location problem is one of the classic problems in
computational geometry. Given a planar subdivision $G$,\footnote{A
\emph{planar subdivision} is a partitioning of the plane into points
(called \emph{vertices}), open line segments (call \emph{edges}), and
open polygons (called \emph{faces}).} the planar point location
problems asks us to construct a data structure so that, for any query
point $p$, we can quickly determine which face of $G$ contains
$p$.\footnote{In the degenerate case where $p$ is vertex or contained
in an edge of $G$ any face incident on that vertex/edge may be
returned as an answer.} The history of the planar point location
problem parallels, in many ways, the study of binary search trees.

After a few initial attempts \cite{dl76,lp77,p81}, asymptotically
optimal (and quite different) linear-space $O(\log n)$ query time
solutions to the planar point location problem were obtained by
Kirkpatrick \cite{k83}, Sarnak and Tarjan \cite{st86}, and
Edelsbrunner \etal\ \cite{egs86} in the mid 1980s.  These results were
based on hierarchical simplification, data structural persistence, and
fractional cascading, respectively.  All three of these techniques have
subsequently found many other applications.  An elegant randomized
solution, combining aspects of all three previous solutions, was later
given by Mulmuley \cite{m90}.  Preparata \cite{p90} gives a
comprehensive survey of the results of this era.

In the 1990s, several authors became interested in determining the
exact constants achievable in the query time.  Goodrich \etal\
\cite{gor97} gave a linear-size data structure that, for any query,
requires at most $2\log n + o(\log n)$ point-line comparisons and
conjectured that this query time was optimal for linear-space data
structures. Here and throughout, logarithms are implicitly base 2
unless otherwise specified. The following year, Adamy and Seidel
\cite{as98} gave a linear-space data structure that answers queries
using $\log n + 2\sqrt{\log n} + O(\log\log n)$ point-line comparisons
and showed that this result is optimal up to the third term.

Still not done with the problem, several authors considered the point
location problem under various assumptions about the query
distribution.  All these solutions compare the expected query time to
the \emph{entropy bound};  in a planar subdivision with $f$ faces, if the query
point $p$ is chosen from a probability measure over $\R^2$ such that
$p_i$ is the probability that $p$ is contained in face $i$ of $G$,
then no algorithm that makes only binary decisions can answer queries
using an expected number of decisions that is fewer than 
\begin{equation}
    H(p_1,\ldots,p_f) = \sum_{i=1}^f p_i\log(1/p_i) \enspace . 
	\eqlabel{entropy}
	\eqlabel{entropy-face}
\end{equation}

In the previous results on planar point location, none of the query
times are affected significantly by the structure of $G$;  they hold
for arbitrary planar subdivisions.  However, when studying point
location under a distribution assumption the problem becomes more
complicated and the results become more specific.  A \emph{convex
subdivision} is a planar subdivision whose faces are all convex
polygons, except the outer face, which is the complement of a convex
polygon.  A \emph{triangulation} is a convex subdivision in which each
face has at most 3 edges on its boundary.  Note that, if every face of
$G$ has a constant number of sides, then $G$ can be augmented, by the
addition of extra edges, so that it is a triangulation without
increasing \eqref{entropy} by more than a constant.  Thus, in the
following we will simply refer to results about triangulations where
it is understood that these also imply the same result for planar
subdivisions with faces of constant size.

Arya \etal\ \cite{acmr00} gave two results for the case where the
query point $p$ is chosen from a known distribution where the $x$ and
$y$ coordinates of $p$ are chosen independently and $G$ is a convex
subdivision.  They give a linear space data structure for which the
expected number of point-line comparisons is at most $4H+O(1)$ and a
quadratic space data structure for which the expected number of
point-line comparisons is at most $2H+O(1)$.  The assumption about the
independence of the $x$ and $y$ coordinates of $p$ is crucial to the
these results.

For arbitrary distributions that are known in advance, several results
exist.  Iacono \cite{i01,i04} showed that, for triangulations, a
simple variant of Kirkpatrick's original point location structure
gives a linear space, $O(H+1)$ expected query time data structure.  A
result by Arya \etal\ \cite{amm00} gives a data structure for
triangulations that uses $H + O(H^{2/3}+1)$ expected number of
comparisons per query and $O(n\log n)$ space.  The space requirement
of this latter data structure was later reduced, by the same authors,
to $O(n\log^* n)$ \cite{amm01a}.  The same three authors
\cite{amm01b} also showed that a variant of Mulmuley's randomized algorithm
gives, for triangulations, a simple $O(H+1)$ expected query time,
linear space data structure.  Most recently, Arya \etal\
\cite{ammw07}, have given an $O(n)$ space structure for point-location
in triangulations with query time $H+O(H^{1/2}+1)$.  This last result
extends to convex subdivisions but only in the case where the query
distribution is uniform within each face.

The above collection of results suggest that point location, and even
distribution-sensitive point location, is a well-studied and
well-understood problem, with solutions that are optimal up to
lower order terms.  However, in the above results there is a glaring
omission.  Given a convex polygon $P$, a folklore $O(\log n)$ time
algorithm exists to test if a query point $p$ is contained in $P$ and
this algorithm is optimal, in the worst case \cite{ps85}.  Testing if
$p\in P$ is a special case of point location in a convex subdivision
in which the subdivision has only 2 faces.  Thus, we might expect that, if
$p$ is drawn according to some distribution over $\R^2$, it may be
possible to do better in many cases. How much better?  It is certainly
not possible to achieve the entropy bound in all cases since, when
$f=2$ the entropy bound is at most 1.

We begin our investigation of distribution-sensitive point location
with the fundamental problem of testing if a query point $p$, drawn
from an arbitrary distribution $D$ over $\R^2$, is contained in a
convex polygon $P$.  We describe a hierarchical triangulation $T$ of
$\R^2$ that we use to simultaneously achieve two objectives:
\begin{enumerate}
\item $T$ is used with a query algorithm to check if a point $p$ 
	is contained in $P$, and
\item $T$ is used to give a lower bound on the expected cost of
	any linear decision tree that tests if a point $p$ selected
	according to $D$ is contained in $P$.
\end{enumerate}
The lower bound in Point~2 matches, to within a constant factor, the
expected query time of the algorithm in Point~1.  Thus, among
algorithms that can be expressed as linear decision trees, our
algorithm is optimal. This result is the first to give
\emph{any} lower bound on the expected complexity of \emph{any} point
location problem that exceeds the entropy bound. Proving the lower
bound is by far the hardest part of our result.  

As an easy consequence of the above result we obtain a data structure
for point location in convex subdivisions.  The expected query time of
the resulting algorithm is optimal in the linear decision tree model
of computation. Note that all known algorithms for planar point
location that do not place special restrictions on the input
subdivision can be described in the linear decision tree model of
computation.\footnote{Although significant breakthroughs have recently been
made in this area \cite{c06,p06}, we deliberately do not survey algorithms
that require the vertices of the subdivision to be on integer
coordinates.}  This data structure for point location in convex
subdivisions where the query point is drawn according to an arbitrary
distribution is the most general result known about planar point
location and implies, to within constant factors, all of the results
discussed above.

The remainder of this paper is organized as follows:  \Secref{prelim}
presents definitions and notations used throughout the paper.
\Secref{polygons} discusses algorithms and lower bounds for point
location in convex polygons.  \Secref{subdivisions} presents algorithms
and lower bounds for point location in convex subdivisions.  Finally,
\Secref{discussion} concludes with a discussion and points out
directions for further research.


\section{Preliminaries}
\seclabel{prelim}

In this section we give definitions, notations, and background
required in subsequent sections.

\paragraph{Triangles and Convex Polygons.}  For the purposes of this paper, a
\emph{triangle} is the common intersection of at most 3 halfplanes.
This includes triangles with infinite area and triangles having 0, 1,
2, or 3, vertices. Similarly, a \emph{convex $k$-gon} is the common
intersection of at most $k$ halfplanes.

\paragraph{Classification Problems and Classification Trees.}

A \emph{classification problem} over a domain $\mathcal{D}$ is a
function $\mathcal{P}:\mathcal{D}\mapsto \{0,\ldots,k-1\}$.  The
special case in which $k=2$ is called a \emph{decision problem}.  A
$d$-ary \emph{classification tree} is a full $d$-ary tree\footnote{A
full $d$-ary tree is a rooted ordered tree in which each non-leaf node
has exactly $d$ children.} in which each internal node $v$ is labelled
with a function $P_v:\mathcal{D}\mapsto\{0,.\ldots,d-1\}$ and for
which each leaf $\ell$ is labelled with a value
$d(\ell)\in\{0,\ldots,k-1\}$. The \emph{search path} of an input $p$
in a classification tree $T$ starts at the root of $T$ and, at each
internal node $v$, evaluates $i=P_v(p)$ and proceeds to the $i$th
child of $v$.  We denote by $T(p)$ the label of the final (leaf) node
in the search path for $p$.  We say that the classification tree $T$
\emph{solves} the classification problem $\mathcal{P}$ over the domain
$\mathcal{D}$ if, for every $p\in \mathcal{D}$, $\mathcal{P}(p)=T(p)$.

Unless specifically mentioned otherwise, classification trees are
binary classification trees.  For a node $v$ in a (binary)
classification tree, its left child, right child, and parent are
denoted by $\lft(v)$, $\rght(v)$ and $\prnt(v)$, respectively.


\paragraph{Probability.}

For a distribution $D$ and an event $X$, we denote by $D_{|X}$ the
distribution $D$ conditioned on $X$.  That is, the distribution where
the probability of an event $Y$ is $\Pr(Y|X)=\Pr(Y\cap X)/\Pr(X)$.
The probability measures used in this paper are usually defined over
$\R^2$.  We make no assumptions about how these measures are
represented, but we assume that an algorithm can perform the following
two operations in constant time:
\begin{enumerate}
\item given an open triangle $\Delta$, compute $\Pr(\Delta)$, and
\item given an open triangle $\Delta$ and a point $t$ at the
intersection of two of $\Delta$'s supporting lines, compute a line $\ell$
that contains $t$ and that partitions $\Delta$ into two open triangles
$\Delta_0$ and $\Delta_1$ such that $\Pr(\Delta_0)\le\Pr(\Delta)/2$
and $\Pr(\Delta_1)\le\Pr(\Delta)/2$.
\end{enumerate}
Requirement 2 is used only for convenience in describing our data
structure.  In \secref{discussion} we show that 
Requirement 2 is not really necessary and that Requirement 1 is sufficient to
implement our data structure.

For a classification tree $T$ that solves a problem
$P:\mathcal{D}\mapsto\{0,\ldots,k-1\}$ and a probability measure $D$
over $\mathcal{D}$, the \emph{expected search time} of $T$ is the
expected length of the search path for $p$ when $p$ is drawn at random
from $\mathcal{D}$ according to $D$.  Note that, for each leaf $\ell$
of $T$ there is a maximal subset $r(\ell)\subseteq \mathcal{D}$ such
that the search path for any $p\in r(\ell)$ ends at $\ell$.  Thus, the
expected search time of $T$ (under distribution $D$) can be written as
\[
     \mu_D(T) = \sum_{\ell\in L(T)} \Pr(r(\ell))\times \depth(\ell)
	\enspace ,
\]
where $L(T)$ denotes the leaves of $T$ and $\depth(\ell)$ denotes the
length of the path from the root of $T$ to $\ell$.

The following theorem, which is a restatement of (half of) Shannon's
Fundamental Theorem for a Noiseless Channel \cite[Theorem 9]{s48}, is
what all existing results on distribution-sensitive planar point
location use to establish their optimality:

\setcounter{thm}{8}
\begin{thm}\thmlabel{shannon}
Let $\mathcal{P}:\mathcal{D}\mapsto \{0,\ldots,k-1\}$ be a classification
problem and let $p\in \mathcal{D}$ be selected from a distibution $D$ such
that $\Pr\{\mathcal{P}(p)= i\}=p_i$, for $0\le i< k$.  Then, any
$d$-ary classification tree $T$ that solves $\mathcal{P}$ has
\begin{equation}
     \mu_D(T) \ge \sum_{i=0}^{k-1} p_i\log_d(1/p_i) \enspace .
	\eqlabel{shannon}
\end{equation}
\end{thm}
\setcounter{thm}{0}
\thmref{shannon} is applied to the point location problem by treating
point location as the problem of classifying the query point $p$ based
on which face of $G$ contains it.  In this way, we obtain the lower
bound in \eqref{entropy-face}.

\section{Point In Convex Polygon Testing}
\seclabel{polygons}

Let $P$ be a convex $n$-gon whose boundary is denoted by $\partial P$
and consider a probability measure $D$ over $\R^2$.  For technical
reasons, we use the convention that $P$ does not contain its boundary
so that $p\in \boundary P$ implies $p\not\in P$.  In this section we
are interested in preprocessing $P$ and $D$ so that we can quickly
solve the decision problem of testing whether a query point $p$, drawn
according to $D$, is contained in $P$. 

In particular, we are interested in algorithms that can be described
as \emph{linear decision trees}.  These are decision trees such that
each internal node $v$ contains a linear inequality $P_v(x,y)=[ax+by
\ge c]$.\footnote{Here, and throughout, we use Iverson's notation
where $[X]=0$ if $X$ is false and $[X]=1$ if $X$ is true \cite{k92}.}
We require that, for every $p\in\R^2$ the leaf reached in the search
path for $p$ is labelled with a 1 if and only if $p\in P$.
Geometrically, each internal node of $T$ is labelled with a directed
line and the decision to go to the left or right child depends on
whether p is to the left or right (or on) this line.  

Our exposition is broken up into three sections.  We begin by
describing a data structure (in fact, a decision tree) that tests if
query point $p$ is contained in a convex polygon  $P$.  Next, we give
an (easy) analysis of the expected search time of this data structure.
Finally, we give a (more difficult) proof that this expected search
time is optimal.


\subsection{Triangle Trees}

At a high level, our data structure works by creating a sequence of
successively finer approximations $A_0,\ldots,A_k$ to $\boundary P$.
Each approximation $A_i$ consists of two convex polygons; an
\emph{outer approximation} that contains $P$ and an \emph{inner
approximation} that is contained in $P$.

Each approximation $A_i$ is completely defined by a set $S_i$ of
points on $\boundary P$.  The inner approximation is simply the convex
hull of $S_i$.  The outer approximation has an edge tangent to $P$ at
each of the points of $S_i$.  More precisely, for each point $x\in
S_i$ there is an edge $e$ of the outer approximation that contains
$x$.  If $x$ is in the interior of an edge of $P$ then $e$ is
contained in the same line that contains that edge. Otherwise ($x$ is
a vertex of $P$) $e$ is supported by the line containing the edge
incident to $x$ that precedes $x$ in counterclockwise order.  We
ensure that successive approximations have a containment relationship,
i.e., $A_i\supseteq A_{i+1}$, by choosing our boundary points so that
$S_i\subseteq S_{i+1}$.

Next we define the sets $S_0,\ldots,S_k$ that define our
approximations.  The set $S_0$ is empty, and we use the convention
that the outer approximation in this case is the entire plane and the
inner approximation is the empty set. The set $S_1$ consists of any
two points, $x$ and $y$ on $\boundary P$ such that $\Pr(h_1)\le 1/2$
and $\Pr(h_2)\le 1/2$, for each of the two open halfspaces $h_1$ and
$h_2$ bounded by the line containing $x$ and $y$. The existence of $x$
and $y$ is guaranteed, for example, by the planar Ham Sandwich Theorem
\cite{bz04}.

We now show how, for $i\ge 1$, to obtain the set $S_{i+1}$ from the
set $S_{i}$.  Let $p_0,\ldots,p_{m_i-1}$ be the points in $S_i$ as
they occur in conterclockwise order around the boundary of $P$.  The
approximation $A_i$ thus consists of $m_i$ open triangles
$\Delta_0,\ldots,\Delta_{m_i-1}$ where $\Delta_j$ is the intersections
of the following halfspaces:

\begin{enumerate}
\item the open halfspace to the right of the directed line through 
	$p_j$ and $p_{j+1}$,
\item the open halfspace bounded by the tangent to $P$ at $p_j$ and that
contains $p_{j+1}$, and 
\item the open halfspace bounded by the tangent to $P$ at $p_{j+1}$ and
that contains $p_j$.
\end{enumerate}
Let $t_j$ be the intersection point of 
the two lines tangent to $P$ at $p_j$ and
$p_{j+1}$.\footnote{Throughout this discussion, subscripts are
implicitly taken modulo $m_i$.}  Refer to \figref{split}. For each
triangle $\Delta_j$ that is not completely contained in $P$
we add a new point to
$S_{i+1}$ as follows:  we subdivide $\Delta_j$ into two open triangles
$\Delta_{j,0}$ and $\Delta_{j,1}$ by a line $\ell$ through $t_j$ and
such that 
\[  
     \Pr(\Delta_{j,b}) \le \Pr(\Delta_{j})/2 \enspace ,
\]
for $b\in\{0,1\}$.
We then select a new point to add to $S_{i+1}$ at the intersection of
$\ell$ and $\boundary P$ that occurs in $\Delta_j$.  Note that the
next level of approximation $A_{i+1}$ now contains two triangles that
are contained in $\Delta_j$.  We call these two triangles the
\emph{children} of $\Delta_j$ and we say that this operation
\emph{splits} $\Delta_j$ into these two triangles.

\begin{figure}
\begin{center}
\begin{tabular}{cc}
\includegraphics{split-a} & \includegraphics{split-b} \\
(a) & (b)
\end{tabular}
\end{center}
\caption{Starting with $\Delta_j$ we (a) subdivide $\Delta_j$ into two triangles
$\Delta_{j,0}$ and $\Delta_{j,1}$ with a line through $t_j$ and then
(b) split $\Delta_j$ into its two children.}
\figlabel{split}
\end{figure}

The entire process terminates at the first value of $k$ for which
$A_k$ is completely contained in $P$.  The approximations
$A_0,\ldots,A_k$ are stored as a binary tree $T=T(P,D)$ that we call a
\emph{triangle tree} for $P$ and $D$.  Each node $v$ of $T$ at level
$i$ in $T$ corresponds to an open triangle $\Delta(v)$ in
approximation $A_{i}$ and the two children of $v$ correspond to the
two open triangles into which $\Delta(v)$ is split. See \figref{tree}.
A crucial property of this construction guaranteed by the splitting
process is that, for any node $v$ at level distance $i$ from the root
of $T$, $\Pr(\Delta(v))\le 1/2^i$.

\begin{figure}
\begin{center}\includegraphics[scale=1.2]{tree}\end{center}
\caption{A convex polygon $P$, the triangles of the triangle 
tree $T$, and the sequence
of approximations $A_1,\ldots,A_3$ that approximate $\boundary P$.}
\figlabel{tree}
\end{figure}

To use the tree $T$ to determine if a point $p$ is contained in $P$ we
proceed top-down, starting at the root.  For a point $p$ contained in
$\Delta(v)$ one of two things can happen: (1) $p$ is contained in one
of the two open triangles $\Delta(\lft(v))$ or $\Delta(\rght(v))$ in
which case we recurse on the right or left child of $v$, respectively,
or (2) we can determine in constant time if $p\in P$.

\subsection{Analysis of the Triangle Tree}

Let $T=T(P,D)$ be a triangle tree for a convex $n$-gon $P$ and a distribution
$D$. Define, for each node $v$ of $T$,
\[
   \Xi(v)=\Delta(v)\setminus (\Delta(\lft(v))\cup \Delta(\rght(v)))
\]
and define $\Pr(v)=\Pr(\Xi(v))$.  Notice that the search for a point
$p$ terminates at $v$ precisely when $p\in\Xi(v)$.  Thus, $\Pr(v)$ is
the probability that a search terminates at node $v$.  For a set $V$
of nodes in $T$ we use the notation $\Pr(V)=\sum_{v\in V}\Pr(v)$ to
denote the probability that the search path ends at some node in $V$.

\begin{thm}\thmlabel{upper-bound}
A triangle tree $T$ contains $O(n)$ nodes and can be constructed in
$O(n)$ time.
Using the triangle tree, the expected number of linear inequalities
required to check if $p\in P$ for a point $p$ drawn from $D$ is at
most
\[
 \mu_D(T) 
   \le O(1)+O(1)\times \sum_{v\in T}\Pr(v)\log(1/\Pr(v)) \enspace .
\]
\end{thm}

\begin{proof}
For each $1\le i\le n$, let $e_i$ be the $i$th edge of $P$, defined so
that it does not include its first (clockwise) endpoint but does
include its second (counterclockwise) endpoint.  Observe that if,
during some iteration of the algorithm for constructing $T$, we select
a point $x\in e_i$ then $e_i$ moves to the boundary of the outer
approximation and no point of $e_i$ will ever be selected again.  This
implies that $T$ has $O(n)$ nodes.

To obtain an $O(n)$ time algorithm to construct $T$ we apply a trick
used by Mehlhorn \cite{m75} in the construction of biased binary
search trees.  Splitting a triangle $\Delta_j$ involves finding the
line $\ell$ and computing the intersection of $\ell$ with
$\boundary P\cap \Delta_j$.  The former operation takes, by
assumption, $O(1)$ time.  The latter operation, by using two
exponential searches in parallel, can be done in $O(\log
(\min\{m-k,k\}))$ time, where $m$ is the number of edges of $P$ that
intersect $\Delta_j$ and $k$ is the rank in this set of the edge that
intersects $\ell$.  In this way, the overall running time of the
construction algorithm is given by the recurrence
\[
    T(n) \le T(n-k) + T(k) + O(\log(\min\{n-k,k\}))
\]
which resolves to $O(n)$.

The expected running time of the query algorithm on $T$ is follows
immediately from the fact that, for any node $v$ at a distance of $i$
from the root of $T$, $\Pr(v) \le \Pr(\Delta(v)) \le 1/2^i$.
\end{proof}


\subsection{Optimality of the Triangle Tree}

Next we will show that the performance bound given by
\thmref{upper-bound} is optimal.  More precisely, we show that there
is no linear decision tree whose expected search time (on distribution
$D$) is asymptotically better than that of the triangle tree.  The key
ingredient in our argument is the following lemma:

\begin{lem}\lemlabel{lower-bound}
Let $V$ be a subset of the vertices of the triangle tree $T$ such that
no vertex in $V$ is the descendent of any other vertex in $V$.  
Let $R=\bigcup_{v\in V} \Xi(v)$. Then, for any linear decision
tree $T'$,
\[
    \mu_{D_{|R}}(T') + \log\left(\mu_{D_{|R}}(T')\right)
	\ge \sum_{v\in V}\Pr(\Xi(v)\mid R)\log(1/\Pr(\Xi(v)\mid R)) \enspace .
\]
\end{lem}

\begin{proof}
We will show that $T'$ can be transformed into a tree $T''$, with
$\mu_{D_{|R}}(T') + \log\left(\mu_{D_{|R}}(T')\right) \ge
\mu_{D_{|R}}(T'')$, and such that $T''$ is a point location structure
for $R$. By then applying \thmref{shannon} to $T''$ we obtain the
lemma.

Recall that, for a leaf $\ell$ in $T'$, $r(\ell)$ is the set of all query
points whose search path, in $T'$, leads to $\ell$.  Because $T'$ is a
linear decision tree, it follows that $r(\ell)$ is a
convex polygon and, if $r(\ell)$ is
a convex $k$-gon, then $\depth(\ell) \ge k$.  To obtain $T''$ we will
subdivide the leaves of $T'$ so that, for each leaf $\ell$, $r(\ell)$
intersects $\Xi(v)$ for at most one $v\in V$.

The tree $T'$ has two kinds of leaves.  An \emph{out leaf} is a leaf
$\ell$ such that $r(\ell)$ is disjoint from $P$.  In this case, it
follows from the fact that $r(\ell)$ is convex and disjoint from $P$,
that $r(\ell)$ intersects $\Xi(v)$ for at most one $v\in V$.

An \emph{in leaf} is a leaf $\ell$ of $T'$ such that $r(\ell)$ is
contained in $P$.  For an in leaf $\ell$, such that $r(\ell)$ is a
$k$-gon, there can be up to $k$ vertices $v$ in $V$ such
that $r(\ell)$ intersects $\Xi(v)$. However, in this case, it is
always possible to partition $r(\ell)$, using a single line, into two generalized polygons,
that intersect $\ceil{k/2}$ and $\floor{k/2}$, respectively, of the
$\Xi(v)$.  By doing this recursively, we obtain a tree $T''$ such
that, for each leaf $\ell'\in T''$, there is at most one $v\in V$ such
that $r(\ell')$ intersects $\Xi(v)$. Furthermore, for any point
$p\in\R^2$ for which the search path ends at $\ell$, respectively
$\ell'$ in $T'$, respectively $T''$, we have $\depth(\ell') \le
\depth(\ell) + \log(\depth(\ell))$.  By labelling the leaves of $T''$
appropriately we obtain a point location structure for $R$. Applying
\thmref{shannon} to this structure we obtain
\begin{eqnarray*}
\sum_{v\in V}\Pr(\Xi(v)\mid R)\log(1/\Pr(\Xi(v)\mid R)
 & \le & \mu_{D_{|R}}(T'')  \\
  & = & 
     \sum_{\ell'\in L(T'') v\in V}\Pr(r(\ell')\mid R)\depth(\ell') \\
  & \le & 
     \sum_{\ell\in L(T') v\in V}
         \Pr(r(\ell)\mid R)
            \left(\depth(\ell)+\log(\depth(\ell))\right) \\
  & \le &
     \mu_{D_{|R}}(T') + \log\left(\mu_{D_{|R}}(T')\right) \enspace ,
\end{eqnarray*}
where the last inequality follows from Jensen's Inequality \cite{j06}
and concavity of the logarithm.
\end{proof}

The remainder of our argument involves carefully piecing together
subsets of the nodes in $T$ and applying 
\lemref{lower-bound} to these subsets.  By doing this carefully, we
will eventually obtain a lower bound that matches the upper bound
in \thmref{upper-bound}.  Let 
\begin{equation}
   H = \sum_{v\in T} \Pr(v)\log (1/\Pr(v)) \eqlabel{H}
\end{equation}
be the entropy of the distribution of search paths in the triangle
tree $T$.  Note that,
ignoring the $O(1)$ term,
the upper bound of \thmref{upper-bound} is within a constant factor of
$H$. Thus, our goal is to show that no linear decision tree has an
expected search time in $o(H)$.

We start our analysis by partitioning the internal nodes of $T$ in
\emph{groups} $G_1,G_2,\ldots$ where
\[
	G_i = \{v\in T : 1/2^{i} \le \Pr(v) < 1/2^{i-1} \} \enspace .
\]
In what follows, we fix some a real number $0< \alpha < 1$ that will be
specified later.  Our first
result shows that, in our lower bound, we can discard a fairly large
number of elements from each group without having much effect on the
overall entropy:

\begin{lem}\lemlabel{garbage}
For each $i$, let $G_i'$ be obtained by deleting at most $2^{\alpha
i}$ elements from $G_i$.  Then
\[
    \sum_{i=1}^\infty \sum_{v\in G_i'} \Pr(v)\log(1/\Pr(v)) 
       \ge H-\blah \enspace .
\]
\end{lem}

\begin{proof}
\begin{eqnarray*}
   H & = & \sum_{v\in T} \Pr(v)\log(1/\Pr(v)) \\
   & = & \sum_{i=1}^{\infty}\sum_{v\in G_i'} \Pr(v)\log (1/\Pr(v)) +
         \sum_{i=1}^{\infty}\sum_{v\in G_i\setminus G_i'} \Pr(v)\log (1/\Pr(v)) \\
   & \le & \sum_{i=1}^{\infty}\sum_{v\in G_i'} \Pr(v)\log (1/\Pr(v)) +
         \sum_{i=1}^{\infty}\sum_{v\in G_i\setminus G_i'} (1/2^{i-1})\log (2^i) \\
   & \le & \sum_{i=1}^{\infty}\sum_{v\in G_i'} \Pr(v)\log (1/\Pr(v)) +
         \sum_{i=1}^{\infty}\sum_{v\in G_i\setminus G_i'} 2i(1/2^{i}) \\
   & \le & \sum_{i=1}^{\infty}\sum_{v\in G_i'} \Pr(v)\log (1/\Pr(v)) +
         \sum_{i=1}^{\infty} 2i(1/2^{i-\alpha i}) \\
   & = & \sum_{i=1}^{\infty}\sum_{v\in G_i'} \Pr(v)\log (1/\Pr(v)) +
         2\sum_{i=1}^{\infty} i/(2^{1-\alpha})^{i} \\
   & = & \sum_{i=1}^{\infty}\sum_{v\in G_i'} \Pr(v)\log (1/\Pr(v)) 
        + \frac{2/2^{1-\alpha}}{(1-1/2^{1-\alpha})^2} \\
   & = & \sum_{i=1}^{\infty}\sum_{v\in G_i'} \Pr(v)\log (1/\Pr(v)) + \blah
	\enspace ,
\end{eqnarray*}
where the last equality is obtained using Taylor series.
\end{proof}

In order to use \lemref{lower-bound} we must partition the vertices of
$T$ into subsets that are compatible with the conditions of the lemma.

\begin{lem}\lemlabel{partition}
Each group $G_i$ can be partitioned into $t_i$ subgroups
$G_{i,1},\ldots,G_{i,t_i}$ such that
\begin{enumerate}
\item $|G_{i,t_i}|\le 2^{\alpha i}$.

\item $|G_{i,j}| \ge 2^{\alpha i} / i$, for all $1\le j< t_i$, and

\item for every $1\le j< t_i$ and every $u,v\in G_{i,j}$ node $u$ is
not an ancestor of node $v$ in $T$. 

\end{enumerate}
\end{lem}

\begin{proof}
Assume that $|G_i|> 2^{\alpha i}$, otherwise there is nothing to
prove.  Observe that all vertices in $G_i$ appear within the first $i$
levels of $T$.  Thus, any node in $G_i$ has at most $i-1$ ancestors in
$T$.  

We can obtain the first subgroup $G_{i,1}$ by first defining all nodes of
$G_i$ to be \emph{unmarked} and \emph{unselected}.  To obtain
$G_{i,1}$ we repeatedly \emph{select} any unselected and unmarked
node $v\in G_i$ that does not have any descendants in $G_i$ and
\emph{mark} the (at most $i-1$) ancestors of $v$ in $T$.  This
process selects at least
\[
   |G_i|/i \ge 2^{\alpha i}/i
\] 
elements to take part in $G_{i,1}$.  We can then apply this process
recursively on $G_i\setminus G_{i,1}$ to obtain the sets
$G_{i,2},\ldots,G_{i,t_i-1}$.  Once this is done, we place the at most
$2^{\alpha i}$ remaining elements in group $G_{i,t_i}$.
\end{proof}

We now have all the tools we need to prove our lower bound.

\begin{thm}\thmlabel{lower-bound}
Let $\epsilon > 0$ be an arbitrarily small constant.
Any linear decision tree $T'$ that solves the problem of testing 
if any query point 
$p\in\R^2$ is contained in $P$ has
\[
   \mu_D(T') \ge H - O(H^{2/3}) \enspace .
\]
\end{thm}

\begin{proof}
Our proof is an application of the little-birdie principle.    We work in a slightly stronger model of
computation in which we are given a triangle tree $T=T(P,D)$ and the partitioning of
the vertices of $T$ into the sets $G_{i,j}$ described in
\lemref{partition}.  In this model, an algorithm consists of a whole
collection of linear decision trees $T_{i,j}$; one for each group $G_{i,j}$.
Now, when the point $p$ is selected according to $D$, a \emph{little
birdie}
tells the algorithm which group $G_{i,j}$ contains the vertex $v$ such
that $p\in\Xi(v)$.  The algorithm then uses the information
provided by the little birdie to select the decision tree $T_{i,j}$
and uses $T_{i,j}$ to determine if $p\in P$.  The cost of this is the
cost of searching in $T_{i,j}$.  Thus, the expected cost of a search
in this model of computation is
\[
     \mu = \sum_{i=1}^\infty \sum_{j=1}^{t_i}
	\Pr(G_{i,j})\mu_{D_{i,j}}(T_{i,j}) \enspace ,
\]
where $D_{i,j}$ denotes the probability distribution $D$ conditioned
on the search for $p$ ending at a node in $G_{i,j}$.  Clearly this
model of computation is at least as strong as the linear decision tree
model since, in this model, there is always the option of ignoring the
birdie's advice by creating a single linear decision tree $T'$ and
setting $T_{i,j}=T'$ for all $i$ and $j$.

Now, applying \lemref{lower-bound} to each group $G_{i,j}$ 
we obtain
\begin{eqnarray*}
\mu & = & \sum_{i=1}^{\infty}\sum_{j=1}^{t_i}
	\Pr(G_{i,j})\mu_{D_{i,j}}(T_{i,j}) \\
& \ge & \sum_{i=1}^{\infty}\sum_{j=1}^{t_i-1}
	\Pr(G_{i,j})\mu_{D_{i,j}}(T_{i,j}) \\
& \ge & \sum_{i=1}^{\infty}\sum_{j=1}^{t_i-1}\Pr(G_{i,j})\times
	\left(
	\sum_{v\in G_{i,j}}\Pr(v\mid G_{i,j})\log(1/\Pr(v\mid G_{i,j}))
	\right) 
        - \log\mu \\
%& = & \sum_{i=1}^{\infty}\sum_{j=1}^{t_i-1}\Pr(G_{i,j})\times
%	\left(
%	\sum_{v\in G_{i,j}}\Pr(v\mid G_{i,j})\log(1/\Pr(v\mid G_{i,j}))
%	\right) 
%        - \log\mu \\
& = & \sum_{i=1}^{\infty}\sum_{j=1}^{t_i-1}
	\sum_{v\in G_{i,j}}\Pr(v)\log(\Pr(G_{i,j})/\Pr(v)) 
        - \log\mu \\
& = & \sum_{i=1}^{\infty}\sum_{j=1}^{t_i-1}
	\sum_{v\in G_{i,j}}\Pr(v)(\log(1/\Pr(v))+ \log(\Pr(G_{i,j})) 
        - \log\mu \\
& \ge & H-\blah +
        \sum_{i=1}^{\infty}\sum_{j=1}^{t_i-1}
	\sum_{v\in G_{i,j}}\Pr(v)\log(\Pr(G_{i,j})) 
        - \log\mu \\
& \ge & H-\blah +
        \sum_{i=1}^{\infty}\sum_{j=1}^{t_i-1}
	\sum_{v\in G_{i,j}}\Pr(v)\log(2^{\alpha i}/i2^{i}) 
        - \log\mu \\
& = & H-\blah 
        +\sum_{i=1}^{\infty}\sum_{j=1}^{t_i-1}
	\sum_{v\in G_{i,j}}\Pr(v)(\alpha i -i -\log i) 
        - \log\mu \\
& \ge & H-\blah + \alpha H - H - \log H - 2 - \log\mu \\
& \ge & \alpha H-\blah - \log H - \log \mu - 2 \enspace . \\
& \ge & \alpha H-\blah - 2\log H - 2 \enspace . \\
\comment{
&=&XXXXXXXXXXXXXXXXXXXXXXx \\
& = & \sum_{i=1}^{\infty}\sum_{j=1}^{t_i-1}
	\sum_{v\in G_{i,j}}\Pr(v)(\alpha i  -1 -\log i) 
        - \log\mu \\
\comment{
& \ge & \frac{1}{4}\sum_{i=1}^{\infty}\sum_{j=1}^{t_i-1}
	\sum_{v\in G_{i,j}}\Pr(v)(i/4 - 3) \\
& \ge & \left(\frac{1}{24}\sum_{i=1}^{\infty}\sum_{j=1}^{t_i-1}
	\sum_{v\in G_{i,j}}\Pr(v)\log (1/\Pr(v))\right)-4 \\
}
& \ge & H - \alpha\blah - O(1) - 2\log\mu \enspace ,
where the last inequality follows from \lemref{garbage} and by setting
$\alpha = 1-\delta$ for a sufficiently small constant
$\delta=\delta(\epsilon) > 0$.
}
\end{eqnarray*}
Setting $\alpha=1-1/H^{1/3}$, and rearranging terms
we obtain
\[
    \mu \ge H - O(H^{2/3}) \enspace ,
\]
completing the proof.
\end{proof}

\section{Convex Subdivisions}
\seclabel{subdivisions}

In this section we consider the problem of point location in convex
subdivisions. Our data structure is simple.  For each internal face
$F_i$ of the convex subdivision we triangulate the interior of $F_i$
using the internal edges of the triangles of a triangle tree
$T_i=(F_i,D_{|F_i})$ for the polygon $F_i$ and the distribution
$D_{|F_i}$. For the outer face, we do the same but keep only the
external edges of the triangles.  Next, we preprocess the resulting
triangulation using some linear-space distribution-sensitive point location
data structure for triangulations \cite{ammw07,i01}.

We have three lower bounds on the expected query time of any linear
classification tree for point location.  The first lower bound follows
from the fact that any classification tree with more than 1 class must
have at least one internal node:
\[
    B_0 = \Omega(1)
\]
The second lower bound is the
entropy bound:
\[
	B_1 = \sum_{i=1}^f \Pr(F_i)\log(1/\Pr(F_i)) \enspace .
\]
The third lower bound is a bit more subtle.  It follows from the fact
that, inside any classification tree for point location is a decision
tree for testing, for each $1\le i\le f$, if a query point $p$ is
contained in $F_i$.  From \thmref{lower-bound} we know that, for any
decision tree $T'$ that determines if a query point $p$ is in $F_i$
\[
   \mu_{D_{|F_i}(T')} \ge \Omega(1)\times \sum_{v\in T_i}\Pr(v\mid
F_i)\log(1/\Pr(v\mid F_i))
	\enspace .
\]
Summing over all faces, we obtain the third lower bound:
\[
   B_2 = \Omega(1)\times \sum_{i=1}^f \sum_{v\in T_i}\Pr(F_i)\Pr(v\mid
F_i)\log(1/\Pr(v\mid F_i))
	\enspace .
\]

Now, because we store all the triangles of each $T_i$ in a point
location structure that achieves the entropy bound, the resulting
structure has an expected query time of 
\begin{eqnarray*}
\mu & = & O(1)+O(1)\times \sum_{i=1}^f \sum_{v\in T_i} \Pr(v)\log(1/\Pr(v)) \\
& = & O(1)+O(1)\times\sum_{i=1}^f \Pr(F_i)\times \sum_{v\in T_i}
\Pr(v\mid F_{i})\log(1/\Pr(v)) \\
& = & O(1)+O(1)\times\sum_{i=1}^f \Pr(F_i)\times \sum_{v\in T_i}
	\Pr(v\mid F_{i})\log(\Pr(F_i)/\Pr(F_i)\Pr(v)) \\
& = & O(1)+O(1)\times\left(\sum_{i=1}^f \Pr(F_i)\log (1/\Pr(F_i)) + 
        \sum_{i=1}^f \sum_{v\in T_i}
	\Pr(F_i)\Pr(v\mid F_{i})\log(1/\Pr(v\mid F_i))\right) \\
& \le & O(1)+O(1)\times(B_1 + B_2) \\
& \le & O(1)\times \max\{B_1, B_2,B_3\} \enspace .
\end{eqnarray*}
This completes the proof of our main result:

\begin{thm}
Given a convex subdivision $G$ with $n$ vertices and a probability measure
$D$ over $\R^2$, a data structure of size
$O(n)$ can be constructed in $O(n)$ time that answers point location queries in 
$G$. 
The expected query time of this data structure, for a point $p$
drawn according to $D$ is $O(\mu_D(T))$, where $T$ is any linear
classification tree that answers point location queries in $G$.
\end{thm}


\section{Discussion}
\seclabel{discussion}

We have presented a data structure for distribution-sensitive point
location in convex subdivisions.  Our data structure achieves, up to
constant factors, the best possible query time for any data structure
in the linear decision tree model of computation.  

Our data structure, as described in \secref{polygons}, uses
comparisons between the query point $p$ and precomputed lines
determined by points on the edges of $P$ that are, in turn, determined
by the distribution $D$.  We note that we can obtain an equally
efficient structure that only does comparisons involving lines through
two vertices of $G$.  To achieve this, we simply modify the splitting
process at the nodes of the triangle tree so that, instead of placing
a single point in the interior of an edge $e$ of $P$ (\figref{split}),
we place one point on each of the endpoints of $e$
(\figref{improved-split}). This modification has another nice
property; it can be implemented so that the only mechanism needed to
access the probability measure $D$ is a primitive for computing the
probability of a triangle.  An interesting corollary of this result is
that, given a convex polygon $P$ and a probability distribution over
the interior of $P$, the above construction gives a linear-time
constant factor approximation for the minimum entropy triangulation of
$P$.

\begin{figure}
\begin{center}\begin{tabular}{cc}
\includegraphics{improved-a} & \includegraphics{improved-b}
\end{tabular}
\end{center}
\caption{The data structure can be realized using only lines through
the vertices of $P$.}
\figlabel{improved-split}
\end{figure}

In practice, performing computations with probability distributions is
at best awkward.  However, one potential real application of our data
structures is when the distribution $D$ is uniform over a set of $m$
points.  Such distributions are easily obtained by sampling some
(unknown) continuous distribution and are often a good enough
approximation of the continuous distribution.  In this case, it is
fairly straightforward to construct our point-location structure in
$O(n+ m\log(mn))$ time.

When discussing distribution-sensitive point location, the exact
definition of the problem and the choice of computational model is
important.  For example, our definition of the problem requires that
our data structure outputs the correct answer for every input point in
$\R^2$.  This means that, for example, we can obtain non-trivial lower
bounds for testing if $p\in P$ even in the case when $\Pr\{p\in
P\}=1$. One could imagine another definition of the problem in which
the data structure is only required to answer correctly for points
that are in the support of $D$.

We have given upper and lower bounds on the expected complexity of
testing if a point is in a convex polygon and these bounds match to
within a constant factor.  More precisely, the expected number of
linear inequalities tested by any linear decision tree is at least
$H-O(H^{2/3})$ and our data structure tests $3H+O(1)$ linear
inequalities. The constant 3 appears in the upper bound because the
expected depth of a search in the triangle tree is at most $H$ and
each node in the triangle tree tests the query point against 3 lines.
However, we could alternatively store the triangles of the triangle
tree using the recent point location structure of Mount \etal\
\cite{ammw07}, in which case the query time becomes $H+O(H^{1/2}+1)$ and
we obtain an $O(n)$ space structure whose query time is optimal to
within a lower-order ($H^{2/3}$) term.

We have studied point location in convex polygons and convex
subdivisions.  From here we could try to extend our results to simple
polygons and general planar subdivisions whose faces are arbitrary
simple polygons. Perhaps a more realistic next step is to study the
related problem of \emph{vertical ray shooting}:  Preprocess a set $S$
of points and open line segments so that, for any query point $p$, we
can determine the first object in $S$ intersected by an upward
vertical ray originating at $p$.  Vertical ray shooting is often used
interchangeably with planar point location but is a strictly harder
problem since it sometimes requires the data structure to distinguish
between edges of the same face.  At the heart of the vertical ray
shooting problem is the following decision problem:

\begin{op}
Let $P$ be a simple polygon whose
boundary consists of one line segment and one $x$-monotone chain
joining the two endpoints of the line segment and let $D$ be a
probability measure over $\R^2$. Preprocess $P$ and $D$ into a data
structure that can test if a query point $p$ is contained in $P$ and
whose expected query time is optimal in the linear decision tree model
of computation.
\end{op}

\bibliographystyle{plain}
\bibliography{entropy}



\end{document}
