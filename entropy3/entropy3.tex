\documentclass{patmorin}
\usepackage{pat}

\title{\MakeUppercase{Point Location in Disconnected Planar Subdivisions}}

\author{Prosenjit~Bose, 
	Karim~Dou\"{\i}eb, 
	Vida~Dujmovi\'c, 
	Jamie~King, and 
	Pat~Morin}



\begin{document}
\maketitle

\section{Introduction}


Collette \etal\ gave an $O(n)$ space data structure that preprocesses
a connected planar subdivision $G$ and a probability measure $D$ over
$\R^2$ such that a point location query in $G$ can be answered in $O(H)$
expected time using.  The expected number of point-line comparisons needed
to answer a query is $H + O(H^{2/3}+1)$.  Here $H$ is a lower-bound on
the expected time required by any linear decision tree for answering
queries on $G$ that are drawn according to $D$. Their work leaves open
the problem of what to do when $G$ is disconnected.


\section{The Data Structure}
\seclabel{data-structure}

In this section we describe our data structure for searching in
disconnected planar subdivisions.  The first tool we use is simplicial
partitions, from the field of geometric range searching:

\begin{thm}[Matou\v{s}ek 1991]\thmlabel{point-partition}
Let $S$ be a set of $n$ points in $\R^2$. There exists a constant
$c$ such that, for any $1\le r \le n$, there exists a sequence
$\langle \Delta_1,\ldots,\Delta_r\rangle$ of triangles such that
  \begin{enumerate}
    \item $S\subseteq \bigcup_{i=1}^r \Delta_i$,
  
    \item $\left|\Delta_i \cap S\setminus
    \left(\bigcup_{j=1}^{i-1}\Delta_j\right)\right| \le 2n/r$, and
  
    \item For any line $\ell$, there are at most $cr^{1/2}$ elements of
  $\{\Delta_1,\ldots,\Delta_r\}$ that intersect $\ell$.
  \end{enumerate}
\end{thm}

Note that Part~2 of \thmref{point-partition} is not the original
statement of the theorem, but follows from Matou\v{s}ek's construction
of $\Delta_1,\ldots,\Delta_r$ \cite{m91}.
Restating \thmref{point-partition} in terms of probability distributions,
we have:

\begin{thm}\thmlabel{prob-partition}
Let $S$ be a set of $n$ points in $\R^2$. There exists a constant
$c$ such that, for any $1\le r \le n$, there exists a sequence
$\langle \Delta_1,\ldots,\Delta_r\rangle$ of triangles such that
  \begin{enumerate}
    \item $\Pr\{\bigcup_{i=1}^r \Delta_i\} = 1$,
  
    \item $\Pr\left\{\Delta_i \cap S\setminus
    \left(\bigcup_{j=1}^{i-1}\Delta_j\right)\right\} \le 2/r$, and
  
    \item For any line $\ell$, there are at most $cr^{1/2}$ elements of
    $\{\Delta_1,\ldots,\Delta_r\}$ that intersect $\ell$.
  \end{enumerate}
\end{thm}

(Someone needs to prove this, or something close, as well as give an
algorithmic version.  Sampling from $D$ and then applying
\thmref{point-partition} should do it.)

We use \thmref{prob-partition} to recursively construct a
\emph{partition tree} $T$.  Let $\alpha > 0$ be a constant that will
specified below.  At the root of $T$, we find the set of triangles
$\Delta=\{\Delta_1,\ldots,\Delta_r\}$ and construct the arrangement of
triangles in $\Delta$ and triangulate its faces to obtain a triangulation
$A$. The exact method we use to triangulate the faces of the arrangement
will be discussed in the next section.  This triangulation is stored at
the root of the tree.

Next, each face $F$ of $A$ becomes a child of the root of $T$.  If $F$
does not intersect any edge or vertex of $G$ then we call $F$ a
\emph{terminal leaf} and label $F$ with the face of $G$ that contains it.
If the current depth of recursion is greater than $\lfloor\log_{\alpha r}
n\rfloor$ then $F$ becomes a \emph{non-terminal leaf} of $T$.  Otherwise
($F$ intersects an edge or vertex of $G$ and its depth is small),
we recursively apply the same procedure on the distribution $D_{|F}$
to obtain a partition tree that becomes a child of the root.

This construction defines a tree $T=T(S,D)$ in which each node has
$O(r^2)$ children and whose height is at most $\log_{\alpha r} n$.
The number of nodes in $T$ is $(O(r^2))^{\log_{\alpha r} n} =
O(n^{2\alpha+\epsilon})$, where $\epsilon$ is a decreasing function
of $r$.  Note that, for $\alpha < 1/2$ and sufficiently large $r$,
the size of $T$ is $o(n)$.

In addition to the tree $T$ we construct a backup data structure $T'$ that
can answer point location queries in $G$ in $O(\log n)$ worst-case time.

To answer a query, $T$ and $T'$ are used as follows.  We search in $T$
for the query point. If this search ends at a terminal leaf $F$ of $T$
then we report the label at $F$ and the query is complete.  Otherwise we
use $T'$ to answer the query in $O(\log n)$ time.

\section{Analysis}
\seclabel{analysis}

Collette \etal\ \cite{cdilm08,cdilm09} show that, up to a lower-order
term, the expected number of comparisons performed by the optimal
decision tree for point location in $G$ is equal to the entropy of the
minimum-entropy Steiner triangulation of $G$.

\begin{thm}[Collette \etal\ 2008]
Let $G$ be a planar subdivision and let $D$ be a probability measure
over $\R^2$.  Let $T^*$ be a  minimum-entropy Steiner triangulation of
$G$ and let $H^*$ be the entropy of $T^*$.  Then any linear decision tree
for point location in $G$ has expected cost at least $H^*-O(\log H^*)$.
\end{thm}

Thus, our goal is to prove that our query time approximates the entropy
of the minimum entropy Steiner triangulation of $G$.  We will do this
by an argument similar to that used by Dujmovi\'c \etal\ \cite{dhm09}
in the context of orthogonal range searching.

An $i$-set of a rooted tree $T$ is a set of vertices in $T$ all of
which are at distance at most $i$ from the root of $T$ and in which
no vertex in the set is the ancestor of any other vertex in the set.
Note that if $T$ is a partition tree defined in \secref{data-structure}
then an $i$-set of $T$ is a set of disjoint triangles.

\begin{lem}
Let $T$ be the partition tree defined in \secref{data-structure}
and let $V$ be an $i$-set of $T$.  Then, the number of elements of
$V$ intersected by any line $\ell$ is $O(|V|^{1/2+\epsilon})$, where
$\epsilon$ is a decreasing function of $r$.
\end{lem}

\begin{proof}
Use the ``Shoot a ray take a walk'' triangulation of the faces of $A$.
\end{proof}

We say that a set of regions $X=\{X_1,\ldots,X_m\}$, $X_i\subseteq\R^2$,
is in $k$-general position if there is no line that intersects $k$
elements of $X$.

\begin{lem}
Let $T$ be the partition tree defined in \secref{data-structure} and let $V$ be an $i$-set of $T$.
Let $V$ be a subset of the nodes of $T$ such that no node no node in $V$
is the ancestor of any other node and that all nodes in $V$ have depth
at most $i$.  Then $V$ contains a subset $V'\subseteq V$ that is in $k$-general position and has size at least $V'/Q$.
\end{lem}

\begin{proof}
\end{proof}


\bibliographystyle{plain}
\bibliography{entropy3}
\end{document}
