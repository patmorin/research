\documentclass[lotsofwhite]{patmorin}
\usepackage{html}
\usepackage{pat}
\definecolor{linkblue}{named}{Blue}
\hypersetup{colorlinks=true, linkcolor=linkblue,  anchorcolor=linkblue,
citecolor=linkblue, filecolor=linkblue, menucolor=linkblue, pagecolor=linkblue,
urlcolor=linkblue, pdfcreator=Me, pdfproducer=Me} \setlength{\parskip}{1ex}

\DeclareMathOperator{\spn}{span}
\DeclareMathOperator{\tp}{top}
\DeclareMathOperator{\bttom}{bottom}

\title{2-Trees as Subgraphs of Rectangle Intersection Graphs}
\author{Banff 2016}

\begin{document}
\maketitle

\section{Introduction}

We prove a Ramsey-Type Theorem about representing 2-trees as subgraphs
of rectangle intersection graphs.  A consequence of this result is
that 2-trees do not have a representation as the product of two path
decompsitions whose bags have bounded intersection.

\subsection{Definitions}

We define some geometric terminology and a particular graph.

\subsubsection{Rectangles}

Throughout this paper, the word \emph{rectangle} means open axis-aligned
rectangle: a subset of $\R^2$ of the form $(\ell,r)\times (b,t)$,
where $\ell< r$ and $b<t$.  The rectangle $R=(\ell,r)\times (b,t)$
has four \emph{corners} $c_1=(\ell,b)$, $c_2=(\ell,t)$, $c_3=(r,b)$,
$c_4=(r,t)$.\footnote{We use the word corner rather than vertex to reduce
confusion between vertices of rectangles and vertices of graphs.}  $R$
also has four \emph{sides} that are open line segments: the top side
has endpoints $c_2$ and $c_4$; the bottom side has endpoints $c_1$ and
$c_3$; the left side has endpoints $c_1$ and $c_2$; and the right side
has endpoints $c_3$ and $c_4$.

Let $v$ and $w$ be two rectangles that intersect and such that $w$ does
not contain a corner of $v$.  The intersection $R=v\cap w$ is a rectangle.
We say that $(v,w)$ is an h-pair if the left or right side of $R$ is
contained in $\partial v$.  We say that $(v,w)$ a v-pair if the top or
bottom side of $R$ is contained in $\partial(v)$.  If $(v,w)$ is not an
h-pair or a v-pair, then we call it an o-pair.  Note that, since $w$
does not contain a corner of $v$, $(v,w)$ is exactly one of a v-pair,
an h-pair or an o-pair.

We say that a sequence of rectangles $v_1,\ldots,v_k$ is a \emph{good
sequence} if
\begin{enumerate}
  \item for each $i\in\{2,\ldots,k\}$, $v_i\cap v_{i-1}\neq \emptyset$
        and $v_i$ does not contain any corner of $v_{i-1}$; and
  \item for each $i\in\{3,\ldots,k-1\}$, $(v_{i-1},v_i)$ and
        $(v_i,v_{i+1})$ are not both h-pairs and not both v-pairs.
\end{enumerate}
We will make use of the following simple geometric lemma:
\begin{lem}\lemlabel{hv-triples}
  If $v_1,v_2,v_3$ is a good sequence of rectangles then
  $v_1\cap v_2\cap v_3=v_2\cap v_3 \neq \emptyset$.
\end{lem}

\begin{proof}
  Easy.
\end{proof}

Using the transitivity of set inclusion, \lemref{hv-triples} has the
following corollary:

\begin{cor}\corlabel{hv-sequence}
  If $v_1,\ldots,v_k$ is a good sequence of rectangles then 
  $\bigcap_{i=1}^{k} v_i = v_{k-1}\cap v_k \neq \emptyset$.
\end{cor}

A \emph{rectangle intersection graph} is a graph whose vertices are
rectangles and the edge between two rectangles $u$ and $w$ is present
if and only $u\cap w\neq \emptyset$.

\subsubsection{2-Trees}


The \emph{height-$h$ $d$-branching universal 2-tree}, $T_{h,d}$ is
defined recursively as follows:
\begin{itemize}
  \item $T_{0,d}$ is the empty graph;
  \item $T_{1,d}$ is a two-vertex graph with a single edge;
  \item For $h\ge 1$, $T_{h,d}$ is obtained from $T_{h-1,d}$ by adding,
     for each edge $vw \in E(T_{i-1,d})\setminus E(T_{i-2,d})$,
     a new vertex $x_{vw}$ adjacent to both $v$ and $w$.
\end{itemize}
The \emph{root edge} of $T_{k,d}$ is the single edge of $T_{1,d}$.
The $i$-vertices of $T_{k,d}$ are \[ V_i(T_{k,d}) = V(T_{i,d}\setminus
V(T_{i-1,d}) \] and the $i$ edges of $T_{k,d}$ are \[ E_i(T_{k,d})
= E(T_{i,d}\setminus E(T_{i-1,d}) \enspace . \] For a vertex $v$ of
$T_{k,d}$, we use the notation $N_i(v)$ to denote the set of vertices
joined to $v$ by an $i$-edge.


\section{The Result}

\begin{thm}
  For each $k\in \N$, any rectangle intersection graph that contains
  $T_{4k-7,k^2}$ as a subgraph contains a clique of size $k$.
\end{thm}

\begin{proof}
  The cases $k=1$ and $k=2$ are trivial so, for the remainder of the proof
  we assume that $k\ge 2$.

  Let $G$ be a rectangle intersection graph that contains
  $T=T_{4k-7,BLAH}$ as a subgraph.  We can assume that $V(G)=V(T)$
  so that vertices of $T$ are rectangles in $V(G)$ and if vertices $v$
  and $w$ are adjacent in $T$, then the rectangles $v$ and $w$ intersect
  (but not necessarily \emph{vice versa}).  

  We will attempt to define a path $v_1,\ldots,v_k$ in $T$ such that
  $v_1,\ldots,v_k$ is a good sequence of rectangles and therefore,
  by \corref{hv-sequence}, $v_1,\ldots,v_k$ form a clique in $G$.
  Each vertex, $v_i$, in this path will be a level-$t_i$ vertex of $T$
  for $t_i \le ik$.  The only cases in which we are unable to complete
  this path occur because some point $x\in\R^2$ is contained in a set
  $X$ of $k$ rectangles of $T$.  In these cases, the set $X$ forms a
  $k$-clique in $G$.

  Let $vw$ be the root edge of $T$, set $v_1=v$ and use the convention
  that $v_0=w$.  Suppose now that we have already constructed a partial
  sequence $v_1,\ldots,v_i$ and we want to extend it to $v_{i+1}$.
  Define $R=\bigcap_{j=1}^i v_i$.  By \corref{hv-sequence}, $R$ is
  non-empty and $R=v_{i-1}\cap v_i$.

  If $v_i$ is a $t$-level vertex of $T$ with $t<k^2$, 
  then $v_{i-1}$ and $v_i$ are both adjacent to a set $S$ of $4k-7$
  level-$(t+1)$ vertices of $T$.  All the rectangles in $S$ intersect
  $R$.  
  If all the rectangles in $S$ contain a corner of $R$, then some corner
  of $R$ is contained in at least
  \[  \left\lceil |N_{i+1}(v_i)|/4]\right\rceil 
         = \left\lceil(4k-7)/4]\right\rceil = k-1
  \]
  rectangles.  Since these rectangles are open, there is a point $x$
  contained in these $k-1$ rectangles and in $v_{i}$.  These $k$
  rectangles therefore form a $k$-clique in $G$ and we are done.

  Otherwise, some rectangle $v\in S$ does not contain a corner of $R$.
  Notice that the sequence $v_1,\ldots,v_i,v$ satisifies all the
  properties of a good sequence except, possibly, the pairs $(v_{i-1},v_i)$
  and $(v_i,v)$ are both h-pairs or both v-pairs.  There are two cases
  to consider:
  \begin{enumerate}
     \item $v_1,\ldots,v_i,v$ is a good sequence of rectangles. In this
        case we say that the procedure \emph{succeeds} in this iteration
        and we set $v_{i+1}=v$.
     
     \item  $v_1,\ldots,v_i,v$ is not a good sequence.  In this case,
       we say that the procedure \emph{stalls} in this iteration.
       Without loss of generality, suppose $v_1,\ldots,v_i,v$ is not a
       good sequence because $(v_{i-1},v)$ and $(v_i,v)$ are both h-pairs.

       In this case, we change $v_i$ by setting $v_i=v$. We claim
       that the resulting sequence $v_1,\ldots,v_{i-1},v$ is a good
       sequence of rectangles.  To see why this is so, recall that
       $v_1,\ldots,v_{i-1}$ is a good sequence, so $(v_{i-2},v_{i-1})$
       is not an h-pair.  This means that $v_{i-2}\cap v_{i-1}
       \supseteq v_{i-1}\cap s$ where $s=(\bttom(v_i),\tp(v_i))\times
       (-\infty,\infty)$ is the horizontal strip bounded above and
       below by the top and bottom sides of $v_i$.  Since $(v_{i},v)$
       is an h-pair, $v\subseteq s$.  Therefore, $(v_{i-1},v)$ is also
       an h-pair or an o-pair and $v_1,\ldots,v_{i-1},v$ is a good sequence.

       Note that in this case we have failed to make our path any
       longer. Instead, we have only replaced the last element.
       Regardless, we continue again trying to extend the path with the
       new value of $v_i$.
  \end{enumerate}
  Notice that each iteration of the preceding two steps considers a new
  vertex, $v$, that is one level deeper in $T$ than the vertex considered
  during the previous iteration.  If we repeat these steps $k(k-1)$
  times, then one of two things occurs.
  \begin{enumerate}
     \item During these $k(k-1)$ iterations, the algorithm succeeds
     (Case~1, above) at least $k-1$ times, in which case we have
     successfully found the path $v_1,\ldots,v_k$ and whose vertices
     form a $k$-clique in $G$.

     \item During these $k(k-1)$ iterations, the algorithm stalls during
     $k$ consecutive iterations say while trying to select $v_{i+1}$.
     During this process the vertex $v_i$ takes on $k+1$ different
     values $v_{i,0},\ldots,v_{i,k}$.  Assume, again, without loss
     of generality that $(v_{i-1},v_i)$ is an h-pair.  Then, for each
     $j\in\{0,\ldots,k-1\}$, the pair $(v_{i,j}, v_{i,j+1})$ is an h-pair,
     otherwise the procedure would have selected $v_{i,j+1}$ as $v_{i+1}$.

     Furthermore, for each $j\in\{0,\ldots,k-1\}$, the pair $v_{i-1},v_{i,j}$ is an h-pair, otherwise the procedure would have selected 



, $k$ iterations, the algorithm
     considers a sequence of candidates, $v_{i,1},\ldots,v_{i,k}$ for $v_{i+1}$



  at least $k-1$ times, in which case we have
     successfully found the path $v_1,\ldots,v_k$ and whose vertices
     form a $k$-clique in $G$.
  \end{enumerate}


  If $(v_i,v_{i+1}')$ are an o-pair, then we set $v_2=v_2'$ and continue
  as described below



   To extend the path from $v_{i-1}$ to $v_{i}$ we define
%  the set
%  \[
%     S_i = \left\{v\in N_{i}(v_{i-1})\cup N_{i+1}(v_{i-1}) : \text{$v$ contains no corner of
%           $\bigcap_{j=1}^{i-1} v_j$}\right\}
%  \]
%  We may assume that $S_i$ is non-empty because, otherwise, one of the
%  corners of $v_{i-1}$ is contained in
%  \[  \left\lceil |N_{i+1}(v_i)|/4]\right\rceil 
%      = \left\lceil(4k-7)/4]\right\rceil = k-1
%  \]
%  rectangles. Since these rectangles are open, there is a point $x$
%  contained in these $k-1$ rectangles and in $v_{i-1}$.  These $k$
%  rectangles therefore form a $k$-clique in $G$.   
%  The vertex $v_{i}$ is chosen to be some vertex in $S_i$ that
%  minimizes $\spn(v_{i-1},v_{i})$.
%
%  We claim that, for each $i\in\{2,\ldots,k\}$, $v_1,\ldots,v_i$
%  has the following properties:
% \begin{enumerate}
%    \item $v_{i}$ contains no corners of $v_{i-1}$:
%    By choice, $v_i$ intersects, but contains no corner of
%    $\bigcup_{j=1}^{i-1} v_j$. This immediately implies that $v_i$
%    contains no corner of $v_{i-1}$ (or, even any $v_j$ for $j<i$).
%
%    \item $\spn(v_{i-1},v_i)$ is minimum over all $v_i \in S_i$:  This
%    is the definition of $v_i$.
%
%    \item $\bigcap_{j=1}^i v_j = v_{i-1}\cap v_i$: For $i=2$, this is
%       immediately, so we may assume $i\ge 3$.  By induction, we may 
%       assume that $\bigcap_{j=1}^{i-1} v_j= v_{i-1}\cap v_{i-2}$.
%       We make the following two observations about $v_{i-2}$ and $v_{i-1}$:
%       \begin{enumerate}
%         \item By Property~1, above, $v_{i-2}$ contains no corner of $v_{i-1}$.
%         \item Since $v_{i-2}$ and $v_{i-1}$ are joined by an edge in $T$,
%       $v_{j-1}\cap v_{j-2}\neq\emptyset$.  
%       \end{enumerate}
%       These two observations imply that
%       the rectangle $R=\v_{j-1}\cap v_{j-2}$ has two edges that are
%       contained in $\partial v_{j-1}$ and at most two edges that are
%       contained in $\partial v_{j-2}$.  We claim that $v_j
%
% (Property~1, above)
%       and $v_{
%
%
%=v_{j}$
%
%    \item FIXME: Unnecessary.  $\bigcap_{j=1}^i v_j \neq\emptyset$: For $i=2$, this follows
%    immediately from the fact that $v_1$ and $v_2$ are joined by an edge
%    in $T$, so we may assume that $i\ge 3$.
%
%    By the definitions of $S_i$ and $T$, the edges $v_{i-1}v_i$
%    and $v_{i-2}v_i$ are both in $T$, so that $v_i\cap v_{i-1}\cap
%    v_{i-2}\neq \emptyset$.  Since $v_1,\ldots,v_{i-1}$ has Property~3,
%    $v_{i-1}\cap v_{i-2}=\bigcap_{j=1}^{i-1} v_j$, so
%    \[
%       \bigcap_{j=1}^i v_j = v_i\cap v_{i-1}\cap v_{i-2} \neq \emptyset 
%         \enspace .
%    \]
%  \end{enumerate}
%  This completes the proof of the theorem, since the final property implies
%  that $v_1,\ldots,v_k$ form a $k$-clique in $G$.
\end{proof}



\bibliography{op}
\bibliographystyle{plainurl}
\end{document}
