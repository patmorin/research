
\comment{
%%%%%%%%%%%%%%%%%%%%%%%%%%%%%%%%%%%%%%%%%%%%%%%%%%%%%%%%%%%%%%%%%%%%%%%%%%
\section{Testing Cylinders}\seclabel{cylinders}

Before we can describe a quality testing procedure for cylinders, we
must generalize the notion of quality to cylinders.  In this work, we
are concerned with the roundess of cylinders, and not their height, or
the flatness of their ends.  For these reasons, we will assume that
our manufactured cylinders have length $l$, and that their ends are
perfectly flat.  The object, $J$, that we are interested in testing is
a compact simply connected subset of the space $(x,y,[0,l])$.

We assume that $J$ is resting on the $(x,y)$ plane and that we know
the orientation of the cylinder.  Define $J_h$ to be the set of all
points $(x,y)$ such that $(x,y,h)\in J$. Note that $J_h$ is a planar
object.  We define the {\em outer boundary} of $J$ as
\begin{equation} 
\jout = \bigcup_{0\le h\le_l}J_h \enspace , 
\end{equation}
and we define the {\em inner boundary} of $J$ as 
\begin{equation} 
\jin = \bigcap_{0\le h\le_l}J_h \enspace . 
\end{equation}

For a point $p$ on the plane, we use $R(p,J)$ and $r(p,J)$ to denote
the maximal and minimal distance, respectively, from $p$ to a point in
$\bd(\jout)$ and $\bd(\jin)$, respectively.  For a point $p$, let
\begin{eqnarray}
\qual(p,J) &=& \min\{r(p,J)-(1-\epsilon), (1+\epsilon)-R(p,J)\} \\  
\qual(J) \ &=& \max\{\qual(p, J) : p\in\mathbb{R}^3\} \enspace . 
\end{eqnarray}
We call any point $c_J$ with $\qual(c_J, J)=\qual(J)$ a {\em center}
of $J$.  Note that according to these definitions $J$ is an good
object if there exists an annulus of inner radius $1-\epsilon$ and
outer radius $1+\epsilon$ which covers both $\bd(\jin)$ and
$\bd(\jout)$. Intuitively, $J$ is good if the projection of all
points on the surface of $J$ onto the $(x,y)$ plane can be covered by
an annulus of inner radius $1-\epsilon$ and outer radius $1+\epsilon$.

As part of our minimum quality assumptions for cylinders we have the
following generalizations of Assumptions~\ref{ass:mq} and
\ref{ass:ss}.

\begin{ass}\asslabel{mq-cylinder}
$R(c_J,J)\le 1+\delta$ and $r(c_J,J)\ge 1-\delta$, for some constant
$0<\delta<1/21$, i.e., $\jout\setminus\jin$ is contained
in an annulus of inner radius $1-\delta$ and outer radius $1+\delta$.
\end{ass}

\begin{ass}\asslabel{ss-cylinder}
For all $0\le h\le l$, $J_h$ is a star-shaped object, and its kernel
contains all points $p$ such that $\dist(c_J,p)\le\alpha$, for some
constant $1-\delta>\alpha>2\delta$.
\end{ass}

Note that these assumptions are not sufficient.  In fact, with only
these assumptions, any probing strategy which uses a finite number of
probes can be fooled into accepting an bad object.  This is because it
only takes one infinitely thin ``bad'' cross section to make an bad
object.  Thus, we need another assumption.

\begin{ass}\asslabel{weird-cylinder}
Let $J$ be an object with center $c_J$, and let $J[h-\alpha,h+\alpha]$
be the subset of points in $J$ with $z$ coordinate in
$[h-\alpha,h+\alpha]$.  Let $K$ be the intersection of the infinite
length cylinder of radius $\alpha$, which is perpendicular to the
$(x,y)$ plane and is centered at $O$ with $J[h-\alpha,h+\alpha]$.
Then $J[h-\alpha,h+\alpha]$ is a star shaped object with kernel $K$.
\end{ass}

Note that Assumptions~\ref{ass:mq-cylinder} and \ref{ass:ss-cylinder}
allow us to find a near-center $c_0$ of $J$ in constant time.  Our
procedure for testing cylinders is exactly the same as the procedure
for testing disks described in \secref{full-disk}, except that during
each iteration, we perform $ln/2\pi$ sets of probes along the planes
$z=0, z=2\pi/n, z=4\pi/n,\ldots,z=l$, where each set contains $n$
probes directed at $c_0$. Note that the number of probes performed is
$O(ln^2)$. After collecting these sample points, they are projected
onto the $(x,y)$-plane, and the algorithm of \cite{bbbrw97} or
\cite{dgr97}, determine whether there exists an annulus of inner
radius $r$ and outer radius $R$ which contains them.

Like the procedure for disks, the testing algorithm relies on a
lemma regarding the maximum distance of any point on $\bd(J)$ to the
nearest sample point.  Before stating the lemma, we define the
function}
\begin{equation} 
f''(n) = \fppn \enspace . 
\end{equation}

\begin{lem}\lemlabel{closeness3}
Let $J$ be an object of length $l$ with center $c_J$, and satisfying
Assumptions~\ref{ass:mq-cylinder}, \ref{ass:ss-cylinder}, and
\ref{ass:weird-cylinder}.  Let $c_0$ be any point such that
$\dist(c_I,c_0)\le 2\delta$, and let $S$ be a set of $ln/2\pi$, $n\ge
n'_0$, probes directed at $c_0$ as described above.  Then for any
point $p\in\bd(J)$ there exists a point $p'\in S$ such that $\dist(p,
p')\le f''(n)$.
\end{lem}

%%%%%%%%%%%%%%%%%%%%%%%%%%%%%%%%%%%%%%%%%%%%%%%%%%%%%%%%%%%%%%%%%%%%%%%%%%
% Begin version
%%%%%%%%%%%%%%%%%%%%%%%%%%%%%%%%%%%%%%%%%%%%%%%%%%%%%%%%%%%%%%%%%%%%%%%%%%
\begin{proof}[Sketch]
Note that $S$ contains a ring of probes $S'$ such that for all $p'\in
S$, $|z(p')-z(p)|\le \pi/n$.  With reference to \figref{cylinder}, one
can prove that $\dist(p,p'')\le f'(n)$ using the same arguments used
in the proof of \lemref{closeness2}.  Using \assref{weird-cylinder},
once can also prove that $\dist(p'',p')\le \frac{1}{n}\left(
((1+3\delta)\pi/\alpha)^2 + \pi^2 \right)^\frac{1}{2}$.  The stated
bound then follows from the triangle inequality. 
\end{proof}

\begin{figure}
\centeripe{cylinder}
\caption{The proof of \lemref{closeness3}.}
\figlabel{cylinder}
\end{figure}


}{
\begin{proof}
Note that $S$ contains a ring of probes $S'$ such that for all $p'\in
S$,
\begin{equation}
|z(p')-z(p)|\le \pi/n \enspace . \eqlabel{cyl-z}
\end{equation}
In this ring, there is a point $p'$ such that
\[ \angle(\x(p'),\y(p'))c_0(\x(p),\y(p))\le\pi/n \enspace . \]

Let $L$ be the half-line contained in the plane $z=z(p)$ with endpoint
$c_0$ and passing through $p$.  Consider the intersection of $L$ with
$\bd(J)$.  By \assref{ss-cylinder}, this intersection is a point.
Call this point $p''$ (see \figref{cylinder}).
Assumptions~\ref{ass:mq-cylinder} and \ref{ass:ss-cylinder} and the
same argument used in the proof of \lemref{closeness2} can be used
to prove that
\begin{equation}
\dist(p,p'')\le f'(n) \enspace . \eqlabel{p-p''}
\end{equation}

\begin{figure}
\centeripe{cylinder}
\caption{The proof of \lemref{closeness3}.}
\figlabel{cylinder}
\end{figure}

Next, assume wlog that $c_0=O$, $\x(p'')=0$, and $\z(p'')=0$.  Note
that under these assumptions $\x(p)=0$ and therefore
\begin{equation}
|\x(p)-\x(p'')| = 0 \enspace . \eqlabel{cyl-x}
\end{equation}

Let $L_{pp''}$ be the line passing through $p$ and $p''$.  By
\assref{weird-cylinder}, the equation of $L_{pp''}$ is of the form
$y=m|\z(p')-\z(p'')|$, where
\begin{eqnarray}
m & \in & [-\y(p'')/\alpha,\y(p'')/\alpha] \\
  & \subseteq & [(1+3\delta)/\alpha,(1+3\delta)/\alpha] \enspace .
\end{eqnarray}
Therefore, 
\begin{eqnarray}
|\y(p'')-\y(p')| & \le & (1+3\delta)|\z(p')-\z(p'')|/\alpha \\
   & \le & (1+3\delta)\pi/n\alpha \eqlabel{cyl-y} \enspace .
\end{eqnarray}
Using \eqref{cyl-z}, \eqref{cyl-x}, and \eqref{cyl-y} we obtain 
\begin{equation}
\dist(p'',p')
   \le  \frac{1}{n}\left( ((1+3\delta)\pi/\alpha)^2 + \pi^2 
                \right)^\frac{1}{2} \enspace . \eqlabel{p'-p''}
\end{equation}

Using \eqref{p-p''} and \eqref{p'-p''} along with the triangle
inquality yields the required result. 
\end{proof}
}
%%%%%%%%%%%%%%%%%%%%%%%%%%%%%%%%%%%%%%%%%%%%%%%%%%%%%%%%%%%%%%%%%%%%%%%%%%
% End version
%%%%%%%%%%%%%%%%%%%%%%%%%%%%%%%%%%%%%%%%%%%%%%%%%%%%%%%%%%%%%%%%%%%%%%%%%%

With this result, and following the arguments of
\secref{full-disk}, it is not difficult to prove the correctness
and running time of the testing procedure for cylinders.
}
