\documentclass[charterfonts,lotsofwhite]{patmorin}
 
%\usepackage{amsthm}

\newcommand{\centeripe}[1]{\begin{center}\Ipe{#1}\end{center}}
\newcommand{\comment}[1]{}

\newcommand{\centerpsfig}[1]{\centerline{\psfig{#1}}}

\newcommand{\seclabel}[1]{\label{sec:#1}}
\newcommand{\Secref}[1]{Section~\ref{sec:#1}}
\newcommand{\secref}[1]{\mbox{Section~\ref{sec:#1}}}

\newcommand{\alglabel}[1]{\label{alg:#1}}
\newcommand{\Algref}[1]{Algorithm~\ref{alg:#1}}
\newcommand{\algref}[1]{\mbox{Algorithm~\ref{alg:#1}}}

\newcommand{\applabel}[1]{\label{app:#1}}
\newcommand{\Appref}[1]{Appendix~\ref{app:#1}}
\newcommand{\appref}[1]{\mbox{Appendix~\ref{app:#1}}}

\newcommand{\tablabel}[1]{\label{tab:#1}}
\newcommand{\Tabref}[1]{Table~\ref{tab:#1}}
\newcommand{\tabref}[1]{Table~\ref{tab:#1}}

\newcommand{\figlabel}[1]{\label{fig:#1}}
\newcommand{\Figref}[1]{Figure~\ref{fig:#1}}
\newcommand{\figref}[1]{\mbox{Figure~\ref{fig:#1}}}

\newcommand{\eqlabel}[1]{\label{eq:#1}}
\newcommand{\eqref}[1]{(\ref{eq:#1})}

\newtheorem{thm}{Theorem}{\bfseries}{\itshape}
\newcommand{\thmlabel}[1]{\label{thm:#1}}
\newcommand{\thmref}[1]{Theorem~\ref{thm:#1}}

\newtheorem{lem}{Lemma}{\bfseries}{\itshape}
\newcommand{\lemlabel}[1]{\label{lem:#1}}
\newcommand{\lemref}[1]{Lemma~\ref{lem:#1}}

\newtheorem{cor}{Corollary}{\bfseries}{\itshape}
\newcommand{\corlabel}[1]{\label{cor:#1}}
\newcommand{\corref}[1]{Corollary~\ref{cor:#1}}

\newtheorem{obs}{Observation}{\bfseries}{\itshape}
\newcommand{\obslabel}[1]{\label{obs:#1}}
\newcommand{\obsref}[1]{Observation~\ref{obs:#1}}

\newtheorem{assumption}{Assumption}{\bfseries}{\rm}
\newenvironment{ass}{\begin{assumption}\rm}{\end{assumption}}
\newcommand{\asslabel}[1]{\label{ass:#1}}
\newcommand{\assref}[1]{Assumption~\ref{ass:#1}}

\newcommand{\proclabel}[1]{\label{alg:#1}}
\newcommand{\procref}[1]{Procedure~\ref{alg:#1}}

\newtheorem{rem}{Remark}
\newtheorem{op}{Open Problem}

\newcommand{\etal}{\emph{et al}}

\newcommand{\voronoi}{Vorono\u\i}
\newcommand{\ceil}[1]{\left\lceil #1 \right\rceil}
\newcommand{\floor}[1]{\left\lfloor #1 \right\rfloor}



\newcommand{\defequals}{\stackrel{\mathrm{def}}{=}}

\title{\MakeUppercase{Distribution-Sensitive Point-in-Polygon Testing}}
\author{ULB Algorithms \and John Iacono \and Pat Morin}
\date{}

\begin{document}
\maketitle

\section{Introduction}


For a set $X$, itself containing sets, we denote the union of its
elements by $\cup X=\bigcup_{x\in X} x$.

\section{Collections of Intervals}

Before considering point-in-polygon testing we consider a related
one-dimensional problem on a set of intervals.  Our motivation for
studying this problem is that it gives intuition into how to
characterize the complexity of point-in-polygon testing under random
distributions.


Let $I=\{[a_i,b_i]: i=1,\ldots,n\}$ be a set of $n$ disjoint closed
real intervals, with $b_i < a_{i+1}$ for all $1\le i\le n$.  Consider
a probability distribution $D$.  In this section, all references to
probablity, events, and expectation refer to $D$. We are interested in
the expected query time of data structures for testing if a random
number $p$, distributed according to $D$, is contained in $\cup I$.

More precisely, we considemomentumr the following problem:  Given $I$
and $D$, construct a comparison tree $T$ whose leaves are labeled with
either $0$ or $1$ and such that, for any $p$ in the support of $D$,
the leaf corresponding to $p$ is labelled with $1$ if and only if
$p\in \cup I$.  Note that, for every comparison tree $T$, every
$p\in\R$ corresponds to some leaf of $T$ and thus $D$ induces a
distribution over the leaves of $T$.  The \emph{expected search time}
for $T$ is the expected depth of a leaf in this distribution.  We call
such a tree $T$ a \emph{membership tester} for $(I,D)$.

To study optimal comparison trees we need to define a slightly
different set of intervals than $I$.  Let $J$ be the set of intervals
obtained after the following two operations:

\begin{enumerate}

\item Remove from $I$ all intervals $\ell$ such that $\Pr(\ell) = 0$.

\item Repeat the following operation as many times as possible:  If
$\Pr([b_i,a_{i+1}])=0$ then replace the two intervals $(a_i,b_i)$ and
$(a_{i+1},b_{i+1})$ with the single interval $(a_i,b_{i+1})$
momentum
\end{enumerate}

\begin{lem}\lemlabel{intervals}
Let $\overline{J}$ be the set of maximal open intervals in
$\R\setminus\cup J$ and let $S=J\cup \overline{J}$.  Then the
expected search time of any membership tester for $(I,D)$ is at least
\[
    H(I,D) \defequals 
	-\sum_{\ell\in S} \Pr(\ell)\log \Pr(\ell) \enspace .
\]
\end{lem}

\begin{proof}
Consider two distinct intervals $\ell,m\in S$ and suppose $p\in\ell$
and $q\in m$.  Note that the search paths for $p$ and $q$ must finish
in different leaves of $T$ since, otherwise, there is some point
$r\in[p,q]$ that is incorrectly classified by $T$. (Either $r\in \cup
I$ and the leaf for $r$ is labelled with a 0 or $r\not\in\cup I$ and
the leaf for $r$ is labelled with 1.)  For every interval $\ell\in S$,
select the leaf $\ell_T$ of minimum depth such that the search path
for some point $p\in\ell$ finishes at $\ell_T$.  Using the
standard encoding of paths in a binary search tree, this gives a code
for the intervals of $S$ where the length of the codeword for $\ell$
is the depth of $\ell_T$ in $T$. In this way we obtain
\[
   \E[\mathrm{search time}] 
    = \sum_{r\in\mathrm{leaves}(T)} \Pr(r)\mathrm{depth}(r)
   \ge \sum_{\ell\in S} \Pr(\ell)\mathrm{depth}(\ell_T)
   \ge H(I,D)
\]
where the second inequality follows from Shannon's source coding
theorem \cite{X}.
\end{proof}

Though not the focus of the current paper, it is worth noting that the
lower bound in \lemref{intervals} can be matched to within an additive
constant by storing the intervals of $J$ and $\overline{J}$ at the
leaves of a biased binary search tree \cite{kXX,mXX}.  In particular,
for every $I$ and $D$ there exists a membership tester for $(I,D)$
whose expected search time is $H(I,D)+O(1)$.

\section{Point In Convex Polygon Testing}

Let $P$ be a convex $n$-gon whose boundary is denoted by $\partial P$
and consider a probability measure $D$ over $\R^2$.  In this section
we are interested in the problem of testing whether a query point $p$,
drawn according to $D$, is contained in $P$.   

In particular, we are interested in algorithms that can be described
as linear decision trees.  A \emph{linear decision tree} is a decision
tree whose internal nodes are labelled with lines and whose leaves are
labelled with 0 and 1.  When a query point $p$ enters a node, it goes
to the left child, the right child or stops at that node depending on
whether $p$ is above, below, or on, respectively the nodes line.  We
require that, for every $p$ in the support of $D$, the leaf reached in
the search path for $p$ is labelled with a 1 if and only $p\in P$.
The cost of a linear decision tree $T$ is the expected length of the
search path for $p$ when $p$ is drawn according to distribution $D$.

\begin{lem}
Let $C=P_1\setminus P_2$ where $P_1$ and $P_2$ are closed compact
subsets of $\R^2$ with $P_2\subseteq P_1$.  Let $C_1,\ldots,C_k$ be
the connected components of $C\setminus\partial P$.  Then, any linear
decision tree for $(P,D)$ has cost at least
\[
	\frac{1}{2}\times H(P,D,C) 
		= -\frac{1}{2}\times \sum_{i=1}^k \Pr(C_i)\log \Pr(C_i)
		\enspace .
\]
\end{lem}

\begin{proof} 
We first prove a lower bound for  a slightly different type of
computation tree.  A \emph{chord decision tree} for $C$ is defined in
exactly the same manner as a linear decision tree except that the
internal nodes of the tree are labelled with line segments (chords)
whose endpoints are on the bounary of $C$ and whose interiors do not
intersect the interior of $C$.  A chord partitions $C$ into two parts
and each of these corresponds to one of the children of $C$.

As in the proof of \lemref{one-d} we can assign a code to
$C_1,\ldots,C_k$ such that the length of the codeword for $C_i$ is the
length of the shortest path from the root of $T$ to a leaf
corresponding to some point in $C_i$.  Then we again obtain
\[
  \E[\mathrm{search time}] 
	\ge \sum_{r\in\mathrm{leaves}(T)} \Pr(r)\mathrm{depth}(r) 
	\ge -\sum_{i=1}^k \Pr(C_i)\log\Pr(C_i)
\]

\end{proof}

\section{Conclusion}

\end{document}
