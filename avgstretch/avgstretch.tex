\documentclass{patmorin}
\usepackage{amsthm,amsmath,graphicx,stmaryrd}
\usepackage{pat}

\DeclareMathOperator{\asf}{asf}

\title{\MakeUppercase{Good Average Stretch Geometric Spanners}}
\author{Vida Dujmovi\'c, Pat Morin and Michiel Smid}


\begin{document}
\begin{titlepage}
\maketitle

\begin{abstract}
  The abstract goes here.
\end{abstract}

\end{titlepage}

\section{Introduction}

The \emph{average stretch factor} of a geometric graph, $G$, with vertex
set $V(G)\subset \R^d$ and edge set $E(G)$ is
\[
    \asf(G) = \binom{|V(G)|}{2}^{-1}\sum_{\{u,w\}\in\binom{V(G)}{2}}\frac{\|uw\|_G}{\|uw\|}
\]
\ldots

\section{The Construction}

We make use of the following construct:  A \emph{$k$-partition} of a
set $V$ of $n$ points consists of a set, $D$,
of disks and an assignment $f:V\to D$ such that
\begin{enumerate}
  \item $|D|\in O(n/k)$,
  \item for each $u\in V$, $u\in f(u)$ (i.e., $u$ is a assigned to a
    disk that contains $u$),
  \item for each $\Delta\in D$, $|\{u\in V: f(u)=\Delta\}|\le k$ (i.e.,
   at most $k$ points are assigned to each disk),
  \item for every $r> 0$, $c\ge 2$, and $p\in\R^2$, the number of disks
   whose radius is in the range $[r,2r)$ and that contain $p$ is $O(1)$.
\end{enumerate}
For any set $V$ of $n$ points, a $k$-partition exists and can be found in
$O(n\log n)$ time. [Prove this---I think you get it from Michiel's trick
of separating the fair split tree into components of maximum size $k$.]
Note that, aside of Property~4, there is very little structure to
the disks in $D$. In particular, disks in $D$ may overlap and may
even contain each other.

\begin{thm}
  For every set, $V$, of $n<\infty$ points in $\R^2$, there exists a
  geometric graph $G$ with $V(G)=V$, $|E(G)|\in O(n)$ and $\asf(G)=1+o(1)$.
\end{thm}

\begin{proof}
  Our construction uses values $c,k\in\omega_n(1)$ and $\epsilon\in
  o_n(1)$ that I haven't optimized yet.

  We begin with a $k$-partition $(\{\Delta_1,\ldots,\Delta_{n'}\},f)$
  of $V$, and we use the convention that $\Delta_1,\ldots,\Delta_{n'}$
  are ordered by decreasing size.  For each $i\in \{1,\ldots,n'\}$,
  let $V_i=\{u\in V : f(u)=\Delta_i\}$; that is, $V_i$ is the set of
  points assigned to disk $\Delta_i$.  For each set $V_i$, we choose a
  representative vertex, $u_i$ and add edges joining $u_i$ to each other
  vertex in $V_i$ (see \figref{overview}). Let $N=\{u_1,\ldots,u_{n'}\}$
  denote the set of representative vertices and recall that $|N|=n'\in
  O(n/k)$.
  
  \begin{figure}
    \begin{center} 
      \includegraphics{overview}
    \end{center} 
    \caption{Using a $k$-partition, $G$ contains $O(n/k)$ stars whose
      centers are interconnected by a $(1+1/k)$-spanner}
    \figlabel{overview}
  \end{figure}

  Next, we add two spanner constructions to $G$.  The first is a
  $(1+1/k)$-spanner of $N$ and the second is a $2$-spanner of $V$.
  Thus, for any $u,w\in V$ we have
  \[
     \frac{\|uw\|_G}{\|uw\|} \le \begin{cases}
           1+1/k & \text{if $u,w\in N$} \\
           2 & \text{in any case.}
         \end{cases}
  \]
  
  Let $r_i$ denote the radius of $\Delta_i$ and let $D_i$ be a disk
  centered at the center of $\Delta_i$ and having radius $cr_i$.
  Points of $V$ that are in $D_i\setminus \Delta_i$ can be problematic
  for $V_i$; there is no guarantee that such points have a $1+o(1)$
  spanning path to the points in $V_i$.

  The final step in our spanner construction is to find, for each
  $i\in\{1,\ldots,n'\}$,  a disk, $E_i$, of radius $r_i/c$, with center
  in $D_i$, and that contains the maximum number of points of $V$.
  (Note that this may include points of $V$ in $V_i$ or outside of $D_i$.)
  We then add edges joining each of the points in $V_i$ to some point
  in $E_i$.  See \figref{hitter}.

  \begin{figure}
    \begin{center}
      \includegraphics{hitter}
    \end{center}
    \caption{The disk $E_i$ captures as many points of $V$ as possible
     while still having center in $D_i$.}
    \figlabel{hitter}
  \end{figure}

  As we will see, the only problematic points that remain for $V_i$
  are those contained in $D_i'=D_i\setminus E_i$.  The average stretch
  factor of $G$ can be expressed as
  \begin{align*}
    \asf(G) 
      & = 
      \binom n2^{-1}\left(
        \sum_{i=1}^{n'}\sum_{u,w\in V_i}\frac{\|uw\|_G}{\|uw\|}  
         + \sum_{i=1}^{n'-1}\sum_{j={i+1}}^{n'}
            \sum_{u\in V_i}\sum_{w\in V_j}\frac{\|uw\|_G}{\|uw\|}
      \right)  \\
      & \le 
         \binom n2^{-1}\left(n'k^2 
          + \sum_{i=1}^{n'-1}\sum_{j={i+1}}^{n'}
           \sum_{u\in V_i}\sum_{w\in V_j}\frac{\|uw\|_G}{\|uw\|}
      \right)  \\
      & = 
         O(k/n) + \binom n2^{-1}\left(  
          \sum_{i=1}^{n'-1}\sum_{j={i+1}}^{n'}
           \sum_{u\in V_i}\sum_{w\in V_j}\frac{\|uw\|_G}{\|uw\|}
      \right)  \\
   \end{align*}
   Next, we fix a particular value of $i$ and focus on the sum
   \begin{equation}
       \sum_{j={i+1}}^{n'}
          \sum_{u\in V_i}\sum_{w\in V_j}\frac{\|uw\|_G}{\|uw\|}
          \eqlabel{blah}
   \end{equation}
   that appears above.  Partition the index set $\{i+1,\ldots,n'\}$
   into two sets $I_i$ and $I_i'$, where $I_i$ contains exactly the
   indices $j$ such that $\Delta_j$ intersects $D_i$, so that $I_i$
   indexes the subsets of $V$ that are problematic for $V_i$ (see \figref{indexer}). 
   Then we have

   \begin{figure}
    \begin{center}
      \includegraphics{indexer}
    \end{center}
    \caption{$I_i$ indexes the (red) sets that are problematic for $V_i$
     while $I_i'$ indexes the (green) sets for which there is a $1+o(1)$
     spanning path to the points in $V_i$.}
    \figlabel{hitter}
  \end{figure}


   
   \begin{align*}
     \eqref{blah} 
      & = \sum_{j\in I_i}
           \sum_{u\in V_i}\sum_{w\in V_j}\frac{\|uw\|_G}{\|uw\|}
         + \sum_{j\in I_i'}
           \sum_{u\in V_i}\sum_{w\in V_j}\frac{\|uw\|_G}{\|uw\|} \\
      & = \sum_{j\in I_i}
          \sum_{u\in V_i}\sum_{w\in V_j}\frac{\|uw\|_G}{\|uw\|} 
         + \sum_{j\in I_i'} |V_i||V_j|(1+1/c)(1+1/k) \enspace .
   \end{align*}
   Putting these pieces back together, we have
   \begin{align*}
     \asf(G) & \le O(k/n) + \binom{n}{2}^{-1}\left(
       \sum_{i=1}^{n'}\left(\sum_{j\in I_i}
          \sum_{u\in V_i}\sum_{w\in V_j}\frac{\|uw\|_G}{\|uw\|} 
         + \sum_{j\in I_i'} |V_i||V_j|(1+1/c)(1+1/k)\right)\right) \\
      & = O(k/n) +\binom{n}{2}^{-1}
       \sum_{i=1}^{n'}\left(\sum_{j\in I_i}
          \sum_{u\in V_i}\left(
             \sum_{w\in V_j\cap D_i'}
               \frac{\|uw\|_G}{\|uw\|}
             +\sum_{w\in V_j\cap D_i\setminus D_i'}
               \frac{\|uw\|_G}{\|uw\|}
           \right)
          + \sum_{j\in I_i'} |V_i||V_j|(1+1/c)(1+1/k) \right) \\
      & = O(k/n) +\binom{n}{2}^{-1}
       \sum_{i=1}^{n'}\left(\sum_{j\in I_i}
          \sum_{u\in V_i}\left(
             \sum_{w\in V_j\cap D_i'}
               \frac{\|uw\|_G}{\|uw\|}
             +\sum_{w\in V_j\cap D_i\setminus D_i'}
               (1+1/c)
           \right)
          + \sum_{j\in I_i'} |V_i||V_j|(1+1/c)(1+1/k) \right) \\
     & \le O(k/n) + (1+1/c)(1+1/k) +\binom{n}{2}^{-1}
       \sum_{i=1}^{n'}\sum_{j\in I_i}
          \sum_{u\in V_i}
             \sum_{w\in V_j\cap D_i'}
               \frac{\|uw\|_G}{\|uw\|} \\
     & \le O(k/n) + (1+1/c)(1+1/k) +\binom{n}{2}^{-1}
       \sum_{i=1}^{n'}\sum_{j\in I_i} 2 |V_i||V_j\cap D_i'| \\
     & \le O(k/n) + (1+1/c)(1+1/k) +\binom{n}{2}^{-1}
       2k\sum_{i=1}^{n'}\sum_{j\in I_i} |V_j\cap D_i'| \\
     \enspace .
   \end{align*}
   Thus, all that remains is to show that
   $\sum_{i=1}^{n'}\sum_{j\in I_i} |V_j\cap D_i'| \in o(n^2/k)$.
   Suppose, for the same of contradiction that this is not the case
   and that
   \begin{equation}
      \epsilon n^2/k \le  \sum_{i=1}^{n'}\sum_{j\in I_i} |V_j\cap D_i'| 
         \le \sum_{i=1}^{n'} |V\cap D_i'| \enspace . \eqlabel{kicker}
   \end{equation}
   The right hand side of \eqref{kicker} has $n'\le \alpha n/k$
   terms, for some constant $\alpha >0$ and each of these terms
   is at most $n$.  Therefore, there must exist at least
   $(\epsilon/\alpha)n/k$ values of $i$ such that $|V\cap D_i'|\ge
   (\epsilon/\alpha)n$.  In particular, there exists a point $u\in V$
   and indices $i_1,\ldots,i_\ell$ such that:
   \begin{enumerate}
     \item $u\in V\cap D_{i_j}'$ for all $j\in\{1,\ldots,\ell\}$,
     \item $|V\cap D_{i_j}'|\ge \delta n$ for some constant $\delta>0$
       and all $j\in\{1,\ldots,\ell\}$, and
     \item $\ell\in\Omega(n/k)$.
   \end{enumerate}

   Suppose, without loss of generality, that the smallest disk
   in $D_{i_1},\ldots,D_{i_\ell}$ has unit radius.  Partition
   $i_1\ldots,i_\ell$ into groups $G_0,G_1,\ldots$ such that $G_t$
   contains all indices $i_j$ such that $D_{i_j}$ has radius in the
   interval $[2^t,2^{t+1})$.
   We claim that each such group, $G_t$, has size $O(c^2)$.  To see
   why this is so, observe that, for each group $G_t$, there exists a
   point $u\in V$ that is contained in $\Omega(|G_t|)$ regions $D_{i}'$
   where $i\in G_t$.  $D_{i}'$ has radius at most $c2^{t+1}$.  This means
   that the set of disks
   \[
      \{ \Delta_i : i\in G_t\}
   \]
   is contained in a disk, centered at $u$, of radius at most
   $(c+1)2^{t+1}$.  But then, Property~4 of $k$-partitions ensures that
   the size of $G_t$ is $O(c^2)$.

   Thus far, we have that each group $G_t$ has size $O(c^2)$
   and the total size of all groups is $\Omega(n/k)$.
   Therefore, there must be at least $\Omega(n/(kc^2))$ groups.
   In particular, we can find $h\in\Omega(n/(kc^2\log c))$ groups,
   $G_{t_1},\ldots,G_{t_h}$, such that $t_{i+1} \ge
   t_{i}+2\log c+2$.  By selecting a representative element from each
   of these groups, we obtain a sequence of indices $i_1,\ldots,i_h$
   such that the radius of $\Delta_{i_{j+1}}$ is at least $4c^2$ times the
   radius of $\Delta_{i_j}$ for each $j\in\{1,\ldots,h-1\}$.

   By choice, $D_{i_1}'$ contains at least $\delta n$ elements of $V$
   including the point $u$.  Also by choice, $D_{i_2}'$ contains at least
   $\delta n$ elements of $V$, including $u$.  We claim that $E_{i_2}$
   contains at least $\delta n$ elements of $V$ as well since there
   exists a disk, $E_{i_2}'$, centered at $u$, of radius $r_{i_2}/c >
   4cr_{i_1}$, that contains $D_{i_1}$ and therefore contains all the
   (at least $\delta n$) points in $D_{i_1}'$ (see \figref{containment}).
   The disk $E_{i_2}$ is chosen to contain as many elements of $V$ as
   possible, so it contains at least as many elements as $E_{i_2}'$.
   Therefore, $D_{i_2}'\cup E_{i_2}$ contains at least $2\delta n$
   points of $V$.

   \begin{figure}
     \begin{center}
       \includegraphics{containment}
     \end{center}
     \caption{$D_{i_2}\cup E_{i_2}$ contains at least $2\delta n$ 
              points of $V$.}
     \figlabel{containment}
   \end{figure}

   We can now argue similarly to show that $E_{i_3}$ contains at least
   $2\delta n$ points of $V$ so $D_{i_3}'\cup E_{i_3}$ contains at least
   $3\delta n$ points of $V$.  In general, this argument shows that
   $D_{i_h}\cup E_{i_h}$ contains at least $h\delta n$ points of $V$.
   But this yields a contradiction for $h> 1/\delta$, since $V$
   contains only $n$ points.
\end{proof}



\section*{Acknowledgement}

The authors of this paper are partly funded by NSERC and CFI.

\section*{Authors}

\paragraph{Vida Dujmovi\'c.}
School of Mathematics and Statistics and Department of Systems and Computer Engineering, Carleton University
%, \texttt{vida@cs.mcgill.ca}

\paragraph{Pat Morin and Michiel Smid.}
School of Computer Scence, Carleton University
%, \texttt{\{morin,smid\}@scs.carleton.ca}


\bibliographystyle{plain}
\bibliography{template}





\end{document}


