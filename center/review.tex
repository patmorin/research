\documentclass{article}
\usepackage{fullpage}
\usepackage{amsfonts}

\newcommand{\area}{\mathrm{area}}
\setlength{\parskip}{.3cm}
\setlength{\parindent}{1cm}

\title{Comments on ``Computing the Center of Area of a Convex Polygon'' by
Bra\ss\ and Heinrich-Litan}
\date{}
\author{}

\begin{document}
\maketitle

The paper gives an algorithm for computing the ``center of area''
(i.e., Tukey center) of a convex polygon.  The algorithm runs in
$O(n^2\log^c n)$ time, improving on the previous algorithm of Diaz and
O'Rourke, which ran in $O(n^6\log^c n)$ time.  The paper is reasonably
well-written and the algorithm appears to be correct.  The main source
of the speed up is an application of prune-and-search.

However, I believe that there is a linear time algorithm based on
Chan's randomized optimization technique \cite{cXX}.  Chan's technique
requires two ingredients: (1)~it must be possible to partition the
problem into $r$ subproblems, each of size at most $\alpha n$,
$\alpha< 1$ such that the optimal solution is the maximum of the
solutions to these subproblems and (2)~it must be possible to solve
the decision problem ``is $f(p^*)>\delta$?'' in $f(n)$ time. Here, $r$
and $\alpha$ are constants independent of $n$.  If both these
conditions are satisfied, then the question ``what is the value of
$f(p^*)$?''  (and where is $p^*$?) can be solved in $O(f(n))$ time.

Consider the following generalization of the problem: Let $P$ be a
convex polygon whose edges are assigned non-negative weights.  Assume
that the area of $P$ plus the sum of the weights is 1.  For a
halfspace $h$, define $\area(h\cap P)$ as the area of of $h\cap P$
plus the weights of all edges contained in $h$.  In the generalized
version of this problem we want to maximize (over all
$p\in\mathbb{R}^2)$ the cut-off weighted area function
\[
	f(p)=\min\{\area(h\cap P)\mid \mbox{$h$ is admissible for $p$}\}
\]
where $h$ is admissible for $p$ if it is bounded by a line through $p$
and this line does not intersect any edge with strictly positive weight.

Observe that this is equivalent to the original center of area problem
if all edges of $P$ have weight 0.

Suppose we are given an instance of the above problem.  In order to apply
Chan's technique, we need to partition the problem into $r$ subproblems
each of size at most $\alpha n$.

\end{document}