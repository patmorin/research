\documentclass{jocg}
\usepackage{amsfonts}

% NOTE - the title is UPPERCASE
\title{%
  \MakeTextUppercase{On the Social Properties of a Conic Section,
       and the Theory \newline of Polemical Mathematics}%
  \thanks{This ``research'' was not supported by any funding organizations.
    This sample file contains text from \emph{The Romance of Mathematics}
    by P. Hampson \cite{h1886}.}
}

% NOTE - author names are comma-separated, and extra space is added after
% footnote markers
\author{%
  Peter~Hampson,%
  \thanks{\affil{Oxford University}, 
          \email{pete@oxford.com.au.gov.org}}\,
  Patrick~Morin,%
  \thanks{\affil{Carleton University},
          \email{\{morin,wolfie\}@scs.carleton.ca}}\,
  Joachim~Gudmundsson,%
  \thanks{\affil{NICTA}, 
          \email{joachim.gudmundsson@nicta.com.au}}\,
  and Wolfgang~Snicklefritz\footnotemark[3]
}

% The following are just some examples of how to use the amsthm
% environments, but will probably be used in most submissions
\usepackage{amsthm}
\theoremstyle{plain}
\newtheorem{theorem}{Theorem}
\theoremstyle{definition}
\newtheorem{question}{Question} 


\begin{document}
\maketitle

\begin{abstract}
This sample file contains text copied verbatim from The Romances
of Mathematics \cite{h1886} about the social properties of a conic,
equations to brain waves, social forces, and the laws of political motion.
This material was chosen for this example as it is a convenient source of
copyright-free text, not for any other reason.
\end{abstract}

\section{Introduction}

Most Learned Professors and Students of this University, From the
interest manifested in my first lecture, I conclude that my method of
investigation has not proved altogether unsatisfactory to you, and I hope
ere long to produce certain investigations which will probably startle
you, and revolutionize the current thought of the age. The application
of mathematics to the study of Social Science and Political Government
has curiously enough escaped the attention of those who ought to be most
conversant with these matters. I shall endeavour to prove in the present
lecture that the relations between individuals and the Government are
similar to those which  mathematical knowledge would lead us to postulate,
and to explain on scientific principles the various convulsions which
sometimes agitate the social and political world.

Indeed, by this method we shall be able to prophesy the future of states
and nations, having given certain functions and peculiarities appertaining
to them, just as easily as we can foretell the exact day and hour of an
eclipse of the moon or sun. In order to do this, we must first determine
the \emph{social properties of a conic section}.

\section{Conic Sections}

For the benefit of the unlearned and ignorant, I will first state that
a cone is a solid figure described by the revolution of a right-angled
triangle about one of the sides containing the right angle, which remains
fixed: 
\[
      \{ p\in\mathbb{R}^3 : \angle poq = \alpha \} \enspace .
\]
The fixed side is called the axis of the cone. Conic sections are
obtained by cutting the cone by planes. It may easily be proved that if
the angle between the cutting plane and the axis be equal to the angle
between the axis and the revolving side of the triangle which generates
the cone, the section described  on the surface of the cone is a parabola;
if the former angle be greater than the latter, the curve will be an
ellipse; and if less, the section will be a hyperbola.

But the simplest conic section is, of course, a circle, which is formed by
a plane at right angles to the axis of the cone; and the simplest circle
is that formed by a plane passing through the apex of the cone. All
this is simple mathematics; and let beginners consult more elementary
treatises than this one to satisfy themselves on these points. But if they
will assume these things to be true, they will know quite enough for our
present purpose. The simplest conic section of all has been proved to be
a \emph{point}. Now, this represents the simplest and original form of
society, a \emph{single family}. ``It is not good for man to be alone''
was the first observation made by the wise Creator upon the rational
creature whom He had introduced into Paradise as its lord. Marriage is
the rudiment of all social life, from which all others spring, out of
which all others  are developed. Around the parents' knees soon cluster
a group of children, and in their relation to each other we discern the
earliest forms of law and discipline---the bonds by which society is held
together. When the children grow up, separate households are formed; and
then the multiplication of families, the congregating of men together
for purposes of security and mutual advantages in division of labour;
and thus is gradually formed a state, which is only the development
of the family---the king representing the parent, and ruling on the
same principle.

\section{Chaos and Duality}

Mathematically speaking, our plane no longer passes through the apex. The
point represented the single family; but keeping the plane horizontal,
we move it along the axis, the sections will become \emph{circles},
which represent mathematically the next simplest form of society,
where the centre is the seat of government, which is connected with
each individual member of the social circle by equal radii. The social
property of a circle is that of a monarchical government  in its purest
and simplest form. The larger the circle becomes (i.e., the further
you move the plane from the apex), the greater the distance between the
individual and the monarch. Therefore, the more independent the monarchy
becomes, and the less influence do individuals possess over the ruling
power. Hence, we may infer that as years roll on, the government will
become more despotic; but the stability of the country diminished, and
probably some individual particle, when sufficiently withdrawn from the
attraction of the central head, will begin to revolve on its own account,
and spontaneously generate a government of its own. We may, therefore,
conclude from mathematical reasoning that an unlimited monarchy, though
advantageous for small states, is not a safe form of government for a
large or populous country, inasmuch as the people do not derive much
benefit from the sovereign; the mutual attraction, which ought to exist
in a flourishing state between the ruler and the ruled, is weakened; and
the isolation of the  monarch tends to make him still more despotic. As
a practical example of the truth of the foregoing statement, I may
mention the present condition of Russia, which shows that the result of
an unlimited monarchy, in a large and unwieldy social circle, is such
as we should have reasonably expected from mathematical investigations.

Invariably, under the circumstances which I have described, the country
will become disorganized; the sovereign will cease to have any power
over the people, and the country will become a chaos, without order,
influence, or power.

When the centre of a conic section moves along the axis of the curve to
infinity, banished by the mutual consent of the individual particles
which compose the curve, or the nation, a figure is formed, called a
\emph{parabola}. This is the curve which the most erratic bodies in the
universe describe in space, as they rush along at a speed inconceivable
to human minds, and are supposed to produce all kinds of mischief and
injury to the  worlds whose courses they wend their way among.

This curve, then, represents the position which the nation assumes
when the constituted monarchy, the centre of the system, has been
\emph{banished to infinity}. A revolution has occurred; the monarch has
been dethroned; and it is not hard to see that the same erratic course
which the comet pursues in its flight, is observable with respect to the
social system which is represented by a parabola. We observe with eager
scrutiny the wanderings of these erratic comets. They appear suddenly
with their vapoury tails; sometimes they shine upon us with their soft,
silvery light, brilliant as another moon; sometimes they stand afar off
in the distant skies, and deign not to approach our steady-going earth,
which pursues its regular course day by day, and year by year. Then, after
a few days’ coy inspection of our planet from different points of view,
they fly to other remote parts of the universe, and do not condescend
to show themselves again for a  hundred years or so. Such is the erratic
conduct of a heavenly body whose course is regulated by a parabolic curve
\[
     y = a(x-b)^2 + c \enspace .
\]

We may look for similar eccentric behaviour on the part of a community,
nation, or state, whose centre is at infinity, whose constitution has
been violently disturbed, and whose monarchy is situated in the far-off
regions of unlimited space. The erratic course of Republican rule is
proverbial. There is no stability, no regularity. To-day we may observe
its brilliancy, which seems to laugh at and eclipse the sombre shining of
more steady and enduring worlds; but ere to-morrow's moon has risen,
it may have vanished into the regions of eternal night, and we look
for its bright shining light in the councils of the nations, but it has
ceased to shed its rays, and we are disappointed. Sometimes it is asked,
with fear and trembling: ``What would be the effect if our earth were
to come in contact with the tail of a comet? Should we be destroyed
by the collision, and our ponderous world cease to be?'' But we are
assured that no such disastrous results would follow. We have already
passed through the tails of many comets, but we have not discovered any
inconvenient change in our ordinary mode of procedure. It is probable
that the comet's tail is composed of no solid substance.

We may therefore infer by analogy that a Republican State would not
offer any powerful resistance if it were to come into collision with
a nation possessing a more settled form of government. A shower of
meteoric stones, like passing fireworks, might take place; but beyond
that nothing would occur to excite the fear, or arouse the energies of
the more favoured nation. As an example of the weakness of a Republican
State I may mention France. There we see an industrious race of people,
endowed with many natural gifts and graces, a country rich and productive;
and yet, owing to the unsettled nature of its government, all these
natural advantages are neutralized; its course amongst the nations is
erratic in the extreme, a spectacle of feeble  administration; and it
would offer no more resistance to a colliding Power than the empty vacuum
of a comet's tail. This example will demonstrate to you the truth of
our theory with regard to the instability of a social system which is
geometrically represented by a parabolic curve.

\section{Ellipses}

We will now turn from this picture of insecurity and unrest to another
figure which possesses most advantageous social properties. I refer to
the ellipse. An ellipse is a curve formed by the section of a cone by
a plane surface inclined at an angle to the vertical axis of the cone,
greater than the angle between the axis and the generating line.  This has
nothing to do with the following important result \cite{ah77,rsst97}:

\begin{theorem}[Appel and Haken 1977]
  Every planar graph is $4$-colourable.
\end{theorem} 

\begin{proof}[Proof Sketch]
The proof involves a simple but lengthy discharging argument.
\end{proof}

Now, this is a curve which possesses most attractive properties. (See
Figure~\ref{fig:illustration}.) It is the curve which the earth and
other planetary orbs describe around the centre of the solar system,
as if nature intended that we should take this figure as a guide in
choosing the most advantageous social system. It possesses a centre,
$C$, in view of all the particles which compose the curve, and connected
with them by close  ties. It has two foci, $S$ and $S'$, fixed points,
by the aid of which we may trace the curve.

\begin{figure}
  \begin{center}
    \begin{picture}(65,65)
      \put(0,0){\framebox(60,60)}
      \put(30,30){\circle{100}}
    \end{picture}
  \end{center}
  \caption{Ce\c{c}i n'est pas un cercle.}
  \label{fig:illustration}
\end{figure}

In the interpretation of this figure, the centre of the curve represents
the throne of monarchy. There is no tendency here to revolutionize the
State, to banish the ruling power, and institute a Republican form of
government; but inasmuch as we saw the weakness of an absolute monarchy
in large and populous States, as represented by the circle, the wisdom
of an elliptical social system has ordained that there shall be two foci,
or houses of representatives of the people, who shall assist in regulating
the progress of the nation. Here we have a limited monarchy; the throne
is supported by the representatives of the people; and the nearer these
foci of the nation are to the centre (i.e., in mathematical language,
the less the \emph{eccentricity} of the curve), the more perfect the
system becomes—the greater the happiness of the community.

In cases where the eccentricity becomes very great, the beauty of the
curve is  destroyed, and ultimately the ellipse is merged into one
straight line. Most learned Professors, here we have a terrible warning
of the awful result of too much eccentricity. Whether we regard the life
of the nation or of the individual, let all bear in mind this alarming
fact, that eccentricity of thought, habit, or behaviour may result, as in
the case of this unfortunate ellipse, which once presented such fair and
promising proportions to the student's admiring gaze, in the ``sinister
effacement of a man,'' or the gradual absorption of a State into an
uninteresting thing ``which lies evenly between its extreme points.''

The great examples of Bacon, of Milton, of Newton, of Locke, and
of others, happen to be directly opposed to the popular inference
that eccentricity and thoughtlessness of conduct are the necessary
accompaniments of talent, and the sure indications of genius. I am
indebted to Lacon for that reflection. You may point to Byron, or Savage,
or Rousseau, and say, ``Were not these eccentric people  talented?''
``Certainly,'' I answer; ``but would they not have been better and
greater men if they had been less eccentric, if they had restrained
their caprice, and controlled their passions?'' Do not imagine, my young
students of this university, that by being eccentric you will therefore
become great men and women of genius. The world will not give you credit
for being brilliant because you affect the extravagances which sometimes
accompany genius. It is apparent that many aspirers to fame and talent
are eager  to exhibit their eccentricities to the gaze of the world,
in order that they may persuade the multitude that they possess the
genius of which eccentricity is falsely supposed to be the outward sign.

\section{Conclusions}

I may remark in passing that the eccentricity of a parabolic curve is
always \emph{unity}. What does this prove? You will remember that a
Republican State is represented by a parabola. Therefore, however such
a nation may strive to alter its condition, and secure a settled form of
government, its eccentricity will always remain the same. It will always
be erratic, peculiar, unsettled; and this conclusion substantiates our
previous proposition with regard to the condition of a social system
represented by a parabola.

With regard to other advantages afforded by an elliptical social system,
we will defer the consideration of this important subject until my
next lecture.  For now, we leave you with a question to consider:

\begin{question}
  What is love?
\end{question}

%\bibliographystyle{plain}
%\bibliography{jocg-sample}

\begin{thebibliography}{1}

\bibitem{ah77}
K.~Appel and W.~Haken.
\newblock Every planar map is four colorable.
\newblock {\em Illinois Journal of Mathematics}, 21:439--567, 1977.

\bibitem{h1886}
P.~Hampson.
\newblock {\em The Romance of Mathematics}.
\newblock Oxford Press, Bigg City, 1886.

\bibitem{rsst97}
N.~Robertson, D.~P. Sanders, P.~Seymour, and R.~Thomas.
\newblock The four-colour theorem.
\newblock {\em Journal of Combinatorial Theory, Series B}, 70(1):2--44, 1997.

\end{thebibliography}

\end{document}
