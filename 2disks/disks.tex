\documentclass[twoside]{report}
\usepackage{amsfonts}
%\usepackage{amsthm}
\usepackage{graphicx}
\usepackage{ipe}
\usepackage{jda}

\newcommand{\centeripe}[1]{\begin{center}\Ipe{#1}\end{center}}
%\newcommand{\comment}[1]{}

\newcommand{\centerpsfig}[1]{\centerline{\psfig{#1}}}

\newcommand{\etal}{\emph{et al.}}

\newtheorem{lem}{Lemma}
\newtheorem{thm}{Theorem}

\newcommand{\seclabel}[1]{\label{sec:#1}}
\newcommand{\Secref}[1]{Section~\ref{sec:#1}}
\newcommand{\secref}[1]{\mbox{Section~\ref{sec:#1}}}
\newcommand{\figlabel}[1]{\label{fig:#1}}
\newcommand{\Figref}[1]{Figure~\ref{fig:#1}}
\newcommand{\figref}[1]{\mbox{Fig.~\ref{fig:#1}}}
\newcommand{\eqlabel}[1]{\label{eq:#1}}
\newcommand{\eqref}[1]{(\ref{eq:#1})}

\newcommand{\tdisk}{\textsc{2-disk}}

\newcommand{\x}{\mathrm{x}}
\newcommand{\y}{\mathrm{y}}
\newcommand{\ch}{\mathrm{CH}}


\Title{Packing Two Disks into a Polygonal Environment}
%\thanks{This research was supported by the Natural Sciences and
%        Engineering Research Council of Canada and by the Hong Kong
%Research Grant Council  CERG grant HKUST6137/98E.}}
\ShortAuthor{Prosenjit Bose and Pat Morin and Antoine Vigneron}
\LongAuthor{
\author{Prosenjit Bose}
\address{School of Computer Science, Carleton University.
	E-mail:~jit@cs.carleton.ca} 
\author{Pat Morin}
\address{School of Computer Science, Carleton University.
	E-mail:~morin@cs.carleton.ca} 
\author{Antoine Vigneron}
\address{Department of Computer Science, University of Singapore.
	E-mail:~antoine@comp.nus.edu.sg}
}

\Received{April 26, 2002}

\begin{document} 

\begin{paper}

\begin{abstract}
  We consider the following problem.  Given a polygon $P$,
  possibly with holes, and having $n$ vertices, compute a pair of
  equal radius disks that do not intersect each other, are contained
  in $P$, and whose radius is maximized.  Our main result is a simple
  randomized algorithm whose expected running time, on any
  input, is $O(n\log n)$. This is 
  optimal in the algebraic decision tree model of computation.
\end{abstract}

\Keywords{Disk packing, Computational geometry, Origami}

%========================================================================
\section{Introduction}

Let $P$ be a polygon, possibly with holes, and having $n$ vertices.
We consider the following problem, which we call \tdisk: Find a pair
of disks with radius $r^*$ that do not intersect each other, are
contained in $P$, and such that $r^*$ is maximized.  This problem was
introduced by Biedl \etal\ \cite{bddlt98} who use it to determine the
radius $r^*$ of the largest disk such that an irregularly shaped piece
of paper $P$ can be folded once to cover the disk (see
\figref{disk-hiding}).

\begin{figure}
\begin{center}\begin{tabular}{ccc}
\IpeScale{90}\Ipe{folding-a} & 
   \raisebox{1cm}{$\Rightarrow$} &
   \IpeScale{90}\Ipe{folding-b} \\
\end{tabular}\end{center}
\caption{The solution to \tdisk\ gives the size of the largest disk
         that can be hidden by $P$ using one fold.}\figlabel{disk-hiding}
\end{figure}

Biedl \etal\ \cite{bddlt98} give an $O(n^2)$ time algorithm for
\tdisk.  Bespamyatnikh \cite{b02} gives an algorithm for simple
polygons (i.e., without holes) that runs in $O(n\log^2 n)$ time and is
based on parametric search \cite{m83}.  For the important special case
when $P$ is a convex polygon, Bose \etal\ \cite{bckm98} describe a
linear time algorithm and Kim and Shin \cite{ks00} describe an
$O(n\log n)$ time algorithm.

Another special case occurs when the holes of $P$ degenerate to points.  This
is known as the \emph{maximin 2-site facility location} problem
\cite{bks00,kks02}.  In this formulation we can think of the centers of the two
disks as obnoxious facilities such as smokestacks, or nuclear power plants, and
the points as population centers.  The goal is maximize the minimum distance
between a facility and a population center.  Katz \etal\ \cite{kks02} give an
$O(n\log n)$ time algorithm for the decision version of the 2-site facility
location problem in which one is given a distance $d$ and asked if there exists
a placement of 2 non-intersecting disks of radius $d$, each contained in $P$
such that no point is included in either of the disks.

In this paper we present a simple randomized algorithm for the general
case in which $P$ is not necessarily convex and may contain holes.
Our algorithm runs in $O(n\log n)$ expected time.  It can also be used
to solve the optimization version of the 2-site maximin facility
location problem in $O(n\log n)$ expected time. We also observe that, when we
allow polygons with holes, $\Omega(n\log n)$ is a lower bound for
\tdisk\ by a simple reduction from \textsc{max-gap}.  Thus, our
algorithm is optimal.

The remainder of the paper is organized as follows:
\secref{medial-axis} reviews definitions and previous results
regarding the medial-axis. \secref{algorithm} describes our
algorithm. \secref{conclusions} summarizes and concludes with an open
problem.

\section{The Medial-Axis}\seclabel{medial-axis}

For the remainder of this paper, $P$ will be a polygon,
possibly with holes, and having $n$ vertices.  The \emph{medial-axis}
$M(P)$ of $P$ is the locus of all points $p$ for which there exists a
disk centered at $p$, contained in $P$, and which intersects the
boundary of $P$ in two or more points. See \figref{medial-axis} for an
example.  Alternatively, $M(P)$ is a portion of the \emph{Voronoi
diagram} of the open line segments and vertices defined by the 
edges of $P$. To be more precise, 
we need to remove the Voronoi edges that are outside $P$ and those
associated with an edge and one of its endpoints. 
It is well known that the medial-axis consists of
$O(n)$ straight line segments and parabolic arcs.

\begin{figure}
\centeripe{testing}
\caption{The medial-axis of a polygon with a triangular hole.}
\figlabel{medial-axis}
\end{figure}

Algorithmically, the medial-axis is well understood. There exists an
$O(n)$ time algorithm~\cite{csw99} for computing the medial-axis of a
polygon without holes and $O(n\log n)$ time algorithms for computing
the medial-axis of a polygon with holes~\cite{as-vdco-95}.
Furthermore, these algorithms can compute a representation in which
each segment or arc is represented as a segment or arc in
$\mathbb{R}^3$, where the third dimension gives the radius of the disk
that touches two or more points on the boundary of $P$.

We say that a point $p\in P$ \emph{supports} a disk of radius $r$ if
the disk of radius $r$ centered at $p$ is contained in $P$.  We call a
vertex, parabolic arc or line segment $x$ of $M(P)$ an
\emph{elementary object} if the radius of the largest disk supported
by $p\in x$ is non-decreasing as $p$ moves from one endpoint of $x$ to
the other.  The vertices and straight line segments of $M(P)$ are
elementary objects and each parabolic arc of $M(P)$ can be split into
at most two elementary objects.  Thus, $M(P)$ can be split into $O(n)$
elementary objects whose union is $M(P)$.

\section{The Algorithm}\seclabel{algorithm}

In this section we describe a randomized algorithm for \tdisk\ with
$O(n\log n)$ expected running time. We begin by restating \tdisk\ as a
problem of computing the diameter of a set of elementary objects under
a rather unusual distance function. We then use an algorithm based on
the work of Clarkson and Shor \cite{cs88} to solve this problem in the
stated time.

The following lemma, of which similar versions appear in Bose \etal\ 
\cite{bckm98} and Biedl \etal\ \cite{bddlt98}, tells us that we can
restrict our search to disks whose centers lie on $M(P)$.

\begin{lem}\label{l:restrict}
  Let $D_1$ and $D_2$ be a solution to \tdisk\ which maximizes the
  distance between $D_1$ and $D_2$ and let $p_1$ and $p_2$ be the
  centers of $D_1$ and $D_2$, respectively.  Then $D_1$ and $D_2$ each
  intersect the boundary of $P$ in at least two points and hence $p_1$
  and $p_2$ are points of $M(P)$.
\end{lem}

\begin{proof}
  Refer to \figref{restrict}. Suppose that one of the disks, say
  $D_1$, intersects the boundary of $P$ in at most one point.  Let
  $o_1$ be this point, or if $D_1$ does not intersect the boundary of
  $P$ at all then let $o_1$ be any point on the boundary of $D_1$.
  Note that there is some value of $\epsilon>0$ such that $D_1$ is
  free to move by a distance of $\epsilon$ in either of the two
  directions perpendicular to the direction $\overrightarrow{p_1o_1}$
  while keeping $D_1$ in the interior of $P$. However, movement in at
  least one of these directions will increase the distance $|p_1p_2|$,
  which is a contradiction since this distance was chosen to be
  maximal over all possible solutions to \tdisk.
  \begin{figure}
    \centeripe{movement}
    \caption{The proof of Lemma~\ref{l:restrict}.}
    \figlabel{restrict}
  \end{figure}
\end{proof}

Let $x_1$ and $x_2$ be two elementary objects of $M(P)$.  We define
the \emph{distance} between $x_1$ and $x_2$, denoted $d(x_1,x_2)$ as
$2r$, where $r$ is the radius of the largest pair of equal-radius
non-intersecting disks $D_1$ and $D_2$, contained in $P$ and with
$D_i$ centered on $x_i$, for $i=1,2$. There are two points to note
about this definition of distance: (1)~if the distance between two
elementary objects is $2r$, then we can place two non-intersecting
disks of radius $r$ in $P$, and (2)~the distance from an elementary
object to itself is not necessarily 0.  Given two elementary objects
it is possible, in constant time, to compute the distance between them
as well as the locations of 2 disks that produce this distance
\cite{bddlt98}.

Let $E$ be the set of elementary objects obtained by taking the union
of the following three sets of elementary objects:

\begin{enumerate}
\item the set of vertices of $M(P)$,
\item the set of elementary line segments of $M(P)$ and
\item the set of elementary parabolic arcs obtained by splitting
  each parabolic arc of $M(P)$ into at most two elementary objects.
\end{enumerate}

We call the \emph{diameter} of $E$ the maximum distance between any pair
$x,y\in E$, where distance is defined as above.  By Lemma~\ref{l:restrict},
\tdisk\ can be solved by finding a pair of elementary objects in $E$ whose
distance is equal to the diameter of $E$.\footnote{Here we use the term
``pair'' loosely, since the diameter may be defined by the distance between an
elementary object and itself.}

Thus, all that remains is to devise an algorithm for finding the diameter of
$E$. Let $m$ denote the cardinality of $E$ and note that, initially, $m=O(n)$.
Motivated by Clarkson and Shor \cite{cs88}, we compute the diameter using the
following algorithm. We begin by selecting a random elementary object $x$ from
$E$ and finding the elementary object $x'\in E$ whose distance from $x$ is
maximal, along with the corresponding radius $r$.  This can be done in $O(m)$
time, since each distance computation between two elementary objects can be
done in constant time.  Note that $r$ is a lower bound on $r^*$.  We use this
lower bound to do \emph{trimming} and \emph{pruning} on the objects of $E$.

We trim each object $y\in E$ by partitioning $y$ into two
subarcs,\footnote{We use the term subarc to mean both parts of
segments and parts of parabolic arcs.} each of which may be empty.
The subarc $y_\ge$ is the part of $y$ supporting disks of radius
greater than or equal to $r$.  The subarc $y_<$ is the remainder of
$y$.  We then \emph{trim} $y_<$ from $y$ by removing $y$ from $E$ and
replacing it with $y_\ge$.  During the trimming step we also remove
from $E$ any object that does not support a disk of radius greater
than $r$ (in which case $y_\ge$ is empty).  Each such trimming
operation can be done in constant time, resulting in an $O(m)$ running
time for this step.

Next, we prune $E$.  For any arc $y \in E$, the \emph{lowest point} of $y$ is
its closest point to the boundary of $P$. In the case of ties, we take a point
which is closest to one of the endpoints of $y$.  By the definition of
elementary objects, the lowest point of $y$ is therefore an endpoint of $y$.
The closed disk with radius $r$ centered on the lowest point of $y$ is denoted
by $D(y)$.  To prune, we discard all the objects $y \in E$ such that $D(y)
\cap D(y') \neq \emptyset$ for all $y' \in E$.

Pruning can be performed in $O(m\log m)$ time by computing, for each
lowest endpoint $p$, a matching lowest endpoint $q$ whose distance
from $p$ is maximal and then discarding $p$ if $\|pq\|\le 2r$. This
computation is known as \emph{all-pairs furthest neighbors} and can
be completed in $O(m\log m)$ time \cite{ams91}.

Once all trimming and pruning is done, we have a new set of elementary
objects $E'$ on which we recurse.  The recursion completes when $|E'|
\leq 2$, at which point we compute the diameter of $E'$ in constant
time using a brute-force algorithm. We output the largest pair of
equal-radius non-overlapping disks found during any iteration of the
algorithm.

To prove that this algorithm is correct we consider a pair of
non-intersecting disks $D_1$ and $D_2$, each contained in $P$ and
having radius $r^*$, centered at $p_1$ and $p_2$, respectively, such
that the Euclidean distance $\|p_1p_2\|$ is maximal. The following lemma
shows that $p_1$ and $p_2$ are not discarded from consideration until
an equally good solution is found.

\begin{lem}
  \label{l:correct} If, during the execution of one round,
  $\{p_1,p_2\} \subset \cup E$ and $r<r^*$, then $\{p_1,p_2\} \subset
  \cup E'$ at the end of the round.
\end{lem}

\begin{proof}
  We need to show that at the end of the round, there exists
  elementary objects $y_1,y_2\in E'$ such that $p_1\in y_1$ and
  $p_2\in y_2$.  More specifically, we need to show there exists
  $y_1,y_2\in E$ such that $p_1$, respectively $p_2$ is not trimmed
  from $y_1$, respectively $y_2$, and $y_1$ and $y_2$ are not pruned.
  
  To see that $p_1$ and $p_2$ are not trimmed from any elementary
  object that contains them we simply note that $p_1$ and $p_2$ both
  support disks of radius $r^*> r$ and are therefore not trimmed.
  
  To prove that the elementary objects $y_1$ and $y_2$ containing $p_1$ and
  $p_2$ are not pruned we subdivide the plane into two open halfspaces $H_1$
  and $H_2$ such that all points in $H_1$ are closer to $p_1$ than to $p_2$ and
  vice-versa. We denote by $L$ the line separating these two halfspaces.
  
  Recall that, after trimming, an elementary object $x$ is only pruned
  if $D(x)\cap D(y)\neq\emptyset$ for all $y\in E$.  We will show that
  $D(y_1)\subseteq H_1$ and $D(y_2)\subseteq H_2$, therefore
  $D(y_1)\cap D(y_2)=\emptyset$ and neither $y_1$ nor $y_2$ are
  pruned.  It suffices to prove that $D(y_1)\subseteq H_1$ since the
  same argument shows that $D(y_2)\subseteq H_2$.  We consider three
  separate cases depending on the location of $p_1$ on $M(P)$.

  \noindent\textbf{Case 1:} 
  $p_1$ is a vertex of $M(P)$. In this case we choose $y_1$ to be the
  singleton elementary object $\{p_1\}$. Thus, $D(y_1)$ is centered
  at $p_1$ and $D(y_1)\subseteq D_1\subseteq H_1$, as required. 
  
  \begin{figure}
    \begin{center}
        \Ipe{proof-segment}   
      \end{center}
    \caption{The proof of Lemma~\ref{l:correct} Case 2.}
    \figlabel{case2}
  \end{figure}
 
  \noindent\textbf{Case 2:} $p_1$ lies in the interior of a straight
  line segment $y_1$ of $M(P)$.  Let $p'_1$ be the lower endpoint of
  $y_1$. Let $\theta$ be the angle $\angle p_2p_1p'_1$ (see
  \figref{case2}).  If $\theta \in [-\pi/2,\pi/2]$ then we can move
  $p_1$ slightly in the direction opposite to
  $\overrightarrow{p_1p'_1}$ while keeping $D_1$ inside $P$, thus
  contradicting the assumption that $\|p_1p_2\|$ is maximal. Therefore
  $\theta \in [\pi/2,3\pi/2]$, which implies that $D(y_1)$ lies in
  $H_1$.
  
  \noindent\textbf{Case 3:} $p_1$ lies in the interior of a parabolic
  arc $y_1$ of $M(P)$.  In this case $D_1$ is tangent to an edge $e_1$
  of $P$ and touches one of its vertices $v$. Again, let $p'_1$ denote
  the lower endpoint of $y_1$.  Without loss of generality, assume
  that $e_1$ is parallel to the $x$-axis, $\x(p'_1)\le \x(p_1)$ and $v$
  is below $e_1$ (see \figref{case3}).  Note that $y_1$ is part of a
  parabola whose focus is $v$ and that the radius of the largest disk
  supported by any point $p\in y_1$ is given by the distance between $p$
  and $e_1$.

  Our assumption that $\x(p'_1)\le \x(p_1)$ therefore implies that
  $y(p'_1)\ge y(p_1)$ and that $\x(v)\le \x(p_1)$.  Let $L'$ be the line
  parallel to $L$ that intersects the segment $[p_1,p_2]$ and that is
  tangent to $D_1$. We denote by $o_1$ the point where $e_1$ is
  tangent to $D_1$, and we denote by $o'_1$ the point such that
  $(o_1,o_1')$ is a diameter of $D_1$.

  \begin{figure}
    \begin{center} 
        \Ipe{proof-arc}
    \end{center}
    \caption{The proof of Lemma~\ref{l:correct} Case 3.}
    \figlabel{case3}
  \end{figure}

  It must be that $\x(p_2)>\x(p_1)$. Otherwise $p_1$ and $D_1$ could
  be moved a small amount in the positive $x$ direction and $D_1$
  would remain in $P$ (since $y_1$ is part of a parabola whose maximum
  is obtained at $x(v)$).  This would increase the distance
  $\|p_1p_2\|$, contradicting the assumption that this distance is
  maximal.  It follows that $L'$ is tangent to $D_1$ along the
  counterclockwise arc $[o'_1,o_1]$.  Now, since $\x(p_1')<\x(p_1)$
  and $\y(p_1')\ge \y(p_1)$ and the radius of $D(y_1)$ is not more
  than the radius of $D_1$, $D(y_1)$ does not intersect $L'$.
  Furtheremore, $D(y_1)$ is on the same side of $L'$ as $D_1$, so
  $D(y_1)$ is contained in $H_1$, which completes the proof.
\end{proof}

Let $d_i$ denote the distance of the furthest object in $E$ from
$x_i$, and suppose for the sake of analysis that the elements of $E$
are labeled $x_1,\ldots,x_n$ so that $d_i\le d_{i+1}$.  The following
lemma helps to establish the running time of the algorithm.

\begin{lem}\label{l:split}
If we select $x=x_i$ as the random elementary object, then we discard all
$x_j\in E$ such that $j \leq i$ from $E$.
\end{lem}

\begin{proof}
  For any $j\le i$, either $x_j$ does not support a disk of radius
  greater than $d_i$, or every point on $x_j$ that supports a disk of
  radius $d_i$ is of distance less than $2d_i$ from every other point of
  $M(P)$ that supports a disk of radius $d_i$.
  
  In the first case, $x_j$ is removed from $E$ by trimming.  In the
  second case, $D(x_j)\cap D(x_k)\neq\emptyset$ for all $x_k\in E$ and $x_j$
  is removed by pruning.
\end{proof}

Finally, we state and prove our main theorem.

\begin{thm}
The above algorithm 
solves \tdisk\ in $O(n\log n)$
expected time.
\end{thm}

\begin{proof}
  The algorithm is correct because, by Lemma~\ref{l:correct}, it never
  discards $p_1$ nor $p_2$ until it has found a solution with $r=r^*$,
  at which point it has already found an optimal solution that will be
  reported when the algorithm terminates.
  
  To prove the running time of the algorithm, we use the following
  facts.  Each round of the algorithm can be completed in $O(m\log m)$
  time where $m$ is the cardinality of $E$ at the beginning of the
  round.  By Lemma~\ref{l:split}, when we select $x_i$ as our random
  elementary object, all objects $x_j$ with $j \leq i$ disappear from $E$.
  Therefore, the expected running time of the algorithm is given by
  the recurrence
\[
T(m) \le O(m\log m) + \frac{1}{m}\sum_{i=1}^{m}T(m-i) 
        \enspace, 
\] 
which readily solves to $O(m\log m)$.  Since $m\in O(n)$, this
completes the proof.
\end{proof}


\section{Conclusions}\seclabel{conclusions}

We have given a randomized algorithm for \tdisk\ that runs in $O(n\log
n)$ expected time. The algorithm is considerably simpler than the
$O(n\log^3 n)$ algorithm of Bespamyathnikh \cite{b02} and has the
additional advantage of solving the more general problem of polygons
with holes.  Although we have described our algorithm as performing
computations with distances, these can be replaced with squared
distances to yield an algorithm that uses only algebraic computations.

In the algebraic decision tree model of computation, one can also
prove an $\Omega(n\log n)$ lower bound on any algorithm for \tdisk\ 
through a reduction from \textsc{max-gap} \cite{ps85}.  Suppose that
the input to \textsc{max-gap} is $y_1,\ldots,y_n$.  Without loss of
generality one can assume that $y_1=\min\{y_i:1\le i\le n\}$ and
$y_n=\max\{y_i:1\le i\le n\}$.  We then construct a rectangle with top
and bottom sides at $y_1$ and $y_n$, respectively, and with width
$2(y_n-y_1)$.  The interior of this rectangle is then partitioned into
rectangles with horizontal line segments having $y$ coordinates
$y_1,\ldots, y_n$.  See \figref{reduction} for an example.

\begin{figure}
\centeripe{reduction}
\caption{Reducing \textsc{max-gap} to \tdisk.}
\figlabel{reduction}
\end{figure}

It should then be clear that the solution to \tdisk\ for this problem
corresponds to placing two disks in the rectangle corresponding to the
gap between $y_i$ and $y_{i+1}$ which is maximal, i.e., it gives a
solution to the original \textsc{max-gap} problem.  Since this
reduction can be easily accomplished in linear time and
\textsc{max-gap} has an $\Omega(n\log n)$ lower bound, this yields an
$\Omega(n\log n)$ lower bound on \tdisk.

The above reduction only works because we allow polygons with
holes. An interesting open problem is that of determining the
complexity of \tdisk\ when restricted to simple polygons.  Is there a
linear time algorithm?  More generally, is there an $O(n+n\log h)$ time
algorithm for polygons with at most $h$ holes?

\bibliographystyle{plain}
\bibliography{disks}

\end{paper}

\end{document}






