\documentclass{beamer}

\mode<presentation>{
\definecolor{cured}{rgb}{.8,0,.2}
\usecolortheme[named=cured]{structure}
%\usefonttheme{structurebold}
\usetheme{split}
}

\usepackage{amsopn}

\DeclareMathOperator{\conv}{conv}
\DeclareMathOperator{\average}{average}
\DeclareMathOperator{\support}{support}
\DeclareMathOperator{\boundary}{\partial}

\input{pat-slides}

\newcommand{\pin}{p_{\mathsf{in}}}
\newcommand{\pout}{p_{\mathsf{out}}}
\DeclareMathOperator{\cost}{cost}
\DeclareMathOperator{\depth}{depth}

\title{On the Expected Maximum Degree of Gabriel and Yao Graphs}
\author{Luc Devroye 
	\and Joachim Gudmundsson
	\and Pat Morin}
\date{May 7, 2009 \\ NICTA Algorithms Seminar}

\begin{document}

\frame{\titlepage}

%\section[Outline]{}
%\frame{\tableofcontents}

\section{Wireless Ad-Hoc Networks}
\frame
{
  \frametitle{Wireless \emph{ad hoc} Networks}
  \begin{itemize}
  \item Networks that operate without any central infrastructure, 
	planning, or organization
  \item Nodes are a set of points in $\R^2$
  \item Note $u$ and $v$ can communicate iff $\|uv\|\le r$  
	($r$ is the transmission range)
}


\end{document}

