\documentclass{article}


\title{Work with Jurek on Immobilizing Polygons}
\author{Jurek Czyzowicz \and Pat Morin}

\begin{document}
\maketitle

Here are some things Jurek and I proved about immobilizing polygons.

Stronly immobilizing means Reuleaux's definition, weakly immobilizing
means Czyzowicz' definition.

\begin{enumerate}

\item Strongly immobilizing a polygon with 2 pins is as hard as
Hopcroft's problem.

\item There is an $O(n^{3/2}\log^c n)$ (maybe $O(n^{4/3}\log^c n)$
time algorithm to immobilize a polygon using 2 pins (if possible).

Van der Stappen et al. beat us to this result (with a running time of
$O(n^{4/3}\log^{1/3} n)$).

\item Strongly immobilizing a polygon with 3 pins in 3-sum hard.

\item Strongly immobilizing a polygon with 4 pins can be done in
linear time (this is well known).

\item Testing whether a set of $n$ pins immobilizes a polygon can be
done in $O(n)$ time.  This is basically linear programming in the
3-dimensional configuration space where the halfplanes we use are
those tangent to the helicoids defined by the pins.

One case still needs careful consideration here: When the intersection
of the halfspaces is a line.  

\item From any set of $n$ pins that immobilize a polygon, we can
choose a subset of size 5 that continue to immobilize the polygon.

\item There exists a set of 5 pins that strongly immobilize a polygon
such that any subset of 4 of them do not weakly immobilize it. (a
3-slider plus 2 pins very close together on the boundary of an ellipse
traced by a point close to them).

\item Any polygonal hole can be immobilized with two pins.

\item Any shape with holes can be immobilized with three pins unless
it is a ring.

\end{enumerate}

\section{Paper Outline}

\begin{enumerate}
\item Observations: 

If there is a motion then there is a motion that keeps two pins on the boundary.
\item 
\end{enumerate}

\end{document}

