\documentclass[lotsofwhite]{patmorin}
\usepackage{amsthm}

\newcommand{\CH}{\mathit{CH}}
\newcommand{\Z}{\mathit{Z}}
\newcommand{\DZ}{\mathit{DZ}}
\newcommand{\depth}{\mathrm{depth}}
\newcommand{\argmax}{\mathrm{argmax}}
\newcommand{\drop}{\!\!\downarrow\!\!}
\newcommand{\dual}{\varphi}

 
%\usepackage{amsthm}

\newcommand{\centeripe}[1]{\begin{center}\Ipe{#1}\end{center}}
\newcommand{\comment}[1]{}

\newcommand{\centerpsfig}[1]{\centerline{\psfig{#1}}}

\newcommand{\seclabel}[1]{\label{sec:#1}}
\newcommand{\Secref}[1]{Section~\ref{sec:#1}}
\newcommand{\secref}[1]{\mbox{Section~\ref{sec:#1}}}

\newcommand{\alglabel}[1]{\label{alg:#1}}
\newcommand{\Algref}[1]{Algorithm~\ref{alg:#1}}
\newcommand{\algref}[1]{\mbox{Algorithm~\ref{alg:#1}}}

\newcommand{\applabel}[1]{\label{app:#1}}
\newcommand{\Appref}[1]{Appendix~\ref{app:#1}}
\newcommand{\appref}[1]{\mbox{Appendix~\ref{app:#1}}}

\newcommand{\tablabel}[1]{\label{tab:#1}}
\newcommand{\Tabref}[1]{Table~\ref{tab:#1}}
\newcommand{\tabref}[1]{Table~\ref{tab:#1}}

\newcommand{\figlabel}[1]{\label{fig:#1}}
\newcommand{\Figref}[1]{Figure~\ref{fig:#1}}
\newcommand{\figref}[1]{\mbox{Figure~\ref{fig:#1}}}

\newcommand{\eqlabel}[1]{\label{eq:#1}}
\newcommand{\eqref}[1]{(\ref{eq:#1})}

\newtheorem{thm}{Theorem}{\bfseries}{\itshape}
\newcommand{\thmlabel}[1]{\label{thm:#1}}
\newcommand{\thmref}[1]{Theorem~\ref{thm:#1}}

\newtheorem{lem}{Lemma}{\bfseries}{\itshape}
\newcommand{\lemlabel}[1]{\label{lem:#1}}
\newcommand{\lemref}[1]{Lemma~\ref{lem:#1}}

\newtheorem{cor}{Corollary}{\bfseries}{\itshape}
\newcommand{\corlabel}[1]{\label{cor:#1}}
\newcommand{\corref}[1]{Corollary~\ref{cor:#1}}

\newtheorem{obs}{Observation}{\bfseries}{\itshape}
\newcommand{\obslabel}[1]{\label{obs:#1}}
\newcommand{\obsref}[1]{Observation~\ref{obs:#1}}

\newtheorem{assumption}{Assumption}{\bfseries}{\rm}
\newenvironment{ass}{\begin{assumption}\rm}{\end{assumption}}
\newcommand{\asslabel}[1]{\label{ass:#1}}
\newcommand{\assref}[1]{Assumption~\ref{ass:#1}}

\newcommand{\proclabel}[1]{\label{alg:#1}}
\newcommand{\procref}[1]{Procedure~\ref{alg:#1}}

\newtheorem{rem}{Remark}
\newtheorem{op}{Open Problem}

\newcommand{\etal}{\emph{et al}}

\newcommand{\voronoi}{Vorono\u\i}
\newcommand{\ceil}[1]{\left\lceil #1 \right\rceil}
\newcommand{\floor}[1]{\left\lfloor #1 \right\rfloor}



%\title{An Optimal Algorithm for $d$-Variate Zonoid Depth}
\title{\MakeUppercase{An Optimal Algorithm
	for} $d$\MakeUppercase{-Variate Zonoid Depth}}
\author{Pat Morin}
\date{}

\begin{document}
\maketitle
\begin{abstract}
Algorithms for Zonoid depth in $d$ dimensions are considered.
\end{abstract}

\section{Introduction}

Let $S$ be a set of $n$ points in $\mathbb{R}^d$.
For a real number $k\ge 1$, the \emph{$k$-zonoid} of $S$ is defined as 
\[
      \Z_k(S) = \left\{\sum_{p\in S}p\lambda_p 
	: \mbox{$0\le \lambda_p \le 1/k$ for all $p\in S$  
	   and $\sum_{p\in S}\lambda_p = 1$}  \right\} \enspace .
\] 
Notice that, for $k=1$ the $1$-zonoid of $S$ is the convex hull of
$S$,  i.e., $\Z_1(S)=\CH(S)$.
As $k$ increases, $\Z_k(S)$ becomes smaller and smaller until
the limiting case $k=n$, for which $\Z_n(S)$ consists of a single point,
the mean of $S$.  The \emph{zonoid depth} of a point
$p\in\CH(S)$ with respect to $S$ is defined as
\[
     \depth(p,S) = \sup\{k : p\in Z_k(S) \} \enspace ,
\]
and is a real number in the interval $[1,n]$.

Gopala and Morin \cite{gm06} consider algorithms for bivariate ($d=2$)
zonoid depth and give an $O(n)$ expected time algorithm for computing,
$\depth(p,S)$ when $p$ and $S$ are in $\mathbb{R}^2$.  The current
paper extends their results by giving an $O(n)$ time algorithm to compute
$\depth(p,S)$ when $p$ and $S$ are in $\mathbb{R}^d$ for any constant
dimension $d$.

\section{Chan's Generalized Optimization Technique}

Chan \cite{c2004} used the following theorem to provide an $O(n\log
n)$ time algorithm for maximum Tukey depth.  In the following, and
throughout the remainder of the paper, we use the shorthand $\cap S$
to denote $\bigcap_{s\in S}s$.

\begin{thm}[Chan 2004]\thmlabel{chan}
Let $\mathcal{H}$ denote the set of all halfspaces in $\mathbb{R}^d$,
let $\mathcal{P}$ denote the set of all possible inputs to some problem, 
let $f:\mathcal{P}\mapsto 2^{\mathcal{H}}$ be any function mapping 
problem inputs to sets of halfspaces in $\mathbb{R}^d$,
and let $w:\mathbb{R}^d\mapsto \mathbb{R}$ be
any linear objective function.  Suppose that $f$ and $w$ satisfy:
\begin{enumerate}

\item Given inputs $P_1,\ldots,P_d\in\mathcal{P}$ each of constant
size, a point $p\in\cap (f(P_1)\cup\cdots\cup f(P_d))$ maximizing
$w(p)$ can be found in constant time.

\item Given a point $p\in\mathbb{R}^d$ and an input $P\in\mathcal{P}$
of size $n$, there exists a $D(n)$ time algorithm to determine whether
$p\in\cap f(P)$.

\item There exists constants $\alpha < 1$ and $r$ such that, for any
input $P\in\mathcal{P}$ of size $n$, it is possible to compute, in
$D(n)$ time, inputs $P_1,\ldots,P_r$, each of size at most
$\ceil{\alpha n}$, and such that $\cap f(P) =
\cap(f(P_1)\cup\cdots\cup f(P_r))$.

\end{enumerate}
Then there exists an $O(D(n))$ expected time randomized algorithm to
compute, for any input $P\in\mathcal{P}$ of size $n$ a point
$p\in\cap f(P)$ that maximizes $w(p)$.
\end{thm}

It is worth noting that the codomain of the function $f$ may contain
infinite sets.  That is, it is acceptable (and common) to have inputs
$P\in\mathcal{P}$ that generate an infinite number of constraints,
i.e., $|f(P)|=\infty$.


\section{Dual Zonoids}

The $k$-zonoid $Z_k(S)$ is a convex polytope.  The extreme-most vertex
of $Z_k(S)$ in direction $w$ can be obtained as follows
\cite{gmXX,beXX}:  Denote by $S_1^w,\ldots,S_n^w$ the points of $S$
ordered by their projection onto a line in direction $w$.  That is,
$S_i\cdot w \ge S_{i+1}\cdot w$ for all $1\le i\le n-1$.  Then the
extreme-most vertex of $Z_k(S)$ in direction $w$ is given by 
\[  
\argmax\{p\cdot w: p\in Z_k(S)\} =
        \left(\frac{1}{k}\sum_{i=1}^{\floor{k}} S_i^w\right) +
          S_{\ceil{k}}^w(k-\floor{k})
\]
\cite{beXX,gmXX}.
Consider the point-hyperplane duality function
$\dual$ given by 
\[
    \dual(x)=\{y : y_d = x_1y_1 +\cdots +x_{d-1}y_{d-1} - x_d \}
%   \dual(a_1,\ldots,a_d) = 
%      \{(b_1,\ldots,b_d) : a_db_d = a_1b_1+\cdots a_{d-1}b_{d-1}  \}
\] 
when $x$ is a point in $\mathbb{R}^d$ and
\[
     \dual(X) = \{\dual(x) : x\in X\}
\]
when $S$ is a subset of $\mathbb{R}^d$.  The function $\dual$ takes
the points $p_1,\ldots,p_n$ onto hyperplanes $h_1,\ldots,h_n$,
respectively (see, e.g., Edelsbrunner \cite{eXX}).  For a point $x$
and a hyperplane $h$, let $x\drop h$ denote the $d$th coordinate of
the vertical projection of $x$ onto $h$.  For a set $H$ of
hyperplanes, let $H_i^x$ be the $i$th hyperplane in $H$ encountered by
a downward vertical ray originating at $(x_1,\ldots,x_{d-1},\infty)$.
For ease of notation we use the shorthand $H_{-i}^x=H_{|H|-i+1}^x$.

Let $H=\dual(S)$.  Then,
under this duality, the \emph{dual $k$-zonoid} $\dual(Z_k(S))$ is the set 
of all hyperplanes in $\mathbb{R}^d$
that do not intersect either of two convex sets $R_A(S)$ and $R_B(S)$.
That is,
\[
     \dual(Z_k(S)) = \{ h\in\mathbb{H^d} : h\cap(R_A(S)\cup R_B(S)) = \emptyset \} \enspace ,
\]
where
\begin{equation}
   R_A(S) = \left\{x\in\mathbb{R}^d: x_d \ge 
\left(\frac{1}{k}\sum_{i=1}^{\floor{k}} x\drop H_i^x\right) +
          (x\drop H_{\ceil{k}}^x)(k-\floor{k}) \right\}  \eqlabel{ra}
\end{equation} 
and
\begin{equation}
   R_B(S) = \left\{x\in\mathbb{R}^d: x_d \le 
\left(\frac{1}{k}\sum_{i=1}^{\floor{k}} x\drop H_{-i}^x\right) +
          (x\drop H_{-\ceil{k}}^x)(k-\floor{k}) \right\} \enspace .
\end{equation}
The two sets $R_A(S)$ and $R_B(S)$ are convex, unbounded from above,
respectively, below, and piecewise linear.  Indeed, the linear pieces
of $R_A(S)$ (respectively $R_B(S)$) are in one to one correspondence
with the linear pieces of the $\ceil{k}$-level (respectively
the $(n-\floor{k}+1)$-level) of the hyperplanes in $H$.


\section{The Decision Algorithm}

Next we consider the following decision problem:  Given a point set
$S$ and an integer $k$, is the origin contained in $Z_k(S)$?  By
translation, a solution to this problem allows us to test if an
arbitrary point $p\in\mathbb{R}^d$ is contained in $Z_k(S)$. One
approach to solving this problem is to compute the intersection of
$Z_k(S)$ with a vertical line through the origin (the $d$th coordinate
axis) and then check if this intersection contains the origin. 

Under duality, the above strategy is equivalent to finding the lowest
point on $R_A$ and the highest point on $R_B$ and checking that each
of these points is above, respectively, below, the hyperplane
$\{x:x_d=0\}$.
In the remainder, we focus on determining the lowest point in $R_A$.
Finding the highest point in $R_B$ is done in a symmetric manner.  

To apply \thmref{chan} we need to consider a slightly more general
problem.  For a set $S$ of points in $\mathbb{R}^d$ and a function
$v:S\mapsto\mathbb{Z}^+$ define $S^v$ as the multiset obtained by
duplicating each element $p\in S$ $v(p)$ times.  This
\emph{$v$-weighted} set $S$ allows us to naturally define subproblems.
For a subset $C\subseteq S$, define $v(C)=\sum_{p\in C}v(p)$ and
define $\mu(C)=\sum_{p\in C} p\times v(p)/v(C)$.  The
\emph{contraction} of $(S,v)$ by $C$ is obtained by replacing the
points of $C$ by their weighted average, $\mu(C)$.  More precisely,
the contraction of $(S,v)$ by $C$ is the pair $(S',v')$ where
\[
    S' = (S\setminus C)
	\cup \mu(C)
\]
and 
\[
    v'(p) = \left\{\begin{array}{ll}
                   v(p) & \mbox{if $p\in S\setminus C$} \\
                   v(C) & \mbox{if $p=\mu(C)$}
    \end{array}\right.
\]

The following lemma explains the relationship between the zonoids of
$S^v$ and the zonoids of $S'^{v'}$.

\begin{lem}\lemlabel{contraction}
If $(S',v')$ is a contraction of $(S,v)$ by $C$ then $Z_k(S'^{v'})
\subseteq Z_k(S^v)$.
\end{lem}

We now have all the tools required to apply \thmref{chan} to solve our
decision problem.
\begin{proof}
Let $x$ be any point in $Z_k(S'^{v'})$.  Then
\begin{eqnarray*}
    x &=& \sum_{p\in S'^{v'}} p\lambda_p \\
      &=& \sum_{p\in (S\setminus C)^v} p\lambda_p 
          + \lambda_{\mu(C)}v(C)\mu(C) \\
      &=& \sum_{p\in (S\setminus C)^v} p\lambda_p 
          + \lambda_{\mu(C)}\sum_{p\in C} p\times v(p) \\
      &=& \sum_{p\in (S\setminus C)^v} p\lambda_p  
          + \sum_{p\in C^v}p\lambda_{\mu(C)} \\
      &\in& Z_k(S^{v})
\end{eqnarray*}
as required.
\end{proof}

\begin{thm}\thmlabel{decision}
Given a set $S$ of $n$ points in $\mathbb{R}^d$ and a 
function $v:S\mapsto\mathbb{Z}^+$ that is computable in constant
time, the point $x\in R_A(S^v)$ such
that $x_d$ is minimum can be found in $O(n)$ expected
time.
\end{thm}

\begin{proof}
Let $f$ be a function that maps the pair $(S,v)$ onto a set of
halfspaces whose intersection is $R_A(S^v)$.  We need to show that the
function $f$ satisfies the Conditions~(0)--(2) of \thmref{chan}.

To satisfy Condition 0 we can enumerate all the linear constraints
generated by each of the $d$ subproblems and use any linear
programming algorithm to find a point $x$ that satisfies all
constraints and such that $x_d$ is minimum.

To satisfy Condition 1, we observe that, testing if $x\in R_A(S^v)$
simply involves checking if $x$ satisfies \eqref{ra}.  This can be done
using an $O(n)$ time weighted selection algorithm to compute the
$\ceil{k}$th hyperplane in $\dual(S^v)$ intersected by a downward
vertical ray originating at $(x_1,\ldots,x_{d-1},\infty)$.  This gives
the hyperplanes $H_1^x,\ldots,H_{\ceil{k}}^x$ that allow \eqref{ra} to
be verified directly.

To satisfy Condition~2 we make use of \emph{cuttings} \cite{X}.  In
particular, we use the fact that, in $O(n)$ time, it is possible to
partition $\mathbb{R}^d$ into $O(1)$ simplices
$\Delta_1,\ldots,\Delta_r$ such that the interior of each simplex is
intersected by at most $n/2$ of the hyperplanes in $\dual(S)$.  For
each simplex $\Delta_i$ we create a subproblem $(S_i,v_i)$ as follows:
Let $C\subseteq S$ contain every point $p\in S$ such that $\dual(p)$
is above the interior of $\Delta_i$.  We first construct the pair
$(S_i',v_i)$ by contracting $(S,v)$ by $C$.  Next, we obtain $S_i$ by
removing from $S_i'$ every point $p$ such that $\dual(p)$ is below the
interior of $\Delta_i$.  The subproblems $(S_i,v_i)$ for $1\le i\le r$
that we obtain in this manner are each of size at most $n/2+2$.

It follows from \lemref{contraction} and the definition of $Z_k(S)$
that $Z_k(S_i^{v_i})\subseteq Z_k(S^v)$.  In the dual, this means that
$R_A(S_i^{v_i})\supseteq R_A(S^v)$.  To satisfy Condition~2 we must
show that $\bigcap_{i=1}^r R_A(S_i^{v_i}) = R_A(S^v)$.  To see this,
consider any point $x$ on the boundary of $R_A(S^v)$.  It is
sufficient to show that $x$ is also on the boundary of at least one
region $R_A(S_i^{v_i})$ for $1\le i\le r$.  The point $x$ is defined
by $\ceil{k}$ hyperplanes $h_1,\ldots,h_{\ceil{k}}\in\dual(S)$ that pass
above the point $x\drop h_{\ceil{k}}$.  Consider all the simplices in
$\Delta_1,\ldots,\Delta_r$ whose interiors intersect a vertical line
through $x$.  Because each such simplex intersects at most $n/2$
hyperplanes in $\dual(S)$, one of these simplices $\Delta_i$ must
have an interior that either: 
\begin{enumerate} 
\item intersects all of $h_1,\ldots,h_k$ or
\item intersects none of $h_1,\ldots,h_k$
\end{enumerate}


The point $x$ is given
defined by \eqref{ra} and is obtained as the average of $\ceil{k}$ points




\end{proof}


\section{The Optimization Algorithm}

\end{document}

