\documentclass[lotsofwhite]{patmorin}
\usepackage{amsopn}

\input{pat}
\newcommand{\oja}{W}
\DeclareMathOperator{\vol}{vol}

\title{\MakeUppercase{Towards a Centerpoint Theorem for Oja Depth}}
\author{The Caldes de Malavella Gang}
\date{}

\begin{document}
\maketitle

\begin{abstract}
The \emph{Oja weight} of a point $p\in\R^d$ with respect to a set $S$
of $n$ points in $\R^d$ is the sum of the volumes of all simplices
whose vertices consist of $p$ and $d$ points of $S$.  An \emph{Oja
center} of $S$ is a point $p$ whose Oja weight is minimum.
The current paper presents bounds relating the Oja weight of the Oja 
center to the volume of the convex hull of $S$.
\end{abstract}


\section{Introduction}

Let $S$ be a set of $n$ points in $\R^d$.  The \emph{Oja weight} of
$p\in\R^d$ with respect to $S$ is defined as
\begin{equation}
     \oja(p,S) = \sum_{p_1,\ldots,p_d\in S}\vol\{p,p_1,\ldots,p_d\}
      \enspace , \eqnlabel{oja}
\end{equation}
where $\vol(X)$ denotes the volume of the convex hull of
$X$.  An Oja center of $S$ is any
point $p$ that minimizes $W(p,S)$.  We denote the weight of an Oja
center of $S$ simply as $W(S)$.

Oja weight is one of many depth measures used in the robust statistics
literature \cite{X,X,X,X}.  Other measures include Tukey (halfspace)
depth \cite{X}, zonoid depth \cite{X}, simplicial depth \cite{X},
convex hull peeling depth \cite{X}, projection depth \cite{X}, and
many others \cite{X}.

Many of the depth measures described above come with corresponding
\emph{centerpoint theorems} of the form ``For any set $S$ of $n$ points
in $\R^d$ there exists a point $p\in\R^d$ whose depth with respect to $S$ is
at most (or at least) $f(n)$.''  A point whose depth is at most (or at
least) $f(n)$ is referred to as a \emph{centerpoint}.  For Tukey
depth, Helly's theorem implies the existence of a point $p\in\R^d$ of
depth at least $f_T(n)=\floor{n/(d+1)}$ and there are point sets in
which no point has depth greater than $n/(d+1)$ \cite{X}.  For zonoid
depth, there always exists a point (the mean) of depth $f_Z(n)=n$.
For simplicial depth, a theorem of Furedi \cite{fXX} proves the
existence of a point of depth at least $f_S(n)=blah$.

A nice property of centerpoint theorems is that, in many applications,
a centerpoint is a good enough approximation of a point that optimizes
the depth measure.  For some depth measures, a centerpoint is
considerably simpler to compute than the point of maximum depth. For
example, in $\R^2$, a Tukey centerpoint can be found in $O(n)$ time
\cite{mxXX} while an optimal $O(n\log n)$ time algorithm for computing
the Tukey center has only recently been discovered \cite{c04}.

Unfortunately, for Oja depth no centerpoint theorems are known.  In
the current paper we make a first step towards remedying this
situation by giving upper bounds on $W(S)$ in terms of the volume of
the convex hull of $S$.  In particular we show that, for any set $S$
of $n$ points in $\mathbb{R}^d$, $W(S)\le \vol(S)\times {n\choose
d}/(d+1)$ and that there exist point sets $S$ for which $W(S)\ge
\vol(S)\times(n/(d+1))^d$.  The remainder of the paper is dedicated
to establishing these two results.

\section{Lower Bounds}

\begin{thm}
For every $n$ a multiple of $d+1$,
there exists a multiset $S$ of $n$ points in $\R^d$ such that
$W(S)\ge\vol(S)\times (n/(d+1))^d$.
\end{thm}

\begin{proof}
Let $S=\{(p_1)^{n/(d+1)},\ldots,(p_{d+1})^{n/(d+1)}\}$ where 
$p_1,\ldots,p_{d+1}$ are the vertices of a regular simplex
in $\R^d$.  That is, $S$ consists of $n/(d+1)$ copies of each of the
vertices of a regular simplex.
Since the Oja weight function $W_S(p)=W(p,S)$
is convex, it follows from symmetry that the point $p\in\R^d$ that
minimizes $W(p,S)$ is the center of this simplex.  Observe that, for
any distinct $i_1,\ldots,i_d$ with $1\le i_j\le d+1$ for
$j=1,\ldots,d$ that
\[
   \vol\{p,p_{i_1},\ldots,p_{i_d}\} = \vol(S)/(d+1) \enspace .
\]
There are ${d+1 \choose d}=d+1$ ways of choosing $i_1,\ldots,i_d$ and
for each such choice there are $(n/(d+1))^d$ ways of choosing elements
of $S$ that contribute to \eqnref{oja}. Thus,
\[
    W(p,S) \ge \vol(S)/(d+1)\times (d+1)\times (n/(d+1))^d
     = \vol(S)\times (n/(d+1))^d \enspace ,
\]
as required. 
\end{proof}

\section{Upper Bounds}

Next we give an upper bound on $W(S)$ for any point set $S$.  However,
before we do this we prove a lemma about volumes of simplices within
convex polyhedra. The following lemma is similar to a classical result
of Winternitz \cite{X} that has been rediscovered many times
\cite{X}.\footnote{Ehrhart's generalization of Winternitz' result
states that, for any convex polyhedron $P$ and any halfplane $h$ that
contains the center of gravity of $P$, $\vol(P\cap h) \ge
\vol(P)\times (d/(d+1))^d$. \cite{eXX}}

\begin{lem}\lemlabel{simplex}
Let $P$ be a convex polyhedron in $\R^d$, let $g$ be the center of
gravity of $P$, and let $p_1,\ldots,p_d$ be any $d$ points in $P$. Then
\[
   \vol\{g,p_1,\ldots,p_d\} \le \vol(P) / (d+1) \enspace .
\] 
\end{lem}

\begin{proof}
The proof is by induction on the number of facets of $P$.  The base
case occurs when $P$ is a simplex (and therefore has $d+1$ facets).
By affine invariance, it suffices to establish the result for a
regular simplex, and this is easily done.

For the inductive step, let $P$ be a polyhedron with $k > d+1$
facets.   Let $g$ be the center of gravity of $P$ and let
$p_1,\ldots,p_d\in P$ be a set of points that maximizes
$\vol\{g,p_1,\ldots,p_d\}$.

\end{proof}


With \lemref{simplex} in hand our upper bound is now easy to prove.

\begin{thm}
For any set $S$ of $n$ points in $\R^d$, $W(S) \le \vol(S)\times {n \choose
d}/(d+1)$.
\end{thm}

\begin{proof}
Let $p$ be the center of gravity of the convex hull of $S$.  By
\lemref{simplex},
every choice of $d$ points from $S$ contributes at most
$\vol(S)/(d+1)$ to $W(p,S)$, so
\[
    W(S) \le W(p,S) \le \vol(S)/(d+1)\times {n\choose d} \enspace ,
\]
as required.
\end{proof}

\end{document}
