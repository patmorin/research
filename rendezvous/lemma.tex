%! program = pdflatex
\documentclass[12pt]{article}
%\documentclass{article}
\usepackage{natbib}
\usepackage{graphics}
\usepackage{alltt}
\usepackage{a4wide}
\usepackage{amssymb}
\usepackage{latexsym}
\usepackage{epsfig}
%\usepackage{wrapfigure}

%%%% ENVS
\newtheorem{example}{Example}
\newtheorem{definition}{Definition}
\newtheorem{axiom}{Axiom}
\newtheorem{lemma}{Lemma}
\newtheorem{theorem}{Theorem}
\newtheorem{corollary}{Corollary}
\newtheorem{problem}{Problem}
\newtheorem{claim}{Claim}
\newtheorem{remark}{Remark}
\newtheorem{question}{Question}
%%%%%%%%%%%% ENVS %%%%%%%%%%%%%%%

\def\qed{\hfill\rule{2mm}{2mm}}
\def\bqed{\hfill\rule{2mm}{2mm}}
\def\wqed{\hfill\mbox{$\sqcap$\llap{$\sqcup$}}}

\newcommand{\pn}{\pi}
%\newcommand{\ceomni}{{\cal E}_{\mbox{\tiny {\sc omni}}}}
%\newcommand{\3na}{\mbox{$3$-{\sc na}}}
%\newcommand{\2na}{\mbox{$2$-{\sc na}}}
%\newcommand{\23na}{\mbox{$2/3$-{\sc na}}}


\begin{document}

\title{\bf An Inequality for Random Walks}

\author{
Evangelos Kranakis\footnotemark[1]
\and
Danny Krizanc\footnotemark[2]
\and
Patrick Morin\footnotemark[1]
}


\maketitle

\def\thefootnote{\fnsymbol{footnote}}
\footnotetext[1]{School of Computer Science, Carleton University,
K1S 5B6, Ottawa, Ontario, Canada.
Research supported in part by
Natural Sciences and Engineering Research Council
of Canada (NSERC) and Mathematics of Information Technology and Complex
Systems (MITACS).}
\footnotetext[2]{Department of Mathematics,
Wesleyan University, Middletown CT 06459, USA.}

\begin{abstract}
We consider an inequality that is useful in random walks.
\end{abstract}

%%% BEGIN DOCUMENT
%\begin{document}

%\maketitle
%\tableofcontents

\section{An Inequality}


\begin{lemma}
Assume that $X_1, X_2, \ldots , X_n$ are $\{ +1 , -1\}$-valued
i.i.d. random variables such that $\Pr [X_i = +1 ] = \Pr [X_i = -1 ]  =1/2$, for
all $i=1,2,\ldots ,n$. Further assume that
$Y_1, Y_2, \ldots , Y_n$ are also i.i.d. random variables with
mean $E[Y] \geq 0$. If the random variables
$X_i, Y_i$ are independent for all $i=1,2,\ldots ,n$, then
the following inequality is valid
\begin{equation}
\label{rv1:eq}
E\left[ ~\left| \sum_{i=1}^n X_i Y_i \right| ~ \right] \geq 
E[Y] \cdot E\left[ ~\left| \sum_{i=1}^n X_i \right| ~\right] ,
\end{equation}
where $| \cdot |$ denotes the absolute value function.
\end{lemma}
{\bf Proof.}
First of all recall from the definition of the expectation 
that for any random variable $Z$ we have that 
\begin{equation}
\label{rv2:eq}
| E[Z]| \leq E[|Z|] .
\end{equation}
Consider the vectors $x = (x_1, x_2 ,\ldots ,x_n)$ and
$y = (y_1, y_2, \ldots , y_n)$ attaining all the values in the
range of the vector random variables
$X = (X_1, X_2 ,\ldots ,X_n)$ and
$Y = (Y_1, Y_2, \ldots , Y_n)$, respectively. 
From the definition of the expected
value we obtain that
\begin{eqnarray*}
E\left[ ~\left| \sum_{i=1}^n X_i Y_i \right| ~\right]
& = &
\sum_{x,y} \left(
\Pr [X=x] \cdot \Pr [Y=y] \cdot \left| \sum_{i=1}^n x_i y_i \right| 
\right)\\
& = &
\frac{1}{2^n} 
\sum_{x,y} \Pr [Y=y] \cdot \left| \sum_{i=1}^n x_i y_i \right| 
\mbox{ (since $\Pr [X_i = x_i ] = 1/2$)} \\
& = &
\frac{1}{2^n} 
\sum_x \sum_y \Pr [Y=y] \cdot \left| \sum_{i=1}^n x_i y_i \right| \\
& = &
\frac{1}{2^n} 
\sum_x E_y \left[ ~\left| \sum_{i=1}^n x_i Y_i \right| ~\right] 
\mbox{ (definition of expectation)} \\
& \geq &
\frac{1}{2^n} 
\sum_x \left| E_y \left[  \sum_{i=1}^n x_i Y_i  \right] \right| 
\mbox{ (using Inequality~\ref{rv2:eq})} \\
& = &
\frac{1}{2^n} 
\sum_x \left|   \sum_{i=1}^n x_i E_y [ Y_i]  \right| 
\mbox{ (by linearity of expectation)} \\
& = &
\frac{1}{2^n} 
\sum_x E_y [Y] \cdot \left|   \sum_{i=1}^n x_i   \right| 
\mbox{ (since $E_y [Y_i] = E_y [Y] \geq 0$, for all $i$)} \\
& = &
E[Y] \cdot  \sum_x \left( \Pr [X = x] \cdot \left|   \sum_{i=1}^n x_i   \right| \right)\\
& = &
E[Y] \cdot E\left[ ~\left| \sum_{i=1}^n X_i \right| ~\right] 
\mbox{ (definition of expectation)} \\
\end{eqnarray*}
This completes the proof of the lemma.
\qed



%\bibliographystyle{plainnat}
%\bibliographystyle{plain}
%\bibliography{biblio}

\end{document}
