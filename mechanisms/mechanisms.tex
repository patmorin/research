\documentclass{article}
\usepackage{amsthm}

\input{pat.tex}

\newcommand{\E}{\mathrm{E}}

\title{Notes on Mechanism Design}
\author{Pat Morin}

\begin{document}
\maketitle

\section{Mechanisms for Selection by Rank}

Throughout this section $t_1,\ldots,t_k$ are the types of the agents
in increasing order and the utility of agent $i$ is $-t_i$ if agent
$i$ is selected and 0 otherwise.  Note, selecting the smallest element
in a set is the classic Vickrey auction (sacrificial goat) problem and
is achieved by making a payment of $t_2$ (the second smallest value) to
the element selected and 0 to all other elements.

\begin{thm}\thmlabel{no-select}
There is no truthful mechanism for selecting the $k$-th smallest
element, for any $2\le k\le n$.
\end{thm}

\begin{proof} 
The payment $p_k$ to the selected element must be at least $t_k$
otherwise agent $k$ will be encouraged to lie.  On the other hand, the
payment must be at most $t_{k-1}$ otherwise agent $k-1$ is encouraged
to lie.  So we have $p_k \le t_{k-1}$ and $p_k \ge t_{k}$, which is
impossible because there exist inputs where $t_{k-1}<t_k$.  
\end{proof}

Note, an interesting and non-intuitive special case of
\thmref{no-select} occurs when $k=n$, i.e., we are trying to select
the maximum of $n$ elements.  This is somewhat surprising, since
selecting the smallest element is a classic Vickrey auction.  However,
the two problems are quite different since in a Vickrey auction we can
rely on agent 2 to give us an upper bound on the payment required to
make agent 1's utility non-negative.  For the problem of finding the
maximum the only upper bound we have is given by agent $n$, who will
want to exagerate this value in order to increase his payment.

\begin{op}
Is there a truthful mechanism for selecting an approximate median, i.e.,
some element whose rank is between $(1-\alpha)n/2$ and $(1+\alpha)n/2$
for some constant $0<\alpha < 1/2$?
\end{op}

Next we consider the problem of selecting the $k$ smallest elements
among a set of $n$ elements.

\begin{thm}\thmlabel{k-smallest}
For any $1\le k< n$, there exists a truthful mechanism for selecting
the $k$ smallest elements.  
\end{thm}

\begin{proof} 
The problem of finding the $k$ smallest elements can be restarted as
the problem of finding a subset $X\subseteq\{1,\ldots,n\}$, $|X|=k$
such that $g(X) = \sum_{x\in X} -t_x$ is maximized.  This is a
Vickrey-Clarke-Groves (VCG) type problem and can be solved with a VCG
mechanism.
\end{proof}

A particularly convenient mechanism for implementing
\thmref{k-smallest} is the mechanism that pays 0 to each unselected
element and pays $t_{k+1}$ to each selected element.  

The following ``meta-theorem'' about random sampling allows for
truthful mechanisms to sample in subsets that can be found by truthful
mechanisms.

\begin{thm}\thmlabel{random-sample}
Suppose there exists a truthful mechanism $M$ for selecting a subset
$S$ of some fixed size $k$ according to some criteria.  Then there
exists a weakly truthful mechanism for selecting a (uniformly
distributed) random element from $S$.
\end{thm}

\begin{proof}
Create a new mechanism $M'$ where the selected element is paid his
payment in $M$ and all other elements are paid 0.  The expected
utility of any agent participating in $M'$ is $1/k$ times that agent's
utility when participating in $M$.  Since multiplying by a positive
constant does not change the sign of any expression involving
utilities the mechanism $M'$ is truthful provided that $M$ is
truthful.
\end{proof}

Note that there are other mechanisms that could be used in the proof
of \thmref{random-sample}.  In particular, the mechanism that gives
each element selected by $M$ a payment of $1/k$ times its payment in
$M$.

Also note \thmref{random-sample} requires that the value of $k$ be fixed in
advance, independently of $t_1,\ldots,t_n$.  Otherwise, an agent whose
utility in $M$ greater than 0 might lie in order to to decrease the
value of $k$ and increase his expected utility.


\thmref{random-sample} and \thmref{k-smallest} immediately lead to the
following result:

\begin{cor}
For any $1\le k\le n$, there exists a truthful mechanism for selecting
a random element from the $k$ smallest elements.
\end{cor}


\begin{op}[Rank-Weighted Random Selection]
Is there a truthful mechanism for selecting a non-uniformly
distributed element where the probability of selecting element $i$ is
a function of the type vector $t$?  This can't be true in general
since it would contradict (for example) \thmref{no-select}.
\end{op}

\begin{op}[Agent-Weighted Random Selection]
Is there a truthful mechanism for selecting a non-uniformly
distributed element among the $k$ smallest elements where the
probability of selecting element $i$ is a function only of $i$?
\end{op}

\section{Mechanisms for Selection by Weight}

The following result is particularly useful in approximation
algorithms for identifying approximate minima.

\begin{thm}
There exists a truthful mechanism for selecting all elements $i$ such
that $t_i \le \beta t_1$, for any constant $\beta > 1$.
\end{thm}

\begin{proof}
The mechanism pays $\beta t_2$ to the minimum element, pays $\beta
t_1$ to every other selected element and pays 0 to non-selected
elements.  The minimum element has no incentive to lie because doing
so will never increase its payoff beyond $\beta t_2$.  The other
selected elements have no incentive to lie because doing so will only
either decrease their utility to 0 (if they report more than $\beta
t_1$) or do nothing to their payment (if they report something smaller
than their true value).   The unselected elements have no reason to
lie because, if they are selected, they can not receive more than
$\beta t_1$, which would make their utility negative.
\end{proof}

\begin{conj}
There exists a truthful mechanism for selecting a (uniformly
distributed) random element from among all elements $i$ such that
$t_i\le \beta t_1$, for any constant $\beta > 1$.
\end{conj}

\begin{proof}[Sketch]
This would be a mechanism where the expected payment to the smallest
element, if selected, is
\[
\frac{k'}{k}\beta t_2 + \left(1-\frac{k'}{k} \right)\frac{1}{\beta}
t_{k'+1}
\]
where $k$ is the number of elements less than or equal to $\beta t_2$
and $k'$ is the number of elements less than or equal to $\beta t_1$.
Note that Nisan and Ronin's mechanism is a special case of the above
when $n=2$.  Intuitively, this is fair because the value of $k'$ is
under the control of agent 1, while the value of $k$ is not.
This mechanism makes the expected utility of agent 1 equal to
\begin{eqnarray*}
\E[u_1] & = & \frac{1}{k'}\left(\frac{k'}{k}\beta t_2 
   + \left(1-\frac{k'}{k} \right)\frac{1}{\beta} t_{k'+1}-t_1\right) \\
   & = & \frac{1}{k}\beta t_2 + -
\left(\frac{1}{k'}-\frac{1}{k}\right)\frac{1}{\beta}
t_{k'+1} - \frac{1}{k'}t_1 \enspace .
\end{eqnarray*}
\end{proof}

\end{document}
