\documentclass{patmorin}
\usepackage{pat,graphicx}

\title{\MakeUppercase{Point Location in Disconnected Planar Subdivisions}}

\author{Prosenjit~Bose, 
	Karim~Dou\"{\i}eb, 
	Vida~Dujmovi\'c, 
	Jamie~King, and 
	Pat~Morin}



\begin{document}
\maketitle

\begin{abstract}
Let $G$ be a (possibly disconnected) planar subdivision and let $D$
be a probability measure over $\R^2$.  The current paper shows how to
preprocess $(P,D)$ into an $O(n)$ size data structure that can answer
planar point location queries over $G$.  The expected query time of this
data structure, for a query point drawn according to $D$, is $O(H+1)$,
where $H$ is a lower bound on the expected query time of any linear
decision tree for point location in $G$.
This extends the results of Collette \etal\ from connected planar
subdivisions to disconnected planar subdivisions.
\end{abstract}

\section{Introduction}


Collette \etal\ gave an $O(n)$ space data structure that preprocesses
a connected planar subdivision $G$ and a probability measure $D$
over $\R^2$ such that a point location query in $G$ can be answered
in $O(H)$ expected time using.  The expected number of point-line
comparisons needed to answer a query is $H + O(H^{2/3}+1)$.  Here $H$
is a lower-bound on the expected time required by any linear decision
tree for answering queries on $G$ that are drawn according to $D$. Their
work leaves open the problem of what to do when $G$ is disconnected.


\section{The Data Structure}
\seclabel{data-structure}

In this section we describe our data structure for searching in
disconnected planar subdivisions.  The first tool we use is simplicial
partitions, from the field of geometric range searching:

\begin{thm}[Matou\v{s}ek 1991]\thmlabel{point-partition}
Let $S$ be a set of $n$ points in $\R^2$. There exists a constant
$c$ such that, for any $1\le r \le n$, there exists a sequence
$\langle \Delta_1,\ldots,\Delta_r\rangle$ of triangles such that
  \begin{enumerate}
    \item $S\subseteq \bigcup_{i=1}^r \Delta_i$,
  
    \item $\left|\Delta_i \cap S\setminus
    \left(\bigcup_{j=1}^{i-1}\Delta_j\right)\right| \le 2n/r$, and
  
    \item For any line $\ell$, there are at most $cr^{1/2}$ elements of
  $\{\Delta_1,\ldots,\Delta_r\}$ that intersect $\ell$.
  \end{enumerate}
\end{thm}

Note that Part~2 of \thmref{point-partition} is not in the original
statement of the theorem, but follows from Matou\v{s}ek's construction
of $\Delta_1,\ldots,\Delta_r$ \cite{m91}.
Restating \thmref{point-partition} in terms of probability distributions,
we have:

\begin{thm}\thmlabel{prob-partition}
Let $S$ be a set of $n$ points in $\R^2$. There exists a constant
$c$ such that, for any $1\le r \le n$, there exists a sequence
$\langle \Delta_1,\ldots,\Delta_r\rangle$ of triangles such that
  \begin{enumerate}
    \item $\Pr\{\bigcup_{i=1}^r \Delta_i\} = 1$,
  
    \item $\Pr\left\{\Delta_i \cap S\setminus
    \left(\bigcup_{j=1}^{i-1}\Delta_j\right)\right\} \le 2/r$, and
  
    \item For any line $\ell$, there are at most $cr^{1/2}$ elements of
    $\{\Delta_1,\ldots,\Delta_r\}$ that intersect $\ell$.
  \end{enumerate}
\end{thm}

(We need to prove this, or something similar, as well as give an
algorithmic version.  Sampling from $D$ and then applying
\thmref{point-partition} should do it.)

Assume, without loss of generality, that all vertices of $G$ and the support
of $D$ are contained in the unit square $[0,1]^2$.  This can easily be
justified by scaling, translation, and performing 4 point-line comparisons
involving the query point before using the data structure to answer a query.

We use \thmref{prob-partition} to recursively construct a \emph{partition
tree} $T$.  Let $\alpha > 0$ be a constant that will be specified
below.  Refer to \figref{simp-part}. At the root of $T$, we find the
set of triangles $\Delta=\{\Delta_1,\ldots,\Delta_r\}$ and construct
the arrangement of triangles in $\Delta$.  We then compute a Steiner
triangulation of this arrangement in order to obtain a triangulation
$A$. The exact nature of the Steiner triangulation $A$ will be discussed
in the next section.  The triangulation $A$ is stored at the root of
$T$.

\begin{figure}
  \begin{center}
    \begin{tabular}{cc}
      \includegraphics{simp-part} & \includegraphics{simp-part-triang} \\
      (a) & (b)
    \end{tabular}
  \end{center}
  \caption{The triangles of a simplicial partition (a) form an arrangement of
    triangles and the faces of this arrangement are (Steiner) triangulated
    to form a Steiner triangulation $A$ (b).}
  \figlabel{simp-part}
\end{figure}

Next, each face $F$ of $A$ becomes a child of the root of $T$.  If $F$
does not intersect any edge or vertex of $G$ then we call $F$ a
\emph{terminal leaf} and label $F$ with the face of $G$ that contains it.
If the current depth of recursion is greater than $\lfloor\log_{\alpha r}
n\rfloor$ then $F$ becomes a \emph{non-terminal leaf} of $T$.  Otherwise
($F$ intersects an edge or vertex of $G$ and its depth is small),
we recursively apply the same procedure on the distribution $D_{|F}$
to obtain a partition tree that becomes a child of the root.

This construction defines a tree $T=T(S,D)$ in which each node has
$O(r^2)$ children and whose height is at most $\log_{\alpha r} n$.
The number of nodes in $T$ is $(O(r^2))^{\log_{\alpha r} n} =
O(n^{2\alpha+\epsilon})$, where $\epsilon$ is a decreasing function
of $r$.  Note that, for $\alpha < 1/2$ and sufficiently large $r$,
the size of $T$ is $o(n)$.

In addition to the tree $T$ we construct a backup data structure $T'$ that
can answer point location queries in $G$ in $O(\log n)$ worst-case time.

To answer a query, $T$ and $T'$ are used as follows:  We search in $T$
for the query point. If this search ends at a terminal leaf $F$ of $T$
then we report the label at $F$ and the query is complete.  Otherwise we
use $T'$ to answer the query in $O(\log n)$ time.

\section{Analysis}
\seclabel{analysis}

Collette \etal\ \cite{cdilm08,cdilm09} show that, up to a lower-order
term, the expected number of comparisons performed by the optimal
decision tree for point location in $G$ is equal to the entropy of the
minimum-entropy Steiner triangulation of $G$.

\begin{thm}[Collette \etal\ 2008]
Let $G$ be a planar subdivision and let $D$ be a probability measure
over $\R^2$.  Let $T^*$ be a  minimum-entropy Steiner triangulation of
$G$ and let $H^*$ be the entropy of $T^*$.  Then any linear decision tree
for point location in $G$ has expected cost at least $H^*-O(\log H^*)$.
\end{thm}

Thus, our goal is to prove that our query time approximates the entropy
of the minimum entropy Steiner triangulation of $G$.  We will do this
by an argument similar to that used by Dujmovi\'c \etal\ \cite{dhm09}
in the context of orthogonal range searching.

An $i$-set of a rooted tree $T$ is a set of vertices in $T$ all of
which are at distance at most $i$ from the root of $T$ and in which
no vertex in the set is the ancestor of any other vertex in the set.
Note that if $T$ is a partition tree defined in \secref{data-structure}
then an $i$-set of $T$ is a set of disjoint triangles.

\begin{lem}
Let $T$ be the partition tree defined in \secref{data-structure}.
The number of nodes of $T$ whose depth is at most $i$ that are
by any line $\ell$ is $O(r^{(i/2)+\epsilon})$, where
$\epsilon$ is a decreasing function of $r$.
\end{lem}

\begin{proof}
  To prove this, we need to describe how the arrangement of triangles
  obtained from \thmref{prob-partition} is triangulated.  Let $\Delta$
  be the set of $r$ triangles obtained from \thmref{prob-partition}
  and let $\mathcal{V}$ be set of $3r+4$ points that make up the vertices
  of the triangles in $\Delta$ plus the vertices of square $\Box$ that contains all triangles in $\Delta$.  A classic result of Haussler and Welzl
  \cite{hwXX} proves that $V$ has a spanning tree $T(V)$ such that any
  line crosses $O(r^{1/2})$ edges of $T$.

  Consider the line segment arrangement $L$ consisting of the union of the
  edges in $T(V)$, the triangles in $\Delta$, and the edges of $\Box$.
  Note that any line $\ell$  intersects $O(r^{1/2})$ edges of the
  arrangement $L$; $O(r^{1/2})$ of these intersections are generated
  by edges corresponding to edges of $T(V)$ and and $O(r^{1/2})$
  are generated by edges of triangles in $\Delta$.  What remains is
  to show how to triangulate the faces of $L$ without introducing too
  many crossings.

  By construction, each face $F$ of $L$, except the outer face, is
  a (weakly) simple polygon having $O(r)$ vertices and edges on its
  boundary.  By a result of Hershberger and Suri \cite{hs95}, there exists
  a Steiner triangulation of $A(F)$ using $O(r)$ vertices such that the
  any chord of $F$ intersects at most $O(\log r)$ edges of $A(F)$. We
  therefore triangulate the arrangement $L$ by triangulating each of
  its faces in this way.  This gives a Steiner triangulation $A$ of $L$
  in which any line intersects at most $O(r^{1/2}\log r)$ edges of $A$.

  Now, consider any line $\ell$. By the above, the number of elements of nodes $x(\ell,i)$ of $T$ at level $i$ that intersect $\ell$ is given by the recurrence \[
     x(\ell, i)\le \begin\left\{ (cr^{1/2}\log r)\cdot x(\ell, i-1)$, 


$O(r^{1/2}\log r)^i= O(r^{(i/2)+\epsilon})$.
  



Each face of $\mathcal{A}$ is a polygon, possibly with holes, that has complexity $O(r)$.

Consider


Use the ``Shoot a ray take a walk'' triangulation of the faces of $A$.
\end{proof}

We say that a set of regions $X=\{X_1,\ldots,X_m\}$, $X_i\subseteq\R^2$,
is in $k$-general position if there is no line that intersects $k$
elements of $X$.

\begin{lem}
Let $T$ be the partition tree defined in \secref{data-structure} and let $V$ be an $i$-set of $T$.
Let $V$ be a subset of the nodes of $T$ such that no node no node in $V$
is the ancestor of any other node and that all nodes in $V$ have depth
at most $i$.  Then $V$ contains a subset $V'\subseteq V$ that is in $k$-general position and has size at least $V'/Q$.
\end{lem}

\begin{proof}
\end{proof}


\bibliographystyle{plain}
\bibliography{entropy3}
\end{document}
