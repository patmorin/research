\documentclass{beamer}

\mode<presentation>{
\definecolor{cured}{rgb}{.8,0,.2}
\usecolortheme[named=cured]{structure}
%\usefonttheme{structurebold}
\usetheme{split}
}

\usepackage{amsopn}

\DeclareMathOperator{\val}{profit}
\DeclareMathOperator{\average}{average}
\DeclareMathOperator{\support}{support}
\DeclareMathOperator{\boundary}{\partial}
\newcommand{\eps}{\epsilon}

 
\usepackage{amsthm}

\newcommand{\centeripe}[1]{\begin{center}\Ipe{#1}\end{center}}
\newcommand{\comment}[1]{}

\newcommand{\centerpsfig}[1]{\centerline{\psfig{#1}}}

\newcommand{\seclabel}[1]{\label{sec:#1}}
\newcommand{\Secref}[1]{Section~\ref{sec:#1}}
\newcommand{\secref}[1]{\mbox{Section~\ref{sec:#1}}}

\newcommand{\alglabel}[1]{\label{alg:#1}}
\newcommand{\Algref}[1]{Algorithm~\ref{alg:#1}}
\newcommand{\algref}[1]{\mbox{Algorithm~\ref{alg:#1}}}

\newcommand{\applabel}[1]{\label{app:#1}}
\newcommand{\Appref}[1]{Appendix~\ref{app:#1}}
\newcommand{\appref}[1]{\mbox{Appendix~\ref{app:#1}}}

\newcommand{\tablabel}[1]{\label{tab:#1}}
\newcommand{\Tabref}[1]{Table~\ref{tab:#1}}
\newcommand{\tabref}[1]{Table~\ref{tab:#1}}

\newcommand{\figlabel}[1]{\label{fig:#1}}
\newcommand{\Figref}[1]{Figure~\ref{fig:#1}}
\newcommand{\figref}[1]{\mbox{Figure~\ref{fig:#1}}}

\newcommand{\eqlabel}[1]{\label{eq:#1}}
%\newcommand{\eqref}[1]{(\ref{eq:#1})}
\newcommand{\Eqref}[1]{Equation~(\ref{eq:#1})}

\newtheorem{thm}{Theorem}{\bfseries}{\itshape}
\newcommand{\thmlabel}[1]{\label{thm:#1}}
\newcommand{\thmref}[1]{Theorem~\ref{thm:#1}}

\newtheorem{lem}{Lemma}{\bfseries}{\itshape}
\newcommand{\lemlabel}[1]{\label{lem:#1}}
\newcommand{\lemref}[1]{Lemma~\ref{lem:#1}}

\newtheorem{cor}{Corollary}{\bfseries}{\itshape}
\newcommand{\corlabel}[1]{\label{cor:#1}}
\newcommand{\corref}[1]{Corollary~\ref{cor:#1}}

\newtheorem{obs}{Observation}{\bfseries}{\itshape}
\newcommand{\obslabel}[1]{\label{obs:#1}}
\newcommand{\obsref}[1]{Observation~\ref{obs:#1}}

\newtheorem{cond}{Condition}{\bfseries}{\itshape}

\newtheorem{clm}{Claim}{\bfseries}{\itshape}
\newcommand{\clmlabel}[1]{\label{clm:#1}}
\newcommand{\clmref}[1]{Claim~\ref{clm:#1}}


\newtheorem{dfn}{Definition}{\bfseries}{\rm}

\newtheorem{assumption}{Assumption}{\bfseries}{\rm}
\newenvironment{ass}{\begin{assumption}\rm}{\end{assumption}}
\newcommand{\asslabel}[1]{\label{ass:#1}}
\newcommand{\assref}[1]{Assumption~\ref{ass:#1}}

\newcommand{\proclabel}[1]{\label{alg:#1}}
\newcommand{\procref}[1]{Procedure~\ref{alg:#1}}

\newtheorem{rem}{Remark}
\newtheorem{op}{Open Problem}

\newcommand{\etal}{\emph{et al}}

%\newcommand{\keywords}[1]{\noindent\textbf{Keywords:} #1}
\newcommand{\voronoi}{Vorono\u\i}
\newcommand{\ceil}[1]{\left\lceil #1 \right\rceil}
\newcommand{\floor}[1]{\left\lfloor #1 \right\rfloor}
\newcommand{\R}{\mathbb{R}}
\newcommand{\N}{\mathbb{N}}
\newcommand{\Z}{\mathbb{Z}}
\newcommand{\Sp}{\mathbb{S}}
\newcommand{\E}{\mathrm{E}}

\usepackage{marvosym}

\newcommand{\notice}[1]
{
   {\Lightning}
   \marginpar{
      \begin{flushleft}\raggedright
        \hspace{-1.5mm}\Lightning{\small #1}
      \end{flushleft}
   }
}



\title{Algorithms for Price/Quality Tradeoffs}
\author{Joachim Gudmundsson
	\and Pat Morin
	\and Michiel Smid}
\date{February 19, 2009 \\ NICTA Algorithms Seminar \\
  \includegraphics[width=1in]{I-Beam} }

\begin{document}

\frame{\titlepage}

%\section[Outline]{}
%\frame{\tableofcontents}

\section{Introduction}


\frame
{
  \frametitle{Motivating Example - Construction Industry}
  \begin{itemize}
    \item Customer 1 wants to build small office building
      \begin{itemize}
        \item Needs 100 $\ge$10m steel I-beams
        \item Can afford \$200 per beam
      \end{itemize}
    \item Customer 2 wants to build large office building
      \begin{itemize}
        \item Needs 100 $\ge$20m steel I-beams
        \item Can afford \$500 per beam
      \end{itemize}
    \item Customer 3 wants to build bridge
      \begin{itemize}
        \item Needs 20 $\ge$50m steel I-beams
        \item Can afford \$5000 per beam
      \end{itemize}
  \end{itemize}
}

\frame
{
  \frametitle{Price/Quality Tradeoffs}
  \begin{itemize}
    \item We have a market of customers $\{(p_i,q_i):i\in\{1,\ldots,n\}\}$
    \begin{itemize}
      \item Customer is willing to pay at most $p_i$
      \item Customer wants quality at least $q_i$
    \end{itemize}
    \item We want to market a product $(p,q)$ that maximizes profit
    \begin{itemize}
      \item $q$ is also the cost to manufacture
      \item We want to maximize 
          $\val(p,q) = (p-q)\times(\mbox{\# units sold}$)
    \end{itemize}
  \end{itemize}
}

\frame
{
  \frametitle{Price/Quality Tradeoffs - Graphically}
  \begin{center}
     \only<1>{\includegraphics{s-graphical}}%
     \only<2>{\includegraphics{s-graphical2}}%
     \only<3>{\includegraphics{s-graphical3}}%
     \only<4>{\includegraphics{s-graphical4}}
  \end{center}
}

\frame
{
  \frametitle{Technical Condition}
  \begin{itemize}
    \item We assume all customers are \emph{Pareto optimal}
    \begin{itemize}
      \item There are no $i,j$ such that $p_i < p_j$ and $q_i > q_j$
      \item Customer $i$ wants more for less
    \end{itemize} 
    \includegraphics{s-pareto}
    \item Justified in \emph{saturated} markets
  \end{itemize}
}


\frame
{
  \frametitle{A First Observation}
  \begin{lem}
    The optimal solution is of the form $(p_i, q_j)$ for some 
       $i,j\in\{1,\ldots,n\}$
  \end{lem}
  \begin{center}\includegraphics[scale=0.8]{s-graphical}\end{center}
  \begin{itemize}
    \item<2> Gives easy $O(n^2)$ time algorithm
  \end{itemize}
}

\frame
{
  \frametitle{Price Sweeping}
  \begin{itemize}
    \item Reduce the price from $\infty$ down to $-\infty$,
     pausing at each $p_i$ \\
    \item[] \includegraphics[scale=0.8]{s-graphical}
    \item Need to keep track of the \emph{sweep line status}
  \end{itemize}
}

\frame
{
  \frametitle{A Useful Lemma}
  \begin{lem}
    $\color{blue}{\val(p,q')} \color{black}\ge \color{red}\val(p,q) 
        \color{black}\Rightarrow \color{blue}{\val(p',q')} \color{black}\ge
\color{red}\val(p',q)$
  \end{lem}
  \begin{center}
    \only<1>{\includegraphics{s-lemma-monotone}}%
    \only<2>{\includegraphics{s-lemma-monotone2}}
  \end{center}
}

\frame
{
  \frametitle{The Sweep Line Status}
  \begin{itemize}
    \item For the current price $p$, the sweepline stores a list $L=\langle
\ell_1,\ldots,\ell_k\rangle$
    \begin{enumerate}
      \item $\ell_1>\ell_2>\cdots>\ell_k$
      \item $\val(p,\ell_1)>\val(p,\ell_2)>\cdots>\val(p,\ell_k)$
    \end{enumerate}
    \item When updating $L$, we need to ensure (1) and (2) are maintained
  \end{itemize}
}

\frame
{
  \frametitle{The Sweep Line Status}
  \begin{itemize}
    \item We must maintain
    \begin{enumerate}
      \item $\ell_1>\ell_2>\cdots>\ell_k$
      \item $\val(p,\ell_1)>\val(p,\ell_2)>\cdots>\val(p,\ell_k)$
    \end{enumerate}
    \includegraphics{s-sweepline}
    \item To maintain (1), we append $q_i$ to $L$
  \end{itemize}
}

\frame
{
  \frametitle{The Sweep Line Status}
  \begin{itemize}
    \item We have a list $L$ with
    \begin{enumerate}
      \item $\ell_1>\ell_2>\cdots>\ell_k$
    \end{enumerate}
    \item We want to delete elements from $L$ so that
    \begin{enumerate} 
      \item[2.] $\val(p,\ell_1)>\val(p,\ell_2)>\cdots>\val(p,\ell_k)$
    \end{enumerate}
    \item Consider the inequality $\val(p_{t'},\ell_i) > \val(p_{t'},\ell_{i+1})$
      \[
         (p_{t'}-\ell_i)(a_i+t'-t) \le (p_{t'}-\ell_{i+1})(a_{i+1}+t'-t)  \enspace , 
      \]
      where $t$ is the ``time'' at which $\ell_i$ and $\ell_{i+1}$ became
adjacent in $L$
    \item This is a linear inequality in $p_{t'}$ and $t'$
  \end{itemize}
}

\frame
{
  \frametitle{Storing Linear Inequalities}
  \begin{itemize}
    \item We need a structure $D$ that 
    \begin{itemize}
      \item stores linear inequalities (over $p_t$ and $t$)
      \item can insert and delete inequalities
      \item Given a query $(x,y)$ return an inequality (if any) that is
         violated when $p_t=x$ and $t=y$
    \end{itemize}
    \item Brodal and Jacob (2002) - $O(\log n)$ time per operation
  \end{itemize}
}

\frame
{
  \frametitle{Summary of Algorithm}
  \begin{itemize}
    \item We sweep a price-line over $n$ points and maintain a list $L$
    \item To update $L$ when sweeping over $(p_t,q_t)$
    \begin{itemize}
      \item Append $q_i$ to $L$ ($\ell_k = q_t$)
      \item Insert the inequality $\val(\ell_{k-1}, p) > \val(\ell_{k},p)$
            into $D$
      \item While some inequality $I$ in $D$ is violated by $(p_t,t)$
      \begin{itemize}
        \item Delete $I$ from $D$
        \item Delete the offending element $\ell_j$ from $L$
        \item Insert the inequality $\val(\ell_{j-1}, p) >
\val(\ell_{j+1},p)$ into $D$
      \end{itemize}
    \end{itemize}
    \item<2-> The optimal solution will appear as $(\ell_1,p_t)$ at some time
       $t\in\{1,\ldots,n\}$.
    \item<3-> Runs in $O(n\log n)$ time
  \end{itemize}
}

\frame
{
  \frametitle{Algorithm for 1-d}

  \begin{thm}
    There exists an $O(n\log n)$ time algorithm that, given a set of $n$ customers
$\{(p_i,q_i) : i\in\{1,\ldots,n\}\}$, returns the point $(p^*,q^*)$ that
maximizes $\val(p^*,q^*)$.
  \end{thm}

  \begin{thm}
    Any algebraic decision tree algorithm that provides a $c$-approximation
for $c<2$ requires $\Omega(n\log n)$ time.
  \end{thm}
}

\frame
{
  \frametitle{An Approximation Algorithm}

  \begin{itemize}
    \item Consider the value $r=\max\{p_i-q_i: i\in\{1,\ldots,n\}\}$
    \item Let $H_i = \{(p,q) : p-q = (1-\epsilon)^i r\}$ for
$i=0,\ldots,\lceil\log_{1/(1-\epsilon)} n\rceil$
    \item[] \includegraphics[scale=0.7]{s-gradation}
    \item[]
    \begin{lem}
      For any point $(p^*,q^*)$, there exists some $i$ such that $(p,q)\in
H_i$ and $\val(p,q) \ge (1-\epsilon)\val(p^*,q^*)$
    \end{lem}
  \end{itemize}
}

\frame
{
  \frametitle{An Approximation Algorithm}

  \begin{itemize}
    \item Find the point on each $H_i$ contained in the
largest number of intervals
    \item[] \includegraphics[scale=0.7]{s-gradation}
    \item Can be solved in $O(n\log n)$ time for each $H_i$
    \item Gives an $O(n (\log n)^2)$ time $(1-\epsilon)$ approximation
     algorithm
  \end{itemize}
}

\frame
{
  \frametitle{Higher Dimensions}

  \begin{itemize}
    \item Each customer is modelled as a $(d+1)$-tuple
$(p_i,q_{i,1},\ldots,q_{i,d})$
    \item Customer $i$ will buy $(p,q_1,\ldots,q_d)$ if $p \le p_i$ and $q_j
\ge q_{i,j}$ for each $j\in\{1,\ldots,d\}$
    \item $H_i$ is now a hyperplane in $\R^{d+1}$
    \item Should find the point in $H_i$ contained in the largest number of
       simplices
    \item Can be done (approximately) in $O(n(\log n)^{O(d)})$ time
    \item[]
     \begin{thm}
       There exists an $O(\eps^{-(2d+1)}n(\log n)^d + n(\log n)^{d+1})$
       time $(1-\epsilon)$-approximation algorithm
     \end{thm}
  \end{itemize}
}

\frame
{
  \frametitle{Conclusion}

  \begin{itemize}
    \item $O(n\log n)$ time exact algorithm for $d=1$
    \item $O_{\epsilon,d}(n(\log n)^{O(d)})$ time $(1-\epsilon)$-approximation algorithm
    \item Open Problem: What about the case where we want to introduce $k >
1$ new products into the market?
  \end{itemize}
}

\end{document}

