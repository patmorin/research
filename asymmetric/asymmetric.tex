\documentclass[lotsofwhite]{patmorin}
\usepackage{graphicx,ipe}
\usepackage{amsthm}

\input{pat.tex}

\newcommand{\sbits}{\sigma}
\newcommand{\cbits}{\phi}
\newcommand{\bbq}{\lambda}
\newcommand{\rounds}{\rho}

\newcommand{\path}{d}
\newcommand{\cost}{c}

\title{\MakeUppercase{Asymmetric Communication Protocols via Hotlink
    Assignments}% 
  \thanks{This work was partly supported by NSERC,
    CRM, FCAR and MITACS.}}

\author{Prosenjit Bose%
  \thanks{School of Computer Science, Carleton University, 
    \texttt{\{jit,morin\}@cs.carleton.ca}} \and
  Danny Krizanc%
  \thanks{Wesleyan University, \texttt{krizanc@caucus.cs.wesleyan.edu}} \and
  Stefan Langerman%
  \thanks{School of Computer Science, McGill University, 
    \texttt{sl@cgm.cs.mcgill.ca}} \and
  Pat Morin\footnotemark[2]
}

\date{}

\begin{document}
\maketitle

\begin{abstract}
We exhibit a relationship between the asymmetric communication problem
of Adler and Maggs (1998) and the hotlink assignment problem of Bose
\etal\ (2000). By generalizing previous results on the hotlink problem
and then exploiting this relationship we present a new asymmetric
communication protocol with different performance bounds than previous
protocols.
\end{abstract}

\noindent\textbf{Keywords:} Communication Protocols, Asynchronous
Communication, Hotlink Assignment.


\section{Introduction}

In the last decade or so there have been a number of services
introduced that provide asymmetric broadband communication to homes
and small offices.  With these services, clients can typically
download at rates ranging from 250KBps to 2MBps but can only upload at
much lower rates, often an order of magnitude less than the download
rates.  Sometimes this discrepancy is created artificially by Internet
service providers in order to discourage clients from running servers
on these connections.  In other cases, it is intrinsic to the
technology being used.  An example of the latter case is home satellite
Internet connections in which downloads are received through a
satellite dish, but uploads take place over the telephone line.

Motivated by this, Adler and Maggs \cite{am98} describe a model of an
asymmetric communication channel that captures (roughly) the asymmetry
in these kinds of networks.  In this model, a client is attempting to
send an $n$-bit request string to a server, where the request string
is drawn according to some probability distribution
$D:\{0,1\}^n\rightarrow [0,1]$.  The client does not know the
distribution $D$, but the server has access to $D$ through \emph{black
box} queries in which the server can specify a $k$-bit string $s$,
($k\le n$) and the black box returns the cumulative probability of all
requests that have $s$ as a prefix.

A protocol is a $[\sbits,\cbits,\bbq,\rounds]$-protocol if, using this
protocol, $\sbits$, $\cbits$, $\bbq$, and $\rounds$ are upper bounds
on the expected numbers of: bits sent by the server, bits sent by the
client, black box queries, and rounds, respectively.
\tabref{previous-results} summarizes previous results on protocols for
asymmetric communication channels.  In this table,
$H=-\sum_{r\in\{0,1\}^n} D(r)\log D(r)$ is \emph{entropy} of the
distribution $D$ and $k\ge 1$ is a free integer parameter of the
algorithm.\footnote{Throughout this paper, when the base of a
logarithm is not explicitly specified, it is assumed to be base 2.}

\begin{table}
\begin{center}\begin{tabular}{l|c|c|c|c}
Ref. & $\sbits$ & $\cbits$ & $\bbq$ & $\rounds$ \\ \hline
\cite{am98} & $3n$   & $1.71H+1$ & $3n$ & $1.71H+1$ \\
\cite{am98} & $O(n)$   & $O(H+1)$ & $2^n$ & $O(1)$ \\
\cite{am98} & $O(n)$   & $O(H+1)$ & $n2^k/k$ & $O(\min\{n/k,H+1\})$ \\
\cite{lh02} & $3n$   & $1.08H+1$ & $3n$ & $1.08H+1$ \\
\cite{waf01} & $n(H+2)$   & $H+2$ & $n(H+3)$ & $H+2$ \\
\cite{waf01} & $O(2^knH)$ & $H+2$ & $O(2^knH)$ & $(H+1)/k+2$ \\
\cite{ggs01}  & $nHk+1$ & $H\log_{k-1}k+1$ & $O(nk)$ & $H/\log k+1$ \\
Here & $n(k+2)$ & $\frac{H\log(k+2)}{\log(k+2)-1}+\log(k+2)$ & $n(k+2)$ &
$\frac{H}{\log(k+2)-1}+1$ \\
\end{tabular}\end{center}
\caption{Previous and new results on asymmetric communication protocols.}
\tablabel{previous-results}
\end{table}

In this paper, we exhibit a relationship between the asymmetric
communication problem of Adler and Maggs \cite{am98} and the hotlink
assignment problem of Bose \etal\ \cite{bcgk00}.  We show that hotlink
assignments can be viewed as asymmetric communication protocols.  When
we apply this knowledge along with improved results on hotlink
assignment, we obtain new results on asymmetric communication
protocols.  In particular, we obtain the last row of
\tabref{previous-results}, which is the first protocol that, for any
constant $\epsilon > 0$, can achieve
$\cbits=(1+\epsilon)H+c_\epsilon$, while maintaining
$\sbits=c_\epsilon n$, where $c_\epsilon$ is a function that depends
only the value of $\epsilon$ and is therefore constant for any fixed
$\epsilon$.

The remainder of this paper is organized as follows: In
\secref{hotlink} we describe the hotlink assignment problem and
generalize an approximation algorithm for this problem due to Kranakis
\etal\ \cite{kks01}.  In \secref{relationship} we show how a hotlink
assignment leads to an asymmetric communication protocol and how the
hotlink assignment described in \secref{hotlink} leads to a
particularly efficient protocol.  Finally, \secref{conclusions}
summarizes and makes some concluding remarks.

\section{Hotlink Assignment}\seclabel{hotlink}

Motivated by the goal of improving web site performance by making the
most popular pages more accessible, Bose \etal\ \cite{bcgk00}
introduced the hotlink assignment problem.  Let $T$ be a rooted tree
with $m$ leaves and a probability distribution $D$ on its leaves, and
let the entropy of this distribution be $H$.  The cost, $\cost(v)$ of
a vertex $v$ of $T$ is the sum of the probabilities of all leaves in
the subtree rooted at $v$.  The \emph{cost} $\cost(u,v)$ of an edge
$(u,v)$ of $T$, where $u$ is the parent of $v$ is equal to $\cost(v)$.
The cost, $\cost(T)$ of $T$ is the sum of all its edge costs.
Equivalently, if $L(T)$ is the set of leaves of $T$, and $\path(v,T)$
denotes the length of the path from the root of $T$ to $v$ then \[
\cost(T)=\sum_{v\in L(T)}\cost(v)\path(v,T) \enspace .  \]

An \emph{adoption} operation at a node $u$ of $T$ creates a new tree
$T'$ by taking a descendant $w$ of $u$, detaching $w$ from its parent
and making $u$ the parent of $w$ (see \figref{adoption}). Obviously,
adoptions can change the costs of the edges and internal nodes of
$T'$.  A tree $T'$ is a \emph{$k$-hotlink assignment} of $T$ if it can
be obtained in the following way: Perform up to $k$ adoptions at the
root $r$ of $T$ to obtain a new tree $T_1$ and then create $k$-hotlink
assignments for each of the children of $r$ in $T_1$.  The
\emph{$k$-hotlink assignment problem} is to find the hotlink
assignment of $T$ with minimum cost.

\begin{figure}
\begin{center}\begin{tabular}{ccc}
\Ipe{adoption-a} & \raisebox{.75cm}{$\Longrightarrow$} & \Ipe{adoption-b} \\
\end{tabular}\end{center}
\caption{The adoption of $w$ by $u$.}\figlabel{adoption}
\end{figure}

In \cite{kks01}, an approximation algorithm that obtains a
near-optimal solution to the $1$-hotlink assignment for binary trees
is given.  A natural generalization of that method to the $k$-hotlink
assignment problem on binary trees may be described as follows: When
processing a vertex $u$, we perform adoptions at node $u$ until each
child $w$ of $u$ is either a leaf or satisfies
$\cost(w)\le\frac{2}{k+2}\cost(u)$.  To determine which descendant $w$
of $u$ to adopt next, we start at the non-leaf child of $u$ with
maximum cost and repeatedly move to the child of the current node with
largest cost until reaching a node $w$ such that, either
$\cost(w)\le\frac{2}{k+2}\cost(u)$ or $w$ is a leaf.

The following lemma shows that the above algorithm actually does
produce a $k$-hotlink assignment.
\begin{lem}
During the processing of $u$, at most $k$ adoptions are performed.
\end{lem}

\begin{proof}
By construction, every descendant $w$ of $u$ that is adopted while
processing $u$ is either a leaf, or has weight
$\cost(w)\le\frac{2}{k+2}\cost(u)$.  Thus, we need only show that,
after at most $k$ adoptions, $\cost(x)\le\frac{2}{k+2}\cost(u)$ and
$\cost(y)\le\frac{2}{k+2}\cost(u)$ where $x$ and $y$ are the two
children of $u$ in the original binary tree $T$.  We will show
something stronger; after at most $k$ adoptions,
$\cost(x)+\cost(y)\le\frac{2}{k+2}\cost(u)$.

We claim that every node that $u$ adopts has weight at least
$\frac{1}{k+2}\cost(u)$.  To prove this, we argue that the procedure
for finding the node $w$ never moves to a node $v$ with cost less than
$\frac{1}{k+2}\cost(u)$.  Observe that every descendant of $u$ has at
most two children and the algorithm only moves to a child of the
current node $v$ if $\cost(v)>\frac{2}{k+2}\cost(u)$.  But then, the
child $v'$ that the procedure moves to has cost at least
$\cost(v')\ge\frac{1}{2}\cost(v)>\frac{1}{k+2}\cost(u)$.  Therefore,
the procedure never moves to a node of cost less than
$\frac{1}{k+2}\cost(u)$ and the algorithm never has $u$ adopt a node
of weight less than $\frac{1}{k+2}\cost(u)$, as claimed.

Therefore, each adoption performed at $u$ reduces $\cost(x)+\cost(y)$
by at least $\frac{1}{k+2}\cost(u)$ and after at most $k$ adoptions
\[ \cost(x)+\cost(y)\le \left(1-\frac{k}{k+2}\right)\cost(u)
        = \left(\frac{2}{k+2}\right)\cost(u) \enspace ,
\]
as required.
\end{proof}

\begin{thm}\thmlabel{cost}
A hotlink assignment obtained by the above algorithm has a cost of at most
\[
  \cost(T')\le \frac{H}{\log(k+2)-1} + 1 \enspace .
\]
\end{thm}

\begin{proof}
The hotlink assignment produced by the algorithm has the property that
for every edge $(u,v)$ where $u$ is a parent of $v$ and $v$ is not a
leaf, $\cost(v)\le\frac{2}{k+2}\cost(u)$.  An immediate consequence of
this is that the length of the path from the root of $T'$ to any leaf
$v$ is at most
\[
  \path(v,T')\le \frac{\log(1/\cost(v))}{\log((k+2)/2)} + 1 =
  \frac{\log(1/\cost(v))}{\log(k+2)-1} + 1 \enspace .
\]
Therefore,
\begin{eqnarray*}
\cost(T') & = & \sum_{v\in L(T')}\cost(v)\path(v,T') \\ 
        & \le & \sum_{v\in
L(T')}\left(\cost(v)\left(\frac{\log(1/\cost(v))}{\log(k+2)-1}+1\right)\right) \\ 
        & = & \frac{H}{\log(k+2)-1} + 1 \enspace , 
\end{eqnarray*}
as required.
\end{proof}

\noindent\textbf{Remark:} For the case $k=1$, a different, more
detailed, analysis of the hotlink assignment algorithm shows that the
cost of the resulting tree is at most $H/(\log 3-2/3)+3/2\approx
1.08H+3/2$ \cite{kks01,lh02}.  Repeating this analysis for larger values
of $k$ does not yield as much improvement.  In fact for even $k$, the
detailed analysis yields the same result and yields only a minor
improvement for odd values of $k$.


\section{Hotlink Assignments are Asymmetric Communication Protocols}
\seclabel{relationship}

Next we show how hotlink assignments and asymmetric communication
protocols are related by showing how hotlink assignments give
asymmetric communication protocols.  In this protocol, the server
sends bit strings of variable length and requires that the client know
the lengths of the strings.  Like Adler and Maggs \cite{am98}, we
assume that the server can do this, and the client can determine the
length of these strings.  When this is not the case, the server and
client can use a variable-length encoding to prepend each string with
its length.  This allows the server to send an $m$-bit string using
$m+\log m+O(\log\log m)$ bits \cite{e75}.

Let $T$ be a complete binary tree with $2^n$ leaves and label each
edge of $T$ with a 0 or 1 depending on whether the edge leads to a
left or right child, respectively.  Label each vertex $v$ of $T$ with
the sequence of edge labels encountered on the path from the root of
$T$ to $v$.  In this way, we obtain a bijection between the leaves of
$T$ and $\{0,1\}^n$.

Suppose we have a $k$-hotlink assignment $T'$ for $T$.  Then we use
the following communication protocol to send the $n$-bit request
string $r$ from the client to the server.  The client and server split
$r$ into two parts, $r_1$ and $r_2$ where $r_1$ is the prefix of $r$
that has already been transmitted and $r_2$ is the suffix of $r$ that
remains to be transmitted.  Initially, $r_1$ is empty and the server
does not know $r_2$.

During a round of the protocol, the server considers the node $u$ of
$T'$ whose label is $r_1$.  For each child $w$ of $u$, the server
sends the label of $w$ less the prefix $r_1$ to the client.  The
client then responds with the index of the longest label $l$ that is a
prefix of $r_2$.  The server and client both append $l$ to $r_1$, and
the client removes $l$ from $r_2$.  The client and server both
continue in this manner until the entire request string has been
transmitted (this happens when the length of $r_1$ is $n$).

It is clear that this protocol correctly sends the request string $r$
from the client to the server.  It is also clear that, if $r$ is the
label of the leaf $v$ in $T$, then the number of bits sent by the
client is $\lceil\log(k+2)\rceil\path(v,T')$.  If $D$ is
simultaneously the distribution on request strings and the
distribution on leaves of $T$ (so that $D(r)=D(\mbox{leaf labelled
$r$})$) then the expected number of bits sent by the client in this
protocol is
\[
 \cbits = \cost(T')\lceil\log(k+2)\rceil \enspace .
\]

If $T'$ is a $k$-hotlink assignment produced by the algorithm of the
previous section then, by \thmref{cost},
\[
 \cbits \le \frac{H\lceil\log(k+2)\rceil}{\log(k+2)-1}+\lceil\log(k+2)\rceil \enspace ,
\]
and
\[
    \rounds \le \frac{H}{\log(k+2)-1} + 1
\]
is an upper bound on the expected number of rounds used by the protocol.

Next we analyze the number of bits sent by the server if $T'$ is a
hotlink assignment obtained by the algorithm of the previous section.
Let $E_i$ be the expected number of strings sent by the server to the
client that include the $i$th bit position of $r$.  If it receives any
such string, the client accepts it with probability at least
$1/(k+2)$.  Therefore, the expected number of times the client accepts
a string that includes the $i$th bit position of $r$ is
\[
   A_i \ge E_i/(k+2) \enspace .
\]
However, the client only accepts a string that contains the $i$th bit
position of $r$ once, after which no other strings contain the $i$th
bit position of $r$, so $A_i=1$.  Therefore, $E_i\le k+2$.  Summing
this over all $i$, the expected number of bits sent by the server is
\[
   \sbits\le n(k+2) \enspace .
\]

Finally, we consider the expected number of black box queries required
by this algorithm.  Observe that we can compute the tree $T$ online,
so that we need only compute the adoptions at a vertex $u$ if the
client has accepted the prefix that is the label of $u$.  In computing
these adoptions, the only time we consider the costs of vertices is
when moving from one node $u$ to the child of $u$ with larger cost.
If we already know $\cost(v)$, then this can be implemented as one
black box query by querying, say, the left child of $v$.  Furthermore,
each such query corresponds to exactly one more bit sent by the server
to the client.  Therefore, the expected number of black box queries
performed by the server is
\[
   \bbq = \sbits \le n(k+2) \enspace .
\] 

We have just proven
\begin{thm}
There exists an 
\[
\left[ n(k+2), \left(\frac{H}{\log(k+2)-1}+1\right)\lceil\log(k+2)\rceil,
        n(k+2), \frac{H}{\log(k+2)-1} \right]
\]
asymmetric communication protocol.
\end{thm}


\section{Conclusions and Remarks}\seclabel{conclusions}

We have shown a relationship between the hotlink assignment problem
and protocols for asymmetric communication channels.  By generalizing
previous results for the hotlink assignment problem and then
exploiting this relationship we give the first protocol for asymmetric
communication channels that achieves $\cbits=(1+\epsilon)H+c_\epsilon$
for any constant $\epsilon>0$ while maintaining $\sbits=c_\epsilon n$,
where $c_\epsilon$ is a function that depends only on $\epsilon$.

Previously proposed asymmetric communication protocols all assume that
the universe of requests is the set of all $n$-bit strings.  One nice
property of the relationship between asymmetric communication
protocols and hotlink assignments is that hotlink assignments are not
restricted to complete binary trees.  A consequence of this is that,
if the client and server are attempting to exchange requests from any
universe $U$ of binary strings, a hotlink assignment of the binary
trie containing all the strings in $U$ gives an efficient asymmetric
communication protocol. \comment{In fact, it is straightforward to
extend the results of this paper to obtain a
$[a(k+2),H,a(k+2),H/\log(k+2)]$ protocol, where $a$ is the expected
length of the request string $r$.}

\bibliography{asymmetric} 
\bibliographystyle{plain}

\end{document}
