\documentclass{article}
\usepackage{IJCGA}
\usepackage{amsfonts}
\usepackage{url}
\usepackage{graphicx}
\usepackage{ipe}

\newcommand{\centeripe}[1]{\begin{center}\Ipe{#1}\end{center}}
\newcommand{\comment}[1]{}
\newcommand{\erik}[1]{#1}
\newcommand{\new}{\makebox[0cm]{$\bigotimes$}}
\newcommand{\centerpsfig}[1]{\centerline{\psfig{#1}}}

\newcommand{\etal}{\emph{et al.}}

\newcommand{\seclabel}[1]{\label{sec:#1}}
\newcommand{\secref}[1]{\mbox{Section~\ref{sec:#1}}}

\newcommand{\tablabel}[1]{\label{tab:#1}}
\newcommand{\tabref}[1]{Table~\ref{tab:#1}}

\newcommand{\figlabel}[1]{\label{fig:#1}}
\newcommand{\Figref}[1]{Figure~\ref{fig:#1}}
\newcommand{\figref}[1]{\mbox{Fig.~\ref{fig:#1}}}

\newcommand{\eqlabel}[1]{\label{eq:#1}}
\newcommand{\eqref}[1]{(\ref{eq:#1})}

\newtheorem{thm}{Theorem}{\bfseries}{\itshape}
\newcommand{\thmlabel}[1]{\label{thm:#1}}
\newcommand{\thmref}[1]{Theorem~\ref{thm:#1}}

\newtheorem{lem}{Lemma}{\bfseries}{\itshape}
\newcommand{\lemlabel}[1]{\label{lem:#1}}
\newcommand{\lemref}[1]{Lemma~\ref{lem:#1}}

\newtheorem{cor}{Corollary}{\bfseries}{\itshape}
\newcommand{\corlabel}[1]{\label{cor:#1}}
\newcommand{\corref}[1]{Corollary~\ref{cor:#1}}

\newtheorem{clm}{Claim}{\bfseries}{\itshape}
\newcommand{\clmlabel}[1]{\label{clm:#1}}
\newcommand{\clmref}[1]{Claim~\ref{clm:#1}}

\newtheorem{obs}{Observation}{\bfseries}{\itshape}
\newcommand{\obslabel}[1]{\label{obs:#1}}
\newcommand{\obsref}[1]{Observation~\ref{obs:#1}}

\newtheorem{assumption}{Assumption}{\bfseries}{\rm}
\newenvironment{ass}{\begin{assumption}\rm}{\end{assumption}}
\newcommand{\asslabel}[1]{\label{ass:#1}}
\newcommand{\assref}[1]{Assumption~\ref{ass:#1}}

\newcommand{\proclabel}[1]{\label{alg:#1}}
\newcommand{\procref}[1]{Procedure~\ref{alg:#1}}

\newcommand{\vsrc}{s}
\newcommand{\vdest}{t}
\newcommand{\N}{N}

\newcommand{\cw}{\mathrm{cw}}
\newcommand{\ccw}{\mathrm{ccw}}
\newcommand{\rot}{\mathrm{rot}}
\newcommand{\gc}{\textsc{greedy-compass}}
\newcommand{\rc}{\textsc{random-compass}}
\newcommand{\gct}{\textsc{greedy-compass-2}}
\newcommand{\compass}{\textsc{compass}}
\newcommand{\greedy}{\textsc{greedy}}

\newcommand{\DT}{\mathit{DT}}

\setlength{\parskip}{1ex}

\title{Online Routing in Convex Subdivisions\thanks{ 
	This research was partly funded by the
	Natural Sciences and Engineering Research Council of Canada.}}

\author{Prosenjit Bose\thanks{ School of Computer Science, Carleton University, 1125 Colonel By Dr., Ottawa, Ontario, Canada, \mbox{K1S 5B6}, \texttt{\{jit,morin\}@scs.carleton.ca}}
\and Andrej Brodnik\thanks{IMFM, University of Ljubljana, Jadranska 11, SI-1111 Ljubljana, Slovenia and Department of Computer Science, Lule\aa\ Technical University, SE-971 87 Lule\aa, Sweden.  \texttt{Andrej.Brodnik@IMFM.Uni-Lj.SI}}
\and Svante Carlsson\thanks{ University of Karlskona/Ronneby, 371 41 KARLSKRONA, Sweden, \texttt{svante.carlsson@sm.luth.se}}
\and Erik D. Demaine\thanks{Department of Computer Science, University of Waterloo, Waterloo, Ontario, Canada, \mbox{N2L 3G1}, \texttt{\{eddemain,rudolf,imunro\}@uwaterloo.ca}}
\and Rudolf Fleischer$^\P$
\and Alejandro L\'opez-Ortiz$^\P$
\and Pat Morin$^\dagger$
\and J. Ian Munro$^\P$}

\date{}


\begin{document}
\copyrightheading

% This produces an error
% \symbolfootnote
% so use this instead
\renewcommand{\thefootnote}{\fnsymbol{footnote}}


\textlineskip

\begin{center}
\cgatitle{\MakeUppercase{Online Routing in Convex Subdivisions}%
	\footnote{This research was funded by the Natural Sciences and 
	Engineering Research Council of Canada.}}

\vspace{24pt}

\smalllineskip
{\authorfont \MakeUppercase{Prosenjit Bose}}

\vspace{2pt}

\addressfont{School of Computer Science, Carleton University 
	Ottawa, K1S 5B6, Canada}

\vspace{10pt}
{\authorfont \MakeUppercase{Andrej Brodnik}}

\vspace{2pt}

\addressfont{IMFM, University of Ljubljana, Ljubljana, Slovenia and Department of Computer Science, Lule\aa\ Technical University, SE-971 87 Lule\aa, Sweden}

\vspace{10pt}
{\authorfont \MakeUppercase{Svante Carlsson}}

\vspace{2pt}

\addressfont{University of Karlskona/Ronneby, 371 41 KARLSKRONA, Sweden}

\vspace{10pt}
{\authorfont \MakeUppercase{Erik D. Demaine}}

\vspace{2pt}

\addressfont{Massachusetts Inst. Tech., Lab. for Computer Science
Cambridge, MA, 02139, USA}

\vspace{10pt}
{\authorfont \MakeUppercase{Rudolf Fleischer}}

\vspace{2pt}

\addressfont{Department of Computer Science
The Hong Kong University of Science and Technology, Kowloon
Hong Kong}

\vspace{10pt}
{\authorfont \MakeUppercase{Alejandro L\'opez-Ortiz}}

\vspace{2pt}

\addressfont{Department of Computer Science, University of Waterloo, Waterloo, \mbox{N2L 3G1}, Canada}

\vspace{10pt}
{\authorfont \MakeUppercase{Pat Morin}}

\vspace{2pt}

\addressfont{School of Computer Science, Carleton University
	Ottawa, K1S 5B6, Canada}

\vspace{10pt}

and

\vspace{10pt}

{\authorfont \MakeUppercase{J. Ian Munro}}

\vspace{2pt}

\addressfont{Department of Computer Science, University of Waterloo, Waterloo, \mbox{N2L 3G1}, Canada }

\vspace{10pt}

\vspace{20pt}
%% authors need not care about this
\publisher{(received date)}{(revised date)}{Editor's name}
\end{center}

\alphfootnote


\begin{abstract}
We consider online routing algorithms for finding paths between the
 vertices of plane graphs.  We show (1)~there exists a routing
algorithm for arbitrary triangulations that has no memory and uses
no randomization, (2)~no equivalent result is possible for convex
subdivisions, (3)~there is no competitive online routing algorithm
under the Euclidean distance metric in arbitrary triangulations, and
(4)~there is no competitive online routing algorithm under the link
distance metric even when the input graph is restricted to be a Delaunay,
greedy, or minimum-weight triangulation.
\end{abstract}

%========================================================================
\textlineskip
\section{Introduction}

Path finding, or routing, is central to a number of fields including
geographic information systems, urban planning, robotics, and
communication networks.  In many cases, knowledge about the
environment in which routing takes place is not available beforehand,
and the vehicle/robot/packet must learn this information through
exploration.  Algorithms for routing in these types of environments
are referred to as \emph{online} \cite{be98} routing algorithms.

In this paper we consider online routing in the following abstract
setting \cite{bm99a}: The environment is a plane graph, $G$ (i.e., the
planar embedding of $G$) with $n$ vertices. The source $\vsrc$ and
destination $\vdest$ are vertices of $G$, and a packet can only travel
on edges of $G$.  Initially, a packet only knows the coordinates of
$\vsrc$, $\vdest$, and $\N(\vsrc)$, where $\N(v)$ denotes the set of
vertices adjacent to a node $v$.  When a packet visits a node $v$, it learns
the coordinates of $N(v)$.

\mbox{Bose} and \mbox{Morin} \cite{bm99a} classify routing algorithms
based on their use of memory and/or randomization.  A deterministic
routing algorithm is \emph{memoryless} or \emph{oblivious} if, given a
packet currently at vertex $v$ and destined for node $\vdest$, the
algorithm decides where to forward the packet based only on the
coordinates of $v$, $\vdest$ and $N(v)$. A randomized algorithm is
oblivious if it decides where to move a packet based only on the
coordinates of $v$, $\vdest$, $N(v)$, and the output of a random
oracle.  An algorithm $\mathcal{A}$ is \emph{defeated} by a graph $G$
if there exists a pair of vertices $\vsrc,\vdest\in G$ such that a
packet stored at $\vsrc$ will never reach $\vdest$ when being routed
using $\mathcal{A}$.  Otherwise, we say that $\mathcal{A}$
\emph{works} for $G$.

Let $\mathcal{A}(G,\vsrc,\vdest)$ denote the length of the walk taken
by routing algorithm $\mathcal{A}$ when travelling from vertex $\vsrc$
to vertex $\vdest$ of $G$, and let $\mathit{SP}(G,\vsrc,\vdest)$
denote the length of the shortest path between $\vsrc$ and $\vdest$.
We say that $\mathcal{A}$ is \emph{$c$-competitive} for a class of
graphs $\mathcal{G}$ if
\[
	\frac{\mathcal{A}(G,\vsrc,\vdest)}{\mathit{SP}(G,\vsrc,\vdest)} \le c
\]
for all graphs $G\in\mathcal{G}$ and all $\vsrc,\vdest\in G$,
$\vsrc\neq\vdest$.  We say that $\mathcal{A}$ is simply
\emph{competitive} if $\mathcal{A}$ is $c$-competitive for some
constant $c$.

Recently, several papers have dealt with online routing and related
problems in geometric settings.  Kalyanasundaram and Pruhs
\cite{KaPr94}\ give a 16-competitive algorithm to \emph{explore} any
unknown plane graph, i.e., visit all of its nodes.  This online
exploration problem makes the same assumptions as those made here, but
the goal of the problem is to visit all vertices of $G$, not just
$\vdest$.  This difference leads to inherently different solutions.

Kranakis \etal\ \cite{ksu99} give a deterministic oblivious routing
algorithm that works for any Delaunay triangulation, and give a
deterministic non-oblivious algorithm that works for any connected
plane graph.

Bose and Morin \cite{bm99a} also study online routing in geometric
settings, particularly triangulations.  They give a randomized
oblivious routing algorithm that works for any triangulation, and ask
whether there is a deterministic oblivious routing algorithm for all
triangulations.  They also give a competitive non-oblivious routing
algorithm for Delaunay triangulations.

Cucka et al.~\cite{CuNeRo96}\ experimentally evaluate the performance
of routing algorithms very similar to those described by Kranakis
\etal\ \cite{ksu99} and Bose and Morin \cite{bm99a}.  When considering
the Euclidean distance travelled during point-to-point routing, their
results show that the \greedy\ routing algorithm \cite{bm99a} performs
better than the \compass\ routing algorithm \cite{bm99a,ksu99} on
random graphs, but does not do as well on Delaunay triangulations of
random point sets.\footnote{Cucka \etal\ call these algorithms
\textsc{p-dfs} and \textsc{d-dfs}, respectively.}  However, when one
considers not the Euclidean distance, but the number of edges
traversed (link distance), then the \compass\ routing algorithm is
slightly more efficient for both random graphs and Delaunay
triangulations.

In this paper we present a number of new fundamental theoretical
results that help further the understanding of online routing in plane
graphs.

\begin{enumerate}
\item We give a deterministic oblivious routing algorithm for all
triangulations, solving the open problem posed by Bose and Morin
\cite{bm99a}.

\item We prove that no deterministic oblivious routing algorithm
works for all convex subdivisions, showing some limitations of
deterministic oblivious routing algorithms.

\item We prove that the randomized oblivious routing algorithm \rc\
described by Bose and Morin \cite{bm99a} works for any convex
subdivision.

\item We show that, under the Euclidean metric, no routing algorithm
exists that is competitive for all triangulations, and under the link
distance metric, no routing algorithm exists that is competitive for
all Delaunay, greedy, or minimum-weight triangulations.
\end{enumerate}

The remainder of the paper is organized as follows: In
\secref{triangulations} we give our deterministic oblivious
algorithm for routing in triangulations.  \secref{convex} presents our
results for routing in convex subdivisions.  \secref{competitive}
describes our impossibility results for competitive
algorithms. Finally, \secref{conclusions} summarizes and concludes
with open problems.

%========================================================================
\section{Oblivious Routing in Triangulations}\seclabel{triangulations}

A \emph{triangulation} $T$ is a plane graph for which every face is a
triangle, except the outer face, which is the complement of a convex
polygon.  In this section we describe a deterministic oblivious
routing algorithm that works for all triangulations.  \erik{The algorithm is
a carefully designed combination of two existing algorithms
\cite{bm99a}.  The \greedy\ algorithm always moves a packet to a
neighbouring node that minimizes the distance to $\vdest$.  The
\compass\ algorithm always moves a packet to the node that is most
``inline'' with $\vdest$.  Both these algorithms are defeated by
certain types of triangulations, but the ways in which they are
defeated are very different.  By combining them, we obtain an algorithm
that works for any triangulation.
}

We use the notation $\angle a,b,c$ to denote the angle formed by $a$
$b$ and $c$ as measured in the counterclockwise direction.  Let
$\cw(v)$ be the vertex in $N(v)$ which minimizes the angle $\angle
cw(v),v,\vdest$ and let $\ccw(v)$ be the vertex in $N(v)$ which
minimizes the angle $\angle\vdest,v,\ccw(v)$.  If $v$ has a neighbour
$w$ on the line segment $(v,\vdest)$, then $\cw(v)=\ccw(v)=w$.  In
particular, the vertex $t$ is contained in the wedge
$\cw(v),v,\ccw(v)$.  Refer to \figref{cw-ccw} for an illustration.

\begin{figure}
\centeripe{cw-ccw.ipe}
\caption{Definition of $\cw(v)$ and $\ccw(v)$.}
\figlabel{cw-ccw}
\end{figure}

The \gc\ algorithm always moves to the vertex among
$\{\cw(v),\ccw(v)\}$ that minimizes the distance to $\vdest$.  If the
two distances are equal, or if $\cw(v)=\ccw(v)$, then \gc\ chooses one of
$\{\cw(v),\ccw(v)\}$ arbitrarily.

\begin{thm}\thmlabel{gc}
Algorithm \gc\ works for any triangulation.
\end{thm}

\proof{
Suppose, by way of contradiction that a triangulation $T$ and a pair
of vertices $\vsrc$ and $\vdest$ exist such that \gc\ does not find a
path from $\vsrc$ to $\vdest$.

In this case there must be a cycle of vertices $C=\langle
v_0,\ldots,v_{k-1}\rangle $ of $T$ such that \gc\ moves from $v_i$ to
$v_{i+1}$ for all $0\le i\le k$, i.e., \gc\ gets trapped cycling
through the vertices of $C$ (see also Lemma~1 of
\cite{bm99a}).\footnote{Here, and in the remainder of this proof, all
subscripts are taken $\bmod\ k$.}  Furthermore, it follows from Lemma~2
of \cite{bm99a} that the destination $\vdest$ is contained in the
interior of $C$.

\begin{clm}
All vertices of $C$ must lie on the boundary of a disk $D$ centered at
$\vdest$.
\end{clm}

\proof{(of claim)
Suppose, by way of contradiction, that there is no such disk $D$.
Then let $D$ be the disk centered at $\vdest$ and having the furthest
vertex of $C$ from $\vdest$ on its boundary.  Consider a vertex $v_i$
in the interior of $D$ such that $v_{i+1}$ is on the boundary of
$D$. (Refer to \figref{gc-proof}.)  Assume, w.l.o.g., that
$v_{i+1}=\ccw(v_i)$.  Then it must be that $\cw(v_i)$ is not in the
interior of $D$, otherwise \gc\ would not have moved to $v_{i+1}$.
But then the edge $(\cw(v_i),\ccw(v_i))$ cuts $D$ into two regions,
$R_1$ containing $v_i$ and $R_2$ containing $\vdest$.  Since $C$
passes through both $R_1$ and $R_2$ and is contained in $D$ then it
must be that $C$ enters region $R_1$ at $\cw(v_i)$ and leaves $R_1$ at
$v_{i+1}=\ccw(v_i)$.  However, this cannot happen because both
$\cw(\cw(v_i))$ and $\ccw(\cw(v_i))$ are contained in the halfspace
bounded by the supporting line of $(\cw(v_i),\ccw(v_i))$ and
containing $\vdest$, and are therefore not contained in $R_1$.
\begin{figure}
\centeripe{gc-proof.ipe}
\caption{The proof of \thmref{gc}.}
\figlabel{gc-proof}
\end{figure}
}

Thus, we have established that all vertices of $C$ are on the boundary
of $D$.  However, since $C$ contains $\vdest$ in its interior and the
triangulation $T$ is connected, it must be that for some vertex $v_j$
of $C$, $\cw(v_j)$ or $\ccw(v_j)$ is in the interior of $D$.  Suppose
that it is $\cw(v_j)$.  But then we have a contradiction, since the
\gc\ algorithm would have gone to $\cw(v_j)$ rather than $v_{j+1}$.
}

%========================================================================
\section{Oblivious Routing in Convex Subdivisions}\seclabel{convex}

A \emph{convex subdivision} is an embedded plane graph such that each
face of the graph is a convex polygon, except the outer face which is
the complement of a convex polygon.  Triangulations are a special case
of convex subdivisions in which each face is a triangle; thus it is
natural to ask whether the \gc\ algorithm can be generalized to convex
subdivisions.  In this section, we show that there is no deterministic
oblivious routing algorithm for convex subdivisions.  However, there
is a randomized oblivious routing algorithm that uses only one random
bit per step.

\subsection{Deterministic Algorithms}

\begin{thm}\thmlabel{no-convex}
Every deterministic oblivious routing algorithm is defeated by some
convex subdivision.
\end{thm}

\proof{
We exhibit a finite collection of convex subdivisions such that any
deterministic oblivious routing algorithm is defeated by at least one
of them.

There are 17 vertices that are common to all of our subdivisions. The
destination vertex $\vdest$ is located at the origin.  The other 16
vertices $V=\{v_0,\ldots,v_{15}\}$ are the vertices of a regular
16-gon centered at the origin and listed in counterclockwise
order.\footnote{In the remainder of this proof, all subscripts are
implicitly taken $\bmod\ 16$.}  In all our subdivisions, the
even-numbered vertices $v_0,v_2,\ldots,v_{14}$ have degree 2.  The
degree of the other vertices varies.  All of our subdivisions contain the
edges of the regular 16-gon.

Assume, by way of contradiction, that there exists a routing algorithm
$\mathcal{A}$ that works for any convex subdivision.  Since the
even-numbered vertices in our subdivisions always have the same two
neighbours in all subdivisions, $\mathcal{A}$ always makes the same
decision at a particular even-numbered vertex.  Thus, it makes sense
to ask what $\mathcal{A}$ does when it visits an even-numbered vertex,
without knowing anything else about the particular subdivision that
$\mathcal{A}$ is routing on.

For each vertex $v_i \in V$, we color $v_i$ black or white depending
on the action of $\mathcal{A}$ upon visiting $v_i$, specifically,
black for moving counterclockwise and white for moving clockwise
around the regular 16-gon.  We claim that all even-numbered vertices
in $V$ must have the same color.  If not, then there exists two
vertices $v_i$ and $v_{i+2}$ such that $v_i$ is black and $v_{i+2}$ is
white.  Then, if we take $\vsrc=v_i$ in the convex subdivision shown
in \figref{no-convex}.a, the algorithm becomes trapped on one of the
edges $(v_i,v_{i+1})$ or $(v_{i+1},v_{i+2})$ and never reaches the
destination $\vdest$, contradicting the assumption that $\mathcal{A}$
works for any convex subdivision.


\begin{figure}
\begin{center}\begin{tabular}{cc}
\Ipe{small-blackwhite.ipe} &
\Ipe{small-single.ipe} \\ 
(a) & (b) \\
\multicolumn{2}{c}{\Ipe{small-cyclic.ipe}} \\
\multicolumn{2}{c}{(c)} 
\end{tabular}\end{center}
\caption{The proof of \thmref{no-convex}.}
\figlabel{no-convex}
\end{figure}


Therefore, assume w.l.o.g.~that all even-numbered vertices of $V$ are
black, and consider the convex subdivision shown in \figref{no-convex}.b.
From this figure it is clear that, if we take $\vsrc=v_1$,
$\mathcal{A}$ cannot visit $x$ after $v_1$, since then it gets
trapped among the vertices $\{v_{12},v_{13},v_{14},v_{15},v_0,v_1,x\}$
and never reaches $\vdest$.

Note that we can rotate \figref{no-convex}.b by integral multiples of
$\pi/4$ while leaving the vertex labels in place and make similar
arguments for $v_3$, $v_5$, $v_7$, $v_9$, $v_{11}$, $v_{13}$ and
$v_{15}$.  However, this implies that $\mathcal{A}$ is defeated by the
convex subdivision shown in \figref{no-convex}.c since if it begins at
any vertex of the regular 16-gon, it never enters the interior of the
16-gon.  We conclude that no oblivious online routing algorithm works
for all convex subdivisions.
}

We note that, although our proof uses subdivisions in which some of
the faces are not strictly convex (i.e., have vertices with interior
angle $\pi$), it is possible to modify the proof to use only strictly
convex subdivisions, but doing so leads to more cluttered diagrams.
\erik{These diagrams are shown in \figref{strictly-convex}. We leave
the details to the interested reader.
\begin{figure}
\begin{center}\begin{tabular}{cc}
\IpeFit{1.7in}\Ipe{big-bw.ipe} &
\IpeFit{1.7in}\Ipe{big-single.ipe} \\ 
\multicolumn{2}{c}{\IpeFit{1.7in}\Ipe{big-cyclic.ipe}} \\
\end{tabular}\end{center}
\caption{Strictly convex subdivisions that can be used in the proof of
\thmref{no-convex}.}
\figlabel{strictly-convex}
\end{figure}
}

\subsection{Randomized Algorithms}

Bose and Morin \cite{bm99a} describe the \rc\ algorithm and show that
it works for any triangulation. For a packet stored at node $v$, the
\rc\ algorithm selects a vertex from $\{\cw(v),\ccw(v)\}$ uniformly at
random and moves to it.  In this section we show that \rc\ works for
any convex subdivision.

Although it is well known that a random walk on any graph $G$ will
eventually visit all vertices of $G$, the \rc\ algorithm has two
advantages over a random walk.  The first advantage is that the \rc\
algorithm is more efficient in its use of randomization than a random
walk.  It requires only one random bit per step, whereas a random walk
requires $\log k$ random bits for a vertex of degree $k$.  The second
advantage is that the \rc\ algorithm makes use of geometry to guide
it, and the result is that \rc\ generally arrives at $\vdest$ much
more quickly than a random walk.  Nevertheless, it can be helpful to
think of \rc\ as a random walk on a directed graph in which every node
has out-degree 1 or 2 except for $\vdest$ which is a sink.

Before we can make statements about which graphs defeat \rc, we must
define what it means for a graph to defeat a randomized algorithm.  We
say that a graph $G$ defeats a (randomized) routing algorithm if there
exists a pair of vertices $\vsrc$ and $\vdest$ of $G$ such that a
packet originating at $\vsrc$ with destination $\vdest$ has
probability 0 of reaching $\vdest$ in any finite number of steps.
Note that, for oblivious algorithms, proving that a graph does not
defeat an algorithm implies that the algorithm will reach its
destination with probability 1.

\begin{thm}
Algorithm \rc\ works for any convex subdivision.
\end{thm}

\proof{
Assume, by way of contradiction, that there is a convex subdivision
$G$ with two vertices $\vsrc$ and $\vdest$ such that the probability
of reaching $\vsrc$ from $\vdest$ using \rc\ is 0.  Then there is a
subgraph $H$ of $G$ containing $\vsrc$, but not containing $\vdest$,
such that for all vertices $v\in H$, $\cw(v)\in H$ and $\ccw(v)\in H$.

The vertex $\vdest$ is contained in some face $f$ of $H$.  We claim
that this face must be convex.  For the sake of contradiction, assume
otherwise.  Then there is a reflex vertex $v$ on the boundary of $f$
such that the line segment $(\vdest,v)$ does not intersect any edge of
$H$.  However, this cannot happen, since $\ccw(v)$ and $\cw(v)$ are
in $H$, and hence $v$ would not be reflex.

Since $G$ is connected, it must be that for some vertex $u$ on the
boundary of $f$, $\cw(u)$ or $\ccw(u)$ is contained in the interior of
$f$.  But this vertex in the interior of $f$ is also in $H$,
contradicting the fact that $f$ is a convex face of $H$.  We conclude
that there is no convex subdivision that defeats \rc.
}


%========================================================================
\section{Competitive Routing Algorithms}\seclabel{competitive}

If we are willing to accept more sophisticated routing algorithms that
make use of memory, then it is sometimes possible to find competitive
routing algorithms.  Bose and Morin \cite{bm99a} give a competitive
algorithm for Delaunay triangulations under the Euclidean distance
metric.  Two questions arise from this: (1)~Can this result be
generalized to arbitrary triangulations? and (2)~Can this result be
duplicated for the link distance metric?  In this section we show
that the answer to both these questions is negative.

\subsection{Euclidean Distance}

In this section we show that, under the Euclidean metric, no
deterministic routing algorithm is $o(\sqrt{n})$-competitive for all
triangulations.  Our proof is a modification of that used by
Papadimitriou and Yannakakis \cite{PaYa91}\ to show that no online
algorithm for finding a destination point among $n$ axis-oriented
rectangular obstacles in the plane is $o(\sqrt{n})$-competitive.

\begin{thm}
Under the Euclidean distance metric, no deterministic routing
algorithm is $o(\sqrt{n})$ competitive for all triangulations.
\end{thm}

\begin{figure}
\begin{center}\begin{tabular}{c}
\Ipe{no-eucl-proof-a.ipe} \\
(a) \\[2ex]
\Ipe{no-eucl-proof-b.ipe} \\
(b)
\end{tabular}\end{center}
\caption{(a)~The triangulation $T$ with the path found by
$\mathcal{A}$ indicated. (b)~The resulting triangulation $T'$ with
the ``almost-vertical'' path shown in bold.}  \figlabel{lattice}
\end{figure}

\proof{
Consider an $n\times n$ hexagonal lattice with the following
modifications.  The lattice has had its $x$-coordinates scaled so that
each edge is of length $\Theta(n)$.  The lattice also has two
additional vertices, $\vsrc$ and $\vdest$, centered horizontally, at
one unit below the bottom row and one unit above the top row,
respectively.  Finally, all vertices of the lattice and $\vsrc$ and
$\vdest$ have been completed to a triangulation $T$.  See
\figref{lattice}.a for an illustration.

Let $\mathcal{A}$ be any deterministic routing algorithm and observe
the actions of $\mathcal{A}$ as it routes from $\vsrc$ to $\vdest$.
In particular, consider the first $n+1$ steps taken by $\mathcal{A}$
as it routes from $\vsrc$ to $\vdest$.  Then $\mathcal{A}$ visits at
most $n+1$ vertices of $T$, and these vertices induce a subgraph
$T_\mathrm{vis}$ consisting of all vertices visited by $\mathcal{A}$
and all edges adjacent to these vertices.

For any vertex $v$ of $T$ not equal to $\vsrc$ or $\vdest$, define the
\emph{$x$-span} of $v$ as the interval between the rightmost and
leftmost $x$-coordinate of $N(v)$.  The length of any $x$-span is
$\Theta(n)$, and the width of the original triangulation $T$ is
$\Theta(n^2)$.  This implies that there is some vertex $v_b$ on the
bottom row of $T$ whose $x$-coordinate is at most $n\sqrt{n}$ from the
$x$-coordinate of $\vsrc$ and is contained in $O(\sqrt{n})$ $x$-spans
of the vertices visited in the first $n+1$ steps of $\mathcal{A}$.

We now create the triangulation $T'$ that contains all vertices and
edges of $T_\mathrm{vis}$.  Additionally, $T'$ contains the set of
edges forming an ``almost vertical'' path from $v_b$ to the top row of
$T'$.  This almost vertical path is a path that is vertical wherever
possible, but uses minimal detours to avoid edges of $T_\mathrm{vis}$.
Since only $O(\sqrt n)$ detours are required, the length of this path
is $O(n\sqrt n)$.  Finally, we complete $T'$ to a triangulation in
some arbitrary way that does not increase the degrees of vertices on
the first $n+1$ steps of $\mathcal{A}$.  See \figref{lattice}.b for an
example.

Now, since $\mathcal{A}$ is deterministic, the first $n+1$ steps taken
by $\mathcal{A}$ on $T'$ will be the same as the first $n+1$ steps
taken by $\mathcal{A}$ on $T$, and will therefore travel a distance of
$\Theta(n^2)$.  However, there is a path in $T'$ from $\vsrc$ to
$\vdest$ that first visits $v_b$ (at a cost of $O(n\sqrt n)$), then
uses the ``almost-vertical'' path to the top row of $T'$ (at a cost of
$O(n\sqrt n)$) and then travels directly to $\vdest$ (at a cost of
$O(n\sqrt n)$).  Thus, the total cost of this path, and hence the
shortest path, from $\vsrc$ to $\vdest$ is $O(n\sqrt n)$.

We conclude that $\mathcal{A}$ is not $o(\sqrt{n})$-competitive for
$T'$.  Since the choice of $\mathcal{A}$ is arbitary, and $T'$
contains $O(n)$ vertices, this implies that no deterministic routing
algorithm is $o(\sqrt{n})$ competitive for all triangulations with $n$
vertices.
}




\comment{
 of width at least one
in the plane can be better than
$\sqrt{n}$-competitive, where $n$ is the distance between
the start and destination positions.
We simply observe that their lower bound construction 
can be transformed into our setting of triangulated planar graphs.
Note that the theorem even holds for non-oblivious online algorithms.


\begin{thm}\thmlabel{nocomp}
There is no online routing algorithm (even with unbounded memory)
on triangulated graphs with bounded competitive ratio.
\end{thm}

\proof{
Let $n$ be an arbitrary large integer.
Assume the start position is at $(0,0)$, and the
destination is at $(n,0)$.
While traversing the graph, the algorithm always sees the
same picture as in \figref{fig_nocomp}. 

\begin{figure}
\begin{center}
\input{fig_nocomp.tex}
\caption{All nodes look the same.}
\figlabel{fig_nocomp}
\end{center}
\end{figure}

Let us assume it visits $k$ nodes before reaching any node
with x-coordinate $n$.
Then the rectangle with vertical sides at x-coordinate $0$ and $n$,
respectively, and horizontal sides at y-coordinate
$m$ and $-m$ where $m=\sqrt{k}\cdot n$,
respectively, contains much 'empty' space,
i.e., space which the algorithm has never seen.
We can then extend the explored triangulated part of the graph
such that this empty space can be traversed directly
from left to right.
In particular, there must be one y-coordinate which needs only
${2kn\over m}$ many detours (of length at most $n$ each),
so the shortest path to the destination has length at most
$2m + n + {2kn\over m}\cdot n$.
On the other hand, the algorithm has traveled a distance of
$\Omega(k\cdot n)$.
This gives a competitive ratio of $O(\sqrt{n})$.
Note that the graph constructed will have $O(n)$ many nodes.
}
}

\subsection{Link Distance}

The link distance metric simply measures the number of edges traversed
by a routing algorithm.  For many networking applications, this metric
is more meaningful than Euclidean distance.  In this section we show
that competitive algorithms under the link distance metric are harder
to come by than under the Euclidean distance metric.  Throughout this
section we assume that the reader is familiar with the definitions of
Delaunay, greedy and minimum-weight triangulations (\emph{cf}.~Preparata and
Shamos \cite{ps85}).

We obtain this result by constructing a ``bad'' family of point sets
as follows.  Let $C_i$ be the set of $\sqrt{n}$ points
$\{(i\sqrt{n},1),(i\sqrt{n},2),\ldots,(i\sqrt{n},\sqrt{n})\}$.  We
call $C_i$ the {\em $i$th column.}  Let
$D_i=\{(i\sqrt{n},1),(i\sqrt{n},\sqrt{n})\}$, and define a family of
point sets $S=\bigcup_{j=1}^\infty \{S_{j^2}\}$ where
$S_n=\{S_{n,1},\ldots,S_{n,\sqrt{n}}\}$ and
\begin{equation}
S_{n,i} = \bigcup_{j=1}^{i-1}C_j \cup D_i \cup \bigcup_{j=i+1}^{\sqrt{n}}C_j
	\cup \{(\sqrt{n}/2, 0), (\sqrt{n}/2,\sqrt{n}+1) \} 
\end{equation}
Two members of the set $S_{49}$ are shown in \figref{no-link-proof}.

\begin{figure}
\begin{center}\begin{tabular}{c@{\hspace{.5in}}c}
\Ipe{no-link-proof-b.ipe} \\
(a) \\[1cm]
\Ipe{no-link-proof-a.ipe} \\ 
(b)
\end{tabular}\end{center}
\caption{The point sets (a)~$S_{49,2}$ and (b)~$S_{49,5}$ along with
	their Delaunay triangulations.}
\figlabel{no-link-proof}
\end{figure}


\begin{thm}\thmlabel{no-link-competitive}
Under the link distance metric, no routing algorithm is
$o(\sqrt{n})$-competitive for all Delaunay
triangulations.\index{Delaunay triangulation}
\end{thm}

\proof{
We use the notation $\DT(S_{n,i})$ to denote the Delaunay
triangulation of $S_{n,i}$.  Although the Delaunay triangulation of
$S_{n,i}$ is not unique, we will assume $\DT(S_{n,i})$ is triangulated
as in \figref{no-link-proof}.  Note that, in $\DT(S_{n,i})$, the
shortest path between the topmost vertex $\vsrc$ and bottom-most
vertex $\vdest$ is of length 3, independent of $n$ and $i$.
Furthermore, any path from $\vsrc$ to $\vdest$ whose length is less
than $\sqrt{n}$ must visit vertices from one of the columns $C_{i-1}$,
$C_{i}$, or $C_{i+1}$.

The rest of the proof is based on the following observation: If we
choose an element $i$ uniformly at random from $\{1,\ldots,\sqrt{n}\}$, then
the probability that a routing algorithm $\mathcal{A}$ has visited a
vertex of $C_{i-1}$, $C_{i}$, or $C_{i+1}$ after $k$ steps is at most
$3k/\sqrt{n}$.  Letting $k=\sqrt{n}/6$, we see that the probability
that $\mathcal{A}$ visits a vertex of $C_{i-1}$, $C_{i}$, or $C_{i+1}$
after $\sqrt{n}/6$ steps is at most $1/2$.

Letting $d_i$ denote the (expected, in the case of randomized
algorithms) number of steps when routing from $\vsrc$ to $\vdest$ in
$S_{n,i}$ using routing algorithm $\mathcal{A}$, we have
\begin{equation}
\frac{1}{\sqrt{n}}\cdot\sum_{i=1}^{\sqrt{n}}d_i \ge \sqrt{n}/{12} \enspace .
\end{equation}
Since, for any $S_{n,i}$, the shortest path from $\vsrc$ to $\vdest$
is 3 there must be some $i$ for which the competitive ratio of
$\mathcal{A}$ for $S_{n,i}$ is at least
$\sqrt{n}/36\in\Omega(\sqrt{n})$.
}

\begin{thm}
Under the link distance metric, no routing algorithm is
$o(\sqrt{n})$-competitive for all greedy triangulations.\index{greedy
triangulation}
\end{thm}

\proof{
This follows immediately from the observation that for any $S_{n,i}$,
a Delaunay triangulation of $S_{n,i}$ is also a greedy triangulation
of $S_{n,i}$.
}


\begin{thm}
Under the link distance metric, no routing algorithm is
$o(\sqrt{n})$-competitive for all minimum-weight
triangulations.\index{minimum-weight triangulation}
\end{thm}

\proof{
We claim that for members of $S$, any greedy triangulation is also a
minimum-weight triangulation.  To prove this, we use a result on
minimum-weight triangulations due to Aichholzer \etal\ \cite{aack96}.
Let $K_{n,i}$ be the complete graph on $S_{n,i}$.  Then an edge $e$ of
$K_{n,i}$ is said to be a {\em light edge}\index{light edge} if every
edge of $K_{n,i}$ that crosses $e$ is not shorter than $e$.
Aichholzer \etal\ prove that if the set of light edges contains the
edges of a triangulation then that triangulation is a minimum-weight
triangulation.

There are only 5 different types of edges in the greedy triangulation
of $S_{n,i}$;  (1)~vertical edges within a column, (2)~horizonal edges
between adjacent columns, (3)~diagonal edges between adjacent columns,
(4)~edges used to triangulate column $i$, and (5)~edges used to join
$\vsrc$ and $\vdest$ to the rest of the graph.  It is straightforward
to verify that all of these types of edges are indeed light edges.
}

%========================================================================
\section{Conclusions}\seclabel{conclusions}

We have presented a number of results concerning online routing in
plane graphs.  \tabref{summary} summarizes what is currently known
about online routing in plane graphs.  An arrow in a reference
indicates that the result is implied by the more general result
pointed to by the arrow.  An F indicates that the result is trivial
and/or folklore.

\begin{table}
\begin{center}\begin{tabular}{l|crcrcrcr}
Class of & \multicolumn{2}{c}{Deterministic} 
	& \multicolumn{2}{c}{Randomized} 
	& \multicolumn{2}{c}{Euclidean}
   	& \multicolumn{2}{c}{Link}        \\
graphs  & \multicolumn{2}{c}{oblivious}     
	& \multicolumn{2}{c}{oblivious\footnotemark}  
	& \multicolumn{2}{c}{competitive} 
	& \multicolumn{2}{c}{competitive} \\ \hline
DT       & Yes & \cite[$\downarrow$]{bm99a,ksu99} & Yes & [$\leftarrow$] & Yes & \cite{bm99a}
	& No & [here] \\
GT/MWT   & Yes & [$\downarrow$] & Yes & [$\downarrow$]
	& Yes & \cite{bm00} & No & [here] \\
Triangulations & Yes & [here] & Yes & \cite[$\leftarrow$]{bm99a} & No & [here] 
	& No & [$\uparrow$] \\
Conv. Subdv. & No & [here] & Yes & [here] & No & [$\uparrow$]
	& No & [$\uparrow$] \\
Plane graphs & No & [F] & No & [F] & No & [F] & No & [F] \\
\end{tabular}\end{center}
\caption{A summary of known results for online routing in plane graphs.}
\tablabel{summary}
\end{table}

\footnotetext{In this column, we consider only algorithms that use a
constant number of random bits per step.  Otherwise, it is well known
that a random walk on any graph $G$ will eventually visit all vertices
of $G$.}

We have also implemented a simulation of the \gc\ algorithm as well as
the algorithms described by Bose and Morin \cite{bm99a} and compared
them under the Euclidean distance metric. These results will be
presented in the full version of the paper.  Here we only summarize
our main observations.

For Delaunay triangulations of random point sets, we found that the
performance of \gc\ is comparable to that of the \compass\ and
\greedy\ algorithms \cite{bm99a,CuNeRo96,ksu99}.  For triangulations
obtained by performing Graham's scan \cite{g72} on random point sets,
the \gc\ algorithm does significantly better than the \compass\ or
\greedy\ algorithms.

We also implemented a variant of \gc\ that we call \gct\ that, when
located at a vertex $v$, moves to the vertex $u\in \{\cw(v),\ccw(v)\}$
that minimizes $d(v,u)+d(u,\vdest)$, where $d(a,b)$ denotes the
Euclidean distance between $a$ and $b$.  Although there are
triangulations that defeat \gct, it worked for all our test
triangulations, and in fact seems to be twice as efficient as \gc\ in
terms of the Euclidean distance travelled.

We note that currently, under the link distance metric, there are no
competitive routing algorithms for any interesting class of geometric
graphs (meshes do not count). The reason for this seems to be that the
properties used in defining many geometric graphs make use of
properties of Euclidean space, and link distance in these graphs is
often independent of these properties.  We consider it an open problem
to find competitive algorithms, under the link distance metric, for an
interesting and naturally occuring class of geometric graphs.

\section{Acknowledgements}

This work was initiated at Schloss Dagstuhl Seminar on Data
Structures, held in Wadern, Germany, February--March 2000, and
co-organized by \mbox{Susanne Albers}, \mbox{Ian Munro}, and
\mbox{Peter Widmayer}.  The authors would also like to thank
\mbox{Jorge Urrutia} and \mbox{Lars Jacobsen} for helpful discussions.

\bibliographystyle{plain}
\bibliography{online}

\end{document}






