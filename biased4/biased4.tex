\documentclass{patmorin}

\usepackage{pat,amsopn}
\DeclareMathOperator{\scc}{succ}
\DeclareMathOperator{\interior}{int}
\DeclareMathOperator{\lft}{left}
\DeclareMathOperator{\rght}{right}

\title{\MakeUppercase{Biased Range Trees: 4-Sided Queries}%
	\thanks{This research was partly supported by NSERC, the Canada
	Foundation for Innovation, the Ontario Innovation Trust, the
	Ontario Ministry of Research and Innovation, the University
	of Sydney, National ICT Australia, and the Carty Memorial
	Foundation.}}
\author{Pat Morin%
	\thanks{\affil{Carleton University}, \email{morin@scs.carleton.ca}.
	This research took place while the author was a visiting researcher
	at the University of Sydney and National ICT Australia.}}


\begin{document}
\maketitle

\begin{abstract}
Let $P$ be a set of $n$ points in $\R^2$ and let $D$ be a probability
measure over the space of all axis-aligned rectangles in $\R^2$.
We describe a data structure that preprocesses $P$ and $D$ to answer
rectangular range counting queries over $P$.  When the queries are drawn
according to $D$, the expected query time of this data structure does
not exceed, by more than a constant factor, the expected cost of any
comparison tree that answers rectangular range counting queries over $P$.
\end{abstract}


\section{Introduction}

Let $P$ be a set of $n$ points in $\R^2$ and let $D$ be a probability
measure over the space of all axis-aligned rectangles in $\R^2$.
We describe a data structure that preprocesses $P$ and $D$ to answer
rectangular range counting queries over $P$.  When the queries are drawn
according to $D$, the expected query time of this data structure does
not exceed, by more than a constant factor, the expected cost of any
comparison tree for answering rectangular range counting queries over $P$
for queries drawn according to $D$.

An interesting aspect of our solution is that it makes use of two
geometric data structuring techniques that are known to be efficient
in practice, but that have poor worst-case behaviour.  The primary
form of search tree used is a $k$-d tree \cite{S}, which requires
$\Theta(\sqrt{n})$ time to answer rectangular range queries \cite{dbXX}.
Furthermore, our solution involves lifting the points of $P$ into a point
set in $\R^4$ and performing 4-sided range-counting queries on this lifted
point set.  While this lifting technique is common in practice it almost
invariably leads to (at least) extra logarithmic factors in query times.
Nevertheless, using these two techniques, we obtain expected query times
that are within a constant factor of the best decision tree for answer
rectangular range queries over $P$ with the queries drawn according
to $D$.

\section{Preliminaries}

\subsection{1-Dimensional Range Counting}

As a subroutine, our data structures are required to solve some
1-dimensional range counting problems.  

Given a set $X$ of $n$ real numbers, the 1-sided (1-d) range counting
problem over $X$ asks us to return the number of elements of $X$ greater
than some query value $x_0$.  When the query value $x_0$ is given by a
probability measure $D$ over $\R$, this problem can be solved optimally
using biased search trees \cite{X,X,X}.  In particular, after $O(n\log n)$
preprocessing, a biased search tree can answer a 1-sided range-counting
query over $X$ in $O(\mu_D(T^*))$ expected time, where $T^*$ is any
comparison tree that answers 2-sided range counting queries over $X$.

The 2-sided (1-d) range counting problem over $X$ asks us to return the
number of elements of $x\in X$ such that $x_0 \le x \le x_1$ for some query
interval $(x_0,x_1)$.  By lifting the set $X$ onto a 2-d point set
$X^2=\{(x,x): x\in X\}$ the 2-sided 1-d range counting problem can be
converted to a 2-sided 2-d range counting problem.  When the query interval
$(x_0,x_1)$ is given by a probability measure $D$ over $\R^2$, such queries
can be answered optimally using the recently-discovered biased range trees
\cite{dhm08}.  In particular, after $O(n\log n)$ preprocessing, a biased
range tree can answer a 2-sided range-counting query over $X$ in
$O(\mu_D(T^*))$ expected time, where $T^*$ is any comparison tree that
answers 2-sided range counting queries over $X$.

\subsection{2-Dimensional Range Counting}

As mentioned above, biased range trees can answer 2-sided 2-d range
counting queries in optimal expected time.  In fact, these data structures
give somewhat more information than just the number of points of $P$ in
the query range.  We say that a data structure \emph{locates} a number
$x_0$ among a set $X$ of numbers if the data structure finds (a pointer
to) the value $\scc_X(x) = \min\{ x\in X\cup \{\infty\}: x \ge x_0 \}$.
For a query point $(x_0,y_0)$, a biased range tree locates the value of
$x_0$ within the set
\[
    P_{\ge y} = \{x: \mbox{$(x,y)\in P$ and $y \ge y_0$}\} 
\]
and locates the value of $y_0$ within the set
\[ 
    P_{\ge x} = \{y : \mbox{$(x,y)\in P$ and $x \ge x_0$}\} \enspace .
\]

\section{3-Sided Queries}

A three sided range query $(x_0,y_0,x_1)$ asks for all points of $S$
contained in
\[
   R(x_0,y_0,x_1) = \{ (x,y)\in\R^2 
                       : \mbox{$x_0\le x\le x_1$ and $y_0 \le y$} \} \enspace .
\]
To answer 3-sided range queries we lift the points of $P$ into $\R^3$
using using the map $(x,y) \mapsto (x,y,x)$ to obtain a set $P^3$
of points in $\R^3$.  Now, we construct a kd-tree $T$ for $P^3$.
We call a node $v$ of $T$ a $x_0$-node, a $y_0$-node, or a $x_1$-node
depending on whether the depth of $v$ is congruent to 0, 1, or 2 modulo
3, respectively.

Each non-leaf node $v$ of $T$ splits $r(v)$ into two parts.  If $v$ is an
$x_0$-node or a $y_0$-node, we denote by $\lft(v)$ the child corresponding to
smaller $x$, respectively, $y$, coordinates. For an $x_1$-node, we denote
by $\lft(v)$, the child corresponding to the larger $z$ coordinate.  For
any non-leaf node $v$, $\rght(v)$ denotes the child of $v$ different from
$\lft(v)$.

Consider a path $v_0,\ldots,v_k$ from the root of $T$ to the leaf $v$ that
contains the query $(x_0,y_0,x_1)\in\interior(r(v))$.  Let $r_k=r(v_k)$
and, for each $i\in\{1,\ldots,k-1\}$, let
\[
  r_i = \left\{\begin{array}{ll}
          r(\rght(v_i)) & \mbox{if $v_{i+1}=\lft(v_i)$} \\
          \emptyset     & \mbox{otherwise}
        \end{array}\right.
\]
Observe that each non-empty $r_i$ intersects the query range
\[
    R=\{ (x,y,z)\in\R^3 
          : \mbox{$x_0 \le x$, $y_0 \le y$, and $x_1 \ge z$} \}
\]
and that the $\interior(r_i)\cap \interior(r_j)=\emptyset$ for any $i\neq
j$.  In particular, $r_1,\ldots,r_{k-1}$ and $r(v_k)$ form a cover of $R$
with boxes whose interiors are disjoint.



\section{4-Sided Queries}

To construct a data structure for 4-sided (rectangular) range queries we
lift the point set $P$ into $\R^4$.  Let 

\section{Summary and Conclusions}

\end{document}
