\documentclass{patmorin}
\usepackage{amsthm,amsmath}
\usepackage{pat}

\title{\MakeUppercase{Notes on Robust Spanners}}
\author{Prosenjit Bose, Vida Dujmovi\'c, Pat Morin, and Michiel Smid}



\begin{document}
\maketitle

\begin{abstract}
We consider a new definition of fault-tolerance for $t$-spanners that
we dub robustness.  A $t$-spanner is robust if the removal of any $k$
vertices leaves $n-f(k)\cdot o(n)$ vertices whose induced subgraph is
a $t$-spanner.
\end{abstract}

\section{Introduction}

For a graph $G=(V,E)$ and a subset $S\subseteq V$ of $G$'s vertices,
$G[S]$ denotes the subraph of $G$ induced by $S$ and $G\setminus
S=G[V\setminus S]$ is the graph obtained by removing the elements of $S$
from $V$.

Let $V\subset \R^d$ be a set of $n$ points in $\R^d$.  A graph $G=(V,E)$
is an \emph{$(f(k),g(n))$-robust $t$-spanner} of $S$ if, for every subset
$S\subseteq V$, there exists a superset $S'\supseteq S$, $|S'|\le f(|S|)$,
such that $G\setminus S'$ is a $t$-spanner.  Intuitively, killing the
vertices in $S$ leaves a small number ($f(|S|)$) of vertices stranded
but most of the graph remains a $t$-spanner.

Robustness is related to, but different from $k$-fault tolerance.
A \emph{$k$-fault-tolerant} $t$-spanner $G=(V,E)$ has the property that
$G\setminus S$ is a $t$-spanner for any subset $S\subseteq V$ of size
at most $k$.  At a minimum, a $k$-fault-tolerant spanner must remain
connected after the removal of any $k$ vertices.  This immediately implies
that any $k$-fault-tolerant spanner with $n$ vertices has $\Omega(kn)$
edges.  Several constructions of $k$-fault-tolerant spanners having
$O(kn)$ edges exist.

\section{One Dimensional Point Sets}

This section gives upper and lower-bounds on the number of edges
in $(f(k),1)$ robust $t$-spanners for 1-dimensional point sets.
We first give a simple construction that is $(O(k\log k),1)$-robust
and a lower-bound on the number of edges in any $t$-spanner that is
$(O(k),1)$-robust.

The arguments used to prove these simple bounds contain the main ideas
needed to prove bounds for a more general class of functions, $f(k)$,
that are subsequently discussed.

\subsection{A Simple Construction}

In this section, we consider the following graph, $G=(V,E)$ which is
closely related to the hypercube.  Let $x_1<x_2<\cdots<x_n$ be the
elements of $V$. The edge set, $E$, of $G$ consists of
\[
    E = \{x_ix_{i+2^j} : j\in\{0,\ldots,\floor{\log n}\},\, i\in\{1,\ldots,n-2^j\} \} \enspace .
\] 
Notice that $G$ is 1-spanner since it contains every every edge of the
form $x_i,x_{i+1}$, for $i\in\{1,\ldots,n-1\}$. Furthermore, $G$ has size
$O(n\log n)$ since every vertex has degree at most $\floor{\log n}+1$.

The following theorem shows that $G$ is also $(O(k^2),1)$-robust.
This theorem is not tight; later we will show that $G$ is in fact
$(O(k\log k),1)$-robust. Nevertheless, the proof of this theorem
can be considered as a warm-up to the proof of the more abstract
\thmref{general-upper-bound-1d}.

\begin{thm}[Simple 1-d Upper Bound]\thmlabel{upper-bound-1d}
Let $V\subset \R$ be any set of $n$ real numbers.  Then there exists an
$(O(k^2),1)$-robust $1$-spanner of $S$ of size $O(n\log n)$.
\end{thm}

\begin{proof}
We will prove that $G$ is a robust $1$-spanner. Consider any
$S\in\binom{V}{k}$, with $k\le \sqrt{n}$.  Let $d=2^{\ceil{\log (k+1)}}$
and observe that $k < d < 2k+2$ and that the edge $x_ix_{i+d}$ is in $G$
for every $i\in\{1,\ldots,n-d\}$.

We say that $x_i,\ldots,x_{i+r}$ form a \emph{good block} if $r\ge
2d-1$ and $S$ contains no element of $x_i,\ldots,x_{i+r}$.  We focus
our attention on the set of maximal good blocks.  Note that the number
of maximal good blocks is at most $k+1$ and that the total number of
vertices contained in these blocks is at least $n-k(2d-1)=n-O(k^2)$.
We will show that there exists a set $S'\supseteq S$ that avoids all but
$2(d-1)$ vertices from each good block, and such that $G\setminus S'$
is a 1-spanner.
Therefore, the size of $S'$ is at most $O(k^2)-2(d-1)(k+1)=O(k^2)$.

Initially our set $S'$ contains all vertices in $S$, and
we will add at most $d-1$ vertices from the beginning and
end of each block.  Between any two consecutive good blocks,
$x_i,\ldots,x_{i+r}$ and $x_j,\ldots,x_{j+s}$, $G$ contains $d$
vertex disjoint paths, $P_1,\ldots,P_{d-1}$ each of which begins
at a vertex in $x_{i+r-d+1},\ldots,x_{i+r}$ and ends at a vertex in
$x_{j},\ldots,x_{j+d-1}$.  (For example, one of these paths consists of
$x_i,x_{i+d},x_{i+2d},\ldots,x_{i+cd}$, for some integer $c$.)

Since $d > k$, at least one path, say $P_1$, does not use any vertices
of $S$.  If $P_1$ begins at $x_k$ and ends at $x_\ell$ then we add
to $S'$ all vertices in 
$x_{k+1},\ldots,x_{\ell-1}$ that are not included in $P_1$.  Observe that adds,
to $S'$, at most $d-1$ of the first $d$ vertices of each block
and at most $d-1$ of the last $d$ vertices of each block.  Therefore,
the number of vertices added to $S'$ from all good blocks is at most
$2(d-1)(k+1)$, as claimed.

Let $x_{i_1},\ldots,x_{i_{n'}}$ denote the vertices in $V\setminus
S'$, ordered from left to right.  None of these vertices are in $S'$,
and for each $j\in\{1,\ldots,n'-1\}$, $i_{j+1}-i_{j}\in\{1,d\}$, so the
edge $x_{i_j}x_{i_{j+1}}$ is present in $G$.  Therefore $G\setminus S'$
is a 1-spanner.
\end{proof}

We now prove a better upper-bound on the robustness of $G$ by using the
probabilistic method.

\begin{thm}[Better 1-d Upper Bound]\thmlabel{better-upper-bound-1d}
Let $S\subset \R$ be any set of $n$ real numbers.  Then there exists an
$(O(k\log k),1)$-robust $1$-spanner of $S$ of size $O(n\log n)$.
\end{thm}

\begin{proof}
As before, consider any set $S\in\binom{V}{k}$.  Let $m=ck\log k$
for some sufficiently large constant $c$.  Without loss of generality,
we can assume that $n\le 2km+k = O(k^2\log k)$, since otherwise there
exists a consecutive sequence $x_i,\ldots,x_{i+q}$ of vertices of length
$q\ge 2m$, none of which are are in $S$.

We can then independently study the subproblems on
of $G_1=G[\{x_0,\ldots,x_{i+m}\}]$
and on
$G_2=G[\{x_{i+q-m},\ldots,x_{n}\}]$.
If $S$ contains $k_1$ vertices of $G_1$ and $k_2$ vertices of $G_2$ then
(by induction on $n$, say) we obtain an induced subgraph of $G\setminus S$
that is a 1-spanner and has
\[ n-ck_1\log k_1 - ck_2\log k_2 \ge n - ck\log k
\]
vertices.

Therefore, we assume $n = O(k^2\log k)$.  Pick a random integer
$r\in\{0,\ldots,2^{\floor{\log n}}\}$ and consider the subgraph, $G'$,
of $G$ consisting only of the edges of the form $(x_i,x_{i+k})$ where
$i-r$ is a multiple of $k$.  (For readers with a background in data
structures, this subgraph looks a lot like a skiplist.)

We say that an element $x_i\in S$ \emph{kills} the vertices
$x_{i-2^{j}+1},\ldots,x_{i+2^{j}-1}$ in $G'$ if $i-r$ is a
multiple of $2^j$.  Notice that the probability that $x_i$ kills
$x_{i-2^{j}+1},\ldots,x_{i+2^{j}-1}$ is at most $1/2^j$.  Therefore,
the expected number of vertices killed by $x$ is at most
\[
    \sum_{j=0}^{\lfloor{\log n}} 2(2^j-1)/2^j \le 2\log n + 2 = O(\log k) \enspace ,
\]
since $n=O(k^2\log k)$.  Finally, notice that if we let the set $S'$
be the set of all vertices killed by elements in $S$, then $G'\setminus
S'$ is a 1-spanner whose expected number of vertices is $n-O(k\log k)$.
Therefore, $G\setminus S$ contains, as a vertex induced subgraph,
a 1-spanner of size $n-O(k\log k)$.  (In particular, some choice of $r$ yields the set $S'$ that demonstrates this.)
\end{proof}



\begin{thm}[Simple Lower Bound]\thmlabel{lower-bound-1d}
Let $V=\{1,\ldots,n\}$ and let $t\ge 1$ be a constant.  Then any
$(O(k),1)$-robust $t$-spanner of $S$ has $\Omega(n\log n)$ edges.
\end{thm}

\begin{proof}
Fix an even natural number $k$ and consider some $i\in\{ck,\ldots,n-ck\}$.
We claim that if $G=(V,E)$ is a $(ck,1)$-robust\footnote{Note that we
have gone from $(O(k),1)$-robust in the statement of the theorem to
$(ck,1)$-robust in the proof.  This does not cause a problem so long
as we only consider values of $k$ greater than some constant $k_0$
hidden in the $O$ notation.} $t$-spanner, then $G$ has at least $k/2$
edges, $xy$, such that $x < i < i+k/2 < y$ and such that $y-x < ctk$.
If this is not the case, then there exists a set $S$, $|S|\le k$,
such that any edges of $G$ with one endpoint to the left of $S$
and one endpoint to the right of $S$ have length at least $ctk$.
A straightforward analysis shows that the existence of such an $S$
implies that $G$ is not $(ck,1)$-robust.

Applying the above argument to $i=ck+2jtck$, for
$j\in\{0,\ldots,\floor{(n-ck)/2tck}\}$ implies that $G$ contains $\Omega(n/tc)$
edges whose length is in the range $[k/2,ctk]$.  Applying this
argument for $k\in\{\ceil{(2tck_0)^{j}} : j\in\{0,\ldots,\floor{(\log
n)/\log(2tck_0)}\}$ proves that, for any constants $c,k_0,t>1$, $G$
has $\Omega(n\log n)$ edges.
\end{proof}

Notice that our upper and lower bounds differ by a factor of $\log k$ in the
function $f(k)$.  We conjecture that the lower bound is the true answer:

\begin{conj}
Let $S\subset \R$ be any set of $n$ real numbers.  Then there exists an
$(O(k),1)$-robust $1$-spanner of $S$ of size $O(n\log n)$.
\end{conj}

\subsection{Generalized Bounds}\seclabel{generalized}

Let $k_0\ge 1$ be a constant and let $f(\cdot)$ be any function that is
weakly convex and increasing over the interval $[k_0,\infty)$, and such
that $f(k_0+1)-f(k_0) > 1$.  Let $f^{i}(k)$ be function $f$ iterated
$i$ times on the initial value $k$, i.e.,
\[
   f^{i}(k) = \underbrace{f(f(f(\cdots f}_{i}(k)\cdots))) \enspace .
\]
Then we define the \emph{iterated
$f$-inverse function}
\[
   f^*(n) = \min\{i : f^{i}(k_0) \ge n\} \enspace .
\] 

\begin{thm}[Generalized Upper Bound]\thmlabel{general-upper-bound-1d}
Let $k_0$, $f$, and $f^*$ be defined as above, and let $S\subset \R$ be
any set of $n$ real numbers.  Then there exists an $(O(kf(k+1)),1)$-robust
$1$-spanner of $S$ of size $O(nf^*(n))$.
\end{thm}

\begin{proof}
The upper bound construction is similar to the graph constructed in the
proof of \thmref{upper-bound-1d}.  It contains the edge set
\[
    E = \{x_ix_{i+f^{j}(k_0)} : j\in\{0,\ldots,f^*(n)-1\} \enspace .
\] 
Given a set $S'\in \binom{S}{k}$, we define $d$ as
\[
   d=\min\{f^i(k_0) : f^i(k_0)> k+1,\, i\in\Z\} \enspace .
\]
Thus, we have the inequalities
\[
     f^{-1}(k+1) < d \le f(k+1) \enspace .
\]
As before, we define a good block as a sequence of $r>2d-1$ consecutive
vertices that do not belong to the same block.  The remainder of the
proof is as before.   There are at most $k+1$ maximal good blocks that
contain a total of at least $n-k(2d-1)$ vertices.  A path can be found
from each block to the next block that requires the elimination of at
most $d=f(k+1)$ vertices from the end of one block and the beginning of
the next.  In total, this leaves a set $\bar{S'}$ of size at least
$n-O(kd) = n - O(kf(k+1))$, as required.
\end{proof}

Taking different functions $f(k)$ in \thmref{general-upper-bound-1d} gives several tradeoffs:

\begin{cor}
Let $S\subset \R$ be any set of $n$ real numbers.  Then there exists an
\begin{itemize}
  \item $(O(k^2),1)$-robust $1$-spanner of $S$ of size $O(n\log n)$
  [$f(k)=2k$];
  \item $(O(k^{2+\varepsilon},1)$-robust $1$-spanner of $S$ of size
  $O(n\log\log n)$ [$f(k)=k^{1+\varepsilon}$];
  \item $(O(k(1+\varepsilon)^k),1)$-robust $1$-spanner of $S$ of size
  $O(n\log^* n)$ [$f(k)=(1+\varepsilon)^k$].
\end{itemize}
\end{cor}


\begin{thm}[Generalized Lower Bound]
Let $k_0$, $f$, and $f^*$ be defined as above, and let $S\subset \R$
be any set of $n$ real numbers.  Let $S=\{1,\ldots,n\}$ and let $t\ge
1$ be a constant.  Then any $(f(k),1)$-robust $t$-spanner of $S$ has
$\Omega(nf^*(n)$ edges.
\end{thm}

\begin{proof}[Proof Sketch]
The proof is similar to the proof of \thmref{lower-bound-1d}.
We need only consider $f(k)=\omega(n)$ since, otherwise we can apply
\thmref{lower-bound-1d}.  We group the edges of the graph $G$ into
$\Omega(f^*(n))$ classes and show that each class contains $\Omega(n)$
vertices.  In this case, the edges in each class have lengths in a
range of the form $[f^i(k_0),tcf^{i}(k_0)]$.  For large enough $i$,
these classes do not overlap since $f(k)=\omega(n)$.
\end{proof}





\section{$d$-Dimensional Point Sets}

\begin{thm}[Generalized Upper Bound in $\R^d$]
Let $k_0$, $f$, and $f^*$ be defined as in \secref{generalized}.
Let $S\subset \R^d$ be any set of $n$ points in $\R^d$.  Then, for any
constant $t>1$,  there exists a $(O(kf(k+1)),1)$-robust $t$-spanner of $S$
with $O(nf^*(n))$ edges.
\end{thm}

\begin{proof} 
This construction makes use of dumbbell tree spanners \cite{X}.
A dumbbell tree spanner of $S$ is a set of $O(1)$ binary trees
$\mathcal{T}=\{T_1,\ldots,T_p\}$.  Each binary tree $T_i$ has $n$ leaves,
one for each vertex of $S$.  Each of $T_i$'s internal nodes also stores
a point of $S$, and no point of $S$ is stored in more than one internal
node.  These trees have the property that, for any two points $x,y\in S$,
there exists some tree $T_i$ such that the path from $x$ to $y$ in $T_i$
has length at most $t\|xy\|$.  Thus, the graph $G=(S,E)$ obtained by
taking the union of all edges in all trees is a $t$-spanner.

Fix a value $k$ and recall that, in any binary tree, $T$, with $n$ nodes,
there exists a vertex whose removal disconnects $T$ into at most 3
components each of size at most $n/2$ \cite{S}.  Repeatedly applying this
yields a set of $O(n/k)$ vertices whose removal disconnects $T$ into
components of size at most $k$.  Perform this decomposition for each of
the trees $T_1,\ldots,T_p$ in a dumbbell tree spanner to obtain a set
of $X$ of $O(n/k)$ vertices whose removal disconnects every tree in $T$
into components each of size at most $k$.

Using any of the $k$-fault tolerant spanner constructions discussed in
the introduction, we can construct a $k$-fault tolerant spanner for $X$
having $O(k|X|)=O(n)$ edges. 

Suppose, now that we are given a set $S'\subset S$ of $k$ vertices.

  Now, observe that for any pair of vertices
$x,y\in S$, there is a tree $T_i$ such that the path from $x$ to $y$ in $T_i$ has length at most $t\|xy\|$.



\end{proof}





\end{document}
