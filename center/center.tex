\documentclass{ws-ijcga}
%\usepackage{charter}
%\usepackage[numbers,sort&compress]{natbib}
\usepackage{graphicx}
%\usepackage{fullpage}
%\usepackage{amsfonts}
%\usepackage{algorithmic}

 
%\usepackage{amsthm}

\newcommand{\centeripe}[1]{\begin{center}\Ipe{#1}\end{center}}
\newcommand{\comment}[1]{}

\newcommand{\centerpsfig}[1]{\centerline{\psfig{#1}}}

\newcommand{\seclabel}[1]{\label{sec:#1}}
\newcommand{\Secref}[1]{Section~\ref{sec:#1}}
\newcommand{\secref}[1]{\mbox{Section~\ref{sec:#1}}}

\newcommand{\alglabel}[1]{\label{alg:#1}}
\newcommand{\Algref}[1]{Algorithm~\ref{alg:#1}}
\newcommand{\algref}[1]{\mbox{Algorithm~\ref{alg:#1}}}

\newcommand{\applabel}[1]{\label{app:#1}}
\newcommand{\Appref}[1]{Appendix~\ref{app:#1}}
\newcommand{\appref}[1]{\mbox{Appendix~\ref{app:#1}}}

\newcommand{\tablabel}[1]{\label{tab:#1}}
\newcommand{\Tabref}[1]{Table~\ref{tab:#1}}
\newcommand{\tabref}[1]{Table~\ref{tab:#1}}

\newcommand{\figlabel}[1]{\label{fig:#1}}
\newcommand{\Figref}[1]{Figure~\ref{fig:#1}}
\newcommand{\figref}[1]{\mbox{Figure~\ref{fig:#1}}}

\newcommand{\eqlabel}[1]{\label{eq:#1}}
\newcommand{\eqref}[1]{(\ref{eq:#1})}

\newtheorem{thm}{Theorem}{\bfseries}{\itshape}
\newcommand{\thmlabel}[1]{\label{thm:#1}}
\newcommand{\thmref}[1]{Theorem~\ref{thm:#1}}

\newtheorem{lem}{Lemma}{\bfseries}{\itshape}
\newcommand{\lemlabel}[1]{\label{lem:#1}}
\newcommand{\lemref}[1]{Lemma~\ref{lem:#1}}

\newtheorem{cor}{Corollary}{\bfseries}{\itshape}
\newcommand{\corlabel}[1]{\label{cor:#1}}
\newcommand{\corref}[1]{Corollary~\ref{cor:#1}}

\newtheorem{obs}{Observation}{\bfseries}{\itshape}
\newcommand{\obslabel}[1]{\label{obs:#1}}
\newcommand{\obsref}[1]{Observation~\ref{obs:#1}}

\newtheorem{assumption}{Assumption}{\bfseries}{\rm}
\newenvironment{ass}{\begin{assumption}\rm}{\end{assumption}}
\newcommand{\asslabel}[1]{\label{ass:#1}}
\newcommand{\assref}[1]{Assumption~\ref{ass:#1}}

\newcommand{\proclabel}[1]{\label{alg:#1}}
\newcommand{\procref}[1]{Procedure~\ref{alg:#1}}

\newtheorem{rem}{Remark}
\newtheorem{op}{Open Problem}

\newcommand{\etal}{\emph{et al}}

\newcommand{\voronoi}{Vorono\u\i}
\newcommand{\ceil}[1]{\left\lceil #1 \right\rceil}
\newcommand{\floor}[1]{\left\lfloor #1 \right\rfloor}



%\setlength{\parskip}{.3cm}
%\setlength{\parindent}{1cm}
\newcommand{\area}{\mathrm{area}}

\begin{document}

\markboth{P. Bra\ss, L. Heinrich-Litan and P. Morin}
{Computing the Center of Area of Convex Polygon}

\catchline

\title{\MakeUppercase{Computing the Center of Area of a Convex Polygon}%
  \thanks{This research was partly funded by Deutsche 
    Forschungsgemeinschaft (DFG), grant BR 1465/5-2, and by
    the IST Programme of the EU as a Shared-cost RTD (FET Open) 
    Project under Contract Number IST-2000-26473 (ECG - Effective 
    Computational Geometry for Curves and Surfaces).}}

\author{Peter Bra\ss}
\address{Department of Computer Science R8/206, City College CUNY,
    138th Street at Convent Ave. \\
    New York NY 10031, USA \\
    peter@cs.ccny.cuny.edu} 

\author{Laura Heinrich-Litan}
\address{Institut f\"ur Informatik, Freie Universit\"at Berlin \\
    D-14195 Berlin, Germany \\ 
    litan@inf.fu-berlin.de}

\author{Pat Morin}
\address{School of Computer Science, Carleton University,
    1125 Colonel By Drive \\ 
    Ottawa, Ontario, K1S~5B6, Canada \\
    morin@cs.carleton.ca}

\maketitle

\pub{Received September 18, 2002}{Revised July 25, 2003}{Communicated by
Pankaj K. Agarwal}

\begin{abstract}
The center of area of a convex planar set $X$ is the point $p$ for
which the minimum area of $X$ intersected by any halfplane containing
$p$ is maximized.  We describe a simple randomized linear-time
algorithm for computing the center of area of a convex $n$-gon.
\end{abstract}

\keywords{Geometric optimization; center of area, Tukey center}

\section{Introduction}
Let $X$ be a convex planar set with unit area.  The \emph{center of
area} of $X$ is a point $p^*$ that maximizes the \emph{cut off area
function}
\[
f(p) = \min\{\area(h\cap X) \mid \mbox{$h$ is a halfplane that contains $p$} \}
	\enspace ,
\]
and the value $\delta^*=f(p^*)$ is known as \emph{Winternitz's measure of
symmetry}.\cite{g63} The $\delta$-level $\Gamma_\delta$ of $X$ is
defined as
\[
\Gamma_\delta = \{p \mid f(p)=\delta  \} \enspace .
\]
It is known that $\Gamma_\delta$ is a closed convex curve and
that $\Gamma_{\delta_1}$ is strictly contained in $\Gamma_{\delta_2}$
if $\delta_1>\delta_2$.  From this it follows that $p^*$ is unique.

There is a long history of work on the center of area of convex sets.
A classical result of Winternitz,\cite[pp.~54--55]{b23} which has
been rediscovered many times,\cite{e55b,ll35,n45,n58,yb51} is
that $f(g)\ge 4/9$ where $g$ is the centroid of $X$, with equality if
and only if $X$ is a triangle. (In $d$ dimensions, Ehrhart\cite{e55a} showed
that $f(g)\ge d^d/(d+1)^d$ with equality if and only if $X$ is a
pyramid on any (d-1)-dimensional convex base.)  For centrally
symmetric sets, $f(g)=1/2$, since any line through the point of
symmetry cuts $X$ into two pieces of equal area.  Thus, $4/9\le
f(g)\le 1/2$ with $f(g)=4/9$ for triangles and $f(g)$ close to $1/2$
for highly symmetric sets.

Although much is known about the center of area, it is quite
nontrivial to determine the center of area for a given convex set.  In
a series of papers, D\'{\i}az and O'Rourke\cite{do89,do91,do94} develop
an $O(n^6\log^2 n)$ time algorithm for finding the center of area of a
convex $n$-gon.  The same authors give an approximation algorithm that
runs in $O(GK(n+K))$ time, where $G$ is the bit-precision of the input
polygon $P$ and $K$ is the output bit-precision of the point $p^*$.
Bra\ss\ and Heinrich-Litan\cite{bh02} describe an $O(n^2\log^3
n\alpha(n))$ time algorithm for computing the center of area of a
convex $n$-gon.  As an application of tools for searching in
arrangments of lines, Langerman and Steiger\cite{ls02} present an
$O(n\log^3 n)$ time algorithm for finding the center of area of a
convex $n$-gon.  All of these algorithms are deterministic.

In this paper we give a simple randomized linear-time algorithm for
finding the center of area of a convex $n$-gon $P$, which also
computes Winternitz's measure of symmetry for $P$.  We proceed by first
giving a linear-time algorithm for the decision problem: Does there
exist a point $p$ such that $f(p)>\delta$?  We then apply a randomized
technique due to Chan\cite{c99} to turn this decision algorithm into a
linear-time optimization algorithm.  For convenience, our model of
computation is the real RAM,\cite{ps85} though we do not use any functions that
are specific to this model.  We require only that it is possible to
to compute the exact area of a convex polygon.

The remainder of the paper is organized as follows.  \secref{decision}
describes our algorithm for the decision problem and
\secref{optimization} shows how to convert this decision algorithm
into an optimization algorithm.  \secref{conclusions} summarizes and
concludes with directions for future research.

%%%%%%%%%%%%%%%%%%%%%%%%%%%%%%%%%%%%%%%%%%%%%%%%%%%%%%%%%%%%%%%%%%%%%%%
\section{The Decision Algorithm}\seclabel{decision}

In this section, we give an $O(n)$ time algorithm for the following
decision problem: Is there a point $p$ such that $f(p)\ge \delta$?  An
alternative statement of this problem is: is $\Gamma_\delta$
non-empty?  In what follows, we show that $\Gamma_\delta$ can be
computed in $O(n)$ time.

A $\delta$-cut of $P$ is a directed line segment $uv$ with endpoints
$u$ and $v$ on the boundary of $P$ such that the area of $P$ to the
right of $uv$ is at most $\delta$.  Note that, for any $\delta$-cut
$uv$, the point $p$ cannot be to the right of $uv$. On the other
hand, if there is no $\delta$-cut $uv$ with $p$ on its right, then
$f(p)\ge\delta$.  Therefore, each $\delta$-cut defines a linear
constraint on the location of $p$, which we call a
$\delta$-constraint.  The answer to the decision problem is
affirmative if and only if there is a point $p$ that simultaneously
satisfies all $\delta$-constraints.  If such a point $p$ exists, we
call the constraints feasible, otherwise we call them infeasible.

Unfortunately, every polygon has an infinite number of $\delta$-cuts
and hence an infinite number of $\delta$-constraints.  However, we
will show that all constraints imposed by these $\delta$-cuts can be
represented succinctly as $O(n)$ non-linear (but convex) constraints
that can be computed in $O(n)$ time.

To generate a representation of all $\delta$-constraints, we begin by
choosing a point $u$ on the boundary of $P$ and finding the unique
point $v$ so that $uv$ is a $\delta$-cut.  Next, we sweep the points
$u$ and $v$ counterclockwise along the boundary of $P$ maintaining the
invariant that $uv$ has an area of exactly $\delta$ to its right.  We
continue this process until we return to the original points $u$ and
$v$.

Observe that, as long as $u$ and $v$ do not cross a vertex of $P$, the
intersection of all $\delta$-constraints belonging to an edge pair is
a convex region whose boundary consists of at most 2 linear pieces and
1 non-linear piece.  (See \figref{delta-cut}.) In fact, this
non-linear piece is a hyperbolic arc.  This is due to the well known
fact that a line tangent to a hyperbola forms a triangle of constant
area with the asymptotes of the hyperbola.  Furthermore, the
description complexity of these pieces is constant, since they are
defined by a four-tuple of vertices of $P$.  Thus, the intersection of
all these $\delta$-constraints can be computed explicitly in constant
time.  Since $u$ and $v$ sweep over each vertex exactly once, we
obtain $2n$ such convex constraints whose intersection is equal to the
intersection of all $\delta$-constraints.

\begin{figure}[b]
\begin{center}\includegraphics{delta-cut}\end{center}
\caption{The intersection of all $\delta$-constraints defined by a
pair of edges makes a convex region whose boundary consists of at
most 3 pieces.}\figlabel{delta-cut}
\end{figure}

Therefore, the decision problem reduces to determining if the
intersection of $2n$ convex regions is empty.  We can compute an
explicit representation of this intersection in $O(n)$ time, as
follows: Separately compute the intersection of all
$\delta$-constraints that contain the point $(0,+\infty)$ and all
$\delta$-constraints that contain the point $(0,-\infty)$ and then
compute the intersection of the two resulting convex regions.  Since
the $\delta$-constraints are generated sorted by slope, the first step
is easily done in $O(n)$ time using an algorithm similar to Graham's
Scan.\cite{a79,g72}  Since the two boundaries of the two resulting
regions are $x$-monotone and upwards, respectively downwards, convex,
their intersection (step two) can be computed in $O(n)$ time using a
left-to-right plane sweep.\cite{bo79}

We have just proven:

\begin{theorem}
Let $P$ be a convex $n$-gon and $\delta>0$ a real parameter.  Then
there exists an $O(n)$ time algorithm for the decision problem: Does
there exist a point $p$ such that $f(p)\ge\delta$?
\end{theorem}

%%%%%%%%%%%%%%%%%%%%%%%%%%%%%%%%%%%%%%%%%%%%%%%%%%%%%%%%%%%%%%%%%%%%%%%%
\section{The Optimization Algorithm}\seclabel{optimization}

In this section, we show how to use the decision algorithm of the
previous section along with a technique of Chan\cite{c99} to solve the
optimization problem: What is the largest value of $\delta$ such that
$\Gamma_\delta$ is non-empty?  Chan's technique requires only that we
be able to (1)~solve the decision problem in $D(n)=\Omega(n^\epsilon)$
time, $\epsilon>0$ and (2)~generate a set of $r>1$ subproblems each of
size $\alpha n$, $\alpha <1$, such that the solution to the original
problem is the minimum (or maximum) of the solutions to the
subproblems.  Under these conditions, the optimization problem can be
solved by a randomized algorithm in $O(D(n))$ expected time.

To apply Chan's technique, we need a suitable definition of
subproblem.  Let $S$ be a subset of edges of $P$.  The $S$-induced
$\delta$-constraints are the set of all $\delta$-constraints $uv$,
where $u$ and $v$ are both on edges of $S$.  The type of subproblems
we consider are those of determining for a given set $S$ and a value
$\delta$ whether or not the $S$-induced $\delta$-constraints feasible.
To obtain a linear-time algorithm, we must be able to solve such
subproblems in $O(|S|)$ time.

For a given set $S$, computing a representation of the $S$-induced
$\delta$-constraints, can be done using a modification of the sweep
algorithm from the previous section so that it only considers
$\delta$-cuts $uv$ where $u$ and $v$ are on elements of $S$.  The only
technical tool required for this modification is a data structure
that, given two points $u$ and $v$ on elements of $S$ (the boundary of
$P$) tells us the area of $P$ to the right of $uv$ in constant time.
This data structure is provided by Czyzowicz \etal\cite{ccu98} who
show that any convex $n$-gon can be preprocessed in $O(n)$ time so
that the area of the polygon to the right of any chord $uv$ can be
computed in $O(1)$ time.  Using this data structure, it is
straightforward to generate a representation of $S$-induced
$\delta$-constraints in $O(|S|)$ time.  Once we have computed these
constraints, we can test if they are feasible in $O(|S|)$ time.  Thus,
Condition~1 required to use Chan's technique is satisfied with
$D(n)=\Theta(n)$.

Next, we observe that Helly's theorem in the plane (c.f.,
Eckhoff\cite{e93}) implies that for any $\delta>\delta^*$ there exists
a set of three $\delta$-constraints whose intersection is empty.  In
our context, this means that $P$ contains 6 edges such that, for any
$\delta>\delta^*$, the $\delta$-constraints induced by those edges are
infeasible.  Therefore, if a set $S$ contains those 6 edges, then the
$S$-induced $\delta$-constraints are feasible if and only if
$\delta\le\delta^*$.

Therefore, we can solve our maximization problem as follows: Partition
the edges of $P$ in $7$ groups, $E_1,\ldots,E_7$, each of size
approximately $n/7$.  Next, generate subsets $S_1,\ldots,S_7$, by
taking all $7$ 6-tuples of $E_1,\ldots,E_7$.  Note that, for each
$S_i$, the $S_i$-induced $\delta$-constraints are satisfiable if
$\delta\le\delta^*$, since they are just a subset of the original
constraints.  On the other hand, for the set $S_j$ that contains the 6
edges guaranteed by Helly's theorem, the $S_j$-induced
$\delta$-constraints are not satisfiable for any $\delta>\delta^*$.
Therefore,
\[
\delta^* = \min\left\{\max\left\{\delta \mid 
	\mbox{$S_i$-induced $\delta$-constraints are satisfiable} 
	\right\} \mid 1\le i\le 7\right\} \enspace .
\]
Finally, observe that each $S_i$ is of size at most $\alpha n$, for
$\alpha=6/7$.  Therefore, we have satisfied the second condition
required to apply Chan's optimization technique.  This completes the
proof of:

\begin{theorem}
There exists a randomized $O(n)$ expected time algorithm for the
optimization problem: What is the largest value $\delta^*$ for which
$\Gamma_{\delta^*}$ is non-empty?
\end{theorem}

Of course, once $\delta^*$ is known, an explicit representation of
$\Gamma_{\delta^*}$ can be computed in $O(n)$ time.  Alternatively,
Chan's technique can also be made to output a point
$p^*\in\Gamma_{\delta^*}$.\cite{c99}


%%%%%%%%%%%%%%%%%%%%%%%%%%%%%%%%%%%%%%%%%%%%%%%%%%%%%%%%%%%%%%%%%%%%%%%%%
\section{Conclusions}\seclabel{conclusions}

We have given a randomized linear-time algorithm for determining the
center of area of a convex $n$-gon.  This algorithm is simple,
implementable, and is asymptotically faster than any previously known
algorithm.

Although our algorithm is simple and easy to implement, the constants
hidden in the $O$-notation are enormous.  A close examination of
Chan's technique reveals that the number of subproblems generated in
our application is actually $r\ge {k\choose 6}$, where $k$ is an
integer that satisfies $\ln {k\choose 6}+1 < k/6$. The smallest such
value of $k$ is $146$, which leads to $r={146\choose 6}=12\, 122\,
560\, 164$ subproblems.  Reducing this constant while maintaining the
$O(n)$ asymptotic running time remains an open problem.  One possible
approach is to treat the problem as an LP-type problem and try to use
the Matou\v sek-Sharir-Welzl algorithm.\cite{msw92}  The difficulty with
this approach is that the underlying LP-type problems consists of
as many as $n\choose 2$ constraints (though only $O(n)$ apply to any given value of
$\delta$).  A linear-time deterministic algorithm is also an open
problem. The current fastest deterministic algorithm runs in
$O(n\log^3 n)$ time.\cite{ls02}

Finally, we have not considered the problem of computing the center of
area of a non-convex polygon.  There are two different versions of
this problem, depending on whether a cut is defined as a chord of $P$,
which partitions $P$ into two polygons, or a line which may partition
$P$ into many polygons.  Approximation algorithms for the second case
are considered by D\'{\i}az and O'Rourke.\cite{do89}  To the best of
our knowledge, there are no exact algorithms for either version.


\section*{Acknowledgements}

The authors would like to thank Stefan~Langerman for helpful
discussions and for bringing his paper\cite{ls02} to our attention.

\bibliography{center}
\bibliographystyle{plain}




\end{document}
