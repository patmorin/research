\documentclass[lotsofwhite,charterfonts]{patmorin}
\usepackage{graphicx}

 
%\usepackage{amsthm}

\newcommand{\centeripe}[1]{\begin{center}\Ipe{#1}\end{center}}
\newcommand{\comment}[1]{}

\newcommand{\centerpsfig}[1]{\centerline{\psfig{#1}}}

\newcommand{\seclabel}[1]{\label{sec:#1}}
\newcommand{\Secref}[1]{Section~\ref{sec:#1}}
\newcommand{\secref}[1]{\mbox{Section~\ref{sec:#1}}}

\newcommand{\alglabel}[1]{\label{alg:#1}}
\newcommand{\Algref}[1]{Algorithm~\ref{alg:#1}}
\newcommand{\algref}[1]{\mbox{Algorithm~\ref{alg:#1}}}

\newcommand{\applabel}[1]{\label{app:#1}}
\newcommand{\Appref}[1]{Appendix~\ref{app:#1}}
\newcommand{\appref}[1]{\mbox{Appendix~\ref{app:#1}}}

\newcommand{\tablabel}[1]{\label{tab:#1}}
\newcommand{\Tabref}[1]{Table~\ref{tab:#1}}
\newcommand{\tabref}[1]{Table~\ref{tab:#1}}

\newcommand{\figlabel}[1]{\label{fig:#1}}
\newcommand{\Figref}[1]{Figure~\ref{fig:#1}}
\newcommand{\figref}[1]{\mbox{Figure~\ref{fig:#1}}}

\newcommand{\eqlabel}[1]{\label{eq:#1}}
\newcommand{\eqref}[1]{(\ref{eq:#1})}

\newtheorem{thm}{Theorem}{\bfseries}{\itshape}
\newcommand{\thmlabel}[1]{\label{thm:#1}}
\newcommand{\thmref}[1]{Theorem~\ref{thm:#1}}

\newtheorem{lem}{Lemma}{\bfseries}{\itshape}
\newcommand{\lemlabel}[1]{\label{lem:#1}}
\newcommand{\lemref}[1]{Lemma~\ref{lem:#1}}

\newtheorem{cor}{Corollary}{\bfseries}{\itshape}
\newcommand{\corlabel}[1]{\label{cor:#1}}
\newcommand{\corref}[1]{Corollary~\ref{cor:#1}}

\newtheorem{obs}{Observation}{\bfseries}{\itshape}
\newcommand{\obslabel}[1]{\label{obs:#1}}
\newcommand{\obsref}[1]{Observation~\ref{obs:#1}}

\newtheorem{assumption}{Assumption}{\bfseries}{\rm}
\newenvironment{ass}{\begin{assumption}\rm}{\end{assumption}}
\newcommand{\asslabel}[1]{\label{ass:#1}}
\newcommand{\assref}[1]{Assumption~\ref{ass:#1}}

\newcommand{\proclabel}[1]{\label{alg:#1}}
\newcommand{\procref}[1]{Procedure~\ref{alg:#1}}

\newtheorem{rem}{Remark}
\newtheorem{op}{Open Problem}

\newcommand{\etal}{\emph{et al}}

\newcommand{\voronoi}{Vorono\u\i}
\newcommand{\ceil}[1]{\left\lceil #1 \right\rceil}
\newcommand{\floor}[1]{\left\lfloor #1 \right\rfloor}



\newcommand{\Z}{\mathbb{Z}}

\newcommand{\foureight}{B_4,W_8}
\newcommand{\N}{\mathrm{N}}
\newcommand{\NE}{\mathrm{NE}}
\newcommand{\E}{\mathrm{E}}
\newcommand{\SE}{\mathrm{SE}}
\renewcommand{\S}{\mathrm{S}}
\newcommand{\SW}{\mathrm{SW}}
\newcommand{\W}{\mathrm{W}}
\newcommand{\NW}{\mathrm{NW}}

\newcommand{\ic}[2]{\langle #1,#2 \rangle}

\title{\MakeUppercase{Connectivity Preserving Transformations 
	of Binary Images}}
\author{The 2002 Barbados Gang}
\date{}

\begin{document}
\maketitle

\begin{abstract} 
We consider a local modification of a binary image
in which a black pixel can be swapped with an adjacent white pixel
provided that this preserves the topology of the image.  We show that,
under several definitions of topology, any image with $n$ black pixels
can be converted to any other image with $n$ black pixels using
$O(n^2)$ of these modifications.  
\end{abstract}

\section{Introduction}

We call a function $I:\Z^2\rightarrow\{0,1\}$ a \emph{binary image}.
We call the elements of $\Z^2$ \emph{pixels} and we say that a pixel
$p$ is \emph{black} (respectively, \emph{white}) if $I(p)=1$
(respectively, $I(p)=0$).  We say that an image is \emph{finite} if it
has a finite number of black pixels.  For a vector $\vec{v}$, we use
the notation $I+\vec{v}$ to denote the image $I'(p) = I(p-\vec{v})$,
i.e., the image $I$ translated by the vector $\vec{v}$.

Let $G_4$ be the graph whose vertex set is $\Z^2$ (the set of all
pixels) and in which two pixels $(x_1,y_1)$ and $(x_2,y_2)$ are
adjacent iff $(x_1-x_2)^2+(y_1-y_2)^2=1$, i.e., $G_4$ is the integer
lattice.  The graph $G_8$ is the graph whose vertex set is $\Z^2$ and
in which two pixels $(x_1,y_1)$ and $(x_2,y_2)$ are adjacent iff
$(x_1-x_2)^2+(y_1-y_2)^2\le 2$, i.e., $G_8$ is the integer lattice in
which two diagonals have been added to every face.  Given a binary
image $I$, the graph $B_4(I)$ (respectively $B_8(I)$) is the subgraph
of $G_4$ (respectively, $G_8$) induced by the black pixels and the
graph $W_4(I)$ (respectively $W_8(I)$) is the subgraph of $G_4$
(respectively, $G_8$) induced by the white pixels (see
\figref{doggy}). For $a,b\in\{4,8\}$ we say that an image is
$B_a,W_b$-connected iff the graphs $B_a$ and $W_b$ are each connected,
i.e., each has a single connected component.

\begin{figure}[htbp]
\begin{center}\begin{tabular}{ccc}
\includegraphics[scale=0.9]{dog} & 
\includegraphics[scale=0.9]{dog-48} &
\includegraphics[scale=0.9]{dog-84} \\
(a) & (b) & (c) \\
\end{tabular}\end{center}
\comment{
\begin{center}\begin{tabular}{cc}
\includegraphics{dog-44} & \includegraphics{dog-88} \\
(d) & (e) \\
\end{tabular}\end{center}
}
\caption{(a) a binary image $I$, (b) the graphs $B_4(I)$ and $W_8(I)$, and (c)
the graphs $B_8(I)$ and $W_4(I)$.}
\figlabel{doggy}
\end{figure}


In this paper we consider a local modification operation on binary
images in which a black pixel $p$ and a white pixel $q$ are
interchanged.  More precisely, we perform the \emph{interchange}
$\ic{p}{q}$ on $I$ to obtain the image $I'$ where
\[
     I'(x) = \left\{\begin{array}{ll}
         I(p) & \mbox{if $x=q$} \\
         I(q) & \mbox{if $x=p$} \\
         I(x) & \mbox{otherwise.}\end{array}\right.
\]

We say that the interchange $\ic{p}{q}$ is $4$-local (respectively
$8$-local) if $p$ and $q$ are adjacent in $G_4$ (respectively, $G_8$).
In this paper we are primarily concerned with $8$-local interchanges
and we are interested in whether two images with the same number of
black pixels differ by a sequence of connectivity-preserving
interchanges.  More precisely, we say that two image $I$ and $J$ that
are each $B_a,W_b$-connected are \emph{$(a,b)$-IP-equivalent}
\cite{rn02} if there exists a vector $\vec{v}$ and a sequence of
images $I_0=I,I_1,\ldots,I_r=J+\vec{v}$ such that each $I_i$ is
$B_a,W_b$-connected and $I_{i}$ can be converted into $I_{i+1}$ by a
single (8-local) interchange.

\subsection{Previous Work}

The study of connectivity in digital images seems to have been first
initiated by Rosenfeld \cite{r70,r73,r74} and has since become part of
the field of \emph{digital topology} \cite{hr96,kr89}.  The idea of
using connectivity-preserving interchanges (IP-equivalence) to convert
one image into another appears in a sequence of papers by Rosenfeld
\etal\ \cite{rkn98,rn02,rsn01}.

Rosenfeld, Saha, and Nakamura \cite{rsn01} study interchanges and
(among other things) show that any two $B_4,W_8$-connected digitial
arcs\footnote{An image $I$ is a \emph{digital arc} if the graph
$B_4(I)$ is a path.} with the same number of black pixels are
$(4,8)$-IP-equivalent. The same authors conjectured that any two
$B_4,W_8$-connected images are $(4,8)$-IP-equivalent.

Rosenfeld and Nakamura \cite{rn02} later resolved this conjecture in
the affirmative by giving an algorithm for computing a sequence of
interchanges to convert any $B_4,W_8$-connected image $I$ with $n$
black pixels into any other $B_4,W_8$-connected image $J$ with $n$
black pixels. Their algorithm achieves this by scanning $I$ with a
horizontal line from top to bottom and performing interchanges to
maintain the invariant that the part of the image above the scan line
consist of a set of disjoint vertical line segments. As the scan line
advances, the line segments above the scan line are moved and/or
merged in order to preserve this invariant.   Although the authors are
not concerned with the number of interchanges required to perform this
conversion, examining their algorithm reveals that the number of
interchanges is bounded by $O(n^3)$ and there exists examples for
which their algorithm performs $\Omega(n^3)$ interchanges.

Motivated by applications in robotics, and apparently unaware of
Ref.~\cite{rn02}, Dumitrescu and Pach \cite{dp04} consider the problem
of converting one image into another while preserving connectivity of
the graph $B_4$.  Although their definition of interchange is more
restricted than that presented here, they show that any image $I$ for
which $B_4(I)$ is connected can be converted into any image $J$ for
which $B_4(J)$ is connected using a sequence of $O(n^2)$ 8-local
interchanges that preserve the connectivity of the graph
$B_4$-connectivity.  They achieve this result by collecting all black
pixels on a line segment.  To add a new black pixel to the line
segment they select a very particular pixel and move it around the
boundary of the black pixels until it lies on the line segment.

\subsection{New Results}

In this paper we prove that, for any $(a,b)\in\{(4,8),(8,4),(8,8)\}$,
any two $B_a,W_b$-connected images $I$ and $J$ each with $n$ black
pixels are $(a,b)$-IP-equivalent and one can be converted into the
other with a sequence of $O(n^2)$ 8-local interchanges. To the best of
our knowledge, these are the first results on $(8,4)$- and
$(8,8)$-IP-equivalence. This is also the first quadratic bound on the
number of interchanges used to show $(4,8)$-IP-equivalence to two
arbitrary $B_4,W_8$-connected images.  These results are optimal up to
constant factors since it is easy to see that converting a horizontal
line segment into a vertical line segment requires $\Omega(n^2)$
8-local interchanges.

It is also worth noting that our proof technique, and resulting
algorithm, is of a different style than those used by Rosenfeld and
Nakamura \cite{rn02} and Dumitrescu and Pach \cite{dp04}.  We obtain
our results by showing that, as long as $I$ is not a vertical segment,
there is always a particular black pixel $p$ that can be moved upwards
or to the left.  By repeatedly performing this local move the image
organizes itself into a vertical line segment.  This is unlike
previous algorithms \cite{dp04,rn02} in that the entire process takes
place without any long-term planning about the movement of a pixel or
group of pixels.  

The remainder of the paper is organized as follows: \Secref{foureight}
gives the proof for $(4,8)$-IP-equivalence.  \Secref{eightfour}
presents the proof for $(8,4)$-IP-equivalence.  \Secref{eighteight}
handles the $(8,8)$ case.  Finally \secref{conclusions} summarizes and
concludes with open problems.

\section{Maintaining $\foureight$-Connectivity}\seclabel{foureight}

For a pixel $p=(x,y)$ we use the notation $\N(p)$ (respectively,
$\E(p)$, $\S(p)$, $\W(p)$) to denote the pixel $(x,y+1)$
(respectively, $(x+1,y)$, $(x,y-1)$, $(x-1,y)$).  We allow
concatenation of these modifiers so that, for example
$\NE(p)=\N(\E(p))$, $\N\NE(p)=\N(\N(\E(p)))$, and so on.  We use the
shorthand $\N^{(0)}(p)=p$ and, for $k>0$,
$\N^{(k)}(p)=\N\N^{(k-1)}(p)$.  We also use the regular expression
notations $*$ and $+$ so that, for example $\N^+(p)=\{\N^{(k)}(p):
k>0\}$ and $N^*(p)=\{\N^{(k)}(p): k\ge 0\}$.

A \emph{cut vertex} of a connected graph $G$ is a vertex whose removal
disconnects $G$.  We start with a simple but useful observation that
gives a sufficient condition for an interchange to preserve
connectivity.

\begin{obs}\obslabel{cut-vertex}
Let $p$ be a black pixel and $q$ be a white pixel in some
$B_a,W_b$-connected image $I$.  Then the interchange $\ic{p}{q}$
preserves $B_a,W_b$-connectivity if $p$ is not a cut vertex of $B_a(I)$,
$q$ is not a cut vertex of $W_b(I)$, $p$ is adjacent to some white pixel
in $W_b(I)$ and $q$ is adjacent to some black pixel in $B_a(I)$.
\end{obs}

The following lemma is the main step in the proof that two
$B_4,W_8$-connected images $I$ and $J$ differ by a sequence of
interchanges.  It tells us that, unless our current image is a
vertical line segment we can always move some black pixel, not in the
leftmost column, either up or to the left.

\begin{lem}\lemlabel{foureight}
For any finite $\foureight$-connected image $I$ there exists an
$8$-local interchange $\ic{p}{q}$ such that $p$ is black, $q$ is
white, $q\in\{\W(p),\NW(p),\N(p),\NE(p)\}$ and the resulting image
$I'$ is $\foureight$-connected. Furthermore, unless all black pixels
of $I$ have the same $x$-coordinate, $p$ is not a black
pixel with minimum $x$-coordinate.
\end{lem}


\begin{proof}
Let $p'=(x,y)$ be the pixel such that
\begin{enumerate}
  \item $p'$ is black, 
  \item $S(p')$ is white,
  \item There exists an integer $k\ge 0$ such that all pixels
	$\N^{(1)}(p'),\ldots,\N^{(k)}(p')$ are black and 
        all pixels in $\N^{+}\N^{(k)}(p')$ are white,
  \item all pixels in $\E^+\N^*(p)$ are white, and
  \item $y$ is maximum.
\end{enumerate}
We know that such a pixel exists because a pixel satisfying the first
four conditions can be found in the set of black pixels with maximum
$x$-coordinate and a pixel satisfying the fifth condition is
guaranteed by finiteness.  We will show that the pixel $p$ in the
interchange $\ic{p}{q}$ we are looking for must exist somewhere near
$p'$.  We distinguish between two cases:

\paragraph{Case 1: $p'$ is not a cut vertex of $B_4(I)$.}  In this
case, if $\N(p')$ is black (\figref{case-one}.a) then we can perform
the interchange $\ic{p'}{\NE(p')}$.  Since $p'$ is not a cut vertex of
$B_4(I)$ and $\NE(p')$ is not a cut vertex of $W_8(I)$,
\obsref{cut-vertex}, implies that this interchange preserves
connectedness.

\begin{figure}
\begin{center}
\begin{tabular}{ccccccc}
\includegraphics{case1a} & 
\includegraphics{case1b} & 
\includegraphics{case1c} & 
\includegraphics{case1d} & 
\includegraphics{case1e} & 
\includegraphics{case1f} \\
(a) & (b) & (c) & (d) & (e) & (f)
\end{tabular}
\end{center}
\caption{Illustrating Case~1 in the proof of \lemref{foureight}.}
\figlabel{case-one}
\end{figure}

Therefore, we may assume that $\N(p')$ is white.  But in this case,
$\W(p')$ must be black (\figref{case-one}.b), otherwise $p'$ is an
isolated vertex in $B_4$.

If $\NW(p')$ is black (\figref{case-one}.c),then we can perform the
interchange $\ic{p'}{\N(p')}$.  Again, neither $p'$ nor $\N(p')$ are
cut-vertices in their respective graphs, so this interchange preserves
connectedness by \obsref{cut-vertex}.

Otherwise $\NW(p')$ is white (\figref{case-one}.d) and we claim that
the interchange $\ic{p'}{\NW(p')}$ preserves connectedness.  To see
why this is so, we first observe that, by the choice of $p'$, all
pixels in $\N^+(p')$ and $N^+W(p')$ are white (\figref{case-one}.d).
Thus, the only way in which $\NW(p')$ could be a cut vertex in $W_8(I)$
(as required by \obsref{cut-vertex}) is if $\W\NW(p')$ is black
(\figref{case-one}.e). But in this case, the choice of $p'$ ensures
that $\W\W(p')$ is black (\figref{case-one}.f).  But this contradicts
the requirement that $\NW(p')$ be a cut vertex of $W_8(I)$.

\paragraph{Case 2: $p'$ is a cut vertex of $B_4$.} In this case,
$\N(p')$ and $\W(p')$ must be black or else $p'$ would have less than
2 neighbours in $B_4(I)$ and could not be a cut vertex.  Also,
$\NW(p')$ must be white otherwise $p'$ would not be a cut vertex
(\figref{case-two}.a). By the same reasoning, if the interchange
$\ic{p'}{\NW(p')}$ does not preserve connectivity, it is because
$\NW(p')$ is a cut vertex in $W_8$.  There are two ways in which this
can happen:

\begin{figure}
\begin{center}
\begin{tabular}{ccccccc}
\includegraphics{case2a} & 
\includegraphics{case2b} & 
\includegraphics{case2c} & 
\includegraphics{case2d} & 
\includegraphics{case2e} & 
\includegraphics{case2f} \\
(a) & (b) & (c) & (d) & (e) & (f)
\end{tabular}
\end{center}
\caption{Illustrating Case~2 in the proof of \lemref{foureight}.}
\figlabel{case-two}
\end{figure}

If $\N\NW(p')$ is black (\figref{case-two}.b) then it must be the case
that $\N(\N(p'))$ is white (\figref{case-two}.c), otherwise $\NW(p')$
could not be a cut vertex of $W_8(I)$.  But this contradicts our
choice of $p'$.

Otherwise $\W\NW(p')$ must be black (\figref{case-two}.d) and
$\W\W(p')$ must be white (otherwise $\NW(p')$ is not a cut vertex of
$W_8(I)$.  There are 2 subcases to consider:

\noindent Case 2a: Every path in $B_4(I)$ from $p'$ to $\W\NW(p')$
visits $\W(p')$ (\figref{case-two}.e).  In this case, we claim that
the interchange $\ic{\W(p')}{\NW(p')}$ preserves connectivity.  

To see that $W_8(I')$ is connected, we use the well-known fact that
$W_8(I')$ is connected if and only if $B_4(I')$ does not contain a
simple cycle with a white vertex in its interior \cite{r70}.  Imagine
coloring $\NW(p')$ black to obtain the image $I_2$.  This would create
one or more cycles in $B_4(I_2)$ that have a white vertex in their
interior.  Notice that, since every path in $B_4(I)$ from $p'$ to
$\W\N\W(p')$ uses $\W(p')$, all cycles in $B_4(I_2)$ that contain a
white pixel in their interior contain $\W\W(p')$ in their interior and
all such cycles include the vertices $\NW(p')$ and $\W(p')$.  But the
interchange $\ic{\W(p')}{\N\W(p')}$ makes $\W(p')$ white, which
eliminates all the cycles in $B_4(I_2)$ that have a white vertex in
their interior.  Thus, the graph $B_4(I')$ does not contain a cycle
with a white pixel in its interior so $W_8(I')$ is connected.

To see that $B_4(I')$ is connected we simply observe that, because
$\W(p')$ has only two neighbours in $B_4(I)$, every path in $B_4(I)$
that has $\W(p')$ in its interior is of the form
$p_1,p_2,\ldots,p',\W(p'),\S\W(p'),\ldots,p_m$. But $B_4(I')$ does
contain a path $P$ from $p'$ to $\S\W(p')$ that uses $\N(p')$ and
$\NW(p')$. Thus, $B_4(I')$ contains the path
$p_1,p_2,\ldots,P,\ldots,p_m$.  Thus, since $B_4(I)$ is connected,
$B_4(I')$ is connected. 

\noindent Case 2b: Every path in $B_4(I)$ from $p'$ to $\W\NW(p')$
visits $\N(p')$ (\figref{case-two}.f).  In this case, we claim that
the interchange $\ic{\N(p')}{\NW(p')}$ preserves connectivity.  The
proof of this (which we omit) is almost identical to the proof for
Case~1a.

Thus far, we have proven the existence of the interchange $\ic{p}{q}$
we are looking for.  What remains is to argue that the interchange
$\ic{p}{q}$ we perform does not involve a black pixel with minimum
$x$-coordinate unless all black pixels of $I$ have the same
$x$-coordinate.  However, this is evident simply by examining the
cases individually to see that for any of the black pixels $p$ that we
select there exists a black pixel with smaller $x$-coordinate
(Figures~\ref{fig:case-one} and \ref{fig:case-two} help with this).
The only case in which this may not be true is Case~1a, which
corresponds to the case where all pixels of $p$ have the same
$x$-coordinate.  \end{proof}

\begin{lem}\lemlabel{segment}
Any $B_4,W_8$-connected image $I$ with $n$ black pixels is
$(4,8)$-IP-equivalent to some $B_4,W_8$-connected ``vertical line
segment'' image $J$ in which all black pixels have $x$-coordinate $0$
and $I$ can be converted into $J$ by a sequence of $O(n^2)$
8-local interchanges.
\end{lem}

\begin{proof} 
Without loss of generality, assume that the minimum $x$-coordinate of
all black pixels of $I$ is 0 and that of all black pixels with
$x$-coordinate 0, the minimum $y$-coordinate is 0.  Let $p_0$ be the
black pixel $(0,0)$. Define the \emph{potential} of a black pixel
$p=(x,y)$ as $\Phi(p)=x+2n-2y$ and define the potential $\Phi(I)$ of
image $I$ as the sum the potentials of all black pixels in $I$.
Because $p_0$ is black and $B_4(I)$ is connected, it is easily
verified that $\Phi(p)\le 5n$ for any black pixel $p$ and therefore
$\Phi(I)\le 5n^2$.

Next, suppose that we perform a sequence of connectivity-preserving
interchanges on $I$ none of which involves a black pixel with
$x$-coordinate 0 to obtain an image $I'$.  Then $\Phi(I')\ge 0$ since
the $x$-coordinate of every black pixel in $I'$ is at least 0 and the
$y$-coordinate of every black pixel is at most $n$ (because $p_0$ is a
black pixel of $I'$ and $B_4(I')$ is connected). 

Finally, observe that any time we apply \lemref{foureight} the
potential of the resulting image decreases by at least 1.  Thus, by
applying \lemref{foureight} at most $5n^2+1$ times we obtain an image
$J'$ such that $\Phi(J')< 0$.  However, from the above discussion this
cannot  occur unless we, at some point, performed an interchange
involving a black pixel with $x$-coordinate 0.  But the final
condition of \lemref{foureight} guarantees that this can only happen
if the image just before the interchange is a (vertical translation
of) the vertical line segment $J$ we are looking for.  \end{proof}

By applying \lemref{segment} to convert any binary image $I$ into a
vertical line segment and then converting that line segment into any
other binary image $J$ we obtain our first theorem:

\begin{thm}
Any two $B_4,W_8$-connected images $I$ and $J$ each having $n$ black
are $(4,8)$-IP-equivalent and $I$ can be converted into $J$ using a
sequence of $O(n^2)$ 8-local interchanges.
\end{thm}


\section{Maintaining $B_4,W_8$-Connectivity}\seclabel{eightfour}

\section{Maintaining $B_8,W_8$-Connectivity}\seclabel{eighteight}

We should note that if $I$ and $J$ are $B_4,W_8$-connected or
$B_8,w_4$-connected then $(8,8)$-IP-equivalence of $I$ and $J$ follows
from the two previous results (even if $I$ is $B_4,W_8$-connected and
$J$ is $B_8,W_4$-connected).

\section{Conclusions}\seclabel{conclusions}

We have shown that, for any $(a,b)\in\{(4,8),(8,4),(8,8)\}$, any two
$B_a,W_b$-connected images $I$ and $J$ each with $n$ black pixels
differ by a sequence of $O(n^2)$ interchanges.   The obvious open
problem is whether any two $B_4,W_4$-connected images with the same
number of black pixels differ by a sequence of interchanges.  This
version of the problem appears to be quite different from the other
three in the sense that it is not always possible to move a black
pixel up or to the left (see \figref{bad-example}).  Thus, it appears
that an algorithm for this case requires a more careful plan for the
movement of the pixels.


\begin{figure}
\begin{center}
\includegraphics{bad-example}
\end{center}
\caption{An example of a $B_4,W_4$-connected image for which no black
pixel can be moved $\W$, $\NW$, $\N$, or $\NE$ while maintaining
$B_4,W_4$-connectivity.}
\figlabel{bad-example}
\end{figure}



\bibliographystyle{plain}
\bibliography{pixels}



\end{document}
