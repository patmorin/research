\documentclass[charterfonts,lotsofwhite]{patmorin}
\usepackage[noend]{algorithmic}

\input{pat}
\newcommand{\td}{\mathrm{depth}}
\newcommand{\lp}{\mathrm{LP}}

\title{\MakeUppercase{Output-Sensitive Algorithms for Tukey Depth}}
\author{David Bremner \and
	John Iacono \and
	Stefan Langerman \and
	Pat Morin \and
	Others?}
\date{}
\begin{document}
\maketitle

\begin{abstract}
The \emph{Tukey depth} of a point $p$ with respect to a finite set $S$ of
points is the minimum number of elements of $S$ contained in any
closed halfspace that contains $p$.  Algorithms for computing the
Tukey depth of a point in various dimensions are considered.  The
running times of these algorithms depend on the value of the output,
making them suited to situations such as outlier removal when the
value of the output is typically small.
\end{abstract}

\section{Introduction}

Let $S$ be a set of $n$ points in $\R^d$.
The \emph{Tukey depth}, or \emph{halfspace depth} of a point $p\in\R^d$ with
respect to $S$ can be defined in several equivalent ways \cite{t73}:
\begin{eqnarray}
\td(p,S) & = & \min\{ |h\cap S| :
	             \mbox{$h$ is a halfspace containing $p$} \} \\ 
                     \eqlabel{tuk-orig}
            & = & \min\{ |h\cap S| :
                      \mbox{$h$ is a halfspace with $p$ on its boundary} \} \\ 
                       \eqlabel{tuk-boundary}
            & = & \min\{ |S'| :
                      \mbox{$p$ is outside the convex hull of 
                           $S\setminus S'$} \}
                      \eqlabel{tuk-hull}
\end{eqnarray}

In this paper we consider algorithms for computing the Tukey depth of
a point $p$ with respect to a set $S$, whose running time is dependent
the value, $k$, of the output. In particular, we present:

\begin{enumerate}
\item A simple $O(n + k\log k)$ time algorithm for points in $\R^2$
(\secref{2d}).  The most complicated data structure used in this
algorithms is a binary heap.

\item An $O(n + (n-k)\log(n-k))$ time algorithm to find the largest
clique in a circular-arc graph, where $k$ is the size of the clique
found (\secref{max-clique}).  This problem is a generaliation of the
Tukey depth problem in $\R^2$.

\item An $O(n\log n + k^2\log n)$ time algorithm for points in $\R^3$
and an $O(n + k^{11/4}n^{1/4}\log^{O(1)})$ time algorithm for points
in $\R^4$ (\secref{3-4-d}).  These algorithms rely on results of Chan
on linear programming with violated constraints \cite{X} which in turn
rely on sophisticated range searching data structures \cite{X} and/or
dynamic convex hull data structures \cite{X}.

\item A simple $O(d^k \lp(n,d-1))$ time algorithm for points in
$\R^d$, where $\lp(n,d)$ denotes the time required to determine the
feasibility of a linear program having $n$ constraints and $d$
variables (\secref{d-d}).  Not surprisingly, this algorithm is also
based on linear programming with violated constraints and is obtained
by presenting a fixed-parameter tractable algorithm for a
parameterization of the NP-hard \textsc{Maximum-Feasible-Subsystem}
problem.
\end{enumerate}

For the remainder of this paper we use the following notations: For
points $p,q\in\mathbb{R}^d$, $p_i$ denotes the $i$th coordinate of
$p$, $\|p\|=(\sum_{i=1}^d p_i^2)^{1/2}$, and $p\cdot
q=\sum_{i=1}^d p_iq_i$.  The unit sphere in $\R^{d+1}$ is denoted by
$\Sp^d =\{ p\in\R^{d+1} : \|p\|=1\}$. The top side of this sphere is
denoted by $\Sp^d_+ =\{ p\in\Sp^{d} : p_{d+1} > 0 \}$ and the bottom
side is denoted by $\Sp^d_- =\{ p\in\Sp^{d} : p_{d+1} < 0 \}$.



\section{An Algorithm for Points in $\R^2$}
\seclabel{2d}

In this section we give a simple $O(n + k\log k)$ time algorithm to
compute the Tukey depth of a point $p\in\R^2$ with respect to
a set $S$ of $n$ points in $\R^2$.  The algorithm begins by
partitioning $\R^2$ into 4 quadrants $Q_0,\ldots,Q_3$ around $p$.
The algorithm then simultaneously begins computing the 4 quantities
$\td_0(p,S),\ldots,\td_3(p,S)$ where 
\begin{equation}
     \td_i(p,S) = \min\{|h\cap S| : \mbox{$h$ is a halfspace containing
$Q_i$} \} \enspace . \eqlabel{four-min}
\end{equation}
Clearly, $\td(p,S) = \min\{\td_i(p,S): 0\le i \le 3 \}$ since any
halfspace containing $p$ contains at least one of the four quadrants.
In the remainder of this section we will describe how to compute
$k_i=\td_i(p,S)$ in $O(n + k_i\log k_i)$ time.  Since the
computation can stop once $\td_i(p,S)$ has been computed for
the index $i$ that minimizes \eqref{four-min}, this yields an $O(n +
k\log k)$ time algorithm, where $k=\td(p,S)$.

Let $S_i=S\cap Q_i$. To compute $\td_i(p,S)$ we create two binary
heaps $H_{i-1}$ and $H_{i+1}$ that store the elements of $S_{i-1}$,
respectively $S_{i+1}$, in counterclockwise, respectively, clockwise,
order around $p$. Creating these two heaps takes $O(n)$ time using the
standard bottom-up algorithm to construct a binary heap \cite{clrXX}.
Next we extract elements one at a time from each of $H_{i-1}$ and
$H_{i+1}$ until either (a)~one of the heaps is empty or (b)~we extract
two elements $q$ from $H_{i-1}$ and $r$ from $H_{i+1}$ such that the
angle $\angle spq > \pi$.  Suppose we have extracted $\ell$ elements
each from $H_{i-1}$ and $H_{i+1}$ when this occurs.  Then it is easy
to verify that 
\[  
  |S_i| + \ell - 1 \le \td_i(p,S) \le |S_i| + 2\ell \enspace .
\]

Next, we continue to extract as many elements as possible from each of
$H_{i-1}$ and $H_{i+1}$ up to a maximum of $\ell$ elements each. The
total time required to extract these at most $4\ell$ from the two
heaps is $O(\ell\log n)$.  By sorting and scanning all the elements
extracted from the heap plus the elements of $S_i$ we can then
compute $\td_i(p,S)$ in an additional
\[
     O((|S_i|+\ell)\log n) = O(k_i\log n)
\] 
time.  This yields an a total running time of 
\[  
   O(n + k_i\log n) = O(n + k_i\log k_i) \enspace ,
\]
as required.  This completes the proof of:

\begin{thm}\thmlabel{2-d}
The Tukey depth of a point $p$ with respect to a set $S$ of $n$ points
in $\R^2$ can be computed in $O(n + k\log k)$ time, where
$k$ is the value of the output.
\end{thm}

\section{An Algorithm for \textsc{Max-Clique} in Circular-Arc Graphs}
\seclabel{max-clique}

The problem of computing Tukey depth in $\R^2$ can be viewed as a
problem on a set of circular arcs.  By \eqref{tuk-boundary}, computing
the Tukey depth of $p$ is equivalent to finding a unit normal vector
$v$ such that the halfspace with $p$ on its boundary and having inner
normal $v$ contains as few points of $S$ as possible.
Note that the set of unit normals in $\R^2$ is the unit circle $\Sp^1$
and that each point $q\in S$ defines a circular arc such that all
choices of $v$ in this circular arc yield a halfspace that does not
contain $q$.  Thus, the Tukey depth problem reduces to the problem of
finding a vector $v$ that is contained in the largest number of
circular arcs. The partitioning into 4 quadrants used in for the
algorithm of \thmref{2-d} works because all the circular arcs are
actually half circles.  In this section we extend the results of the
previous section to the case of arbitrary circular arcs.

Let $C$ be a set of $n$ circular arcs of the unit circle $\Sp^1$.  We
describe an $O(n+(n-k)\log (n-k))$ time algorithm to find a point
$v\in\Sp^1$ that is contained in the largest number of elements of
$C$.  Let $p_1,\ldots,p_{2n}$ denote the $2n$ endpoints of the arcs in
$C$, as they occur in counterclockwise order.  We use the notation
$[a,b]$ to denote the circular arc that begins at $a$ and proceeds
counterclockwise to $b$.  The following observations imply that all
the points of very high depth are clustered close together.

\begin{obs}
Let $q\in[p_i,p_{i+1}]$ be a point contained in $n-k$ arcs of $C$.
Then, for any $0\le r\le n$, every point $q'\in[p_{i-r},p_{i+r}]$ is
contained at least $n-k-r$ arcs of $C$.
\end{obs}

\begin{obs}
Let $q\in[p_i,p_{i+1}]$ be a point contained in $n-k$ arcs of $C$.
Then, for any $k \le r\le n$, every point $q'\notin[p_{i-r},p_{i+r}]$ 
is contained in at most $n+k-r$ arcs of $C$.
\end{obs}

To help isolate the point contained in the largest number of circular
arcs we take a sample $s_0,\ldots,s_t\subseteq \{p_1,\ldots,p_{2n}\}$
such that any interval $[s_i,s_{i+1}]$ between two consecutive sample
points contains at most $n^\alpha$ points of $\{p_1,\ldots,p_{2n}\}$

\section{Algorithms for Points in $\R^3$ and $\R^4$}
\seclabel{3-4-d}

\notice{general position and no zero}
The previous section showed how the problem of computing the Tukey
depth of a point in $\R^2$ is equivalent to the problem of finding a
point contained in the largest number of halfcircles on the unit
circle $\Sp^1$.  A similar statement is true in $\R^d$:  Each point
$q\in S$ defines an open halfsphere $q^*=\{ v\in\Sp^{d-1} : v\cdot q <
0\}$.  That is, all vectors in $q^*$ are the inner normals of planes
that contain $p$ but do not contain $q$.  Thus, the problem of
determining the Tukey depth of $p$ reduces to the problem of finding
the point contained in the largest number of halfspheres in $S^*=\{q^*
: q\in S\}$.

We simply observe that this problem can be solved by solving two
problems in $\R^{d-1}$.  Each open halfsphere $q^*\in S^*$ is the
intersection of an open halfspace $q^\#$ with $\Sp^{d-1}$.  Consider
the intersection of $q^\#$ with the hyperplane
$H_+=\{(x_1,\ldots,x_d):x_d=1\}$.  By central projection, there is a
1-1 correspondence between points in $\Sp^{d-1}_+$ and $H_+$ and this
projection has the property that $r\in\Sp^{d-1}_+$ is in $q^*$ if and
only if the projection of $r$ is in $q^\#\cap H_+$.  Thus, finding the
point in $\Sp^{d-1}_+$ contained in the largest number of halfspheres
is equivalent to finding a point in $H_+$ contained in the largest
number of halfspaces.  A similar statement holds regarding
$\Sp^{d-1}_-$ using the hyperplane $H_-=\{(x_1,\ldots,x_d):x_d=1\}$.

The above discussion shows that computing the Tukey depth of a point
in $\R^d$ reduces to two instances of the problem
\textsc{MaximumFeasibleSubsystem} in $\R^{d-1}$: Given set $K$ of $n$
halfspaces in $\mathbb{R}^{d-1}$, find the subset $K'$ of $K$ of
minimum cardinality such that $\cap (K\setminus K')$ is non-empty.
The current best results for \textsc{MaximumFeasibleSubsystem} in
small dimensions are due to Chan \cite{c04}.  Using two instances of
his algorithm for \textsc{MaximumFeasibleSubsystem} in $\R^2$,
respectively, $\R^3$, and running them in parallel gives:

\begin{thm}
The Tukey depth of a point $p$ with respect to a set $S$ of $n$ points
in $\R^3$ can be computed in $O(n\log n + k^2\log n)$ time, where
$k$ is the value of the output.
\end{thm}


\begin{thm}
The Tukey depth of a point $p$ with respect to a set $S$ of $n$ points
in $\R^4$ can be computed in $O(n\log n + k^{11/4}n^{1/4}\log^{O(1)}
n)$ time, where $k$ is the value of the output.
\end{thm}

\section{An Algorithm for Points in $\R^d$}
\seclabel{d-d}

Finally, we consider the general case of point sets in $\R^d$.  
In the previous section we showed that computing the Tukey depth of
a point $p$ with respect to a point set $S$ of $n$ points in $\R^d$
can be reduced to two instances of \textsc{MaximumFeasibleSubsystem}
in $\R^{d-1}$.  Here we give a fixed-parameter tractable \cite{dfXX}
algorithm for \textsc{MaximumFeasibleSubsystem}.  The algorithm uses
linear programming as a subroutine in the following way.  Given a
collection $K$ of halfspaces in $\R^{d-1}$, an algorithm for linear
programming can be used to either
\begin{enumerate}
\item Determine a point $p\in\cap K$ if such a point exists or,
\item report a subset $B\subseteq K$, $|B|\le d$, such that $\cap B=\emptyset$.
\end{enumerate}
Let $\textsc{IIS}(K)$ denote a routine that outputs, in the latter
case, the set
$B$ or outputs, in the former case, the empty set.
Then the following algorithm solves the
\textsc{MaximumFeasibleSubsystem} decision problem:

\noindent$\textsc{MFS}(K,k)$
\begin{algorithmic}[1]
\STATE{\COMMENT{$\star$ determine if there exists $K'\subseteq K$, $|K'|\le
k$, such that $\cap(K\setminus K')\neq\emptyset$ $\star$} }
\STATE{$B\gets\textsc{IIS}(K)$}
\IF{$B=\emptyset$}
   \STATE{\textbf{return} true}
\ENDIF
\IF{$k=0$}
   \STATE{\textbf{return} false}
\ENDIF
\FOR{each $h\in K'$}
   \IF{$\textsc{MFS}(K\setminus\{h\},k-1)=\mathrm{true}$}
      \STATE{\textbf{return} true}
   \ENDIF
\ENDFOR
\STATE{\textbf{return} false}
\end{algorithmic}

Correctness of the above algorithm is easily established by induction
on the value of $k$.  The running time of the algorithm is given by
the recurrence
\[
    T(n,k) \le \lp(n,d-1)+ dT(n-1,k-1) \enspace ,
\]
where $\lp(n,d)$ denotes the running time of an algorithm for solving
a linear programming with $n$ constraints and $d$ variables.  This
recurrence readily resolves to $O(d^k\lp(n,d-1))$.  Running this
algorithm with the values $k=1,2,\ldots$ completes the
proof of:

\begin{thm}\thmlabel{d-d}
The Tukey depth of a roint $p$ with respect to a set $S$ of $n$ points
in $\R^d$ can be computed in $O(d^k\lp(n,d-1))$ time, where $k$ is
the value of the output and $\lp(n,d)$ is the time to solve a linear
program with $n$ constraints and $d$ variables.
\end{thm}



\comment{As in
the previous section, we formulate our solution by considering the
third definition of Tukey depth, i.e., Equation~\eqref{tuk-hull}. That
is, we wish to remove the smallest possible set $S'\subseteq S$ such
that $p$ is outside the convex hull of $S\setminus S'$.  We begin by
considering the problem of determining if $p$ is outside the convex
hull of $S$, i.e., if $\td(p,S)=0$.

Without loss of generality assume $p$ is the origin. Then testing if
$p$ is outside the convex hull of $S$ can be formulated as the problem
of finding coefficients $a_1,\ldots,a_d$ such that $\sum_{i=1}^da_i
q_i > 0$ for all $q\in S$.  \notice{is it easy to make this a
non-strict inequality?} Note that this is a set of $n$ linear
inequalities in $d$ variables. Any of the standard algorithms for
linear programming will either report that this set of inequalities is
feasible and give a vector $(a_1,\ldots,a_d)$ satisfying all
constraints or will report a set of $d+1$ constraints that not
feasible. These $d+1$ constraints correspond to $d+1$ points of $S$
whose convex hull (a simplex) contains $p$.  Denote by
$\textsc{Carath\'eodoryWitness}(p,S)$ a routine that takes as input
$p$ and $S$ and either returns a set of at most $d+1$ points in $S$
whose convex hull contains $p$ or, in the case where $p$ is outside
the convex hull of $S$, the empty set.  The following algorithm
determines whether $\td(p,S)\le k$:

\noindent$\textsc{TukeyDepthDecision}(p,S,k)$
\begin{algorithmic}
\STATE{\COMMENT{ return true if $\td(p,S)\le k$ and false otherwise }}
\STATE{$\Delta\gets \textsc{Carath\'eodoryWitness}(p,S)$}
\IF{$\Delta=\emptyset$} 
   \STATE{\textbf{return} true} 
\ENDIF
\IF{$k= 0$}
   \STATE{\textbf{return} false}
\ENDIF
\FOR{each $q\in \Delta$}
   \IF{$\textsc{TukeyDepthDecision}(p,S\setminus\{q\},k-1)=\mathrm{true}$}
     \STATE{\textbf{return} true}
   \ENDIF
\ENDFOR
\STATE{\textbf{return} false}
\end{algorithmic}

Correctness of the above algorithm follows easily by induction on
$\td(p,S)$.  The running time of the algorithm is given by the
recurrence
\[
  T(n,d,k) \le \lp(n,d) + (d+1) T(n-1,d,k-1)
\]
which resolves easily to $O((d+1)^k\lp(n,d))$ where $\lp(n,d)$ is the
time to solve a linear program with $n$ constraints and $d$ variables.
Running the above decision algorithm repeatedly with $k=0,1,\ldots$
completes the proof of:

\begin{thm}\thmlabel{d-d}
The Tukey depth of a roint $p$ with respect to a set $S$ of $n$ points
in $\R^d$ can be computed in $O((d+1)^k\lp(n,d))$ time, where
$k$ is the value of the output and $\lp(n,d)$ is the time to solve a
linear program with $n$ constraints and $d$ variables.
\end{thm}
}


\end{document}

