\documentclass[lotsofwhite]{patmorin}
 
%\usepackage{amsthm}

\newcommand{\centeripe}[1]{\begin{center}\Ipe{#1}\end{center}}
\newcommand{\comment}[1]{}

\newcommand{\centerpsfig}[1]{\centerline{\psfig{#1}}}

\newcommand{\seclabel}[1]{\label{sec:#1}}
\newcommand{\Secref}[1]{Section~\ref{sec:#1}}
\newcommand{\secref}[1]{\mbox{Section~\ref{sec:#1}}}

\newcommand{\alglabel}[1]{\label{alg:#1}}
\newcommand{\Algref}[1]{Algorithm~\ref{alg:#1}}
\newcommand{\algref}[1]{\mbox{Algorithm~\ref{alg:#1}}}

\newcommand{\applabel}[1]{\label{app:#1}}
\newcommand{\Appref}[1]{Appendix~\ref{app:#1}}
\newcommand{\appref}[1]{\mbox{Appendix~\ref{app:#1}}}

\newcommand{\tablabel}[1]{\label{tab:#1}}
\newcommand{\Tabref}[1]{Table~\ref{tab:#1}}
\newcommand{\tabref}[1]{Table~\ref{tab:#1}}

\newcommand{\figlabel}[1]{\label{fig:#1}}
\newcommand{\Figref}[1]{Figure~\ref{fig:#1}}
\newcommand{\figref}[1]{\mbox{Figure~\ref{fig:#1}}}

\newcommand{\eqlabel}[1]{\label{eq:#1}}
\newcommand{\eqref}[1]{(\ref{eq:#1})}

\newtheorem{thm}{Theorem}{\bfseries}{\itshape}
\newcommand{\thmlabel}[1]{\label{thm:#1}}
\newcommand{\thmref}[1]{Theorem~\ref{thm:#1}}

\newtheorem{lem}{Lemma}{\bfseries}{\itshape}
\newcommand{\lemlabel}[1]{\label{lem:#1}}
\newcommand{\lemref}[1]{Lemma~\ref{lem:#1}}

\newtheorem{cor}{Corollary}{\bfseries}{\itshape}
\newcommand{\corlabel}[1]{\label{cor:#1}}
\newcommand{\corref}[1]{Corollary~\ref{cor:#1}}

\newtheorem{obs}{Observation}{\bfseries}{\itshape}
\newcommand{\obslabel}[1]{\label{obs:#1}}
\newcommand{\obsref}[1]{Observation~\ref{obs:#1}}

\newtheorem{assumption}{Assumption}{\bfseries}{\rm}
\newenvironment{ass}{\begin{assumption}\rm}{\end{assumption}}
\newcommand{\asslabel}[1]{\label{ass:#1}}
\newcommand{\assref}[1]{Assumption~\ref{ass:#1}}

\newcommand{\proclabel}[1]{\label{alg:#1}}
\newcommand{\procref}[1]{Procedure~\ref{alg:#1}}

\newtheorem{rem}{Remark}
\newtheorem{op}{Open Problem}

\newcommand{\etal}{\emph{et al}}

\newcommand{\voronoi}{Vorono\u\i}
\newcommand{\ceil}[1]{\left\lceil #1 \right\rceil}
\newcommand{\floor}[1]{\left\lfloor #1 \right\rfloor}



\title{\MakeUppercase{Diameter and Smallest Enclosing Circle Peeling Depth}}
\author{Prosenjit Bose \and Pat Morin}
\date{}

\begin{document}
\maketitle
\begin{abstract}
We introduce two new notions of data depth based on a generalization
of the fact that the median is obtained by repeatedly removing the
minimum and maximum of a set until only one or two elements remain.
\end{abstract}

\section{Introduction}

Let $S$ be a set of $n$ points in $\mathbb{R}^d$.  The \emph{diameter} 
of a point set $d(S)$ is defined as
\[
    d(S)=\max\left\{\|xy\|:x,y\in S\right\} \enspace .
\]  
A pair of points $x$ and $y$ such that
$\|xy\|=D(S)$ is called a \emph{diametrical pair}. Let $D_0=S$ and
let $D_i$ be obtained from $D_{i-1}$ by removing a diametrical pair.
The \emph{diameter peeling depth} of $x$ with respect to $S$ is the
maximum value $i$ such that $x\in D_i$.

The smallest enclosing ball $B(S)$ is the smallest closed ball that
contains all points of $S$.  The radius of $B(S)$, denoted $r(S)$ is
called the \emph{radius} of $S$.  Note that, in general, $B(S)$ has
anywhere between $2$ and $d+1$ points of $S$ on its boundary.  Let
$B_0=S$ and let $B_i$ be obtained from $B_{i-1}$ by removing all
points on the boundary of $B(B_{i-1})$.  The \emph{smallest enclosing
ball peeling depth} of $x$ with respect to $S$ is the maximum value
$i$ such that $x\in B_{i}$.


\section{Diameter-Peeling Depth}

\begin{thm}
Let $S$ be a set of $n$ points in $\mathbb{R}^d$ and let $X\subset S$ and
$Y$ each be of size at most $n/2-1$.  Then the diameter-peeling center
of $S\cup Y\setminus X$ is contained in the convex hull of $S$.
\end{thm}


\begin{thm}
Let $S$ be a set of $n$ points in $\mathbb{R}^d$.  The
diameter-peeling depth of every point in $S$ can be computed in
$O(dn^2\log n)$ time.
\end{thm}

\begin{thm}
Let $S$ be a set of $n$ points in $\mathbb{R}^3$.  The
diameter-peeling depth of every
point in $S$ can be computed in $O(n\log^{O(1)} n)$ time.
\end{thm}

\begin{thm}
Let $S$ be a set of $n$ points in $\mathbb{R}^2$.  The
diameter-peeling depth of every point in $S$ can be computed in
$O(n\log n)$ time.
\end{thm}


\section{Smallest Enclosing Circle-Peeling Depth}

\begin{thm}
Let $S$ be a set of $n$ points in $\mathbb{R}^d$ and let $X\subset S$ and
$Y$ each be of size at most $n/3-1$.  Then the smallest enclosing
ball-peeling center of $S\cup Y\setminus X$ is contained in the convex
hull of $S$.
\end{thm}

\begin{thm}
Let $S$ be a set of $n$ points in $\mathbb{R}^d$.  The
smallest enclosing ball-peeling depth of every point in $S$ 
can be computed in $O(d!n^2)$ time.
\end{thm}

\begin{thm}
Let $S$ be a set of $n$ points in $\mathbb{R}^2$.  The
smallest enclosing ball-peeling depth of every point in $S$ 
can be computed in $O(n^{3/2}\log^{O(1)} n)$ time.
\end{thm}



\section{Conclusions}
\end{document}
