\documentclass[12pt]{article}
\usepackage{fullpage,amsthm,graphicx}

\newtheorem{lem}{Lemma}
\newtheorem{fact}{Fact}

\begin{document}

Let $a(X)$ denote area and let $g(X)=(x(X),y(X))$
denote center of gravity of a set $X$.

\begin{fact}
For two disjoint sets $X$ and $Y$
\[
    g(X\cup Y)=
         \frac{a(X)\times g(X) + a(Y)\times g(Y)}{a(X) +a(Y)}
\]
\end{fact}

\begin{lem}
Let $X$ be a convex body in the plane with unit area and center of gravity $g$,
let $\ell$ be any line through $g$, and let $p$ and $q$ denote the
intersections of $\ell$ with the boundary of $X$.  Then
\[
        \|pg\| / \|pq\| \ge 1/3 \enspace .
\]
\end{lem}

\begin{proof}
(Refer to the figure on the following page.)
Without loss of generality, assume $\ell$ is horizontal, $x(X)=0$, $p$
is to the left of $g$ and $q$ is to the right of $g$.  Consider the
line $m$ perpendicular to $\ell$ (vertical) that contains $g$.  The
line $m$ intersects the boundary of $X$ in two points $r$, above, and
$s$, below. Consider the unique triangle $T=tuq$ with $tu$ parallel to
a tangent of $X$ at $p$, with $r$ and $s$ on its boundary, and whose
center of gravity is on the line $m$.  Note that the point $g$ is
$1/3$ of the distance from the intersection $v$ of $\ell$ and $tu$ to the point
$q$.\footnote{This is a property of triangles that can be shown by
dissecting a triangle into 9 similar triangles.} Thus,
if the segment $tu$ intersects the interior of $X$ then the proof is
complete since then
\[
       \|pg\| = \|pv\|+\|vg\| = \|pv\| + (1/3)\|vq\| \ge
(1/3)(\|pv\|+\|vq\|) = (1/3)\|pq\| \enspace .
\]

Otherwise, $m$ partitions $X$ (respectively, $T$) into two parts.  Let
$X^+$ (respectively, $T^+$) denote the part of $X$ to the right of $m$
and let $X^-=X\setminus X^+$ (respectively, $T^-=T\setminus T^+$).
Observe that $a(T^+) < a(X^+)$ and that $x(T^+)\le x(X^+)$.  Also
$a(T^-) > a(X^-)$ and $x(T^-) < x(X^-)$.  Thus,
\begin{eqnarray*}
   x(T) & = & \frac{a(T^+)\times x(T^+)
                   + a(T^-)\times x(T^-)}{a(T)} \\
        & \le & \frac{a(X^+)\times x(X^+) 
                   + a(T^-)\times x(T^-)}{a(T)} \\
        & < & \frac{a(X^+)\times x(X^+)
                   + a(X^-)\times x(X^-)}{a(T)} \\
        & = & 0 \\
        & = & x(T)
\end{eqnarray*}
contradicting the assumption that $tu$ does not intersect $X$.
\end{proof}
\begin{center}
  \includegraphics{figure}
\end{center}


\end{document}
