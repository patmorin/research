%% -*- LaTeX -*- This is LaTeX2e code

\section{Other Space-Efficient Geometric Algorithms}

In this section, we present some simple observations about well-known
algorithms for three geometric problems and show how to space-efficiently
encode the results.

\subsection{Convex Hull of a Simple Polygon}

Using an space-efficient stack, e.g., along the lines of Br\"onnimann
\etal\'{}s~\cite{bronnimann:convex} description of Graham\'{}s
Scan~\cite{graham:efficient}, we can implement an optimal
space-efficient algorithm for computing the convex hull of a simple
polygon. To do so, we only need to observe that the only operations
performed in the algorithm described by Preparata and
Shamos~\cite[Chap. 4.1.4]{preparata:computational} are push and pop
operations.

\begin{lemma}
  The convex hull of a simple polygon with $n$ vertices can
  be computed \emph{in-place} in \Oh{n} time.
\end{lemma}

\subsection{Diameter of a Point Set}

Using the space-efficient algorithm of Br\"onnimann
\etal~\cite{bronnimann:convex}, we immediately obtain an optimal
space-efficient algorithm for computing the diameter of a point set.
This is due to the fact that the algorithm for enumerating antipodal
pairs as described by Preparata and
Shamos~\cite[Chap. 4.2.3]{preparata:computational} already is an
\emph{in-place} traversal of the boundary of the convex hull. To
encode the output without any extra space, we use an \emph{in-place}
partitioning algorithm, e.g. \cite{katajainen:partitioning}, to move
an antipodal pair determining the diameter to the start of the array
containing the points.

\begin{lemma}
  The diameter of a set of $n$ points in the plane can be computed
  \emph{in place} in \Oh{n \log_2 n} time. The encoding of the result
  does not need any extra space.
\end{lemma}

\subsection{Minimum Enclosing Circle}

A close look at Welzl\'{}s incremental
algorithm~\cite{welzl:smallest} shows that this algorithm already
works \emph{in-place}. We can modify the output of this algorithm such
that the array containing the points is reorganized in the following
way: Either the first three points determine the minimum enclosing
circle or the first two points determine the diameter of the circle.
Which of these cases applies, can be checked by first assuming that
the first two points determine the diameter and then checking whether
the third point is contained within this circle. As we can always
arrange the three points that determine the minimum enclosing circle
in such a way that the circle given by first two points does not
contain the third point (take the endpoints of the shortest edge in
the triangle formed by these points), this shows that the result
can be encoded without any extra space.

\begin{lemma}
  The Minimum Enclosing Circle of a set of $n$ points in the plane can
  be computed \emph{in-place} in \Oh{n} expected time. The encoding of
  the result does not need any extra space.
\end{lemma}

%%% Local Variables:
%%% mode: latex
%%% TeX-master: "paper"
%%% End:
