\documentclass{patmorin}
\usepackage{amsthm,amsmath,graphicx,stmaryrd}
\usepackage{pat}

\DeclareMathOperator{\asf}{asf}

\title{\MakeUppercase{Good Average Stretch Geometric Spanners}}
\author{Vida Dujmovi\'c, Pat Morin and Michiel Smid}


\begin{document}
\begin{titlepage}
\maketitle

\begin{abstract}
  The abstract goes here.
\end{abstract}

\end{titlepage}

\section{Introduction}

The \emph{average stretch factor} of a geometric graph, $G$, with vertex
set $V(G)\subset \R^d$ and edge set $E(G)$ is
\[
    \asf(G) = \binom{|V(G)|}{2}^{-1}\sum_{\{u,w\}\in\binom{V(G)}{2}}\frac{\|uw\|_G}{\|uw\|}
\]
\ldots

\section{The Construction}

\begin{thm}
  For every set, $V$, of $n<\infty$ points in $\R^2$, there exists a
  geometric graph $G$ with $V(G)$, $|E(G)|\in O(n)$ and $\asf(G)=1+o(1)$.
  Given $V$, the graph $G$ can be constructed in \ldots time.
\end{thm}

\begin{proof}
  We construct the graph $G$ by first adding the edges of some 2-spanner
  of $V$.  Next, we find a $(k/n)$-net $N\subset V$ with respect to
  equilateral triangles. The set $N$ has the property that any equilateral
  triangle that contains at least $k$ points of $S$ contains at least
  one point of $N$.  Halld\'orson and Tokuyama show that there exists
  such a set, $N$, of size $O(n/k)$, and this set can be computed in
  $O(n\log n)$ time.

  We then connect each point $u\in V\setminus N$ to its nearest neighbour
  in $N$.  For each point $w\in N$, consider a partition of $\R^2$
  into six 60 degree cones, $C_1(w)$,\ldots,$C_6(w)$, each having an
  apex at $w$.  For each $i\in \{1,\ldots,6\}$, let $\Delta_i(w)$ denote
  the minimal equaliateral triangle with two sides bounded by $C_i(w)$
  that contains all the neighbours of $w$ in $C_i(w)$.  Since $N$ is a
  $(k/n)$-net for equilateral triangles, it follows that $\Delta_i(w)$ 
  contains at most $k$ points.

  Let $\Delta_1,\ldots,\Delta_{n'}$ contain exactly the triangles in the
  set $\{\Delta_i(w):i\in\{1,\ldots,k\}, w\in N\}$ ordered by decreasing
  size.  The average stretch factor of $G$ can be expressed as
  \begin{align*}
    \asf(G) & = 
         \binom n2^{-1}\left(\sum_{i=1}^{n'}\sum_{u,w\in\Delta_i\cap S}\frac{\|uw\|_G}{\|uw\|}  
             + \sum_{i=1}^{n'-1}\sum_{j={i+1}}^{n'}\sum_{u\in\Delta_i\cap S}\sum_{w\in\Delta_j\cap S}\frac{\|uw\|_G}{\|uw\|}\right)  \\
      & = 
         O(k^2/n) + \binom n2^{-1}\left(  
             \sum_{i=1}^{n'-1}\sum_{j={i+1}}^{n'}\sum_{u\in\Delta_i\cap S}\sum_{w\in\Delta_j\cap S}\frac{\|uw\|_G}{\|uw\|}\right)  \\
   \end{align*}
\end{proof}



\section*{Acknowledgement}

The authors of this paper are partly funded by NSERC and CFI.

\section*{Authors}

\paragraph{Vida Dujmovi\'c.}
School of Mathematics and Statistics and Department of Systems and Computer Engineering, Carleton University
%, \texttt{vida@cs.mcgill.ca}

\paragraph{Pat Morin and Michiel Smid.}
School of Computer Scence, Carleton University
%, \texttt{\{morin,smid\}@scs.carleton.ca}


\bibliographystyle{plain}
\bibliography{template}





\end{document}


