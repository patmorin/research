\documentclass{patmorin}
\usepackage[noend]{algorithmic}

\title{Space-Efficient Geometric Divide-and-Conquer Algorithms}
\author{Prosenjit Bose \and
	Anil Maheshwari \and
	Pat Morin \and
	Jason Morrison \and
	Michiel Smid \and
	Jan Vahrenhold}
\date{}

\newcommand{\rt}{\mathrm{right\_turn}}

\begin{document}
\maketitle

\begin{abstract}
A \emph{space-efficient} algorithm is one which uses very little space
beyond the space in which the input is given.  We describe space
efficient algorithms for some classic geometric problems in the plane,
including closest-pairs, bichromatic closest-pair and orthogonal segment
intersection.  These algorithms avoid the use of complicated in-place
merging and partitioning algorithms by using the concept of
"reversible" operations.
\end{abstract}

\section{Reversible Operations}

\subsection{Convex Hull}

\begin{description}
\item{Input:} An array $A=A[0],\ldots,A[n-1]$ of points sorted by
$x$-coordinate.

\item{Output (Before Restoring):} The upper convex hull of these
points stored in $A[0],\ldots,A[h-1]$ and the interior points stored in
$A[h],\ldots,A[n-1]$ 

\item{Restored Output:} The original points of $A$ sorted by $x$
coordinate.
\end{description}

The following algorithm, given by Br\"onnimann etal produces the
desired output in linear time

\vspace{1ex}
\noindent\begin{minipage}{\textwidth}
\textsc{Graham-InPlace-Scan}$(A, n)$
\begin{algorithmic}[1]
\STATE{$h\gets 2$}
\FOR{$i\gets 2\ldots n$}
  \WHILE{$h\ge 2$ \textbf{and} \textbf{not} $\rt(S[h-2], S[h-1], S[i])$}
     \STATE{$h\gets h-1$ \COMMENT{ pop top element from the stack }}
  \ENDWHILE
  \STATE{\textbf{swap} $S[i]\leftrightarrow S[h]$}
  \STATE{$h\gets h+1$}
\ENDFOR
\STATE{\textbf{return} $h$}
\end{algorithmic}
\end{minipage}
\vspace{1ex}
 
The following algorithm undoes the operations of Graham's Scan

\vspace{1ex}
\noindent\begin{minipage}{\textwidth}
\textsc{Graham-InPlace-Undo}$(A, h, n)$
\begin{algorithmic}[1]
\FOR{$i\gets n,\ldots,2$}
  \STATE{$h\gets h-1$}
  \STATE{\textbf{swap} $A[i]\leftrightarrow A[h]$}
  \WHILE{$h < i$ \textbf{and} $A[h].x < A[h+1].x$}
	\STATE{$h\gets h+1$}
  \ENDWHILE
\ENDFOR
\end{algorithmic}
\end{minipage}
\vspace{1ex}
 
Finally, note that even if the value of $h$ is not known, it can be
computed easily in linear time since $A[h-1].x \ge A[h].x$ but
$A[i-1].x<A[i].x$ for all $1\le i< h$.

\subsection{Partitioning}

Suppose $A=A[0],\ldots,A[n-1]$ is an array of points sorted by
$y$-coordinate and we wish to partition it into two sets based on the
value $A[i].x$. The following algorithm is motivated by the 

\vspace{1ex}
\noindent\begin{minipage}{\textwidth}
\textsc{Graham-Partition}$(A, i, n)$
\begin{algorithmic}[1]
\STATE{\textbf{swap} $A[0]\leftrightarrow A[i]$}
\STATE{$h\gets 1$}
\FOR{$i\gets 1\ldots n-1$}
  \IF{$A[i].x < A[h-1].x$}
       \STATE{\textbf{swap} $A[i]\leftrightarrow A[h]$}
       \STATE{\textbf{swap} $A[h-1]\leftrightarrow A[i]$}
       \STATE{$h\gets h+1$}
  \ENDIF
\ENDFOR
\STATE{\textbf{return} $h$}
\end{algorithmic}
\end{minipage}
\vspace{1ex}

\vspace{1ex}
\noindent\begin{minipage}{\textwidth}
\textsc{Graham-Partition-Undo}$(A, h, n)$
\begin{algorithmic}[1]
\FOR{$i\gets n-1,\ldots,1$}
   \IF{$A[h-2].y > A[i].y$}
      \STATE{$h\gets h-1$}
      \STATE{\textbf{swap} $A[h-1]\leftrightarrow A[i]$}
      \STATE{\textbf{swap} $A[h-1]\leftrightarrow A[h]$}
   \ENDIF
\ENDFOR

\end{algorithmic}
\end{minipage}
\vspace{1ex}




\end{document}
