\documentclass[12pt]{article}
%\usepackage{amsthm}
\newtheorem{question}{Question}

\title{Rhythm Reconstruction Questions}
\author{Barbados 2008 Gang}

\begin{document}
\maketitle

\section{Reconstruction from Substrings}

For a string $S=s_0,\ldots,s_{n-1}$, let $S(i)$ denote the $s_{i\bmod
n}$.

Let $S_1,\ldots,S_n$ be bitstrings each of length $k$.  We say that a
string $S$ is a \emph{reconstruction} of $S_1,\ldots,S_n$ if there
exists a permutation $\pi_1,\ldots,\pi_n$ such that
$S_{\pi_i}(j)=S(i+j)$ for all $1\le i\le n$ and all $1\le j\le k$.

\begin{question}
What value of $k$ guarantees that $S_1,\ldots,S_n$ has at most one
reconstruction, modulo rotations?
\end{question}

Currently we know that $k\ge n-2$ guarantees a unique reconstruction
and $k=n/2-c$ for sufficiently large constant $c$ does not guarantee a
large reconstruction.

This question is related to reconstruction from substrings that occur
in computational molecular biology.  We should consult the relevant
literature about the shotgun algorithm and the shortest superstring
problem. 


\section{Reconstruction from Spectra}

A \emph{full spectrum} is a set of $n$ strings $S_1,\ldots,S_n$ over
the alphabet $\{0,1,2\}$, each of length $n/2$.  We say that a binary
string $S$ is a \emph{reconstruction} of a full spectrum
$S_1,\ldots,S_n$ if there exists a permutation $\pi_1,\ldots,\pi_n$
such that $S_{\pi_i}(j) = S(i-j) + S(i+j)$ for all $1\le i\le n$ and
all $0\le j\le n/2$.

\begin{question}
Does any full spectrum have at most one reconstruction, modulo
rotations?
\end{question}



\end{document}
