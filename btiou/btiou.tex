\documentclass[lotsofwhite]{patmorin}


\title{IOUs in BitTorrent}
\author{Pat Morin}

\newcommand{\field}[1]{\texttt{#1}}

\begin{document}
\maketitle

\begin{abstract}
We present a proposal for adding IOUs to the BitTorrent protocol.
\end{abstract}


\section{The Protocol Extensions}

The peer-to-peer part of the BitTorrent protocol should be extended
with two new kinds of messages

\subsection{The IOU Message}

The IOU message contains two parts, a \field{bytes} field indicating
how many bytes the client is offering to owe and a \field{checksum}
field that the receiver must store and resend in order to redeem the
IOU.  The form of the \field{checksum} field is left up to the
implementation. It is expected that the receiver of an IOU will store
these values on some persistent media along with the \field{peer\_id}
of the sender.

\noindent\textbf{Remark:}  The documentation of the BitTorrent
protocol says that \field{peer\_id} is a string of length 20 that is
generated at random at the start of a new download.   The proposed
change would require that this \field{peer\_id} be created at the time
of installation (or first use) of BitTorrent and continue to be used
for all BitTorrent connections.  This does not seem to cause any
problems.

\subsection{The UOME Message}

The UOME message contains two parts, a \field{peer\_id} and a
\field{checksum}.  The \field{checksum} field is copied from a
previous IOU message received from a peer that identified itself with
\field{peer\_id}.

\section{Semantics}

The intended meaning of an IOU message is:  ``I owe you \field{bytes}
bytes of data and I will do my best to repay these to you when I
receive a UOME message containing \field{checksum}.''  Similarly, the
meaning of a UOME message is:  ``You owe me some data and to prove it
here is the \field{checksum}.''

It is intended that a BitTorrent client should keep track of all IOUs
that it sends but for which it has not yet received a UOME message.
When a client receives a UOME message it should remove the
corresponding entry from its list of outstanding IOUs and adjust its
internal structures to treat the sender of the UOME message with the
appropriate priority.

Similarly, when a client receives an IOU message, it should store this
IOU along with the \field{peer\_id} of the sender so that it may
redeem this IOU in the future.  At the same time, it should adjust its
internal structures to treat the sender of the IOU message with the
appropriate priority.

\section{Implementation Issues}

One item that has been left out so far is how much the client should
increase the priority of an outgoing connection in response to an IOU
or a UOME message.  This is a delicate question because there is a
possibility of writing bad IOUs.  A misbehaved client could, in an
attempt to increase its download speed, happily write IOUs but never
change its behaviour upon receipt of a UOME.  Or, a client could
change its \field{peer\_id} at each session so that other clients
would never recognize it as the sender of a previous IOU.

To avoid these types of abuse, the value of IOUs must be carefully
tempered.  In particular, a client should not react the same way to a
peer





\end{document} 
