\documentclass[lotsofwhite]{patmorin}
\usepackage{fullpage,graphicx,graphics}

 
%\usepackage{amsthm}

\newcommand{\centeripe}[1]{\begin{center}\Ipe{#1}\end{center}}
\newcommand{\comment}[1]{}

\newcommand{\centerpsfig}[1]{\centerline{\psfig{#1}}}

\newcommand{\seclabel}[1]{\label{sec:#1}}
\newcommand{\Secref}[1]{Section~\ref{sec:#1}}
\newcommand{\secref}[1]{\mbox{Section~\ref{sec:#1}}}

\newcommand{\alglabel}[1]{\label{alg:#1}}
\newcommand{\Algref}[1]{Algorithm~\ref{alg:#1}}
\newcommand{\algref}[1]{\mbox{Algorithm~\ref{alg:#1}}}

\newcommand{\applabel}[1]{\label{app:#1}}
\newcommand{\Appref}[1]{Appendix~\ref{app:#1}}
\newcommand{\appref}[1]{\mbox{Appendix~\ref{app:#1}}}

\newcommand{\tablabel}[1]{\label{tab:#1}}
\newcommand{\Tabref}[1]{Table~\ref{tab:#1}}
\newcommand{\tabref}[1]{Table~\ref{tab:#1}}

\newcommand{\figlabel}[1]{\label{fig:#1}}
\newcommand{\Figref}[1]{Figure~\ref{fig:#1}}
\newcommand{\figref}[1]{\mbox{Figure~\ref{fig:#1}}}

\newcommand{\eqlabel}[1]{\label{eq:#1}}
\newcommand{\eqref}[1]{(\ref{eq:#1})}

\newtheorem{thm}{Theorem}{\bfseries}{\itshape}
\newcommand{\thmlabel}[1]{\label{thm:#1}}
\newcommand{\thmref}[1]{Theorem~\ref{thm:#1}}

\newtheorem{lem}{Lemma}{\bfseries}{\itshape}
\newcommand{\lemlabel}[1]{\label{lem:#1}}
\newcommand{\lemref}[1]{Lemma~\ref{lem:#1}}

\newtheorem{cor}{Corollary}{\bfseries}{\itshape}
\newcommand{\corlabel}[1]{\label{cor:#1}}
\newcommand{\corref}[1]{Corollary~\ref{cor:#1}}

\newtheorem{obs}{Observation}{\bfseries}{\itshape}
\newcommand{\obslabel}[1]{\label{obs:#1}}
\newcommand{\obsref}[1]{Observation~\ref{obs:#1}}

\newtheorem{assumption}{Assumption}{\bfseries}{\rm}
\newenvironment{ass}{\begin{assumption}\rm}{\end{assumption}}
\newcommand{\asslabel}[1]{\label{ass:#1}}
\newcommand{\assref}[1]{Assumption~\ref{ass:#1}}

\newcommand{\proclabel}[1]{\label{alg:#1}}
\newcommand{\procref}[1]{Procedure~\ref{alg:#1}}

\newtheorem{rem}{Remark}
\newtheorem{op}{Open Problem}

\newcommand{\etal}{\emph{et al}}

\newcommand{\voronoi}{Vorono\u\i}
\newcommand{\ceil}[1]{\left\lceil #1 \right\rceil}
\newcommand{\floor}[1]{\left\lfloor #1 \right\rfloor}



\newcommand{\boundary}{\partial}
\newcommand{\interior}{\mathrm{int}}
\newcommand{\z}[1]{{\hat{#1}}}
\newcommand{\depth}{\mathrm{depth}}

\begin{document}
\section{Introduction}


\section{Introduction}
\seclabel{intro}

The planar point location problem is one of the classic problems in
computational geometry. Given a planar subdivision $G$,\footnote{A
\emph{planar subdivision} is a partitioning of the plane into points
(called \emph{vertices}), open line segments (call \emph{edges}), and
open polygons (called \emph{faces}).} the planar point location
problem asks us to construct a data structure so that, for any query
point $p$, we can quickly determine which face of $G$ contains
$p$.\footnote{In the degenerate case where $p$ is a vertex or contained
in an edge of $G$ any face incident on that vertex/edge may be
returned as an answer.}

The history of the planar point location problem parallels, in many
ways, the study of binary search trees.  After a few initial attempts
\cite{dl76,lp77,p81}, asymptotically optimal (and quite different)
linear-space $O(\log n)$ query time solutions to the planar point
location problem were obtained by Kirkpatrick \cite{k83}, Sarnak and
Tarjan \cite{st86}, and Edelsbrunner \etal\ \cite{egs86} in the mid
1980s.  These results were based on hierarchical simplification, data
structural persistence, and fractional cascading, respectively.  All
three of these techniques have subsequently found many other
applications.  An elegant randomized solution, combining aspects of
all three previous solutions, was later given by Mulmuley \cite{m90},
and uses randomized incremental construction, a technique that has
since become pervasive in computational geometry
\cite[Section~9.5]{bcko08}.  Preparata \cite{p90} gives a
comprehensive survey of the results of this era.

In the 1990s, several authors became interested in determining the
exact constants achievable in the query time.  Goodrich \etal\
\cite{gor97} gave a linear-size data structure that, for any query,
requires at most $2\log n + o(\log n)$ point-line comparisons and
conjectured that this query time was optimal for linear-space data
structures. Here and throughout, logarithms are implicitly base 2
unless otherwise specified. The following year, Adamy and Seidel
\cite{as98} gave a linear-space data structure that answers queries
using $\log n + 2\sqrt{\log n} + O(\log\log n)$ point-line comparisons
and showed that this result is optimal up to the third term.

Still not done with the problem, several authors considered the point
location problem under various assumptions about the query
distribution.  All these solutions compare the expected query time to
the \emph{entropy bound};  in a planar subdivision with $f$ faces, if the query
point $p$ is chosen from a probability measure over $\R^2$ such that
$p_i$ is the probability that $p$ is contained in face $i$ of $G$,
then no algorithm that makes only binary decisions can answer queries
using an expected number of decisions that is fewer than 
\begin{equation}
    H(p_1,\ldots,p_f) = \sum_{i=1}^f p_i\log(1/p_i) \enspace . 
	\eqlabel{entropy}
	\eqlabel{entropy-face}
\end{equation}

In the previous results on planar point location, none of the query
times are affected significantly by the structure of $G$;  they hold
for arbitrary planar subdivisions.  However, when studying point
location under a distribution assumption the problem becomes more
complicated and the results become more specific.  A \emph{convex
subdivision} is a planar subdivision whose faces are all convex
polygons, except the outer face, which is the complement of a convex
polygon.  A \emph{triangulation} is a convex subdivision in which each
face has at most 3 edges on its boundary.

Note that, if every face of $G$ has a constant number of sides, then
$G$ can be augmented, by the addition of extra edges, so that it is a
triangulation without increasing \eqref{entropy} by more than a
constant.  Similarly, in a convex subdivision $G$ where the query
distribution $D$ is uniform within each face of $G$, the faces of the
subdivision can be triangulated without increasing the entropy by more
than a constant \cite{amm00}. Thus, in the following we will simply
refer to results about triangulations where it is understood that
these also imply the same result for planar subdivisions with faces of
constant size or convex subdivisions when the query distribution is
uniform within each face.

Arya \etal\ \cite{acmr00} gave two results for the case where the
query point $p$ is chosen from a known distribution where the $x$ and
$y$ coordinates of $p$ are chosen independently and $G$ is a convex
subdivision.  They give a linear space data structure for which the
expected number of point-line comparisons is at most $4H+O(1)$ and a
quadratic space data structure for which the expected number of
point-line comparisons is at most $2H+O(1)$.  The assumption about the
independence of the $x$ and $y$ coordinates of $p$ is crucial to the
these results.

For arbitrary distributions that are known in advance, several results
exist.  Iacono \cite{i01,i04} showed that, for triangulations, a
simple variant of Kirkpatrick's original point location structure
gives a linear space, $O(H+1)$ expected query time data structure.  A
result by Arya \etal\ \cite{amm00} gives a data structure for
triangulations that uses $H + O(H^{2/3}+1)$ expected number of
comparisons per query and $O(n\log n)$ space.  The space requirement
of this latter data structure was later reduced, by the same authors,
to $O(n\log^* n)$ \cite{amm01a}.  The same three authors
\cite{amm01b} also showed that a variant of Mulmuley's randomized algorithm
gives, for triangulations, a simple $O(H+1)$ expected query time,
linear space data structure.  Very recently, Arya \etal\
\cite{ammw07}, have given an $O(n)$ space structure for point-location
in triangulations with query time $H+O(H^{1/2}+1)$. 

In a preliminary version of this paper, the current authors
\cite{soda07} showed that, when $G$ is a convex subdivision, there
exists a data structure of size $O(n)$ that can answer point location
queries using $\tilde H + O(\tilde H^{2/3}+1)$ point/line comparisons.
Here, $\tilde H=\tilde H(G,D)$ is a lower bound on the expected cost
of point location for any linear decision tree for point location in
$G$ when the query point $p$ is drawn from $D$.  In general, $\tilde
H$ is larger than $H$.

In the current paper, we show that, for any connected planar
subdivision (not necessarily convex), there exists a data structure of
size $O(n)$ that can answer point location queries using $\tilde H +
O(\tilde H^{2/3}+1)$ point/line comparisons.  Again, $\tilde H=\tilde
H(G,D)$ is a lower bound on the expected cost of any linear decision
tree that solves this problem.  We achieve this result by showing how
to compute a near-minimum-entropy Steiner triangulation $\Delta$ of
$G$ and then proving that the entropy of a minimum-entropy Steiner
triangulation of $G$ is a lower bound on the cost of any linear
decision tree for point location in $G$.  By then applying the recent
result of Arya \etal\ to the triangulation $\Delta$ we obtain nearly
matching upper and lower bounds.

Note that all known algorithms for planar point location that do not
place special restrictions on the input subdivision can be described
in the linear decision tree model of computation.\footnote{Although
significant breakthroughs have recently been made in this area
\cite{c06,p06}, we deliberately do not survey algorithms that require
the vertices of the subdivision to be on integer coordinates.}  The
data structures presented in the current paper are the most general
results known about planar point location and imply, to within a
lower order term, all of the results discussed in the introduction.

The remainder of this paper is organized as follows:  \Secref{prelim}
presents definitions and notations used throughout the paper.
\Secref{polygons} shows how to compute a near-minimum-entropy
triangulation of a simple polygon.  Finally, \Secref{subdivisions}
presents our point location structure for connected planar
subdivisions.

\section{Definitions}
Polygon, Reflex chain, pseudotriangle, 2-convex pseudotriangle,
geodesic triangle, partition, $V(T)$, $\z t$, decision trees,
classification problems, linear decision trees, $\depth(\ell)$.

\section{Minimum Entropy Triangulations} 

Let $P$ be a simple polygon with $n$ vertices, denoted
$p_0,\ldots,p_{n-1}$ as they occur, in counterclockwise order, on the
boundary of $P$.  We denote the boundary of $P$ by $\boundary P$ and
the interior of $P$ by $\interior(P)$.  We will show how to find a
triangulation of $P$ that has near-minimum entropy.  That is, we will
find a triangulation $\Delta=\Delta(P,D)$ such that $H(\Delta) =
\sum_{t\in\Delta} \Pr(t|P)\log(1/\Pr(t|P))$ is near-minimum over all
triangulations of $P$.  

\subsection{The Triangulation $\Delta=\Delta(P,D)$}

Our triangulation algorithm is recursive and takes as input a polygon
$P$ and a reflex chain $p_i,\ldots,p_j$ on the boundary of $P$.  If
$P$ is a triangle, then there is nothing to do, so the algorithm
outputs $P$ and terminates. Otherwise, the
algorithm first selects a point $p_k$ on the boundary of $P$ and adds
all the edges of the geodesic triangle $t=\triangle p_ip_jp_k$ to the
triangulation $\Delta$.
Observe that removing $t$ from $P$ disconnects $P$ into
components $P_1,\ldots,P_m$ where $P_i$ is a polygon that shares a
reflex chain $C_i$ with the pseudotriangle $t$ (see
\figref{ii}).  The point
$p_k$ is selected in such a way that, for all $i\in\{1,\ldots,m\}$, $\Pr(P_i)
\le (1/2)\Pr(P)$.\footnote{The existence of such a point $p_k$ is
readily established by a standard continuity argument; see Bose \etal
\cite{bXX} for an example.} Each of the sub-polygons $P_1,\ldots,P_m$
can then be triangulated recursively by applying the algorithm to
$P_i$ and the reflex chain $C_i$.

\begin{figure}
  \begin{center}
    \begin{tabular}{cc}
      \includegraphics{ii-1} & \includegraphics{ii-2} 
    \end{tabular}
  \end{center}
  \caption{The geodesic triangle $t=\triangle p_i p_j p_k$ partitions $P$ into several
pieces $P_1,\ldots,P_m$.}
  \figlabel{ii}
\end{figure}


To complete the triangulation $\Delta$ all that remains is to
partition $\z t$ into triangles.  To do this, we first
partition $\z t$ into at most 1 triangle $t'$ and 3 2-convex
pseudotriangles $t_0,t_1,t_2$ as shown in \figref{pt-partition}.a. Let
$Q_i$ be the connected component of $(P\setminus \z t)\cup t_i$ that
contains $t_i$.  To complete the triangulation we will partition
$t_i$ into triangles, for each $i\in\{0,1,2\}$, using a recursive
algorithm.  This algorithm selects an edge $e_i$ of the reflex chain
in $t_i$ and
extends $e_i$ in both directions until it reaches the boundary of $t_i$
(see \figref{pt-partition}.b).  The resulting line segment partitions
$t_i$ into a triangle $t_i'$, and 2 2-convex pseudotriangles $t_{i,0}$
and
$t_{i,1}$.  At the same time, $Q_i$ is partitioned into up to 4 pieces
(see \figref{pt-partition}.c):
\begin{enumerate}
\item The triangle $t_i'$, and
\item A subpolygon $P_j$ incident to $e_i$,
\item The two connected components $Q_{i,0}$
and $Q_{i,1}$ of $Q_i\setminus t_i'$ that contain $t_{i,0}$ and
$t_{i,1}$, respectively.
\end{enumerate}
The edge $e_i$ is selected
so that $\Pr(Q_{i,b})\le (1/2)\Pr(Q_i)$ for each
$b\in\{0,1\}$.\footnote{The existence of such an edge $e_i$ is assured
by yet another continuity argument.}  This
completes the description of the triangulation $\Delta$.  A partially
completed triangulation is show in \figref{delta-example}.

\begin{figure}
  \begin{center}
    \begin{tabular}{ccc}
      \includegraphics{pt-partition} & 
      \includegraphics{t-partition} &
      \includegraphics{t-partition-2} \\
      (a) & (b) & (c)
    \end{tabular}
  \end{center}
  \caption{Partitioning (a) a pseudotriangle $t$ into 3 2-convex
pseudotriangles $t_0,t_1,t_2$ and one triangle $t'$ (b) a 2-convex
pseudotriangle $t_i$ into one triangle $t_i'$ and 2 2-convex
pseudotriangles $t_{i,0}$ and $t_{i,1}$, and (c) $Q_i$ into 4 pieces.}
  \figlabel{pt-partition}
\end{figure}

\begin{figure}
  \begin{center}
      \includegraphics{delta-example}
  \end{center}
  \caption{The triangles obtained during the first level of recursive
triangulation.  The yellow subpolygons are triangulated recursively.}
  \figlabel{delta-example}
\end{figure}

\subsection{Analysis of $H(\Delta)$}

In order to study the entropy of triangulation $\Delta$ defined above,
we will impose a tree structure on the pieces of $P$ induced by the
triangles in $\Delta$.  The $\Delta$-tree $T=T(P,D)$ for $P$ is a tree
whose nodes are subpolygons of $P$ and which has the property that,
for any node $y$ that is the child of a node $x$, $y\subseteq x$. The
root $r$ of $T$ is the polygon $P$.  The root of $T$ has the following
children (defined in terms of the construction algorithm in the
previous section; see \figref{delta-tree}):
\begin{enumerate}
\item Each subpolygon $P_i$ whose boundary does not share a segment
      with $\z t$ is a child of $r$.  
\item In addition, the subpolygons $Q_0,Q_1,Q_2$ are each children
     of $r$.  The subtree rooted at $Q_i$ is a binary tree corresponding
     to the recursive partitioning of $t_i$ and $Q_i$ done by the algorithm.
\end{enumerate}
Note that the above definition yields a tree whose leaves are the
subpolygons $P_1,\ldots,P_m$  obtained by removing $t$ from $P$.  The
subtree rooted at each such leaf is therefore obtained recursively
from the recursive triangulation of $P_i$.

\begin{figure}
 \begin{center}\includegraphics{delta-tree}\end{center}
   \caption{The $\Delta$-tree $T$.  Yellow leaves in this tree are the
     root of subtrees obtained recursively.  Grey areas show the
portions of a node not covered by its children.}
  \figlabel{delta-tree}
\end{figure}

In the tree $T$, we will distinguish between two different kinds of
edges.  The edges joining $r$ to $Q_i$, for each $i\in\{0,1,2\}$ are
called \emph{red edges}, and the remaining edges are called
\emph{black} edges.  We call a node $x$ a \emph{black child} if the
edge from $x$ to its parent is black, otherwise we call $x$ a
\emph{red child}.

\begin{lem}\lemlabel{t-halving}\lemlabel{A}
Let $T=T(P,D)$ be the $\Delta$-tree for $(P,D)$ and
Let $x$ and $y$ be nodes in $T$ where $x$ is the parent of $y$ and
$xy$ is a black edge.  Then $\Pr(y)\le(1/2)\Pr(x)$.
\end{lem}

\begin{proof}
This follows immediately from the definition of $T$.
\end{proof}

\begin{lem}\lemlabel{t-red-black}
Let $T=T(P,D)$ be the $\Delta$-tree for $(P,D)$ and
Let $x$ be a node at depth $i$ in $T$.  Then the number of black edges
on the path from the root of $T$ to $x$ is at least $\floor{i/2}$.
\end{lem}

\begin{proof}
This follows from the fact that any red child has only (2,
1, or 0) black children.
\end{proof}


\begin{lem}\lemlabel{t-depth}
Let $T=T(P,D)$ be the $\Delta$-tree for $(P,D)$ and
let $x$ be a node at depth $i$ in $T$.  Then $\Pr(x)\le 1/2^{\floor{i/2}}$.
\end{lem}

\begin{proof}
This is an immediate consequence of \lemref{t-halving} and
\lemref{t-red-black}.
\end{proof}

We call a set of nodes $\{x_1,\ldots,x_m\}$ in $T$ RB-independent if 
for each $i,j\in\{1,\ldots,m\}$, $i\neq j$, $x_i$ is not an ancestor
of $x_j$ and the path from $x_i$ to $x_j$ in $T$ contains at least one
black edge.

\begin{lem}\lemlabel{t-intersect-rb-indep}
Let $T=T(P,D)$ be the $\Delta$-tree for $(P,D)$ and
let $\{x_1,\ldots,x_m\}$ be an RB-independent set of vertices in $T$.
Then any triangle contained in $P$ intersects at most 3 elements of
$w_1,\ldots,w_m$.
\end{lem}

\begin{proof}
Coming soon.
\end{proof}

\begin{lem}\lemlabel{t-big-rb-indep}
Let $S=\{x_1,\ldots,x_m\}$ be a set of nodes in $T$ each of depth at
most $i$.  Then there exists an RB-independent subset of $S$ whose
size is at least $m/3i$.
\end{lem}

\begin{proof}
Simply select elements greedily from $S$. When the element $x_i$ is
selected, remove from consideration all ancestors of $x_i$ and all red
children of these ancestors.  Since each step selects one node and 
eliminates at most $3(i-1)+1$ nodes from consideration, this process 
will select at least $m/((3(i-1)+1) \ge m/3i$ nodes.
\end{proof}

\subsection{Minimum-Entropy Triangulation}

Next, we show that the triangulation $\Delta$ defined above is
nearly-minimum entropy over all possible triangulations of $P$.  We do
this by developing a technique for lower-bounding the entropy of one
triangulation in terms of the entropy of another triangulation
(\lemref{pieces}).  We then show how to apply this technique to lower
bound the entropy of any triangulation $\Delta^*$ in terms of the
entropy of $\Delta$ (\lemref{partition} and \lemref{min-H-triangulation}).
  
To obtain lower bounds on the entropy of a triangulation $\Delta^*$,
consider the following easily proven observation: If each triangle in
$\Delta^*$ intersects at most $c$ triangles of some triangulation
$\Delta$ then $H(\Delta^*) \ge H(\Delta) - \log c$.\footnote{Proof:
Consider the set $X=\{ t^*\cap t : t^*\in\Delta^*, t\in \Delta\}$.
Each triangle of $\Delta^*$ contributes at most $c$ pieces to $X$, so
we have $H(\Delta) \le H(X) \le H(\Delta^*) + \log c$.}  This
observation allows us to use $\Delta$ to prove a lower bound on the
entropy of a triangulation $\Delta^*$.  Unfortunately, the condition
that each triangle of $\Delta^*$ intersect at most $c$ triangles of
$\Delta$ is too restrictive for our purposes.  Instead, we require
following stronger result:

\begin{lem}\lemlabel{pieces}
Let $D$ be a probability measure over $\R^2$.  Let $\Delta$ and
$\Delta^*$ be triangulations, and let $\{\Delta_1,\ldots,\Delta_m\}$
be a partition of the triangles in $\Delta$.  Suppose that, for all
$i\in\{1,\ldots,m\}$ and for each triangle $t^*\in\Delta^*$, $t^*$
intersects at most $c$ triangles in $\Delta_i$.  Then
\begin{eqnarray*}
   H(\Delta) \le 
	 H(\Delta^*) + H(\{\cup\Delta_1,\ldots,\cup\Delta_m\}) + \log c
 \enspace . 
\end{eqnarray*}
\end{lem}

Intuitively, \lemref{pieces} can be thought of as follows:  If we tell
an observer which of the $\Delta_i$ a point $p$ drawn according to $D$
occurs in then the amount of information we are giving the observer
about the experiment is at most
$H(\{\cup\Delta_1,\ldots,\cup\Delta_m\})$.  However, after giving away
this information, we are able to apply the simple observation in the
previous paragraph, since each triangle in $\Delta^*$ intersects at
most $c$ elements of each $\Delta_i$.  Thus, \lemref{pieces} is really
just $m$ applications of the simple observation.  The following proof
formalizes this:

\begin{proof} Help Vida: Justify the first line of the following please 
--- I'm stuck.
Intuition: LHS represents picking a point $p$ according to $D$ and
asking how much uncertainty there is about which triangle of
$\Delta^*$ contains $p$.  The RHS represents picking $p$, telling you
which of the $\Delta_i$ contains $p$ and then asking how much
uncertainty remains about which triangle of $\Delta^*$ contains $p$.
Surely knowing which of the $\Delta_i$ contains $p$ does not increase
the uncertainty.
\begin{eqnarray*}
   H(\Delta^*) 
     & \ge & \sum_{i=1}^m \Pr(\Delta_i)H(\Delta^*|\Delta_i) \\
     &  =  & \sum_{i=1}^m \Pr(\Delta_i)\sum_{t^*\in\Delta^*}
       \Pr(t^*|\Delta_i)\log(1/\Pr(t^*|\Delta_i)) \\
     &  =  & \sum_{i=1}^m \sum_{t^*\in\Delta^*}
       \Pr(t^*\cap\Delta_i)\log(\Pr(\Delta_i)/\Pr(t^*\cap\Delta_i)) \\
     &  =  & \sum_{i=1}^m \sum_{t^*\in\Delta^*}
       \Pr(t^*\cap\Delta_i)\log(1/\Pr(t^*\cap\Delta_i)) 
        + \sum_{i=1}^m \Pr(\Delta_i)\log(\Pr(\Delta_i)) \\
     &  =  & \sum_{i=1}^m \sum_{t^*\in\Delta^*}
       \Pr(t^*\cap\Delta_i)\log(1/\Pr(t^*\cap\Delta_i)) 
        - H(\{\cup\Delta_1,\ldots,\cup\Delta_m\}) \\
     & \ge  & \sum_{i=1}^m \sum_{t^*\in\Delta^*}\sum_{t\in\Delta_i}
       \Pr(t^*\cap t)\log(1/\Pr(t^*\cap t)) 
        -\log c - H(\{\cup\Delta_1,\ldots,\cup\Delta_m\}) \\
     & \ge  & \sum_{i=1}^m \sum_{t\in\Delta_i}
       \Pr(t)\log(1/\Pr(t)) 
        -\log c - H(\{\cup\Delta_1,\ldots,\cup\Delta_m\}) \\
     &  =  & H(\Delta) -\log c - H(\{\cup\Delta_1,\ldots,\cup\Delta_m\}) 
            \enspace ,
\end{eqnarray*}
and this completes the proof.
\end{proof}

The remainder of our argument involves carefully partitioning the
triangles of $\Delta$ into subsets $\Delta_1,\ldots,\Delta_m$ that are
compatible with \lemref{pieces}, and then showing that
$H(\Delta_1,\ldots,\Delta_m)$ is not too big.  To help us, we will use
the $\Delta$-tree $T$.  For a node $x$ in $T$ with children
$x_1,\ldots,x_k$, let $t(x) = x \setminus (\bigcup_{i=1}^m x_i)$ be
the portion of $x$ not covered by $x$'s children.  Note that $t(x)$ is
always either the empty set or is a triangle in $\Delta$.  In fact,
for every triangle $t\in\Delta$, there is exactly one $x\in V(T)$ such
that $t(x)=t$, and for every $x\in V(T)$ such that $t(x)$ is non-empty
there is exactly one $t\in\Delta$ such that $t(x)=t$.  This implies
that\footnote{Here, and throughout the remainder, we slightly abuse notation by
using the convention that $0\cdot\log(1/0)=0$.}
\[
    H(\Delta) = \sum_{t\in\Delta}\Pr(t|P)\log(1/\Pr(t|P)) =
       \sum_{x\in V(T)}\Pr(t(x)|P)\log(1/\Pr(t(x)|P)) \enspace .
\]

For a node $x\in V(T)$, we define $\Pr(x)=\Pr(t(x))$ is the
probability that a point drawn from $D$ is contained in $t(x)$.
We begin by partitioning the nodes of $T$ into
\emph{groups} $G_1,G_2,\ldots$ where
\[
	G_i = \{x\in V(T) : 1/2^{i} \le \Pr(x) < 1/2^{i-1} \} \enspace .
\]
In what follows, we fix some a real number $0< \alpha < 1$ that will be
specified later.   The following lemma gives the structure of the sets
that will be used in the application of \lemref{pieces}.

\begin{lem}\lemlabel{partition}
Each group $G_i$ can be partitioned into $r_i$ subgroups
$G_{i,1},\ldots,G_{i,r_i}$ such that
\begin{enumerate}
\item There is an integer $t_i$ such that $r_i-t_i\le 2^{\alpha i}$,

\item $|G_{i,j}| \ge 2^{\alpha i} / 6i$, for all $j\in\{1,\ldots,r_i\}$, and 

\item $G_{i,j}$ is RB-independent, for all $j\in\{1,\ldots,r_i\}$.
\end{enumerate}
\end{lem}

\begin{proof}
Note that \lemref{t-depth} implies that any node in group $G_{i}$ has
depth at most $2i+1$ in $T$.  Therefore, as long as 
$|G_i|>2^{\alpha i}$, we can apply \lemref{t-big-rb-indep} to obtain
an RB-independent subset of $G_i$ of size at least $2^{\alpha i}/6i$,
make this into a group $G_{i,j}$, and remove these elements from $G_i$.
We repeat this until, after $t_i$ iterations, $|G_i| \le 2^{\alpha
i}$.  At this point we simply create $|G_i|$ singleton groups
$G_{i,t_i},\ldots,G_{i,r_i}$ and observe that
all groups $G_{i,j}$ are RB-independent (Condition~3),
$G_{i,1},\ldots,G_{i,t_i}$ satisfy Condition~2, and $r_i$ and $t_i$ satisfy
Condition~1.
\end{proof}

Next we apply \lemref{pieces} and \lemref{partition} to obtain a lower
bound on the entropy of any triangulation $\Delta^*$. 

\begin{lem}\lemlabel{min-H-triangulation}\lemlabel{Z}
Let $P$ be a simple polygon, let $D$ be a probability measure over
$\R^2$, and consider the triangulation $\Delta=\Delta(P,D)$.
Then, for any triangulation $\Delta^*$ of $P$,
\[
    H(\Delta) \le H(\Delta^*) + O(H(\Delta)^{2/3}+1) \enspace .
\]
\end{lem}

\begin{proof}
Let $T=T(P,D)$ be the $\Delta$-tree for $(P,D)$ and apply
\lemref{partition} to obtain groups $\{G_{i,j}: i=1,\ldots,\infty,
j=1,\ldots,r_i\}$.  For any group $G_{i,j}$ Condition~3 and
\lemref{t-intersect-rb-indep} ensures that any triangle of
$\Delta^*$ can intersect at most 3 triangles of $G_{i,j}$.  Therefore,
applying \lemref{pieces} with $c=3$, we obtain:
\[ 
 H(\Delta) \le 
   H(\Delta^*) + H(\{{\cup G_{i,j}} : i\in\N , j\in\{1,\ldots,r_i\}\}) 
   + O(1)  \enspace .
\]
Thus, all that remains is to show that the contribution of $\overline
H=H(\cup\{G_{i,j}\} : i\in\N , j\in\{1,\ldots,r_i\}\})$ is at most
$O(H(\Delta)^{2/3})$. 
\begin{eqnarray*}
\overline H 
 & = & H(\cup\{G_{i,j}\} : i\in\N , j\in\{1,\ldots,r_i\}\}) \\
 & = & \sum_{i=1}^\infty
         \sum_{j=1}^{r_i}\Pr(\cup G_{i,j})\log(1/\Pr(\cup G_{i,j})) \\
 & = & \sum_{i=1}^\infty
         \sum_{j=1}^{t_i}\Pr(\cup G_{i,j})\log(1/\Pr(\cup G_{i,j})) 
         + \sum_{i=1}^\infty
         \sum_{j=t_i+1}^{r_i}\Pr(\cup G_{i,j})\log(1/\Pr(\cup G_{i,j})) \\
 & \le & \sum_{i=1}^\infty
         \sum_{j=1}^{t_i-1}\Pr(\cup G_{i,j})\log(1/\Pr(\cup G_{i,j})) 
         + \sum_{i=1}^{\infty} (2^{\alpha i}/ 2^{i-1})\log(2^i) \\
 &  =  & \sum_{i=1}^\infty
         \sum_{j=1}^{t_i-1}\Pr(\cup G_{i,j})\log(1/\Pr(\cup G_{i,j})) 
         + 2\cdot\sum_{i=1}^{\infty} i/2^{(1-\alpha)i} \\
 &  =  & \sum_{i=1}^\infty
         \sum_{j=1}^{t_i-1}\Pr(\cup G_{i,j})\log(1/\Pr(\cup G_{i,j}))
         + O(1/(1-\alpha)^2) \enspace ,
\end{eqnarray*}
where the last equality is obtained using Taylor series.
Continuing, we get 
\begin{eqnarray*}
\overline H
 &  =  & \sum_{i=1}^\infty
         \sum_{j=1}^{t_i-1}\Pr(\cup G_{i,j})\log(1/\Pr(\cup G_{i,j}))
         + O(1/(1-\alpha)^2) \\
 & \le  & \sum_{i=1}^\infty
         \sum_{j=1}^{t_i-1} \Pr(\cup G_{i,j})
              \log(i2^{i}/2^{\alpha i})
         + O(1/(1-\alpha)^2) \\
 & \le  & \sum_{i=1}^\infty
         \sum_{j=1}^{t_i-1} \Pr(\cup G_{i,j})((1-\alpha)i+\log i)
         + O(1/(1-\alpha)^2) \\
 & \le  & \sum_{i=1}^\infty
         \sum_{j=1}^{t_i-1} \Pr(\cup G_{i,j})((1-\alpha)\log(1/\Pr(\cup G_{i,j})) + \log\log(1/\Pr(\cup G_{i,j})))
         + O(1 + 1/(1-\alpha)^2) \\
 & \le &  (1-\alpha)H(\Delta) + \log(H(\Delta)) + O(1 + 1/(1-\alpha)^2) \\
 & \le &  O(H(\Delta)^{2/3}+ 1)
\end{eqnarray*} 
where the second last inequality follows from Jensen's Inequality and the
last inequality is obtained by setting $\alpha=1-1/H(\Delta)^{1/3}$.
This completes the proof.
\end{proof}

\lemref{min-H-triangulation} shows that the triangulation
$\Delta=\Delta(P,D)$ defined previously is nearly minimum-entropy over
all triangulations of $P$.  The following theorem gives an algorithmic
version of \lemref{min-H-triangulation}

\begin{thm}\thmlabel{min-H-triangulation}
Let $P$ be a simple polygon with $n$ vertices, and let $D$ be a
probability measure over $\R^2$.  Then there exists an $O(n\log n)$
time algorithm that computes a triangulation $\Delta'$ of $P$ having
$O(n)$ triangles and such that, for any triangulation $\Delta^*$ of
$P$,
\[
    H(\Delta') \le H(\Delta^*) + O(H(\Delta')^{2/3}+1) \enspace .
\]
\end{thm}

\begin{proof}
To construct $\Delta$ we first find the third vertex $p_k$ of the
geodesic triangle $t=\triangle p_i p_j p_k$.  This can be accomplished
in $O(n)$ time by computing the shortest path trees from
$p_i$ and $p_j$ to all other vertices of $P$ and using these to find $p_k$.
For an example of a similar computation, see Bose \etal \cite[Lemma~X]{BXX}.

Next, $\z t$ is split into 3 2-convex pseudotriangles $t_0,t_1,t_2$,
which is easily accomplished in $O(n)$ time.  The last step, before
recursing, is to triangulate each of $t_0,t_1,t_2$.  This step can be
accomplished in $O(n)$ time using a 2-sided exponential searching
trick that was used by Mehlhorn \cite{mXX} in the construction of
biased binary search trees (see also, Dujmovi\'c \etal
\cite[Lemma~X]{soda07}).

Finally, the algorithm recurses on each of the pieces
$P_1,\ldots,P_m$.  In this way, we obtain a divide-and-conquer
algorithm for constructing $\Delta$.  Unfortunately, this algorithm
may have running time $\Omega(n^2)$ since there is no bound
significantly smaller than $n$ on the size of an individual subproblem
$P_i$.  To overcome this, before recursing on a subproblem $P_i$ we
check if it contains more than $n/2$ vertices.  If so, then rather
than recursing normally on $P_i$ we choose a geodesic triangle of
$P_i$ whose removal leaves only subpolygons $P_{i,1},\ldots,P_{i,m_i}$
each with at most $n/2$ vertices.  This modification then yields an
algorithm whose recursion tree has depth $O(\log n)$ and at which the
work done at each level is $O(n)$, so the total running time of this
algorithm is $O(n\log n)$.

Note that this algorithm yields a triangulation $\Delta'$ that is
different from $\Delta$.  In particular, there may exist one $P_{i,j}$
with $\Pr(P_{i,j})>\Pr(P_i)/2$.  In this case, $P_{i,j}$ would become
a red child of its parent in the $\Delta'$-tree.  Despite this, all the
proofs of \lemref{A}-\lemref{Z} continue to hold almost without
modification.  The only difference occurs in \lemref{t-red-black} only
guarantees a bound of $\floor{i/3}$ on the number of black edges, but
this has almost no effect on subsequent computations.

Help: Can someone offer a short proof that the number of vertices is
$O(n)$, perhaps by arguing that the number of leaves in $T$ is at most
$2n$?
\end{proof}

\section{Point Location in Simple Planar Subdivisions}

Next, we consider the problem of point location in simple
subdivisions.  The following theorem of Arya \etal\ \cite{ammw07}
shows that triangulations can be used to give upper bounds on the cost
of a point location structure:

\begin{thm}[Arya \etal\ 2007]\thmlabel{ammw07}
Let $D$ be a probability measure over $\R^2$ and let $\Delta$ be a
triangulation of $\R^2$ having a total of $n$ triangles.  Then there exists a
data structure of size $O(n)$ that can be constructed in $O(n\log n)$
time, and for which the expected number of point/line comparisons
required to locate the face of $G$ containing a query point $p$, drawn
according to $D$, is $H(\Delta) + O(H(\Delta)^{1/2}+1)$.
\end{thm}

The following lemma shows that the minimum-entropy triangulation can
be used to give a lower bound on the cost of any point location
structure.

\begin{lem}\lemlabel{triangulate}
Let $T^*$ be any linear decision tree for a classification problem
$\mathcal{P}$ over $\R^2$.  Then there exists a linear decision tree
$T'$ for $\mathcal{P}$, such that, for each leaf $\ell$ of $T'$,
$r(\ell)$ is a triangle and $T'$ satisifies
\[
    \mu_D(T') \le \mu_D(T^*) + O(\log\mu_D(T^*))
\]
for any probability measure $D$ over $\R^2$.
\end{lem}

\begin{proof}
Each leaf $\ell$ of $T^*$ has a region $r(\ell)$ that is a convex
polygon.  If $r(\ell)$ has $k$ sides then the depth of $\ell$ in $T$
is at least $k$.  To obtain the tree $T'$ replace each such leaf
$\ell$ of $T^*$ by a balanced binary tree of depth $O(\log k)$ by
repeatedly splitting the leaf into two children $\ell_1$ and $\ell_2$
whose regions have $\ceil{(k+2)/2}$ and $\floor{(k+2)/2}$ vertices.
For a leaf $\ell\in L(T)$, let $s(\ell)$ denote the set of leaves in $T'$
in the subtree of $\ell$.   Then
\begin{eqnarray*}
   \mu_D(T^*) 
     &  =  & \sum_{\ell\in L(T^*)} \Pr(r(\ell))\cdot \depth(\ell) \\
     &  =  & \sum_{\ell\in L(T^*)}\sum_{\ell'\in s(\ell)} 
              \Pr(r(\ell'))\cdot \depth(\ell) \\
     & \ge & \sum_{\ell\in L(T^*)} 
             \sum_{\ell'\in s(\ell)}\Pr(r(\ell'))\cdot (\depth(\ell')
                   - O(\log (\depth(\ell)))) \\
     &  =  & \mu_D(T') - \sum_{\ell\in L(T^*)} 
             \sum_{\ell'\in s(\ell)}\Pr(r(\ell'))\cdot O(\log (\depth(\ell))) \\
     &  =  & \mu_D(T') - \sum_{\ell\in L(T^*)} 
             \Pr(r(\ell))\cdot O(\log (\depth(\ell))) \\
     & \ge & \mu_D(T') - O(\log(\mu_D(T^*)) \enspace , 
\end{eqnarray*}
where the last inequality is an application of Jensen's Inequality.
\end{proof}

\lemref{triangulate} says that for any linear decision tree for point
location, there is an underlying triangulation.  The entropy of this
triangulation gives a lower bound on the cost of the decision tree.
Thus, one way to give lower bounds for linear decision trees is to
give lower bounds on the entropy of a minimum-entropy triangulation.

Now, our point location structure is simple.  Let $G$ be a connected
planar subdivision whose faces are $F_1,\ldots,F_m$ and let $D$ be a
probability measure over $\R^2$.  We triangulate each face $F_i$ of
$G$ (a simple polygon) using \thmref{min-H-triangulation} to obtain a
triangulation $\Delta_i$.\footnote{Note that the outer face is the
complement of a simple polygon.  This can be handled by, for example,
working with a central projection of $G$ onto a sphere.} The union of
all $\Delta_i$ is a triangulation $\Delta$ of $\R^2$, to which we
apply \thmref{ammw07} to obtain a point location structure $R=R(G,D)$
for point location in $\Delta$ and hence also in $G$.  The following
theorem shows that $R$ is nearly optimal:

\begin{thm}
Given a connected planar subdivision $G$ with $n$ vertices and a probability
measure $D$ over $\R^2$, a data structure $R=R(G,D)$ of size $O(n)$ can be
constructed in $O(n\log n)$ time that answers point location queries in $G$.
The expected number of point/line comparisons performed by $R$, 
for a point $p$ drawn according to $D$ is 
\[
  \mu_D(R) \le \mu_D(T^*) + O(\mu_D(R)^{2/3}+1) \enspace , 
\] 
where $T^*$ is any linear classification tree that answers point
location queries in $G$.
\end{thm}

\begin{proof}
The space and preprocessing requirements follow from
\thmref{min-H-triangulation} and \thmref{ammw07}.
To prove the bound on the expected query time, apply
\lemref{triangulate} to the tree $T^*$ and consider the resulting tree
$T'$, each of whose leaves have regions that are triangles and such
that
\[
     \mu_D(T') \le \mu_D(T^*) + O(\log \mu_D(T^*)) \enspace .
\]
Observe that each leaf of $T'$ corresponds to a triangle in $\R^2$
that is completely contained in one of the faces of $G$.  Let
$\Delta'$ denote this set of triangles and let $\Delta'_i$ denote the
subset of $\Delta'$ contained in $F_i$.
Consider the entropy $H(\Delta')$ of the distribution induced by the
triangles of $T'$:
\begin{eqnarray*}
   H(\Delta') 
     & = & \sum_{i=1}^f \sum_{t\in \Delta'_i}\Pr(t)\log(1/\Pr(t)) \\
     & = & \sum_{i=1}^f \Pr(F_i)\sum_{t\in \Delta'_i}
            \Pr(t|F_i)\log(1/\Pr(t)) \\
     & = & \sum_{i=1}^f \Pr(F_i)\sum_{t\in \Delta'_i}
            \Pr(t|F_i)
            \left(
              \log(1/\Pr(t|F_i))-\log(\Pr(F_i))
            \right) \\
     & = & \sum_{i=1}^f \Pr(F_i)\sum_{t\in \Delta'_i}
            \Pr(t|F_i)\log(1/\Pr(t|F_i)) 
          +
      \sum_{i=1}^f \Pr(F_i) \log(1/\Pr(F_i)) \\
     & = & \sum_{i=1}^f \Pr(F_i) H(\Delta'_i) + H(F) \enspace .
\end{eqnarray*}
Similarly, the entropy of $\Delta$ is given by 
\begin{eqnarray*}
   H(\Delta) 
     & = & \sum_{i=1}^f \sum_{t\in \Delta_i}\Pr(t)\log(1/\Pr(t)) \\
     & = & \sum_{i=1}^f \Pr(F_i)\sum_{t\in \Delta_i}
            \Pr(t|F_i)\log(1/\Pr(t|F_i)) 
          + H(F) \\
     & = & \sum_{i=1}^f \Pr(F_i) H(\Delta_i) + H(F) \enspace .
\end{eqnarray*}
By \thmref{min-H-triangulation}, the triangles in $\Delta_i$ form a
nearly-minimum entropy triangulation of $F_i$.  More specifically, 
\[  
   H(\Delta_i) \le H(\Delta'_i) + O(H(\Delta_i)^{2/3}+1)  \enspace .
\]
Putting this all together, we have
\begin{eqnarray*}
  H(\Delta) 
    &  =  & \sum_{i=1}^f\Pr(F_i) H(\Delta_i) + H(F) \\ 
    & \le & \sum_{i=1}^f\Pr(F_i) (H(\Delta'_i) + O(H(\Delta_i)^{2/3}+1)) + H(F) \\ 
    &  =  & H(\Delta') + \sum_{i=1}^f\Pr(F_i) O(H(\Delta_i)^{2/3}+1) \\ 
    & \le & \mu_D(T') + \sum_{i=1}^f\Pr(F_i) O(H(\Delta_i)^{2/3}+1) \\ 
    & \le & \mu_D(T') + O(H(\Delta)^{2/3}+1) \\
    & \le & \mu_D(T^*) + O(H(\Delta)^{2/3}+1)
\end{eqnarray*}
Finally, since we preprocess $\Delta$ using \thmref{ammw07}, the
expected number of comparisons required to answer a query is
\begin{eqnarray*}
  H(\Delta) + O(H(\Delta)^{1/2} + 1)
   & = & \mu_D(T^*) +  O(\mu_D(T^*)^{1/2} + H(\Delta)^{2/3} + H(\Delta)^{1/3} + 1) \\
   & = & \mu_D(T^*) + O(\mu_D(R)^{2/3} + 1)
\end{eqnarray*}
and this completes the proof.
\end{proof}
\end{document}
