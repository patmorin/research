\documentclass{beamer}
\usepackage{amsopn}

\DeclareMathOperator{\conv}{conv}
\DeclareMathOperator{\average}{average}

\usepackage{beamerthemesplit}
\input{pat}

\title{An Optimal Randomized Algorithm \\ for $d$-Variate Zonoid Depth}
\author{Pat Morin}
\date{\today}

\begin{document}

\frame{\titlepage}

\section[Outline]{}
\frame{\tableofcontents}

\section{Introduction}
\subsection{Depth Functions}
\frame
{
  \frametitle{Depth Functions}
  \begin{itemize}
  \item $S$: A set of $n$ points in $\R^d$
  \item A \emph{depth function} assigns a real value $D(p,S)$ to 
            each point in $\R^d$
  \item Generalizes 1-dimensional \emph{order statistics} 
  \item Produces robust multivariate statistics
  \item Many depth functions have been defined, including Tukey depth,
projection depth, Delaunay depth, Fermat-Weber depth, simplicial
depth, convex-hull peeling depth,\ldots,
  \item<1->zonoid depth
  \end{itemize}
}
    
\subsection{Zonoid Depth}
\frame
{
   \frametitle{Zonoid Depth}
   \begin{itemize}
   \item<1-> Recall the definition of the \emph{convex hull}
    \[ \conv(S) = \left\{\sum_{p\in S} p\lambda_p : 
         \mbox{$0\le\lambda_p\le 1$ and $\sum_{p\in S}\lambda_p = 1$} 
        \right\} 
    \]
   \item<2-> Change it slightly to get the \emph{$k$-zonoid}
    \[ Z_k(S) = \left\{\sum_{p\in S} p\lambda_p : 
         \mbox{$0\le\lambda_p\le 1/k$ and $\sum_{p\in S}\lambda_p = 1$} 
        \right\} 
    \]
   \item<3-> Note that $Z_1(S)=\conv(S)$ and $Z_n(S)=\{\average(S)\}$ 
   \item<4-> The \emph{zonoid depth} of $p$ is 
     \[ Z(p,S)=\max\{k : p\in Z_k(S)\} \]
   \end{itemize}
}

\frame
{
    \frametitle{Zonoid Depth}
    \begin{center}
     \includegraphics{zonoid-eg}
    \end{center}
}


\subsection{Zonoid Depth Computation}
\frame
{
  \frametitle{Zonoid Depth Computation}
  \begin{itemize}
  \item<1-> How quickly can we compute the zonoid depth of $p$?
     \begin{itemize}  
       \item<2-> Bern and Eppstein 2001:  $O(n(Ld\log n)^c)$ time where $L$
            is the bit precision of the input  (uses ellipsoid method)
       \item<3-> Gopala and Morin 2004: $O(n)$ expected time, but only
            for $d=2$
       \item<4-> Here: $O(n)$ expected time for any constant $d$ (but
            with a superpolynomial dependence on $d$)
     \end{itemize}
     \item<5-> In this talk we focus mainly on the decision 
          problem:  \\ ``Is $p\in Z_k(S)$?''
  \end{itemize}
}

\section{The Algorithm}

\subsection{Chan's Optimization Theorem}
\frame
{
  \frametitle{Chan's Optimization Theorem (2004)}
  Suppose:  
  \begin{enumerate}
   \item<1-> $\mathcal{P}$ is a computational problem
   \item<2-> $f:\mathcal{P}\mapsto C$ maps problem instances onto closed 
         convex subsets of $\R^d$
   \item<3-> $w:\R^d\mapsto \R$ is a linear objective function
   \item<4-> For any $p\in\R^d$ and any $P\in\mathcal{P}$ of size $n$ we
         can test if $p\in f(P)$ in $D(n)$ time
   \item<5-> For any $P\in\mathcal{P}$ of size $n$ we can find
         $P_1,\ldots, P_r\in\mathcal{P}$ such that $|P_i|\le\alpha n$ 
         and $f(P)=\bigcap_{i=1}^r f(P_i)$ \hfill{[$r=O(1)$ and $\alpha<1$]}
  \end{enumerate}
  Then:
  \begin{itemize}
    \item<6-> For any $P\in\mathcal{P}$ of size $n$ we can find the point
    $q\in f(P)$ that maximizes $w(q)$
  \end{itemize}
} 

\frame
{
   \frametitle{Chan's Theorem}

   \begin{tabular}{ccc}
    $P$ & $\stackrel{f}{\Rightarrow}$ & \includegraphics[scale=.2]{chan-1} \\
    $\Downarrow$ \\
    $P_1$,$P_2$,$P_3$,$P_4$ & $\stackrel{f}{\Rightarrow}$ 
	& \includegraphics[scale=.2]{chan-2} \\
   \end{tabular}
}

\subsection{Geometric Duality}
\frame
{ \frametitle{Geometric Duality}
  \begin{center}
   \includegraphics{duality-1}
   Points become hyperplanes
  \end{center}
  \begin{itemize}
    \item Points become hyperplanes
    \item Polytopes become sets of hyperplanes that avoid two polytopes
  \end{itemize}
}



\end{document}
