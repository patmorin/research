\documentclass[acmtoalg]{acmtrans2m}
\let\orgsetcounter\setcounter  % workaround for broken acmtrans2m
\usepackage{amsopn,algorithmic}
\usepackage{pat}

\newcommand{\bigrand}{\textsc{BigRand}}
\newcommand{\A}{\mathcal{A}}
\newcommand{\note}[1]{\enspace\enspace\hfill{\mbox{[#1]}}}
\DeclareMathOperator{\sign}{sign}
\DeclareMathOperator{\bias}{bias}
\DeclareMathOperator{\reset}{reset}
\DeclareMathOperator{\var}{var}

%\acmVolume{V}
%\acmNumber{N}
%\acmYear{YY}
%\acmMonth{M}
\markboth{Evangelos Kranakis et al.}{Randomized Rendez-Vous with Limited Memory}
\title{Randomized Rendez-Vous with Limited Memory}
\author{\MakeUppercase{Evangelos Kranakis} \\
           Carleton University \and
	\MakeUppercase{Danny Krizanc} \\ 
           Wesleyan University \and
	\MakeUppercase{Pat Morin} \\
           Carleton University}

\begin{abstract}
We present a tradeoff between the expected time for two identical
agents to rendez-vous on a synchronous, anonymous, oriented ring and
the memory requirements of the agents. In particular, we show  there
exists a $2t$ state agent, which can achieve rendez-vous on an $n$
node ring in expected time $O( n^2/2^{t} + 2^{t} )$ and that any $t/2$
state agent requires expected time $\Omega( n^2/2^t )$. As a corollary
we observe that $\Theta(\log \log n)$ bits of memory are necessary and
sufficient to achieve rendez-vous in linear time.
\end{abstract}

\category{H.3.4}{Systems and Software}{Distributed Systems}
\category{G.3}{Probability and Statistics}{}[Probabilistic Algorithms]


\terms{Algorithms, Theory}

\keywords{Rendez-Vous, Distributed Computing}

\begin{document}

\setcounter{page}{1}

{\let\setcounter\orgsetcounter
\begin{bottomstuff}
Authors' addresses: E.~Kranakis and P.~Morin, School of Computer Science,
Carleton University, 1125 Colonel By Drive, Ottawa, ON, CANADA K1S~5B6.\newline
D.~Krizanc, Computer Science Group, Mathematics Department, Wesleyan
University, Middletown, CT 06459 USA.
\end{bottomstuff}
}
\maketitle


\section{Introduction}
\seclabel{intro}

The problem of {\em rendez-vous} (the gathering of agents
widely dispersed in some domain at a common place and
time) has been studied under many guises and in many settings
\cite{alpernGal03,Marco,dmsvw08,dessmark,dobrev,fkkss,fkklss,gakra,kowalski1,kowalski2,kkss,mgkkpv06,markou,RoyDudek,sawchuk04,Yamashita,yu}.  
(See Kranakis et al.\ 2006 for a survey of recent results.) In this
paper we consider the problem of autonomous mobile software agents
gathering in a distributed network. This is a fundamental operation
useful in such applications as web-crawling, distributed search, meeting
scheduling, etc. In particular, we study the problem of two identical
agents attempting to rendez-vous on a synchronous anonymous ring.

\subsection{The Model}

Network rendez-vous problems have many different parameters, including
the topology of the underlying network, the agents' knowledge (if any) of
the network, labellings (if any) of the edges and vertices of the network,
the capabilities of the agents (unlimited memory, access to random bits,
the ability to clone, the ability to leave messages at vertices), the
types of agents (identical or not), the synchrony of the agents, and
the reliability of the agents. 

We consider the standard model of a synchronous anonymous oriented
$n$-node ring \cite{santoro}. The nodes are assumed to have no identities,
the computation proceeds in synchronous steps and the edges of the ring
are labelled {\tt clockwise} and {\tt counterclockwise} in a consistent
fashion.  We model the agents as identical probabilistic finite automata
$A = \langle S, \delta, s_0 \rangle$ where $S$ is the set of states of
the automata including $s_0$ the initial state and the special state ${\tt
halt}$, and $\delta:S \times C \times P \to S \times M$ where $C = \{ H,
T \}$ represents a random coin flip, $P = \{ \tt present, notpresent \}$
represents a predicate indicating the presence of the other agent at a
node, and $M = \{ -1, 0, +1 \}$ represents the potential moves the agent
may make, $+1$ representing clockwise, $-1$ counterclockwise and $0$
stay at the current node.  During each synchronous step, depending upon
its current state, the answer to a query for the presence of the other
agent, and the value of an independent random coin flip with probability
of $\tt heads$ equal to $1/2$, the agent uses $\delta$ in order to change
its state and either move across the edge labelled $\tt clockwise$, move
across the edge labelled $\tt counterclockwise$ or stay at the current
node. We assume that the agent halts once it detects the presence of
the other agent at a node.  Rendezvous occurs when both agents halt on
the same node.  The complexity measures  we are interested in are the
expected time (the number of synchronous steps) to rendez-vous (where the
expectation is taken over all sequences of coin flips of the two agents)
and the size ($|S|$) or memory requirement ($\log_2 |S|$) of the agents.

We assume for simplicity that $n$, the size of the ring is an even number
and that the two agents start an even distance apart. This avoids the
possibility that the two agents simultaneously cross the same edge in
opposite directions without achieving a rendez-vous.  This assumption
does not significantly affect our (asymptotic) results.  We can achieve
a similar rendez-vous time without this assumption by having agents use
a coin periodically to determine whether they should pause, for one unit
of time, at their currently location.

\subsection{Related Work and New Results}

A number of researchers have observed that using random walks one can
design $O(1)$ state agents that will rendez-vous in polynomial $O(n^3)$
number steps on any network \cite{coppersmith}. For the ring the expected
time for two random walks to meet is easily shown to be $O(n^2)$. (See
Reference~\cite{kk} for an example proof of this fact.)

This expected time bound can be improved by considering the following
strategy.  Repeat the following until rendez-vous is achieved: flip a
fair coin and walk $n/2$ steps clockwise if the result is heads, $n/2$
steps counterclockwise if the result is tails.  If the two agents choose
different directions (which they do with probability 1/2) then they will
rendez-vous (at least in the case where they start at an even distance
apart).  The expected time to rendez-vous in this case satisfies 
\[
  T \le (1/2)(n/2) + (1/2)(n/2+T)
\]
and is therefore at most $3n/2$.  Alpern refers to this strategy
as \emph{Coin Half Tour} and studies it in detail \cite{alpern95}.
A variant of Coin Half Tour, in which each agent either travels $n-1$
steps in the same direction or remains stationary for $n-1$ time units,
was studied by Alpern \etal\ \cite{abe99}.  When the agents have uniformly
distributed starting positions this strategy achieves an expected meeting
time, for odd $n$, of $n-\Theta(1)$. The case for even values of $n$
is complicated by the fact that the agents are more likely to pass each
other without meeting. In this case, the agents can still rendez-vous
in $1.254122768n + O(1)$ expected time.  A generalization of the above
strategy, in which each agent either searches exhaustively for $2n$
steps or waits for $2n$ steps, allows two agents to rendez-vous in any
$n$ vertex graph in expected time at most $4n$ \cite[Section~4]{abe99}.

Note that these strategies require the agent to count up to at least $n/2$
and thus require $\Omega(n)$ states or $\Omega(\log n)$ bits of memory.
The main result of this paper is that this memory requirement can be
reduced to $O(\log\log n)$ bits while still achieving rendez-vous in
$O(n)$ expected time, and this is optimal.

Below we show a tradeoff between the (memory) size of the agents and
the time required for them to rendez-vous. We prove there exists a $2t$
state algorithm,   which can achieve rendez-vous on an $n$ node ring in
expected time $O(n^2/2^{t} + 2^{t} )$ and that any $t/2$ state algorithm
requires expected time $\Omega( n^2/2^t )$.  As a corollary we observe
that $\Theta(\log \log n)$ bits of memory are necessary and sufficient
to achieve rendez-vous in linear time.

A preliminary version of these results was presented at the 8th
Latin American Theoretical Informatics Conference (LATIN~2008)
\cite{latin}. \Secref{prelim} contains some preliminary results,
\secref{upper-bound} our upper bound and \secref{lower-bound} the
lower bound.


\section{Preliminaries}
\seclabel{prelim}

\subsection{Martingales, Stopping Times, and Wald's Equations}

In this section, we review some results on stochastic processes that
are used several times in our proofs.  The material in this section is
based on the presentation in Ross' textbook \cite[Chapter 6]{ross}.
Let $X=X_1,X_2,X_3,\ldots$ be a sequence of random variables and let
$Q=Q_1,Q_2,Q_3\ldots$ be a sequence of random variables in which $Q_i$
is a function of $X_1,\ldots,X_i$.  Then we say that \emph{$Q$ is a
martingale with respect to $X$} if, for all $i$, $\E[|Q_i|] < \infty$
and $\E[Q_{i+1}\mid X_1,\ldots,X_i] = Q_i$.

A positive integer-valued random variable $T$ is a \emph{stopping
time} for the sequence $X_1,X_2,X_3,\ldots$ if the event $T=i$ is
determined by the values $X_1,\ldots,X_i$.  In particular, the event
$T=i$ is independent of the values $X_{i+1},X_{i+2},\ldots$.  Some of
our results rely on the Martingale Stopping Theorem:

\begin{thm}[Martingale Stopping Theorem]\thmlabel{martingale}
If $Q_1,Q_2,Q_3,\ldots$ is a martingale with respect to
$X_1,X_2,X_3,\ldots$ and $T$ is a stopping time for
$X_1,X_2,X_3,\ldots$ then
\[
   \E[Q_T] = \E[Q_1] \enspace 
\]
provided that at least one of the following holds:
\begin{enumerate}
\item $Q_i$ is uniformly bounded for all $i\le T$,
\item $T$ is bounded, or
\item $\E[T] < \infty$ and there exists an $M<\infty$ such that
\[ \E\left[|Q_{i+1}-Q_i| \mid X_1,\ldots,X_i \right] < M  \enspace . \]
\end{enumerate}
\end{thm}

If $X_1,X_2,X_3,\ldots$ is a sequence of i.i.d.\ random variables with
expected value $\E[X]<\infty$ and variance $\var(X)<\infty$ then 
the sequence $Q_i=\sum_{j=1}^i(X_j-\E[X])$ is a Martingale and the
assumption that $\var(X)<\infty$ implies that this sequence satisfies
Condition~3 of \thmref{martingale}, so we obtain \emph{Wald's Equation
for expectation}:
\begin{equation}
    \E\left[\sum_{i=1}^T X_i \right] = \E[T]\cdot\E[X] \eqlabel{wald-1}
\end{equation}
whenever $T$ is a stopping time for $X_1,X_2,X_3,\ldots$  Similarly,
we can derive a version of \emph{Wald's Equation for variance} by considering 
the martingale 
\[
   Q_i=\left(\sum_{j=1}^i (X_j-\E[X]\right)^2 - i\cdot\var(X)
\]
to obtain
\begin{equation}
    \var\left(\sum_{i=1}^T X_i\right) = \E\left[\left(\sum_{i=1}^T (X_i-\E[X_i])\right)^2 \right] =
\E[T]\cdot\var(X) \enspace . \eqlabel{wald-2}
\end{equation}

\comment{
Our upper bound uses a special kind of random walk using probabilistic
counters to solve the rendez-vous problem.  We therefore begin with
some preliminary results on random walks and probabilistic counters.}

\subsection{A Lemma on Random Walks}

Let $X_1,X_2,X_3,\ldots\in\{-1,+1\}$ be independent random variables with
\[ \Pr\{X_i=-1\}=\Pr\{X_i=+1\}=1/2
\]
and let $S_i=\sum_{j=1}^i X_j$.  The sequence $S_1,S_2,S_3,\ldots$ is
a \emph{simple random walk} on the line, where each $X_i$ represents a
step to the left ($X_i=-1$) or a step to the right ($X_i=+1$).
Define the \emph{hitting time} $h_m$ as
\[
   h_m = \min\left\{i:|S_i|= m\right\} \enspace ,
\]
which is the number of steps in a simple random walk
before it travels a distance of $m$ from its starting location.
The following result is well-known (see, e.g.,
Reference~\cite{mu05}):
\begin{lem}\lemlabel{hitting-time}
$\E[h_m] = m^2$.
\end{lem}
Applying Markov's Inequality with \lemref{hitting-time} yields the
following useful corollary
\begin{cor}\corlabel{hitting-time}
$\Pr\{\max\{|S_i|:i\in\{1,\ldots,2m^2\}\} \ge m\} \ge 1/2$ .
\end{cor}
In other words, \corref{hitting-time} says that, at least half the
time, at some point
during the first $2m^2$ steps of a simple random walk, the 
walk is at distance $m$ from its starting
location.

Let $Y_1,\ldots,Y_m$ be i.i.d.\
non-negative random variables with finite expectation $r=\E[Y_i]$,
independent of $X_1,\ldots,X_m$, and with the
property that 
\[ \Pr\{Y_i \ge \alpha r\} \ge 1/2 \]
for some constant
$\alpha > 0$.  The following lemma considers a modified random walk in
which the $i$th step is of length $Y_i$:
\begin{lem}\lemlabel{walk}
Let $X_1,\ldots,X_m$ and $Y_1,\ldots,Y_m$ be defined as above.  Then
there exists constants $\beta,\kappa>0$ such that
\[
  \Pr\left\{\max\left\{
           \left|\sum_{i=1}^{m'} X_iY_i\right| : m'\in\{1,\ldots,m\}\right\}
            \ge \beta r\sqrt{m}\right\} 
     \ge \kappa \enspace .
\]
\end{lem}

\begin{proof}
We will define 3 events $E_1,E_2,E_3$ such that
$\Pr\{E_1\cap E_2\cap E_3\} \ge 1/8$ and, if $E_1$, $E_2$, and $E_3$
all occur, then there exists a value $m'\in\{1,\ldots,m\}$ such that
$\left|\sum_{i=1}^{m'}
X_iY_i\right| \ge \alpha r \sqrt{m}/2^{3/2}$.  This will prove the
lemma for $\kappa = 1/8$ and $\beta = \alpha/2^{3/2}$.

Let $E_1$ be the event that there exists a value $m'\in\{1,\ldots,m\}$
such that 
\[
   \left|\sum_{i=1}^{m'} X_i\right| \ge \sqrt{m/2} \enspace .
\]
By \corref{hitting-time}, $\Pr\{E_1\}\ge 1/2$.  Assume $E_1$ occurs
and, without loss of generality, assume $\sum_{i=1}^{m'} X_i > 0$.

Let $I^{+} = \{i\in\{1,\ldots,m'\}: X_i=+1\}$ and
$I^-=\{1,\ldots,m'\}\setminus I^+$.  We further partition $I^+$ into
two sets $I^+_1$ and $I^+_2$ where $I^+_1$ contains the smallest
$|I^-|$ elements of $I^+$ and $I^+_2$ contains the remaining elements.
Note that, with these definitions, $|I^+_1|=|I^-|$ and that
$|I^+_2|=\sum_{i=1}^{m'} X_i$.  Let $E_2$ be the event that
\[
     \sum_{i\in I^+_1}X_iY_i + \sum_{i\in I^-} X_iY_i \ge 0
\]
which is equivalent to $\sum_{i\in I^+_1}Y_i \ge \sum_{i\in I^-} Y_i$
and observe that, by symmetry, $\Pr\{E_2|E_1\}\ge 1/2$.

Finally, let $E_3$ be the event
\[
   \sum_{i\in I^+_2} X_iY_i \ge \alpha r|I^+_2|/2
\]
To bound $\Pr\{E_3|E_1\cap E_2\}$, let $T=\left|\{i\in I^+_2:Y_i\ge
\alpha r\}\right|$ and observe that
$T\ge |I^+_2|/2$ implies $E_3$. Now, $T$ is a
$\mathrm{binomial}(|I^+_2|,p)$ random variable for $p\ge 1/2$ so its
median value is at least $p|I^+_2| \ge |I^+_2|/2$ and therefore
$\Pr\{E_3|E_1\cap E_2\} \ge \Pr\{T\ge |I^+_2|/2\} \ge 1/2$.

We have just shown that $\Pr\{E_1\cap E_2\cap E_3\}\ge 1/8$.  To
complete the proof we observe that, if $E_1$, $E_2$ and $E_3$ occur
then
\begin{eqnarray*}
  \sum_{i=1}^{m'} X_i Y_i 
    & = & \sum_{i\in I^+_1} X_i Y_i
           + \sum_{i\in I^-} X_i Y_i 
           + \sum_{i\in I^+_2} X_i Y_i \\
    & \ge & \sum_{i\in I^+_2} X_i Y_i \\
    & \ge & \alpha r|I^+_2|/2 \\
    & \ge & \alpha r\sqrt{m}/2^{3/2} \enspace .
\end{eqnarray*}
\comment{
We will define 4 events $E_1,\ldots,E_4$, such that, if all of these
events occurs then $\left|\sum_{i=1}^m X_iY_i\right| \ge \alpha
r\sqrt{m}/2$.  Furthermore, $\Pr\{E_i\mid E_1,\ldots,E_{i-1}\}\ge 1/2$
for each $i\in\{1,\ldots,4\}$.  Therefore, the probability that all 4
events occur is at least $1/16$.  In this way, the lemma will be
proven for $\beta = \alpha/2$ and $\kappa=1/16$.

Consider the index set $I=\{i\in\{1,\ldots,m\}: Y_i \ge \alpha r\}$.
Let $E_1$ be the event $|I| \ge m/2$.  Then, it follows from the fact
that $\ceil{m/2}$ is a median of the $\textrm{binomial}(1/2,m)$
distribution, that $\Pr\{E_1\} \ge 1/2$. Define $m' = \max\{i\in I\}$
and observe that, if $E_1$ occurs, then $m'\ge |I| \ge m/2$.  Let
$\overline{I}=\{1,\ldots,m'\}\setminus I$.

Let $E_2$ be the event
\begin{equation}
  \left|\sum_{i=1}^{m'}X_iY_i\right| 
     = \left|\sum_{i\in I} X_iY_i\right| 
        + \left|\sum_{i\in \overline{I}} X_iY_i\right| \eqlabel{sign}
\end{equation}
and observe that, again, $\Pr\{E_2\mid E_1\} \ge 1/2$.

Let $I_j=\{i\in I: i\le j\}$ and let $E_3$ be the event
\begin{equation}
	\max\left\{\left|\sum_{i\in I_j} X_i\right|:j\in I\right\} \ge
\sqrt{|I|/2} \enspace .
        \eqlabel{big}
\end{equation}
By \corref{hitting-time} $\Pr\{E_3|E_1,E_2\}\ge 1/2$.

Finally, let $I^-=\{i\in I:X_i=-1\}$, let $I^+=\{i\in I:X_i=+1\}$ and
assume, without loss of generality, that $|I^+| \ge |I^-|$.  Partition
$I^+$ arbitrarily into two sets $I^+_1$ and $I^+_2$ so that
$|I^+_1|=|I^-|$. Let $E_4$ be the event that
\[
     \sum_{i\in I^+_1} Y_i \ge  \sum_{i\in I^-} Y_i 
\]
and observe that $\Pr\{E_4\mid E_1,E_2,E_3\} \ge 1/2$, so
$\Pr\{E_1,\ldots,E_4\} \ge 1/16$.  Finally we observe that, if
$E_1,\ldots,E_4$ all occur, then
\begin{eqnarray*}
  \left|\sum_{i=1}^{m'} X_iY_i\right|
   & = & \left|\sum_{i\in I} X_iY_i\right|  
           + \left|\sum_{i\not\in I} X_iY_i\right|  \\
   & \ge & \left|\sum_{i\in I} X_iY_i\right| 
           \note{because of $E_2$}\\
   & = & \left|\sum_{i\in I^+_2} Y_i
       +\sum_{i\in I^+_1} Y_i - \sum_{i\in I^-} Y_i\right| 
           \\
   & \ge & \sum_{i\in I^+_2} Y_i 
           \note{because of $E_4$} \\
   & \ge & \sum_{i\in I^+_2} \alpha r \\
   & = & |I^+_2|\alpha r \\
   & \ge & \alpha r\sqrt{m'/2} 
           \note{because of $E_3$} \\
   & \ge & \alpha r\sqrt{m}/2 
           \note{because of $E_1$} \enspace ,\\
\end{eqnarray*}
as required.
}
\end{proof}


\subsection{An Approximate Counter}

In the previous section we have shown that, if we can generate random
variables $Y_i$ that are frequently large, then we can speed up the
rate at which a random walk moves away from its starting location.  In
this section we consider how to generate these frequently-large random
variables.  Consider a random variable $Y$ generated by the following
algorithm:

\noindent
\begin{minipage}{\textwidth}
$\bigrand(t)$
\begin{algorithmic}[1]
\STATE{$Y\gets C \gets 0$}
\WHILE{$C < t$}
  \STATE{$Y\gets Y + 1$}
  \IF {a coin toss comes up heads}
    \STATE{$C\gets C + 1$}
  \ELSE
    \STATE{$C\gets 0$}
  \ENDIF
\ENDWHILE
\RETURN{$Y$}
\end{algorithmic}
\end{minipage}

\begin{lem}\lemlabel{counter}
Let $Y$ be the output of Algorithm $\bigrand(t)$.  Then 
\begin{enumerate}
\item $\E[Y]=2^t(2-1/2^{t-1})$ and
\item $\var(Y) \le 2^{t+1} < \infty$
\item $\Pr\{Y \ge \E[Y] / 2\} \ge 1/2$.
\end{enumerate}
\end{lem}

\begin{proof}
To compute the expected value of $Y$ we observe that the algorithm
begins by tossing a sequence of $i-1$ heads and then either
(a)~returning to the initial state if the $i$th coin toss is a tail or
(b)~terminating if $i=2^t$.  The first case occurs with probability
$1/2^i$ and the second case occurs with probability $1/2^t$.  Call the
interval between consecutive visits to the initial state a
\emph{round}.  The number of rounds, $T$, is a
$\mathrm{geometric}(1/2^t)$ random variable and therefore $\E[T] =
2^t$.  The length $X_i$ of the $i$th round is dominated\footnote{A random
variable $X$ dominates a random variable $Y$ if $\Pr\{X > x\}\ge
\Pr\{Y > x\}$ for all $x\in\R$.} by a
$\mathrm{geometric}(1/2)$ random variable and its expectation and
variance are easily shown to satisfy
$\E[X_i]=2-1/2^{t-1}$ and $\var(X_i) \le 2$.  Therefore, Parts~1 and
2 of the lemma follow from Wald's Equations for expectation and
variance of $Y=\sum_{i=1}^T X_i$, respectively.

%In this way, we obtain the equation
%\[
%   \E[Y] = \sum_{i=1}^{t} \frac{1}{2^i}\left(i + \E[Y]\right) +
%\frac{t}{2^t} \enspace .
%\]
%Rearranging terms and multiplying by $2^{t}$, we obtain
%\[
%   \E[Y] = 2^t(2-1/2^{t-1}) \enspace .  
%\]

To prove the second part of the lemma, consider only the random
variable $T$ that counts the number of rounds.  Since $T$ is a
geometric($1/2^t$) random variable, its median is $\ceil{-
\log 2/\log(1-1/2^t)}$ and is therefore at least $2^t$, for $t \ge 1$.
Since the value of $T$ is a lower bound on the value of $Y$, this
completes the proof.
\end{proof}

\section{The Rendez-Vous Algorithm}
\seclabel{algorithm}
\seclabel{upper-bound}

Consider the following algorithm used by an agent to make a random walk
on a ring.  The agent repeatedly performs the following steps: (1)~toss
a coin to determine a direction $d\in\{\mbox{{\tt clockwise}},\mbox{{\tt
counterclockwise}}\}$ then (2)~run algorithm $\bigrand(t)$ replacing
each increment of the variable $Y$ with a step in direction $d$.
By using $t$ states for a clockwise counter and $t$ states for a
counterclockwise counter this algorithm can be implemented by a $2t$
state finite automata. (Or using one bit to remember the direction $d$
and $\log t$ bits to keep track of the counter $C$ in the $\bigrand$
algorithm, it can be implemented by an agent having only $1+\log_2 t$
bits of memory.)

We call $m$ iterations of the above algorithm a \emph{round}.  Together,
\lemref{walk} and \lemref{counter} imply that, during a round, with
probability at least $\kappa$, an agent will travel a distance of at
least $\beta 2^t \sqrt{m}$ from its starting location.  Set
\[
   m= \left\lceil \frac{n^2}{\beta^2 2^{2t}} \right\rceil 
\]
and consider what happens when two agents $A$ and $B$ both execute
this rendez-vous algorithm.  During the first round of $A$'s
execution, with probability at least $\kappa$, agent $A$ will have
visited agent $B$'s starting location. Furthermore, with probability
at least $1/2$ agent $B$ will not have moved away from $A$ when this
occurs, so the paths of agents $A$ and $B$ will cross, and a
rendez-vous will occur, with probability at least $\kappa/2$.  If we
define $T$ as the round in which agents $A$ and $B$ rendez-vous, we
therefore have $E[T]\le 2/\kappa$.

By \lemref{counter}, the expected number of steps taken for $A$ to
execute the $i$th round is at most
\[
    \E[M_i] \le m 2^{t}(2-1/2^{t-1}) 
\]
and
\[
    \var(M_i) \le m 2^{t+1} 
\]
The variables $M_1,M_2,\cdots$ are independent and the algorithm
terminates when $A$ and $B$ rendez-vous.  The time for two agents to
rendez-vous is bounded by 
\[
   \sum_{i=1}^T M_i \enspace .
\]
Note that the event $T=j$ is independent of $M_{j+1},M_{j+2},\ldots$
so $T$ is a stopping time for the sequence
$M_1,M_2,\ldots$.  Furthermore, $\var(M_i)<\infty$, so by Wald's
Equation for the expectation
\[
   E\left[\sum_{i=1}^T M_i\right] = E[T]\cdot E[M_1] 
     \le \frac{2}{\kappa}\cdot m 2^{t}(2-1/2^{t-1}) \enspace .
\]
This completes the proof of our first theorem.
\begin{thm}\thmlabel{upper-bound}
There exists a rendez-vous algorithm in which each agent has at most
$2t$ states and whose expected rendez-vous time is
$O(n^2/2^{t} + 2^{t})$.
\end{thm}

\section{The Lower Bound}
\seclabel{lower-bound}

Next we show that the algorithm in \secref{algorithm} is
%, in some sense, 
optimal.

The model of computation for the lower bound represents a rendez-vous
algorithm $\A$ as a probablistic finite automata having $t$ states.
Each vertex of the automata has two outgoing edges representing the
two possible results of a coin toss and each edge $e$ is labelled with
a real number $\ell(e)\in[-1,+1]$.  The edge label of $e$ represents a
step of length $|\ell(e)|$ with this step being counterclockwise if
$\ell(e)<0$ and clockwise if $\ell(e)>0$.  As before, both agents use
identical automata and start in the same state.  The rendez-vous
process is complete once the distance between the two agents is at
most 1.  This model is stronger than the model used for upper bound of
\thmref{upper-bound} since the edge labels are no longer restricted to
be in the discrete set $\{-1,0,+1\}$ and the definition of a
rendez-vous has been slightly relaxed.

\subsection{Well-Behaved Algorithms and Reset Times}

We say that an algorithm is \emph{well-behaved} if the directed graph
of its state machine has only one strongly connected component that
contains all nodes.  We are particularly interested in intervals
between consecutive visits to the start state, which we will call
\emph{rounds}.

\begin{lem}\lemlabel{reset}
Let $R$ be the number of steps during a round. Then 
$\E[R] \le 2^t$ and $\E[R^2] \le 2\cdot2^{2t}$.
\end{lem}

\begin{proof}
For each state $v$ of $\mathcal{A}$'s automata fix a shortest path (a
sequence of edges) leading from $v$ to the start state.  For an
automata that is currently at $v$ we say that the next step is a
\emph{success} if it traverses the first edge of this path, otherwise
we say that the next step is a \emph{failure}.

Each round can be further refined into \emph{phases}, where every
phase consists of 0 or more successes followed by either a failure or
by reaching the start vertex.  Let $X_i$ denote the length of the
$i$th phase and note that $X_i$ is dominated by a
$\mathrm{geometric}(1/2)$ random variable $X_i'$, so
$\E[X_i]\le\E[X_i'] = 2$.  On the other hand, if a phase lasts $t-1$
steps then the start vertex is reached.  Therefore, the probability of
reaching the start vertex during any particular phase is at least
$1/2^{t-1}$ and the number $T$ of phases is dominated by a
$\mathrm{geometric}(1/2^{t-1})$ random variable $T'$, so
$\E[T]\le\E[T']\le 2^{t-1}$.  Therefore, by Wald's Equation
\[
  \E[R] = \E\left[\sum_{i=1}^T X_i\right]\le \E\left[\sum_{i=1}^{T'}
X_i'\right] = \E[T']\cdot\E[X_1'] = 2^t \enspace .
\]
For the second part of the lemma, we can apply Wald's Equation for
the variance \eqref{wald-2} to
obtain
\begin{eqnarray*}
  \E[R^2] & = & \E\left[\left(\sum_{i=1}^T X_i\right)^2\right] \\
          &\le& \E\left[\left(\sum_{i=1}^{T'} X_i'\right)^2\right] \\
          & = & \var\left(\sum_{i=1}^{T'} X_i'\right) + (\E[T']\cdot\E[X_1'])^2 \\
          & = & \E[T']\cdot\var(X_1) + (\E[T']\cdot\E[X_1'])^2 \\
          & \le & 2^{t-1}\cdot 2 + 2^{2t} \\
          & \le & 2\cdot2^{2t}
\comment{
          & = & \E\left[\sum_{i=1}^{T'} (X_i')^2
                + \sum_{i=1}^{T'}\sum_{j=i+1}^{T'} 2X_i'X_j' \right] \\
          & = & \E\left[\sum_{i=1}^{T'} (X_i')^2\right]
                + \E\left[\sum_{i=1}^{T'}\sum_{j=i+1}^{T'} 2X_i'X_j'\right] \\
          & = & \E[T']\cdot \E\left[(X_i')^2\right]
                + \E\left[\sum_{i=1}^T\sum_{j=i+1}^{T'} 2X_i'X_j'\right] \\
          & = & \E[T']\cdot \E\left[(X_i')^2\right]
                + \E\left[\sum_{i=1}^T\sum_{j=i+1}^{T'} 2X_i'X_j'\right] \\

}
\end{eqnarray*}
as required.
\end{proof}


\subsection{Unbiasing Algorithms}

Note that $\E[R]$ can be expressed another way: For an edge $e$ of the
state machine, let $f(e)$ be the expected number of times the edge $e$
is traversed during a round.  The
\emph{reset time} of algorithm $\A$ is then defined as
\[
   \reset(\A) = \sum_e f(e) = \E[R] \enspace .
\]
The \emph{bias} of a well-behaved algorithm $\A$ is defined as
\[
    \bias(\A) = \sum_{e} f(e)\cdot\ell(e) \enspace ,
\]
which is the expected sum of the edge labels encoutered during a round.
We say that $\A$ is \emph{unbiased} if $\bias(\A)=0$, otherwise we say
that $\A$ is \emph{biased}.

Biased algorithms are somewhat more difficult to study.  However,
observe that, for any algorithm $\mathcal{A}$ we can replace every
edge label $\ell(e)$ with the value $\ell(e)-x$ for any real number
$x$ and obtain an equivalent algorithm in the sense that, if two
agents $A$ and $B$ execute the modified algorithm following the same
sequence of state transitions then $A$ and $B$ will rendez-vous after
exactly the same number of steps.  In particular, if we replace each
edge label $\ell(e)$ with the value
\[
   \ell'(e) = \ell(e) - \frac{\bias(\A)}{\reset(\A)}
\]
then we obtain an algorithm $\A'$ with $\bias(\A')=0$.  Furthermore, since
$|\bias(\A)|\le\reset(\A)$, every edge label $\ell'(e)$ has
$-2 \le \ell'(e)\le 2$.  This gives the following relation between
biased and unbiased algorithms:

\begin{lem}\lemlabel{unbiased}
Let $\A$ be a well-behaved $t$-state algorithm with expected rendez-vous time
$R$.  Then there exists a well-behaved unbiased $t$-state algorithm $\A'$ with
expected rendez-vous time at most $2R$.
\end{lem}

\subsection{The Lower Bound for Well-Behaved Algorithms}

We now have all the tools in place to prove the lower bound for the
case of well-behaved algorithms.

\begin{lem}\lemlabel{well-behaved}
Let $\A$ be a well-behaved $t$-state algorithm.  Then the expected rendez-vous
time of $\A$ is $\Omega(n^2/2^{t})$.
\end{lem}

\begin{proof}
Suppose the agents are placed at antipodal locations on a ring of size
$n$,
so that the distance between them is $n/2$.  We will show that there
exists constants $c >0$ and $p> 0$ such that, after $cn^2/2^t$ steps,
with probability at least $p$ neither agent will have travelled a
distance greater than $n/4$ from their starting location.  Thus, the
expected rendez-vous time is at least $p cn^2/2^t = \Omega(n^2/2^t)$.

By \lemref{unbiased} we can assume that $\A$ is unbiased.
Consider the actions of a single agent starting at location 0.  The
actions of the agent proceed in rounds where, during the $i$th round,
the agent takes $R_i$ steps and the sum of edge labels encountered
during these steps is $X_i$.  Note that the random variables
$X_1,X_2,\ldots$ are i.i.d.\ with expectation $\E[X]=0$ and variance
$\E[X^2]$. Since the absolute value of $X_i$ is bounded from above by
$R_i$, we have the inequalities $\E[|X_i|]\le \E[R_i]$ and
$\E[X_i^2]\le\E[R_i^2]$. 

Let $S_i=\left|\sum_{j=1}^i X_j\right|$, for
$i=0,1,2\ldots$ be the agent's
distance from their starting location at the end of the $i$th round.
Let $Q_i= S_i^2 - i\E[X^2]$ and observe that 
\comment{
\begin{eqnarray*}
 \E\left[Q_{i+1}|X_1,\ldots,X_i\right]
 & = & \sum_{x} \Pr\{X=x\} (\left(S_i + x\right)^2 - (i+1)\E[X^2]) \\
 & = & (S_i)^2 + S_i\E[X] + \E[X^2] - (i+1)\E[X^2] \\
 & = & (S_i)^2 + i\E[X^2] \\
 & = & Q_i  \enspace .
\end{eqnarray*}}
the sequence $Q_1,Q_2,\ldots$ is a martingale with respect to
the sequence $X_1,X_2,\ldots$ \cite[Example 6.1d]{ross}.  Define
\[
    T = \min\{i : S_i \ge m \} \enspace ,
\]
and observe that this is equivalent to 
\[
    T = \min\{i : Q_i \ge m^2 - i\E[X^2] \} \enspace .
\]
The random variable $T$ is a \emph{stopping time} for the martingale
$Q_1,Q_2,\ldots$.  Furthermore,
\begin{eqnarray*}
  \E[|Q_{i+1}-Q_i|\mid X_1,\ldots,X_i]  
    & = & \E\left[\left|S_{i+1}^2-(i+1)\E[X^2] 
                  - S_i^2 + i\E[X^2]  \right| \mid X_1,\ldots,X_i \right] \\
    & = & \E\left[\left|S_{i+1}^2 - S_i^2 - \E[X^2] \right| 
                                \mid X_1,\ldots,X_i  \right] \\
    & = & \E\left[\left|\left(\sum_{j=1}^{i+1} X_j\right)^2
                   - \left(\sum_{j=1}^i X_j\right)^2
                   - \E[X^2] \right|  \mid X_1,\ldots,X_i \right] \\
    & = & \E\left[\left| \sum_{j=1}^{i+1} X_{i+1}X_j
                   - \E[X^2] \right|  \mid X_1,\ldots,X_i \right] \\
    & = & \E\left[\left| X_{i+1}^2 - \E[X^2] \right|  
                 \mid X_1,\ldots,X_i \right] \\
    & = & \E\left[\left| X_{i+1}^2 - \E[X^2] \right|\right] \\
    & \le & \E[X^2] \le \E[R^2] < \infty \enspace .
\end{eqnarray*}
Therefore, by the \thmref{martingale}
\begin{equation}
  \E[Q_T] 
    = \E[Q_1] 
    = \E[(X_1)^2 - \E[X^2]] = 0 \enspace . \eqlabel{mart1}
\end{equation}
However, by definition $Q_T\ge m^2-T\cdot\E[X^2]$, so 
\begin{equation}
  \E[Q_T] 
    \ge \E[m^2 - T\cdot\E[X^2]] 
    = m^2 - \E[T]\cdot\E[X^2] \enspace . \eqlabel{mart2}
\end{equation}
Equating the right hand sides of \eqref{mart1} and \eqref{mart2} gives 
\[
   \E[T] \ge \frac{m^2}{\E[X^2]} \enspace .
\]
Furthermore, the expected number of steps taken by the agent during
these $T$ rounds is, by Wald's Equation, 
\begin{equation}
  \E\left[\sum_{i=1}^T R_i\right] 
    =  \E[T]\cdot\E[R_1] 
    \ge \frac{m^2 \E[R]}{\E[X^2]} 
    \ge \frac{m^2 \E[R]}{\E[R^2]}
    \ge \frac{m^2 \E[R]}{c2^{2t}} 
    \ge \frac{m^2}{c2^{2t}} 
   \enspace ,  \eqlabel{gap}
\end{equation}
%\begin{equation}
%  \E\left[\sum_{i=1}^T R_i\right] 
%    =  \E[T]\cdot\E[R_1] 
%    \ge \frac{m^2 \E[R]}{\E[X^2]} 
%    \ge \frac{m^2 \E[R]}{\E[R^2]} 
%    \ge \frac{m^2 \E[R]}{c2^{2t}} 
%    \ge \frac{m^2}{c2^{2t}} 
%   \enspace ,  \eqlabel{gap}
%\end{equation}
where the last two inequalities follow from \lemref{reset} and the
fact that $R\ge 1$.
\end{proof}

\subsection{Badly-Behaved Algorithms}

Finally, we consider the case where the algorithm $\mathcal{A}$ is not
well-behaved.  In this case, $\mathcal{A}$'s automata contains a set
of \emph{terminal components}.  These are disjoint sets of vertices of the
automata that are strongly connected and that have no outgoing edges (edges
with source in the component and target outside the component). From each 
terminal component, select an arbitrary vertex and call it
the \emph{terminal start state} for that terminal component.
An argument similar to that given in \lemref{reset} proves:

\begin{lem}\lemlabel{component}
The expected time to reach some terminal start state is at most $2^t$.
\end{lem}

Observe that each terminal component defines a well-behaved algorithm.
Let $c$ be the number of terminal components and let $t_1,\ldots,t_c$ be
the sizes of these terminal components.  When two agents execute the same
algorithm $\mathcal{A}$, \lemref{component} and Markov's Inequality
imply that the probability
that both agents reach the same terminal component after at most $2^{t+2}$
steps is at least $1/2c$.  By applying \lemref{well-behaved} to each component, we can therefore lower bound the expected rendez-vous time by
\begin{eqnarray*}
  \frac{1}{2c}\Omega(n^2/2^{t-c})
    & \ge & \Omega(n^2/2^{2t}) \enspace .
\end{eqnarray*}
Substituting $t'=t/2$ into the above completes the proof of our second theorem:

\begin{thm}
Any $t/2$-state rendez-vous algorithm has expected rendez-vous time
$\Omega(n^2/2^{t})$.
\end{thm}

\subsection{Linear Time Rendez-vous}

We observe that Theorems 2 and 3 immediately imply:

\begin{thm}
$\Theta( \log \log n)$ bits of memory are necessary and sufficient to 
achieve rendez-vous in linear time on an $n$ node ring.
\end{thm}

\section{Conclusions}

We have given upper and lower bounds on the expected rendez-vous time
for two identical agents to rendez-vous on a ring as a function of the
ring size $n$ and the the memory available to the agents.  In
particular, we have shown that $O(\log\log n)$ bits of memory are
necessary and sufficient for two agents to rendez-vous in $O(n)$
expected time.

A gap remains in our upper and lower bounds.  When expressed in terms
of the number of states $t$ available to the agents, our upper and
lower bounds differ by a factor of 4.  We believe that the upper bound
is tight and this gap is an artifact of the lower bound proof.
Closing the gap remains an open problem.

The current paper studies symmetric rendez-vous with limited memory
when the underlying graph is a ring. Another possibility is to consider
rendez-vous with limited memory in other graphs.  Possibilities include
rendez-vous an on $n$-vertex torus or, more generally, on any $n$-vertex
vertex-transitive graph.\footnote{A vertex-transitive graph is a graph
such that, for any pair of vertices $u$ and $w$, there is an isometry that
maps $u$ onto $w$ \cite[Chapter~15]{alpernGal03}.} With complete knowledge
of the underlying graph and unlimited memory, rendezvous can be achieved
in $O(n)$ expected time for any graph \cite[Section~15.2]{alpernGal03}.
On the other hand, if both agents take a random walk (which requires
no memory and no knowledge of the underlying graph) then their
expected meeting time is $O(n^3)$ and this is tight for some graphs
\cite{coppersmith}.

Another possibility is to consider the effects of memory limitations on
randomized algorithms for the rendezvous of multiple (greater than two)
agents on an $n$ node ring.  In particular, what is the expected time
required for $k$ identical agents, each having $t$ states, to achieve
rendez-vous on a synchronous, anonymous, oriented $n$ node ring?



%To close
%this gap, there are two locations in the lower bound proof that need
%to be improved.  The first is in \lemref{reset}, since it seems
%impossible for an unbiased algorithm to achieve $\E[R]=2^t$ and the
%bound achieved in the upper bound is $\E[R] = 2^{t/2}$.  The second
%part of the lower bound that needs to be tightened is in \eqref{gap},
%where a relationship needs to be developed between $\E[R]$ and
%$\E[X^2]$.  In particular, one needs to show that $\E[R]/\E[X^2] \ge
%c/2^t$ for some constant $c$.

\bibliographystyle{acmtrans}
\bibliography{rendezvous}

\end{document}

