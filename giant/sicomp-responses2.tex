\documentclass{patmorin}
\usepackage{amsmath}
\usepackage{fancyvrb}

\setlength{\parindent}{0mm}

\title{Responses to Reviewers}
\author{Prosenjit Bose, Vida Dujmovi\'c, Pat Morin, and Michiel Smid}

\begin{document}
\maketitle

This document contains a summary of changes made to our paper in response
to referee comments.  

\section{Editorial Comments}

To address the following comment, we have mentioned, in the preceding paragraph, that the edge weights of the graph are given by the Euclidean distance between their endpoints.  We have also reminded the reader (using parentheses) that the shortest paths we're talking about are weighted.

\begin{Verbatim}[frame=single]
top of page 2 (line 1 or item #2): I know it's an expository part, but 
perhaps it is not completely clear to every reader that edges lengths 
are always equal to the Euclidean distance 
\end{Verbatim}

To address the following comment, we have added the word \emph{fixed}.

\begin{Verbatim}[frame=single]
page 4 line -4 (and again just before Section 1.4): for FIXED c>0 
\end{Verbatim}

The following comment makes a good point.  Allowing $x=0$ is meaningless so we reset the range to $x\in\{1,...,|V|/2\}$.

\begin{Verbatim}[frame=single]
line -3: why do you allow x=0 and not start from x=1? 
\end{Verbatim}

With respect to the following comment, the referee is correct and the argument falls apart when $x>n/2$. We chose to fix this with a minimal amount of extraneous discussion by simply upper-bounding the value of $k$ to which the argument applies.  In particular, we specify that $k\in\{1,\ldots,\lfloor\max\{k':f(k')\le n/2\}\rfloor\}$.

\begin{Verbatim}[frame=single]
page 5 para 1: I have a problem when the set size passes the n/2 threshold 
say e.g. f(k) = k^2, and k is such that it is =3n/4. Please rewrite this 
argument taking with more care to this possibility.
\end{Verbatim}

To address the following two comments, we have added text before Section 1.1 that defines $n$, states that interpoint distances are always Euclidean, and shortest paths always refer to shortest Euclidean paths that use only edges of the graph.

\begin{Verbatim}[frame=single]
BTW, where did you define n=|V|? 

page 5 line -2: For EUCLIDEAN point sets 
More generally, you should say somewhere that throughout, you refer to 
R^d with Euclidean distances 
\end{Verbatim}

To address the following comment, we have added two remarks.  

The first remark, which appears at the end of Section~2, discusses how our lower bounds depend on the value of $t$.  This remark works out the dependence on $t$ in Theorem~3 and 4 and for one case of Corollary 2.

We have also added a second remark at the end of Section~3.1 discussing how Theorem~5 and Corollary~3 depend on $t$. 

We chose to leave these remarks out of the theorems so as not to distract the reader from the main arguments in the proofs.  Once the reader understands these arguments for a fixed $t>1$, it is easy to work out the dependence on $t$ (if one understands how the previous work depends on $t$).

\begin{Verbatim}[frame=single]
All results in the paper (e.g. in Section 1.4 para 2, or alternatively 
in the specific theorems and corollaries): How do the size of the 
spanner and f(k) depend on the stretch t>1? ...
\end{Verbatim}

To address the following comment, we have pointed out that $t$ is not necessarily an integer (immediately after the definition of $t$-spanner in Section 1.1).

\begin{Verbatim}[frame=single]
...You might want to 
emphasize that t need not be an integer (often referred to as 
t=1+epsilon). I know it's standard for geometric spanners, but 
remember an integer stretch is common for spanners of general graphs. 
\end{Verbatim}

\section*{Referee \#4}

To address the following comment, we have added a paragraph explaining and motivating ``induced spanners'' to the end of Section~1.1. (Our apologies to the referee for not recognizing the term ``induced spanners''---which we use as a paragraph heading!)

\begin{Verbatim}[frame=single]
The authors addressed all my comments, except for the last one.
They say that they are not sure what do I mean by "induced spanners". 
The definition of those appears in their paper in the paragraph that
precedes the Acknowledgements section.
\end{Verbatim}


\end{document}
