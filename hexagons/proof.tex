\documentclass{article}
\usepackage{graphicx}
\usepackage{amsthm}

\newtheorem{theorem}{Theorem}

\begin{document}

\begin{theorem}\label{thm:hex-deflate}
Every general position hexagon is montonically deflatable.
\end{theorem}


\begin{proof}
Let $P$ be any hexagon in general position.  The proof is by induction on
the number, $m$, of pairs of mutually visible non-adjacent vertices of
$P$.  The base case, $m=3$, happens when the hexagon is already deflated
(it has a unique triangulation with 4 triangles and 3 non-polygon edges).

The inductive step is made using an enormous case analysis.
Figures~\ref{fig:hexagons-a}--\ref{fig:hexagons-c} enumerate all possible
hexagons with distinct order types.  [Include information here about
how these were generated.]  In each case, we show how moving a single
vertex along a linear trajectory monotonically reduces the number of
visible pairs.

A simple argument, presented in the next paragraph, suffices to handle
all 49 cases in which $P$ has exactly one reflex vertex.  The single
case where $P$ has no reflex vertices is easily reduced to this case
(just move one vertex inwards until it becomes reflex), thus this handles
all examples in which $P$ has at most one reflex vertex.

Refer to Figure~\ref{fig:one-reflex} for what follows.  Let $a$, $b$,
$c$, $d$, $e$, and $f$ be the vertices of $P$ in the order they occur on
the boundary of $P$ and suppose, without loss of generality, that $a$
is the unique reflex vertex of $P$.  Suppose, again without loss of
generality that there is a closed halfplane with $a$ on its boundary that
contains $a$, $b$, $c$, and $d$, but not $f$. Then $abcd$ is a convex
quadrilateral contained in $P$ and moving $c$ directly towards $a$ until
it crosses $bd$ removes at least one visible pair, namely $bd$, from $P$.
This motion is monotone because the only vertices not visible from $c$
(possibly $f$ and $e$) remain hidden ``behind'' $a$.  In particular,
the orientiations of the triangles $fac$ and $eac$ do not change during
this motion.

\begin{figure}
  \begin{center}
    \includegraphics{one-reflex}
  \end{center}
  \caption{The inductive step of Theorem~\ref{thm:hex-deflate} when $P$ has
    one reflex vertex.}
  \label{fig:one-reflex}
\end{figure}

The remaining 119 cases have 2 or 3 reflex vertices
and are each handled using a motion illustrated in
Figures~\ref{fig:hexagons-a}--\ref{fig:hexagons-c}.  All these motions
move a single vertex, say $a$, along a linear trajectory until it crosses
a chord of $P$.  All these motions have two properties that make it easy
to check their correctness:

\begin{enumerate}
\item There is a convex polygon, $C$, whose vertices are a subset of those
of $P$, that contains $a$, $b$, and $f$, and whose interior intersects
the boundary of $P$ only in $ab$ and $af$.   Throughout the motion,
$a$ remains within $C$, except at the end, where it passes through
the interior of an edge of $C$ that is interior to $P$ and stops an
arbitrarily small distance outside of $C$. This guarantees that $P$
remains simple throughout the motion.  (See Figure~\ref{fig:example},
where $C$ is the triangle $bcf$.)

\item The motion of $a$ is such that it results in a continous sequence of
nested polygons.  In the notation of Section~\ref{sectioniXXX},  $P^{t'}
\subseteq P^t$ for all $0\le t\le t'\le 1$.  This ensures that no pair of
vertices $x,y\in\{b,c,d,e,f\}$ ever becomes visible during the motion.
That is, the only possibility of the motion being non-monotone comes
from the possibility that $a$ may gain visibilities as it moves.
\end{enumerate}

The only remaining check, for each case, is to ensure that no new visible
pair involving $a$ appears during the motion. This can be done case by
case using only order type information about $P$.  We now illustrate
one example, in Figure~\ref{fig:example}.  In this example, $a$ is
moved toward the interior of $P$ along the line through $ab$ until
it crosses the segment $fc$.  This eliminates the visible pair $fc$.
This motion satisfies properties 1 and 2, above, so the polygon remains
simple throughout the motion and no new visible pairs not involving
$a$ are created.  To check that no new visible pair involving $a$ is
created during the motion, observe that, initially, the only vertex not
visible from $a$ is $e$.  In particular, this is because the sequence
$efa$ forms a right turn.  This remains true at the end of the motion
because $efc$ forms a right turn and, at the end of the motion, $a$
is arbitrarily close to the segment $fc$.  Therefore, by convexity,
$efa$ forms a right turn throughout the motion and at no point during
the motion does the pair $ae$ become visible.

\begin{figure}
  \begin{center}
    \begin{tabular}{cc}
      \includegraphics{example-before} &
      \includegraphics{example-after} 
    \end{tabular}
  \end{center}
  \caption{One case in the proof of Theorem~\ref{thm:hex-deflate}.}
  \label{fig:example}
\end{figure}

Similar statements can be verified for all the polygons in
Figures~\ref{fig:hexagons-a}--\ref{fig:hexagons-c}.  We wish the reader
good luck with their verification.
\end{proof}

\begin{figure}
  \begin{center}
    \includegraphics[width=\textwidth]{hexagons-1}
  \end{center}
  \caption{Cases in proof of \ref{thm:hex-deflate}, Part I}
  \label{fig:hexagons-a}
\end{figure}

\begin{figure}
  \begin{center}
    \includegraphics[width=\textwidth]{hexagons-2}
  \end{center}
  \caption{Cases in proof of \ref{thm:hex-deflate}, Part II}
  \label{fig:hexagons-b}
\end{figure}

\begin{figure}
  \begin{center}
    \includegraphics[width=\textwidth]{hexagons-3}
  \end{center}
  \caption{Cases in proof of \ref{thm:hex-deflate}, Part III}
  \label{fig:hexagons-c}
\end{figure}




\end{document}
