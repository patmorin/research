\documentclass[lotsofwhite,charterfonts]{patmorin}
\usepackage{graphicx}
 
%\usepackage{amsthm}

\newcommand{\centeripe}[1]{\begin{center}\Ipe{#1}\end{center}}
\newcommand{\comment}[1]{}

\newcommand{\centerpsfig}[1]{\centerline{\psfig{#1}}}

\newcommand{\seclabel}[1]{\label{sec:#1}}
\newcommand{\Secref}[1]{Section~\ref{sec:#1}}
\newcommand{\secref}[1]{\mbox{Section~\ref{sec:#1}}}

\newcommand{\alglabel}[1]{\label{alg:#1}}
\newcommand{\Algref}[1]{Algorithm~\ref{alg:#1}}
\newcommand{\algref}[1]{\mbox{Algorithm~\ref{alg:#1}}}

\newcommand{\applabel}[1]{\label{app:#1}}
\newcommand{\Appref}[1]{Appendix~\ref{app:#1}}
\newcommand{\appref}[1]{\mbox{Appendix~\ref{app:#1}}}

\newcommand{\tablabel}[1]{\label{tab:#1}}
\newcommand{\Tabref}[1]{Table~\ref{tab:#1}}
\newcommand{\tabref}[1]{Table~\ref{tab:#1}}

\newcommand{\figlabel}[1]{\label{fig:#1}}
\newcommand{\Figref}[1]{Figure~\ref{fig:#1}}
\newcommand{\figref}[1]{\mbox{Figure~\ref{fig:#1}}}

\newcommand{\eqlabel}[1]{\label{eq:#1}}
\newcommand{\eqref}[1]{(\ref{eq:#1})}

\newtheorem{thm}{Theorem}{\bfseries}{\itshape}
\newcommand{\thmlabel}[1]{\label{thm:#1}}
\newcommand{\thmref}[1]{Theorem~\ref{thm:#1}}

\newtheorem{lem}{Lemma}{\bfseries}{\itshape}
\newcommand{\lemlabel}[1]{\label{lem:#1}}
\newcommand{\lemref}[1]{Lemma~\ref{lem:#1}}

\newtheorem{cor}{Corollary}{\bfseries}{\itshape}
\newcommand{\corlabel}[1]{\label{cor:#1}}
\newcommand{\corref}[1]{Corollary~\ref{cor:#1}}

\newtheorem{obs}{Observation}{\bfseries}{\itshape}
\newcommand{\obslabel}[1]{\label{obs:#1}}
\newcommand{\obsref}[1]{Observation~\ref{obs:#1}}

\newtheorem{assumption}{Assumption}{\bfseries}{\rm}
\newenvironment{ass}{\begin{assumption}\rm}{\end{assumption}}
\newcommand{\asslabel}[1]{\label{ass:#1}}
\newcommand{\assref}[1]{Assumption~\ref{ass:#1}}

\newcommand{\proclabel}[1]{\label{alg:#1}}
\newcommand{\procref}[1]{Procedure~\ref{alg:#1}}

\newtheorem{rem}{Remark}
\newtheorem{op}{Open Problem}

\newcommand{\etal}{\emph{et al}}

\newcommand{\voronoi}{Vorono\u\i}
\newcommand{\ceil}[1]{\left\lceil #1 \right\rceil}
\newcommand{\floor}[1]{\left\lfloor #1 \right\rfloor}



\title{\MakeUppercase{Testing the Mobility of Polygons}}
\author{Jurek Czyzowicz \and Pat Morin}
\date{}


\newcommand{\Rtwo}{\mathbb{R}^2}

\begin{document}
\maketitle

\begin{abstract}
We consider the problem of testing whether a given set of $n$ frictionless
point contacts (pins) on the edges of a polygon $P$ immobilizes $P$ in the
sense that any non-trivial continuous motion of $P$ will cause at least one of
the pins to enter the interior of $P$ during the motion.  Our main result is
that this problem can be solved in linear time.  A corollary of our algorithm
is a Helly-type theorem which states that, if a set of $n$ pins immobilizes $P$
then there is a subset of $k\le 5$ of the pins that also immobilizes $P$.  This
theorem is tight in that there exists an example of a polygon and 5 pins that
is immobilized but removing any one of the pins causes the polygon to become
mobile. 
\end{abstract}

%%%%%%%%%%%%%%%%%%%%%%%%%%%%%%%%%%%%%%%%%%%%%%%%%%%%%%%%%%%%%%%%%%%%%%%%%%%%%%%%
\section{Introduction}

Let $P$ be a simple polygon with $m$ vertices and let $S$ be a set of
$n$ points in the interiors of the edges of $P$.  We refer to the
elements of $S$ as \emph{pins}.  The set $S$ of pins
\emph{immobilizes} $P$ if, during any continuous non-trivial rigid
motion of $P$, some pin enters the interior of $P$.  In
\figref{examples}, the first three polygons are immobilized (see
Czyzowicz \etal\ \cite{X} for the proof that the triangle in
\figref{examples}.c is immobilized) and the last two polygons are
mobile and a mobilizing motion is shown.

\begin{figure}
\begin{center}\begin{tabular}{ccccc}
\includegraphics{im-square} &
\includegraphics{im-wings} &
\includegraphics{im-triangle} &
\includegraphics{m-square} &
\includegraphics{m-square2} \\
(a) & (b) & (c) & (d) & (e)
\end{tabular}\end{center}
\caption{Examples of three polygons (a, b and c) that are immobile and two polygons
(c and d) that are mobile.}
\figlabel{examples}
\end{figure}

This definition of immobility seems to have been first introduced by Czyzowicz
\etal\ \cite{X}.  There are many other definitions of immobility, which may be
more or less relevant depending on the application being considered.  One of
the most well-studied of these is Reuleaux's definition of \emph{form closure}
\cite{X} which disallows infinitesmal motions. Under Reuleaux's definition, the
3 pins in \figref{examples}.c do not achieve form closure since an infinitesmal
rotation of the triangle about the small circle is possible. (If we rotate the
pins about the small circle they trace arcs that are tangent to the edges of
the triangle.)

Many authors have studied the problem of immobilizing convex polygons using a
small set of pins.  Czyzowicz \etal\ \cite{X} show that 4 pins which can be
found in $O(m)$ time are always sufficient and sometimes necessary to
immobilize any convex polygon.  The same authors show that, if the polygon
does not have any parallel sides then $3$ pins are always sufficient.  The
number of pins required to immobilize a convex polytope in $d$ dimensions is
always between $d+1$ and $2d$ \cite{X}.  So and So \cite{ssXX} have shown that
4 pins are always necessary and sufficient to achieve form closure on any
convex polygon and the locations of these pins can be computed in linear time
\cite{X}.

For non-convex polygons these problems become somewhat more complicated.  While
4 pins is always sufficient to immobilize any simple polygon, sometimes 2 or 3
pins are also sufficient, and determining the minimum number required seems
difficult.  Cheong \etal\ study this problem and give an algorithm that
determines if 2 pins are sufficient to achieve form closure on a polygon in
$O(m^{4/3}\log^{1/3} m)$ time.  The same authors also give an algorithm for
determinining if 3 pins are sufficient to achieve form closure that runs in
$O(m^2 \log^2 m)$ time and an algorithm to test if 3 pins are sufficient to
achieve immobility that runs in $O(m^2\log^4 m)$ time. 

In this paper we study a closely related problem, namely that of testing if a
given set $S$ of $n$ pins immobilizes a polygon $P$ with $m$ edges.  Our
algorithm runs in $O(n)$ time if the input is given in such a way that each
element of $S$ has a pointer to the edge of $P$ on which it lies.  Otherwise,
these pointers can be precomputed in $O(n + m\log n)$ time.  A somewhat
surprising consequence of our algorithm is the following analog of
\emph{Helly's Theorem}: If $S$ immobilizes $P$, then $S$ contains a subset $S'$
of size at most 5 that also immobilizes $P$, and this is tight;  there are
examples in which a set of 5 pins immobilizes $P$ but no subset of 4 pins
immobilizes $P$.  

Note that testing if $S$ achieves form closure is easy since it involves
testing if a certain set of 3-dimensional vectors contains the origin in the
interior of its convex hull \cite{A,B,C,D}.  Therefore, using linear-time
linear programming \cite{X,X} we can solve this problem in linear time.  Our
algorithm for testing immobility uses this same linear program, but requires
significant care in handling the special case where the origin is on the
boundary of the convex hull. 

We also consider lower bounds for determining the minimum number of pins
required to achieve form closure on a simple polygon.  We show that testing if
2 pins is sufficient to achieve form closure is as hard as Hopcroft's problem
and testing if 3 pins is sufficient is as hard as determining if any 3 points
of a point set lie on a common line.  These lower bounds show that the results
of Cheong \etal\ cite{X} are probably close to optimal since they are within a
polylogarithmic factor of the conjectured lower bounds for these two problems.

The remainder of this paper is organized as follows:  In \secref{prelim} we
review some background and present some basic preliminary results.  In
\secref{testing} we describe our algorithm for testing if a set $S$ of pins
immobilizes a polygon $P$. In \secref{lower-bounds} we present our lower-bound
constructions.  Finally, in \secref{conclusions} we summarize and conclude with
open problems.

%===============================================================================
\section{Motion Space, Configuration Space and Helicoidal Halfspaces}

Any rigid motion of $\Rtwo$ can be expressed as a rotation about the origin
followed by a translation.  Thus, we can express any such motion as a triple
$(x,y,\theta)\in \mathbb{R}^3$, where $\theta$ represents the angle of rotation
and $(x,y)$ is the vector of translation.  This gives us a 3-dimensional
\emph{motion space} in which the origin $(0,0,0)$ represents the trivial
motion.  For a motion $M$ in the motion space, define $M(P)$ as the image of
$P$ under the motion $M$.

We are interested in a subset of the motion space called the
\emph{configuration space of $P$ with respect to $S$} or, simply, the
\emph{configuration space}.  This is the subset of the motion space that
consists of all motions $M$ such that $M(P)$ does not contain any points of $S$
in its interior.  In general, the configuration space can have many connected
components.  Because the set $S$ consists of points on the boundary of $P$, we
know that one of these components contains the trivial motion $(0,0,0)$.  

We represent a continuous motion of $P$ as a continuous curve
$f:[0,1]\rightarrow \mathbb{R}^3$ in motion space where $f(0)=(0,0,0)$, so that
$f(t)$ defines the motion of $P$ at time $t$.  The continuous motion $f$ is
\emph{allowable} if $f(t)$ is contained in the configuration space for all
$0\le t\le 1$. The continuous motion $f$ is \emph{non-trivial} if $f(t)\neq
(0,0,0)$ for some $0\le t\le 1$.  Thus, determining if $S$ immobilizes $P$ is
equivalent to determining if there is some non-trivial allowable continuous
motion $f$.  Of course, such a motion $f$ exists if and only if $(0,0,0)$ is
not the only point in its component of the configuration space.

One method of determining whether $P$ is immobilized by $S$ would be to
explicitly compute the entire configuration space, but this would result in an
algorithm with a large (but still polynomial) running time.  Rather than take
this approach, we will approximate the configuration space at two different
levels.  At the first level, we approximate by only considering a small part of
the configuration space.  At the second level, we approximate the (non-linear)
configuration space by a set of linear constraints. 

First, observe that there is a non-trivial allowable continuous motion $f$ if
and only if there is a non-trivial allowable continuous motion $f'$ such that
$f'(t)$ is contained in a ball of radius $\epsilon$ centered at $(0,0,0)$, for
any $\epsilon >0$.  The utility of this is obvious:  We need only consider the
part of the configuration space that is contained in a ball of radius
$\epsilon$ centered at $(0,0,0)$.

In order to get some idea of what the configuration space looks like, we will
consider the case where $P$ is a halfplane with one pin on its boundary.  Note
that, if we consider only motions with no rotational component, i.e., in the
motion plane $\theta=0$, then the set of allowable translations is a closed
halfplane whose bounding line is parallel to the bounding line of $P$ and that
contains the origin $(0,0)$.  Blah, blah, blah the configuration space is
closed and its boundary is a helicoid.  We call such an object a
\emph{helicoidal halfspace}.

\begin{figure}
\begin{center}\begin{tabular}{cccc}
\includegraphics{halfplane-a} & \includegraphics{halfplane-b}
\end{tabular}\end{center}
\end{figure}

Now, suppose $P$ is a polygon and the pins in $S$ are in the interiors of the
is edges of $P$.  Let $p$ be some pin in $S$.  Observe that, since $p$ is in
the interior of an edge of $P$, then there is a value of $\epsilon >0$ such
that the intersection of the configuration space of $P$ with respect to $\{p\}$
and the ball of radius $\epsilon$ is the same as the intersection of the
configuration space of a halfplane with respect to $\{p\}$ and the ball of
radius $\epsilon>0$. Therefore, for each pin $p\in S$ we obtain a helicoidal
halfspace that is the configuration space of a halfplane with respect to
$\{p\}$.  The polygon $P$ is immobilized if and only if the intersection of
these helicoidal halfspaces leaves the point $(0,0,0)$ in a component with no
other points. 

To summarize, we have reduced the problem of determining if $S$ immobilizes $P$
to a problem of determining whether the intersection of a set of $n$ helicoidal
halfspaces, each of which contains $(0,0,0)$ on its boundary leaves the point
$(0,0,0)$ in a component by itself.  This does not seem like much progress
since (1)~the problem is non-linear and (2)~helicoidal halfspaces not convex,
so the problem is not even convex.  In the next section we will show how a
careful use of linear programming will solve both these problems.

%%%%%%%%%%%%%%%%%%%%%%%%%%%%%%%%%%%%%%%%%%%%%%%%%%%%%%%%%%%%%%%%%%%%%%%%%%%%%%%%
\section{Reduction to Linear Programming}

Recall that we are only interested in studying the configuration space
restricted to an $\epsilon$ ball around the origin $(0,0,0)$ and that each of
our helicoidal halfspaces contains the origin on its boundary.  To avoid
dealing with the complexities of helicoids, we will approximate each helicoid
by a closed halfspace whose bounding plane is tangent to the helicoid at the
origin.  We call such a halfspace the \emph{halfspace approximation} of the
helicoid.

Let $h$ be a helicoidal halfspace and let $h'$ be the halfspace approximation
of $h$.  Let $C$ be a cone contained in $h$ whose apex is at $(0,0,0)$, whose
axis is orthogonal to $h'$, and that has internal angle $\alpha <\pi$.  Then,
there exists a value of $\epsilon >0$ such that the intersection of the
$\epsilon$ ball centered at $(0,0,0)$ with $C$ is contained in $h$.  

Suppose we compute the intersection of the $n$ halfspace approximations.  Then
there are several cases depending on the dimension of the resulting set.

%===============================================================================
\subsection{The Intersection has Volume}

If the intersection of the $n$ halfspace approximations has volume then we
claim that $P$ is not immobized by $S$.  To see this, just take any ray
originating at $(0,0,0)$ that enters the volume and observe that by \lemref{X}
there is an initial segment of this ray that defines a non-trivial allowable
continuous motion of $P$.

%===============================================================================
\subsection{The Intersection is the Point $(0,0,0)$}

If the intersection of the $n$ halfspace approximations is the point $(0,0,0)$
then we claim that $P$ is immobilized.\footnote{Indeed, in this case it is not
hard to verify that the set $S$ achieves the stronger notion of form closure
\cite{X}.}  To see this, consider any smooth motion $f$ and let $r$ be the ray
tangent to $f$ at $(0,0,0)$.  Since the halfspace approximations are closed, it
must be the case that $r$ is not contained on the boundary of any of the
halfspaces (otherwise their intersection would contain $r$).  Therefore, there
is some halfspace approximation $h'$ that does not contain $r$.  Since $r$ is
tangent to $f$ at $(0,0,0)$ but not tangent to $h'$, it follows that $f$ is not
tangent to $h'$.  Therefore, by \lemref{X}, some part of $f$ very close to
$(0,0,0)$ is not contained in one of the helicoidal halfspaces so $f$ is not an
allowable motion. 

%===============================================================================
\subsection{The Intersection is a Line Segment or Ray}


%===============================================================================
\subsection{The Intersection is a Surface}

If the intersection of the $n$ halfspace approximations is a 2-dimensional
surface then two of the halfspace approximations must meet in a plane and the
surface is contained in this plane.  These two halfspace approximations
correspond to a very special configuration of two pins.   Note that this plane
intersects the plane $\theta=0$ in a line, which means that the two pins are on
parallel edges.  Furthermore, these two pins define only one plane which means
that the line joining the two pins is orthogonal to the edges that contain
these pins.  Finally, since the two halfspace approximatios are oriented in
opposite directions this means that one of the pins is 




%%%%%%%%%%%%%%%%%%%%%%%%%%%%%%%%%%%%%%%%%%%%%%%%%%%%%%%%%%%%%%%%%%%%%%%%%%%%%%%%
\section{Testing Mobility}\seclabel{testing}

%%%%%%%%%%%%%%%%%%%%%%%%%%%%%%%%%%%%%%%%%%%%%%%%%%%%%%%%%%%%%%%%%%%%%%%%%%%%%%%%
\section{Lower Bounds for Computing Form Closure Grasps}\seclabel{lower-bounds}

%%%%%%%%%%%%%%%%%%%%%%%%%%%%%%%%%%%%%%%%%%%%%%%%%%%%%%%%%%%%%%%%%%%%%%%%%%%%%%%%
\section{Summary and Conclusions}\seclabel{conclusions}

\end{document}

