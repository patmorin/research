\documentclass{article}
\usepackage{amsthm}

\input{pat.tex}
\newcommand{\cn}{\mathsf{cr}}
\newcommand{\dn}{\mathsf{dn}}

\begin{document}

\begin{thm}
For any outerplanar graph $G$, $\dn(G)\le 2$.  Furthermore, for any
outpuerplanar graph $G$, $\dn(G)$ can be computed in $O(n)$ time.
\end{thm}

\begin{proof} 

We begin with the proof that $\dn(G)\le 2$. Without loss of
generality, assume that $G$ is maximal outerplanar. That is, in any
planar embedding of $G$ all faces except the outer face are triangles.
First, color the vertices of $G$ using the colors $\{1,2\}$ so that
each triangle of $G$ has one vertex of color 1 and two vertices of
color 2.\notice{explain why this is possible}  We obtain an
embedding of $G$ using two edge lengths by assigning the length of an
edge $uv$ to be 1 if $u$ and $v$ are assigned different colors or
$\gamma$ if $u$ and $v$ are assigned the same color (which must be 2).
We call these the \emph{unit edges} and the \emph{$\gamma$ edges},
respectively.  Note that, with this embedding, each triangle of $G$
becomes an isosceles triangle with two sides of length 1 and one side
of length $\gamma$.

We first show that no two vertices $u$ and $v$ of $G$ are embedded at
the same point.  Select an arbitrary vertex $r$ and consider a
shortest path from $r$ to $u$ in the graph $G$.  In the embedding this
translates into a sequence of vectors $u_1,\ldots,u_k$ such that
$r+u_1+\cdots+u_k=u$ and where $\|u_i\|\in \{1,\gamma\}$ for all $1\le
i\le k$.  Furthermore, there are no two vectors of $u$ that 


We must show that no two vertices of

\end{proof}


\end{document}
