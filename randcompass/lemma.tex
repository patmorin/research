\documentclass{article}
\usepackage{amsthm}

\newtheorem{clm}{Claim}
\newtheorem{lem}{Lemma}
\newcommand{\E}{\mathrm{E}\,}

\begin{document}


A \emph{walk} on $\{-n,\ldots,n\}$ is a sequence $\langle
w_1,\ldots,w_k\rangle$ where each $w_i\in\{-n,\ldots,n\}$ and
$|w_i-w_{i+1}| = 1$ for all $i\in\{1,\ldots,k-1\}$.  Note that
there is an injective mapping from walks of length $k$ onto vectors
in $\{-1,1\}^{k-1}$, where $\langle w_1,\ldots,w_k\rangle$ maps
to $\langle w_2-w_1,w_3-w_2,\ldots,w_k-w_{k-1}\rangle$.

We want to study a random walk on $\{-n,\ldots,n\}$ defined by a vector
of probabilities $R=(r_{-n},\ldots,r_n)$ with $r_{-n} = 1$, $r_n = 0$,
and $0< r_i < 1$, for all $i \in \{-n+1,...,n-1\}$. When at location $i$,
the walk moves ``right'' to $i+1$ with probability $r_i$.  Otherwise (with
probability $1-r_i$) the walk moves ``left'' to position $i-1$.

Let $W$ be the set of all walks on $\{-n,\ldots,n\}$ that start at
0, include both $n$ and $-n$, visiting $n$ first and ending at the
first visit to $-n$.  That is, an element in $W$ is a walk $\langle
w_1,\ldots,w_k\rangle$ where $w_1=0$, $w_k = -n$, $w_i = n$ for some
$i\in\{2,\ldots,k-1\}$ and $w_i\neq -n$ for any $i\in\{1,\ldots,k-1\}$.

Define the \emph{head} of a walk in $w\in W$ as the shortest prefix
of $w$ that includes $n$ followed by a return to 0. That is, the head
of $w$ is the shortest prefix $\langle w_1,\ldots,w_i\rangle$ of $w$
such that $w_i=0$ and $w_j = n$ for some $j\in\{1,\ldots,i-1\}$.

\begin{clm}
Let $W_{\le k}$ be the subset of $W$ consisting of walks whose head is of
length at most $k$.  Then $\Pr(W_k) \le k^2 e^{-\Omega(n^2/k)}$.
\end{clm}

\begin{proof}
If we could show that $\Pr(w) \le 2^{-k}$ when $w$ is a head of length $k$, then we have:
\[
  \Pr(W_{\le k}) 
  = \sum_{j=1}^k\sum_{w\in W_{j}} 2^{-j}
  = \sum_{j=1}^k|W_{j}| 2^{-j}
  \le \sum_{j=1}^k 2^je^{-\Omega(n^2/k)}2^{-j}
  = ke^{-\Omega(n^2/k)} \enspace .
\]
The only sticky situation we have is when $w$ uses the edge $(n-1,n)$ a lot.
\end{proof}




\begin{clm}\label{claim:one}
Let $W'$ be the set of all walks in $W$ that share some common head $w'$.
Then $\Pr(W') \le (1/2)^{2n}$.
\end{clm}

\begin{proof}
Observe that $\Pr(w')$ is the product of $|w|$ factors and these factors
include $r_i$ and $1-r_i$ for all $i\in\{0,\ldots,n-1\}$.  Therefore,
since $r_i(1-r_i) \le 1/4$, we have that $\Pr(w') \le (1/4)^n =
(1/2)^{2n}$, as required.
\end{proof}


\begin{clm}
Let $W_k$ be the subset of $W$ consisting of walks whose head is of
length at most $k$.  Then $\Pr(W_k) \le k^2 2^{k-2n}e^{-\Omega(n^2/k)}$.
\end{clm}

\begin{proof}
Partition $W_k$ into classes $W_{k,1},\ldots,W_{k,t}$ where each class has
the same head so that
\begin{equation}
  \Pr(W_k) = \sum_{i=1}^t \Pr(W_{k,i}) \le t(1/2)^{2n} \enspace ,
  \label{eq:prob-a}
\end{equation}
where the inequality follows from Claim~\ref{claim:one}.  Therefore, we
can complete the proof by upper bounding $t$, the number of heads of length at most $k$.
We do this using Chernoff's bounds.
Chernoff's Bounds show that, for a 
vector $V$ chosen uniformly at random from $\{-1,1\}^{k'}$ with $k' \le k$,
\[
   \Pr\{|\Sigma V| \ge \epsilon k\} \le e^{-\Omega(\epsilon^2 k)} \enspace .
\]
Applying this bound with $\epsilon = n/k$ we obtain
\[
   \Pr\{|\Sigma V| \ge n\} \le e^{-\Omega(n^2/k)} \enspace .
\]
Since $V$ is chosen uniformly at random from $2^{k'}$ possibilities,
this means that the number of $V$ with $|\Sigma V| \ge n$ is at most
\begin{equation}
  2^{k'}e^{-\Omega(n^2/k)} \enspace.
  \label{eq:counting}
\end{equation}

The head of any class $W_{k,i}$ maps to a vector $V=v_1,\ldots,v_{k'}$,
with $k'\le k$, $\left|\sum_{i=1}^{k''} v_i \right| \ge n$, and
$\left|\sum_{i=k''}^{k'} v_i \right| \ge n$ where $k'' \le k'$ ($k''$
is the first time at which the walk visits $n$).

Therefore, by (\ref{eq:counting}), the number of such classes is at most
\begin{eqnarray*}
  t & \le & \sum_{k'=1}^{k}\sum_{k''=1}^{k'} 
     \left(2^{k''}e^{-\Omega(n^2/k)}\right)
     \left(2^{k'-k''}e^{-\Omega(n^2/k)}\right) \\
   & \le & \sum_{k'=1}^{k}\sum_{k''=1}^{k'} 2^{k'}e^{-\Omega(n^2/k)} \\
   & \le & k^2 2^{k'}e^{-\Omega(n^2/k)} \\
   & \le & k^2 2^{k}e^{-\Omega(n^2/k)} \enspace ,
\end{eqnarray*}
so the claim follows from (\ref{eq:prob-a}).
\end{proof}

We are now ready to prove the main result of this section.  Let $h(i,j)$ denote the expected number of steps for a walk that starts at position $i$ to reach position $j$.

\begin{lem}
For any choice of $R$, $\max\{h(0,-n), h(0,n)\} = \Omega(n^2)$.
\end{lem}

\begin{proof}
Without loss of generality (by symmetry), assume that every walk is a
walk in $W$.  If $\max\{h(0,-n), h(0,n)\}=k/2$ then, with probability at
least $1/2$, the walk has length at most $k$.  In particular, the walk has
a head of length at most $k$. By Claim~\ref{clm:two}, we therefore have
\[
   1/2 \le \Pr(W_k) \le k^2 2^{k-2n}e^{-\Omega(n^2/k)}
\]
or
\[  -1 \le 2\log k + k - 2n - (n^2/k)\log(e) \]
\[ (n^2/k)\log(e) +  2n - 1 \le 2\log k + k  \]

Suppose, without loss of generality, that, with probability at least $1/2$, the
walk reaches $n$ before reaching $-n$.  Then $\max\{h(0,n), h(0,n)\} = \Omega(n^2)$.



\end{proof}






%
%Let $h(i,j)$ denote the expected number of steps for a walk that starts at position $i$ to reach position $j$.
%
%\begin{clm}
%For any choice of $R$, $\max\{h(0,-n), h(0,n)\} = \Omega(n^2)$
%\end{clm}
%
%\begin{proof}
%Assume, without loss of generality, that with probability
%at least $1/2$, the walk reaches $n$ before reaching $-n$.  Then
%\[
%    h(0, -n) \ge (1/2)(h(0,n) + h(n,0)) \enspace .
%\]
%
%Let $S_k$ be the set of all walks $\langle w_1,\ldots,w_k\rangle$ on
%$\{-n,\ldots,n\}$ where $w_1 = 0$, $w_k=n$ and $w_i\in\{1,\ldots,n-1\}$ for any
%$i\in\{1,\ldots,k-1\}$.  From the theory of Markov processes, we have
%\[
%   \sum_{k=n}^\infty \sum_{\vec{w}\in S_k} k(1/2)^k = \Omega(n^2)
%\]
%\cite{X,Y}.
%
%For some walk $\overrightarrow w$ in $S_k$, let $\overleftarrow w$ denote the reversal of $w$.  Note that the probability of $\overrightarrow w$ is given by
%\[
%  \Pr\{\overrightarrow{w}\} = \prod_{i=1}^{k-1} x_i
%\]
%where $x_i = r_i$ or $x_i=(1-r_i)$, depending on wether $w_{i+1}=w_i+1$ or $w_{i+1} = w_i -1$, respectively.  On the other hand, for the reverse walk $\overleftarrow{w}$ we have    
%\[
%  \Pr\{\overleftarrow{w}\} = \prod_{i=1}^{k-1} (1-x_i)
%\]
%Finally, note that
%\[
%   h(0,n) + h(n,0) \ge 
%    \sum_{k=n}^\infty \sum_{\overrightarrow{w}\in S_k}  
%      k(\Pr(\overrightarrow{w} + \Pr(\overleftarrow{w}))
%\]
%
%
%Consider a fair version of the random walk, in which $r_i=1/2$ for all
%$i\in\{-n+1,\ldots,n-1\}$.  It is well-know that, for this model,
%$\E[\min\{h(0,-n),h(0,n)\}]=\Omega(n^2)$ \cite{something}.
%
%
%
%Suppose, without loss of generality that, with probability at least $1/2$, the walk reaches $n$ before reaching $j$
%\end{proof}
%
\end{document}

