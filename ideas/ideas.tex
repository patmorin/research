\documentclass[12pt]{article}
\usepackage{graphicx,ipe}
\usepackage{fullpage}

\input{pat.tex}

\setlength{\parskip}{.5cm}

\begin{document}
\section{Data Structures}

\subsection{Ordered Theta Graphs}

Let $S$ be a set of $n$ points in the plane and process the points
one at a time by repeatedly choosing an unprocessed point $p$ and
\emph{inserting} $p$ as follows:
Draw two edges; one joining $p$ to its nearest previously-inserted
neighbour in $p$'s upper right quadrant and an edge joining $p$ to its
nearest previously-inserted neighbour in $p$'s lower left quadrant.

\centeripe{theta}

The result of this process is a geometric graph.  In general, a
vertex of this graph can have degree as large as $n-1$.

Prove or disprove: For any point set $S$, there exists an ordering of
the points of $S$ such that, in the resulting graph, every point has
degree at most $\Delta$, for some constant $\Delta$.

What is known:  There exists an ordering of the points of $S$ such that
the resulting graph has degree at most $2H_n+2$ \cite{bgm02}.

\subsection{Working Set Ternary Tries}

Consider dictionary data structures that store keys $k_1,\ldots,k_n$
and let $t(k_i)$ be defined as follows.  If $k_i$ was never accessed
before then $t(k_i)=n$.  Otherwise, $t(k_i)$ is the number of distinct
items accessed since the last time $k_i$ was accessed.

Open Problem: Use ternary tries to develop a data structure for
storing strings where the worst-case cost to search for the string $s$
is $O(|s| + \log t(s))$.

What is known: Splay trees can be used to access $s$ in $O(|s| +\log
t(s))$ amortized time, but the worst case search time can be
$\Theta(|s|+ n)$.  If we apply Iacono's $2^{2^i}$ trick \cite{i01}
then we get a data structure that accesses $s$ in $O(|s|+\log t(s)
+\log\log n)$ time.  A good place to start would be with Iacono's
trick and see if it's possible to drop the extra $\log\log n$ term.

\subsection{Queueish Dictionaries}

Open Problem: Find a dictionary data structure in which every access
to item $x$ after the first access takes $O(\log (n-t(x)))$ time.
This data structure is good for periodic access sequences.  An example
is the Ministry of Transport, where people return every 4 years to
renew their licenses.

What is known: There is a dictionary data structure that takes $O(\log
(n-t(x)) + \log\log n)$ time to access $x$ \cite{il02}.  This data
structure is also based on Iacono's $2^{2^i}$ trick.

\bibliographystyle{plain}
\bibliography{ideas}

\end{document}