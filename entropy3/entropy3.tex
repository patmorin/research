\documentclass{patmorin}
\usepackage{pat}

\title{Point Location in Disconnected Planar Subdivisions}

\author{Prosenjit~Bose, 
	Karim~Dou\"{\i}eb, 
	Vida~Dujmovi\'c, 
	Jamie~King, and 
	Pat~Morin}



\begin{document}
\maketitle

\section{Introduction}

$G$ is a planar subdivision $D$ is a probability measure over $\R^2$.

After a bunch of other papers, Collette \etal\ gave an $O(n)$ space data
structure that preprocesses a connected planar subdivision $G$ and a
probability measure $D$ over $\R^2$ such that a point location query in
$G$ can be answered in $O(H)$ expected time using.  The expected number
of point-line comparisons needed to answer a query is $H + O(H^{2/3}+1)$.
Here $H$ is a lower-bound on the expected time required by any linear
decision tree for answering queries on $G$ that are drawn according
to $D$. Their work leaves open the problem of what to do when $G$
is disconnected.


\section{Review}

Collette \etal\ show that, up to a lower-order term, the expected
number of comparisons performed by the optimal decision tree for point
location in $G$ is equal to the entropy of the minimum-entropy Steiner
triangulation of $G$.  Thus, our goal is to find a near-minimum entropy
Steiner triangulation of $G$.

The first tool we use is simplicial partitions, from the field of geometric
range searching: 

\begin{thm}[Matou\v{s}ek 1991]\thmlabel{point-partition}
Let $S$ be a set of $n$ points in $\R^2$. There exists a constant
$c$ such that, for any $1\le r \le n$, there exists a sequence
$\langle \Delta_1,\ldots,\Delta_r\rangle$ of triangles such that
  \begin{enumerate}
    \item $S\subseteq \bigcup_{i=1}^r \Delta_i$,
  
    \item $\left|\Delta_i \cap S\setminus
    \left(\bigcup_{j=1}^{i-1}\Delta_j\right)\right| \le 2n/r$, and
  
    \item For any line $\ell$, there are at most $cr^{1/2}$ elements of
  $\{\Delta_1,\ldots,\Delta_r\}$ that intersect $i$.
  \end{enumerate}
\end{thm}

Note that Part~2 of \thmref{point-partition} is not the original
statement of the theorem, but follows from Matou\v{s}ek's proof \cite{m91}.
Restating \thmref{point-partition} in terms of probability distributions,
we have:

\begin{thm}\thmlabel{prob-partition}
Let $S$ be a set of $n$ points in $\R^2$. There exists a constant
$c$ such that, for any $1\le r \le n$, there exists a sequence
$\langle \Delta_1,\ldots,\Delta_r\rangle$ of triangles such that
  \begin{enumerate}
    \item $\Pr\{\bigcup_{i=1}^r \Delta_i\} = 1$,
  
    \item $\Pr\left\{\Delta_i \cap S\setminus
    \left(\bigcup_{j=1}^{i-1}\Delta_j\right)\right\} \le 2/r$, and
  
    \item For any line $\ell$, there are at most $cr^{1/2}$ elements of
    $\{\Delta_1,\ldots,\Delta_r\}$ that intersect $i$.
  \end{enumerate}
\end{thm}

(Someone needs to prove this, or something close, as well as give an
algorithmic version.  Sampling from $D$ and then applying
\thmref{point-partition} should do it.)

We use \thmref{prob-partition} as follows.  We find the set of
triangles $\Delta=\{\Delta_1,\ldots,\Delta_r\}$.  We construct the
arrangement of triangles in $\Delta$ and triangulate its faces to obtain
a triangulation $A$.  Next, for each face $F$ of $A$ that intersects
some vertex or edge of $G$, we recursively apply the same procedure
on the distribution $D_{|F}$.  We stop this recursion when a face $F$
is empty or when the level of recursion exceeds $\log_{\alpha r} n$.

This construction defines a tree $T=T(S,D)$ in which each node has
$O(r^2)$ children.  Each node $v$ of $T$ is associated with a triangle
$\Delta(v)$.  We call $v$ a \emph{terminal node} if $\Delta(v)$ does not
intersect any edge or vertex of $G$.  Note that every terminal node is a
leaf of $T$ but, since we limit the depth of recursion to $\log_{\alpha r}
n$, not every leaf is a terminal node.

The number of nodes in $T$ is $(O(r^2))^{\log_{\alpha r} n} =
O(n^{2\alpha+\epsilon})$, where $\epsilon$ is a decreasing function of $r$.


\bibliographystyle{plain}
\bibliography{entropy3}
\end{document}
