\documentclass{patmorin}
\usepackage{amsthm,amsmath,graphicx,stmaryrd}
\usepackage{pat}

\newcommand{\eps}{\varepsilon}

\title{\MakeUppercase{On the Average Stretch of Theta Graph Paths}}
\author{Prosenjit Bose,
        Jean-Lou De Carufel,
        Nima Hoda, and
        Pat Morin\thanks{All authors are members of the School of Computer Science at Carleton University.}}



\begin{document}
\maketitle

\begin{abstract}
  The abstract goes here.
\end{abstract}

\newcommand{\tkgraph}{$\theta_k$-graph}

\section{Introduction}

For any point set $V\subset\R^2$, the \tkgraph\ of $V$ \cite{keilXX}
is a geometric graph whose vertex set is $V$.  For $k\ge 6$, this graph
is known to be a \emph{spanner}:  For any pair of vertices $x$ and $y$,
\begin{equation}
     \frac{\|xy\|_\theta}{\|xy\|} \le \frac{1}{1-2\sin(\pi/k)} \eqlabel{spanning}
\end{equation}
where $\|xy\|_\theta$ denotes the length of the Euclidean shortest
path from $x$ to $y$ that uses only edges of the \tkgraph\ graph
and $\|xy\|$ denotes the Euclidean distance between $x$ and $y$.
The proof of \eqref{spanning} constructs a path of length at most
$\|xy\|/(1-2\sin(\pi/k))$ using a simple online algorithm.  Call a path constructed by this algorithm a $\theta$-path and denote by
$\|xy\|_{\theta'}$ the length of the $\theta$-path from $x$ to $y$.

In this paper, we study the average length of paths found by the online
routing algorithm in \tkgraph s of random point sets.  In particular, for a set, $V$, of $n$ points independently and uniformly distributed in $[0,1]^2$, we consider the quantity
\[
   \E\left[\frac{\|xy\|_{\theta'}}{\|xy\|}\right]
 =      \binom{n}{2}^{-1}
        \left(
          \sum_{x\in V}
          \sum_{y\in V\setminus\{x\}} \frac{\|xy\|_{\theta'}}{\|xy\|}
        \right) \enspace .
\]
This quantity is the average stretch factor of a $\theta$-paths in a
random point set.

\section{The First Section}

Like many problems of this sort, care must be taken to handle
boundary effects.  We call a vertex $x$ if $G_\theta$ \emph{good} if
$x\in\left[\sqrt{(c\log n)/n},1-\sqrt{(c\log n)/n}\right]^2$.

Our first lemma says that, with high probability, $G_\theta$ does not
contain long edges except possibly near the boundary of $[0,1]^2$.

\begin{lem}\lemlabel{no-long-edge}
  With probability $1-\exp(-\Omega(cn))$, $G_\theta$ contains no edge
  $xy$ such that both $x$ and $y$ are good and $\|xy\|\ge c\sqrt{(\log n)/n}$.
\end{lem}

\begin{proof}[Proof Sketch]
  This would imply a triangle of area $\Omega(c(\log n)/n)$ that is
  contained in $[0,1]^2$ but that contains no points of $V$.
\end{proof}


\begin{lem}\lemlabel{angle}
  Let $x$ and $y$ be a pair of good vertices in $V$;
  let $x=x_1,x_2,\ldots,x_r=y$ denote the $\theta$-path from $x$ to $y$; and 
  let $\alpha_i$ denote the smaller of the two angles between the line
  through $x_i$ and $x_{i+1}$ and the line through $x$ and $y$.  Then
  $\E[1/\cos(\alpha_i)] \ge z_k-o(1)$ for all $i\in\{1,\ldots,r-2\}$.
\end{lem}

\begin{proof}
  The construction of the $\theta$-path from $x$ to $y$ begins by growing
  an isosceles triangle with apex at $x$ and apex angle $2\pi/k$; see
  \figref{grow}.  The growth process stops when the base of this triangle;
  \begin{enumerate}
    \item touches the boundary of $[0,1]^2$,
    \item touches $y$, or
    \item touches a point $x_2\in V\setminus\{x,y\}$.
  \end{enumerate}
  Observe that, in the third case, the point $x_2$ is uniformly
  distributed on the base of the triangle.  This implies that [with
  some work to determine the value of $z_k$], in the third case
  $\E[1/\cos(\alpha_i)] \ge z_k$.  

  \begin{figure}
    \begin{center}
      \includegraphics{grow}
    \end{center}
    \caption{Routing from $x$ to $y$ starts by growing a triangle.}   
  \end{figure}
\end{proof}   

Let $Z$ be the set of pairs $(x,y)\in V$, $x\neq y$, such that both $x$ and $y$ are good and $\|xy\| > r\sqrt{c\log n/n}$. 

\begin{lem}
  For a random pair $(x,y)\in Z$,
  \[ 
    \E\left[\frac{\|xy\|_\theta'}{\|xy\|}\right] \ge (1-O(1/r))f(\theta)
  \]
\end{lem}

\begin{proof}[Proof Sketch]
  Since, by \lemref{no-long-edge}, there are no long edges on the path
  from $x$ to $y$, the path has at least $r$ edges.  By \lemref{angle},
  all but the last of these edges deviates by an (expected) angle of $z_k$
  from the direction of the straight path from $x$ to $y$.
\end{proof}


\begin{lem}
   With high probability, $|Z| = \binom{n}{2}-O((n^{3/2}+nr^2)\log n)$.
\end{lem}

\begin{proof}[Proof Sketch]
  The number of not-good vertices is $O(n^{1/2}\log n)$ w.h.p.  Each of
  these eliminates $n-1$ possible pairs from $Z$, so these eliminate
  $O(n^{3/2}\log n)$ possible pairs.   The number of vertices at
  distance at most $k\sqrt{c(\log n)/n}$ from any vertex is, w.h.p.,
  $O(r^2\log n)$.  Therefore, this condition eliminates $O(nr^2\log n)$
  possible pairs from $Z$.  Any of the $\binom{n}{2}$ pairs not eliminated
  by one of these two conditions is in $Z$.
\end{proof}

\begin{thm}
  The average stretch factor of a $\theta$-path in $G_\theta$ is
  at least $(1-o(1))f(z_k)$.
\end{thm}

\begin{proof}
   The expected total stretch factor is
  \begin{eqnarray*}
  \E\left[\sum_{\{x,y\} \subset V} \frac{\|xy\|_{\theta'}}{\|xy\|}\right]
  & \ge & \E\left[\sum_{(x,y)\in Z} \frac{\|xy\|_{\theta'}}{\|xy\|}\right] \\
  & \ge & \left(\binom{n}{2}-O((n^{3/2}+nr^2)\log n)\right)(1-1/r)f(\theta) \\
  \end{eqnarray*}
\end{proof}

\section{The Second Section}

\section{Summary}
\seclabel{summary}

\section*{Acknowledgement}

The work of Pat Morin was partly funded by NSERC and CFI.

\bibliographystyle{plain}
\bibliography{stretch}


\end{document}


