\documentclass{article}
\usepackage{amsthm}

 
%\usepackage{amsthm}

\newcommand{\centeripe}[1]{\begin{center}\Ipe{#1}\end{center}}
\newcommand{\comment}[1]{}

\newcommand{\centerpsfig}[1]{\centerline{\psfig{#1}}}

\newcommand{\seclabel}[1]{\label{sec:#1}}
\newcommand{\Secref}[1]{Section~\ref{sec:#1}}
\newcommand{\secref}[1]{\mbox{Section~\ref{sec:#1}}}

\newcommand{\alglabel}[1]{\label{alg:#1}}
\newcommand{\Algref}[1]{Algorithm~\ref{alg:#1}}
\newcommand{\algref}[1]{\mbox{Algorithm~\ref{alg:#1}}}

\newcommand{\applabel}[1]{\label{app:#1}}
\newcommand{\Appref}[1]{Appendix~\ref{app:#1}}
\newcommand{\appref}[1]{\mbox{Appendix~\ref{app:#1}}}

\newcommand{\tablabel}[1]{\label{tab:#1}}
\newcommand{\Tabref}[1]{Table~\ref{tab:#1}}
\newcommand{\tabref}[1]{Table~\ref{tab:#1}}

\newcommand{\figlabel}[1]{\label{fig:#1}}
\newcommand{\Figref}[1]{Figure~\ref{fig:#1}}
\newcommand{\figref}[1]{\mbox{Figure~\ref{fig:#1}}}

\newcommand{\eqlabel}[1]{\label{eq:#1}}
\newcommand{\eqref}[1]{(\ref{eq:#1})}

\newtheorem{thm}{Theorem}{\bfseries}{\itshape}
\newcommand{\thmlabel}[1]{\label{thm:#1}}
\newcommand{\thmref}[1]{Theorem~\ref{thm:#1}}

\newtheorem{lem}{Lemma}{\bfseries}{\itshape}
\newcommand{\lemlabel}[1]{\label{lem:#1}}
\newcommand{\lemref}[1]{Lemma~\ref{lem:#1}}

\newtheorem{cor}{Corollary}{\bfseries}{\itshape}
\newcommand{\corlabel}[1]{\label{cor:#1}}
\newcommand{\corref}[1]{Corollary~\ref{cor:#1}}

\newtheorem{obs}{Observation}{\bfseries}{\itshape}
\newcommand{\obslabel}[1]{\label{obs:#1}}
\newcommand{\obsref}[1]{Observation~\ref{obs:#1}}

\newtheorem{assumption}{Assumption}{\bfseries}{\rm}
\newenvironment{ass}{\begin{assumption}\rm}{\end{assumption}}
\newcommand{\asslabel}[1]{\label{ass:#1}}
\newcommand{\assref}[1]{Assumption~\ref{ass:#1}}

\newcommand{\proclabel}[1]{\label{alg:#1}}
\newcommand{\procref}[1]{Procedure~\ref{alg:#1}}

\newtheorem{rem}{Remark}
\newtheorem{op}{Open Problem}

\newcommand{\etal}{\emph{et al}}

\newcommand{\voronoi}{Vorono\u\i}
\newcommand{\ceil}[1]{\left\lceil #1 \right\rceil}
\newcommand{\floor}[1]{\left\lfloor #1 \right\rfloor}


\newcommand{\cn}{\mathsf{cr}}
\newcommand{\dn}{\mathsf{dn}}

\begin{document}

\begin{thm}
For any outerplanar graph $G$, $\dn(G)\le 2$.  Furthermore, for any
outpuerplanar graph $G$, $\dn(G)$ can be computed in $O(n)$ time.
\end{thm}

\begin{proof} 

We begin with the proof that $\dn(G)\le 2$. Without loss of
generality, assume that $G$ is maximal outerplanar. That is, in any
planar embedding of $G$ all faces except the outer face are triangles.
First, color the vertices of $G$ using the colors $\{1,2\}$ so that
each triangle of $G$ has one vertex of color 1 and two vertices of
color 2.\notice{explain why this is possible}  We obtain an
embedding of $G$ using two edge lengths by assigning the length of an
edge $uv$ to be 1 if $u$ and $v$ are assigned different colors or
$\gamma$ if $u$ and $v$ are assigned the same color (which must be 2).
We call these the \emph{unit edges} and the \emph{$\gamma$ edges},
respectively.  Note that, with this embedding, each triangle of $G$
becomes an isosceles triangle with two sides of length 1 and one side
of length $\gamma$.

We first show that no two vertices $u$ and $v$ of $G$ are embedded at
the same point.  Select an arbitrary vertex $r$ and consider a
shortest path from $r$ to $u$ in the graph $G$.  In the embedding this
translates into a sequence of vectors $u_1,\ldots,u_k$ such that
$r+u_1+\cdots+u_k=u$ and where $\|u_i\|\in \{1,\gamma\}$ for all $1\le
i\le k$.  Furthermore, there are no two vectors of $u$ that 


We must show that no two vertices of

\end{proof}


\end{document}
