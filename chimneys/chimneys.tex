\documentclass[12pt]{article}
\usepackage{amsopn,amsthm,amsmath}

\newtheorem{thm}{Theorem}
\newtheorem{lem}{Lemma}
\DeclareMathOperator{\cn}{cr}

\newcommand{\const}{268.19}

\begin{document}
The proof of the upper-bound is closely related to Sz\'ekely's proof of
the Szem\'eredi-Trotter Theorem.  Indeed, a version of our upper bound,
with weaker constants, can be proven using only the Szemer\'edi-Trotter
Theorem.

We make use of the following version of the Crossing Lemma, which was
proven by Pach, Radoici\'c, Tardos, and T\'oth (SoCG~2004):

\begin{lem}[Crossing Lemma]\label{lem:crossing}
Let
$\beta=103/6$, $\gamma = 1024/31827$, and let
$G$ be a graph with no self loops, no parallel edges, $v$ vertices, and
$e > \beta v$ edges.  Then
\[
  \cn(G) \ge \gamma \cdot \frac{e^3}{v^2} \enspace ,
\]
\end{lem}


\begin{thm}
Let $S$ be a set of $r$ points and let $H_0,\ldots,H_{t-1}$ be a
set of lines such that the orthogonal projection of $S$ onto $H_{j}$
gives exactly $i+j$ distinct values, for some $i\ge 8$ and for each
$j\in\{0,\ldots,t-1\}$.  Then $t \le \const i+1$.
\end{thm}

\begin{proof}
The proof is by contradiction. Each projection direction $H_j$ defines a set $L_j$ of $i+j$ parallel lines,
each of which contains at least one point of $S$.  In total, this defines
a set $L=L_0\cup\cdots\cup L_{t-1}$ of
\[
   \ell(t) = \sum_{j=0}^{t-1} (i+j) = ti + t(t-1)/2 \le ti + t^2/2 \enspace .
\]
lines.  The total number of incidences between the these lines and
the points in $S$ is $tr$.

Next we observe that the number of points, $r$, satisfies
\[  r \ge i+t-1 
\]
since there are $i+t-1$ parallel lines in $L_{t-1}$ and each of these
contains at least one point of $S$.  Furthermore, the points of $S$
are all contained in the intersections of lines in $L_{i}$ with lines
in $L_{i+1}$, so
\[
   r \le i(i+1) \enspace .
\]

Define a geometric graph $G$ having $r+2$ vertices consisting
of the points in $S$ plus two points ``at infinity,'' as in
Figure~\ref{fig:G}. Observe that $G$ is a graph with no self-loops and no
parallel edges. The number of
edges in $G$ is exactly $tr + \ell(t)$.

Observe that we have a drawing of $G$ so that the only crossings between
edges occur where lines in $L$ intersect each other.  The total number of
intersecting pairs of lines in $L$ is 
\[
  \begin{aligned}
    \chi(t) 
      & = \sum_{j=1}^{t-1}(i+j)\cdot\sum_{k=0}^{j-1}(i+k) \\
      & \le \sum_{j=1}^{t-1}(i+j)(ij + j^2/2) \\
      & \le \sum_{j=1}^{t-1}(i^2j+3ij^2/2 + j^3/2) \\
      & \le i^2t^2/2 + it^3/2 + t^4/8 \enspace .
  \end{aligned}
\]
At this point, one would normally apply the Crossing Lemma
(Lemma~\ref{lem:crossing}) to $G$.  However, this turns out to be
insufficient to prove a linear bound on $t$.  Instead, we will consider
a subgraph of $G$.  Let $G_{s}$, $s\in\{1,\ldots,t\}$, be defined like
$G$, but only with respect to the set of lines in $L_0,\ldots,L_{s-1}$.
Then, by Lemma~\ref{lem:crossing},
\begin{equation}
   sr + \ell(s) \le \beta(r+2)
     \label{eq:few-edges}
\end{equation}
or
\begin{equation}
  \chi(s) \ge \gamma\cdot\frac{(sr+\ell(s))^3}{(r+2)^2}
     \label{eq:many-crossings}
\end{equation}
for each $s\in\{0,\ldots,t-1\}$.
In particular, select $s=2i$.  Then Eq.~(\ref{eq:few-edges}) implies that
\[  
   2ir + \ell(2i)  \le \beta(r+2)
\]
which implies that $r \le 2/(2i/\beta-1)-\ell(s)/\beta$, so 
\[
  t < 2/(2i/\beta-1) - \ell(s)/\beta - i + 1 \le \const i + 1
\]
for every non-negative integer $i$.
Therefore, we may assume that 
Eq.~(\ref{eq:many-crossings}) holds.
Letting $t=ci$ and 
rewriting Eq.~(\ref{eq:many-crossings}), we get
\[
  \begin{aligned}
  s 
   &\le \frac{\chi(s)^{1/3} (r+2)^{2/3}}{\gamma^{1/3}r} 
       - \frac{\ell(s)}{r} \\ 
   &\le \frac{\chi(s)^{1/3} (r^{2/3}+2^{2/3})}{\gamma^{1/3}r} 
      - \frac{\ell(s)}{r} \\
   & = \frac{\chi(s)^{1/3}r^{2/3}+\chi(s)^{1/3}2^{2/3}}{\gamma^{1/3} r} 
          - \frac{\ell(s)}{r}  \\
   & = \frac{\chi(s)^{1/3}r^{2/3}}{\gamma^{1/3} r} 
           +\frac{\gamma^{-1/3}(2\ell(s))^{2/3} - \ell(s)}{r} 
    & \mbox{ (since $\chi(s) \le \ell(s)^2$)} \\
   &\le \frac{\chi(s)^{1/3}}{(\gamma r)^{1/3}}
    & \mbox{ (for $\ell(s) \ge 125 \Leftrightarrow i\ge 8$))} \\
   &\le \frac{(i^2s^2/2 + is^3/2 + s^4/8)^{1/3}}{(\gamma r)^{1/3} }  \\
   &\le \frac{((is)^{2/3}/2^{1/3} + i^{1/3}s/2^{1/3} + s^{4/3}/2)}{(\gamma r)^{1/3}}  \\
   &\le \frac{(2^{1/3}+2^{2/3}+2^{1/3})i^{4/3}}{(\gamma r)^{1/3}}  
    & \mbox{ (since $s=2i$)} \\
   & = \frac{(2^{4/3} + 2^{2/3})i^{4/3}}{(\gamma r)^{1/3}}  \\
   & =  \frac{(2^{4/3} + 2^{2/3})i^{4/3}}{(\gamma(i+t-1))^{1/3}} \\ 
   &\le  \frac{(2^{4/3} + 2^{2/3})i^{4/3}}{(\gamma(1+c-1/i)i)^{1/3}}  
    & \mbox{ (since $t=ci$)} \\
   & =  \frac{(2^{4/3} + 2^{2/3})i}{(\gamma(1+c-1/i))^{1/3}} \\ 
   & < 2i = s
  \end{aligned}
\]
where the last inequality yields the desired contradiction provided that
$c > (2^{1/3} + 2^{-1/3})^3/\gamma-1+1/i$.  Selecting $c=\const+1/i$
gives $t=\const i + 1$ and completes the proof.
\end{proof}



\end{document}
