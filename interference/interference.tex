\documentclass{patmorin}
\usepackage{amsmath,amsthm}
\usepackage{pat}

\newtheorem{cor}{Corollary}

\title{Maximum Interference for Transmitters Uniformly Distributed on a Segment}
\author{Evangelos Kranakis, Danny Krizanc, and Pat Morin}

\begin{document}
\maketitle

\begin{abstract}
We consider the following problem: A set of $n$ transmitters are
place independently and uniformly at random on a unit interval.  Each
transmitter, except the two extreme ones, adjust their transmission ranges
so that they can be heard by the further of their left neighbour and right
neighbour.  What is the maximum number of transmitters that can be heard
by any individual transmitter?  We show that, with high probability,
there exists some transmitter that hears $\Omega(\sqrt{\log n})$
transmitters and all transmitters hear $O(\sqrt{\log n})$ transmitters.
\end{abstract}

\section{Introduction}

Related Work.

\section{The Lower Bound}

In this section we show that, with high probability, in a set of $n$
elements uniformly distributed on an interval, there exists an element
that has interference $\Omega(\sqrt{\log n})$.  We do this by defining
a configuration of points that leads to an element with interference
$c\sqrt{\log n}$ and then showing that, with high probability, this
configuration occurs somewhere in our point set.

A sequence of numbers $X_0,\ldots,X_k$ forms a \emph{$k$-frame} if
\[
     X_0 \le 1
\]
and
\[
     X_{i-1}/4 \le X_i \le X_{i-1}/2 \enspace ,
\]
for all $i\in\{1,\ldots,k\}$.  Later, we will show that a $k$-frame
implies a node that has interference $k$.  For now, we start by showing
that frames are not that unlikely:

\begin{lem}
If $X_0,\ldots,X_k$ are a sequence of independent Exponential(1) random
variables, then the probability that $X_0,\ldots,X_k$ form a $k$-frame
is at least $2^{-O(k^2)}$.
\end{lem}

\begin{proof}
Recall that an Exponential(1) random variable $X$ has cumulative
distribution function
\[
   \Pr\{X \le x\} = 1-e^{-x} \enspace .
\]

Next, observe that, in a frame,
\[
                 4^{-i} \le X_i \le 2^{-i}  \enspace ,
\]
for all $i\in\{0,\ldots,k\}$.  Let $F(X)$ be the event ``$X$ is a frame.''
Then,
\begin{align*}
     \Pr\{F(X_0,\ldots,X_{i+1}) \mid F(X_0,\ldots,X_{i})\} 
        & = \Pr\{X_{i+1} \in [X_{i}/4,X_{i}/2] \mid F(X_0,\ldots,X_{i})\} \\
        & \ge \Pr\{X_{i+1} \in [4^{-i}/4,4^{-i}/2]\} \\
        & = \Pr\{X_{i+1} \in [2^{-(2i+2)},2^{-(2i+1)}]\} \\
        & = \exp(-2^{-(2i+2)}) - \exp(2^{-(2i+1)}) \\
        & \sim -2^{-(2i+2)} + 2^{-(2i+1)} \\
        & = 2^{-(2i+2)} \enspace .
\end{align*}
Therefore,
\begin{align*}
     \Pr\{F(X_0,\ldots,X_{k})\}
   & = \Pr\{X_0\le 1\}
         \cdot\prod_{i=1}^k \Pr\{F(X_0,\ldots,X_{i})
                                 \mid F(X_0,\ldots,X_{i-1})\} \\
   & = (1-1/e)
         \cdot\prod_{i=1}^k \Pr\{F(X_0,\ldots,X_{i})
                                 \mid F(X_0,\ldots,X_{i-1})\} \\
   & \ge (1-1/e)\cdot\prod_{i=1}^k 2^{-2i} \\
   & = (1-1/e)\cdot2^{-\sum_{i=1}^k 2i} \\
   & = (1-1/e)\cdot2^{-k(k+1)} \\
   & \ge 2^{-(k+1)^2} \\
\end{align*}
as required.
\end{proof}

%\begin{thm}
%Let $S$ be a set of $n$ real numbers each selected independently and
%uniformly from the real interval $[0,1]$.  Then, the expected number
%of elements of $S$ that have interference at least $k$ is at least
%$n2^{-O(k^2)}$.
%\end{thm}
%
%\begin{proof}
%Let $X_0,\ldots,X_n$ be a sequence of Exponential(1) random variables,
%let $x_i=\sum_{j=0}^{i-1} X_i$, and let $\tilde x_i = x_i/x_{n+1}$.
%It is well-known that the sequence $\tilde x_1,\ldots,\tilde x_n$ has
%the same distribution that one would obtain by sorting the points of $S$.
%Thus, it suffices to study the set $S=\{\tilde x_1,\ldots,\tilde x_n\}$.
%
%Observe that, for any $i > k$, if $X_{i-k-1},\ldots,X_{i-1}$ is a
%$k$-frame, then $\tilde x_i$ has interference at least $k$.  Therefore
%\[
%  \Pr\{\mbox{$\tilde x_i$ has interference at least $k$}\} \ge 2^{-O(k^2)}
%    \enspace ,
%\]
%for all $i\in\{1,\ldots,n\}$.\footnote{For $i <k$, we can reverse
%the argument.} Therefore, the expected number of elements that have
%interference at least $k$ is $n2^{-O(k^2)}$, as required.
%\end{proof}

This doesn't quite prove that elements with interference at least
$\sqrt{\alpha\log n}$ have a high probability of occuring.  For that, we
can consider blocks of $k=\sqrt{\alpha\log n}$ consecutive elements.  The
last element in each block has probability $n^{-O(\alpha)}$ of being the
last element in a $k$-frame, and this is independent of whether any other
block ends with the last element in a $k$-frame.  Using this, we obtain:

\begin{cor}
Let $S$ be a set of $n$ real numbers each selected independently and
uniformly from the real interval $[0,1]$.  Then, for any $\alpha > 0$,
the probability that no element of $S$ has interference at
least $\sqrt{\alpha\log n}$ is at most $e^{-n^{1-O(\alpha)}}$.
\end{cor}

\section{The Upper Bound}

In this section we show that, with high probability, the maximum
interference among $n$ points uniformly distributed in an interval is
$O(\sqrt{\log n})$.

\begin{lem}\lemlabel{no-big-gap}
With probability at least $1-n^{-\Omega(c)}$, there is no open interval
$I\subseteq[0,1]$ of length greater $c\log n / n$ that is empty of points
in $S$.
\end{lem}

\begin{proof}
Fix a particular interval of length $c\log n /n$.  The probability that
this interval is empty of point is at most
\[
   p = (1-c\log n/n))^{n-1} \le n\exp(-(n-1)c\log n/n) = n^{-\Omega(c)} \enspace .
\]
Any such maximal empty interval either contains $1$, or is bounded from
above by some $x_i\in[c\log n /n,1]$.  Therefore, the probability that
there exists \emph{any} such interval is most $(n+1)p= n^{-\Omega(c)}$,
as required.
\end{proof}

\begin{proof}
We define the \emph{left-interference} of an element $x_t\in S$ as
the number of elements $x_i\in S$ such that $x_i < x_t$ and $x_t-x_i
\le \max\{x_i-x_{i-1},x_{i+1}-x_i\}$.  We define the \emph{short-range
left-interference} of $x_t$ in the same way, except only counting those
elements $x_i$ such that $X_{i-1} \le 1$. (Note that this implies $x_t-x_i
\le 2$.)

Shortly, we will show that the probability that the short-range
left-interference of any $x_t\in S$ exceeds $c\sqrt{\log n}\le
n^{-\Omega(c)}$.  This will be sufficient, for two reasons: (1)~a
symmetric argument bounds the probability that the right-interference
exceeds $c\sqrt{\log n}$ and (2)~\lemref{no-big-gap} implies that, for any
$x_t\in S$ the number of elements $x_i$ that contribute to the left
interference of $x_t$ and have $X_{i-1}>1$ is at most $c\log \log n$ with
probability at least $n^{-\Omega(c)}$.

Consider the following process: Start with a \emph{length} $\ell_0 \le 1$
and Generate i.i.d.\ Exponential(1) random variables $X_1,\ldots,X_i$
until $\sum_{j=1}^i {X_j} \ge \ell_0/2$. If $\sum_{j=1}^i {X_j} \ge
\ell_0$ then the process stops.  Otherwise, the process begins another
round with a new input length $\ell_1=\ell_0-\sum_{j=1}^i {X_j}$, and
the process continues in this manner until the stopping condition is met.

Observe that, in this process, the length $\ell_i$ during the $i$th
round (starting at round 0) satisfies $\ell_i\le (1/2)^i$.  Therefore,
the probability of continuing onto round $i+1$ from round $i$ is at most
\[
   \Pr\{X_1 \le 1/2^i\} = 1-e^{-1/2^i} \le 1/2^{i} \enspace .
\]
Therefore, the probability that this process continues beyond round $k$
is at most
\[
    \prod_{i=0}^{k} (1/2)^i = (1/2)^{\sum_{i=0}^{k} i} \le (1/2)^{k^2}
\enspace .
\]
The relationship between the above process and interference is the
following:  Suppose that some element $x_t\in S$ has interference $k$
and $x_i$ contributes to the left-interference of $x_t$ and $X_{i-1}\le
1$.  Then, we let $\ell_0=X_{i-1}$ and observe that the elements
$x_{i_1}=x_i,\ldots,x_{i_k}$ that contribute the left-interference of
$x_t$ correspond to elements that terminate a round in the above process,
with the next-to-last round being terminated by $x_t$ itself.

By setting $k=\sqrt{c\log n}$, we find that the probability that $X_{1}$
starts a process leading to an $x_t$ whose short-range left-interference is
greater than $k$ is at most $n^{-c}$.  Therefore, the probability that
there exists \emph{any} $X_i$ that starts a process leading an $x_t$ with
short-range left-interference greater than $k$ is at most
$n^{-c+1}=n^{-\Omega(c)}$, completing the proof.
\end{proof}



\end{document}

