\documentclass[lotsofwhite,12pt]{patmorin}
\usepackage{amsopn}
\usepackage{algorithmic}
 
%\usepackage{amsthm}

\newcommand{\centeripe}[1]{\begin{center}\Ipe{#1}\end{center}}
\newcommand{\comment}[1]{}

\newcommand{\centerpsfig}[1]{\centerline{\psfig{#1}}}

\newcommand{\seclabel}[1]{\label{sec:#1}}
\newcommand{\Secref}[1]{Section~\ref{sec:#1}}
\newcommand{\secref}[1]{\mbox{Section~\ref{sec:#1}}}

\newcommand{\alglabel}[1]{\label{alg:#1}}
\newcommand{\Algref}[1]{Algorithm~\ref{alg:#1}}
\newcommand{\algref}[1]{\mbox{Algorithm~\ref{alg:#1}}}

\newcommand{\applabel}[1]{\label{app:#1}}
\newcommand{\Appref}[1]{Appendix~\ref{app:#1}}
\newcommand{\appref}[1]{\mbox{Appendix~\ref{app:#1}}}

\newcommand{\tablabel}[1]{\label{tab:#1}}
\newcommand{\Tabref}[1]{Table~\ref{tab:#1}}
\newcommand{\tabref}[1]{Table~\ref{tab:#1}}

\newcommand{\figlabel}[1]{\label{fig:#1}}
\newcommand{\Figref}[1]{Figure~\ref{fig:#1}}
\newcommand{\figref}[1]{\mbox{Figure~\ref{fig:#1}}}

\newcommand{\eqlabel}[1]{\label{eq:#1}}
\newcommand{\eqref}[1]{(\ref{eq:#1})}

\newtheorem{thm}{Theorem}{\bfseries}{\itshape}
\newcommand{\thmlabel}[1]{\label{thm:#1}}
\newcommand{\thmref}[1]{Theorem~\ref{thm:#1}}

\newtheorem{lem}{Lemma}{\bfseries}{\itshape}
\newcommand{\lemlabel}[1]{\label{lem:#1}}
\newcommand{\lemref}[1]{Lemma~\ref{lem:#1}}

\newtheorem{cor}{Corollary}{\bfseries}{\itshape}
\newcommand{\corlabel}[1]{\label{cor:#1}}
\newcommand{\corref}[1]{Corollary~\ref{cor:#1}}

\newtheorem{obs}{Observation}{\bfseries}{\itshape}
\newcommand{\obslabel}[1]{\label{obs:#1}}
\newcommand{\obsref}[1]{Observation~\ref{obs:#1}}

\newtheorem{assumption}{Assumption}{\bfseries}{\rm}
\newenvironment{ass}{\begin{assumption}\rm}{\end{assumption}}
\newcommand{\asslabel}[1]{\label{ass:#1}}
\newcommand{\assref}[1]{Assumption~\ref{ass:#1}}

\newcommand{\proclabel}[1]{\label{alg:#1}}
\newcommand{\procref}[1]{Procedure~\ref{alg:#1}}

\newtheorem{rem}{Remark}
\newtheorem{op}{Open Problem}

\newcommand{\etal}{\emph{et al}}

\newcommand{\voronoi}{Vorono\u\i}
\newcommand{\ceil}[1]{\left\lceil #1 \right\rceil}
\newcommand{\floor}[1]{\left\lfloor #1 \right\rfloor}




\DeclareMathOperator{\td}{depth}
\DeclareMathOperator{\ch}{conv}
\newlength{\algwidth}
\setlength{\algwidth}{\textwidth}
\addtolength{\algwidth}{-5mm}

\title{\MakeUppercase{Algorithm Ideas for Tukey Depth}}
\author{Prosenjit Bose \and Dan Chen \and John Chinneck \and Pat Morin}

\begin{document}
\maketitle

\section{Introduction}
\seclabel{intro}

Let $S$ be a set of $n$ points in $\R^d$.
The \emph{Tukey depth}, or \emph{halfspace depth} of a point $p\in\R^d$ with
respect to $S$ can be defined in several equivalent ways:
\begin{eqnarray}
\td(p,S) & = & \min\{ |h\cap S| :
                     \mbox{$h$ is a closed halfspace containing $p$} \} 
                       \eqlabel{tuk-orig} \\ 
            & = & \min\{ |h\cap S| :
                      \mbox{$h$ is a closed halfspace 
                            with $p$ on its boundary} \} 
                        \eqlabel{tuk-boundary} \\ 
            & = & \min\{ |S'| :
                      \mbox{$p$ is outside the convex hull of 
                           $S\setminus S'$} \}
                      \eqlabel{tuk-hull}
\end{eqnarray}

We use the notation $a\cdot b$ to denote the orhogonal projection of
$a$ onto $b$. When $a$ and $b$ are vectors, this is the usual
dot-product, but in general, $a$ is a set of points and $b$ is a
linear subspace of $\R^d$.


\section{Exact Algorithms}
\seclabel{exact}

If $d=1$ then computing $\td(p,S)$ can be done in $O(n)$ time by
counting the number of points greater than $p$ and the number of
points less than $p$ and taking the minimum of these two values.  If
$d=2$ then $\td(p,S)$ can be computed in $O(n\log n)$ time by
a \emph{sort-and-scan} algorithm that sorts the points of $S$ radially
about $p$ and sweeps a line around $p$.

\subsection{Enumeration of Halfspaces}

Otherwise, the halfspace $h$ that minimizes \eqref{tuk-boundary} is
arbitrarily close to a halfspace $h^*$ containing $p$ and $d-1$ points
of $S$ on its boundary.  We can find $h^*$ by enumerating all $n
\choose d-1$ points of $S$ and testing each one in $O(n)$ time. This
gives an $O(n^d)$ time algorithm to find $h^*$.  With a non-degeneracy
assumption,\footnote{In particular, we assume that no $d+1$ points of
$S\cup\{p\}$ lie on a common hyperplane.} this allows us to compute
$\td(p,S)$.

A recursive implementation of the above algorithm (with running time
$O(n^{d-1}\log n)$) might look like:

\noindent\framebox{
\begin{minipage}{\algwidth}
\noindent{\textsc{Tukey$(p,S,d)$}}
\begin{algorithmic}
\IF{$d=2$}
  \STATE{\textbf{return} $\td(p,S)$ 
     \COMMENT{use $O(n\log n)$ ``sort-and-scan'' algorithm}}
\ENDIF
\STATE{$m\gets \infty$}
\FOR{$x\in S$}
  \STATE{let $\pi$ be the hyperplane orthogonal to $\vec{px}$}
  \STATE{$m\gets\min\{m, \textsc{Tukey}(p,S\cdot \pi,d)\}$}
\ENDFOR
\STATE{\textbf{return} $m$}
\end{algorithmic}
\end{minipage}}

\subsection{Branch-and-Bound}

[Dan's approach.]

\subsection{Reduction to $d$-Hitting Set}

The approach described in 
[D. Bremner, D. Chen, J. Iacono, S. Langerman, and P. Morin.
Output-sensitive algorithms for Tukey depth and related problems.
Submitted to Statistics and Computing, November 2006.]

Can a partition of space into cones that have ``Carath\'eodory
witnesses'' speed up this algorithm?

\section{Upper Bounds}
\seclabel{ub}

An upper bound on $\td(p,S)$ can be obtained by selecting any
non-trivial vector $v\in\R^d$ and computing the Tukey depth of $p\cdot
v$ in
the one dimensional point set 
\[
      S\cdot v = \{x\cdot v : x\in S\} \enspace .
\]
Notice that if we choose $v$ as the inner-normal of the halfspace $h$
that minimizes \eqref{tuk-boundary} then 
\begin{equation}
      \td(p,S) = \td(p\cdot v,S\cdot v) \enspace . \eqlabel{ub-1d}
\end{equation}

More generally, given any $i$-flat\footnote{An $i$-flat in $\R^d$ is
an $i$-dimensional linear subspace of $\R^d$. A 0-flat is a point, a
1-flat is a line, a 2-flat is a plane\ldots} $f$, we have
\begin{equation}
      \td(p,S) \le \td(p\cdot f, S\cdot f) \eqlabel{ub-id}
\end{equation}
with equality if $f$ is orthogonal to the boundary of the halfspace $h$
that minimizes \eqref{tuk-boundary}.

These observations lead to several different heuristics for bounding
$\td(p,S)$ from above:

\begin{enumerate}
\item Let $v$ be parallel to each of the coordinate axes in turn and
apply \eqref{ub-1d} to obtain an upper bound on $\td(p,S)$.

\item Repeatedly select random elements from $S$ until obtaining a set
$T$ such that there is a unique hyperplane $\pi$ containing
$T\cup\{p\}$.\footnote{With the non-degeneracy assumption this occurs
as soon as we have selected $d-1$ elements of $S$.} Let $v$ be
orthogonal to $\pi$ and apply \eqref{ub-1d} to get an upper bound on
$\td(p,S)$.

\item More generally, for some fixed $i$, repeatedly select random
elements from $S$ until obtaining a set $T$ such that there is a
unique $(d-i)$-flat $\pi$  that contains $T\cup\{p\}$.\footnote{With
the non-degeneracy assumption this occurs as soon as we have selected
$d-i$ elements of $S$.}  Let $f$ be an
$i$-flat orthogonal to $\pi$ and apply \eqref{ub-id} to get an upper
bound on $\td(p,S)$.

\end{enumerate}


Computationally, the third option with $i=2$ seems the most attractive
since the expected number of samples required to find the halfspace
$h^*$ discussed in \secref{exact} is ${n\choose d-2}/2$ and each
sample leaves us with a 2-dimensional problem that can be solved in
$O(n\log n)$ time.  Indeed, this is a randomized version of the
recursive algorithm described in \secref{exact}.

\section{Lower Bounds}
\seclabel{lb}

Formulation \eqref{tuk-hull} gives a method for obtaining lower bounds
on Tukey depth by linear programming.  For a set $S$ of $n$ points,
the question ``Is $p$ contained in the convex hull of $S$?'' can be
expressed as 2 linear programs $P^+$ and $P^-$, each with $d$
variables and $n$ constraints, and whose constraints form a bijection
with the points in $S$.  

Both linear programs are feasible if and only if $p$ is outside the
convex hull of $S$. Otherwise, one of the linear programs, say $P^+$
is infeasible, so it contains a \emph{basic infeasible subsystem} of
at most $d+1$ constraints that, by themselves, define an infeasible
linear program.  These constraints correspond to a set $B$ of at most
$d+1$ points in $S$ whose convex hull contains the point $p$.  We
obtain a lower bound on $\td(p,S)$ by observing that the set $S'$
minimizing \eqref{tuk-hull} must contain at least one element of $B$.
This leads to the following algorithm:

\noindent\framebox{
\begin{minipage}{\algwidth}
\noindent{\textsc{TukeyLowerBound$(p,S)$}}
\begin{algorithmic}
\STATE{$\ell\gets 0$}
\WHILE{$p\in \ch(S)$}
  \STATE{Let $B\subset S$, $|B|\le d+1$, such that $p\in\ch(B)$}
  \STATE{$S\gets S\setminus B$}
  \STATE{$\ell \gets \ell+1$}  
\ENDWHILE
\STATE{\textbf{return} $\ell$}
\end{algorithmic}
\end{minipage}}




\end{document}
