%\setcounter{\chapter}{10}
\chapter{Hash Tables}

\newcommand{\E}{\mathrm{E}}
\newcommand{\floor}[1]{\lfloor #1 \rfloor}
\newcommand{\ceil}[1]{\lceil #1 \rceil}

\newcommand{\etal}{et al.}

\begin{chapterauthors}
\chapterauthor{Pat Morin}{Carleton University}
\end{chapterauthors}

\index{hash table}

\section{Introduction}

A \emph{set abstract data type (set ADT)}\index{set ADT} is an
abstract data type that maintains a set $S$ under the following three
operations:

\begin{enumerate}

\item \textsc{Insert}$(x)$:  Add the key $x$ to the set.

\item \textsc{Delete}$(x)$: Remove the key $x$ from the set.

\item \textsc{Search}$(x)$: Determine if $x$ is contained in the set,
and if so, return a pointer to $x$. 

\end{enumerate}
One of the most practical and widely used methods of implementing the
set ADT is with \emph{hash tables}.  

Note that the three set ADT operations can easily be implemented to
run in $O(\log n)$ time per operation using balanced binary search
trees (See Chapter~\ref{external:12:BBST}).  If we assume that the
input data are integers in the set $U=\{0,\ldots,u-1\}$ then they can
even be implemented to run in $O(\log\log u)$ time using data
structures for integer searching
(Chapter~\ref{external:38?:int-searching}).  However, these data
structures actually do more than the three basic operations we
require.  In particular if we search for an element $x$ that is not
present in $S$ then these data structures can report the smallest item
in $S$ that is larger than $x$ (the \emph{successor} of $x$) and/or
the largest item in $S$ that is smaller than $x$ (the
\emph{predecessor} of $x$).  

Hash tables do away with this extra functionality of finding
predecessors and successors and only perform exact searches. If we
search for an element $x$ in a hash table and $x$ is not present then
the only information we obtain is that $x\notin S$. By dropping this
extra functionality hash tables can give better performance bounds.
Indeed, any reasonable hash table implementation performs each of the
three set ADT operations in $O(1)$ expected time.

The main idea behind all hash table implementations discussed in this
chapter is to store a set of $n=|S|$ elements in an array (the hash
table) $A$ of length $m\ge n$.  In doing this, we require a function
that maps any element $x$ to an array location.  This function is
called a \emph{hash function}\index{hash function} $h$ and the value
$h(x)$ is called the \emph{hash value} of $x$.  That is, the element
$x$ gets stored at the array location $A[h(x)]$.  The \emph{occupancy}
of a hash table is the ratio $\alpha=n/m$ of stored elements to the
length of $A$.  

The study of hash tables follows two very different lines. Many
implementations of hash tables are based on the \emph{integer universe
assumption}:\index{integer universe assumption} All elements stored in
the hash table come from the universe $U=\{0,\ldots,u-1\}$.  In this
case, the goal is to design a hash function $h:U\rightarrow
\{0,\ldots,m-1\}$ so that for each $i\in \{0,\ldots,m-1\}$, the number
of elements $x\in S$ such that $h(x)=i$ is as small as possible.
Ideally, the hash function $h$ would be such that each element of $S$
is mapped to a unique value in $\{0,\ldots,m-1\}$.  Most of the hash
functions designed under the integer universe assumption are
number-theoretic constructions. Several of these are described in
Section~\ref{hash:sec:integers}.

Historically, the integer universe assumption seems to have been
justified by the fact that any data item in a computer is represented
as a sequence of bits that can be interpreted as a binary number.
However, many complicated data items require a large (or variable)
number of bits to represent and this make $u$ the size of the universe
very large.  In many applications $u$ is much larger than the largest
integer that can fit into a single word of computer memory.  In this
case, the computations performed in number-theoretic hash functions
become inefficient.

This motivates the second major line of research into hash tables.
This research works is based on the \emph{random probing
assumption\emph{random probing assumption}}: Each element $x$ that is
inserted into a hash table is a black box that comes with an infinite
random \emph{probe sequence}\index{probe sequence}
$x_0,x_1,x_2,\ldots$ where each of the $x_i$ is independently and
uniformly distributed in $\{0,\ldots,m-1\}$.  Hash table
implementations based on the random probing assumption are described
in Section~\ref{hash:sec:uniform}.

Both the integer universe assumption and the random probing
assumption have their place in practice.  When there is an easily
computing mapping of data elements onto machine word sized integers
then hash tables for integer universes are the method of choice.  When
such a mapping is not so easy to compute (variable length strings are
an example) it might be better to use the bits of the input items to
build a good pseudorandom sequence and use this sequence as the probe
sequence for some random probing data structure. 

To guarantee good performance, many hash table implementations require
that the occupancy $\alpha$ be a constant strictly less than $1$.
Since the number of elements in a hash table changes over time, this
requires that the array $A$ be resized periodically.  This is easily
done, without increasing the amortized cost of hash table operations
by choosing three constants $0< \alpha_1 < \alpha_2 < \alpha_3 < 1$ so
that, whenever $n/m$ is not the interval $(\alpha_1,\alpha_3)$ the
array $A$ is resized so that its size is $n/\alpha_2$. A simple
amortization argument (Chapter~\ref{external:1:analysis}) shows that
the amortized cost of this resizing is $O(1)$ per update
(Insert/Delete) operation.


\section{Hash Tables for Integer Keys}
\label{hash:sec:integers}

In this section we consider hash tables under the integer universe
assumption, in which the key values $x$ come from the universe
$U=\{0,\ldots,u-1\}$.   A \emph{hash function}\index{hash function}
$h$ is a function whose domain is $U$ and whose range is the set
$\{0,\ldots,m-1\}$, $m\le u$.  A hash function $h$ is said to be a
\emph{perfect hash function}\index{hash
function!perfect}\index{perfect hash function} for a set $S\subseteq
U$ if, for every $x\in S$, $h(x)$ is unique.  A perfect hash function
$h$ for $S$ is \emph{minimal}\index{hash function!minimal
perfect}\index{minimal perfect hash function} if $m=|S|$, i.e., $h$ is
a bijection between $S$ and $\{0,\ldots,m-1\}$.  Obviously a minimal
perfect hash function for $S$ is desirable since it allows us to store
all the elements of $S$ in a single array of length $n$.
Unfortunately, perfect hash functions are rare, even for $m$ much
larger than $n$.  If each element of $S$ is mapped independently and
uniformly to a random element of $\{0,\ldots,m-1\}$ then the birthday
paradox (See, for example, Feller~\cite{f68}) states that, if $m$ is
much less than $n^2$ then there will almost surely exist two elements
of $S$ that have the same hash value.

We begin our discussion with two commonly used hashing schemes that
are heuristic in nature.  That is, we can not make any non-trivial
statements about the performance of these schemes when storing an
arbitrary set $S$.  We then discuss several schemes that have provably
good performance.

\subsection{Hashing by Division}
\index{hashing!by division}

In \emph{hashing by division}, we use the hash function
\[ h(x) = x\bmod m \enspace . \]
To use this hash function in a data structure, we maintain an array
$A[0],\ldots,A[m-1]$ where each element of this array is a pointer to
the head of a linked list (Section~\ref{external:2:linked-list}).  The
linked list $L_i$ pointed to by the array element $A[i]$ contains all
the elements $x$ such that $h(x)=i$.  This technique of maintaining an
array of lists is called \emph{hashing with
chaining}\index{hashing!with chaining}.

In such a hash table, inserting an element $x$ takes $O(1)$ time;  we
compute $i=h(x)$ and append (or prepend) $x$ to the list $L_i$.
However, searching for and/or deleting an element $x$ is not so easy.
We have to compute $i=h(x)$ and then traverse the list $L_i$ until we
either find $x$ or reach the end of the list.  The cost of this is
proportional to the length of $L_i$.  Obviously, if our set $S$
consists of the elements $0,m,2m,3m,\ldots,nm$ then all elements are
stored in the list $L_0$ and searches and deletions take linear time.

However, one hopes that such pathological cases do not occur in
practice. For example, if the elements of $S$ are uniformly and
independently distributed in $U$ and $u$ is a multiple of $m$ then the
expected size of any list $L_i$ is only $n/m$.  In this case, searches
and deletions take $O(1+\alpha)$ expected time.  To help avoid
pathological cases, the choice of $m$ is important.  In particular,
$m$ a power of 2 is usually avoided since, in a binary computer,
taking the remainder modulo a power of 2 means simply discarding some
high-order bits.  Taking $m$ to be a prime not too close to a power of
2 is recommended \cite{k97}.

\subsection{Hashing by Multiplication}
\index{hashing!by multiplication}

The implementation of a hash table using \emph{hashing by
multiplication} is exactly the same as that hashing by division
except that the hash function 

\[
  h(x) = \floor{mxA} \bmod m
\]
is used.  Here $A$ is a real-valued constant whose choice we discuss
below. The advantage of the multiplication method is that the value of
$m$ is not critical.  We can take $m$ to be a power of 2, which makes
it convenient for use on binary computers. 

Although any value of $A$ gives a hash function, some values of $A$
are better than others.  (Setting $A=0$ is clearly not a good idea.)

Knuth \cite{k97} suggests using the \emph{golden ratio}\index{golden
ratio} for $A$, i.e.,
setting 
\[  
    A=(\sqrt{5}-1)/2=0.6180339887\ldots
\]   
This choice of $A$ is motivated by a theorem, first conjectured by
Oderfeld and later proven by S\'wierczkowski \cite{s58}.  This theorem
states that the sequence 
\[
     mA \bmod m,\,\, 2mA \bmod m,\,\, 3mA \bmod m,\,\,\ldots,\,\, nmA\bmod m
\] 
partitions the interval $(0,m)$ into $n+1$ intervals having only three
distinct lengths.  Furthermore, the next element $(n+1)mA\bmod m$ in
the sequence is always contained in one of the largest
intervals.\footnote{In fact, any irrational number has this property
\cite{sos57}.  The golden ratio is especially good because it is
not too close to a whole number.}

Of course, no matter what value of $A$ we select, the pigeonhole
principle implies that for $u\ge nm$ then there will always exist some
hash value $i$ and some $S\subseteq U$ of size $n$ such that $h(x)=i$
for all $x\in S$.  In other words, we can always find a set $S$ all of
whose elements get stored in the same list $L_i$. Thus, the worst case
of hashing by multiplication is as bad as hashing by division.


\subsection{Universal Hashing}
\label{hash:sec:universal}
\index{hashing!universal}
\index{universal hashing}

The argument used at the end of the previous section applies equally
well to any hash function $h$.  That is, if the table size $m$ is much
smaller than the universe size $u$ then for any hash function there is
some large (of size at least $\ceil{u/m}$) subset of $U$ that has the
same hash value.  To get around this difficulty we need a collection
of hash functions of which we can choose one that works well for $S$.
Even better would be a collection of hash functions such that, for any
given $S$, most of the hash functions work well for $S$.  Then we
could simply pick one of the functions at random and have a good
chance of it working well.

Let $\mathcal{H}$ be a collection of hash functions, i.e., functions
from $U$ onto $\{0,\ldots,m-1\}$.  We say that $\mathcal{H}$ is
\emph{universal}\index{hash function!universal} if, for each $x,y\in
U$ the number of $h\in \mathcal{H}$ such that $h(x)=h(y)$ is at most
$|\mathcal{H}|/m$.  Consider any $S\subseteq U$ of size $n$ and
suppose we choose a random hash function $h$ from a universal
collection of hash functions.  Consider some value $x\in U$.  The
probability that any key $y\in S$ has the same hash value as $x$ is
only $1/m$.  Therefore, the expected number of keys in $S$, not equal
to $x$, that have the same hash value as $x$ is only \[ n_{h(x)} =
\left\{\begin{array}{ll} (n-1)/m & \mbox{if $x\in S$} \\ n/m &
\mbox{if $x\notin S$} \end{array}\right.  \] Therefore, if we store
$S$ in a hash table using the hash function $h$ then the expected time
to search for, or delete, $x$ is $O(1+\alpha)$.

From the preceding discussion, it seems that a universal collection of
hash functions from which we could quickly select one at random would be
very handy indeed. With such a collection at our disposal we get an
implementation of the set ADT that has $O(1)$ insertion time and
$O(1)$ expected search and deletion time.

Carter and Wegman \cite{cw79} describe three different collections of
universal hash functions.  If the universe size $u$ is a
prime number\footnote{This is not a major restriction since, for any
$u>1$, there always exists a prime number in the set
$\{u,u+1,\ldots,2u\}$.  Thus we can enforce this assumption by
increasing the value of $u$ by a constant factor.} then
\[
  \mathcal{H} = \{ h_{k_1,k_2,m}(x)=((k_1x+k_2)\bmod u))\bmod m : 1\le k_1< u,
0\le k_2< u\} 
\]
is a collection of universal hash functions.  Clearly, choosing a
function uniformly at random from $\mathcal{H}$ can be done easily by
choosing two random values $k_1\in \{1,\ldots,u-1\}$ and $k_2\in
\{0,\ldots,u-1\}$.  Thus, we have an implementation of the set ADT with
$O(1)$ expected time per operation.

 
\subsection{Static Perfect Hashing}
\index{perfect hashing!static}
\index{hashing!static perfect}
\index{static perfect hashing}
\label{hash:sec:fks}

The result of Carter and Wegman on universal hashing is very strong,
and from a practical point of view, it is probably the strongest
result most people will ever need.  The only thing that could be
improved about their result is to make it deterministic, so that the
running times of all operations are $O(1)$ \emph{worst-case}.
Unfortunately, this is not possible, as shown by Dietzfelbinger \etal\
\cite{dkmm94}.  

Since there is no hope of getting $O(1)$ worst-case time for all three
set ADT operations, the next best thing would be to have searches that
take $O(1)$ worst-case time.  In this section we describe the method
of Fredman, Koml\'os and Szemer\'edi \cite{fks84}.  This is a static
data structure that takes as input a set $S\subseteq U$ and builds a
data structure of size $O(n)$ that can test if an element $x$ is in
$S$ in $O(1)$ worst-case time.  Like the universal hash functions from
the previous section, this method also requires that $u$ be a prime
number. This scheme uses hash functions of the form
\[
  h_{k,m}(x) = (kx\bmod u))\bmod m \enspace .\footnote{Actually, it
turns out that any universal hash function also works in the FKS
scheme \cite[Section~11.5]{clrs01}.}
\]

Let $B_{k,m}(S,i)$ be the number of elements $x\in S$ such that
$h_{k,m}(x)=i$, i.e., the number of elements of $S$ that have hash
value $i$ when using the hash function $h_{k,m}$.  The function
$B_{k,m}$ gives complete information about the distribution of hash
values of $S$.  The main lemma used by Fredman \etal\ is that, if we
choose $k\in U$ uniformly at random then
\begin{equation}
\E\left[\sum_{i=0}^{m-1} {B_{k,m}(S,i)\choose 2}\right] < \frac{n^2}{m}  
  \enspace . \label{hash:eq:fks-main}
\end{equation}
There are two important special cases of this result.

In the \emph{sparse case} we take $m=n^2/\alpha$, for some constant
$0<\alpha<1$.  In this case, the expectation in
(\ref{hash:eq:fks-main}) is less than $\alpha$.  Therefore, by
Markov's inequality, the probability that this sum is greater than or
equal to 1 is at most $\alpha$.  But, since this sum is a non-negative
integer, then with probability at least $1-\alpha$ it must be equal to
0.  In other words, with probability at least $1-\alpha$,
$B_{k,m}(S,i)\le 1$ for all $0\le i\le m-1$, i.e., the hash function
$h_{k,m}$ is perfect for $S$.  Of course this implies that we can find
a perfect hash function very quickly by trying a small number of
random elements $k\in U$ and testing if they result in perfect hash
functions. (The expected number of elements that we will have to try
is only $1/(1-\alpha)$.) Thus, if we are willing to use quadratic
space then we can perform searches in $O(1)$ worst-case time.

In the \emph{dense case} we assume that $m$ is close to $n$ and
discover that, for many values of $k$, the hash values are distributed
fairly evenly among the set $1,\ldots,m$.  More precisely, if we use a
table of size $m=n$, then 
\[
   \E \left[ \sum_{i=0}^{m-1} B_{k,m}(S,i)^2 \right] \le 3n \enspace .
      \label{hash:eq:fks-space-a}
\]
By Markov's inequality this means that
\begin{equation}
   \Pr\left\{ \sum_{i=0}^{m-1} B_{k,m}(S,i)^2  \le 3n/\alpha \right\}
\ge
	1-\alpha \enspace .
      \label{hash:eq:fks-space}
\end{equation}
Again, we can quickly find a value of $k$ satisfying
(\ref{hash:eq:fks-space}) by testing a few randomly chosen values of
$k$.

These two properties are enough to build a two-level data structure
that uses linear space and executes searches in worst-case constant
time.  We call the following data structure the FKS-$\alpha$ data
structure, after its inventors Fredman, Koml\'os and Szemer\'edi.  At
the top level, the data structure consists of an array
$A[0],\ldots,A[m-1]$ where $m=n$.  The elements of this array are
pointers to other arrays $A_0,\ldots,A_{m-1}$, respectively.  To
decide what will be stored in these other arrays, we build a hash
function $h_{k,m}$ that satisfies the conditions of
(\ref{hash:eq:fks-space}).  This gives us the top-level hash function
$h_{k,m}(x)=(kx\bmod u)\bmod m$. Each element $x\in S$ gets stored in
the array pointed to by $A[h_{k,m}(x)]$.

What remains is to describe how we use the arrays
$A_0,\ldots,A_{m-1}$.  Let $S_i$ denote the set of elements $x\in S$
such that $h_{k,m}(s)=i$.  The elements of $S_i$ will be stored in
$A_i$.  The size of $S_i$ is $n_i=B_{k,m}(S,i)$.  To store the
elements of $S_i$ we set the size of $A_i$ to
$m_i=n_i{}^2/\alpha=B_{k,n}(S,i)^2/\alpha$.  Observe that, by
(\ref{hash:eq:fks-space}), all the $A_i$'s take up a total space of
$O(n)$, i.e., $\sum_{i=0}^{m-1} m_i = O(n)$.  Furthermore, by trying a
few randomly selected integers we can quickly find a value $k_i$ such
that the hash function $h_{k_i,m_i}$ is perfect for $S_i$.  Therefore,
we store the element $x\in S_i$ at position $A_i[h_{k_i,m_i}(x)]$ and
$x$ is the unique element stored at that location.  With this scheme
we can search for any value $x\in U$ by computing two hash values
$i=h_{k,m}(x)$ and $j=h_{k_i,m_i}(x)$ and checking if $x$ is stored in
$A_i[j]$.

Building the array $A$ and computing the values of
$n_0,\ldots,n_{m-1}$ takes $O(n)$ expected time since for a given
value $k$ we can easily do this in $O(n)$ time and the expected number
of values of $k$ that we must try before finding one that satisfies
(\ref{hash:eq:fks-space}) is $O(1)$.  Similarly, building each subarray
$A_i$ takes $O(n_i{}^2)$ expected time, resulting in an overall expected
running time of $O(n)$.  Thus, for any constant $0<\alpha<1$, an
FKS-$\alpha$ data structure can be constructed in $O(n)$ expected time
and this data structure can execute a search for any $x\in U$ in
$O(1)$ worst-case time.

\subsection{Dynamic Perfect Hashing}
\label{hash:sec:dphash}
\index{hashing!dynamic perfect}
\index{dynamic perfect hashing}
\index{perfect hashing!dynamic}
The FKS-$\alpha$ data structure is nice in that it allows for searches
in $O(1)$ time, in the worst case.  Unfortunately, it is only static;
it does not support insertions or deletions of elements.  In this
section we describe a result of Dietzfelbinger \etal\ \cite{dkmm94}
that shows how the FKS-$\alpha$ data structure can be made dynamic
with some judicious use of partial rebuilding
(Section~\ref{external:12:Andersson-trees-partial-rebuilding}).

The main idea behind the scheme is simple: be lazy at both the upper
and lower levels of the FKS-$\alpha$ data structure.  That is, rebuild
parts of the data structure only when things go wrong.  At the top
level, we relax the condition that the size $m$ of the upper array $A$
is exactly $n$ and allow $A$ to have size anywhere between $n$ and
$2n$.  Similarly, at the lower level we allow the array $A_i$ to have
a size $m_i$ anywhere between $n_i{}^2/\alpha$ and $2n_i{}^2/\alpha$.

Periodically, we will perform a \emph{global rebuilding}\index{global
rebuilding} operation in
which we remove all $n$ elements from the hash table.  Some elements
which have previously been marked as deleted will be discarded,
thereby reducing the value of $n$.  We put the remaining elements in a
list, and recompute a whole new FKS-$(\alpha/2)$ data structure for
the elements in the list.  This data structure is identical to the
standard FKS-$(\alpha/2)$ data structure except that, at the top level
we use an array of size $m=2n$.

Searching in this data structure is exactly the same as for the static
data structure.  To search for an element $x$ we compute
$i=h_{k,m}(x)$ and $j=h_{k_i,m_i}(x)$ and look for $x$ at location
$A_i[j]$.   Thus, searches take $O(1)$ worst-case time.

Deleting in this data structure is done in the laziest manner
possible.  To delete an element we only search for it and then mark it
as deleted.  We will use the convention that this type of deletion
does not change the value of $n$ since it does not change the number
of elements actually stored in the data structure.  While doing this,
we also keep track of the number of elements that are marked as
deleted.  When this number exceeds $n/2$ we perform a global
rebuilding operation.  The global rebuilding operation takes $O(n)$
expected time, but only occurs during one out of every $n/2$
deletions.  Therefore, the amortized cost of this operation is $O(1)$
per deletion.

The most complicated part of the data structure is the insertion
algorithm and its analysis.  To insert a key $x$ we know, because of
how the search algorithm works, that we must ultimately store $x$ at
location $A_i[j]$ where $i=h_{k,m}(x)$ and $j=h_{k_i,m_i}(x)$.
However, several things can go wrong during the insertion of $x$:

\begin{enumerate} 
\item The value of $n$ increases by 1, so it may be that $n$ now
exceeds $m$.  In this case we perform a global rebuilding
operation and we are done.

\item We compute $i=h_{k,m}(x)$ and discover that $\sum_{i=0}^{m-1}
n_i{}^2 > 3n/\alpha$.  In this case, the hash function
$h_{k,m}$ used at the top level is no longer any good since it is
producing an overall hash table that is too large.  In this case we
perform a global rebuilding operation and we are done.

\item We compute $i=h_{k,m}(x)$ and discover that, since the value of
$n_i$ just increased by one, $n_i{}^2/\alpha > m_i$. In this case, the
array $A_i$ is too small to guarantee that we can quickly find a
perfect hash function.  To handle this, we copy the elements of $A_i$
into a list $L$ and allocate a new array $A_i$ with the new size
$m_i=2n_i{}^2/\alpha$.  We then find a new value $k_i$ such that
$h_{k_i,m_i}$ is a perfect hash function for the elements of $L$ and
we are done.

\item The array location $A_i[j]$ is already occupied by some other
element $y$.  But in this case, we know that $A_i$ is large enough to
hold all the elements (otherwise we would already be done after
Case~3), but the value $k_i$ being used in the hash function
$h_{k_i,m_i}$ is the wrong one since it doesn't give a perfect hash
function for $S_i$.  Therefore we simply try new values for $k_i$
until we find a find a value $k_i$ that yields a perfect hash function
and we are done.  
\end{enumerate}

If none of the preceding 4 cases occurs then we can simply place $x$
at location $A_i[j]$ and we are done.

Handling Case~1 takes $O(n)$ expected time since it involves a global
rebuild of the entire data structure.  However, Case~1 only happens
during one out of every $\Theta(n)$ insertions, so the amortized cost
of all occurrences of Case~1 is only $O(1)$ per insertion.

Handling Case~2 also takes $O(n)$ expected time.  The question is: How
often does Case~2 occur?  To answer this question, consider the
\emph{phase}\index{phase} that occurs between two consecutive
occurrences of Case~1.  During this phase, the data structure holds at
most $m$ distinct elements.  Call this set of elements $S$.  With
probability at least $(1-\alpha)$ the hash function $h_{k,m}$ selected
at the beginning of the phase satisfies (\ref{hash:eq:fks-space}) so
that Case~2 never occurs during the phase.  Similarly, the probability
that Case~2 occurs exactly once during the phase is at most
$\alpha(1-\alpha)$.  In general, the probability that Case~2 occurs
exactly $i$ times during a phase is at most $\alpha^i(1-\alpha)$.  Thus, the
expected cost of handling all occurrences of Case~2 during the entire
phase is at most 
\[ \sum_{i=0}^\infty \alpha^i(1-\alpha)i\times O(n) =
	O(n) \enspace .  
\] 
But since a phase involves $\Theta(n)$ insertions this means that the
amortized expected cost of handling Case~2 is $O(1)$ per insertion.

Next we analyze the total cost of handling Case~3.  Define a
\emph{subphase}\index{subphase} as the period of time between two
global rebuilding operations triggered either as a result of a
deletion, Case~1 or Case~2.   We will show that the total cost of
handling all occurrences of Case~3 during a subphase is $O(n)$ and
since a subphase takes $\Theta(n)$ time anyway this does not
contribute to the cost of a subphase by more than a constant factor.
When Case~3 occurs at the array $A_i$ it takes $O(m_i)$ time.
However, while handling Case~3, $m_i$ increases by a constant factor,
so the total cost of handling Case~3 for $A_i$ is dominated by the
value of $m_i$ at the end of the subphase.  But we maintain the
invariant that $\sum_{i=0}^{m-1} m_i = O(n)$ during the entire
subphase.  Thus, handling all occurrences of Case~3 during a subphase
only requires $O(n)$ time. 

Finally, we consider the cost of handling Case~4.  For a particular
array $A_i$, consider the \emph{subsubphase}\index{subsubphase}
between which two occurrences of Case~3 cause $A_i$ to be rebuilt or a
global rebuilding operation takes place.  During this subsubphase the
number of distinct elements that occupy $A_i$ is at most
$\alpha\sqrt{m_i}$.  Therefore, with probability at least $1-\alpha$
any randomly chosen value of $k_i\in U$ is a perfect hash function for
this set.   Just as in the analysis of Case~2, this implies that the
expected cost of handling all occurrences of Case~3 at $A_i$ during a
subsubphase is only $O(m_i)$.  Since a subsubphase ends with
rebuilding all of $A_i$ or a global rebuilding, at a cost of
$\Omega(m_i)$ all the occurrences of Case~4 during a subsubphase do not
contribute to the expected cost of the subsubphase by more than a
constant factor.

To summarize, we have shown that the expected cost of handling all
occurrences of Case~4 is only  a constant factor times the cost of
handling all occurrences of Case~3.  The cost of handling all
occurrences of Case~3 is no more than a constant factor times the
expected cost of all global rebuilds.  The cost of handling all the
global rebuilds that occur as a result of Case~2 is no more than a
constant factor times the cost of handling all occurrences of global
rebuilds that occur as a consequence of Case~1.  And finally, the cost
of all global rebuilds that occur as a result of Case~1 or of
deletions is $O(n)$ for a sequence of $n$ update operations.
Therefore, the total expected cost of $n$ update operation is $O(n)$.


\section{Random Probing}
\label{hash:sec:uniform}
\index{random probing}
\index{probing!random}

Next we consider hash table implementations under the random probing
assumption: Each element $x$ stored in the hash table comes with a
random sequence $x_0,x_1,x_2,\ldots$ where each of the $x_i$ is
independently and uniformly distributed in
$\{1,\ldots,m\}$.\footnote{A variant of the random probing assumption,
referred to as the \emph{uniform hashing}\index{uniform
hashing}\index{hashing!uniform} assumption, assumes that
$x_0,\ldots,x_{m-1}$ is a random permutation of $0,\ldots,m-1$.}  We
begin with a discussion of the two basic paradigms: hashing with
chaining and open addressing.  Both these paradigms attempt to store
the key $x$ at array position $A[x_0]$.  The difference between these
two algorithms is their \emph{collision resolution
strategy}\index{collision resolution strategy}, i.e., what the
algorithms do when a user inserts the key value $x$ but array position
$A[x_0]$ already contains some other key.

\subsection{Hashing with Chaining}
\index{hashing!with chaining}

In \emph{hashing with chaining}, a collision is resolved by allowing
more than one element to live at each position in the table.  Each
entry in the array $A$ is a pointer to the head of a linked list.  To
insert the value $x$, we simply append it to the list $A[x_0]$.  To
search for the element $x$, we perform a linear search in the list
$A[x_0]$. To delete the element $x$, we search for $x$ in the list
$A[x_0]$ and splice it out.

It is clear that insertions take $O(1)$ time, even in the worst case.
For searching and deletion, the running time is proportional to a
constant plus the length of the list stored at $A[x_0]$.  Notice that
each of the at most $n$ elements not equal to $x$ is stored in
$A[x_0]$ with probability $1/m$, so the expected length of $A[x_0]$ is
either $\alpha=n/m$ (if $x$ is not contained in the table) or
$1+(n-1)/m$ (if $x$ is contained in the table). Thus, the expected
cost of searching for or deleting an element is $O(1+\alpha)$.

The above analysis shows us that hashing with chaining supports the
three set ADT operations in $O(1)$ expected time per operation, as
long as the occupancy, $\alpha$, is a constant. It is worth noting
that this does not require that the value of $\alpha$ be less than 1. 

If we would like more detailed information about the cost of searching, we
might also ask about the \emph{worst-case search time} defined as
\[
   W = \max\{\mbox{length of the list stored at $A[i]$} : 0\le i\le m-1 \} 
	\enspace .
\]
It is very easy to prove something quite strong about $W$ using only
the fact that the length of each list $A[i]$ is a
$\mathrm{binomial}(n,1/m)$ random variable.  Using Chernoff's bounds
on the tail of the binomial distribution \cite{c52}, this
immediately implies that   
\[
   \Pr\{\mbox{length of $A[i]$} \ge \alpha c\ln n\} \le n^{-\Omega(c)} \enspace .
\]
Combining this with Boole's inequality ($\Pr\{\mbox{$A$ or $B$}\} \le
\Pr\{A\}+\Pr\{B\}$) we obtain
\[
   \Pr\{W \ge \alpha c\ln n\} \le n\times n^{-\Omega(c)} = n^{-\Omega(c)} \enspace .
\]

Thus, with very high probability, the worst-case search time is
logarithmic in $n$.  This also implies that $\E[W]=O(\log n)$.  The
distribution of $W$ has been carefully studied and it is known that,
\emph{with high probability}\index{high probability}, i.e., with
probability $1-o(1)$, $W=(1+o(1))\ln n/\ln\ln n$
\cite{jk77,ksc78}.\footnote{Here, and throughout this chapter, if an
asymptotic notation does not contain a variable then the variable that
tends to infinity is implicitly $n$.  Thus, for example, $o(1)$ is the
set of non-negative functions of $n$ that tend to 0 as
$n\rightarrow\infty$.} Gonnet has proven a more accurate result that
$W=\Gamma^{-1}(n) -3/2 + o(1)$ with high probability. Devroye
\cite{d85} shows that similar results hold even when the distribution
of $x_0$ is not uniform.

\subsection{Hashing with Open Addressing}
\index{open addressing}
\index{hashing!with open addressing}

\label{hash:sec:open-addressing}

\emph{Hashing with open addressing} differs from hashing with chaining in
that each table position $A[i]$ is allowed to store only one value.
When a collision occurs at table position $i$, one of the two elements
involved in the collision must move on to the next element in its
probe sequence.  In order to implement this efficiently and correctly
we require a method of marking elements as deleted.  This method could
be an auxiliary array that contains one bit for each element of $A$,
but usually the same result can be achieved by using a special key
value $\mathbf{del}$ that does not correspond to any valid key.

To search for an element $x$ in the hash table we look for $x$ at
positions $A[x_0]$, $A[x_1]$, $A[x_2]$, and so on until we either
(1)~find $x$, in which case we are done or (2)~find an empty table
position $A[x_i]$ that is not marked as deleted, in which case we can
be sure that $x$ is not stored in the table (otherwise it would be
stored at position $x_i$).  To delete an element $x$ from the hash
table we first search for $x$.  If we find $x$ at table location
$A[x_i]$ we then simply mark $A[x_i]$ as deleted.  To insert a value
$x$ into the hash table we examine table positions $A[x_0]$, $A[x_1]$,
$A[x_2]$, and so on until we find a table position $A[x_i]$ that is
either empty or marked as deleted and we store the value $x$ in
$A[x_i]$.

Consider the cost of inserting an element $x$ using this method.  Let
$i_x$ denote the smallest value $i$ such that $x_{i_x}$ is either empty or
marked as deleted when we insert $x$.  Thus, the cost of inserting $x$
is a constant plus $i_x$.  The probability that the table position
$x_0$ is occupied is at most $\alpha$ so, with probability at least
$1-\alpha$, $i_x=0$.  Using the same reasoning, the probability
that we store $x$ at position $x_i$ is at most 
\begin{equation}
   \Pr\{i_x = i\} \le \alpha^{i}(1-\alpha)  \label{hash:eq:pro}
\end{equation}
since the table locations $x_0,\ldots,x_{i-1}$ must be occupied,
the table location $x_i$ must not be occupied and the $x_i$ are
independent.  Thus, the expected number of steps taken by the
insertion algorithm is 
\[
   \sum_{i=1}^\infty i\Pr\{i_x = i\} = (1-\alpha)\sum_{i=1}^\infty i\alpha^{i-1}  = 1/(1-\alpha)
\]
for any constant $0<\alpha< 1$.  The cost of searching for $x$ and
deleting $x$ are both proportional to the cost of inserting $x$, so
the expected cost of each of these operations is
$O(1/(1-\alpha))$.\footnote{Note that the expected cost of searching
for or deleting an
element $x$ is proportional to the value of $\alpha$ \emph{at the time
$x$ was inserted.}  If many deletions have taken place, this may be
quite different than the current value of $\alpha$.}   

We should compare this with the cost of hashing with chaining.  In
hashing with chaining,the occupancy $\alpha$ has very little effect on
the cost of operations.  Indeed, any constant $\alpha$, even greater
than $1$ results in $O(1)$ time per operation.  In contrast, open
addressing is very dependent on the value of $\alpha$.  If we take
$\alpha>1$ then the expected cost of insertion using open addressing
is infinite since the insertion algorithm never finds an empty table
position.  Of course, the advantage of hashing with chaining is that
it does not require lists at each of the $A[i]$.  Therefore, the
overhead of list pointers is saved and this extra space can be used
instead to maintain the invariant that the occupancy $\alpha$ is a
constant strictly less than 1.

Next we consider the worst case search time of hashing with open
addressing.  That is, we study the value $W=\max\{i_x: \mbox{$x$ is
stored in the table at location $i_x$}\}$.  Using (\ref{hash:eq:pro}) and Boole's
inequality it follows almost immediately that 
\[
  \Pr\{W > c\log n\} \le n^{-\Omega(c)} .
\]

Thus, with very high probability, $W$, the worst case search time, is
$O(\log n)$.  Tighter bounds on $W$ are known when the probe sequences
$x_0,\ldots,x_{m-1}$ are random permutations of $0,\ldots,m-1$.  In this
case, Gonnet shows that
\[
  \E[W]=\log_{1/\alpha} n - \log_{1/\alpha}(\log_{1/\alpha} n) + O(1)
\]
\cite{g81}. 

Open addressing under the random probing assumption has many nice
theoretical properties and is easy to analyze.  Unfortunately, it is
often criticized as being an unrealistic model because it requires a
long random sequences $x_0,x_1,x_2,\ldots$ for each element $x$ that
is to be stored or searched for.  Several variants of open addressing
discussed in the next few sections try to overcome this problem by
using only a few random values. 


\subsection{Linear Probing}
\index{linear probing}
\index{hashing!with linear probing}
 
\emph{Linear probing} is a variant of open addressing that requires less
randomness.  To obtain the probe sequence $x_0,x_1,x_2,\ldots$ we
start with a random element $x_0\in \{0,\ldots,m-1\}$.  The element
$x_i$, $i>0$ is given by $x_i=(i+x_0)\bmod m$.  That is, one first
tries to find $x$ at location $x_0$ and if that fails then one looks
at $(x_0+1)\bmod m$, $(x_0+2)\bmod m$ and so on.  

The performance of linear probing is discussed by Knuth \cite{k97} who
shows that the expected number of probes performed during an
unsuccessful search is at most 
\[
	(1+1/(1-\alpha)^2)/2
\] 
and the expected
number of probes performed during a successful search is at most
\[
	(1+1/(1-\alpha))/2 \enspace . 
\] 
This is not quite as good as for standard hashing with open
addressing, especially in the unsuccessful case.

Linear probing suffers from the problem of \emph{primary
clustering}\index{primary clustering}\index{clustering!primary}.  If
$j$ consecutive array entries are occupied then a newly inserted
element will have probability $j/m$ of hashing to one of these
entries. This results in $j+1$ consecutive array entries being
occupied and increases the probability (to $(j+1)/m$) of another newly
inserted element landing in this cluster.  Thus, large clusters of
consecutive elements have a tendency to grow larger.

\subsection{Quadratic Probing}
\index{quadratic probing}
\index{hashing!with quadratic probing}

\emph{Quadratic probing} is similar to linear probing;  an element $x$
determines its entire probe sequence based on a single random choice,
$x_0$.  Quadratic probing uses the probe sequence
$x_0,(x_0+k_1+k_2)\bmod m, (x_0+2k_1+2^2k_2)\bmod m,\ldots$.  In
general, the $i$th element in the probe sequence is
$x_i=(x_0+ik_1+i^2k_2)\bmod m$.  Thus, the final location of an
element depends quadratically on how many steps were required to
insert it.  This method seems to work much better in practice than
linear probing, but requires a careful choice of $m$, $k_1$ and $k_2$
so that the probe sequence contains every element of $\{0,\ldots,m-1\}$.

The improved performance of quadratic probing is due to the fact that
if there are two elements $x$ and $y$ such that $x_i=y_j$ then it is
not necessarily true (as it is with linear probing) that
$x_{i+1}=y_{j+1}$.  However, if $x_0=y_0$ then $x$ and $y$ will have
exactly the same probe sequence.  This lesser phenomenon is called
\emph{secondary clustering}\index{secondary
clustering}\index{clustering!secondary}.  Note that this secondary
clustering phenomenon implies that neither linear nor quadratic
probing can hope to perform any better than hashing with chaining.
This is because all the elements that have the same initial hash $x_0$
are contained in an implicit chain.  In the case of linear probing,
this chain is defined by the sequence $x_0,x_0+1,x_0+2,\ldots$ while
for quadratic probing it is defined by the sequence
$x_0,x_0+k_1+k_2,x_0+2k_1+4k_2,\ldots$

\subsection{Double Hashing}\label{hash:sec:double}
\index{hashing!double}
\index{double hashing}

\emph{Double hashing} is another method of open addressing that uses two hash
values $x_0$ and $x_1$.  Here $x_0$ is in the set $\{0,\ldots,m-1\}$
and $x_1$ is in the subset of $\{1,\ldots,m-1\}$ that is relatively
prime to $m$.  With double hashing, the probe sequence for element $x$
becomes $x_0,(x_0+x_1)\bmod m, (x_0+2x_1)\bmod m,\ldots$.  In general,
$x_i=(x_0+ix_1)\bmod m$, for $i>0$.  The expected number of probes
required by double hashing seems difficult to determine exactly.
Guibas has proven that, asymptotically, and for occupancy $\alpha\le
.31$, the performance of double hashing is asymptotically equivalent
to that of uniform hashing.  Empirically, the performance of double
hashing matches that of open addressing with random probing
regardless of the occupancy $\alpha$ \cite{k97}.

\subsection{Brent's Method}\label{hash:sec:brent}
\index{Brent's method}
\index{hashing!Brent's method}

\emph{Brent's method} \cite{b73} is a heuristic that attempts to
minimize the average time for a successful search in a hash table with
open addressing.  Although originally described in the context of
double hashing (Section~\ref{hash:sec:double}) Brent's method applies
to any open addressing scheme.  The \emph{age}\index{age} of an element $x$
stored in an open addressing hash table is the minimum value $i$ such
that $x$ is stored at $A[x_i]$.  In other words, the age is one less
than the number of locations we will probe when searching for $x$. 

Brent's method attempts to minimize the total age of all elements in
the hash table.  To insert the element $x$ we proceed as follows:  We
find the smallest value $i$ such that $A[x_i]$ is empty; this is where
standard open-addressing would insert $x$.  Consider the element $y$
stored at location $A[x_{i-2}]$.  This element is stored there because
$y_j=x_{i-2}$, for some $j\ge 0$.  We check if the array location
$A[y_{j+1}]$ is empty and, if so, we move $y$ to location $A[y_{j+1}]$
and store $x$ at location $A[x_{i-2}]$.  Note that, compared to
standard open addressing, this decreases the total age by 1.  In
general, Brent's method checks, for each $2\le k\le i$ the array entry
$A[x_{i-k}]$ to see if the element $y$ stored there can be moved to
any of $A[y_{j+1}],A[y_{j+2}],\ldots,A[y_{j+k-1}]$ to make room for
$x$.  If so, this represents a decrease in the total age of all
elements in the table and is performed.

Although Brent's method seems to work well in practice, it is
difficult to analyze theoretically.  Some theoretical analysis of
Brent's method applied to double hashing is given by Gonnet and Munro
\cite{gm79}.  Lyon \cite{lyon78}, Munro and Celis \cite{mc99} and
Poblete \cite{p77} describe some variants of Brent's method.


\subsection{Multiple-Choice Hashing}
\label{hash:sec:multiplechoice}
\index{hashing!multiple-choice}
\index{multiple-choice hashing}

It is worth stepping back at this point and revisiting the comparison
between hash tables and binary search trees.  For balanced binary
search trees, the average cost of searching for an element is $O(\log
n)$.  Indeed, it easy to see that for at least $n/2$ of the elements,
the cost of searching for those elements is $\Omega(\log n)$.  In
comparison, for both the random probing schemes discussed so far, the
expected cost of search for an element is $O(1)$.  However, there are
a handful of elements whose search cost is $\Theta(\log n/\log \log
n)$ or $\Theta(\log n)$ depending on whether hashing with chaining or
open addressing is used, respectively.  Thus there is an inversion:
Most operations on a binary search tree cost $\Theta(\log n)$ but a
few elements (close to the root) can be accessed in $O(1)$ time.  Most
operations on a hash table take $O(1)$ time but a few elements (in long
chains or with long probe sequences) require $\Theta(\log n/\log\log
n)$ or $\Theta(\log n)$ time to access.  In the next few sections we
consider variations on hashing with chaining and open addressing that
attempt to reduce the worst-case search time $W$.

\emph{Multiple-choice hashing} is hashing with chaining in which,
during insertion, the element $x$ has the choice of $d\ge 2$ different
lists in which it can be stored.  In particular, when we insert $x$ we
look at the lengths of the lists pointed to by
$A[x_0],\ldots,A[x_{d-1}]$ and append $x$ to $A[x_i]$, $0\le i< d$
such that the length of the list pointed to by $A[x_i]$ is minimum.
When searching for $x$, we search for $x$ in each of the lists
$A[x_0],\ldots,A[x_{d-1}]$ \emph{in parallel}.  That is, we look at
the first elements of each list, then the second elements of each
list, and so on until we find $x$.  As before, to delete $x$  we first
search for it and then delete it from whichever list we find it in.  

It is easy to see that the expected cost of searching for an element
$x$ is $O(d)$ since the expected length of each the $d$ lists is
$O(1)$.  More interestingly, the worst case search time is bounded by
$O(dW)$ where $W$ is the length of the longest list.  Azar \etal\
\cite{abku99} show that 
\begin{equation}
    \E[W]=\frac{\ln\ln n}{\ln d} + O(1) \enspace .
	\label{hash:eq:multiple-choice} 
\end{equation}
Thus, the expected worst case search time for multiple-choice hashing
is $O(\log\log n)$ for any constant $d\ge 2$.


\subsection{Asymmetric Hashing}
\index{hashing!asymmetric}
\index{asymmetric hashing}

\emph{Asymmetric hashing} is a variant of multiple-choice hashing in
which the hash table is split into $d$ blocks, each of size $n/d$.
(Assume, for simplicity, that $n$ is a multiple of $d$.) The probe
value $x_i$, $0\le i< d$ is drawn uniformly from
$\{in/d,\ldots,(i+1)n/d-1\}$.   As with multiple-choice hashing, to
insert $x$ the algorithm examines the lengths of the lists $A[x_0],
A[x_1],\ldots,A[x_{d-1}]$ and appends $x$ to the shortest of these
lists.  In the case of ties, it appends $x$ to the list with smallest
index.  Searching and deletion are done exactly as in multiple-choice
hashing.

V\"ocking \cite{v99} shows that, with asymmetric hashing the expected
length of the longest list is 
\[
   \E[W] \le \frac{\ln\ln n}{d\ln\phi_d} +O(1) \enspace .
\] 
The function $\phi_d$ is a generalization of the \emph{golden
ratio}\index{golden ratio},
so that $\phi_2=(1+\sqrt{5})/2$.  Note that this improves
significantly on standard multiple-choice hashing (\ref{hash:eq:multiple-choice}) for larger values of
$d$.

\subsection{LCFS Hashing}
\index{hashing!LCFS}
\index{LCFS hashing}

\emph{LCFS hashing} is a form of open addressing that changes the
collision resolution strategy.\footnote{Amble and Knuth \cite{ak74}
were the first to suggest that, with open addressing, any collision
resolution strategy could be used.} Reviewing the algorithm for
hashing with open addressing reveals that when two elements collide,
priority is given to the first element inserted into the hash table
and subsequent elements must move on.  Thus, hashing with open
addressing could also be referred to as \emph{FCFS (first-come
first-served) hashing}\index{FCFS hashing}\index{hashing!FCFS}.

With LCFS (last-come first-served) hashing, collision resolution is
done in exactly the opposite way.  When we insert an element $x$, we
always place it at location $x_0$.  If position $x_0$ is already
occupied by some element $y$ because $y_j=x_0$ then we place $y$ at
location $y_{j+1}$, possibly displacing some element $z$, and so on.

Poblete and Munro \cite{pm89} show that, after inserting $n$ elements
into an initially empty table, the expected worst case search time is
bounded above by
\[  \E[W] \le 
   1 
   + \Gamma^{-1}(\alpha n)
   \left( 1+\frac{\ln\ln(1/(1-\alpha))}{\ln \Gamma^{-1}(\alpha n)}
   + O\left(\frac{1}{\ln^2\Gamma^{-1}(\alpha n)} \right)\right)
\enspace ,
\]
where $\Gamma$ is the gamma function and 
\[
  \Gamma^{-1}(\alpha n) = \frac{\ln n}{\ln\ln n}
	\left(1+\frac{\ln\ln\ln n}{\ln\ln n} + 
	O\left(\frac{1}{\ln\ln n}\right)\right) \enspace .
\]
Historically, LCFS hashing is the first version of open addressing
that was shown to have an expected worst-case search time that is
$o(\log n)$.

\subsection{Robin-Hood Hashing}
\label{hash:sec:robinhood}
\index{hashing!Robin-Hood}
\index{Robin-Hood hashing}

Robin-Hood hashing \cite{c86,clm85,vp98} is a form of open addressing
that attempts to equalize the search times of elements by using a
fairer collision resolution strategy.  During insertion, if we are
trying to place element $x$ at position $x_i$ and there is already an
element $y$ stored at position $y_j=x_i$ then the ``younger'' of the
two elements must move on.  More precisely, if $i\le j$ then we will
try to insert $x$ at position $x_{i+1}$, $x_{i+2}$ and so on.
Otherwise, we will store $x$ at position $x_i$ and try to to insert
$y$ at positions $y_{j+1}$, $y_{j+2}$ and so on.

Devroye \etal\ \cite{dmv03} show that, after performing $n$ insertions on
an initially empty table of size $m=\alpha n$ using the Robin-Hood
insertion algorithm, the worst case search time has expected value
\[
	\E[W] = \Theta(\log\log n)
\]
and this bound is tight.
Thus, Robin-Hood hashing is a form of open addressing that
has doubly-logarithmic worst-case search time.  This makes it
competitive with the multiple-choice hashing method of
Section~\ref{hash:sec:multiplechoice}.

\subsection{Cuckoo Hashing}\label{hash:sec:cuckoo}
\index{hashing!cuckoo}
\index{cuckoo hashing}

Cuckoo hashing \cite{pr01} is a form of multiple choice hashing in
which each element $x$ lives in one of two tables $A$ or $B$, each of
size $m=n/\alpha$.  The element $x$ will either be stored at location
$A[x_A]$ or $B[x_B]$. There are no other options.  This makes
searching for $x$ an $O(1)$ time operation since we need only check
two array locations.

The insertion algorithm for cuckoo hashing proceeds as
follows:\footnote{The algorithm takes its name from the large but lazy
cuckoo bird which, rather than building its own nest, steals the nest
of another bird forcing the other bird to move on.} Store $x$ at
location $A[x_A]$. If $A[x_A]$ was previously occupied by some element
$y$ then store $y$ at location $B[y_B]$.  If $B[y_B]$ was previously
occupied by some element $z$ then store $z$ at location $A[z_A]$, and
so on.  This process ends when we place an element into a previously
empty table slot or when it has gone on for more than $c\log n$ steps.
In the former case, the insertion of $x$ completes successfully.  In
the latter case the insertion is considered a failure, and the entire
hash table is reconstructed from scratch using a new probe sequence
for each element in the table.  That is, if this reconstruction
process has happened $i$ times then the two hash values we use for an
element $x$ are $x_A=x_{2i}$ and $x_B=x_{2i+1}$.

Pagh and Rodler \cite{pr01} (see also Devroye and Morin \cite{dm02})
show that, during the insertion of $n$ elements, the probability of
requiring a reconstruction is $O(1/n)$. This, combined with the fact
that the expected insertion time is $O(1)$ shows that the expected
cost of $n$ insertions in a Cuckoo hashing table is $O(n)$.  Thus,
Cuckoo hashing offers a somewhat simpler alternative to the dynamic
perfect hashing algorithms of Section~\ref{hash:sec:dphash}.

\section{Historical Notes}

In this section we present some of the history of hash tables. The
idea of hashing seems to have been discovered simultaneously by two
groups of researchers. Knuth \cite{k97} cites an internal IBM
memorandum in January 1953 by H. P. Luhn that suggested the use of
hashing with chaining.  Building on Luhn's work, A. D. Linh suggested
a method of open addressing that assigns the probe sequence
$x_0,\floor{x_0/10},\floor{x_0/100},\floor{x_0/1000},\ldots$ to the
element $x$.

At approximately the same time, another group of researchers at IBM:
G. M. Amdahl, E.~M.~Boehme, N. Rochester and A. L. Samuel implemented
hashing in an assembly program for the IBM 701 computer.  Amdahl is
credited with the idea of open addressing with linear probing.

The first published work on hash tables was by A. I. Dumey \cite{d56},
who described hashing with chaining and discussed the idea of using
remainder modulo a prime as a hash function.  Ershov \cite{e58},
working in Russia and independently of Amdahl, described open
addressing with linear probing.

Peterson \cite{p57} wrote the first major article discussing the
problem of searching in large files and coined the term ``open
addressing.''  Buchholz \cite{b63} also gave a survey of the searching
problem with a very good discussion of hashing techniques at the time.
Theoretical analyses of linear probing were first presented by Konheim
and Weiss \cite{kw66} and Podderjugin.  Another, very influential,
survey of hashing was given by Morris \cite{m68}.  Morris' survey is
the first published use of the word ``hashing'' although it was
already in common use by practitioners at that time.



\section{Other Developments}

The study of hash tables has a long history and many researchers have
proposed methods of implementing hash tables.  Because of this, the
current chapter is necessarily incomplete. (At the time of writing,
the hash.bib bibliography on hashing contains over 800 entries.) We
have summarized only a handful of the major results on hash tables in
internal memory.  In this section we provide a few references to the
literature for some of the other results.   For more information on
hashing, Knuth \cite{k97}, Vitter and Flajolet \cite{vf90}, Vitter
and Chen \cite{vc87}, and Gonnet and Baeza-Yates \cite{gb-y91}
% and Flajolet \etal \cite{X} 
are useful references.

Brent's method (Section~\ref{hash:sec:brent}) is a collision
resolution strategy for open addressing that reduces the expected
search time for a successful search in a hash table with open
addressing.  Several other methods exist that either reduce the
expected or worst-case search time.  These include \emph{binary tree
hashing}\index{hashing!binary tree}\index{binary tree hashing}
\cite{m77, gm79}, \emph{optimal
hashing}\index{hashing!optimal}\index{optimal hashing}
\cite{p76,r78,gm79}, Robin-Hood hashing
(Section~\ref{hash:sec:robinhood}), and \emph{min-max
hashing}\index{hashing!min-max}\index{min-max hashing} \cite{g81,c86}.
One interesting method, due to Celis\index{Celis'
method}\index{hashing!Celis' method} \cite{c86}, applies to any open
addressing scheme.  The idea is to study the distribution of the ages
of elements in the hash table, i.e., the distribution give by 
\[
    D_i=\Pr\{\mbox{$x$ is stored at position $x_i$}\} 
\] 
and start searching for $x$ at the locations at which we are most
likely to find it, rather than searching the table positions
$x_0,x_1,x_2\ldots$ in order.

Perfect hash functions seem to have been first studied by Sprugnoli
\cite{s77} who gave some heuristic number theoretic constructions of
minimal perfect hash functions for small data sets.  Sprugnoli is
responsible for the terms ``perfect hash function'' and ``minimal
perfect hash function.''  A number of other researchers have presented
algorithms for discovering minimal and near-minimal perfect hash
functions.  Examples include Anderson and Anderson \cite{aa79},
Cichelli \cite{c80a,c80b}, Chang \cite{c84a,c84b}, Gori and Soda
\cite{gs89}, and Sager \cite{s85}.  Berman \etal\ \cite{bbdo86} and
K\"orner and Marton \cite{km88} discuss the theoretical limitations of
perfect hash functions.  A comprehensive, and recent, survey of
perfect hashing and minimal perfect hashing is given by Czech \etal\
\cite{chm97}.

Tarjan and Yao \cite{ty79} describe a set ADT implementation that
gives $O(\log u/\log n)$ worst-case access time.  It is obtained by
combining a trie (Section~\ref{external:32:tries}) of degree $n$ with a
compression scheme for arrays of size $n^2$ that contain only $n$
non-zero elements.  (The trie has $O(n)$ nodes each of which has $n$
pointers to children, but there are only a total of $O(n)$ children.)
Although their result is superseded by the results of Fredman \etal\
\cite{fks84} discussed in Section~\ref{hash:sec:fks}, they are the
first theoretical results on worst-case search time for hash tables.

Dynamic perfect hashing (Section~\ref{hash:sec:dphash}) and cuckoo
hashing (Section~\ref{hash:sec:cuckoo}) are methods of achieving
$O(1)$ worst case search time in a dynamic setting.  Several other
methods have been proposed  \cite{dm90,dgmp92,bm99}.
 
Yao \cite{y81} studies the \emph{membership problem}\index{membership
problem}.  Given a set $S\subseteq U$, devise a data structure that
can determine for any $x\in U$ whether $x$ is contained in $S$. Yao
shows how, under various conditions, this problem can be solved using
a very small number of memory accesses per query.  However, Yao's
algorithms sometimes derive the fact that an element $x$ is in $S$
without actually finding $x$.  Thus, they don't solve the set ADT
problem discussed at the beginning of this chapter since they can not
recover a pointer to $x$.

The ``power of two random choices,'' as used in multiple-choice
hashing, (Section~\ref{hash:sec:multiplechoice}) has many applications
in computer science.  Karp, Luby and Meyer auf der Heide
\cite{klm92,klm93} were the first to use this paradigm for simulating
PRAM computers on computers with fewer processors. The book chapter by
Mitzenmacher \etal\ \cite{mrs01} surveys results and applications of
this technique.

A number of table implementations have been proposed that are suitable
for managing hash tables in external memory.\index{hashing!external
memory}\index{external memory!hashing}  Here, the goal is to reduce
the number of disk blocks that must be accessed during an operation,
where a disk block can typically hold a large number of elements.
These schemes include \emph{linear
hashing}\index{hashing!linear}\index{linear hashing} \cite{l80},
\emph{dynamic hashing}\index{hashing!dynamic}\index{dynamic hashing}
\cite{l78}, \emph{virtual hashing}\index{virtual
hashing}\index{hashing!virtual} \cite{lit78}, \emph{extendible
hashing}\index{hashing!extendible}\index{extendible hashing}
\cite{fnps79},  \emph{cascade
hashing}\index{hashing!cascade}\index{cascade hashing} \cite{kz84},
and \emph{spiral storage}\index{spiral storage}\index{hashing!spiral
storage} \cite{m85}.  In terms of hashing, the main difference between
internal memory and external memory is that, in internal memory, an
array is allocated at a specific size and this can not be changed
later.  In contrast, an external memory file may be appended to or be
truncated to increase or decrease its size, respectively.  Thus, hash
table implementations for external memory can avoid the periodic
global rebuilding operations used in internal memory hash table
implementations.

\section*{Acknowledgement}

The author is supported by a grant from the Natural Sciences and
Engineering Research Council of Canada (NSERC).

%%\setcounter{\chapter}{10}
\chapter{Hash Tables}

\newcommand{\E}{\mathrm{E}}
\newcommand{\floor}[1]{\lfloor #1 \rfloor}
\newcommand{\ceil}[1]{\lceil #1 \rceil}

\newcommand{\etal}{et al.}

\begin{chapterauthors}
\chapterauthor{Pat Morin}{Carleton University}
\end{chapterauthors}

\index{hash table}

\section{Introduction}

A \emph{set abstract data type (set ADT)}\index{set ADT} is an
abstract data type that maintains a set $S$ under the following three
operations:

\begin{enumerate}

\item \textsc{Insert}$(x)$:  Add the key $x$ to the set.

\item \textsc{Delete}$(x)$: Remove the key $x$ from the set.

\item \textsc{Search}$(x)$: Determine if $x$ is contained in the set,
and if so, return a pointer to $x$. 

\end{enumerate}
One of the most practical and widely used methods of implementing the
set ADT is with \emph{hash tables}.  

Note that the three set ADT operations can easily be implemented to
run in $O(\log n)$ time per operation using balanced binary search
trees (See Chapter~\ref{external:12:BBST}).  If we assume that the
input data are integers in the set $U=\{0,\ldots,u-1\}$ then they can
even be implemented to run in $O(\log\log u)$ time using data
structures for integer searching
(Chapter~\ref{external:38?:int-searching}).  However, these data
structures actually do more than the three basic operations we
require.  In particular if we search for an element $x$ that is not
present in $S$ then these data structures can report the smallest item
in $S$ that is larger than $x$ (the \emph{successor} of $x$) and/or
the largest item in $S$ that is smaller than $x$ (the
\emph{predecessor} of $x$).  

Hash tables do away with this extra functionality of finding
predecessors and successors and only perform exact searches. If we
search for an element $x$ in a hash table and $x$ is not present then
the only information we obtain is that $x\notin S$. By dropping this
extra functionality hash tables can give better performance bounds.
Indeed, any reasonable hash table implementation performs each of the
three set ADT operations in $O(1)$ expected time.

The main idea behind all hash table implementations discussed in this
chapter is to store a set of $n=|S|$ elements in an array (the hash
table) $A$ of length $m\ge n$.  In doing this, we require a function
that maps any element $x$ to an array location.  This function is
called a \emph{hash function}\index{hash function} $h$ and the value
$h(x)$ is called the \emph{hash value} of $x$.  That is, the element
$x$ gets stored at the array location $A[h(x)]$.  The \emph{occupancy}
of a hash table is the ratio $\alpha=n/m$ of stored elements to the
length of $A$.  

The study of hash tables follows two very different lines. Many
implementations of hash tables are based on the \emph{integer universe
assumption}:\index{integer universe assumption} All elements stored in
the hash table come from the universe $U=\{0,\ldots,u-1\}$.  In this
case, the goal is to design a hash function $h:U\rightarrow
\{0,\ldots,m-1\}$ so that for each $i\in \{0,\ldots,m-1\}$, the number
of elements $x\in S$ such that $h(x)=i$ is as small as possible.
Ideally, the hash function $h$ would be such that each element of $S$
is mapped to a unique value in $\{0,\ldots,m-1\}$.  Most of the hash
functions designed under the integer universe assumption are
number-theoretic constructions. Several of these are described in
Section~\ref{hash:sec:integers}.

Historically, the integer universe assumption seems to have been
justified by the fact that any data item in a computer is represented
as a sequence of bits that can be interpreted as a binary number.
However, many complicated data items require a large (or variable)
number of bits to represent and this make $u$ the size of the universe
very large.  In many applications $u$ is much larger than the largest
integer that can fit into a single word of computer memory.  In this
case, the computations performed in number-theoretic hash functions
become inefficient.

This motivates the second major line of research into hash tables.
This research works is based on the \emph{random probing
assumption\emph{random probing assumption}}: Each element $x$ that is
inserted into a hash table is a black box that comes with an infinite
random \emph{probe sequence}\index{probe sequence}
$x_0,x_1,x_2,\ldots$ where each of the $x_i$ is independently and
uniformly distributed in $\{0,\ldots,m-1\}$.  Hash table
implementations based on the random probing assumption are described
in Section~\ref{hash:sec:uniform}.

Both the integer universe assumption and the random probing
assumption have their place in practice.  When there is an easily
computing mapping of data elements onto machine word sized integers
then hash tables for integer universes are the method of choice.  When
such a mapping is not so easy to compute (variable length strings are
an example) it might be better to use the bits of the input items to
build a good pseudorandom sequence and use this sequence as the probe
sequence for some random probing data structure. 

To guarantee good performance, many hash table implementations require
that the occupancy $\alpha$ be a constant strictly less than $1$.
Since the number of elements in a hash table changes over time, this
requires that the array $A$ be resized periodically.  This is easily
done, without increasing the amortized cost of hash table operations
by choosing three constants $0< \alpha_1 < \alpha_2 < \alpha_3 < 1$ so
that, whenever $n/m$ is not the interval $(\alpha_1,\alpha_3)$ the
array $A$ is resized so that its size is $n/\alpha_2$. A simple
amortization argument (Chapter~\ref{external:1:analysis}) shows that
the amortized cost of this resizing is $O(1)$ per update
(Insert/Delete) operation.


\section{Hash Tables for Integer Keys}
\label{hash:sec:integers}

In this section we consider hash tables under the integer universe
assumption, in which the key values $x$ come from the universe
$U=\{0,\ldots,u-1\}$.   A \emph{hash function}\index{hash function}
$h$ is a function whose domain is $U$ and whose range is the set
$\{0,\ldots,m-1\}$, $m\le u$.  A hash function $h$ is said to be a
\emph{perfect hash function}\index{hash
function!perfect}\index{perfect hash function} for a set $S\subseteq
U$ if, for every $x\in S$, $h(x)$ is unique.  A perfect hash function
$h$ for $S$ is \emph{minimal}\index{hash function!minimal
perfect}\index{minimal perfect hash function} if $m=|S|$, i.e., $h$ is
a bijection between $S$ and $\{0,\ldots,m-1\}$.  Obviously a minimal
perfect hash function for $S$ is desirable since it allows us to store
all the elements of $S$ in a single array of length $n$.
Unfortunately, perfect hash functions are rare, even for $m$ much
larger than $n$.  If each element of $S$ is mapped independently and
uniformly to a random element of $\{0,\ldots,m-1\}$ then the birthday
paradox (See, for example, Feller~\cite{f68}) states that, if $m$ is
much less than $n^2$ then there will almost surely exist two elements
of $S$ that have the same hash value.

We begin our discussion with two commonly used hashing schemes that
are heuristic in nature.  That is, we can not make any non-trivial
statements about the performance of these schemes when storing an
arbitrary set $S$.  We then discuss several schemes that have provably
good performance.

\subsection{Hashing by Division}
\index{hashing!by division}

In \emph{hashing by division}, we use the hash function
\[ h(x) = x\bmod m \enspace . \]
To use this hash function in a data structure, we maintain an array
$A[0],\ldots,A[m-1]$ where each element of this array is a pointer to
the head of a linked list (Section~\ref{external:2:linked-list}).  The
linked list $L_i$ pointed to by the array element $A[i]$ contains all
the elements $x$ such that $h(x)=i$.  This technique of maintaining an
array of lists is called \emph{hashing with
chaining}\index{hashing!with chaining}.

In such a hash table, inserting an element $x$ takes $O(1)$ time;  we
compute $i=h(x)$ and append (or prepend) $x$ to the list $L_i$.
However, searching for and/or deleting an element $x$ is not so easy.
We have to compute $i=h(x)$ and then traverse the list $L_i$ until we
either find $x$ or reach the end of the list.  The cost of this is
proportional to the length of $L_i$.  Obviously, if our set $S$
consists of the elements $0,m,2m,3m,\ldots,nm$ then all elements are
stored in the list $L_0$ and searches and deletions take linear time.

However, one hopes that such pathological cases do not occur in
practice. For example, if the elements of $S$ are uniformly and
independently distributed in $U$ and $u$ is a multiple of $m$ then the
expected size of any list $L_i$ is only $n/m$.  In this case, searches
and deletions take $O(1+\alpha)$ expected time.  To help avoid
pathological cases, the choice of $m$ is important.  In particular,
$m$ a power of 2 is usually avoided since, in a binary computer,
taking the remainder modulo a power of 2 means simply discarding some
high-order bits.  Taking $m$ to be a prime not too close to a power of
2 is recommended \cite{k97}.

\subsection{Hashing by Multiplication}
\index{hashing!by multiplication}

The implementation of a hash table using \emph{hashing by
multiplication} is exactly the same as that hashing by division
except that the hash function 

\[
  h(x) = \floor{mxA} \bmod m
\]
is used.  Here $A$ is a real-valued constant whose choice we discuss
below. The advantage of the multiplication method is that the value of
$m$ is not critical.  We can take $m$ to be a power of 2, which makes
it convenient for use on binary computers. 

Although any value of $A$ gives a hash function, some values of $A$
are better than others.  (Setting $A=0$ is clearly not a good idea.)

Knuth \cite{k97} suggests using the \emph{golden ratio}\index{golden
ratio} for $A$, i.e.,
setting 
\[  
    A=(\sqrt{5}-1)/2=0.6180339887\ldots
\]   
This choice of $A$ is motivated by a theorem, first conjectured by
Oderfeld and later proven by S\'wierczkowski \cite{s58}.  This theorem
states that the sequence 
\[
     mA \bmod m,\,\, 2mA \bmod m,\,\, 3mA \bmod m,\,\,\ldots,\,\, nmA\bmod m
\] 
partitions the interval $(0,m)$ into $n+1$ intervals having only three
distinct lengths.  Furthermore, the next element $(n+1)mA\bmod m$ in
the sequence is always contained in one of the largest
intervals.\footnote{In fact, any irrational number has this property
\cite{sos57}.  The golden ratio is especially good because it is
not too close to a whole number.}

Of course, no matter what value of $A$ we select, the pigeonhole
principle implies that for $u\ge nm$ then there will always exist some
hash value $i$ and some $S\subseteq U$ of size $n$ such that $h(x)=i$
for all $x\in S$.  In other words, we can always find a set $S$ all of
whose elements get stored in the same list $L_i$. Thus, the worst case
of hashing by multiplication is as bad as hashing by division.


\subsection{Universal Hashing}
\label{hash:sec:universal}
\index{hashing!universal}
\index{universal hashing}

The argument used at the end of the previous section applies equally
well to any hash function $h$.  That is, if the table size $m$ is much
smaller than the universe size $u$ then for any hash function there is
some large (of size at least $\ceil{u/m}$) subset of $U$ that has the
same hash value.  To get around this difficulty we need a collection
of hash functions of which we can choose one that works well for $S$.
Even better would be a collection of hash functions such that, for any
given $S$, most of the hash functions work well for $S$.  Then we
could simply pick one of the functions at random and have a good
chance of it working well.

Let $\mathcal{H}$ be a collection of hash functions, i.e., functions
from $U$ onto $\{0,\ldots,m-1\}$.  We say that $\mathcal{H}$ is
\emph{universal}\index{hash function!universal} if, for each $x,y\in
U$ the number of $h\in \mathcal{H}$ such that $h(x)=h(y)$ is at most
$|\mathcal{H}|/m$.  Consider any $S\subseteq U$ of size $n$ and
suppose we choose a random hash function $h$ from a universal
collection of hash functions.  Consider some value $x\in U$.  The
probability that any key $y\in S$ has the same hash value as $x$ is
only $1/m$.  Therefore, the expected number of keys in $S$, not equal
to $x$, that have the same hash value as $x$ is only \[ n_{h(x)} =
\left\{\begin{array}{ll} (n-1)/m & \mbox{if $x\in S$} \\ n/m &
\mbox{if $x\notin S$} \end{array}\right.  \] Therefore, if we store
$S$ in a hash table using the hash function $h$ then the expected time
to search for, or delete, $x$ is $O(1+\alpha)$.

From the preceding discussion, it seems that a universal collection of
hash functions from which we could quickly select one at random would be
very handy indeed. With such a collection at our disposal we get an
implementation of the set ADT that has $O(1)$ insertion time and
$O(1)$ expected search and deletion time.

Carter and Wegman \cite{cw79} describe three different collections of
universal hash functions.  If the universe size $u$ is a
prime number\footnote{This is not a major restriction since, for any
$u>1$, there always exists a prime number in the set
$\{u,u+1,\ldots,2u\}$.  Thus we can enforce this assumption by
increasing the value of $u$ by a constant factor.} then
\[
  \mathcal{H} = \{ h_{k_1,k_2,m}(x)=((k_1x+k_2)\bmod u))\bmod m : 1\le k_1< u,
0\le k_2< u\} 
\]
is a collection of universal hash functions.  Clearly, choosing a
function uniformly at random from $\mathcal{H}$ can be done easily by
choosing two random values $k_1\in \{1,\ldots,u-1\}$ and $k_2\in
\{0,\ldots,u-1\}$.  Thus, we have an implementation of the set ADT with
$O(1)$ expected time per operation.

 
\subsection{Static Perfect Hashing}
\index{perfect hashing!static}
\index{hashing!static perfect}
\index{static perfect hashing}
\label{hash:sec:fks}

The result of Carter and Wegman on universal hashing is very strong,
and from a practical point of view, it is probably the strongest
result most people will ever need.  The only thing that could be
improved about their result is to make it deterministic, so that the
running times of all operations are $O(1)$ \emph{worst-case}.
Unfortunately, this is not possible, as shown by Dietzfelbinger \etal\
\cite{dkmm94}.  

Since there is no hope of getting $O(1)$ worst-case time for all three
set ADT operations, the next best thing would be to have searches that
take $O(1)$ worst-case time.  In this section we describe the method
of Fredman, Koml\'os and Szemer\'edi \cite{fks84}.  This is a static
data structure that takes as input a set $S\subseteq U$ and builds a
data structure of size $O(n)$ that can test if an element $x$ is in
$S$ in $O(1)$ worst-case time.  Like the universal hash functions from
the previous section, this method also requires that $u$ be a prime
number. This scheme uses hash functions of the form
\[
  h_{k,m}(x) = (kx\bmod u))\bmod m \enspace .\footnote{Actually, it
turns out that any universal hash function also works in the FKS
scheme \cite[Section~11.5]{clrs01}.}
\]

Let $B_{k,m}(S,i)$ be the number of elements $x\in S$ such that
$h_{k,m}(x)=i$, i.e., the number of elements of $S$ that have hash
value $i$ when using the hash function $h_{k,m}$.  The function
$B_{k,m}$ gives complete information about the distribution of hash
values of $S$.  The main lemma used by Fredman \etal\ is that, if we
choose $k\in U$ uniformly at random then
\begin{equation}
\E\left[\sum_{i=0}^{m-1} {B_{k,m}(S,i)\choose 2}\right] < \frac{n^2}{m}  
  \enspace . \label{hash:eq:fks-main}
\end{equation}
There are two important special cases of this result.

In the \emph{sparse case} we take $m=n^2/\alpha$, for some constant
$0<\alpha<1$.  In this case, the expectation in
(\ref{hash:eq:fks-main}) is less than $\alpha$.  Therefore, by
Markov's inequality, the probability that this sum is greater than or
equal to 1 is at most $\alpha$.  But, since this sum is a non-negative
integer, then with probability at least $1-\alpha$ it must be equal to
0.  In other words, with probability at least $1-\alpha$,
$B_{k,m}(S,i)\le 1$ for all $0\le i\le m-1$, i.e., the hash function
$h_{k,m}$ is perfect for $S$.  Of course this implies that we can find
a perfect hash function very quickly by trying a small number of
random elements $k\in U$ and testing if they result in perfect hash
functions. (The expected number of elements that we will have to try
is only $1/(1-\alpha)$.) Thus, if we are willing to use quadratic
space then we can perform searches in $O(1)$ worst-case time.

In the \emph{dense case} we assume that $m$ is close to $n$ and
discover that, for many values of $k$, the hash values are distributed
fairly evenly among the set $1,\ldots,m$.  More precisely, if we use a
table of size $m=n$, then 
\[
   \E \left[ \sum_{i=0}^{m-1} B_{k,m}(S,i)^2 \right] \le 3n \enspace .
      \label{hash:eq:fks-space-a}
\]
By Markov's inequality this means that
\begin{equation}
   \Pr\left\{ \sum_{i=0}^{m-1} B_{k,m}(S,i)^2  \le 3n/\alpha \right\}
\ge
	1-\alpha \enspace .
      \label{hash:eq:fks-space}
\end{equation}
Again, we can quickly find a value of $k$ satisfying
(\ref{hash:eq:fks-space}) by testing a few randomly chosen values of
$k$.

These two properties are enough to build a two-level data structure
that uses linear space and executes searches in worst-case constant
time.  We call the following data structure the FKS-$\alpha$ data
structure, after its inventors Fredman, Koml\'os and Szemer\'edi.  At
the top level, the data structure consists of an array
$A[0],\ldots,A[m-1]$ where $m=n$.  The elements of this array are
pointers to other arrays $A_0,\ldots,A_{m-1}$, respectively.  To
decide what will be stored in these other arrays, we build a hash
function $h_{k,m}$ that satisfies the conditions of
(\ref{hash:eq:fks-space}).  This gives us the top-level hash function
$h_{k,m}(x)=(kx\bmod u)\bmod m$. Each element $x\in S$ gets stored in
the array pointed to by $A[h_{k,m}(x)]$.

What remains is to describe how we use the arrays
$A_0,\ldots,A_{m-1}$.  Let $S_i$ denote the set of elements $x\in S$
such that $h_{k,m}(s)=i$.  The elements of $S_i$ will be stored in
$A_i$.  The size of $S_i$ is $n_i=B_{k,m}(S,i)$.  To store the
elements of $S_i$ we set the size of $A_i$ to
$m_i=n_i{}^2/\alpha=B_{k,n}(S,i)^2/\alpha$.  Observe that, by
(\ref{hash:eq:fks-space}), all the $A_i$'s take up a total space of
$O(n)$, i.e., $\sum_{i=0}^{m-1} m_i = O(n)$.  Furthermore, by trying a
few randomly selected integers we can quickly find a value $k_i$ such
that the hash function $h_{k_i,m_i}$ is perfect for $S_i$.  Therefore,
we store the element $x\in S_i$ at position $A_i[h_{k_i,m_i}(x)]$ and
$x$ is the unique element stored at that location.  With this scheme
we can search for any value $x\in U$ by computing two hash values
$i=h_{k,m}(x)$ and $j=h_{k_i,m_i}(x)$ and checking if $x$ is stored in
$A_i[j]$.

Building the array $A$ and computing the values of
$n_0,\ldots,n_{m-1}$ takes $O(n)$ expected time since for a given
value $k$ we can easily do this in $O(n)$ time and the expected number
of values of $k$ that we must try before finding one that satisfies
(\ref{hash:eq:fks-space}) is $O(1)$.  Similarly, building each subarray
$A_i$ takes $O(n_i{}^2)$ expected time, resulting in an overall expected
running time of $O(n)$.  Thus, for any constant $0<\alpha<1$, an
FKS-$\alpha$ data structure can be constructed in $O(n)$ expected time
and this data structure can execute a search for any $x\in U$ in
$O(1)$ worst-case time.

\subsection{Dynamic Perfect Hashing}
\label{hash:sec:dphash}
\index{hashing!dynamic perfect}
\index{dynamic perfect hashing}
\index{perfect hashing!dynamic}
The FKS-$\alpha$ data structure is nice in that it allows for searches
in $O(1)$ time, in the worst case.  Unfortunately, it is only static;
it does not support insertions or deletions of elements.  In this
section we describe a result of Dietzfelbinger \etal\ \cite{dkmm94}
that shows how the FKS-$\alpha$ data structure can be made dynamic
with some judicious use of partial rebuilding
(Section~\ref{external:12:Andersson-trees-partial-rebuilding}).

The main idea behind the scheme is simple: be lazy at both the upper
and lower levels of the FKS-$\alpha$ data structure.  That is, rebuild
parts of the data structure only when things go wrong.  At the top
level, we relax the condition that the size $m$ of the upper array $A$
is exactly $n$ and allow $A$ to have size anywhere between $n$ and
$2n$.  Similarly, at the lower level we allow the array $A_i$ to have
a size $m_i$ anywhere between $n_i{}^2/\alpha$ and $2n_i{}^2/\alpha$.

Periodically, we will perform a \emph{global rebuilding}\index{global
rebuilding} operation in
which we remove all $n$ elements from the hash table.  Some elements
which have previously been marked as deleted will be discarded,
thereby reducing the value of $n$.  We put the remaining elements in a
list, and recompute a whole new FKS-$(\alpha/2)$ data structure for
the elements in the list.  This data structure is identical to the
standard FKS-$(\alpha/2)$ data structure except that, at the top level
we use an array of size $m=2n$.

Searching in this data structure is exactly the same as for the static
data structure.  To search for an element $x$ we compute
$i=h_{k,m}(x)$ and $j=h_{k_i,m_i}(x)$ and look for $x$ at location
$A_i[j]$.   Thus, searches take $O(1)$ worst-case time.

Deleting in this data structure is done in the laziest manner
possible.  To delete an element we only search for it and then mark it
as deleted.  We will use the convention that this type of deletion
does not change the value of $n$ since it does not change the number
of elements actually stored in the data structure.  While doing this,
we also keep track of the number of elements that are marked as
deleted.  When this number exceeds $n/2$ we perform a global
rebuilding operation.  The global rebuilding operation takes $O(n)$
expected time, but only occurs during one out of every $n/2$
deletions.  Therefore, the amortized cost of this operation is $O(1)$
per deletion.

The most complicated part of the data structure is the insertion
algorithm and its analysis.  To insert a key $x$ we know, because of
how the search algorithm works, that we must ultimately store $x$ at
location $A_i[j]$ where $i=h_{k,m}(x)$ and $j=h_{k_i,m_i}(x)$.
However, several things can go wrong during the insertion of $x$:

\begin{enumerate} 
\item The value of $n$ increases by 1, so it may be that $n$ now
exceeds $m$.  In this case we perform a global rebuilding
operation and we are done.

\item We compute $i=h_{k,m}(x)$ and discover that $\sum_{i=0}^{m-1}
n_i{}^2 > 3n/\alpha$.  In this case, the hash function
$h_{k,m}$ used at the top level is no longer any good since it is
producing an overall hash table that is too large.  In this case we
perform a global rebuilding operation and we are done.

\item We compute $i=h_{k,m}(x)$ and discover that, since the value of
$n_i$ just increased by one, $n_i{}^2/\alpha > m_i$. In this case, the
array $A_i$ is too small to guarantee that we can quickly find a
perfect hash function.  To handle this, we copy the elements of $A_i$
into a list $L$ and allocate a new array $A_i$ with the new size
$m_i=2n_i{}^2/\alpha$.  We then find a new value $k_i$ such that
$h_{k_i,m_i}$ is a perfect hash function for the elements of $L$ and
we are done.

\item The array location $A_i[j]$ is already occupied by some other
element $y$.  But in this case, we know that $A_i$ is large enough to
hold all the elements (otherwise we would already be done after
Case~3), but the value $k_i$ being used in the hash function
$h_{k_i,m_i}$ is the wrong one since it doesn't give a perfect hash
function for $S_i$.  Therefore we simply try new values for $k_i$
until we find a find a value $k_i$ that yields a perfect hash function
and we are done.  
\end{enumerate}

If none of the preceding 4 cases occurs then we can simply place $x$
at location $A_i[j]$ and we are done.

Handling Case~1 takes $O(n)$ expected time since it involves a global
rebuild of the entire data structure.  However, Case~1 only happens
during one out of every $\Theta(n)$ insertions, so the amortized cost
of all occurrences of Case~1 is only $O(1)$ per insertion.

Handling Case~2 also takes $O(n)$ expected time.  The question is: How
often does Case~2 occur?  To answer this question, consider the
\emph{phase}\index{phase} that occurs between two consecutive
occurrences of Case~1.  During this phase, the data structure holds at
most $m$ distinct elements.  Call this set of elements $S$.  With
probability at least $(1-\alpha)$ the hash function $h_{k,m}$ selected
at the beginning of the phase satisfies (\ref{hash:eq:fks-space}) so
that Case~2 never occurs during the phase.  Similarly, the probability
that Case~2 occurs exactly once during the phase is at most
$\alpha(1-\alpha)$.  In general, the probability that Case~2 occurs
exactly $i$ times during a phase is at most $\alpha^i(1-\alpha)$.  Thus, the
expected cost of handling all occurrences of Case~2 during the entire
phase is at most 
\[ \sum_{i=0}^\infty \alpha^i(1-\alpha)i\times O(n) =
	O(n) \enspace .  
\] 
But since a phase involves $\Theta(n)$ insertions this means that the
amortized expected cost of handling Case~2 is $O(1)$ per insertion.

Next we analyze the total cost of handling Case~3.  Define a
\emph{subphase}\index{subphase} as the period of time between two
global rebuilding operations triggered either as a result of a
deletion, Case~1 or Case~2.   We will show that the total cost of
handling all occurrences of Case~3 during a subphase is $O(n)$ and
since a subphase takes $\Theta(n)$ time anyway this does not
contribute to the cost of a subphase by more than a constant factor.
When Case~3 occurs at the array $A_i$ it takes $O(m_i)$ time.
However, while handling Case~3, $m_i$ increases by a constant factor,
so the total cost of handling Case~3 for $A_i$ is dominated by the
value of $m_i$ at the end of the subphase.  But we maintain the
invariant that $\sum_{i=0}^{m-1} m_i = O(n)$ during the entire
subphase.  Thus, handling all occurrences of Case~3 during a subphase
only requires $O(n)$ time. 

Finally, we consider the cost of handling Case~4.  For a particular
array $A_i$, consider the \emph{subsubphase}\index{subsubphase}
between which two occurrences of Case~3 cause $A_i$ to be rebuilt or a
global rebuilding operation takes place.  During this subsubphase the
number of distinct elements that occupy $A_i$ is at most
$\alpha\sqrt{m_i}$.  Therefore, with probability at least $1-\alpha$
any randomly chosen value of $k_i\in U$ is a perfect hash function for
this set.   Just as in the analysis of Case~2, this implies that the
expected cost of handling all occurrences of Case~3 at $A_i$ during a
subsubphase is only $O(m_i)$.  Since a subsubphase ends with
rebuilding all of $A_i$ or a global rebuilding, at a cost of
$\Omega(m_i)$ all the occurrences of Case~4 during a subsubphase do not
contribute to the expected cost of the subsubphase by more than a
constant factor.

To summarize, we have shown that the expected cost of handling all
occurrences of Case~4 is only  a constant factor times the cost of
handling all occurrences of Case~3.  The cost of handling all
occurrences of Case~3 is no more than a constant factor times the
expected cost of all global rebuilds.  The cost of handling all the
global rebuilds that occur as a result of Case~2 is no more than a
constant factor times the cost of handling all occurrences of global
rebuilds that occur as a consequence of Case~1.  And finally, the cost
of all global rebuilds that occur as a result of Case~1 or of
deletions is $O(n)$ for a sequence of $n$ update operations.
Therefore, the total expected cost of $n$ update operation is $O(n)$.


\section{Random Probing}
\label{hash:sec:uniform}
\index{random probing}
\index{probing!random}

Next we consider hash table implementations under the random probing
assumption: Each element $x$ stored in the hash table comes with a
random sequence $x_0,x_1,x_2,\ldots$ where each of the $x_i$ is
independently and uniformly distributed in
$\{1,\ldots,m\}$.\footnote{A variant of the random probing assumption,
referred to as the \emph{uniform hashing}\index{uniform
hashing}\index{hashing!uniform} assumption, assumes that
$x_0,\ldots,x_{m-1}$ is a random permutation of $0,\ldots,m-1$.}  We
begin with a discussion of the two basic paradigms: hashing with
chaining and open addressing.  Both these paradigms attempt to store
the key $x$ at array position $A[x_0]$.  The difference between these
two algorithms is their \emph{collision resolution
strategy}\index{collision resolution strategy}, i.e., what the
algorithms do when a user inserts the key value $x$ but array position
$A[x_0]$ already contains some other key.

\subsection{Hashing with Chaining}
\index{hashing!with chaining}

In \emph{hashing with chaining}, a collision is resolved by allowing
more than one element to live at each position in the table.  Each
entry in the array $A$ is a pointer to the head of a linked list.  To
insert the value $x$, we simply append it to the list $A[x_0]$.  To
search for the element $x$, we perform a linear search in the list
$A[x_0]$. To delete the element $x$, we search for $x$ in the list
$A[x_0]$ and splice it out.

It is clear that insertions take $O(1)$ time, even in the worst case.
For searching and deletion, the running time is proportional to a
constant plus the length of the list stored at $A[x_0]$.  Notice that
each of the at most $n$ elements not equal to $x$ is stored in
$A[x_0]$ with probability $1/m$, so the expected length of $A[x_0]$ is
either $\alpha=n/m$ (if $x$ is not contained in the table) or
$1+(n-1)/m$ (if $x$ is contained in the table). Thus, the expected
cost of searching for or deleting an element is $O(1+\alpha)$.

The above analysis shows us that hashing with chaining supports the
three set ADT operations in $O(1)$ expected time per operation, as
long as the occupancy, $\alpha$, is a constant. It is worth noting
that this does not require that the value of $\alpha$ be less than 1. 

If we would like more detailed information about the cost of searching, we
might also ask about the \emph{worst-case search time} defined as
\[
   W = \max\{\mbox{length of the list stored at $A[i]$} : 0\le i\le m-1 \} 
	\enspace .
\]
It is very easy to prove something quite strong about $W$ using only
the fact that the length of each list $A[i]$ is a
$\mathrm{binomial}(n,1/m)$ random variable.  Using Chernoff's bounds
on the tail of the binomial distribution \cite{c52}, this
immediately implies that   
\[
   \Pr\{\mbox{length of $A[i]$} \ge \alpha c\ln n\} \le n^{-\Omega(c)} \enspace .
\]
Combining this with Boole's inequality ($\Pr\{\mbox{$A$ or $B$}\} \le
\Pr\{A\}+\Pr\{B\}$) we obtain
\[
   \Pr\{W \ge \alpha c\ln n\} \le n\times n^{-\Omega(c)} = n^{-\Omega(c)} \enspace .
\]

Thus, with very high probability, the worst-case search time is
logarithmic in $n$.  This also implies that $\E[W]=O(\log n)$.  The
distribution of $W$ has been carefully studied and it is known that,
\emph{with high probability}\index{high probability}, i.e., with
probability $1-o(1)$, $W=(1+o(1))\ln n/\ln\ln n$
\cite{jk77,ksc78}.\footnote{Here, and throughout this chapter, if an
asymptotic notation does not contain a variable then the variable that
tends to infinity is implicitly $n$.  Thus, for example, $o(1)$ is the
set of non-negative functions of $n$ that tend to 0 as
$n\rightarrow\infty$.} Gonnet has proven a more accurate result that
$W=\Gamma^{-1}(n) -3/2 + o(1)$ with high probability. Devroye
\cite{d85} shows that similar results hold even when the distribution
of $x_0$ is not uniform.

\subsection{Hashing with Open Addressing}
\index{open addressing}
\index{hashing!with open addressing}

\label{hash:sec:open-addressing}

\emph{Hashing with open addressing} differs from hashing with chaining in
that each table position $A[i]$ is allowed to store only one value.
When a collision occurs at table position $i$, one of the two elements
involved in the collision must move on to the next element in its
probe sequence.  In order to implement this efficiently and correctly
we require a method of marking elements as deleted.  This method could
be an auxiliary array that contains one bit for each element of $A$,
but usually the same result can be achieved by using a special key
value $\mathbf{del}$ that does not correspond to any valid key.

To search for an element $x$ in the hash table we look for $x$ at
positions $A[x_0]$, $A[x_1]$, $A[x_2]$, and so on until we either
(1)~find $x$, in which case we are done or (2)~find an empty table
position $A[x_i]$ that is not marked as deleted, in which case we can
be sure that $x$ is not stored in the table (otherwise it would be
stored at position $x_i$).  To delete an element $x$ from the hash
table we first search for $x$.  If we find $x$ at table location
$A[x_i]$ we then simply mark $A[x_i]$ as deleted.  To insert a value
$x$ into the hash table we examine table positions $A[x_0]$, $A[x_1]$,
$A[x_2]$, and so on until we find a table position $A[x_i]$ that is
either empty or marked as deleted and we store the value $x$ in
$A[x_i]$.

Consider the cost of inserting an element $x$ using this method.  Let
$i_x$ denote the smallest value $i$ such that $x_{i_x}$ is either empty or
marked as deleted when we insert $x$.  Thus, the cost of inserting $x$
is a constant plus $i_x$.  The probability that the table position
$x_0$ is occupied is at most $\alpha$ so, with probability at least
$1-\alpha$, $i_x=0$.  Using the same reasoning, the probability
that we store $x$ at position $x_i$ is at most 
\begin{equation}
   \Pr\{i_x = i\} \le \alpha^{i}(1-\alpha)  \label{hash:eq:pro}
\end{equation}
since the table locations $x_0,\ldots,x_{i-1}$ must be occupied,
the table location $x_i$ must not be occupied and the $x_i$ are
independent.  Thus, the expected number of steps taken by the
insertion algorithm is 
\[
   \sum_{i=1}^\infty i\Pr\{i_x = i\} = (1-\alpha)\sum_{i=1}^\infty i\alpha^{i-1}  = 1/(1-\alpha)
\]
for any constant $0<\alpha< 1$.  The cost of searching for $x$ and
deleting $x$ are both proportional to the cost of inserting $x$, so
the expected cost of each of these operations is
$O(1/(1-\alpha))$.\footnote{Note that the expected cost of searching
for or deleting an
element $x$ is proportional to the value of $\alpha$ \emph{at the time
$x$ was inserted.}  If many deletions have taken place, this may be
quite different than the current value of $\alpha$.}   

We should compare this with the cost of hashing with chaining.  In
hashing with chaining,the occupancy $\alpha$ has very little effect on
the cost of operations.  Indeed, any constant $\alpha$, even greater
than $1$ results in $O(1)$ time per operation.  In contrast, open
addressing is very dependent on the value of $\alpha$.  If we take
$\alpha>1$ then the expected cost of insertion using open addressing
is infinite since the insertion algorithm never finds an empty table
position.  Of course, the advantage of hashing with chaining is that
it does not require lists at each of the $A[i]$.  Therefore, the
overhead of list pointers is saved and this extra space can be used
instead to maintain the invariant that the occupancy $\alpha$ is a
constant strictly less than 1.

Next we consider the worst case search time of hashing with open
addressing.  That is, we study the value $W=\max\{i_x: \mbox{$x$ is
stored in the table at location $i_x$}\}$.  Using (\ref{hash:eq:pro}) and Boole's
inequality it follows almost immediately that 
\[
  \Pr\{W > c\log n\} \le n^{-\Omega(c)} .
\]

Thus, with very high probability, $W$, the worst case search time, is
$O(\log n)$.  Tighter bounds on $W$ are known when the probe sequences
$x_0,\ldots,x_{m-1}$ are random permutations of $0,\ldots,m-1$.  In this
case, Gonnet shows that
\[
  \E[W]=\log_{1/\alpha} n - \log_{1/\alpha}(\log_{1/\alpha} n) + O(1)
\]
\cite{g81}. 

Open addressing under the random probing assumption has many nice
theoretical properties and is easy to analyze.  Unfortunately, it is
often criticized as being an unrealistic model because it requires a
long random sequences $x_0,x_1,x_2,\ldots$ for each element $x$ that
is to be stored or searched for.  Several variants of open addressing
discussed in the next few sections try to overcome this problem by
using only a few random values. 


\subsection{Linear Probing}
\index{linear probing}
\index{hashing!with linear probing}
 
\emph{Linear probing} is a variant of open addressing that requires less
randomness.  To obtain the probe sequence $x_0,x_1,x_2,\ldots$ we
start with a random element $x_0\in \{0,\ldots,m-1\}$.  The element
$x_i$, $i>0$ is given by $x_i=(i+x_0)\bmod m$.  That is, one first
tries to find $x$ at location $x_0$ and if that fails then one looks
at $(x_0+1)\bmod m$, $(x_0+2)\bmod m$ and so on.  

The performance of linear probing is discussed by Knuth \cite{k97} who
shows that the expected number of probes performed during an
unsuccessful search is at most 
\[
	(1+1/(1-\alpha)^2)/2
\] 
and the expected
number of probes performed during a successful search is at most
\[
	(1+1/(1-\alpha))/2 \enspace . 
\] 
This is not quite as good as for standard hashing with open
addressing, especially in the unsuccessful case.

Linear probing suffers from the problem of \emph{primary
clustering}\index{primary clustering}\index{clustering!primary}.  If
$j$ consecutive array entries are occupied then a newly inserted
element will have probability $j/m$ of hashing to one of these
entries. This results in $j+1$ consecutive array entries being
occupied and increases the probability (to $(j+1)/m$) of another newly
inserted element landing in this cluster.  Thus, large clusters of
consecutive elements have a tendency to grow larger.

\subsection{Quadratic Probing}
\index{quadratic probing}
\index{hashing!with quadratic probing}

\emph{Quadratic probing} is similar to linear probing;  an element $x$
determines its entire probe sequence based on a single random choice,
$x_0$.  Quadratic probing uses the probe sequence
$x_0,(x_0+k_1+k_2)\bmod m, (x_0+2k_1+2^2k_2)\bmod m,\ldots$.  In
general, the $i$th element in the probe sequence is
$x_i=(x_0+ik_1+i^2k_2)\bmod m$.  Thus, the final location of an
element depends quadratically on how many steps were required to
insert it.  This method seems to work much better in practice than
linear probing, but requires a careful choice of $m$, $k_1$ and $k_2$
so that the probe sequence contains every element of $\{0,\ldots,m-1\}$.

The improved performance of quadratic probing is due to the fact that
if there are two elements $x$ and $y$ such that $x_i=y_j$ then it is
not necessarily true (as it is with linear probing) that
$x_{i+1}=y_{j+1}$.  However, if $x_0=y_0$ then $x$ and $y$ will have
exactly the same probe sequence.  This lesser phenomenon is called
\emph{secondary clustering}\index{secondary
clustering}\index{clustering!secondary}.  Note that this secondary
clustering phenomenon implies that neither linear nor quadratic
probing can hope to perform any better than hashing with chaining.
This is because all the elements that have the same initial hash $x_0$
are contained in an implicit chain.  In the case of linear probing,
this chain is defined by the sequence $x_0,x_0+1,x_0+2,\ldots$ while
for quadratic probing it is defined by the sequence
$x_0,x_0+k_1+k_2,x_0+2k_1+4k_2,\ldots$

\subsection{Double Hashing}\label{hash:sec:double}
\index{hashing!double}
\index{double hashing}

\emph{Double hashing} is another method of open addressing that uses two hash
values $x_0$ and $x_1$.  Here $x_0$ is in the set $\{0,\ldots,m-1\}$
and $x_1$ is in the subset of $\{1,\ldots,m-1\}$ that is relatively
prime to $m$.  With double hashing, the probe sequence for element $x$
becomes $x_0,(x_0+x_1)\bmod m, (x_0+2x_1)\bmod m,\ldots$.  In general,
$x_i=(x_0+ix_1)\bmod m$, for $i>0$.  The expected number of probes
required by double hashing seems difficult to determine exactly.
Guibas has proven that, asymptotically, and for occupancy $\alpha\le
.31$, the performance of double hashing is asymptotically equivalent
to that of uniform hashing.  Empirically, the performance of double
hashing matches that of open addressing with random probing
regardless of the occupancy $\alpha$ \cite{k97}.

\subsection{Brent's Method}\label{hash:sec:brent}
\index{Brent's method}
\index{hashing!Brent's method}

\emph{Brent's method} \cite{b73} is a heuristic that attempts to
minimize the average time for a successful search in a hash table with
open addressing.  Although originally described in the context of
double hashing (Section~\ref{hash:sec:double}) Brent's method applies
to any open addressing scheme.  The \emph{age}\index{age} of an element $x$
stored in an open addressing hash table is the minimum value $i$ such
that $x$ is stored at $A[x_i]$.  In other words, the age is one less
than the number of locations we will probe when searching for $x$. 

Brent's method attempts to minimize the total age of all elements in
the hash table.  To insert the element $x$ we proceed as follows:  We
find the smallest value $i$ such that $A[x_i]$ is empty; this is where
standard open-addressing would insert $x$.  Consider the element $y$
stored at location $A[x_{i-2}]$.  This element is stored there because
$y_j=x_{i-2}$, for some $j\ge 0$.  We check if the array location
$A[y_{j+1}]$ is empty and, if so, we move $y$ to location $A[y_{j+1}]$
and store $x$ at location $A[x_{i-2}]$.  Note that, compared to
standard open addressing, this decreases the total age by 1.  In
general, Brent's method checks, for each $2\le k\le i$ the array entry
$A[x_{i-k}]$ to see if the element $y$ stored there can be moved to
any of $A[y_{j+1}],A[y_{j+2}],\ldots,A[y_{j+k-1}]$ to make room for
$x$.  If so, this represents a decrease in the total age of all
elements in the table and is performed.

Although Brent's method seems to work well in practice, it is
difficult to analyze theoretically.  Some theoretical analysis of
Brent's method applied to double hashing is given by Gonnet and Munro
\cite{gm79}.  Lyon \cite{lyon78}, Munro and Celis \cite{mc99} and
Poblete \cite{p77} describe some variants of Brent's method.


\subsection{Multiple-Choice Hashing}
\label{hash:sec:multiplechoice}
\index{hashing!multiple-choice}
\index{multiple-choice hashing}

It is worth stepping back at this point and revisiting the comparison
between hash tables and binary search trees.  For balanced binary
search trees, the average cost of searching for an element is $O(\log
n)$.  Indeed, it easy to see that for at least $n/2$ of the elements,
the cost of searching for those elements is $\Omega(\log n)$.  In
comparison, for both the random probing schemes discussed so far, the
expected cost of search for an element is $O(1)$.  However, there are
a handful of elements whose search cost is $\Theta(\log n/\log \log
n)$ or $\Theta(\log n)$ depending on whether hashing with chaining or
open addressing is used, respectively.  Thus there is an inversion:
Most operations on a binary search tree cost $\Theta(\log n)$ but a
few elements (close to the root) can be accessed in $O(1)$ time.  Most
operations on a hash table take $O(1)$ time but a few elements (in long
chains or with long probe sequences) require $\Theta(\log n/\log\log
n)$ or $\Theta(\log n)$ time to access.  In the next few sections we
consider variations on hashing with chaining and open addressing that
attempt to reduce the worst-case search time $W$.

\emph{Multiple-choice hashing} is hashing with chaining in which,
during insertion, the element $x$ has the choice of $d\ge 2$ different
lists in which it can be stored.  In particular, when we insert $x$ we
look at the lengths of the lists pointed to by
$A[x_0],\ldots,A[x_{d-1}]$ and append $x$ to $A[x_i]$, $0\le i< d$
such that the length of the list pointed to by $A[x_i]$ is minimum.
When searching for $x$, we search for $x$ in each of the lists
$A[x_0],\ldots,A[x_{d-1}]$ \emph{in parallel}.  That is, we look at
the first elements of each list, then the second elements of each
list, and so on until we find $x$.  As before, to delete $x$  we first
search for it and then delete it from whichever list we find it in.  

It is easy to see that the expected cost of searching for an element
$x$ is $O(d)$ since the expected length of each the $d$ lists is
$O(1)$.  More interestingly, the worst case search time is bounded by
$O(dW)$ where $W$ is the length of the longest list.  Azar \etal\
\cite{abku99} show that 
\begin{equation}
    \E[W]=\frac{\ln\ln n}{\ln d} + O(1) \enspace .
	\label{hash:eq:multiple-choice} 
\end{equation}
Thus, the expected worst case search time for multiple-choice hashing
is $O(\log\log n)$ for any constant $d\ge 2$.


\subsection{Asymmetric Hashing}
\index{hashing!asymmetric}
\index{asymmetric hashing}

\emph{Asymmetric hashing} is a variant of multiple-choice hashing in
which the hash table is split into $d$ blocks, each of size $n/d$.
(Assume, for simplicity, that $n$ is a multiple of $d$.) The probe
value $x_i$, $0\le i< d$ is drawn uniformly from
$\{in/d,\ldots,(i+1)n/d-1\}$.   As with multiple-choice hashing, to
insert $x$ the algorithm examines the lengths of the lists $A[x_0],
A[x_1],\ldots,A[x_{d-1}]$ and appends $x$ to the shortest of these
lists.  In the case of ties, it appends $x$ to the list with smallest
index.  Searching and deletion are done exactly as in multiple-choice
hashing.

V\"ocking \cite{v99} shows that, with asymmetric hashing the expected
length of the longest list is 
\[
   \E[W] \le \frac{\ln\ln n}{d\ln\phi_d} +O(1) \enspace .
\] 
The function $\phi_d$ is a generalization of the \emph{golden
ratio}\index{golden ratio},
so that $\phi_2=(1+\sqrt{5})/2$.  Note that this improves
significantly on standard multiple-choice hashing (\ref{hash:eq:multiple-choice}) for larger values of
$d$.

\subsection{LCFS Hashing}
\index{hashing!LCFS}
\index{LCFS hashing}

\emph{LCFS hashing} is a form of open addressing that changes the
collision resolution strategy.\footnote{Amble and Knuth \cite{ak74}
were the first to suggest that, with open addressing, any collision
resolution strategy could be used.} Reviewing the algorithm for
hashing with open addressing reveals that when two elements collide,
priority is given to the first element inserted into the hash table
and subsequent elements must move on.  Thus, hashing with open
addressing could also be referred to as \emph{FCFS (first-come
first-served) hashing}\index{FCFS hashing}\index{hashing!FCFS}.

With LCFS (last-come first-served) hashing, collision resolution is
done in exactly the opposite way.  When we insert an element $x$, we
always place it at location $x_0$.  If position $x_0$ is already
occupied by some element $y$ because $y_j=x_0$ then we place $y$ at
location $y_{j+1}$, possibly displacing some element $z$, and so on.

Poblete and Munro \cite{pm89} show that, after inserting $n$ elements
into an initially empty table, the expected worst case search time is
bounded above by
\[  \E[W] \le 
   1 
   + \Gamma^{-1}(\alpha n)
   \left( 1+\frac{\ln\ln(1/(1-\alpha))}{\ln \Gamma^{-1}(\alpha n)}
   + O\left(\frac{1}{\ln^2\Gamma^{-1}(\alpha n)} \right)\right)
\enspace ,
\]
where $\Gamma$ is the gamma function and 
\[
  \Gamma^{-1}(\alpha n) = \frac{\ln n}{\ln\ln n}
	\left(1+\frac{\ln\ln\ln n}{\ln\ln n} + 
	O\left(\frac{1}{\ln\ln n}\right)\right) \enspace .
\]
Historically, LCFS hashing is the first version of open addressing
that was shown to have an expected worst-case search time that is
$o(\log n)$.

\subsection{Robin-Hood Hashing}
\label{hash:sec:robinhood}
\index{hashing!Robin-Hood}
\index{Robin-Hood hashing}

Robin-Hood hashing \cite{c86,clm85,vp98} is a form of open addressing
that attempts to equalize the search times of elements by using a
fairer collision resolution strategy.  During insertion, if we are
trying to place element $x$ at position $x_i$ and there is already an
element $y$ stored at position $y_j=x_i$ then the ``younger'' of the
two elements must move on.  More precisely, if $i\le j$ then we will
try to insert $x$ at position $x_{i+1}$, $x_{i+2}$ and so on.
Otherwise, we will store $x$ at position $x_i$ and try to to insert
$y$ at positions $y_{j+1}$, $y_{j+2}$ and so on.

Devroye \etal\ \cite{dmv03} show that, after performing $n$ insertions on
an initially empty table of size $m=\alpha n$ using the Robin-Hood
insertion algorithm, the worst case search time has expected value
\[
	\E[W] = \Theta(\log\log n)
\]
and this bound is tight.
Thus, Robin-Hood hashing is a form of open addressing that
has doubly-logarithmic worst-case search time.  This makes it
competitive with the multiple-choice hashing method of
Section~\ref{hash:sec:multiplechoice}.

\subsection{Cuckoo Hashing}\label{hash:sec:cuckoo}
\index{hashing!cuckoo}
\index{cuckoo hashing}

Cuckoo hashing \cite{pr01} is a form of multiple choice hashing in
which each element $x$ lives in one of two tables $A$ or $B$, each of
size $m=n/\alpha$.  The element $x$ will either be stored at location
$A[x_A]$ or $B[x_B]$. There are no other options.  This makes
searching for $x$ an $O(1)$ time operation since we need only check
two array locations.

The insertion algorithm for cuckoo hashing proceeds as
follows:\footnote{The algorithm takes its name from the large but lazy
cuckoo bird which, rather than building its own nest, steals the nest
of another bird forcing the other bird to move on.} Store $x$ at
location $A[x_A]$. If $A[x_A]$ was previously occupied by some element
$y$ then store $y$ at location $B[y_B]$.  If $B[y_B]$ was previously
occupied by some element $z$ then store $z$ at location $A[z_A]$, and
so on.  This process ends when we place an element into a previously
empty table slot or when it has gone on for more than $c\log n$ steps.
In the former case, the insertion of $x$ completes successfully.  In
the latter case the insertion is considered a failure, and the entire
hash table is reconstructed from scratch using a new probe sequence
for each element in the table.  That is, if this reconstruction
process has happened $i$ times then the two hash values we use for an
element $x$ are $x_A=x_{2i}$ and $x_B=x_{2i+1}$.

Pagh and Rodler \cite{pr01} (see also Devroye and Morin \cite{dm02})
show that, during the insertion of $n$ elements, the probability of
requiring a reconstruction is $O(1/n)$. This, combined with the fact
that the expected insertion time is $O(1)$ shows that the expected
cost of $n$ insertions in a Cuckoo hashing table is $O(n)$.  Thus,
Cuckoo hashing offers a somewhat simpler alternative to the dynamic
perfect hashing algorithms of Section~\ref{hash:sec:dphash}.

\section{Historical Notes}

In this section we present some of the history of hash tables. The
idea of hashing seems to have been discovered simultaneously by two
groups of researchers. Knuth \cite{k97} cites an internal IBM
memorandum in January 1953 by H. P. Luhn that suggested the use of
hashing with chaining.  Building on Luhn's work, A. D. Linh suggested
a method of open addressing that assigns the probe sequence
$x_0,\floor{x_0/10},\floor{x_0/100},\floor{x_0/1000},\ldots$ to the
element $x$.

At approximately the same time, another group of researchers at IBM:
G. M. Amdahl, E.~M.~Boehme, N. Rochester and A. L. Samuel implemented
hashing in an assembly program for the IBM 701 computer.  Amdahl is
credited with the idea of open addressing with linear probing.

The first published work on hash tables was by A. I. Dumey \cite{d56},
who described hashing with chaining and discussed the idea of using
remainder modulo a prime as a hash function.  Ershov \cite{e58},
working in Russia and independently of Amdahl, described open
addressing with linear probing.

Peterson \cite{p57} wrote the first major article discussing the
problem of searching in large files and coined the term ``open
addressing.''  Buchholz \cite{b63} also gave a survey of the searching
problem with a very good discussion of hashing techniques at the time.
Theoretical analyses of linear probing were first presented by Konheim
and Weiss \cite{kw66} and Podderjugin.  Another, very influential,
survey of hashing was given by Morris \cite{m68}.  Morris' survey is
the first published use of the word ``hashing'' although it was
already in common use by practitioners at that time.



\section{Other Developments}

The study of hash tables has a long history and many researchers have
proposed methods of implementing hash tables.  Because of this, the
current chapter is necessarily incomplete. (At the time of writing,
the hash.bib bibliography on hashing contains over 800 entries.) We
have summarized only a handful of the major results on hash tables in
internal memory.  In this section we provide a few references to the
literature for some of the other results.   For more information on
hashing, Knuth \cite{k97}, Vitter and Flajolet \cite{vf90}, Vitter
and Chen \cite{vc87}, and Gonnet and Baeza-Yates \cite{gb-y91}
% and Flajolet \etal \cite{X} 
are useful references.

Brent's method (Section~\ref{hash:sec:brent}) is a collision
resolution strategy for open addressing that reduces the expected
search time for a successful search in a hash table with open
addressing.  Several other methods exist that either reduce the
expected or worst-case search time.  These include \emph{binary tree
hashing}\index{hashing!binary tree}\index{binary tree hashing}
\cite{m77, gm79}, \emph{optimal
hashing}\index{hashing!optimal}\index{optimal hashing}
\cite{p76,r78,gm79}, Robin-Hood hashing
(Section~\ref{hash:sec:robinhood}), and \emph{min-max
hashing}\index{hashing!min-max}\index{min-max hashing} \cite{g81,c86}.
One interesting method, due to Celis\index{Celis'
method}\index{hashing!Celis' method} \cite{c86}, applies to any open
addressing scheme.  The idea is to study the distribution of the ages
of elements in the hash table, i.e., the distribution give by 
\[
    D_i=\Pr\{\mbox{$x$ is stored at position $x_i$}\} 
\] 
and start searching for $x$ at the locations at which we are most
likely to find it, rather than searching the table positions
$x_0,x_1,x_2\ldots$ in order.

Perfect hash functions seem to have been first studied by Sprugnoli
\cite{s77} who gave some heuristic number theoretic constructions of
minimal perfect hash functions for small data sets.  Sprugnoli is
responsible for the terms ``perfect hash function'' and ``minimal
perfect hash function.''  A number of other researchers have presented
algorithms for discovering minimal and near-minimal perfect hash
functions.  Examples include Anderson and Anderson \cite{aa79},
Cichelli \cite{c80a,c80b}, Chang \cite{c84a,c84b}, Gori and Soda
\cite{gs89}, and Sager \cite{s85}.  Berman \etal\ \cite{bbdo86} and
K\"orner and Marton \cite{km88} discuss the theoretical limitations of
perfect hash functions.  A comprehensive, and recent, survey of
perfect hashing and minimal perfect hashing is given by Czech \etal\
\cite{chm97}.

Tarjan and Yao \cite{ty79} describe a set ADT implementation that
gives $O(\log u/\log n)$ worst-case access time.  It is obtained by
combining a trie (Section~\ref{external:32:tries}) of degree $n$ with a
compression scheme for arrays of size $n^2$ that contain only $n$
non-zero elements.  (The trie has $O(n)$ nodes each of which has $n$
pointers to children, but there are only a total of $O(n)$ children.)
Although their result is superseded by the results of Fredman \etal\
\cite{fks84} discussed in Section~\ref{hash:sec:fks}, they are the
first theoretical results on worst-case search time for hash tables.

Dynamic perfect hashing (Section~\ref{hash:sec:dphash}) and cuckoo
hashing (Section~\ref{hash:sec:cuckoo}) are methods of achieving
$O(1)$ worst case search time in a dynamic setting.  Several other
methods have been proposed  \cite{dm90,dgmp92,bm99}.
 
Yao \cite{y81} studies the \emph{membership problem}\index{membership
problem}.  Given a set $S\subseteq U$, devise a data structure that
can determine for any $x\in U$ whether $x$ is contained in $S$. Yao
shows how, under various conditions, this problem can be solved using
a very small number of memory accesses per query.  However, Yao's
algorithms sometimes derive the fact that an element $x$ is in $S$
without actually finding $x$.  Thus, they don't solve the set ADT
problem discussed at the beginning of this chapter since they can not
recover a pointer to $x$.

The ``power of two random choices,'' as used in multiple-choice
hashing, (Section~\ref{hash:sec:multiplechoice}) has many applications
in computer science.  Karp, Luby and Meyer auf der Heide
\cite{klm92,klm93} were the first to use this paradigm for simulating
PRAM computers on computers with fewer processors. The book chapter by
Mitzenmacher \etal\ \cite{mrs01} surveys results and applications of
this technique.

A number of table implementations have been proposed that are suitable
for managing hash tables in external memory.\index{hashing!external
memory}\index{external memory!hashing}  Here, the goal is to reduce
the number of disk blocks that must be accessed during an operation,
where a disk block can typically hold a large number of elements.
These schemes include \emph{linear
hashing}\index{hashing!linear}\index{linear hashing} \cite{l80},
\emph{dynamic hashing}\index{hashing!dynamic}\index{dynamic hashing}
\cite{l78}, \emph{virtual hashing}\index{virtual
hashing}\index{hashing!virtual} \cite{lit78}, \emph{extendible
hashing}\index{hashing!extendible}\index{extendible hashing}
\cite{fnps79},  \emph{cascade
hashing}\index{hashing!cascade}\index{cascade hashing} \cite{kz84},
and \emph{spiral storage}\index{spiral storage}\index{hashing!spiral
storage} \cite{m85}.  In terms of hashing, the main difference between
internal memory and external memory is that, in internal memory, an
array is allocated at a specific size and this can not be changed
later.  In contrast, an external memory file may be appended to or be
truncated to increase or decrease its size, respectively.  Thus, hash
table implementations for external memory can avoid the periodic
global rebuilding operations used in internal memory hash table
implementations.

\section*{Acknowledgement}

The author is supported by a grant from the Natural Sciences and
Engineering Research Council of Canada (NSERC).

%%\setcounter{\chapter}{10}
\chapter{Hash Tables}

\newcommand{\E}{\mathrm{E}}
\newcommand{\floor}[1]{\lfloor #1 \rfloor}
\newcommand{\ceil}[1]{\lceil #1 \rceil}

\newcommand{\etal}{et al.}

\begin{chapterauthors}
\chapterauthor{Pat Morin}{Carleton University}
\end{chapterauthors}

\index{hash table}

\section{Introduction}

A \emph{set abstract data type (set ADT)}\index{set ADT} is an
abstract data type that maintains a set $S$ under the following three
operations:

\begin{enumerate}

\item \textsc{Insert}$(x)$:  Add the key $x$ to the set.

\item \textsc{Delete}$(x)$: Remove the key $x$ from the set.

\item \textsc{Search}$(x)$: Determine if $x$ is contained in the set,
and if so, return a pointer to $x$. 

\end{enumerate}
One of the most practical and widely used methods of implementing the
set ADT is with \emph{hash tables}.  

Note that the three set ADT operations can easily be implemented to
run in $O(\log n)$ time per operation using balanced binary search
trees (See Chapter~\ref{external:12:BBST}).  If we assume that the
input data are integers in the set $U=\{0,\ldots,u-1\}$ then they can
even be implemented to run in $O(\log\log u)$ time using data
structures for integer searching
(Chapter~\ref{external:38?:int-searching}).  However, these data
structures actually do more than the three basic operations we
require.  In particular if we search for an element $x$ that is not
present in $S$ then these data structures can report the smallest item
in $S$ that is larger than $x$ (the \emph{successor} of $x$) and/or
the largest item in $S$ that is smaller than $x$ (the
\emph{predecessor} of $x$).  

Hash tables do away with this extra functionality of finding
predecessors and successors and only perform exact searches. If we
search for an element $x$ in a hash table and $x$ is not present then
the only information we obtain is that $x\notin S$. By dropping this
extra functionality hash tables can give better performance bounds.
Indeed, any reasonable hash table implementation performs each of the
three set ADT operations in $O(1)$ expected time.

The main idea behind all hash table implementations discussed in this
chapter is to store a set of $n=|S|$ elements in an array (the hash
table) $A$ of length $m\ge n$.  In doing this, we require a function
that maps any element $x$ to an array location.  This function is
called a \emph{hash function}\index{hash function} $h$ and the value
$h(x)$ is called the \emph{hash value} of $x$.  That is, the element
$x$ gets stored at the array location $A[h(x)]$.  The \emph{occupancy}
of a hash table is the ratio $\alpha=n/m$ of stored elements to the
length of $A$.  

The study of hash tables follows two very different lines. Many
implementations of hash tables are based on the \emph{integer universe
assumption}:\index{integer universe assumption} All elements stored in
the hash table come from the universe $U=\{0,\ldots,u-1\}$.  In this
case, the goal is to design a hash function $h:U\rightarrow
\{0,\ldots,m-1\}$ so that for each $i\in \{0,\ldots,m-1\}$, the number
of elements $x\in S$ such that $h(x)=i$ is as small as possible.
Ideally, the hash function $h$ would be such that each element of $S$
is mapped to a unique value in $\{0,\ldots,m-1\}$.  Most of the hash
functions designed under the integer universe assumption are
number-theoretic constructions. Several of these are described in
Section~\ref{hash:sec:integers}.

Historically, the integer universe assumption seems to have been
justified by the fact that any data item in a computer is represented
as a sequence of bits that can be interpreted as a binary number.
However, many complicated data items require a large (or variable)
number of bits to represent and this make $u$ the size of the universe
very large.  In many applications $u$ is much larger than the largest
integer that can fit into a single word of computer memory.  In this
case, the computations performed in number-theoretic hash functions
become inefficient.

This motivates the second major line of research into hash tables.
This research works is based on the \emph{random probing
assumption\emph{random probing assumption}}: Each element $x$ that is
inserted into a hash table is a black box that comes with an infinite
random \emph{probe sequence}\index{probe sequence}
$x_0,x_1,x_2,\ldots$ where each of the $x_i$ is independently and
uniformly distributed in $\{0,\ldots,m-1\}$.  Hash table
implementations based on the random probing assumption are described
in Section~\ref{hash:sec:uniform}.

Both the integer universe assumption and the random probing
assumption have their place in practice.  When there is an easily
computing mapping of data elements onto machine word sized integers
then hash tables for integer universes are the method of choice.  When
such a mapping is not so easy to compute (variable length strings are
an example) it might be better to use the bits of the input items to
build a good pseudorandom sequence and use this sequence as the probe
sequence for some random probing data structure. 

To guarantee good performance, many hash table implementations require
that the occupancy $\alpha$ be a constant strictly less than $1$.
Since the number of elements in a hash table changes over time, this
requires that the array $A$ be resized periodically.  This is easily
done, without increasing the amortized cost of hash table operations
by choosing three constants $0< \alpha_1 < \alpha_2 < \alpha_3 < 1$ so
that, whenever $n/m$ is not the interval $(\alpha_1,\alpha_3)$ the
array $A$ is resized so that its size is $n/\alpha_2$. A simple
amortization argument (Chapter~\ref{external:1:analysis}) shows that
the amortized cost of this resizing is $O(1)$ per update
(Insert/Delete) operation.


\section{Hash Tables for Integer Keys}
\label{hash:sec:integers}

In this section we consider hash tables under the integer universe
assumption, in which the key values $x$ come from the universe
$U=\{0,\ldots,u-1\}$.   A \emph{hash function}\index{hash function}
$h$ is a function whose domain is $U$ and whose range is the set
$\{0,\ldots,m-1\}$, $m\le u$.  A hash function $h$ is said to be a
\emph{perfect hash function}\index{hash
function!perfect}\index{perfect hash function} for a set $S\subseteq
U$ if, for every $x\in S$, $h(x)$ is unique.  A perfect hash function
$h$ for $S$ is \emph{minimal}\index{hash function!minimal
perfect}\index{minimal perfect hash function} if $m=|S|$, i.e., $h$ is
a bijection between $S$ and $\{0,\ldots,m-1\}$.  Obviously a minimal
perfect hash function for $S$ is desirable since it allows us to store
all the elements of $S$ in a single array of length $n$.
Unfortunately, perfect hash functions are rare, even for $m$ much
larger than $n$.  If each element of $S$ is mapped independently and
uniformly to a random element of $\{0,\ldots,m-1\}$ then the birthday
paradox (See, for example, Feller~\cite{f68}) states that, if $m$ is
much less than $n^2$ then there will almost surely exist two elements
of $S$ that have the same hash value.

We begin our discussion with two commonly used hashing schemes that
are heuristic in nature.  That is, we can not make any non-trivial
statements about the performance of these schemes when storing an
arbitrary set $S$.  We then discuss several schemes that have provably
good performance.

\subsection{Hashing by Division}
\index{hashing!by division}

In \emph{hashing by division}, we use the hash function
\[ h(x) = x\bmod m \enspace . \]
To use this hash function in a data structure, we maintain an array
$A[0],\ldots,A[m-1]$ where each element of this array is a pointer to
the head of a linked list (Section~\ref{external:2:linked-list}).  The
linked list $L_i$ pointed to by the array element $A[i]$ contains all
the elements $x$ such that $h(x)=i$.  This technique of maintaining an
array of lists is called \emph{hashing with
chaining}\index{hashing!with chaining}.

In such a hash table, inserting an element $x$ takes $O(1)$ time;  we
compute $i=h(x)$ and append (or prepend) $x$ to the list $L_i$.
However, searching for and/or deleting an element $x$ is not so easy.
We have to compute $i=h(x)$ and then traverse the list $L_i$ until we
either find $x$ or reach the end of the list.  The cost of this is
proportional to the length of $L_i$.  Obviously, if our set $S$
consists of the elements $0,m,2m,3m,\ldots,nm$ then all elements are
stored in the list $L_0$ and searches and deletions take linear time.

However, one hopes that such pathological cases do not occur in
practice. For example, if the elements of $S$ are uniformly and
independently distributed in $U$ and $u$ is a multiple of $m$ then the
expected size of any list $L_i$ is only $n/m$.  In this case, searches
and deletions take $O(1+\alpha)$ expected time.  To help avoid
pathological cases, the choice of $m$ is important.  In particular,
$m$ a power of 2 is usually avoided since, in a binary computer,
taking the remainder modulo a power of 2 means simply discarding some
high-order bits.  Taking $m$ to be a prime not too close to a power of
2 is recommended \cite{k97}.

\subsection{Hashing by Multiplication}
\index{hashing!by multiplication}

The implementation of a hash table using \emph{hashing by
multiplication} is exactly the same as that hashing by division
except that the hash function 

\[
  h(x) = \floor{mxA} \bmod m
\]
is used.  Here $A$ is a real-valued constant whose choice we discuss
below. The advantage of the multiplication method is that the value of
$m$ is not critical.  We can take $m$ to be a power of 2, which makes
it convenient for use on binary computers. 

Although any value of $A$ gives a hash function, some values of $A$
are better than others.  (Setting $A=0$ is clearly not a good idea.)

Knuth \cite{k97} suggests using the \emph{golden ratio}\index{golden
ratio} for $A$, i.e.,
setting 
\[  
    A=(\sqrt{5}-1)/2=0.6180339887\ldots
\]   
This choice of $A$ is motivated by a theorem, first conjectured by
Oderfeld and later proven by S\'wierczkowski \cite{s58}.  This theorem
states that the sequence 
\[
     mA \bmod m,\,\, 2mA \bmod m,\,\, 3mA \bmod m,\,\,\ldots,\,\, nmA\bmod m
\] 
partitions the interval $(0,m)$ into $n+1$ intervals having only three
distinct lengths.  Furthermore, the next element $(n+1)mA\bmod m$ in
the sequence is always contained in one of the largest
intervals.\footnote{In fact, any irrational number has this property
\cite{sos57}.  The golden ratio is especially good because it is
not too close to a whole number.}

Of course, no matter what value of $A$ we select, the pigeonhole
principle implies that for $u\ge nm$ then there will always exist some
hash value $i$ and some $S\subseteq U$ of size $n$ such that $h(x)=i$
for all $x\in S$.  In other words, we can always find a set $S$ all of
whose elements get stored in the same list $L_i$. Thus, the worst case
of hashing by multiplication is as bad as hashing by division.


\subsection{Universal Hashing}
\label{hash:sec:universal}
\index{hashing!universal}
\index{universal hashing}

The argument used at the end of the previous section applies equally
well to any hash function $h$.  That is, if the table size $m$ is much
smaller than the universe size $u$ then for any hash function there is
some large (of size at least $\ceil{u/m}$) subset of $U$ that has the
same hash value.  To get around this difficulty we need a collection
of hash functions of which we can choose one that works well for $S$.
Even better would be a collection of hash functions such that, for any
given $S$, most of the hash functions work well for $S$.  Then we
could simply pick one of the functions at random and have a good
chance of it working well.

Let $\mathcal{H}$ be a collection of hash functions, i.e., functions
from $U$ onto $\{0,\ldots,m-1\}$.  We say that $\mathcal{H}$ is
\emph{universal}\index{hash function!universal} if, for each $x,y\in
U$ the number of $h\in \mathcal{H}$ such that $h(x)=h(y)$ is at most
$|\mathcal{H}|/m$.  Consider any $S\subseteq U$ of size $n$ and
suppose we choose a random hash function $h$ from a universal
collection of hash functions.  Consider some value $x\in U$.  The
probability that any key $y\in S$ has the same hash value as $x$ is
only $1/m$.  Therefore, the expected number of keys in $S$, not equal
to $x$, that have the same hash value as $x$ is only \[ n_{h(x)} =
\left\{\begin{array}{ll} (n-1)/m & \mbox{if $x\in S$} \\ n/m &
\mbox{if $x\notin S$} \end{array}\right.  \] Therefore, if we store
$S$ in a hash table using the hash function $h$ then the expected time
to search for, or delete, $x$ is $O(1+\alpha)$.

From the preceding discussion, it seems that a universal collection of
hash functions from which we could quickly select one at random would be
very handy indeed. With such a collection at our disposal we get an
implementation of the set ADT that has $O(1)$ insertion time and
$O(1)$ expected search and deletion time.

Carter and Wegman \cite{cw79} describe three different collections of
universal hash functions.  If the universe size $u$ is a
prime number\footnote{This is not a major restriction since, for any
$u>1$, there always exists a prime number in the set
$\{u,u+1,\ldots,2u\}$.  Thus we can enforce this assumption by
increasing the value of $u$ by a constant factor.} then
\[
  \mathcal{H} = \{ h_{k_1,k_2,m}(x)=((k_1x+k_2)\bmod u))\bmod m : 1\le k_1< u,
0\le k_2< u\} 
\]
is a collection of universal hash functions.  Clearly, choosing a
function uniformly at random from $\mathcal{H}$ can be done easily by
choosing two random values $k_1\in \{1,\ldots,u-1\}$ and $k_2\in
\{0,\ldots,u-1\}$.  Thus, we have an implementation of the set ADT with
$O(1)$ expected time per operation.

 
\subsection{Static Perfect Hashing}
\index{perfect hashing!static}
\index{hashing!static perfect}
\index{static perfect hashing}
\label{hash:sec:fks}

The result of Carter and Wegman on universal hashing is very strong,
and from a practical point of view, it is probably the strongest
result most people will ever need.  The only thing that could be
improved about their result is to make it deterministic, so that the
running times of all operations are $O(1)$ \emph{worst-case}.
Unfortunately, this is not possible, as shown by Dietzfelbinger \etal\
\cite{dkmm94}.  

Since there is no hope of getting $O(1)$ worst-case time for all three
set ADT operations, the next best thing would be to have searches that
take $O(1)$ worst-case time.  In this section we describe the method
of Fredman, Koml\'os and Szemer\'edi \cite{fks84}.  This is a static
data structure that takes as input a set $S\subseteq U$ and builds a
data structure of size $O(n)$ that can test if an element $x$ is in
$S$ in $O(1)$ worst-case time.  Like the universal hash functions from
the previous section, this method also requires that $u$ be a prime
number. This scheme uses hash functions of the form
\[
  h_{k,m}(x) = (kx\bmod u))\bmod m \enspace .\footnote{Actually, it
turns out that any universal hash function also works in the FKS
scheme \cite[Section~11.5]{clrs01}.}
\]

Let $B_{k,m}(S,i)$ be the number of elements $x\in S$ such that
$h_{k,m}(x)=i$, i.e., the number of elements of $S$ that have hash
value $i$ when using the hash function $h_{k,m}$.  The function
$B_{k,m}$ gives complete information about the distribution of hash
values of $S$.  The main lemma used by Fredman \etal\ is that, if we
choose $k\in U$ uniformly at random then
\begin{equation}
\E\left[\sum_{i=0}^{m-1} {B_{k,m}(S,i)\choose 2}\right] < \frac{n^2}{m}  
  \enspace . \label{hash:eq:fks-main}
\end{equation}
There are two important special cases of this result.

In the \emph{sparse case} we take $m=n^2/\alpha$, for some constant
$0<\alpha<1$.  In this case, the expectation in
(\ref{hash:eq:fks-main}) is less than $\alpha$.  Therefore, by
Markov's inequality, the probability that this sum is greater than or
equal to 1 is at most $\alpha$.  But, since this sum is a non-negative
integer, then with probability at least $1-\alpha$ it must be equal to
0.  In other words, with probability at least $1-\alpha$,
$B_{k,m}(S,i)\le 1$ for all $0\le i\le m-1$, i.e., the hash function
$h_{k,m}$ is perfect for $S$.  Of course this implies that we can find
a perfect hash function very quickly by trying a small number of
random elements $k\in U$ and testing if they result in perfect hash
functions. (The expected number of elements that we will have to try
is only $1/(1-\alpha)$.) Thus, if we are willing to use quadratic
space then we can perform searches in $O(1)$ worst-case time.

In the \emph{dense case} we assume that $m$ is close to $n$ and
discover that, for many values of $k$, the hash values are distributed
fairly evenly among the set $1,\ldots,m$.  More precisely, if we use a
table of size $m=n$, then 
\[
   \E \left[ \sum_{i=0}^{m-1} B_{k,m}(S,i)^2 \right] \le 3n \enspace .
      \label{hash:eq:fks-space-a}
\]
By Markov's inequality this means that
\begin{equation}
   \Pr\left\{ \sum_{i=0}^{m-1} B_{k,m}(S,i)^2  \le 3n/\alpha \right\}
\ge
	1-\alpha \enspace .
      \label{hash:eq:fks-space}
\end{equation}
Again, we can quickly find a value of $k$ satisfying
(\ref{hash:eq:fks-space}) by testing a few randomly chosen values of
$k$.

These two properties are enough to build a two-level data structure
that uses linear space and executes searches in worst-case constant
time.  We call the following data structure the FKS-$\alpha$ data
structure, after its inventors Fredman, Koml\'os and Szemer\'edi.  At
the top level, the data structure consists of an array
$A[0],\ldots,A[m-1]$ where $m=n$.  The elements of this array are
pointers to other arrays $A_0,\ldots,A_{m-1}$, respectively.  To
decide what will be stored in these other arrays, we build a hash
function $h_{k,m}$ that satisfies the conditions of
(\ref{hash:eq:fks-space}).  This gives us the top-level hash function
$h_{k,m}(x)=(kx\bmod u)\bmod m$. Each element $x\in S$ gets stored in
the array pointed to by $A[h_{k,m}(x)]$.

What remains is to describe how we use the arrays
$A_0,\ldots,A_{m-1}$.  Let $S_i$ denote the set of elements $x\in S$
such that $h_{k,m}(s)=i$.  The elements of $S_i$ will be stored in
$A_i$.  The size of $S_i$ is $n_i=B_{k,m}(S,i)$.  To store the
elements of $S_i$ we set the size of $A_i$ to
$m_i=n_i{}^2/\alpha=B_{k,n}(S,i)^2/\alpha$.  Observe that, by
(\ref{hash:eq:fks-space}), all the $A_i$'s take up a total space of
$O(n)$, i.e., $\sum_{i=0}^{m-1} m_i = O(n)$.  Furthermore, by trying a
few randomly selected integers we can quickly find a value $k_i$ such
that the hash function $h_{k_i,m_i}$ is perfect for $S_i$.  Therefore,
we store the element $x\in S_i$ at position $A_i[h_{k_i,m_i}(x)]$ and
$x$ is the unique element stored at that location.  With this scheme
we can search for any value $x\in U$ by computing two hash values
$i=h_{k,m}(x)$ and $j=h_{k_i,m_i}(x)$ and checking if $x$ is stored in
$A_i[j]$.

Building the array $A$ and computing the values of
$n_0,\ldots,n_{m-1}$ takes $O(n)$ expected time since for a given
value $k$ we can easily do this in $O(n)$ time and the expected number
of values of $k$ that we must try before finding one that satisfies
(\ref{hash:eq:fks-space}) is $O(1)$.  Similarly, building each subarray
$A_i$ takes $O(n_i{}^2)$ expected time, resulting in an overall expected
running time of $O(n)$.  Thus, for any constant $0<\alpha<1$, an
FKS-$\alpha$ data structure can be constructed in $O(n)$ expected time
and this data structure can execute a search for any $x\in U$ in
$O(1)$ worst-case time.

\subsection{Dynamic Perfect Hashing}
\label{hash:sec:dphash}
\index{hashing!dynamic perfect}
\index{dynamic perfect hashing}
\index{perfect hashing!dynamic}
The FKS-$\alpha$ data structure is nice in that it allows for searches
in $O(1)$ time, in the worst case.  Unfortunately, it is only static;
it does not support insertions or deletions of elements.  In this
section we describe a result of Dietzfelbinger \etal\ \cite{dkmm94}
that shows how the FKS-$\alpha$ data structure can be made dynamic
with some judicious use of partial rebuilding
(Section~\ref{external:12:Andersson-trees-partial-rebuilding}).

The main idea behind the scheme is simple: be lazy at both the upper
and lower levels of the FKS-$\alpha$ data structure.  That is, rebuild
parts of the data structure only when things go wrong.  At the top
level, we relax the condition that the size $m$ of the upper array $A$
is exactly $n$ and allow $A$ to have size anywhere between $n$ and
$2n$.  Similarly, at the lower level we allow the array $A_i$ to have
a size $m_i$ anywhere between $n_i{}^2/\alpha$ and $2n_i{}^2/\alpha$.

Periodically, we will perform a \emph{global rebuilding}\index{global
rebuilding} operation in
which we remove all $n$ elements from the hash table.  Some elements
which have previously been marked as deleted will be discarded,
thereby reducing the value of $n$.  We put the remaining elements in a
list, and recompute a whole new FKS-$(\alpha/2)$ data structure for
the elements in the list.  This data structure is identical to the
standard FKS-$(\alpha/2)$ data structure except that, at the top level
we use an array of size $m=2n$.

Searching in this data structure is exactly the same as for the static
data structure.  To search for an element $x$ we compute
$i=h_{k,m}(x)$ and $j=h_{k_i,m_i}(x)$ and look for $x$ at location
$A_i[j]$.   Thus, searches take $O(1)$ worst-case time.

Deleting in this data structure is done in the laziest manner
possible.  To delete an element we only search for it and then mark it
as deleted.  We will use the convention that this type of deletion
does not change the value of $n$ since it does not change the number
of elements actually stored in the data structure.  While doing this,
we also keep track of the number of elements that are marked as
deleted.  When this number exceeds $n/2$ we perform a global
rebuilding operation.  The global rebuilding operation takes $O(n)$
expected time, but only occurs during one out of every $n/2$
deletions.  Therefore, the amortized cost of this operation is $O(1)$
per deletion.

The most complicated part of the data structure is the insertion
algorithm and its analysis.  To insert a key $x$ we know, because of
how the search algorithm works, that we must ultimately store $x$ at
location $A_i[j]$ where $i=h_{k,m}(x)$ and $j=h_{k_i,m_i}(x)$.
However, several things can go wrong during the insertion of $x$:

\begin{enumerate} 
\item The value of $n$ increases by 1, so it may be that $n$ now
exceeds $m$.  In this case we perform a global rebuilding
operation and we are done.

\item We compute $i=h_{k,m}(x)$ and discover that $\sum_{i=0}^{m-1}
n_i{}^2 > 3n/\alpha$.  In this case, the hash function
$h_{k,m}$ used at the top level is no longer any good since it is
producing an overall hash table that is too large.  In this case we
perform a global rebuilding operation and we are done.

\item We compute $i=h_{k,m}(x)$ and discover that, since the value of
$n_i$ just increased by one, $n_i{}^2/\alpha > m_i$. In this case, the
array $A_i$ is too small to guarantee that we can quickly find a
perfect hash function.  To handle this, we copy the elements of $A_i$
into a list $L$ and allocate a new array $A_i$ with the new size
$m_i=2n_i{}^2/\alpha$.  We then find a new value $k_i$ such that
$h_{k_i,m_i}$ is a perfect hash function for the elements of $L$ and
we are done.

\item The array location $A_i[j]$ is already occupied by some other
element $y$.  But in this case, we know that $A_i$ is large enough to
hold all the elements (otherwise we would already be done after
Case~3), but the value $k_i$ being used in the hash function
$h_{k_i,m_i}$ is the wrong one since it doesn't give a perfect hash
function for $S_i$.  Therefore we simply try new values for $k_i$
until we find a find a value $k_i$ that yields a perfect hash function
and we are done.  
\end{enumerate}

If none of the preceding 4 cases occurs then we can simply place $x$
at location $A_i[j]$ and we are done.

Handling Case~1 takes $O(n)$ expected time since it involves a global
rebuild of the entire data structure.  However, Case~1 only happens
during one out of every $\Theta(n)$ insertions, so the amortized cost
of all occurrences of Case~1 is only $O(1)$ per insertion.

Handling Case~2 also takes $O(n)$ expected time.  The question is: How
often does Case~2 occur?  To answer this question, consider the
\emph{phase}\index{phase} that occurs between two consecutive
occurrences of Case~1.  During this phase, the data structure holds at
most $m$ distinct elements.  Call this set of elements $S$.  With
probability at least $(1-\alpha)$ the hash function $h_{k,m}$ selected
at the beginning of the phase satisfies (\ref{hash:eq:fks-space}) so
that Case~2 never occurs during the phase.  Similarly, the probability
that Case~2 occurs exactly once during the phase is at most
$\alpha(1-\alpha)$.  In general, the probability that Case~2 occurs
exactly $i$ times during a phase is at most $\alpha^i(1-\alpha)$.  Thus, the
expected cost of handling all occurrences of Case~2 during the entire
phase is at most 
\[ \sum_{i=0}^\infty \alpha^i(1-\alpha)i\times O(n) =
	O(n) \enspace .  
\] 
But since a phase involves $\Theta(n)$ insertions this means that the
amortized expected cost of handling Case~2 is $O(1)$ per insertion.

Next we analyze the total cost of handling Case~3.  Define a
\emph{subphase}\index{subphase} as the period of time between two
global rebuilding operations triggered either as a result of a
deletion, Case~1 or Case~2.   We will show that the total cost of
handling all occurrences of Case~3 during a subphase is $O(n)$ and
since a subphase takes $\Theta(n)$ time anyway this does not
contribute to the cost of a subphase by more than a constant factor.
When Case~3 occurs at the array $A_i$ it takes $O(m_i)$ time.
However, while handling Case~3, $m_i$ increases by a constant factor,
so the total cost of handling Case~3 for $A_i$ is dominated by the
value of $m_i$ at the end of the subphase.  But we maintain the
invariant that $\sum_{i=0}^{m-1} m_i = O(n)$ during the entire
subphase.  Thus, handling all occurrences of Case~3 during a subphase
only requires $O(n)$ time. 

Finally, we consider the cost of handling Case~4.  For a particular
array $A_i$, consider the \emph{subsubphase}\index{subsubphase}
between which two occurrences of Case~3 cause $A_i$ to be rebuilt or a
global rebuilding operation takes place.  During this subsubphase the
number of distinct elements that occupy $A_i$ is at most
$\alpha\sqrt{m_i}$.  Therefore, with probability at least $1-\alpha$
any randomly chosen value of $k_i\in U$ is a perfect hash function for
this set.   Just as in the analysis of Case~2, this implies that the
expected cost of handling all occurrences of Case~3 at $A_i$ during a
subsubphase is only $O(m_i)$.  Since a subsubphase ends with
rebuilding all of $A_i$ or a global rebuilding, at a cost of
$\Omega(m_i)$ all the occurrences of Case~4 during a subsubphase do not
contribute to the expected cost of the subsubphase by more than a
constant factor.

To summarize, we have shown that the expected cost of handling all
occurrences of Case~4 is only  a constant factor times the cost of
handling all occurrences of Case~3.  The cost of handling all
occurrences of Case~3 is no more than a constant factor times the
expected cost of all global rebuilds.  The cost of handling all the
global rebuilds that occur as a result of Case~2 is no more than a
constant factor times the cost of handling all occurrences of global
rebuilds that occur as a consequence of Case~1.  And finally, the cost
of all global rebuilds that occur as a result of Case~1 or of
deletions is $O(n)$ for a sequence of $n$ update operations.
Therefore, the total expected cost of $n$ update operation is $O(n)$.


\section{Random Probing}
\label{hash:sec:uniform}
\index{random probing}
\index{probing!random}

Next we consider hash table implementations under the random probing
assumption: Each element $x$ stored in the hash table comes with a
random sequence $x_0,x_1,x_2,\ldots$ where each of the $x_i$ is
independently and uniformly distributed in
$\{1,\ldots,m\}$.\footnote{A variant of the random probing assumption,
referred to as the \emph{uniform hashing}\index{uniform
hashing}\index{hashing!uniform} assumption, assumes that
$x_0,\ldots,x_{m-1}$ is a random permutation of $0,\ldots,m-1$.}  We
begin with a discussion of the two basic paradigms: hashing with
chaining and open addressing.  Both these paradigms attempt to store
the key $x$ at array position $A[x_0]$.  The difference between these
two algorithms is their \emph{collision resolution
strategy}\index{collision resolution strategy}, i.e., what the
algorithms do when a user inserts the key value $x$ but array position
$A[x_0]$ already contains some other key.

\subsection{Hashing with Chaining}
\index{hashing!with chaining}

In \emph{hashing with chaining}, a collision is resolved by allowing
more than one element to live at each position in the table.  Each
entry in the array $A$ is a pointer to the head of a linked list.  To
insert the value $x$, we simply append it to the list $A[x_0]$.  To
search for the element $x$, we perform a linear search in the list
$A[x_0]$. To delete the element $x$, we search for $x$ in the list
$A[x_0]$ and splice it out.

It is clear that insertions take $O(1)$ time, even in the worst case.
For searching and deletion, the running time is proportional to a
constant plus the length of the list stored at $A[x_0]$.  Notice that
each of the at most $n$ elements not equal to $x$ is stored in
$A[x_0]$ with probability $1/m$, so the expected length of $A[x_0]$ is
either $\alpha=n/m$ (if $x$ is not contained in the table) or
$1+(n-1)/m$ (if $x$ is contained in the table). Thus, the expected
cost of searching for or deleting an element is $O(1+\alpha)$.

The above analysis shows us that hashing with chaining supports the
three set ADT operations in $O(1)$ expected time per operation, as
long as the occupancy, $\alpha$, is a constant. It is worth noting
that this does not require that the value of $\alpha$ be less than 1. 

If we would like more detailed information about the cost of searching, we
might also ask about the \emph{worst-case search time} defined as
\[
   W = \max\{\mbox{length of the list stored at $A[i]$} : 0\le i\le m-1 \} 
	\enspace .
\]
It is very easy to prove something quite strong about $W$ using only
the fact that the length of each list $A[i]$ is a
$\mathrm{binomial}(n,1/m)$ random variable.  Using Chernoff's bounds
on the tail of the binomial distribution \cite{c52}, this
immediately implies that   
\[
   \Pr\{\mbox{length of $A[i]$} \ge \alpha c\ln n\} \le n^{-\Omega(c)} \enspace .
\]
Combining this with Boole's inequality ($\Pr\{\mbox{$A$ or $B$}\} \le
\Pr\{A\}+\Pr\{B\}$) we obtain
\[
   \Pr\{W \ge \alpha c\ln n\} \le n\times n^{-\Omega(c)} = n^{-\Omega(c)} \enspace .
\]

Thus, with very high probability, the worst-case search time is
logarithmic in $n$.  This also implies that $\E[W]=O(\log n)$.  The
distribution of $W$ has been carefully studied and it is known that,
\emph{with high probability}\index{high probability}, i.e., with
probability $1-o(1)$, $W=(1+o(1))\ln n/\ln\ln n$
\cite{jk77,ksc78}.\footnote{Here, and throughout this chapter, if an
asymptotic notation does not contain a variable then the variable that
tends to infinity is implicitly $n$.  Thus, for example, $o(1)$ is the
set of non-negative functions of $n$ that tend to 0 as
$n\rightarrow\infty$.} Gonnet has proven a more accurate result that
$W=\Gamma^{-1}(n) -3/2 + o(1)$ with high probability. Devroye
\cite{d85} shows that similar results hold even when the distribution
of $x_0$ is not uniform.

\subsection{Hashing with Open Addressing}
\index{open addressing}
\index{hashing!with open addressing}

\label{hash:sec:open-addressing}

\emph{Hashing with open addressing} differs from hashing with chaining in
that each table position $A[i]$ is allowed to store only one value.
When a collision occurs at table position $i$, one of the two elements
involved in the collision must move on to the next element in its
probe sequence.  In order to implement this efficiently and correctly
we require a method of marking elements as deleted.  This method could
be an auxiliary array that contains one bit for each element of $A$,
but usually the same result can be achieved by using a special key
value $\mathbf{del}$ that does not correspond to any valid key.

To search for an element $x$ in the hash table we look for $x$ at
positions $A[x_0]$, $A[x_1]$, $A[x_2]$, and so on until we either
(1)~find $x$, in which case we are done or (2)~find an empty table
position $A[x_i]$ that is not marked as deleted, in which case we can
be sure that $x$ is not stored in the table (otherwise it would be
stored at position $x_i$).  To delete an element $x$ from the hash
table we first search for $x$.  If we find $x$ at table location
$A[x_i]$ we then simply mark $A[x_i]$ as deleted.  To insert a value
$x$ into the hash table we examine table positions $A[x_0]$, $A[x_1]$,
$A[x_2]$, and so on until we find a table position $A[x_i]$ that is
either empty or marked as deleted and we store the value $x$ in
$A[x_i]$.

Consider the cost of inserting an element $x$ using this method.  Let
$i_x$ denote the smallest value $i$ such that $x_{i_x}$ is either empty or
marked as deleted when we insert $x$.  Thus, the cost of inserting $x$
is a constant plus $i_x$.  The probability that the table position
$x_0$ is occupied is at most $\alpha$ so, with probability at least
$1-\alpha$, $i_x=0$.  Using the same reasoning, the probability
that we store $x$ at position $x_i$ is at most 
\begin{equation}
   \Pr\{i_x = i\} \le \alpha^{i}(1-\alpha)  \label{hash:eq:pro}
\end{equation}
since the table locations $x_0,\ldots,x_{i-1}$ must be occupied,
the table location $x_i$ must not be occupied and the $x_i$ are
independent.  Thus, the expected number of steps taken by the
insertion algorithm is 
\[
   \sum_{i=1}^\infty i\Pr\{i_x = i\} = (1-\alpha)\sum_{i=1}^\infty i\alpha^{i-1}  = 1/(1-\alpha)
\]
for any constant $0<\alpha< 1$.  The cost of searching for $x$ and
deleting $x$ are both proportional to the cost of inserting $x$, so
the expected cost of each of these operations is
$O(1/(1-\alpha))$.\footnote{Note that the expected cost of searching
for or deleting an
element $x$ is proportional to the value of $\alpha$ \emph{at the time
$x$ was inserted.}  If many deletions have taken place, this may be
quite different than the current value of $\alpha$.}   

We should compare this with the cost of hashing with chaining.  In
hashing with chaining,the occupancy $\alpha$ has very little effect on
the cost of operations.  Indeed, any constant $\alpha$, even greater
than $1$ results in $O(1)$ time per operation.  In contrast, open
addressing is very dependent on the value of $\alpha$.  If we take
$\alpha>1$ then the expected cost of insertion using open addressing
is infinite since the insertion algorithm never finds an empty table
position.  Of course, the advantage of hashing with chaining is that
it does not require lists at each of the $A[i]$.  Therefore, the
overhead of list pointers is saved and this extra space can be used
instead to maintain the invariant that the occupancy $\alpha$ is a
constant strictly less than 1.

Next we consider the worst case search time of hashing with open
addressing.  That is, we study the value $W=\max\{i_x: \mbox{$x$ is
stored in the table at location $i_x$}\}$.  Using (\ref{hash:eq:pro}) and Boole's
inequality it follows almost immediately that 
\[
  \Pr\{W > c\log n\} \le n^{-\Omega(c)} .
\]

Thus, with very high probability, $W$, the worst case search time, is
$O(\log n)$.  Tighter bounds on $W$ are known when the probe sequences
$x_0,\ldots,x_{m-1}$ are random permutations of $0,\ldots,m-1$.  In this
case, Gonnet shows that
\[
  \E[W]=\log_{1/\alpha} n - \log_{1/\alpha}(\log_{1/\alpha} n) + O(1)
\]
\cite{g81}. 

Open addressing under the random probing assumption has many nice
theoretical properties and is easy to analyze.  Unfortunately, it is
often criticized as being an unrealistic model because it requires a
long random sequences $x_0,x_1,x_2,\ldots$ for each element $x$ that
is to be stored or searched for.  Several variants of open addressing
discussed in the next few sections try to overcome this problem by
using only a few random values. 


\subsection{Linear Probing}
\index{linear probing}
\index{hashing!with linear probing}
 
\emph{Linear probing} is a variant of open addressing that requires less
randomness.  To obtain the probe sequence $x_0,x_1,x_2,\ldots$ we
start with a random element $x_0\in \{0,\ldots,m-1\}$.  The element
$x_i$, $i>0$ is given by $x_i=(i+x_0)\bmod m$.  That is, one first
tries to find $x$ at location $x_0$ and if that fails then one looks
at $(x_0+1)\bmod m$, $(x_0+2)\bmod m$ and so on.  

The performance of linear probing is discussed by Knuth \cite{k97} who
shows that the expected number of probes performed during an
unsuccessful search is at most 
\[
	(1+1/(1-\alpha)^2)/2
\] 
and the expected
number of probes performed during a successful search is at most
\[
	(1+1/(1-\alpha))/2 \enspace . 
\] 
This is not quite as good as for standard hashing with open
addressing, especially in the unsuccessful case.

Linear probing suffers from the problem of \emph{primary
clustering}\index{primary clustering}\index{clustering!primary}.  If
$j$ consecutive array entries are occupied then a newly inserted
element will have probability $j/m$ of hashing to one of these
entries. This results in $j+1$ consecutive array entries being
occupied and increases the probability (to $(j+1)/m$) of another newly
inserted element landing in this cluster.  Thus, large clusters of
consecutive elements have a tendency to grow larger.

\subsection{Quadratic Probing}
\index{quadratic probing}
\index{hashing!with quadratic probing}

\emph{Quadratic probing} is similar to linear probing;  an element $x$
determines its entire probe sequence based on a single random choice,
$x_0$.  Quadratic probing uses the probe sequence
$x_0,(x_0+k_1+k_2)\bmod m, (x_0+2k_1+2^2k_2)\bmod m,\ldots$.  In
general, the $i$th element in the probe sequence is
$x_i=(x_0+ik_1+i^2k_2)\bmod m$.  Thus, the final location of an
element depends quadratically on how many steps were required to
insert it.  This method seems to work much better in practice than
linear probing, but requires a careful choice of $m$, $k_1$ and $k_2$
so that the probe sequence contains every element of $\{0,\ldots,m-1\}$.

The improved performance of quadratic probing is due to the fact that
if there are two elements $x$ and $y$ such that $x_i=y_j$ then it is
not necessarily true (as it is with linear probing) that
$x_{i+1}=y_{j+1}$.  However, if $x_0=y_0$ then $x$ and $y$ will have
exactly the same probe sequence.  This lesser phenomenon is called
\emph{secondary clustering}\index{secondary
clustering}\index{clustering!secondary}.  Note that this secondary
clustering phenomenon implies that neither linear nor quadratic
probing can hope to perform any better than hashing with chaining.
This is because all the elements that have the same initial hash $x_0$
are contained in an implicit chain.  In the case of linear probing,
this chain is defined by the sequence $x_0,x_0+1,x_0+2,\ldots$ while
for quadratic probing it is defined by the sequence
$x_0,x_0+k_1+k_2,x_0+2k_1+4k_2,\ldots$

\subsection{Double Hashing}\label{hash:sec:double}
\index{hashing!double}
\index{double hashing}

\emph{Double hashing} is another method of open addressing that uses two hash
values $x_0$ and $x_1$.  Here $x_0$ is in the set $\{0,\ldots,m-1\}$
and $x_1$ is in the subset of $\{1,\ldots,m-1\}$ that is relatively
prime to $m$.  With double hashing, the probe sequence for element $x$
becomes $x_0,(x_0+x_1)\bmod m, (x_0+2x_1)\bmod m,\ldots$.  In general,
$x_i=(x_0+ix_1)\bmod m$, for $i>0$.  The expected number of probes
required by double hashing seems difficult to determine exactly.
Guibas has proven that, asymptotically, and for occupancy $\alpha\le
.31$, the performance of double hashing is asymptotically equivalent
to that of uniform hashing.  Empirically, the performance of double
hashing matches that of open addressing with random probing
regardless of the occupancy $\alpha$ \cite{k97}.

\subsection{Brent's Method}\label{hash:sec:brent}
\index{Brent's method}
\index{hashing!Brent's method}

\emph{Brent's method} \cite{b73} is a heuristic that attempts to
minimize the average time for a successful search in a hash table with
open addressing.  Although originally described in the context of
double hashing (Section~\ref{hash:sec:double}) Brent's method applies
to any open addressing scheme.  The \emph{age}\index{age} of an element $x$
stored in an open addressing hash table is the minimum value $i$ such
that $x$ is stored at $A[x_i]$.  In other words, the age is one less
than the number of locations we will probe when searching for $x$. 

Brent's method attempts to minimize the total age of all elements in
the hash table.  To insert the element $x$ we proceed as follows:  We
find the smallest value $i$ such that $A[x_i]$ is empty; this is where
standard open-addressing would insert $x$.  Consider the element $y$
stored at location $A[x_{i-2}]$.  This element is stored there because
$y_j=x_{i-2}$, for some $j\ge 0$.  We check if the array location
$A[y_{j+1}]$ is empty and, if so, we move $y$ to location $A[y_{j+1}]$
and store $x$ at location $A[x_{i-2}]$.  Note that, compared to
standard open addressing, this decreases the total age by 1.  In
general, Brent's method checks, for each $2\le k\le i$ the array entry
$A[x_{i-k}]$ to see if the element $y$ stored there can be moved to
any of $A[y_{j+1}],A[y_{j+2}],\ldots,A[y_{j+k-1}]$ to make room for
$x$.  If so, this represents a decrease in the total age of all
elements in the table and is performed.

Although Brent's method seems to work well in practice, it is
difficult to analyze theoretically.  Some theoretical analysis of
Brent's method applied to double hashing is given by Gonnet and Munro
\cite{gm79}.  Lyon \cite{lyon78}, Munro and Celis \cite{mc99} and
Poblete \cite{p77} describe some variants of Brent's method.


\subsection{Multiple-Choice Hashing}
\label{hash:sec:multiplechoice}
\index{hashing!multiple-choice}
\index{multiple-choice hashing}

It is worth stepping back at this point and revisiting the comparison
between hash tables and binary search trees.  For balanced binary
search trees, the average cost of searching for an element is $O(\log
n)$.  Indeed, it easy to see that for at least $n/2$ of the elements,
the cost of searching for those elements is $\Omega(\log n)$.  In
comparison, for both the random probing schemes discussed so far, the
expected cost of search for an element is $O(1)$.  However, there are
a handful of elements whose search cost is $\Theta(\log n/\log \log
n)$ or $\Theta(\log n)$ depending on whether hashing with chaining or
open addressing is used, respectively.  Thus there is an inversion:
Most operations on a binary search tree cost $\Theta(\log n)$ but a
few elements (close to the root) can be accessed in $O(1)$ time.  Most
operations on a hash table take $O(1)$ time but a few elements (in long
chains or with long probe sequences) require $\Theta(\log n/\log\log
n)$ or $\Theta(\log n)$ time to access.  In the next few sections we
consider variations on hashing with chaining and open addressing that
attempt to reduce the worst-case search time $W$.

\emph{Multiple-choice hashing} is hashing with chaining in which,
during insertion, the element $x$ has the choice of $d\ge 2$ different
lists in which it can be stored.  In particular, when we insert $x$ we
look at the lengths of the lists pointed to by
$A[x_0],\ldots,A[x_{d-1}]$ and append $x$ to $A[x_i]$, $0\le i< d$
such that the length of the list pointed to by $A[x_i]$ is minimum.
When searching for $x$, we search for $x$ in each of the lists
$A[x_0],\ldots,A[x_{d-1}]$ \emph{in parallel}.  That is, we look at
the first elements of each list, then the second elements of each
list, and so on until we find $x$.  As before, to delete $x$  we first
search for it and then delete it from whichever list we find it in.  

It is easy to see that the expected cost of searching for an element
$x$ is $O(d)$ since the expected length of each the $d$ lists is
$O(1)$.  More interestingly, the worst case search time is bounded by
$O(dW)$ where $W$ is the length of the longest list.  Azar \etal\
\cite{abku99} show that 
\begin{equation}
    \E[W]=\frac{\ln\ln n}{\ln d} + O(1) \enspace .
	\label{hash:eq:multiple-choice} 
\end{equation}
Thus, the expected worst case search time for multiple-choice hashing
is $O(\log\log n)$ for any constant $d\ge 2$.


\subsection{Asymmetric Hashing}
\index{hashing!asymmetric}
\index{asymmetric hashing}

\emph{Asymmetric hashing} is a variant of multiple-choice hashing in
which the hash table is split into $d$ blocks, each of size $n/d$.
(Assume, for simplicity, that $n$ is a multiple of $d$.) The probe
value $x_i$, $0\le i< d$ is drawn uniformly from
$\{in/d,\ldots,(i+1)n/d-1\}$.   As with multiple-choice hashing, to
insert $x$ the algorithm examines the lengths of the lists $A[x_0],
A[x_1],\ldots,A[x_{d-1}]$ and appends $x$ to the shortest of these
lists.  In the case of ties, it appends $x$ to the list with smallest
index.  Searching and deletion are done exactly as in multiple-choice
hashing.

V\"ocking \cite{v99} shows that, with asymmetric hashing the expected
length of the longest list is 
\[
   \E[W] \le \frac{\ln\ln n}{d\ln\phi_d} +O(1) \enspace .
\] 
The function $\phi_d$ is a generalization of the \emph{golden
ratio}\index{golden ratio},
so that $\phi_2=(1+\sqrt{5})/2$.  Note that this improves
significantly on standard multiple-choice hashing (\ref{hash:eq:multiple-choice}) for larger values of
$d$.

\subsection{LCFS Hashing}
\index{hashing!LCFS}
\index{LCFS hashing}

\emph{LCFS hashing} is a form of open addressing that changes the
collision resolution strategy.\footnote{Amble and Knuth \cite{ak74}
were the first to suggest that, with open addressing, any collision
resolution strategy could be used.} Reviewing the algorithm for
hashing with open addressing reveals that when two elements collide,
priority is given to the first element inserted into the hash table
and subsequent elements must move on.  Thus, hashing with open
addressing could also be referred to as \emph{FCFS (first-come
first-served) hashing}\index{FCFS hashing}\index{hashing!FCFS}.

With LCFS (last-come first-served) hashing, collision resolution is
done in exactly the opposite way.  When we insert an element $x$, we
always place it at location $x_0$.  If position $x_0$ is already
occupied by some element $y$ because $y_j=x_0$ then we place $y$ at
location $y_{j+1}$, possibly displacing some element $z$, and so on.

Poblete and Munro \cite{pm89} show that, after inserting $n$ elements
into an initially empty table, the expected worst case search time is
bounded above by
\[  \E[W] \le 
   1 
   + \Gamma^{-1}(\alpha n)
   \left( 1+\frac{\ln\ln(1/(1-\alpha))}{\ln \Gamma^{-1}(\alpha n)}
   + O\left(\frac{1}{\ln^2\Gamma^{-1}(\alpha n)} \right)\right)
\enspace ,
\]
where $\Gamma$ is the gamma function and 
\[
  \Gamma^{-1}(\alpha n) = \frac{\ln n}{\ln\ln n}
	\left(1+\frac{\ln\ln\ln n}{\ln\ln n} + 
	O\left(\frac{1}{\ln\ln n}\right)\right) \enspace .
\]
Historically, LCFS hashing is the first version of open addressing
that was shown to have an expected worst-case search time that is
$o(\log n)$.

\subsection{Robin-Hood Hashing}
\label{hash:sec:robinhood}
\index{hashing!Robin-Hood}
\index{Robin-Hood hashing}

Robin-Hood hashing \cite{c86,clm85,vp98} is a form of open addressing
that attempts to equalize the search times of elements by using a
fairer collision resolution strategy.  During insertion, if we are
trying to place element $x$ at position $x_i$ and there is already an
element $y$ stored at position $y_j=x_i$ then the ``younger'' of the
two elements must move on.  More precisely, if $i\le j$ then we will
try to insert $x$ at position $x_{i+1}$, $x_{i+2}$ and so on.
Otherwise, we will store $x$ at position $x_i$ and try to to insert
$y$ at positions $y_{j+1}$, $y_{j+2}$ and so on.

Devroye \etal\ \cite{dmv03} show that, after performing $n$ insertions on
an initially empty table of size $m=\alpha n$ using the Robin-Hood
insertion algorithm, the worst case search time has expected value
\[
	\E[W] = \Theta(\log\log n)
\]
and this bound is tight.
Thus, Robin-Hood hashing is a form of open addressing that
has doubly-logarithmic worst-case search time.  This makes it
competitive with the multiple-choice hashing method of
Section~\ref{hash:sec:multiplechoice}.

\subsection{Cuckoo Hashing}\label{hash:sec:cuckoo}
\index{hashing!cuckoo}
\index{cuckoo hashing}

Cuckoo hashing \cite{pr01} is a form of multiple choice hashing in
which each element $x$ lives in one of two tables $A$ or $B$, each of
size $m=n/\alpha$.  The element $x$ will either be stored at location
$A[x_A]$ or $B[x_B]$. There are no other options.  This makes
searching for $x$ an $O(1)$ time operation since we need only check
two array locations.

The insertion algorithm for cuckoo hashing proceeds as
follows:\footnote{The algorithm takes its name from the large but lazy
cuckoo bird which, rather than building its own nest, steals the nest
of another bird forcing the other bird to move on.} Store $x$ at
location $A[x_A]$. If $A[x_A]$ was previously occupied by some element
$y$ then store $y$ at location $B[y_B]$.  If $B[y_B]$ was previously
occupied by some element $z$ then store $z$ at location $A[z_A]$, and
so on.  This process ends when we place an element into a previously
empty table slot or when it has gone on for more than $c\log n$ steps.
In the former case, the insertion of $x$ completes successfully.  In
the latter case the insertion is considered a failure, and the entire
hash table is reconstructed from scratch using a new probe sequence
for each element in the table.  That is, if this reconstruction
process has happened $i$ times then the two hash values we use for an
element $x$ are $x_A=x_{2i}$ and $x_B=x_{2i+1}$.

Pagh and Rodler \cite{pr01} (see also Devroye and Morin \cite{dm02})
show that, during the insertion of $n$ elements, the probability of
requiring a reconstruction is $O(1/n)$. This, combined with the fact
that the expected insertion time is $O(1)$ shows that the expected
cost of $n$ insertions in a Cuckoo hashing table is $O(n)$.  Thus,
Cuckoo hashing offers a somewhat simpler alternative to the dynamic
perfect hashing algorithms of Section~\ref{hash:sec:dphash}.

\section{Historical Notes}

In this section we present some of the history of hash tables. The
idea of hashing seems to have been discovered simultaneously by two
groups of researchers. Knuth \cite{k97} cites an internal IBM
memorandum in January 1953 by H. P. Luhn that suggested the use of
hashing with chaining.  Building on Luhn's work, A. D. Linh suggested
a method of open addressing that assigns the probe sequence
$x_0,\floor{x_0/10},\floor{x_0/100},\floor{x_0/1000},\ldots$ to the
element $x$.

At approximately the same time, another group of researchers at IBM:
G. M. Amdahl, E.~M.~Boehme, N. Rochester and A. L. Samuel implemented
hashing in an assembly program for the IBM 701 computer.  Amdahl is
credited with the idea of open addressing with linear probing.

The first published work on hash tables was by A. I. Dumey \cite{d56},
who described hashing with chaining and discussed the idea of using
remainder modulo a prime as a hash function.  Ershov \cite{e58},
working in Russia and independently of Amdahl, described open
addressing with linear probing.

Peterson \cite{p57} wrote the first major article discussing the
problem of searching in large files and coined the term ``open
addressing.''  Buchholz \cite{b63} also gave a survey of the searching
problem with a very good discussion of hashing techniques at the time.
Theoretical analyses of linear probing were first presented by Konheim
and Weiss \cite{kw66} and Podderjugin.  Another, very influential,
survey of hashing was given by Morris \cite{m68}.  Morris' survey is
the first published use of the word ``hashing'' although it was
already in common use by practitioners at that time.



\section{Other Developments}

The study of hash tables has a long history and many researchers have
proposed methods of implementing hash tables.  Because of this, the
current chapter is necessarily incomplete. (At the time of writing,
the hash.bib bibliography on hashing contains over 800 entries.) We
have summarized only a handful of the major results on hash tables in
internal memory.  In this section we provide a few references to the
literature for some of the other results.   For more information on
hashing, Knuth \cite{k97}, Vitter and Flajolet \cite{vf90}, Vitter
and Chen \cite{vc87}, and Gonnet and Baeza-Yates \cite{gb-y91}
% and Flajolet \etal \cite{X} 
are useful references.

Brent's method (Section~\ref{hash:sec:brent}) is a collision
resolution strategy for open addressing that reduces the expected
search time for a successful search in a hash table with open
addressing.  Several other methods exist that either reduce the
expected or worst-case search time.  These include \emph{binary tree
hashing}\index{hashing!binary tree}\index{binary tree hashing}
\cite{m77, gm79}, \emph{optimal
hashing}\index{hashing!optimal}\index{optimal hashing}
\cite{p76,r78,gm79}, Robin-Hood hashing
(Section~\ref{hash:sec:robinhood}), and \emph{min-max
hashing}\index{hashing!min-max}\index{min-max hashing} \cite{g81,c86}.
One interesting method, due to Celis\index{Celis'
method}\index{hashing!Celis' method} \cite{c86}, applies to any open
addressing scheme.  The idea is to study the distribution of the ages
of elements in the hash table, i.e., the distribution give by 
\[
    D_i=\Pr\{\mbox{$x$ is stored at position $x_i$}\} 
\] 
and start searching for $x$ at the locations at which we are most
likely to find it, rather than searching the table positions
$x_0,x_1,x_2\ldots$ in order.

Perfect hash functions seem to have been first studied by Sprugnoli
\cite{s77} who gave some heuristic number theoretic constructions of
minimal perfect hash functions for small data sets.  Sprugnoli is
responsible for the terms ``perfect hash function'' and ``minimal
perfect hash function.''  A number of other researchers have presented
algorithms for discovering minimal and near-minimal perfect hash
functions.  Examples include Anderson and Anderson \cite{aa79},
Cichelli \cite{c80a,c80b}, Chang \cite{c84a,c84b}, Gori and Soda
\cite{gs89}, and Sager \cite{s85}.  Berman \etal\ \cite{bbdo86} and
K\"orner and Marton \cite{km88} discuss the theoretical limitations of
perfect hash functions.  A comprehensive, and recent, survey of
perfect hashing and minimal perfect hashing is given by Czech \etal\
\cite{chm97}.

Tarjan and Yao \cite{ty79} describe a set ADT implementation that
gives $O(\log u/\log n)$ worst-case access time.  It is obtained by
combining a trie (Section~\ref{external:32:tries}) of degree $n$ with a
compression scheme for arrays of size $n^2$ that contain only $n$
non-zero elements.  (The trie has $O(n)$ nodes each of which has $n$
pointers to children, but there are only a total of $O(n)$ children.)
Although their result is superseded by the results of Fredman \etal\
\cite{fks84} discussed in Section~\ref{hash:sec:fks}, they are the
first theoretical results on worst-case search time for hash tables.

Dynamic perfect hashing (Section~\ref{hash:sec:dphash}) and cuckoo
hashing (Section~\ref{hash:sec:cuckoo}) are methods of achieving
$O(1)$ worst case search time in a dynamic setting.  Several other
methods have been proposed  \cite{dm90,dgmp92,bm99}.
 
Yao \cite{y81} studies the \emph{membership problem}\index{membership
problem}.  Given a set $S\subseteq U$, devise a data structure that
can determine for any $x\in U$ whether $x$ is contained in $S$. Yao
shows how, under various conditions, this problem can be solved using
a very small number of memory accesses per query.  However, Yao's
algorithms sometimes derive the fact that an element $x$ is in $S$
without actually finding $x$.  Thus, they don't solve the set ADT
problem discussed at the beginning of this chapter since they can not
recover a pointer to $x$.

The ``power of two random choices,'' as used in multiple-choice
hashing, (Section~\ref{hash:sec:multiplechoice}) has many applications
in computer science.  Karp, Luby and Meyer auf der Heide
\cite{klm92,klm93} were the first to use this paradigm for simulating
PRAM computers on computers with fewer processors. The book chapter by
Mitzenmacher \etal\ \cite{mrs01} surveys results and applications of
this technique.

A number of table implementations have been proposed that are suitable
for managing hash tables in external memory.\index{hashing!external
memory}\index{external memory!hashing}  Here, the goal is to reduce
the number of disk blocks that must be accessed during an operation,
where a disk block can typically hold a large number of elements.
These schemes include \emph{linear
hashing}\index{hashing!linear}\index{linear hashing} \cite{l80},
\emph{dynamic hashing}\index{hashing!dynamic}\index{dynamic hashing}
\cite{l78}, \emph{virtual hashing}\index{virtual
hashing}\index{hashing!virtual} \cite{lit78}, \emph{extendible
hashing}\index{hashing!extendible}\index{extendible hashing}
\cite{fnps79},  \emph{cascade
hashing}\index{hashing!cascade}\index{cascade hashing} \cite{kz84},
and \emph{spiral storage}\index{spiral storage}\index{hashing!spiral
storage} \cite{m85}.  In terms of hashing, the main difference between
internal memory and external memory is that, in internal memory, an
array is allocated at a specific size and this can not be changed
later.  In contrast, an external memory file may be appended to or be
truncated to increase or decrease its size, respectively.  Thus, hash
table implementations for external memory can avoid the periodic
global rebuilding operations used in internal memory hash table
implementations.

\section*{Acknowledgement}

The author is supported by a grant from the Natural Sciences and
Engineering Research Council of Canada (NSERC).

%%\setcounter{\chapter}{10}
\chapter{Hash Tables}

\newcommand{\E}{\mathrm{E}}
\newcommand{\floor}[1]{\lfloor #1 \rfloor}
\newcommand{\ceil}[1]{\lceil #1 \rceil}

\newcommand{\etal}{et al.}

\begin{chapterauthors}
\chapterauthor{Pat Morin}{Carleton University}
\end{chapterauthors}

\index{hash table}

\section{Introduction}

A \emph{set abstract data type (set ADT)}\index{set ADT} is an
abstract data type that maintains a set $S$ under the following three
operations:

\begin{enumerate}

\item \textsc{Insert}$(x)$:  Add the key $x$ to the set.

\item \textsc{Delete}$(x)$: Remove the key $x$ from the set.

\item \textsc{Search}$(x)$: Determine if $x$ is contained in the set,
and if so, return a pointer to $x$. 

\end{enumerate}
One of the most practical and widely used methods of implementing the
set ADT is with \emph{hash tables}.  

Note that the three set ADT operations can easily be implemented to
run in $O(\log n)$ time per operation using balanced binary search
trees (See Chapter~\ref{external:12:BBST}).  If we assume that the
input data are integers in the set $U=\{0,\ldots,u-1\}$ then they can
even be implemented to run in $O(\log\log u)$ time using data
structures for integer searching
(Chapter~\ref{external:38?:int-searching}).  However, these data
structures actually do more than the three basic operations we
require.  In particular if we search for an element $x$ that is not
present in $S$ then these data structures can report the smallest item
in $S$ that is larger than $x$ (the \emph{successor} of $x$) and/or
the largest item in $S$ that is smaller than $x$ (the
\emph{predecessor} of $x$).  

Hash tables do away with this extra functionality of finding
predecessors and successors and only perform exact searches. If we
search for an element $x$ in a hash table and $x$ is not present then
the only information we obtain is that $x\notin S$. By dropping this
extra functionality hash tables can give better performance bounds.
Indeed, any reasonable hash table implementation performs each of the
three set ADT operations in $O(1)$ expected time.

The main idea behind all hash table implementations discussed in this
chapter is to store a set of $n=|S|$ elements in an array (the hash
table) $A$ of length $m\ge n$.  In doing this, we require a function
that maps any element $x$ to an array location.  This function is
called a \emph{hash function}\index{hash function} $h$ and the value
$h(x)$ is called the \emph{hash value} of $x$.  That is, the element
$x$ gets stored at the array location $A[h(x)]$.  The \emph{occupancy}
of a hash table is the ratio $\alpha=n/m$ of stored elements to the
length of $A$.  

The study of hash tables follows two very different lines. Many
implementations of hash tables are based on the \emph{integer universe
assumption}:\index{integer universe assumption} All elements stored in
the hash table come from the universe $U=\{0,\ldots,u-1\}$.  In this
case, the goal is to design a hash function $h:U\rightarrow
\{0,\ldots,m-1\}$ so that for each $i\in \{0,\ldots,m-1\}$, the number
of elements $x\in S$ such that $h(x)=i$ is as small as possible.
Ideally, the hash function $h$ would be such that each element of $S$
is mapped to a unique value in $\{0,\ldots,m-1\}$.  Most of the hash
functions designed under the integer universe assumption are
number-theoretic constructions. Several of these are described in
Section~\ref{hash:sec:integers}.

Historically, the integer universe assumption seems to have been
justified by the fact that any data item in a computer is represented
as a sequence of bits that can be interpreted as a binary number.
However, many complicated data items require a large (or variable)
number of bits to represent and this make $u$ the size of the universe
very large.  In many applications $u$ is much larger than the largest
integer that can fit into a single word of computer memory.  In this
case, the computations performed in number-theoretic hash functions
become inefficient.

This motivates the second major line of research into hash tables.
This research works is based on the \emph{random probing
assumption\emph{random probing assumption}}: Each element $x$ that is
inserted into a hash table is a black box that comes with an infinite
random \emph{probe sequence}\index{probe sequence}
$x_0,x_1,x_2,\ldots$ where each of the $x_i$ is independently and
uniformly distributed in $\{0,\ldots,m-1\}$.  Hash table
implementations based on the random probing assumption are described
in Section~\ref{hash:sec:uniform}.

Both the integer universe assumption and the random probing
assumption have their place in practice.  When there is an easily
computing mapping of data elements onto machine word sized integers
then hash tables for integer universes are the method of choice.  When
such a mapping is not so easy to compute (variable length strings are
an example) it might be better to use the bits of the input items to
build a good pseudorandom sequence and use this sequence as the probe
sequence for some random probing data structure. 

To guarantee good performance, many hash table implementations require
that the occupancy $\alpha$ be a constant strictly less than $1$.
Since the number of elements in a hash table changes over time, this
requires that the array $A$ be resized periodically.  This is easily
done, without increasing the amortized cost of hash table operations
by choosing three constants $0< \alpha_1 < \alpha_2 < \alpha_3 < 1$ so
that, whenever $n/m$ is not the interval $(\alpha_1,\alpha_3)$ the
array $A$ is resized so that its size is $n/\alpha_2$. A simple
amortization argument (Chapter~\ref{external:1:analysis}) shows that
the amortized cost of this resizing is $O(1)$ per update
(Insert/Delete) operation.


\section{Hash Tables for Integer Keys}
\label{hash:sec:integers}

In this section we consider hash tables under the integer universe
assumption, in which the key values $x$ come from the universe
$U=\{0,\ldots,u-1\}$.   A \emph{hash function}\index{hash function}
$h$ is a function whose domain is $U$ and whose range is the set
$\{0,\ldots,m-1\}$, $m\le u$.  A hash function $h$ is said to be a
\emph{perfect hash function}\index{hash
function!perfect}\index{perfect hash function} for a set $S\subseteq
U$ if, for every $x\in S$, $h(x)$ is unique.  A perfect hash function
$h$ for $S$ is \emph{minimal}\index{hash function!minimal
perfect}\index{minimal perfect hash function} if $m=|S|$, i.e., $h$ is
a bijection between $S$ and $\{0,\ldots,m-1\}$.  Obviously a minimal
perfect hash function for $S$ is desirable since it allows us to store
all the elements of $S$ in a single array of length $n$.
Unfortunately, perfect hash functions are rare, even for $m$ much
larger than $n$.  If each element of $S$ is mapped independently and
uniformly to a random element of $\{0,\ldots,m-1\}$ then the birthday
paradox (See, for example, Feller~\cite{f68}) states that, if $m$ is
much less than $n^2$ then there will almost surely exist two elements
of $S$ that have the same hash value.

We begin our discussion with two commonly used hashing schemes that
are heuristic in nature.  That is, we can not make any non-trivial
statements about the performance of these schemes when storing an
arbitrary set $S$.  We then discuss several schemes that have provably
good performance.

\subsection{Hashing by Division}
\index{hashing!by division}

In \emph{hashing by division}, we use the hash function
\[ h(x) = x\bmod m \enspace . \]
To use this hash function in a data structure, we maintain an array
$A[0],\ldots,A[m-1]$ where each element of this array is a pointer to
the head of a linked list (Section~\ref{external:2:linked-list}).  The
linked list $L_i$ pointed to by the array element $A[i]$ contains all
the elements $x$ such that $h(x)=i$.  This technique of maintaining an
array of lists is called \emph{hashing with
chaining}\index{hashing!with chaining}.

In such a hash table, inserting an element $x$ takes $O(1)$ time;  we
compute $i=h(x)$ and append (or prepend) $x$ to the list $L_i$.
However, searching for and/or deleting an element $x$ is not so easy.
We have to compute $i=h(x)$ and then traverse the list $L_i$ until we
either find $x$ or reach the end of the list.  The cost of this is
proportional to the length of $L_i$.  Obviously, if our set $S$
consists of the elements $0,m,2m,3m,\ldots,nm$ then all elements are
stored in the list $L_0$ and searches and deletions take linear time.

However, one hopes that such pathological cases do not occur in
practice. For example, if the elements of $S$ are uniformly and
independently distributed in $U$ and $u$ is a multiple of $m$ then the
expected size of any list $L_i$ is only $n/m$.  In this case, searches
and deletions take $O(1+\alpha)$ expected time.  To help avoid
pathological cases, the choice of $m$ is important.  In particular,
$m$ a power of 2 is usually avoided since, in a binary computer,
taking the remainder modulo a power of 2 means simply discarding some
high-order bits.  Taking $m$ to be a prime not too close to a power of
2 is recommended \cite{k97}.

\subsection{Hashing by Multiplication}
\index{hashing!by multiplication}

The implementation of a hash table using \emph{hashing by
multiplication} is exactly the same as that hashing by division
except that the hash function 

\[
  h(x) = \floor{mxA} \bmod m
\]
is used.  Here $A$ is a real-valued constant whose choice we discuss
below. The advantage of the multiplication method is that the value of
$m$ is not critical.  We can take $m$ to be a power of 2, which makes
it convenient for use on binary computers. 

Although any value of $A$ gives a hash function, some values of $A$
are better than others.  (Setting $A=0$ is clearly not a good idea.)

Knuth \cite{k97} suggests using the \emph{golden ratio}\index{golden
ratio} for $A$, i.e.,
setting 
\[  
    A=(\sqrt{5}-1)/2=0.6180339887\ldots
\]   
This choice of $A$ is motivated by a theorem, first conjectured by
Oderfeld and later proven by S\'wierczkowski \cite{s58}.  This theorem
states that the sequence 
\[
     mA \bmod m,\,\, 2mA \bmod m,\,\, 3mA \bmod m,\,\,\ldots,\,\, nmA\bmod m
\] 
partitions the interval $(0,m)$ into $n+1$ intervals having only three
distinct lengths.  Furthermore, the next element $(n+1)mA\bmod m$ in
the sequence is always contained in one of the largest
intervals.\footnote{In fact, any irrational number has this property
\cite{sos57}.  The golden ratio is especially good because it is
not too close to a whole number.}

Of course, no matter what value of $A$ we select, the pigeonhole
principle implies that for $u\ge nm$ then there will always exist some
hash value $i$ and some $S\subseteq U$ of size $n$ such that $h(x)=i$
for all $x\in S$.  In other words, we can always find a set $S$ all of
whose elements get stored in the same list $L_i$. Thus, the worst case
of hashing by multiplication is as bad as hashing by division.


\subsection{Universal Hashing}
\label{hash:sec:universal}
\index{hashing!universal}
\index{universal hashing}

The argument used at the end of the previous section applies equally
well to any hash function $h$.  That is, if the table size $m$ is much
smaller than the universe size $u$ then for any hash function there is
some large (of size at least $\ceil{u/m}$) subset of $U$ that has the
same hash value.  To get around this difficulty we need a collection
of hash functions of which we can choose one that works well for $S$.
Even better would be a collection of hash functions such that, for any
given $S$, most of the hash functions work well for $S$.  Then we
could simply pick one of the functions at random and have a good
chance of it working well.

Let $\mathcal{H}$ be a collection of hash functions, i.e., functions
from $U$ onto $\{0,\ldots,m-1\}$.  We say that $\mathcal{H}$ is
\emph{universal}\index{hash function!universal} if, for each $x,y\in
U$ the number of $h\in \mathcal{H}$ such that $h(x)=h(y)$ is at most
$|\mathcal{H}|/m$.  Consider any $S\subseteq U$ of size $n$ and
suppose we choose a random hash function $h$ from a universal
collection of hash functions.  Consider some value $x\in U$.  The
probability that any key $y\in S$ has the same hash value as $x$ is
only $1/m$.  Therefore, the expected number of keys in $S$, not equal
to $x$, that have the same hash value as $x$ is only \[ n_{h(x)} =
\left\{\begin{array}{ll} (n-1)/m & \mbox{if $x\in S$} \\ n/m &
\mbox{if $x\notin S$} \end{array}\right.  \] Therefore, if we store
$S$ in a hash table using the hash function $h$ then the expected time
to search for, or delete, $x$ is $O(1+\alpha)$.

From the preceding discussion, it seems that a universal collection of
hash functions from which we could quickly select one at random would be
very handy indeed. With such a collection at our disposal we get an
implementation of the set ADT that has $O(1)$ insertion time and
$O(1)$ expected search and deletion time.

Carter and Wegman \cite{cw79} describe three different collections of
universal hash functions.  If the universe size $u$ is a
prime number\footnote{This is not a major restriction since, for any
$u>1$, there always exists a prime number in the set
$\{u,u+1,\ldots,2u\}$.  Thus we can enforce this assumption by
increasing the value of $u$ by a constant factor.} then
\[
  \mathcal{H} = \{ h_{k_1,k_2,m}(x)=((k_1x+k_2)\bmod u))\bmod m : 1\le k_1< u,
0\le k_2< u\} 
\]
is a collection of universal hash functions.  Clearly, choosing a
function uniformly at random from $\mathcal{H}$ can be done easily by
choosing two random values $k_1\in \{1,\ldots,u-1\}$ and $k_2\in
\{0,\ldots,u-1\}$.  Thus, we have an implementation of the set ADT with
$O(1)$ expected time per operation.

 
\subsection{Static Perfect Hashing}
\index{perfect hashing!static}
\index{hashing!static perfect}
\index{static perfect hashing}
\label{hash:sec:fks}

The result of Carter and Wegman on universal hashing is very strong,
and from a practical point of view, it is probably the strongest
result most people will ever need.  The only thing that could be
improved about their result is to make it deterministic, so that the
running times of all operations are $O(1)$ \emph{worst-case}.
Unfortunately, this is not possible, as shown by Dietzfelbinger \etal\
\cite{dkmm94}.  

Since there is no hope of getting $O(1)$ worst-case time for all three
set ADT operations, the next best thing would be to have searches that
take $O(1)$ worst-case time.  In this section we describe the method
of Fredman, Koml\'os and Szemer\'edi \cite{fks84}.  This is a static
data structure that takes as input a set $S\subseteq U$ and builds a
data structure of size $O(n)$ that can test if an element $x$ is in
$S$ in $O(1)$ worst-case time.  Like the universal hash functions from
the previous section, this method also requires that $u$ be a prime
number. This scheme uses hash functions of the form
\[
  h_{k,m}(x) = (kx\bmod u))\bmod m \enspace .\footnote{Actually, it
turns out that any universal hash function also works in the FKS
scheme \cite[Section~11.5]{clrs01}.}
\]

Let $B_{k,m}(S,i)$ be the number of elements $x\in S$ such that
$h_{k,m}(x)=i$, i.e., the number of elements of $S$ that have hash
value $i$ when using the hash function $h_{k,m}$.  The function
$B_{k,m}$ gives complete information about the distribution of hash
values of $S$.  The main lemma used by Fredman \etal\ is that, if we
choose $k\in U$ uniformly at random then
\begin{equation}
\E\left[\sum_{i=0}^{m-1} {B_{k,m}(S,i)\choose 2}\right] < \frac{n^2}{m}  
  \enspace . \label{hash:eq:fks-main}
\end{equation}
There are two important special cases of this result.

In the \emph{sparse case} we take $m=n^2/\alpha$, for some constant
$0<\alpha<1$.  In this case, the expectation in
(\ref{hash:eq:fks-main}) is less than $\alpha$.  Therefore, by
Markov's inequality, the probability that this sum is greater than or
equal to 1 is at most $\alpha$.  But, since this sum is a non-negative
integer, then with probability at least $1-\alpha$ it must be equal to
0.  In other words, with probability at least $1-\alpha$,
$B_{k,m}(S,i)\le 1$ for all $0\le i\le m-1$, i.e., the hash function
$h_{k,m}$ is perfect for $S$.  Of course this implies that we can find
a perfect hash function very quickly by trying a small number of
random elements $k\in U$ and testing if they result in perfect hash
functions. (The expected number of elements that we will have to try
is only $1/(1-\alpha)$.) Thus, if we are willing to use quadratic
space then we can perform searches in $O(1)$ worst-case time.

In the \emph{dense case} we assume that $m$ is close to $n$ and
discover that, for many values of $k$, the hash values are distributed
fairly evenly among the set $1,\ldots,m$.  More precisely, if we use a
table of size $m=n$, then 
\[
   \E \left[ \sum_{i=0}^{m-1} B_{k,m}(S,i)^2 \right] \le 3n \enspace .
      \label{hash:eq:fks-space-a}
\]
By Markov's inequality this means that
\begin{equation}
   \Pr\left\{ \sum_{i=0}^{m-1} B_{k,m}(S,i)^2  \le 3n/\alpha \right\}
\ge
	1-\alpha \enspace .
      \label{hash:eq:fks-space}
\end{equation}
Again, we can quickly find a value of $k$ satisfying
(\ref{hash:eq:fks-space}) by testing a few randomly chosen values of
$k$.

These two properties are enough to build a two-level data structure
that uses linear space and executes searches in worst-case constant
time.  We call the following data structure the FKS-$\alpha$ data
structure, after its inventors Fredman, Koml\'os and Szemer\'edi.  At
the top level, the data structure consists of an array
$A[0],\ldots,A[m-1]$ where $m=n$.  The elements of this array are
pointers to other arrays $A_0,\ldots,A_{m-1}$, respectively.  To
decide what will be stored in these other arrays, we build a hash
function $h_{k,m}$ that satisfies the conditions of
(\ref{hash:eq:fks-space}).  This gives us the top-level hash function
$h_{k,m}(x)=(kx\bmod u)\bmod m$. Each element $x\in S$ gets stored in
the array pointed to by $A[h_{k,m}(x)]$.

What remains is to describe how we use the arrays
$A_0,\ldots,A_{m-1}$.  Let $S_i$ denote the set of elements $x\in S$
such that $h_{k,m}(s)=i$.  The elements of $S_i$ will be stored in
$A_i$.  The size of $S_i$ is $n_i=B_{k,m}(S,i)$.  To store the
elements of $S_i$ we set the size of $A_i$ to
$m_i=n_i{}^2/\alpha=B_{k,n}(S,i)^2/\alpha$.  Observe that, by
(\ref{hash:eq:fks-space}), all the $A_i$'s take up a total space of
$O(n)$, i.e., $\sum_{i=0}^{m-1} m_i = O(n)$.  Furthermore, by trying a
few randomly selected integers we can quickly find a value $k_i$ such
that the hash function $h_{k_i,m_i}$ is perfect for $S_i$.  Therefore,
we store the element $x\in S_i$ at position $A_i[h_{k_i,m_i}(x)]$ and
$x$ is the unique element stored at that location.  With this scheme
we can search for any value $x\in U$ by computing two hash values
$i=h_{k,m}(x)$ and $j=h_{k_i,m_i}(x)$ and checking if $x$ is stored in
$A_i[j]$.

Building the array $A$ and computing the values of
$n_0,\ldots,n_{m-1}$ takes $O(n)$ expected time since for a given
value $k$ we can easily do this in $O(n)$ time and the expected number
of values of $k$ that we must try before finding one that satisfies
(\ref{hash:eq:fks-space}) is $O(1)$.  Similarly, building each subarray
$A_i$ takes $O(n_i{}^2)$ expected time, resulting in an overall expected
running time of $O(n)$.  Thus, for any constant $0<\alpha<1$, an
FKS-$\alpha$ data structure can be constructed in $O(n)$ expected time
and this data structure can execute a search for any $x\in U$ in
$O(1)$ worst-case time.

\subsection{Dynamic Perfect Hashing}
\label{hash:sec:dphash}
\index{hashing!dynamic perfect}
\index{dynamic perfect hashing}
\index{perfect hashing!dynamic}
The FKS-$\alpha$ data structure is nice in that it allows for searches
in $O(1)$ time, in the worst case.  Unfortunately, it is only static;
it does not support insertions or deletions of elements.  In this
section we describe a result of Dietzfelbinger \etal\ \cite{dkmm94}
that shows how the FKS-$\alpha$ data structure can be made dynamic
with some judicious use of partial rebuilding
(Section~\ref{external:12:Andersson-trees-partial-rebuilding}).

The main idea behind the scheme is simple: be lazy at both the upper
and lower levels of the FKS-$\alpha$ data structure.  That is, rebuild
parts of the data structure only when things go wrong.  At the top
level, we relax the condition that the size $m$ of the upper array $A$
is exactly $n$ and allow $A$ to have size anywhere between $n$ and
$2n$.  Similarly, at the lower level we allow the array $A_i$ to have
a size $m_i$ anywhere between $n_i{}^2/\alpha$ and $2n_i{}^2/\alpha$.

Periodically, we will perform a \emph{global rebuilding}\index{global
rebuilding} operation in
which we remove all $n$ elements from the hash table.  Some elements
which have previously been marked as deleted will be discarded,
thereby reducing the value of $n$.  We put the remaining elements in a
list, and recompute a whole new FKS-$(\alpha/2)$ data structure for
the elements in the list.  This data structure is identical to the
standard FKS-$(\alpha/2)$ data structure except that, at the top level
we use an array of size $m=2n$.

Searching in this data structure is exactly the same as for the static
data structure.  To search for an element $x$ we compute
$i=h_{k,m}(x)$ and $j=h_{k_i,m_i}(x)$ and look for $x$ at location
$A_i[j]$.   Thus, searches take $O(1)$ worst-case time.

Deleting in this data structure is done in the laziest manner
possible.  To delete an element we only search for it and then mark it
as deleted.  We will use the convention that this type of deletion
does not change the value of $n$ since it does not change the number
of elements actually stored in the data structure.  While doing this,
we also keep track of the number of elements that are marked as
deleted.  When this number exceeds $n/2$ we perform a global
rebuilding operation.  The global rebuilding operation takes $O(n)$
expected time, but only occurs during one out of every $n/2$
deletions.  Therefore, the amortized cost of this operation is $O(1)$
per deletion.

The most complicated part of the data structure is the insertion
algorithm and its analysis.  To insert a key $x$ we know, because of
how the search algorithm works, that we must ultimately store $x$ at
location $A_i[j]$ where $i=h_{k,m}(x)$ and $j=h_{k_i,m_i}(x)$.
However, several things can go wrong during the insertion of $x$:

\begin{enumerate} 
\item The value of $n$ increases by 1, so it may be that $n$ now
exceeds $m$.  In this case we perform a global rebuilding
operation and we are done.

\item We compute $i=h_{k,m}(x)$ and discover that $\sum_{i=0}^{m-1}
n_i{}^2 > 3n/\alpha$.  In this case, the hash function
$h_{k,m}$ used at the top level is no longer any good since it is
producing an overall hash table that is too large.  In this case we
perform a global rebuilding operation and we are done.

\item We compute $i=h_{k,m}(x)$ and discover that, since the value of
$n_i$ just increased by one, $n_i{}^2/\alpha > m_i$. In this case, the
array $A_i$ is too small to guarantee that we can quickly find a
perfect hash function.  To handle this, we copy the elements of $A_i$
into a list $L$ and allocate a new array $A_i$ with the new size
$m_i=2n_i{}^2/\alpha$.  We then find a new value $k_i$ such that
$h_{k_i,m_i}$ is a perfect hash function for the elements of $L$ and
we are done.

\item The array location $A_i[j]$ is already occupied by some other
element $y$.  But in this case, we know that $A_i$ is large enough to
hold all the elements (otherwise we would already be done after
Case~3), but the value $k_i$ being used in the hash function
$h_{k_i,m_i}$ is the wrong one since it doesn't give a perfect hash
function for $S_i$.  Therefore we simply try new values for $k_i$
until we find a find a value $k_i$ that yields a perfect hash function
and we are done.  
\end{enumerate}

If none of the preceding 4 cases occurs then we can simply place $x$
at location $A_i[j]$ and we are done.

Handling Case~1 takes $O(n)$ expected time since it involves a global
rebuild of the entire data structure.  However, Case~1 only happens
during one out of every $\Theta(n)$ insertions, so the amortized cost
of all occurrences of Case~1 is only $O(1)$ per insertion.

Handling Case~2 also takes $O(n)$ expected time.  The question is: How
often does Case~2 occur?  To answer this question, consider the
\emph{phase}\index{phase} that occurs between two consecutive
occurrences of Case~1.  During this phase, the data structure holds at
most $m$ distinct elements.  Call this set of elements $S$.  With
probability at least $(1-\alpha)$ the hash function $h_{k,m}$ selected
at the beginning of the phase satisfies (\ref{hash:eq:fks-space}) so
that Case~2 never occurs during the phase.  Similarly, the probability
that Case~2 occurs exactly once during the phase is at most
$\alpha(1-\alpha)$.  In general, the probability that Case~2 occurs
exactly $i$ times during a phase is at most $\alpha^i(1-\alpha)$.  Thus, the
expected cost of handling all occurrences of Case~2 during the entire
phase is at most 
\[ \sum_{i=0}^\infty \alpha^i(1-\alpha)i\times O(n) =
	O(n) \enspace .  
\] 
But since a phase involves $\Theta(n)$ insertions this means that the
amortized expected cost of handling Case~2 is $O(1)$ per insertion.

Next we analyze the total cost of handling Case~3.  Define a
\emph{subphase}\index{subphase} as the period of time between two
global rebuilding operations triggered either as a result of a
deletion, Case~1 or Case~2.   We will show that the total cost of
handling all occurrences of Case~3 during a subphase is $O(n)$ and
since a subphase takes $\Theta(n)$ time anyway this does not
contribute to the cost of a subphase by more than a constant factor.
When Case~3 occurs at the array $A_i$ it takes $O(m_i)$ time.
However, while handling Case~3, $m_i$ increases by a constant factor,
so the total cost of handling Case~3 for $A_i$ is dominated by the
value of $m_i$ at the end of the subphase.  But we maintain the
invariant that $\sum_{i=0}^{m-1} m_i = O(n)$ during the entire
subphase.  Thus, handling all occurrences of Case~3 during a subphase
only requires $O(n)$ time. 

Finally, we consider the cost of handling Case~4.  For a particular
array $A_i$, consider the \emph{subsubphase}\index{subsubphase}
between which two occurrences of Case~3 cause $A_i$ to be rebuilt or a
global rebuilding operation takes place.  During this subsubphase the
number of distinct elements that occupy $A_i$ is at most
$\alpha\sqrt{m_i}$.  Therefore, with probability at least $1-\alpha$
any randomly chosen value of $k_i\in U$ is a perfect hash function for
this set.   Just as in the analysis of Case~2, this implies that the
expected cost of handling all occurrences of Case~3 at $A_i$ during a
subsubphase is only $O(m_i)$.  Since a subsubphase ends with
rebuilding all of $A_i$ or a global rebuilding, at a cost of
$\Omega(m_i)$ all the occurrences of Case~4 during a subsubphase do not
contribute to the expected cost of the subsubphase by more than a
constant factor.

To summarize, we have shown that the expected cost of handling all
occurrences of Case~4 is only  a constant factor times the cost of
handling all occurrences of Case~3.  The cost of handling all
occurrences of Case~3 is no more than a constant factor times the
expected cost of all global rebuilds.  The cost of handling all the
global rebuilds that occur as a result of Case~2 is no more than a
constant factor times the cost of handling all occurrences of global
rebuilds that occur as a consequence of Case~1.  And finally, the cost
of all global rebuilds that occur as a result of Case~1 or of
deletions is $O(n)$ for a sequence of $n$ update operations.
Therefore, the total expected cost of $n$ update operation is $O(n)$.


\section{Random Probing}
\label{hash:sec:uniform}
\index{random probing}
\index{probing!random}

Next we consider hash table implementations under the random probing
assumption: Each element $x$ stored in the hash table comes with a
random sequence $x_0,x_1,x_2,\ldots$ where each of the $x_i$ is
independently and uniformly distributed in
$\{1,\ldots,m\}$.\footnote{A variant of the random probing assumption,
referred to as the \emph{uniform hashing}\index{uniform
hashing}\index{hashing!uniform} assumption, assumes that
$x_0,\ldots,x_{m-1}$ is a random permutation of $0,\ldots,m-1$.}  We
begin with a discussion of the two basic paradigms: hashing with
chaining and open addressing.  Both these paradigms attempt to store
the key $x$ at array position $A[x_0]$.  The difference between these
two algorithms is their \emph{collision resolution
strategy}\index{collision resolution strategy}, i.e., what the
algorithms do when a user inserts the key value $x$ but array position
$A[x_0]$ already contains some other key.

\subsection{Hashing with Chaining}
\index{hashing!with chaining}

In \emph{hashing with chaining}, a collision is resolved by allowing
more than one element to live at each position in the table.  Each
entry in the array $A$ is a pointer to the head of a linked list.  To
insert the value $x$, we simply append it to the list $A[x_0]$.  To
search for the element $x$, we perform a linear search in the list
$A[x_0]$. To delete the element $x$, we search for $x$ in the list
$A[x_0]$ and splice it out.

It is clear that insertions take $O(1)$ time, even in the worst case.
For searching and deletion, the running time is proportional to a
constant plus the length of the list stored at $A[x_0]$.  Notice that
each of the at most $n$ elements not equal to $x$ is stored in
$A[x_0]$ with probability $1/m$, so the expected length of $A[x_0]$ is
either $\alpha=n/m$ (if $x$ is not contained in the table) or
$1+(n-1)/m$ (if $x$ is contained in the table). Thus, the expected
cost of searching for or deleting an element is $O(1+\alpha)$.

The above analysis shows us that hashing with chaining supports the
three set ADT operations in $O(1)$ expected time per operation, as
long as the occupancy, $\alpha$, is a constant. It is worth noting
that this does not require that the value of $\alpha$ be less than 1. 

If we would like more detailed information about the cost of searching, we
might also ask about the \emph{worst-case search time} defined as
\[
   W = \max\{\mbox{length of the list stored at $A[i]$} : 0\le i\le m-1 \} 
	\enspace .
\]
It is very easy to prove something quite strong about $W$ using only
the fact that the length of each list $A[i]$ is a
$\mathrm{binomial}(n,1/m)$ random variable.  Using Chernoff's bounds
on the tail of the binomial distribution \cite{c52}, this
immediately implies that   
\[
   \Pr\{\mbox{length of $A[i]$} \ge \alpha c\ln n\} \le n^{-\Omega(c)} \enspace .
\]
Combining this with Boole's inequality ($\Pr\{\mbox{$A$ or $B$}\} \le
\Pr\{A\}+\Pr\{B\}$) we obtain
\[
   \Pr\{W \ge \alpha c\ln n\} \le n\times n^{-\Omega(c)} = n^{-\Omega(c)} \enspace .
\]

Thus, with very high probability, the worst-case search time is
logarithmic in $n$.  This also implies that $\E[W]=O(\log n)$.  The
distribution of $W$ has been carefully studied and it is known that,
\emph{with high probability}\index{high probability}, i.e., with
probability $1-o(1)$, $W=(1+o(1))\ln n/\ln\ln n$
\cite{jk77,ksc78}.\footnote{Here, and throughout this chapter, if an
asymptotic notation does not contain a variable then the variable that
tends to infinity is implicitly $n$.  Thus, for example, $o(1)$ is the
set of non-negative functions of $n$ that tend to 0 as
$n\rightarrow\infty$.} Gonnet has proven a more accurate result that
$W=\Gamma^{-1}(n) -3/2 + o(1)$ with high probability. Devroye
\cite{d85} shows that similar results hold even when the distribution
of $x_0$ is not uniform.

\subsection{Hashing with Open Addressing}
\index{open addressing}
\index{hashing!with open addressing}

\label{hash:sec:open-addressing}

\emph{Hashing with open addressing} differs from hashing with chaining in
that each table position $A[i]$ is allowed to store only one value.
When a collision occurs at table position $i$, one of the two elements
involved in the collision must move on to the next element in its
probe sequence.  In order to implement this efficiently and correctly
we require a method of marking elements as deleted.  This method could
be an auxiliary array that contains one bit for each element of $A$,
but usually the same result can be achieved by using a special key
value $\mathbf{del}$ that does not correspond to any valid key.

To search for an element $x$ in the hash table we look for $x$ at
positions $A[x_0]$, $A[x_1]$, $A[x_2]$, and so on until we either
(1)~find $x$, in which case we are done or (2)~find an empty table
position $A[x_i]$ that is not marked as deleted, in which case we can
be sure that $x$ is not stored in the table (otherwise it would be
stored at position $x_i$).  To delete an element $x$ from the hash
table we first search for $x$.  If we find $x$ at table location
$A[x_i]$ we then simply mark $A[x_i]$ as deleted.  To insert a value
$x$ into the hash table we examine table positions $A[x_0]$, $A[x_1]$,
$A[x_2]$, and so on until we find a table position $A[x_i]$ that is
either empty or marked as deleted and we store the value $x$ in
$A[x_i]$.

Consider the cost of inserting an element $x$ using this method.  Let
$i_x$ denote the smallest value $i$ such that $x_{i_x}$ is either empty or
marked as deleted when we insert $x$.  Thus, the cost of inserting $x$
is a constant plus $i_x$.  The probability that the table position
$x_0$ is occupied is at most $\alpha$ so, with probability at least
$1-\alpha$, $i_x=0$.  Using the same reasoning, the probability
that we store $x$ at position $x_i$ is at most 
\begin{equation}
   \Pr\{i_x = i\} \le \alpha^{i}(1-\alpha)  \label{hash:eq:pro}
\end{equation}
since the table locations $x_0,\ldots,x_{i-1}$ must be occupied,
the table location $x_i$ must not be occupied and the $x_i$ are
independent.  Thus, the expected number of steps taken by the
insertion algorithm is 
\[
   \sum_{i=1}^\infty i\Pr\{i_x = i\} = (1-\alpha)\sum_{i=1}^\infty i\alpha^{i-1}  = 1/(1-\alpha)
\]
for any constant $0<\alpha< 1$.  The cost of searching for $x$ and
deleting $x$ are both proportional to the cost of inserting $x$, so
the expected cost of each of these operations is
$O(1/(1-\alpha))$.\footnote{Note that the expected cost of searching
for or deleting an
element $x$ is proportional to the value of $\alpha$ \emph{at the time
$x$ was inserted.}  If many deletions have taken place, this may be
quite different than the current value of $\alpha$.}   

We should compare this with the cost of hashing with chaining.  In
hashing with chaining,the occupancy $\alpha$ has very little effect on
the cost of operations.  Indeed, any constant $\alpha$, even greater
than $1$ results in $O(1)$ time per operation.  In contrast, open
addressing is very dependent on the value of $\alpha$.  If we take
$\alpha>1$ then the expected cost of insertion using open addressing
is infinite since the insertion algorithm never finds an empty table
position.  Of course, the advantage of hashing with chaining is that
it does not require lists at each of the $A[i]$.  Therefore, the
overhead of list pointers is saved and this extra space can be used
instead to maintain the invariant that the occupancy $\alpha$ is a
constant strictly less than 1.

Next we consider the worst case search time of hashing with open
addressing.  That is, we study the value $W=\max\{i_x: \mbox{$x$ is
stored in the table at location $i_x$}\}$.  Using (\ref{hash:eq:pro}) and Boole's
inequality it follows almost immediately that 
\[
  \Pr\{W > c\log n\} \le n^{-\Omega(c)} .
\]

Thus, with very high probability, $W$, the worst case search time, is
$O(\log n)$.  Tighter bounds on $W$ are known when the probe sequences
$x_0,\ldots,x_{m-1}$ are random permutations of $0,\ldots,m-1$.  In this
case, Gonnet shows that
\[
  \E[W]=\log_{1/\alpha} n - \log_{1/\alpha}(\log_{1/\alpha} n) + O(1)
\]
\cite{g81}. 

Open addressing under the random probing assumption has many nice
theoretical properties and is easy to analyze.  Unfortunately, it is
often criticized as being an unrealistic model because it requires a
long random sequences $x_0,x_1,x_2,\ldots$ for each element $x$ that
is to be stored or searched for.  Several variants of open addressing
discussed in the next few sections try to overcome this problem by
using only a few random values. 


\subsection{Linear Probing}
\index{linear probing}
\index{hashing!with linear probing}
 
\emph{Linear probing} is a variant of open addressing that requires less
randomness.  To obtain the probe sequence $x_0,x_1,x_2,\ldots$ we
start with a random element $x_0\in \{0,\ldots,m-1\}$.  The element
$x_i$, $i>0$ is given by $x_i=(i+x_0)\bmod m$.  That is, one first
tries to find $x$ at location $x_0$ and if that fails then one looks
at $(x_0+1)\bmod m$, $(x_0+2)\bmod m$ and so on.  

The performance of linear probing is discussed by Knuth \cite{k97} who
shows that the expected number of probes performed during an
unsuccessful search is at most 
\[
	(1+1/(1-\alpha)^2)/2
\] 
and the expected
number of probes performed during a successful search is at most
\[
	(1+1/(1-\alpha))/2 \enspace . 
\] 
This is not quite as good as for standard hashing with open
addressing, especially in the unsuccessful case.

Linear probing suffers from the problem of \emph{primary
clustering}\index{primary clustering}\index{clustering!primary}.  If
$j$ consecutive array entries are occupied then a newly inserted
element will have probability $j/m$ of hashing to one of these
entries. This results in $j+1$ consecutive array entries being
occupied and increases the probability (to $(j+1)/m$) of another newly
inserted element landing in this cluster.  Thus, large clusters of
consecutive elements have a tendency to grow larger.

\subsection{Quadratic Probing}
\index{quadratic probing}
\index{hashing!with quadratic probing}

\emph{Quadratic probing} is similar to linear probing;  an element $x$
determines its entire probe sequence based on a single random choice,
$x_0$.  Quadratic probing uses the probe sequence
$x_0,(x_0+k_1+k_2)\bmod m, (x_0+2k_1+2^2k_2)\bmod m,\ldots$.  In
general, the $i$th element in the probe sequence is
$x_i=(x_0+ik_1+i^2k_2)\bmod m$.  Thus, the final location of an
element depends quadratically on how many steps were required to
insert it.  This method seems to work much better in practice than
linear probing, but requires a careful choice of $m$, $k_1$ and $k_2$
so that the probe sequence contains every element of $\{0,\ldots,m-1\}$.

The improved performance of quadratic probing is due to the fact that
if there are two elements $x$ and $y$ such that $x_i=y_j$ then it is
not necessarily true (as it is with linear probing) that
$x_{i+1}=y_{j+1}$.  However, if $x_0=y_0$ then $x$ and $y$ will have
exactly the same probe sequence.  This lesser phenomenon is called
\emph{secondary clustering}\index{secondary
clustering}\index{clustering!secondary}.  Note that this secondary
clustering phenomenon implies that neither linear nor quadratic
probing can hope to perform any better than hashing with chaining.
This is because all the elements that have the same initial hash $x_0$
are contained in an implicit chain.  In the case of linear probing,
this chain is defined by the sequence $x_0,x_0+1,x_0+2,\ldots$ while
for quadratic probing it is defined by the sequence
$x_0,x_0+k_1+k_2,x_0+2k_1+4k_2,\ldots$

\subsection{Double Hashing}\label{hash:sec:double}
\index{hashing!double}
\index{double hashing}

\emph{Double hashing} is another method of open addressing that uses two hash
values $x_0$ and $x_1$.  Here $x_0$ is in the set $\{0,\ldots,m-1\}$
and $x_1$ is in the subset of $\{1,\ldots,m-1\}$ that is relatively
prime to $m$.  With double hashing, the probe sequence for element $x$
becomes $x_0,(x_0+x_1)\bmod m, (x_0+2x_1)\bmod m,\ldots$.  In general,
$x_i=(x_0+ix_1)\bmod m$, for $i>0$.  The expected number of probes
required by double hashing seems difficult to determine exactly.
Guibas has proven that, asymptotically, and for occupancy $\alpha\le
.31$, the performance of double hashing is asymptotically equivalent
to that of uniform hashing.  Empirically, the performance of double
hashing matches that of open addressing with random probing
regardless of the occupancy $\alpha$ \cite{k97}.

\subsection{Brent's Method}\label{hash:sec:brent}
\index{Brent's method}
\index{hashing!Brent's method}

\emph{Brent's method} \cite{b73} is a heuristic that attempts to
minimize the average time for a successful search in a hash table with
open addressing.  Although originally described in the context of
double hashing (Section~\ref{hash:sec:double}) Brent's method applies
to any open addressing scheme.  The \emph{age}\index{age} of an element $x$
stored in an open addressing hash table is the minimum value $i$ such
that $x$ is stored at $A[x_i]$.  In other words, the age is one less
than the number of locations we will probe when searching for $x$. 

Brent's method attempts to minimize the total age of all elements in
the hash table.  To insert the element $x$ we proceed as follows:  We
find the smallest value $i$ such that $A[x_i]$ is empty; this is where
standard open-addressing would insert $x$.  Consider the element $y$
stored at location $A[x_{i-2}]$.  This element is stored there because
$y_j=x_{i-2}$, for some $j\ge 0$.  We check if the array location
$A[y_{j+1}]$ is empty and, if so, we move $y$ to location $A[y_{j+1}]$
and store $x$ at location $A[x_{i-2}]$.  Note that, compared to
standard open addressing, this decreases the total age by 1.  In
general, Brent's method checks, for each $2\le k\le i$ the array entry
$A[x_{i-k}]$ to see if the element $y$ stored there can be moved to
any of $A[y_{j+1}],A[y_{j+2}],\ldots,A[y_{j+k-1}]$ to make room for
$x$.  If so, this represents a decrease in the total age of all
elements in the table and is performed.

Although Brent's method seems to work well in practice, it is
difficult to analyze theoretically.  Some theoretical analysis of
Brent's method applied to double hashing is given by Gonnet and Munro
\cite{gm79}.  Lyon \cite{lyon78}, Munro and Celis \cite{mc99} and
Poblete \cite{p77} describe some variants of Brent's method.


\subsection{Multiple-Choice Hashing}
\label{hash:sec:multiplechoice}
\index{hashing!multiple-choice}
\index{multiple-choice hashing}

It is worth stepping back at this point and revisiting the comparison
between hash tables and binary search trees.  For balanced binary
search trees, the average cost of searching for an element is $O(\log
n)$.  Indeed, it easy to see that for at least $n/2$ of the elements,
the cost of searching for those elements is $\Omega(\log n)$.  In
comparison, for both the random probing schemes discussed so far, the
expected cost of search for an element is $O(1)$.  However, there are
a handful of elements whose search cost is $\Theta(\log n/\log \log
n)$ or $\Theta(\log n)$ depending on whether hashing with chaining or
open addressing is used, respectively.  Thus there is an inversion:
Most operations on a binary search tree cost $\Theta(\log n)$ but a
few elements (close to the root) can be accessed in $O(1)$ time.  Most
operations on a hash table take $O(1)$ time but a few elements (in long
chains or with long probe sequences) require $\Theta(\log n/\log\log
n)$ or $\Theta(\log n)$ time to access.  In the next few sections we
consider variations on hashing with chaining and open addressing that
attempt to reduce the worst-case search time $W$.

\emph{Multiple-choice hashing} is hashing with chaining in which,
during insertion, the element $x$ has the choice of $d\ge 2$ different
lists in which it can be stored.  In particular, when we insert $x$ we
look at the lengths of the lists pointed to by
$A[x_0],\ldots,A[x_{d-1}]$ and append $x$ to $A[x_i]$, $0\le i< d$
such that the length of the list pointed to by $A[x_i]$ is minimum.
When searching for $x$, we search for $x$ in each of the lists
$A[x_0],\ldots,A[x_{d-1}]$ \emph{in parallel}.  That is, we look at
the first elements of each list, then the second elements of each
list, and so on until we find $x$.  As before, to delete $x$  we first
search for it and then delete it from whichever list we find it in.  

It is easy to see that the expected cost of searching for an element
$x$ is $O(d)$ since the expected length of each the $d$ lists is
$O(1)$.  More interestingly, the worst case search time is bounded by
$O(dW)$ where $W$ is the length of the longest list.  Azar \etal\
\cite{abku99} show that 
\begin{equation}
    \E[W]=\frac{\ln\ln n}{\ln d} + O(1) \enspace .
	\label{hash:eq:multiple-choice} 
\end{equation}
Thus, the expected worst case search time for multiple-choice hashing
is $O(\log\log n)$ for any constant $d\ge 2$.


\subsection{Asymmetric Hashing}
\index{hashing!asymmetric}
\index{asymmetric hashing}

\emph{Asymmetric hashing} is a variant of multiple-choice hashing in
which the hash table is split into $d$ blocks, each of size $n/d$.
(Assume, for simplicity, that $n$ is a multiple of $d$.) The probe
value $x_i$, $0\le i< d$ is drawn uniformly from
$\{in/d,\ldots,(i+1)n/d-1\}$.   As with multiple-choice hashing, to
insert $x$ the algorithm examines the lengths of the lists $A[x_0],
A[x_1],\ldots,A[x_{d-1}]$ and appends $x$ to the shortest of these
lists.  In the case of ties, it appends $x$ to the list with smallest
index.  Searching and deletion are done exactly as in multiple-choice
hashing.

V\"ocking \cite{v99} shows that, with asymmetric hashing the expected
length of the longest list is 
\[
   \E[W] \le \frac{\ln\ln n}{d\ln\phi_d} +O(1) \enspace .
\] 
The function $\phi_d$ is a generalization of the \emph{golden
ratio}\index{golden ratio},
so that $\phi_2=(1+\sqrt{5})/2$.  Note that this improves
significantly on standard multiple-choice hashing (\ref{hash:eq:multiple-choice}) for larger values of
$d$.

\subsection{LCFS Hashing}
\index{hashing!LCFS}
\index{LCFS hashing}

\emph{LCFS hashing} is a form of open addressing that changes the
collision resolution strategy.\footnote{Amble and Knuth \cite{ak74}
were the first to suggest that, with open addressing, any collision
resolution strategy could be used.} Reviewing the algorithm for
hashing with open addressing reveals that when two elements collide,
priority is given to the first element inserted into the hash table
and subsequent elements must move on.  Thus, hashing with open
addressing could also be referred to as \emph{FCFS (first-come
first-served) hashing}\index{FCFS hashing}\index{hashing!FCFS}.

With LCFS (last-come first-served) hashing, collision resolution is
done in exactly the opposite way.  When we insert an element $x$, we
always place it at location $x_0$.  If position $x_0$ is already
occupied by some element $y$ because $y_j=x_0$ then we place $y$ at
location $y_{j+1}$, possibly displacing some element $z$, and so on.

Poblete and Munro \cite{pm89} show that, after inserting $n$ elements
into an initially empty table, the expected worst case search time is
bounded above by
\[  \E[W] \le 
   1 
   + \Gamma^{-1}(\alpha n)
   \left( 1+\frac{\ln\ln(1/(1-\alpha))}{\ln \Gamma^{-1}(\alpha n)}
   + O\left(\frac{1}{\ln^2\Gamma^{-1}(\alpha n)} \right)\right)
\enspace ,
\]
where $\Gamma$ is the gamma function and 
\[
  \Gamma^{-1}(\alpha n) = \frac{\ln n}{\ln\ln n}
	\left(1+\frac{\ln\ln\ln n}{\ln\ln n} + 
	O\left(\frac{1}{\ln\ln n}\right)\right) \enspace .
\]
Historically, LCFS hashing is the first version of open addressing
that was shown to have an expected worst-case search time that is
$o(\log n)$.

\subsection{Robin-Hood Hashing}
\label{hash:sec:robinhood}
\index{hashing!Robin-Hood}
\index{Robin-Hood hashing}

Robin-Hood hashing \cite{c86,clm85,vp98} is a form of open addressing
that attempts to equalize the search times of elements by using a
fairer collision resolution strategy.  During insertion, if we are
trying to place element $x$ at position $x_i$ and there is already an
element $y$ stored at position $y_j=x_i$ then the ``younger'' of the
two elements must move on.  More precisely, if $i\le j$ then we will
try to insert $x$ at position $x_{i+1}$, $x_{i+2}$ and so on.
Otherwise, we will store $x$ at position $x_i$ and try to to insert
$y$ at positions $y_{j+1}$, $y_{j+2}$ and so on.

Devroye \etal\ \cite{dmv03} show that, after performing $n$ insertions on
an initially empty table of size $m=\alpha n$ using the Robin-Hood
insertion algorithm, the worst case search time has expected value
\[
	\E[W] = \Theta(\log\log n)
\]
and this bound is tight.
Thus, Robin-Hood hashing is a form of open addressing that
has doubly-logarithmic worst-case search time.  This makes it
competitive with the multiple-choice hashing method of
Section~\ref{hash:sec:multiplechoice}.

\subsection{Cuckoo Hashing}\label{hash:sec:cuckoo}
\index{hashing!cuckoo}
\index{cuckoo hashing}

Cuckoo hashing \cite{pr01} is a form of multiple choice hashing in
which each element $x$ lives in one of two tables $A$ or $B$, each of
size $m=n/\alpha$.  The element $x$ will either be stored at location
$A[x_A]$ or $B[x_B]$. There are no other options.  This makes
searching for $x$ an $O(1)$ time operation since we need only check
two array locations.

The insertion algorithm for cuckoo hashing proceeds as
follows:\footnote{The algorithm takes its name from the large but lazy
cuckoo bird which, rather than building its own nest, steals the nest
of another bird forcing the other bird to move on.} Store $x$ at
location $A[x_A]$. If $A[x_A]$ was previously occupied by some element
$y$ then store $y$ at location $B[y_B]$.  If $B[y_B]$ was previously
occupied by some element $z$ then store $z$ at location $A[z_A]$, and
so on.  This process ends when we place an element into a previously
empty table slot or when it has gone on for more than $c\log n$ steps.
In the former case, the insertion of $x$ completes successfully.  In
the latter case the insertion is considered a failure, and the entire
hash table is reconstructed from scratch using a new probe sequence
for each element in the table.  That is, if this reconstruction
process has happened $i$ times then the two hash values we use for an
element $x$ are $x_A=x_{2i}$ and $x_B=x_{2i+1}$.

Pagh and Rodler \cite{pr01} (see also Devroye and Morin \cite{dm02})
show that, during the insertion of $n$ elements, the probability of
requiring a reconstruction is $O(1/n)$. This, combined with the fact
that the expected insertion time is $O(1)$ shows that the expected
cost of $n$ insertions in a Cuckoo hashing table is $O(n)$.  Thus,
Cuckoo hashing offers a somewhat simpler alternative to the dynamic
perfect hashing algorithms of Section~\ref{hash:sec:dphash}.

\section{Historical Notes}

In this section we present some of the history of hash tables. The
idea of hashing seems to have been discovered simultaneously by two
groups of researchers. Knuth \cite{k97} cites an internal IBM
memorandum in January 1953 by H. P. Luhn that suggested the use of
hashing with chaining.  Building on Luhn's work, A. D. Linh suggested
a method of open addressing that assigns the probe sequence
$x_0,\floor{x_0/10},\floor{x_0/100},\floor{x_0/1000},\ldots$ to the
element $x$.

At approximately the same time, another group of researchers at IBM:
G. M. Amdahl, E.~M.~Boehme, N. Rochester and A. L. Samuel implemented
hashing in an assembly program for the IBM 701 computer.  Amdahl is
credited with the idea of open addressing with linear probing.

The first published work on hash tables was by A. I. Dumey \cite{d56},
who described hashing with chaining and discussed the idea of using
remainder modulo a prime as a hash function.  Ershov \cite{e58},
working in Russia and independently of Amdahl, described open
addressing with linear probing.

Peterson \cite{p57} wrote the first major article discussing the
problem of searching in large files and coined the term ``open
addressing.''  Buchholz \cite{b63} also gave a survey of the searching
problem with a very good discussion of hashing techniques at the time.
Theoretical analyses of linear probing were first presented by Konheim
and Weiss \cite{kw66} and Podderjugin.  Another, very influential,
survey of hashing was given by Morris \cite{m68}.  Morris' survey is
the first published use of the word ``hashing'' although it was
already in common use by practitioners at that time.



\section{Other Developments}

The study of hash tables has a long history and many researchers have
proposed methods of implementing hash tables.  Because of this, the
current chapter is necessarily incomplete. (At the time of writing,
the hash.bib bibliography on hashing contains over 800 entries.) We
have summarized only a handful of the major results on hash tables in
internal memory.  In this section we provide a few references to the
literature for some of the other results.   For more information on
hashing, Knuth \cite{k97}, Vitter and Flajolet \cite{vf90}, Vitter
and Chen \cite{vc87}, and Gonnet and Baeza-Yates \cite{gb-y91}
% and Flajolet \etal \cite{X} 
are useful references.

Brent's method (Section~\ref{hash:sec:brent}) is a collision
resolution strategy for open addressing that reduces the expected
search time for a successful search in a hash table with open
addressing.  Several other methods exist that either reduce the
expected or worst-case search time.  These include \emph{binary tree
hashing}\index{hashing!binary tree}\index{binary tree hashing}
\cite{m77, gm79}, \emph{optimal
hashing}\index{hashing!optimal}\index{optimal hashing}
\cite{p76,r78,gm79}, Robin-Hood hashing
(Section~\ref{hash:sec:robinhood}), and \emph{min-max
hashing}\index{hashing!min-max}\index{min-max hashing} \cite{g81,c86}.
One interesting method, due to Celis\index{Celis'
method}\index{hashing!Celis' method} \cite{c86}, applies to any open
addressing scheme.  The idea is to study the distribution of the ages
of elements in the hash table, i.e., the distribution give by 
\[
    D_i=\Pr\{\mbox{$x$ is stored at position $x_i$}\} 
\] 
and start searching for $x$ at the locations at which we are most
likely to find it, rather than searching the table positions
$x_0,x_1,x_2\ldots$ in order.

Perfect hash functions seem to have been first studied by Sprugnoli
\cite{s77} who gave some heuristic number theoretic constructions of
minimal perfect hash functions for small data sets.  Sprugnoli is
responsible for the terms ``perfect hash function'' and ``minimal
perfect hash function.''  A number of other researchers have presented
algorithms for discovering minimal and near-minimal perfect hash
functions.  Examples include Anderson and Anderson \cite{aa79},
Cichelli \cite{c80a,c80b}, Chang \cite{c84a,c84b}, Gori and Soda
\cite{gs89}, and Sager \cite{s85}.  Berman \etal\ \cite{bbdo86} and
K\"orner and Marton \cite{km88} discuss the theoretical limitations of
perfect hash functions.  A comprehensive, and recent, survey of
perfect hashing and minimal perfect hashing is given by Czech \etal\
\cite{chm97}.

Tarjan and Yao \cite{ty79} describe a set ADT implementation that
gives $O(\log u/\log n)$ worst-case access time.  It is obtained by
combining a trie (Section~\ref{external:32:tries}) of degree $n$ with a
compression scheme for arrays of size $n^2$ that contain only $n$
non-zero elements.  (The trie has $O(n)$ nodes each of which has $n$
pointers to children, but there are only a total of $O(n)$ children.)
Although their result is superseded by the results of Fredman \etal\
\cite{fks84} discussed in Section~\ref{hash:sec:fks}, they are the
first theoretical results on worst-case search time for hash tables.

Dynamic perfect hashing (Section~\ref{hash:sec:dphash}) and cuckoo
hashing (Section~\ref{hash:sec:cuckoo}) are methods of achieving
$O(1)$ worst case search time in a dynamic setting.  Several other
methods have been proposed  \cite{dm90,dgmp92,bm99}.
 
Yao \cite{y81} studies the \emph{membership problem}\index{membership
problem}.  Given a set $S\subseteq U$, devise a data structure that
can determine for any $x\in U$ whether $x$ is contained in $S$. Yao
shows how, under various conditions, this problem can be solved using
a very small number of memory accesses per query.  However, Yao's
algorithms sometimes derive the fact that an element $x$ is in $S$
without actually finding $x$.  Thus, they don't solve the set ADT
problem discussed at the beginning of this chapter since they can not
recover a pointer to $x$.

The ``power of two random choices,'' as used in multiple-choice
hashing, (Section~\ref{hash:sec:multiplechoice}) has many applications
in computer science.  Karp, Luby and Meyer auf der Heide
\cite{klm92,klm93} were the first to use this paradigm for simulating
PRAM computers on computers with fewer processors. The book chapter by
Mitzenmacher \etal\ \cite{mrs01} surveys results and applications of
this technique.

A number of table implementations have been proposed that are suitable
for managing hash tables in external memory.\index{hashing!external
memory}\index{external memory!hashing}  Here, the goal is to reduce
the number of disk blocks that must be accessed during an operation,
where a disk block can typically hold a large number of elements.
These schemes include \emph{linear
hashing}\index{hashing!linear}\index{linear hashing} \cite{l80},
\emph{dynamic hashing}\index{hashing!dynamic}\index{dynamic hashing}
\cite{l78}, \emph{virtual hashing}\index{virtual
hashing}\index{hashing!virtual} \cite{lit78}, \emph{extendible
hashing}\index{hashing!extendible}\index{extendible hashing}
\cite{fnps79},  \emph{cascade
hashing}\index{hashing!cascade}\index{cascade hashing} \cite{kz84},
and \emph{spiral storage}\index{spiral storage}\index{hashing!spiral
storage} \cite{m85}.  In terms of hashing, the main difference between
internal memory and external memory is that, in internal memory, an
array is allocated at a specific size and this can not be changed
later.  In contrast, an external memory file may be appended to or be
truncated to increase or decrease its size, respectively.  Thus, hash
table implementations for external memory can avoid the periodic
global rebuilding operations used in internal memory hash table
implementations.

\section*{Acknowledgement}

The author is supported by a grant from the Natural Sciences and
Engineering Research Council of Canada (NSERC).

%\input{hashing/hashing.bbl}
\bibliographystyle{plain}
\bibliography{hashing/hashing}

\bibliographystyle{plain}
\bibliography{hashing/hashing}

\bibliographystyle{plain}
\bibliography{hashing/hashing}

\bibliographystyle{plain}
\bibliography{hashing/hashing}
