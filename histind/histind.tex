\documentclass[lotsofwhite,charterfonts]{patmorin}

 
%\usepackage{amsthm}

\newcommand{\centeripe}[1]{\begin{center}\Ipe{#1}\end{center}}
\newcommand{\comment}[1]{}

\newcommand{\centerpsfig}[1]{\centerline{\psfig{#1}}}

\newcommand{\seclabel}[1]{\label{sec:#1}}
\newcommand{\Secref}[1]{Section~\ref{sec:#1}}
\newcommand{\secref}[1]{\mbox{Section~\ref{sec:#1}}}

\newcommand{\alglabel}[1]{\label{alg:#1}}
\newcommand{\Algref}[1]{Algorithm~\ref{alg:#1}}
\newcommand{\algref}[1]{\mbox{Algorithm~\ref{alg:#1}}}

\newcommand{\applabel}[1]{\label{app:#1}}
\newcommand{\Appref}[1]{Appendix~\ref{app:#1}}
\newcommand{\appref}[1]{\mbox{Appendix~\ref{app:#1}}}

\newcommand{\tablabel}[1]{\label{tab:#1}}
\newcommand{\Tabref}[1]{Table~\ref{tab:#1}}
\newcommand{\tabref}[1]{Table~\ref{tab:#1}}

\newcommand{\figlabel}[1]{\label{fig:#1}}
\newcommand{\Figref}[1]{Figure~\ref{fig:#1}}
\newcommand{\figref}[1]{\mbox{Figure~\ref{fig:#1}}}

\newcommand{\eqlabel}[1]{\label{eq:#1}}
\newcommand{\eqref}[1]{(\ref{eq:#1})}

\newtheorem{thm}{Theorem}{\bfseries}{\itshape}
\newcommand{\thmlabel}[1]{\label{thm:#1}}
\newcommand{\thmref}[1]{Theorem~\ref{thm:#1}}

\newtheorem{lem}{Lemma}{\bfseries}{\itshape}
\newcommand{\lemlabel}[1]{\label{lem:#1}}
\newcommand{\lemref}[1]{Lemma~\ref{lem:#1}}

\newtheorem{cor}{Corollary}{\bfseries}{\itshape}
\newcommand{\corlabel}[1]{\label{cor:#1}}
\newcommand{\corref}[1]{Corollary~\ref{cor:#1}}

\newtheorem{obs}{Observation}{\bfseries}{\itshape}
\newcommand{\obslabel}[1]{\label{obs:#1}}
\newcommand{\obsref}[1]{Observation~\ref{obs:#1}}

\newtheorem{assumption}{Assumption}{\bfseries}{\rm}
\newenvironment{ass}{\begin{assumption}\rm}{\end{assumption}}
\newcommand{\asslabel}[1]{\label{ass:#1}}
\newcommand{\assref}[1]{Assumption~\ref{ass:#1}}

\newcommand{\proclabel}[1]{\label{alg:#1}}
\newcommand{\procref}[1]{Procedure~\ref{alg:#1}}

\newtheorem{rem}{Remark}
\newtheorem{op}{Open Problem}

\newcommand{\etal}{\emph{et al}}

\newcommand{\voronoi}{Vorono\u\i}
\newcommand{\ceil}[1]{\left\lceil #1 \right\rceil}
\newcommand{\floor}[1]{\left\lfloor #1 \right\rfloor}



\newcommand{\E}{\mathrm{E}}

\title{\MakeUppercase{$B$-Treaps}}
\author{Stefan Langerman \and Pat Morin}
\date{}

\begin{document}
\maketitle

\begin{abstract}
We describe a randomized version of B-trees called B-treaps and show
that they execute update and search operations in $O(\log_B n)$ time
for all operations.
\end{abstract}

\section{Introduction}

\section{Randomized $B$-Trees}

A $B$-tree is a search tree in which each node contains at most $B-1$
elements and has at most $B$ children.   A node is \emph{full} if
contains exactly $B-1$ elements.  To insert the element $x$ into a
$B$-tree we follow the search path for $x$ until we reach a non-full
node and then add $x$ to that node.  Note this implies, that after any
sequence of insertions, the only non-full nodes are leaves.  

We consider the random process of inserting a random permutation of
$\{1,\ldots,n\}$ into an initially empty $B$-tree.  We call the
resulting tree a \emph{random $B$-tree}.  Let $T_B$ denote the
resulting tree and, for a node $u$ in $T_B$, let $S_B(u)$ denote the
set of elements stored in node $u$.  For $B=2$, the resulting tree
$T_2$ is a random binary search tree and this process is fairly well
understood.  In particular, it is known that the expected depth of any
particular node in $T_2$ is at most $2\ln n$ and at least $\ln n$
\cite{X} and the expected height (depth of the deepest node) of $T_2$
is at most $c\ln n$, where $c=4.31107\ldots$ \cite{X}.  

We first observe that the two trees $T_2$ and $T_B$, $B>2$ are closely
related (see \figref{bin-to-b}).  For any node $u$ in $T_B$, $S_B(u)$
induces a connected subtree of $T_2$.  Note, furthermore, that for
every permutation that produces the tree $T_B$, the elements of
$S_B(u)$ can permuted arbitrarily and the same tree $T_B$ results.
This implies that the subtree of $T_2$ induced by $S_B(u)$ is a random
binary search tree and has the properties described above.  In
particular, the shape of the subtree is independent of the tree $T_B$.

Let $u_1,\ldots,u_k$ be the sequence of nodes visited while searching
for some element $x$ in $T_B$ and observe that $u_1,\ldots,u_{k-1}$
are full.  Consider one such node $u_i$, $1<i<k$ and let $V(u_i)$
denote the number of elements in $u$ that are also visited by the
search path for $x$ in $T_2$.  From the above remarks, it follows that
\[ 
     \ln B \le \E[V(u_i)] \le 2\ln B \enspace ,  
\] 
since $V(u)$ is the number of nodes visited during a search of a
random binary search tree of size $B$.  Furthermore, all the $V(u_i)$
are independent and they are independent of $k$.  Therefore, by Wald's
Equation
\begin{eqnarray*}
    \E\left[k-1\right]\times \E\left[V(u_i)\right] 
	&  =  & \E\left[\sum_{i=1}^{k-1} V(u_i)\right] \\
        & \le & \E\left[\sum_{i=1}^{k} V(u_i)\right] \\
        & \le & 2\ln n 
	\enspace . 
\end{eqnarray*}
Stated another way,
\begin{equation}
	\E[k] \le 2\ln n/\E[|V(u_i)|]+1 \le  2\log_B n + 1  \enspace .
	\eqlabel{search-time}
\end{equation}
Thus, the search time in a random $B$-tree is within a constant factor
of the optimal search time.

\begin{rem}
The constant 2 in \eqref{search-time} is probably not tight. To
improve this constant, we need a lower bound on $E[V(u)]$.  The bound
$\ln i+\ln (B-i)$ for searching the element of rank $i$ might be more
useful, but it is difficult to get a handle on the values $i$.
\end{rem}

\section{$B$-Treaps}

Here I want to dynamize random $B$-trees using priorities.  However,
this is trickier than with treaps because a single rotation in $T_2$
causes a whole cascade of changes in $T_B$.  The solution will be to
follow the search paths for the inserted value until we find a node
whose minimum priority is less than the priority of the element we are
inserting then rebuild (at most) 4 subtrees rooted at that node.  The
following lemma shows that there's not very much to rebuild.

\begin{lem}
Let $s(i)$ denote the size of the subtree rooted at $i$ in a random
binary search tree.  Then  $E[s(i)] = O(\log n)$.
\end{lem}

\begin{proof}
We first show that the expected number of elements in the right
subtree of $i$ is $O(\log (n-i))$.  By a symmetric argument, the
number of elements in the left subree is $O(\log i)$, which is
sufficient to prove the lemma.

The elements in the right subtree of $i$ are all greater than $i$, so
consider only the permutation of the $n-i+1$ elements greater than or
equal to $i$.  In this permutation, there is some minimum value $j$
that appears before $i$ (if $i$ is the first element, then we use the
convention that $j=n+1$).  The elements in the right subtree of $i$
are precisely those elements that appear after $i$ in the permutation
and are less than $j$.  Therefore, what we are really studying is the
expected value of $j-i$.  Of course, this is highly dependent on the
position of $i$ in the permutation: \begin{eqnarray*} E[j-i] & = &
\sum_{k=1}^{n-i+1} \Pr\{\mbox{$i$ is at location $k$}\} \times
E[j-i\mid\mbox{$i$ is at location $k$}] \\ & = &
\frac{1}{n-i+1}\sum_{k=1}^{n-i+1} E[j-i\mid\mbox{$i$ is at location
$k$}] \end{eqnarray*} Now note that $E[j-i\mid\mbox{$i$ is at location
$k$}]$ is the minimum of $k-1$ numbers chosen at random without
replacement from the set $\{0,\ldots,n-i\}$.  An easy computation
shows that this is $O((n-i)/k)$ (What is the exact value?), which
allows us to complete the preceding computation: \begin{eqnarray*}
E[j-i] & = & \frac{1}{n-i+1}\sum_{k=1}^{n-i+1} O(n-i/k) \\ & = &
O(H_{n-i}) \\ & = & O(\log (n-i)) \enspace , \end{eqnarray*} and this
completes the proof.  \end{proof}


Since the four subtrees that we rebuild in $T_B$ contain only elements
that are descendents of the newly inserted node in $T_2$, the above
lemma says that there are not very many nodes involved in the
rebuilding.  What we need now is a lemma that says that a random
$B$-tree on $m$ values has $O(m/B)$ nodes.  I currently have no idea
how to prove this.



\end{document}
