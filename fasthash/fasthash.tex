\documentclass[lotsofwhite,charterfonts]{patmorin}
\usepackage{amsthm}

\title{\MakeUppercase{Efficient Universal Hashing}}
\author{Prosenjit Bose \and Pat Morin}
\date{}

\newcommand{\Z}{\mathbb{Z}}
\newcommand{\F}{\mathcal{F}}
\newcommand{\xor}{\oplus}
\newcommand{\rev}[1]{\overleftarrow{#1}}
\input{pat}

\begin{document}
\maketitle
\begin{abstract}
We introduce a new family of hash functions that maps $b$-bit
non-negative integers onto $r$-bit non-negative integers ($r\le b$).
This family is defined as the set
\[
 \mathcal{F}_\mathrm{BM} = \left\{ h_{k,l}(x)=((x\times k) \xor (\rev{x}\times
l))\bmod 2^r :
k,l\in\{0,\ldots,2^b-1\} \right\} 
\]
where $\rev{x}$ denotes the integer obtained by reversing the bits in
the binary representation of $x$ and $\xor$ denotes the bitwise
exclusive-or operation.  We show that, besides being very fast to
implement on a binary computer, this family of hash functions has a
very strong \emph{avalanche} property.  Namely, for any distinct
$x,y\in\{0,\ldots,2^b\}$, if we choose $k$ and $l$ randomly then
$h_{k,l}(x)\xor h_{k,l}(y)$ is a random bit-string of length $r$.  We
show that this family of hash functions has applications in data
structures and cryptography.  
\end{abstract}


\section{Introduction}

A family $\mathcal{H}$ of functions from $U=\{0,\ldots,u-1\}$ onto
$M=\{0,\ldots,m-1\}$ is said to be a \emph{universal} family of hash
functions if, for every distinct $x,y\in U$ choosing a random element
$h\in \mathcal{H}$ gives
\[
    \Pr\{h(x)=h(y)\} \le 1/m \enspace .
\]
As a data structuring tool, universal families of hash functions are
extremely useful since, given a set $S\subseteq U$ of size $O(m)$ a
randomly chosen member of $\mathcal{H}$ can be used to map the
elements of $S$ into a table of size $m$ such that the expected number
of elements mapped into any particular table position is $O(1)$.  This
gives a data structure for querying membership in $S$ that takes
$O(1)$ expected time per query.  With slightly more work, families of
universal hash functions can be used to build a table of size $O(m)$
in which no element is mapped to more than one table entry
\cite{fksXX,clrsXX}.  Families of universal hash functions also have
applications to message authentication in the field of cryptography
\cite{cXX,cXX,cXX}.

Universal families of hash functions were introduced by Carter and
Wegman \cite{cwXX} who suggested the family
\[
\mathcal{F}^{(u,m)} = \{f_{k,l}(x)=(kx+l)\bmod u\bmod m  :
k\in\{1,\ldots,m-1\}, l\in\{0,\ldots,m-1\} \}
\]
and showed that if $u$ is a prime then this family of hash functions
is universal.  Unfortunately, the requirement that $u$ is a prime
makes this function awkward to use in implementations.  For example,
if we wish to build a hash table to store 32 bit integers then the
smallest prime value not less than $2^{32}$ is $u=2^{32}+15$.
Therefore, to implement this hash function, the multiplication
$k\times x$ requires the use of 64-bit arithmetic as does the modular
arithmetic operation that computes the result modulo u.  The 64-bit
arithmetic is costly, but even more costly is the reduction modulo u.
On current architectures the $\bmod$ operation typically takes an order of
magnitude longer to execute than an integer multiplication or addition
operation. 

Being aware of this problem, Halevi and Krawczyk \cite{X} showed how
the reduction modulo $u=2^{32}+15$ can be implemented without division
using a constant number of multiplication, addition, subtraction,
negation, shifting, and branching instructions, all done using 64-bit
arithmetic.

Black \etal\ \cite{X} introduce a new class of hash functions for use
in message authentication that operates on $b$-bit binary numbers,
where $b=2w$.  This family is defined as
\[
	\mathcal{G}^{(2^b,2^b)} = \{ g_{k,l}(x) = ((k+x_\mathrm{hi})\bmod
2^w)(l+x_\mathrm{lo})\bmod 2^w) : k,l\in\{0,\ldots,2^w-1\} \}	\enspace ,
\]
where $x_\mathrm{hi}$ and $x_\mathrm{lo}$ denote the high order and
low order $w$ bits of $x$, respectively.  This family of hash
functions is extremely well-suited to implementation on modern
computers since the modular reductions are done using only powers of 2 and can
therefore be implemented as a bitwise binary AND operation.
Furthermore, all arithmetic takes place using $b$-bit arithmetic.
That is, if the input $x$ is a 32-bit number then all arithmetic
operations can be implemented using 32-bit arithmetic.  Unfortunately,
the family $\mathcal{F}_\mathrm{B+}$ is not universal.  In fact the
strongest statement one can make is that, for distinct $x,y\in
\{0,\ldots2^{2w}-w\}$ and randomly chosen $k,l\in\{0,\ldots,2^{w}-1\}$,
\[ 
    \Pr\{h_{k,l}(x)=h_{k,l}(y)\} \le 1/2^w \enspace .  
\] 
Compared with universal families of hash functions, this is quite weak
since the total number of possible hash values is $m=2^b=2^{2w}$, but
two keys have the same hash value with probability $1/2^w=1/\sqrt{m}$.
Other families of hash functions, designed with the goal of optimizing
software implementations have been proposed \cite{X,X,X,X,X} but none
of these is universal. (Though some give better guarantees on the
collision probability.)

In this paper we propose a new family of hash functions that maps $b$ bit
integers onto $r$ bit integers.  This family is defined as
\[
	\mathcal{H}^{(2^b,2^r)} = 
	\{
	  (kx\oplus l\rev{x}) \bmod 2^r : k,l\in \Z_{2^b}\}
	\} \enspace 	
\]
where $\rev{x}$ denotes the integer obtained by reversing the bits of
$x$ and $\xor$ denotes the bitwise exclusive-or operation.
This family of hash functions is well-suited for implementation on
modern computers because its main ingredients are multiplication and
modular reduction over a power of 2 and bitwise exclusive-or.  The
only troublesome operation is the $\rev{x}$ operation which is not
implemented as a primitive in most programming languages.  However,
we show that it can be implemented efficiently using lookup tables.
Furthermore, unlike modular reduction, there is no fundamental
difficulty involved in implementing the $\rev{x}$ operation in
hardware.  Indeed, it can be implemented by a circuit that requires 0
gates.

On the theoretical side, we show that this family has a very strong
\emph{avalanche} property:\footnote{We call this an avalanche property
because a small (1 bit) change in the input results in a large
($b/2$-bit) expected change in the output \cite{X}.}
\begin{thm}\thmlabel{random}
Consider two distinct values  $x,y\in \{0,\ldots,2^b-1\}$ and select a
random member $h\in \mathcal{H}^{(2^b,2^b)}$.  Then
$X = h(x)\xor h(y)$ is a random binary string
of length $b$.  That is, each bit of $X$ is set to 0 or 1 with
probability $1/2$ independently of all other bits of $X$.
\end{thm}

Since $h(x)=h(y)$ if and only if $h(x)\xor h(y)= 0$, \thmref{random} leads
immediately to the following corollary, which makes this family of
hash functions excellent for use in the implementation of hash tables.
\begin{cor}
The family $H^{(2^b,2^r)}$ is universal for all $1\le r\le b$.
\end{cor}

\thmref{random} also leads immediately to the following corrolary
which makes this family of hash functions useful for message
authentication applications:

\begin{cor}
Let $k_i$ and $l_i$, $1\le i\le p$ be random $b$-bit strings and let
$x_i$ and $y_i$, $1\le i\le p$ be such that there exists at least one
$i$ in which $x_i\neq y_i$.  Then
\[
   \Pr\left\{\bigoplus_{i=1}^p h^{(b,b)}_{k_i,l_i}(x_i) = 
	\bigoplus_{i=1}^p h^{(b,b)}_{k_i,l_i}(y_i) \right\} \le 1/2^b
\enspace .
\]
\end{cor}

The remainder of this paper is organized as follows.   In
\secref{proof} we prove \thmref{random}.  In \secref{implementation}
we describe an implementation of this algorithm and some experimental
results comparing it to other algorithms.  In \secref{conclusions} we
summarize and conclude with open problems.

\section{Proof of \thmref{random}}\seclabel{proof}

\begin{proof}[Proof (of \thmref{random})]

Let $x_i$ denote the $i$th bit of $x$.  Since $x\neq y$, there exists
some $i$ such that $x_i$ and $y_i$ differ.  Consider the bit $X_j$.
There are two cases to consider:

\noindent\textbf{Case 1:} $j \ge i$.  In this case, the value of $X_j$
can be expressed as an exclusive or of many bits, one of which is
$x_i\and k_{j-i}$ and another of which is $y_j\and k_{j-i}$. 

 that includes
\begin{enumerate}
\item

\end{enumerate}


\end{proof}

\section{Implementation and Experiments}

\section{Conclusions}

\end{document}
