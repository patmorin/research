\documentclass[lotsofwhite]{patmorin}

\title{\MakeUppercase{Centerpoint Theorems for Wedges}}
\author{Jeff Erickson \and
	Ferran Hurtado \and
	Pat Morin \and
	Perouz Taslakian \and
	David Wood}
\date{}

\begin{document}
\maketitle

\begin{abstract}
The \emph{Centerpoint Theorem} states that, for any set $S$ of $n$ points
in $\mathbb{R}^2$, there exists a point $p$ in $\mathbb{R}^2$ such
that every halfplane containing $p$ contains at least $n/3$ points of
$S$ and such a point can be found in $O(n)$ time using an algorithm of
Mukhopadhya and XXX (1996).  We consider a generalization of the
Centerpoint Theorem in which halfplanes are replaced with cones of
angle $\alpha$.
\end{abstract}

\section{Introduction}

Let $S$ be a set of $n$ points in $\mathbb{R}^d$.  The \emph{halfspace
depth} \cite{t73} of a point $p$ with respect to $S$ is defined as
\begin{equation}
H(p) = \min\{|h\cap S| : \mbox{$h$ is a closed halfspace that contains $p$} \}
\enspace .
\end{equation}
The \emph{Centerpoint Theorem}, which is a simple consequence of
Helly's Theorem \cite{X} states that for any point set $S$, there
exists a point whose halfspace depth is at least $n/(d+1)$.  

In this paper we consider a generalization of halfspace depth called
\emph{$\alpha$ wedge depth}.  The $\alpha$ wedge depth
\end{document}
