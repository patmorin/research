\documentclass[lotsofwhite,charterfonts]{patmorin}
\usepackage{amsthm,amsfonts}
\input{pat}

\newcommand{\rep}[1]{^{\langle#1\rangle}}

\newcommand{\toomuchdetail}[1]{#1}
\newcommand{\degreesum}{Condition~(a)}
\newcommand{\maxdegree}{Condition~(b)}
\newcommand{\twotwos}{Condition~(c)}
\newcommand{\onebig}{Condition~(d)}
\newcommand{\fourds}{Condition~(e)}


\title{\MakeUppercase{Degree Sequences of 2-Trees}%
	\thanks{This research was partly funded by NSERC.}}

\author{
	Greg Aloupis \and
	Prosenjit Bose \and
	David Bremner \and
	Mirela Damian-Lordache \and
	Erik Demaine \and 
	Martin Demaine \and
	Dania El-Khechen \and
	Vida Dujmovi\'{c} \and
	Robin Flatland \and
	Francisco Gomez-Martin \and
	John Iacono \and
	Danny Kri\c{z}anc \and
	Stefan Langerman \and
	Erin Leigh McLeish \and
	Henk Meijer \and
	Pat Morin \and
	Sumeeta Ramaswami \and
	David Rappaport \and
	Diane Souvaine \and
	Perouz Taslakian \and
	Godfried Toussaint \and
	Dmitri Tymoczko \and
	David Wood \and
	Stefanie Wuhrer \and
}

\date{}

\begin{document}
\maketitle
\begin{abstract}
A characterization of the degree sequences of 2-trees is
given.  This characterization yields a linear-time algorithm for
recognizing and realizing degree sequences of 2-trees.
\end{abstract}

\section{Introduction}

The \emph{degree sequence} of a graph $G=(V,E)$ is the sequence
obtained by listing the degrees of the vertices of $G$ in
non-decreasing order.  If $D=\langle d_1,\ldots,d_n\rangle$ is the
degree sequence of $G$ then we say that $G$ is a \emph{realization} of
$D$ or that $G$ \emph{realizes} $D$.  

In this paper we consider degree sequences of a particular class of
graphs.  \emph{2-trees} are defined inductively as follows:  The
complete graph $K_3$ on 3 vertices is a 2-tree and any 2-tree $T$ on
$n$ vertices can be obtained from some 2-tree $T'$ on $n-1$ vertices
by adding a new vertex $u$ adjacent to two vertices $v$ and $w$ that
are adjacent in $T'$.  We call this process \emph{attaching} the
vertex $u$ to the edge $\{v,w\}$.

It is helpful to have a notation for elements of a sequence that occur
often.  We denote by $a\rep{b}$ the (sub)sequence $a,a,\ldots,a$ of
length $b$. We prove the following result:

\begin{thm}\thmlabel{main}
The following two statements are equivalent:
\begin{enumerate}
\item $D=\langle d_1,\ldots,d_n\rangle$ is a non-decreasing sequence of
      natural numbers such that
\begin{enumerate}
\item $\sum_{i=1}^n d_i = 4n-6$,
\item $\max\{d_i : 1\le i\le n\} \le n-1$,
\item $d_1=d_2=2$,
\item $D\not\in \{ 2\rep{p},4\rep{n-p-1}, 2p-2 : \mbox{$n < 2p-1$ or
$n > 3p-3$} \}$, and 
\item $D\not\in \{\langle 2\rep{n-4},d,d,d,d\rangle : d\ge 5\}$.
\end{enumerate}
\item $D$ is the degree sequence of a 2-tree.  
\end{enumerate}
Furthermore, for each $D=\langle d_1,\ldots,d_n\rangle$ satisfying
Conditions~(a)--(e) and a given $d_i \ge 3$ there exists a 2-tree that
realizes $D$ in which a vertex of degree $d_i$ is adjacent to a vertex
of degree $2$.  
\end{thm}

We denote by $\mathcal{D}$ the set of all sequences satisfying
Conditions~(a)--(e) of \thmref{main}.  The remainder of this paper is
organized as follows:  In \secref{if} we show that the every sequence
$D\in \mathcal{D}$ is the degree sequence of a 2-tree.  In
\secref{onlyif} we show that the degree sequence of every 2-tree is
in $\mathcal{D}$.

\section{The Elements of $\mathcal{D}$ are the Degree Sequences of
	2-Trees}\seclabel{if}

In this section we prove that if $D\in \mathcal{D}$ then $D$ is the
degree sequence of a 2-tree.  Our proof is by induction on $n$, the
length of the sequence.  However, during the process of applying
induction we come across several different base cases.

\subsection{The Base Cases}

\begin{lem}\lemlabel{basecase-a}
The sequence $\langle 2,2,2\rangle$ is the degree sequences of a 2-tree.
\end{lem}

\begin{proof}
This follows immmediately from the definition of 2-trees.
\end{proof}


\begin{lem}\lemlabel{basecase-b}
Suppose $D\in \mathcal{D}\cap \{\langle 2\rep{n-2},x,y\rangle:
x,y\ge 3\}$. 
Then there exists a 2-tree that realizes $D$ in which every vertex of
degree greater than $2$ is adjacent to a vertex of degree 2.
\end{lem}

\begin{proof}
From \degreesum\ we know that $2(n-2)+x+y=4n-6$ or, equivalently,
$x+y=2n-2$.  By \maxdegree\ this implies that $x=y=n-1$.  Thus,
we can create a 2-tree realizing $D$ by starting with a $K_3$ and
attaching $n-3$ vertices to one of its edges.
\end{proof}

\begin{lem}\lemlabel{basecase-c}
Suppose $D\in \mathcal{D}\cap \{\langle 2\rep{n-3},x,y,z\rangle:
x,y,z\ge 3\}$. 
Then there exists a 2-tree that realizes $D$ in which every vertex of
degree greater than $2$ is adjacent to a vertex of degree 2.
\end{lem}

\begin{proof}
We create a 2-tree by starting with the graph $K_3$ (which, according
to \lemref{basecase-a}, is a
2-tree) and attaching the following numbers of vertices to its edges:
\[
    e_1=\frac{1}{2}(x+y-z-2) \enspace , \enspace
    e_2=\frac{1}{2}(x-y+z-2) \enspace , \enspace 
    e_3=\frac{1}{2}(-x+y+z-2) \enspace .
\]
It is straightforward to verify that the resulting 2-tree has three
vertices of degree $x$, $y$, and $z$, respectively, and that all other
vertices have degree 2.  All that remains to verify that is that
$e_1$, $e_2$ and $e_3$ are non-negative integers.  These number are
certainly integers because, by \degreesum, $x+y+z$ is even.  

Next we show that $e_1$ is non-negative.  Observe that, by \degreesum,
$2(n-3)+x+y+z = 4n-6$ or, equivalently, $x+y+z=2n$.  By \maxdegree,
this implies that $x+y \ge n+1$ and that $x+y-z \ge 2$.  Thus,
$x+y-z-2 \ge 0$ and $e_1$ is non-negative, as required.  
Exactly the same argument shows that $e_2$ and $e_3$ are also
non-negative.
\end{proof}

\begin{lem}\lemlabel{basecase-d}
Suppose $D\in \mathcal{D}\cap \{\langle 2\rep{n-5},d,d,d,x,d+x-2 \rangle:
\mbox{$d,x\ge 3$}\}$.
Then there exists a 2-tree that realizes $D$ in which every vertex of
degree greater than $2$ is adjacent to a vertex of degree 2.
\end{lem}

\begin{proof}
To do.
\end{proof}


\begin{lem}\lemlabel{basecase-e}
Suppose $D\in \mathcal{D}\cap \{\langle 2\rep{p},d\rep{n-p}\rangle:
\mbox{$d\ge 3$ and $n-p \ge 5$}\}$.
Then there exists a 2-tree that realizes $D$ in which every vertex of
degree greater than $2$ is adjacent to a vertex of degree 2.
\end{lem}

\begin{proof}
To do (possibly use David's reduction instead).
\end{proof}


\begin{lem}\lemlabel{basecase-f}
Suppose $D\in \mathcal{D}\cap \{\langle 2\rep{p},4\rep{n-p-1},2p-2\rangle:
2p-1\le n\le 3p-3\}$.
Then there exists a 2-tree that realizes $D$ in which every vertex of
degree greater than $2$ is adjacent to a vertex of degree 2.
\end{lem}\marginpar{necessary?}

\begin{proof}
To do.
\end{proof}

\begin{lem}\lemlabel{basecase-g}
Suppose $D\in \mathcal{D}\cap \{\langle 2\rep{p},4\rep{n-p-2},x,n-2\rangle:
\mbox{$d\ge 3$}\}$.
Then there exists a 2-tree that realizes $D$ in which every vertex of
degree greater than $2$ is adjacent to a vertex of degree 2.
\end{lem}

\begin{proof}
To do.
\end{proof}

\subsection{The Induction}

With the base cases out of the way, we are ready for an inductive
proof of the first half of \thmref{main}.

\begin{lem}\lemlabel{if}
If $D=\langle d_1,\ldots,d_n\rangle \in \mathcal{D}$ and $d_i\ge 3$
then there exists a 2-tree that realizes $D$ in which a vertex of
degree $d_i$ is adjacent to a vertex of degree 2.  
\end{lem}

\begin{proof}

We are given $D$ and a particular value $d_i$. If $D$ meets the
conditions of Lemmata~\ref{lem:basecase-a}, \ref{lem:basecase-b},
\ref{lem:basecase-c}, \ref{lem:basecase-d}, \ref{lem:basecase-e},
\ref{lem:basecase-f}, or \ref{lem:basecase-g} then we are done.
Otherwise, select a value $d_j$, $d_j\neq d_i$, $d_j\ge 3$.  We know
such a value exists because neither \lemref{basecase-a},
\lemref{basecase-b}, \lemref{basecase-c}, nor \lemref{basecase-e} is
applicable to $D$.  The value $d_j$ is selected as follows:  Let
$a=\min\{d_k : d_k\ge 3\}$ and let $c=d_n$.  If
$d_i=a$ or $a+c-2 \ge n$ then we select $d_j=d_n$ otherwise we select
$d_j=a$.  Without loss of generality, assume $d_i < d_j$ (otherwise
reverse the roles of $d_i$ and $d_j$) and let $x=d_i$.  Create a new
sequence 
\[  
   D'=\langle d_{x-2+1},\ldots,d_{i}-x+2,\ldots,d_{j}-x+2,\ldots,d_n \rangle
\] 
and, if necessary, reorder the elements of $D'$ into non-decreasing
order.  The proof of the following claim is left until later:

\begin{clm}\clmlabel{main}
$D'\in \mathcal{D}$.
\end{clm}

Now, apply the inductive hypothesis on the sequence $D'$ with the
special value $d_j-x+2$ to obtain a 2-tree $T'$ in which a vertex $v$
of degree $d_j-x+2$ is adjacent to a vertex $w$ of degree 2.  Attach
$x-2$ vertices to the edge $\{v,w\}$ to obtain a 2-tree that realizes
$D$ and in which vertices of degree $d_i$ and $d_j$ are each adjacent
to $x-2\ge 1$ vertices of degree $2$.  This completes the proof.
\end{proof}


\begin{proof}[Proof of \clmref{main}]

We must show that $D'$ satisfies Conditions~(a)-(e) of \thmref{main}.
Let $p$ be the number of times 2 appears in $D$.  We begin by
considering \fourds\ and \onebig\ since these can be shown independent
of the choice of $d_j$.

\noindent\fourds:
If $D'\in \{\langle 2\rep{n'-4},d,d,d,d\rangle : d\ge 5\}$ then
$D\in\{\langle 2\rep{n'+x-6},d,d,d,x,d+x-2\rangle : d,x\ge 3\}$ which
is not possible since then \lemref{basecase-f} would apply to $D$.

\noindent\onebig:
Suppose $D'$ is of the form $\langle 2\rep{p'},4\rep{n'-p'-1},
2p'-2\rangle$.   Now, the sequence $D$ must be 
\begin{equation} 
    D=\langle 2\rep{p},4\rep{n-p-2},x,2p-x+2\rangle \enspace , 
		\eqlabel{seq-a}
\end{equation}
possibly with $x=4$.

If $n'< 2p'-1$ then
\[
      n' = n-x+2 < 2p'-1  
         \toomuchdetail{= 2(p-x+3) - 1}
         = 2p-2x + 5 \enspace , 
\]
so $n < 2p-x+3$ but this is no possible since then $D$ does not 
satisfy
\maxdegree.

On the other hand, if $n'>3p' - 3$ then
\[
      n' = n-x+2 > 3p' - 3 
         \toomuchdetail{= 3(p-x+3) - 3}
         = 3p - 3x + 6
\]
so $n > 3p - 2x + 4$.  So what?


Next we show that $D'$ satisfies Conditions~(a)-(c).  There are two
cases to consider, depending on the choice of $d_j$.

\noindent\textbf{Case 1:} $d_j=d_n$ and $a+c-2\ge n$.  

Dispense with the case $a=3$.  In this case $a+c-2=c+1\ge n$ implies $c\ge
n-1$ (is an easy case?). \marginpar{*}

\noindent\degreesum\ and \twotwos:  To show that these conditions hold
for $D'$, it suffices to show that $p'=p-x+3\ge 2$, i.e., that $p\ge
x-1$.  Since the number 3 does not occur in $D$ and $D$ satisified
Condition~1, we have 
\[
             2p+4(n-p-2)+x+c \le 4n-6 \enspace ,
\]
which simplifies to $p\ge (x+c)/2-1\ge x-1$, as required.

\noindent\maxdegree:  Let $b=\max\{d:d\in D'\}$.  The sequence $D'$
has length $n'=n-x+2$ and we must show that $b\le n'-1$.  If
$b=d_n-x+2$ then the assumption that $D$ satisifies \maxdegree\
implies $b\le n'-1$.  Thus, assume $b\neq d_{n}-x+2$ so that the values
$x$, $b$ and $c$ each occur in $D$, i.e., $d_i=x$, $d_j=c$ and $d_k=b$
with $i$, $j$ and $k$ distinct.

Let $t=n-p-3$.  Then, from \degreesum\ on $D$ we have
\[
     2p + ta + x + b + c  \le 4n-6 \enspace , 
\]
which, since $n=p+t+3$, is equivalent to 
\[
     t(a-2) + x + b + c  \le 2n \enspace .
\]
Adding $1-x-n$ to both sides gives
\[
     t(a-2) + b + c - n + 1\le n - x + 1 = n' - 1\enspace .
\]
Since $a+c-n \ge 2$, $t\ge 0$ and $a\ge 4$ we have
\begin{eqnarray*}
     n'-1 & \ge & t(a-2) + b + c - n + 1 \\
          & \ge & t(a-3) + b + 3 \\
          & \ge & b+3 \ge b
\end{eqnarray*}
as required.

\noindent\textbf{Case 2:} $d_i=x=a$ and $a+c-2 \le n-1$.

\noindent\degreesum\ and \twotwos:  To show that these conditions hold for $D'$
it is sufficient to show that $p'=p-a+3 \ge 2$, i.e., that $p\ge a-1$.
Notice that, if $a=3$ then \twotwos\ on $D$ already implies $p\ge
a-1$.  Thus, assume $a\ge 4$.  The, from \degreesum\ we have
\[
      2p + a + 4(n-p-2) + d_j \le 4n-6 \enspace ,
\]
which simplifies to $p\ge (a+d_j)/2 -1 \ge a-1$, as required.

\noindent\maxdegree: To show that \maxdegree\ holds for $D'$, observe that
$\max\{d:d\in D'\}\le c$ and the conditions of Case~2 ensure that
$c \le n-a+1= n'-1$.

\noindent\onebig:

\noindent\fourds:
If $D'\in \{\langle 2\rep{n'-4},d,d,d,d\rangle : d\ge 5\}$ then
$D\in\{\langle 2\rep{n'+x-6},d,d,d,x,d+x-2\rangle : d,x\ge 3\}$ which
is not possible since then \lemref{basecase-f} would apply to $D$.
\end{proof}


\section{Degree Sequences of 2-Trees are in $\mathcal{D}$}\seclabel{onlyif}

\begin{lem}\lemlabel{onlyif-a}
Any sequence $D$ that does not satisfy Conditions~(a)-(c) of
\thmref{main} is not the degree sequence of any 2-tree.
\end{lem}

\begin{proof}
It is easy to verify that every 2-tree has $2n-3$ edges and hence
its degree sequence satisfies Condition~(a).
Every simple graph (and hence every 2-tree) satisfied Condition~(b).
Finally, Condition~(c) follows from the 
well-known fact that every two tree has at least 2 vertices of degree
2 \cite{X}.
\end{proof}

\begin{lem}\lemlabel{onlyif-b}
Any sequence $D\in\{\langle 2\rep{n-4},d,d,d,d\rangle : d\ge 5\}$ is not 
the degree sequence of any 2-tree.
\end{lem}

\begin{proof}
To do (uses a bit of algebra).
\end{proof}


\begin{lem}\lemlabel{onlyif-c}
Any sequence $D\in\{2\rep{p},4\rep{n-p-1}, 2p-2 : \mbox{$n < 2p-1$ or
$n > 3p-3$} \}$ is not the degree sequence of any 2-tree.
\end{lem}

\begin{proof}
To do (uses a charging argument based on the fact that the subgraph
induced by degree 4 vertices is a binary tree - this is closely
related to the fact that every degree 4 is adjacent to the vertex of
degree $2p-2$).
\end{proof}

\begin{lem}\lemlabel{onlyif}
Any sequence $D\not\in \mathcal{D}$ is the not the degree sequence of
any 2-tree.
\end{lem}

\begin{proof}
This follows immediately from \lemref{onlyif-a}, \lemref{onlyif-b} and
\lemref{onlyif-c}.
\end{proof}

\section{Conclusions}

Together, \lemref{if} and \lemref{onlyif} prove both directions of
\thmref{main}.

\begin{itemize}
\item Sketch $O(n)$ time realization algorithm.
\item Mention open problems.
\end{itemize}

\section{Acknowledgements}

This research was initiated at the The 21st Bellairs Winter Workshop
on Computational Geometry, January 27--February 3, 2006.  The authors
are grateful to Godfried~Toussaint for organizing the workshop and to
the other workshop participants, namely
---,
---,
---, and
---,
for providing a stimulating working environment.


\bibliographystyle{plain}
\bibliography{paper}
\end{document}

