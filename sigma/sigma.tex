\documentclass{article}

\begin{document}
Let $\sigma$ be a value such that $G_\sigma(S)=(S,E)$ has $n^{4/3}$ edges.
Let $S_<$ be the vertices of $G_\sigma$ that have degree less than
$n^{4/3}/2$ and let $S_\ge=S\setminus S_<$.  Then we have
\[
     \sum_{v\in S_<}\deg(v) \le n^{4/3}/2
\]
so
\[
     \sum_{v\in S_\ge}\deg(v) \ge n^{4/3}/2 \enspace .
\]
This means that if we want to know approximately the number of edges
in $G_\sigma$ it suffices to sum up the degrees of the vertices in
$S_\ge$.  Note that, since the degree of any vertex is at most $n$,
\[
        |S_\ge| \ge n^{1/3}/2 \enspace .
\]
This means there are a lot of high degree vertices.

Now, if we take a random sample $S'$ of $cn^{2/3}\log^2 n$ vertices from $S$,
then, with probability very close to 1
\[
      (1-\epsilon)|S_\ge|\frac{\log^2 n}{n^{1/3}} 
	\le |S'\cap S_\ge| 
	\ge (1+\epsilon)|S_\ge|\frac{\log^2 n}{n^{1/3}} \enspace .
\]
Actually, it's even stronger than this.  Let $S^{i}_\le$ be the number
of vertices of $S'$

\end{document}
