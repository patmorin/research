\documentclass[lotsofwhite]{patmorin}
\usepackage{amsthm}
\usepackage{url}

\title{\MakeUppercase{Centerpoint Theorems for Wedges}}
\author{Jeff Erickson \and
	Ferran Hurtado \and
	Pat Morin}
\date{}
 
%\usepackage{amsthm}

\newcommand{\centeripe}[1]{\begin{center}\Ipe{#1}\end{center}}
\newcommand{\comment}[1]{}

\newcommand{\centerpsfig}[1]{\centerline{\psfig{#1}}}

\newcommand{\seclabel}[1]{\label{sec:#1}}
\newcommand{\Secref}[1]{Section~\ref{sec:#1}}
\newcommand{\secref}[1]{\mbox{Section~\ref{sec:#1}}}

\newcommand{\alglabel}[1]{\label{alg:#1}}
\newcommand{\Algref}[1]{Algorithm~\ref{alg:#1}}
\newcommand{\algref}[1]{\mbox{Algorithm~\ref{alg:#1}}}

\newcommand{\applabel}[1]{\label{app:#1}}
\newcommand{\Appref}[1]{Appendix~\ref{app:#1}}
\newcommand{\appref}[1]{\mbox{Appendix~\ref{app:#1}}}

\newcommand{\tablabel}[1]{\label{tab:#1}}
\newcommand{\Tabref}[1]{Table~\ref{tab:#1}}
\newcommand{\tabref}[1]{Table~\ref{tab:#1}}

\newcommand{\figlabel}[1]{\label{fig:#1}}
\newcommand{\Figref}[1]{Figure~\ref{fig:#1}}
\newcommand{\figref}[1]{\mbox{Figure~\ref{fig:#1}}}

\newcommand{\eqlabel}[1]{\label{eq:#1}}
\newcommand{\eqref}[1]{(\ref{eq:#1})}

\newtheorem{thm}{Theorem}{\bfseries}{\itshape}
\newcommand{\thmlabel}[1]{\label{thm:#1}}
\newcommand{\thmref}[1]{Theorem~\ref{thm:#1}}

\newtheorem{lem}{Lemma}{\bfseries}{\itshape}
\newcommand{\lemlabel}[1]{\label{lem:#1}}
\newcommand{\lemref}[1]{Lemma~\ref{lem:#1}}

\newtheorem{cor}{Corollary}{\bfseries}{\itshape}
\newcommand{\corlabel}[1]{\label{cor:#1}}
\newcommand{\corref}[1]{Corollary~\ref{cor:#1}}

\newtheorem{obs}{Observation}{\bfseries}{\itshape}
\newcommand{\obslabel}[1]{\label{obs:#1}}
\newcommand{\obsref}[1]{Observation~\ref{obs:#1}}

\newtheorem{assumption}{Assumption}{\bfseries}{\rm}
\newenvironment{ass}{\begin{assumption}\rm}{\end{assumption}}
\newcommand{\asslabel}[1]{\label{ass:#1}}
\newcommand{\assref}[1]{Assumption~\ref{ass:#1}}

\newcommand{\proclabel}[1]{\label{alg:#1}}
\newcommand{\procref}[1]{Procedure~\ref{alg:#1}}

\newtheorem{rem}{Remark}
\newtheorem{op}{Open Problem}

\newcommand{\etal}{\emph{et al}}

\newcommand{\voronoi}{Vorono\u\i}
\newcommand{\ceil}[1]{\left\lceil #1 \right\rceil}
\newcommand{\floor}[1]{\left\lfloor #1 \right\rfloor}



\newcommand{\blau}[1]{\tan(\arctan(#1)+\pi/3)}
\newcommand{\bleu}[1]{\tan(\arctan(#1)-\pi/3)}
\newcommand{\blah}[1]{2\arccos(1/#1)}
\newcommand{\crap}[1]{(?)_d}
\newcommand{\nov}{d(d+1)/2-1}

\begin{document}
\maketitle

\begin{abstract}
The \emph{Centerpoint Theorem} states that, for any set $S$ of $n$
points in $\mathbb{R}^2$, there exists a point $p$ in $\mathbb{R}^2$
such that every closed halfplane containing $p$ contains at least
$\ceil{n/3}$ points of $S$. Furthermore, such a point can be found in
$O(n)$ time using an algorithm of Jadhav and Mukhopadhyay (1994).  We
consider a generalization of the Centerpoint Theorem in which
halfplanes are replaced with wedges (cones) of angle $\alpha$.  We
give bounds that are tight for all values of $\alpha$ and give an
$O(n)$ time algorithm to find a point satisfying these bounds.
\end{abstract}

\section{Introduction}

Let $S$ be a set of $n$ points in $\mathbb{R}^d$.  The \emph{halfspace
depth} \cite{t75} of a point $p$ with respect to $S$ is defined as
\[
D_\pi(p,S) = 
   \min\{|h\cap S| : \mbox{$h$ is a closed halfspace that contains $p$} \}
    \enspace .
\]
The \emph{Centerpoint Theorem}, which is a simple consequence of
Helly's Theorem \cite{e93} states that for any point set $S$ of size
$n$, there exists a point whose halfspace depth is at least
$\ceil{n/(d+1)}$.  Furthermore, for every $n>0$, there exists a point
set $S$ in $\mathbb{R}^d$ of size $n$ for which no point in
$\mathbb{R}^d$ has halfspace depth greater than $\ceil{n/(d+1)}$. 

In this paper we consider a generalization of halfspace depth that we
call \emph{$\alpha$-wedge depth}. Let $r$ be a ray with endpoint $q$.
An $\alpha$-wedge with apex $q$ and axis $r$ is the set of all points
$x$ such that the angle\footnote{We use the convention that the angle
between two line segments (or, in this case, a ray and a line segment)
with an endpoint in common is the smaller of the two angles occuring
at the common point.} between $pq$ and $r$ is at most $\alpha/2$.  The
$\alpha$-wedge depth of a point $p$ with respect to a point set $S$ is
defined as
\[
D_\alpha(p,S) =
   \min\{|h\cap S| : \mbox{$h$ is an $\alpha$-wedge that contains $p$} \} 
   \enspace .
\]
Define the function $f^d_\alpha(n)$ as follows:
\[
   f^d_\alpha(n) = \min\left\{\max\{D_\alpha(p,S):p\in
\mathbb{R}^d\}: S\subseteq\mathbb{R}^d\, ,\, |S|=n\right\}
\]
That is, $f^d_\alpha$ defines, for each $n$, the maximum value $k$ for
which every point set $S$ of size $n$ is guaranteed to define a point
whose $\alpha$-wedge depth with respect to $S$ is at least $k$.  The
Centerpoint Theorem states that $f^d_\pi(n) = \ceil{n/(d+1)}$.  In this paper
we prove the following Theorem about 2-dimensional point sets:

\begin{thm}\thmlabel{twod-result}
\[
   f^2_\alpha(n) = \left\{\begin{array}{ll}
                        1 & \mbox{if $\alpha < \pi$} \\
                        \ceil{n/3} & \mbox{if $\pi\le \alpha < 4\pi/3$} \\
                        \ceil{n/2} & \mbox{if $4\pi/3\le \alpha < 2\pi$} \\
                        n & \mbox{if $\alpha = 2\pi$\enspace .} 
                       \end{array}\right.
\]
Furthermore, for any $\alpha$ and any point set $S$ of size $n$, a
point $p$ such that $D_\alpha(p,S) \ge
f^2_\alpha(n)$ can be found in $O(n)$ expected time.
\end{thm}

We also prove some partial results about $f^d_\alpha$ for dimensions
$d\ge 3$.

\comment{
then generalize the bounds of \thmref{twod-result} to point sets in
$\mathbb{R}^d$ and obtain the following theorem:

\begin{thm}\thmlabel{dd-result}
Let $\theta_d = 2\pi - \blah{d}$.  Then
\[
   f^d_\alpha(n) = \left\{\begin{array}{ll}
                        1 & \mbox{if $\alpha < \pi$} \\
                        \ceil{n/(d+1)} & \mbox{if $\pi\le \alpha < \theta_d$} \\
                        \ceil{n/2} & \mbox{if $\theta_d\le \alpha < 2\pi$} \\
                        n & \mbox{if $\alpha = 2\pi$\enspace .} 
                       \end{array}\right.
\]
\end{thm} 
The constant $\theta_d$ in \thmref{dd-result} appears in the following
way.  It is the angle $\alpha$ of the smallest $\alpha$-wedge whose
apex is a the center of a regular $d$-simplex and that contains $d$ of
the $d+1$ vertices of this simplex.
}

\section{Proof of \thmref{twod-result}}

In this section we prove a sequence of lemmata that immediately imply
\thmref{twod-result}.

\begin{lem}\lemlabel{a}
If $\alpha < \pi$ then $f^2_\alpha(n) = 1$ and a point $p$ such that
$D_\alpha(p,S)\ge 1$ can be found in $O(1)$ time.
\end{lem}

\begin{proof}
To prove the lower bound, we observe that for any non-empty
point set $S$, every point $p\in S$ satisfies $D_\alpha(p,S)\ge 1$, so
$f^2_\alpha(n) \ge 1$. This proves the lower bound and gives an $O(1)$
time algorithm for finding $p$. 

For the upper bound, consider a set $S$ of points that are all
on the $x$-axis.  For any point $p$ on or above the $x$ axis, the
$\alpha$-wedge whose axis is vertical and upwards intersects the $x$
axis in at most one point, therefore $D_\alpha(p,S) \le 1$.  For any
point $p$ below the $x$ axis, the $\alpha$-wedge whose axis is
vertical and downwards does not intersect the $x$ axis at all, so
$D_\alpha(p,S)=0$.  In either case, $D_\alpha(p,S)\le 1$ so $f^2_\alpha(n)
\le 1$.
\end{proof}

\begin{lem}\lemlabel{b}
If $\pi\le \alpha < 4\pi/3$ then $f^2_\alpha(n) = \ceil{n/3}$ and a
point $p$ such that $D_\alpha(p,S)\ge \ceil{n/3}$ can be found in $O(n)$ time.
\end{lem}

\begin{proof} 
For the lower bound, we observe that every $\alpha$-wedge containing
$p$ also contains a halfspace containing $p$.  Therefore, the
Centerpoint Theorem implies that $f^2_\alpha(n) \ge \ceil{n/3}$.  This
proves the lower bound and the algorithm of Jadhav and Mukhopadhyay
\cite{jm94} gives an $O(n)$ time algorithm for finding $p$.

For the upper bound, consider the following point set.  Start with
three rays originating at the origin such that each pair of rays meet at an
angle of $2\pi/3$.  Place $\ceil{n/3}$ or $\floor{n/3}$ points on each
ray, as appropriate, so that the total number of points is $n$.  For
any point $p\in\mathbb{R}^2$, there exists a $4\pi/3$-wedge whose apex
is at $p$ and whose interior intersects only one of the three rays
(the axis of this wedge is parallel to this ray). This $4\pi/3$ wedge
contains an $\alpha$-wedge that contains $p$ and intersects only one
of the three rays, therefore $D_\alpha(p,S)\le \ceil{n/3}$.  Since the
choice of $p$ is arbitrary, this implies that $f^2_\alpha(n) \le
\ceil{n/3}$.  
\end{proof}

The next part of the proof uses the notion of halving lines.  A line
$\ell$ is a \emph{halving line} of a (2-dimensional) point set $S$ of
size $n$ if $\ell$ contains at least one point of $S$ and each of the
two closed halfplanes bounded by $\ell$ contains at least $\ceil{n/2}$
points of $S$.  The following lemma was proven by Fekete and Meijer
\cite[Lemma~2]{fm00} in a different context.  However, for
completeness, we include a proof because an understanding of the
existence proof is required for the algorithm described in upcoming
\lemref{algorithm}.

\begin{lem}\lemlabel{construction}
For any point set $S$ there exists three concurrent halving lines 
of $S$ such that the angle\footnote{We use the convention that
the angle between a pair of lines is the smaller of the two angles
defined by the two lines.} between any pair of lines is $\pi/3$.
\end{lem}

\begin{proof}
To prove the existence of these three halving lines we start with one
vertical halving line and the other two halving lines forming angles
of $\pi/3$ with it, one having positive slope and the other
having negative slope.  If these three halving lines are concurrent
then the construction is complete. 

Otherwise, let one of the lines, call it $\ell$, be distinguished and
directed so that the intersection point $p$ of the remaining two lines
is to the right of $\ell$.  Imagine continuously rotating the three
lines while maintaining the invariant that they are all halving lines
and that the angle between any two is $\pi/3$.  After having rotated
the lines by an angle of $\pi$, the three halving lines are identical
to their initial configuration except that the direction of $\ell$ is
reversed, so now the intersection point of the other two lines is to
the left of $\ell$.  We conclude that at some point during this
process the intersection point of the other two lines must have been
on $\ell$, at which point the three lines were concurrent.  This
completes proof.  
\end{proof}

\begin{lem}\lemlabel{algorithm}
Three halving lines satisfying the conditions of \lemref{construction}
can be found in $O(n)$ time.
\end{lem}

\begin{proof}

To find the three halving lines we apply the prune-and-search paradigm
in much the same way as the algorithm of Lo, Matou\v{s}ek and Steiger
\cite{lms94} for finding planar ham-sandwich cuts.  By the standard
``computational geometry duality'' \cite[Section~1.3.3]{bkoo97}, our problem
is to find three points on the median level of $n$ lines such that
these points are collinear and their $x$-coordinates satisfy a certain
equation.  

More precisely, given a set $S^*$ of $n$ lines (that are dual to the
points of $S$), let
\[
    h_k(x) = \min\{ y : 
        \mbox{$(x,y)$ is on or above at least $k$ lines of
$S^*$}\} 
\]
and let $h=h_{\ceil{n_2}}$.  The set of all points $(x,y)$ satisfying
$y=h_k(x)$ is called the \emph{$k$-level} of $S^*$ or, for
$k=\ceil{n/2}$, the \emph{median level}.
The dual of our problem is to find a value $x$ such that the three points
$(x,h(x))$, $(g_1(x),h(g_1(x)))$ and $(g_2(x),h(g_2(x)))$ are collinear.
Here $g_1(x)=\blau{x}$ and
$g_2(x)=\bleu{x}$ which captures the condition that each pair of halving
lines form an angle of $\pi/3$.  [Informally, the continuity argument in
the proof of \lemref{construction} is equivalent to the observation
that, if the sequence of points $\langle(-\infty,h(-\infty)),
(g_1(-\infty),h(g_1(-\infty)), (g_2(-\infty),h(g_2(-\infty))\rangle$ form
a right (respectively left) turn then the points $\langle(\infty,h(\infty)),
(g_1(\infty),h(g_1(\infty)), (g_2(\infty),h(g_2(\infty))\rangle$ form a
left (respectively right) turn, so there must be some
$x\in(-\infty,\infty)$ such that 
$(x,h(x))$, $(g_1(x),h(g_1(x)))$ and $(g_2(x),h(g_2(x)))$ are
collinear.]

Each iteration in the algorithm of Lo \etal\ \cite{lms94} constructs,
in time linear in $|S^*|$, a trapezoid $T$ that is guaranteed to
contain a ham-sandwich point and that intersects at most $2n/3$ lines
of $S^*$.  The lines in $S^*$ not intersecting $T$ are then discarded
and the algorithm recurses on the remaining lines.  Since a constant
fraction of the lines are discarded in each iteration, the running
time of the algorithm is a geometrically decreasing series and is
therefore $O(|S^*|)$.

In our setting, we are searching for 3 points, so at each iteration we
construct three trapezoids $T$, $T_1$ and $T_2$ such that each
trapezoid intersects at most $\delta m$ lines, for an arbitrarily
small constant $\delta < 1/3$.  We then discard from $S^*$ any line
not intersecting any of the three trapezoids and recurse on the
remaining lines.  Each iteration (described below) takes $O(|S^*|)$
time and decreases the size of $S^*$ by a factor of $3\delta$, so the
entire algorithm runs in $O(|S^*|)=O(n)$ time.

Because the algorithm is recursive the subproblems it solves are
slightly more general than the original problem.  Given a set $S^*$ of
lines, two $x$-coordinates $x_1$ and $x_2$ and three integers $k$,
$k_1$ and $k_2$, the algorithm finds an $x$-coordinate $x\in[x_1,x_2]$
such that the three points $(x,h_k(x))$, $(g_1(x),h_{k_1}(g_1(x)))$
and $(g_2(x),h_{k_2}(g_2(x)))$ are collinear.  Such a value $x$ is
guaranteed \emph{a priori} to exist.  Note that, for our initial
recursive call we set $x_1=-\infty$, $x_2=\infty$, and
$k=k_1=k_2=\ceil{n/2}$.

All that remains is to show how to implement a single iteration of the
algorithm in $O(|S^*|)$ time.  To begin, we create a set $X$ of
$x$-coordinates that initially contains the values $x_1$ and $x_2$.
Next we add to $X$ an additional $O(1)$ values so that, for any two
consecutive elements of $X$, the arrangement of our $m$ lines contains
at most $(\delta m)^2/16$ vertices that have $x$-coordinates between
these two elements of $X$.  These additional values can be found in
$O(|S^*|)$ time using (e.g.) the algorithm of Matou\v{s}ek \cite{m91}
(or much more simply by random sampling).  Finally, for each value
$x\in X$ we add the values $g_1^{-1}(x)$ and $g_2^{-1}(x)$ to $X$.
This last step guarantees that, for any two consecutive elements
$x_1'$ and $x_2'$ of $X$, the arrangement of the lines in $S^*$
contains at most $(\delta m)^2/16$ vertices whose $x$ coordinates are
in the range $[g_1(x_1'),g_1(x_2')]$ (respectively
$[g_2(x_1'),g_2(x_2')]$). 

Now, using $O(|X|)=O(1)$ applications of a linear time selection
algorithm (e.g., \cite{bea73}) we can 
find, in $O(|S^*|)$ time, two consecutive elements $x_1'$
and $x_2'$ of $X$ such that $x_1',x_2'\in[x_1,x_2]$ and a solution to our
problem lies in the interval $[x_1',x_2']$.  Consider the trapezoid
$T$ whose four corners are given by $(x_1',h_{k\pm\floor{\delta
m/4}}(x_1'))$ and $(x_2',h_{k\pm\floor{\delta m/4}}(x_2'))$.  A simple
argument \cite[Lemma~X]{lms94} shows that this trapezoid intersects at
most $\delta m$ lines of $S^*$ and that the $k$-level of $S^*$ does
not intersect the top or bottom sides of this trapezoid.  Similarly,
there are trapezoids $T_1$ and $T_2$ defined by the four points 
$(g_1(x_1'),h_{k_1\pm\floor{\delta m/4}}(g_1(x_1')))$ and
$(g_1(x_2'),h_{k_1\pm\floor{\delta m/4}}(g_1(x_2')))$ and the four
points
$(g_2(x_1'),h_{k_1\pm\floor{\delta m/4}}(g_2(x_1')))$ and
$(g_2(x_2'),h_{k_1\pm\floor{\delta m/4}}(g_2(x_2')))$, respectively.
The inclusion of the elements of the form $g_1^{-1}(x)$ and
$g_2^{-1}(x)$ in the set $X$ guarantees that, neither $T_1$ 
nor $T_2$ intersect more than $\delta m$ lines in $S^*$
and the $k_1$-level (respectively $k_2$-level) of $S^*$ does not
intersect the top or bottom sides of $T_1$ (respectively $T_2$).

Altogether, this means that there are at least $m-3\delta m$ lines in
$S^*$ that do not intersect any of the trapezoids $T$, $T_1$ or $T_2$.
When we recurse, we discard these lines, set $x_1=x_1'$, $x_2=x_2'$,
and subtract from $k$ (respectively $k_1$ and $k_2$) the number of
discarded lines that pass below $T$ (respectively $T_1$ and $T_2$).
This completes the description of the algorithm and the proof of the
Lemma.
\end{proof}

\begin{lem}\lemlabel{c}
If $4\pi/3\le \alpha < 2\pi$ then $f^2_\alpha(n) = n/2$ and a point
$p$ such that $D_\alpha(p,S)\ge n/2$ can be found in $O(n)$ time.
\end{lem}

\begin{proof}
For the lower bound, consider the three halving lines whose existence
is given by \lemref{construction}.  These three halving lines
naturally define six $\pi/3$-wedges.  Observe that if we take $p$ to
be the common intersection point of the three halving lines then any
$\alpha$-wedge that contains $p$ contains at least 3 consecutive
$\pi/3$-wedges and therefore contains at least $\ceil{n/2}$ points of $S$
(because it contains a closed halfplane bounded by one of the
halving lines).  Therefore, $f^2_\alpha(n)\ge \ceil{n/2}$, and the
point $p$ such that $D_\alpha(p,S)\ge \ceil{n/2}$ can be found in $O(n)$
time using \lemref{algorithm}.

 
For the upper bound, we consider a point set in which the points have
been clustered into two groups of size $\floor{n/2}$ and $\ceil{n/2}$.
Each of the two groups is contained in a unit ball and the distance
between the two groups is very large, say $r$.  Now, observe that any
point $p\in\mathbb{R}^2$ must be at distance $r/2$ from at least one
of the two groups.  This means that, if $r$ is sufficiently large,
then there exists a $(2\pi-\alpha)$-wedge whose apex is $p$ and that
contains this group in its interior.  The complementary $\alpha$-wedge
contains $p$ and does not contain any points of this group.
Therefore, $D_\alpha(p,S)\le \ceil{n/2}$.  Since $p$ was chosen
arbitrarily, we conclude that $f^2_\alpha(n)\le \ceil{n/2}$.
\end{proof}

\begin{proof}[Proof of \thmref{twod-result}] 
The theorem follows immediately from \lemref{a}, \lemref{b}
and \lemref{c}.
\end{proof}

\section{Some Results for $\mathbb{R}^d$}

In this section, we consider $\alpha$-wedge depth in $\mathbb{R}^d$,
$d\ge 3$, and prove  some bounds on the function $f^d_\alpha$.  The
following lemma results from exactly the same arguments used in the
proofs of \lemref{a}, \lemref{b} and \lemref{c}.

\begin{lem}\lemlabel{dd-a}\lemlabel{combined}
$f^\alpha_d$ satisfies the following:
\[\begin{array}{lcll}
  f^d_\alpha(n) & = & 1 & \mbox{if $\alpha < \pi$} \\
  f^d_\alpha(n) & \ge & \ceil{n/(d+1)} & \mbox{if $\alpha \ge \pi$} \\
  f^d_\alpha(n) & \le & \ceil{n/2} & \mbox{if $\alpha < 2\pi$} \\
\end{array}\]
\end{lem}

The following technical lemma is needed for proving an upper bound
that generalizes the construction in \lemref{b}.

\begin{lem}\lemlabel{cone-angle}
Let $T$ be a regular $d$-simplex whose center is at the origin and let
$\theta_d=\blah{d}$.  Then, for any $d$ vertices of $T$, there is a
$\theta_d$-wedge whose apex is at the origin and that contains these
$d$ vertices of $S$.
\end{lem}

\begin{proof} 
Wlog we can consider the regular $d$-simplex whose vertices are
$e_1,\ldots,e_d, ((1-\sqrt{d+1})/d)(e_1+\cdots+e_d)$ where $e_i$ is the
$i$th coordinate unit vector in $\mathbb{R}^d$.  The center of this
simplex is the point $c=\sqrt{d+1}/(d^2+d)(e_1+\cdots+ e_d)$.
Consider the ray $r$ that originates at $c$ and
contains the point $e_1+\cdots+e_d$.  The angle between $r$ and
$e_i$, for any $1\le i\le d$ is easily determined to be
$\theta_d/2$ using the famous formula $\arccos (u\cdot
v/\|u\|\|v\|)$ for the angle between two vectors $u$ and $v$.   Thus,
the $d$-vertices $e_1,\ldots e_d$ are on the boundary of this
$(2\pi-\theta_d)$-wedge, as required.  
\end{proof}

The next lemma is a straightforward generalization of \lemref{b}.
Notice that $2\pi-\theta_d$ approaches $\pi$ from below as $d$ increases.
This means that, for sufficiently large $d$, the upper bound in the
following lemma only holds for $\alpha < \pi+\epsilon$.

\begin{lem}\lemlabel{dd-b}
If $\alpha < 2\pi-\theta_d$ then $f^d_\alpha(n) \le \ceil{n/(d+1)}$.
\end{lem}

\begin{proof} 
We use a generalization of the point set used in the proof of
\lemref{b}.  Let $T$ be a regular $d$-simplex whose center is at the
origin and consider the $d+1$ rays originating at the origin and each
containing a different vertex of $T$.  On each of these rays, place
$\ceil{n/(d+1)}$ or $\floor{n/(d+1)}$ points, as appropriate, to
produce a point set $S$ of size $n$.  We claim that, as in the proof
of \lemref{b}, for any point $p\in\mathbb{R}^2$, there is a
$\theta_d$-wedge that contains $p$ and $d$ of the $d+1$ rays that
contain the points of $S$. 

To see why this is so, let $C_1,\ldots,C_{d+1}$ be the closed cones
obtained by taking the affine hull of each subset of $d$ vertices of
$T$.  Notice that these cones cover $\mathbb{R}^d$ and that each cone
contains $d$ of the $d+1$ rays that contain $S$.  Furthermore, if the
cone $C_i$ contains the point $-p$ then, by \lemref{cone-angle}, there
is a $\theta_d$-wedge whose apex is at $p$ and that contains $C_i$.  

If we consider the complementary $(2\pi-\theta_d)$-wedge then the
interior of this wedge does not intersect $C_i$ and hence intersects
only 1 of the $d+1$ rays that contain $S$. This
$(2\pi-\theta_d)$-wedge contains an $\alpha$-wedge that contains $p$
and contains at most $\ceil{n/(d+1)}$ points of $S$, as required.
\end{proof}

For non-trivial lower bounds on $f^d_\alpha$ we easily obtain the
following result by taking the point $p$ which is common to $d$
orthogonal halving planes of $S$.

\begin{lem}\lemlabel{weak-dd}
If $\alpha \ge \pi+2\arccos(1/\sqrt{d})$ then $f_\alpha^d(n)\ge \ceil{n/2}$.
\end{lem}

\begin{lem}
Let $S$ be a set of $n$ points in $\mathbb{R}^3$.  Then there exists
$4$ concurrent halving hyperplanes of $S$ that are parallel to the
sides a regular tetrahedron.
\end{lem}

\begin{proof}[Proof Sketch]
The columns of the following matrix are the coordinates of the
vertices of a regular tetrahedron.

\[T=\left[\begin{array}{cccccc}
  \sqrt{6}/6 & \sqrt{6}/6 & -\sqrt{6}/3 & 0 \\
  \sqrt{3}/6 & \sqrt{3}/6 & \sqrt{3/6} & -\sqrt{3}/2 \\
  \sqrt{2}/2 & -\sqrt{2}/2 & 0 & 0
\end{array} \right]
\]
Suppose we wish to 
rotate
the vertices of $T$ about the $z$-axis by an angle of $\theta$.
This is accomplished by computing 
$R_\theta T$, where $R_\theta$ is the rotation
matrix
\[R_\theta=\left[\begin{array}{ccccc}
   \cos(\theta)  & -\sin(\theta)  & 0 \\
   \sin(\theta)  & \cos(\theta)   & 0 \\
   0           & 0           & 1  
\end{array}\right]\]
In particular, if we set $\theta=\pi$ then we obtain
\[R_\pi T=\left[\begin{array}{cccccc}
  -\sqrt{6}/6 & -\sqrt{6}/6 & \sqrt{6}/3 & 0 \\
  -\sqrt{3}/6 & -\sqrt{3}/6 & -\sqrt{3/6} & \sqrt{3}/2 \\
  \sqrt{2}/2 & -\sqrt{2}/2 & 0 & 0
\end{array} \right]
\]
Remarkably, the set of vectors we obtain in this way is the reflection
of the original set of vectors.  That is, if we treat $T$ as a set of
column vectors then $R_\theta T=-T$ because $R_\theta v_1=-v2$,
$R_\theta v_2=-v_1$, $R_\theta v_3=-v_3$ and $R_\theta v_4=-v_4$.
(This does not seem to generalize to $d$ dimensions for any $d>3$.)
\end{proof}


Notice that the threshold value for $\alpha$ in the above lemma
approaches $2\pi$ from below as $d$ gets large, so for sufficiently
large $d$, the bound only holds for $\alpha \ge 2\pi -\epsilon$.

Putting everything together, we get the picture so far:

\begin{tabular}{l|cc}
$d$ & Upper bound of $\ceil{n/(d+1)}$ & Lower bound of $\ceil{n/2}$ \\
\hline
2 & $\alpha < 240.00$ & $\alpha \ge 240.00$ \\
3 & $\alpha < 218.94$ & $\alpha \ge 289.47$ \\
4 & $\alpha < 208.96$ & $\alpha \ge ??$ \\
5 & $\alpha < 203.07$ & $\alpha \ge ??$ \\
6 & $\alpha < 199.19$ & $\alpha \ge ??$ \\
\end{tabular}

\begin{lem}
$f^\alpha_d$ satisfies the following:
\[\begin{array}{lcll}
  f^d_\alpha(n) & = & 1 & \mbox{if $\alpha < \pi$} \\
  f^d_\alpha(n) & = & \ceil{n/(d+1)} & \mbox{if $\pi \le \alpha < \theta_d$} \\
  f^d_\alpha(n) & = & ? & \mbox{if $\theta_d \le \alpha < \crap{d}$} \\
  f^d_\alpha(n) & = & \ceil{n/2} & \mbox{if $\crap{d}\le \alpha < 2\pi$} \\
\end{array}\]
\end{lem}





\section{Conclusions}

We have completely determined the function $f^2_\alpha$ and given a
linear-time algorithm for finding a point $p$ such that
$D_\alpha(p.S)\ge f^2_\alpha(|S|)$.  Our main new algorithmic result
is a linear-time algorithm for finding 3 concurrent halving lines,
each pair of which forms an angle of $\pi/3$.  These triples of
halving lines were used by Fekete and Meijer to show that the cost of
a minimum Steiner star of an $n$ point set is at most $2/\sqrt{3}$
times the cost of the minimum-cost perfect matching of the same set.
Our algorithm gives an $O(n)$ time construction of a Steiner star
matching this bound.



We conclude with a list of open problems:

\begin{enumerate}
\item Given a point set $S$ in $\mathbb{R}^2$, what is the complexity
of finding a point $p\in\mathbb{R}^2$ that maximizes $D_\alpha(p,S)$?
For $\alpha=\pi$, i.e., halfplane depth, Chan has recently given an
$O(n\log n)$ time algorithm \cite{c04}.

\item Given a set $S$ of $n$ points in $\mathbb{R}^d$, for $d\ge 3$,
what is the complexity of finding a point $p\in \mathbb{R}^d$ such
that $p\ge f_\alpha^d(n)$?  This problem is still open even for the
case $\alpha=\pi$, though some progress has been made for the case
$d=X$ \cite{X}.

\item In this paper we have only considered $\alpha$-wedges.  This is
mainly because, for $d=2$, these are more or less the only interesting
scale-invariant objects.  However, in higher dimensions, one can
define many scale-invariant shapes.  In general, for any shape $F$,
one can study the properties of $F$-depth:
\[
  D_F(p,S) = \min\left\{ 
     h\cap S : \mbox{$h$ is an $F$ that contains $p$}
  \right\} \enspace .
\] 
\end{enumerate}

\section*{Acknowledgements}

This research was initiated while Pat Morin was visiting Universitat
Polit\`ecnica de Catalunya (UPC).  His visit was supported by NSERC
and XXXXX.  

The authors would like to thank Jit Bose, Anil Maheshwari, Perouz
Taslakian, Michiel Smid, and David Wood for helpful discussions.

\bibliographystyle{plain}
\bibliography{wedge}





\end{document}
