\documentclass[lotsofwhite,charterfonts]{patmorin}
\usepackage{graphicx}
\usepackage{charter}
\usepackage{url}
\usepackage{amsfonts,amsthm}

 
%\usepackage{amsthm}

\newcommand{\centeripe}[1]{\begin{center}\Ipe{#1}\end{center}}
\newcommand{\comment}[1]{}

\newcommand{\centerpsfig}[1]{\centerline{\psfig{#1}}}

\newcommand{\seclabel}[1]{\label{sec:#1}}
\newcommand{\Secref}[1]{Section~\ref{sec:#1}}
\newcommand{\secref}[1]{\mbox{Section~\ref{sec:#1}}}

\newcommand{\alglabel}[1]{\label{alg:#1}}
\newcommand{\Algref}[1]{Algorithm~\ref{alg:#1}}
\newcommand{\algref}[1]{\mbox{Algorithm~\ref{alg:#1}}}

\newcommand{\applabel}[1]{\label{app:#1}}
\newcommand{\Appref}[1]{Appendix~\ref{app:#1}}
\newcommand{\appref}[1]{\mbox{Appendix~\ref{app:#1}}}

\newcommand{\tablabel}[1]{\label{tab:#1}}
\newcommand{\Tabref}[1]{Table~\ref{tab:#1}}
\newcommand{\tabref}[1]{Table~\ref{tab:#1}}

\newcommand{\figlabel}[1]{\label{fig:#1}}
\newcommand{\Figref}[1]{Figure~\ref{fig:#1}}
\newcommand{\figref}[1]{\mbox{Figure~\ref{fig:#1}}}

\newcommand{\eqlabel}[1]{\label{eq:#1}}
\newcommand{\eqref}[1]{(\ref{eq:#1})}

\newtheorem{thm}{Theorem}{\bfseries}{\itshape}
\newcommand{\thmlabel}[1]{\label{thm:#1}}
\newcommand{\thmref}[1]{Theorem~\ref{thm:#1}}

\newtheorem{lem}{Lemma}{\bfseries}{\itshape}
\newcommand{\lemlabel}[1]{\label{lem:#1}}
\newcommand{\lemref}[1]{Lemma~\ref{lem:#1}}

\newtheorem{cor}{Corollary}{\bfseries}{\itshape}
\newcommand{\corlabel}[1]{\label{cor:#1}}
\newcommand{\corref}[1]{Corollary~\ref{cor:#1}}

\newtheorem{obs}{Observation}{\bfseries}{\itshape}
\newcommand{\obslabel}[1]{\label{obs:#1}}
\newcommand{\obsref}[1]{Observation~\ref{obs:#1}}

\newtheorem{assumption}{Assumption}{\bfseries}{\rm}
\newenvironment{ass}{\begin{assumption}\rm}{\end{assumption}}
\newcommand{\asslabel}[1]{\label{ass:#1}}
\newcommand{\assref}[1]{Assumption~\ref{ass:#1}}

\newcommand{\proclabel}[1]{\label{alg:#1}}
\newcommand{\procref}[1]{Procedure~\ref{alg:#1}}

\newtheorem{rem}{Remark}
\newtheorem{op}{Open Problem}

\newcommand{\etal}{\emph{et al}}

\newcommand{\voronoi}{Vorono\u\i}
\newcommand{\ceil}[1]{\left\lceil #1 \right\rceil}
\newcommand{\floor}[1]{\left\lfloor #1 \right\rfloor}



\newcommand{\voronoi}{Vorono\u\i}
\newcommand{\ceil}[1]{\lceil #1 \rceil}
\newcommand{\adt}{\mathrm{ADT}}


\newcommand{\email}[1]{\texttt{#1}}

\title{\MakeUppercase{Output-Sensitive Algorithms for Computing 
	Nearest-Neighbour Decision Boundaries}\thanks{%
	This research was partly funded by the Alexander von Humboldt
	Foundation and The Natural Sciences and Engineering Research
	Council of Canada.}}

\author{David~Bremner\thanks{Faculty of Computer Science,
	University of New Brunswick,
	%Box 4400 Fredericton New Brunswick E3B 5A3 Canada
	\email{bremner@unb.ca}} \and
	Erik~Demaine\thanks{MIT Laboratory for Computer Science,
       	%200 Technology Square, Cambridge, MA 02139, USA,
       	\email{edemaine@mit.edu}}  \and
	Jeff~Erickson\thanks{Computer Science Department, University of Illinois,
	\email{jeffe@cs.uiuc.edu}} \and
	John~Iacono\thanks{Polytechnic University, 
	%5 MetroTech Center, Brooklyn, New York, USA, 11530. 
	\email{jiacono@poly.edu}} \and
	Stefan~Langerman\thanks{Charg\'e de recherches du FNRS, Universit\'e Libre de
	Bruxelles, 
	%CP212, boulevard du Triomphe, 1050 Bruxelles.
	\email{stefan.langerman@ulb.ac.be}} \and
	Pat~Morin\thanks{School of Computer Science, Carleton University,
	%1125 Colonel By Drive, Ottawa, K1S~5B6, CANADA,
	\email{morin@cs.carleton.ca}} \and
	Godfried~Toussaint\thanks{School of Computer Science, McGill University,
	%3480 University St., Montreal, Quebec, CANADA, H3A~2A7.
	\email{godfried@cs.mcgill.ca}}
} 
	

\date{}

\begin{document}

\maketitle

\begin{abstract} 
Given a set $R$ of red points and a set $B$ of blue points, the
\emph{nearest-neighbour decision rule} classifies a new point $q$ as
red (respectively, blue) if the closest point to $q$ in $R\cup B$
comes from $R$ (respectively, $B$).  This rule implicitly partitions
space into a red set and a blue set that are separated by a red-blue
\emph{decision boundary}.  In this paper we develop output-sensitive
algorithms for computing this decision boundary for point sets on the
line and in $\mathbb{R}^2$.  Both algorithms run in time $O(n\log k)$,
where $k$ is the number of points that contribute to the  decision
boundary.  This running time is the best possible when parameterizing
with respect to $n$ and $k$.  
\end{abstract}

%%%%%%%%%%%%%%%%%%%%%%%%%%%%%%%%%%%%%%%%%%%%%%%%%%%%%%%%%%%%%%%%%%%%%%%%
\section{Introduction}

Let $S$ be a set of $n$ points in the plane that is partitioned into a
set of \emph{red points} denoted by $R$ and a set of \emph{blue
points} denoted by $B$.  The \emph{nearest-neighbour decision rule}
classifies a new point $q$ as the colour of the closest point to $q$ in
$S$.  The nearest-neighbour decision rule is popular in pattern
recognition as a means of learning by example.  For this reason, the
set $S$ is often referred to as a \emph{training set}. 

Several properties make the nearest-neighbour decision rule quite
attractive, including its intuitive simplicity and the theorem that
the asymptotic error rate of the nearest-neighbour rule is bounded
from above by twice the Bayes error rate \cite{ch67,d81,s77}.  (See
\cite{t03} for an extensive survey of the nearest-neighbour decision
rule and its relatives.) Furthermore, for point sets in small
dimensions, there are efficient and practical algorithms for
preprocessing a set $S$ so that the nearest neighbour of a query point
$q$ can be found quickly.

The nearest-neighbour decision rule implicitly partitions the plane
into a red set and a blue set that meet at a red-blue \emph{decision
boundary}.  One attractive aspect of the nearest-neighbour decision
rule is that it is often possible to reduce the size of the training
set $S$ without changing the decision boundary.  To see this, consider
the \emph{\voronoi\ diagram} of $S$, which partitions the plane into
convex (possibly unbounded) polygonal \emph{\voronoi\ cells}, where
the \voronoi\ cell of point $p\in S$ is the set of all points that are
closer to $p$ than to any other point in $S$ (see \figref{voronoi}.a).
If the \voronoi\ cell of a red point $r$ is completely surrounded by
the \voronoi\ cells of other red points then the point $r$ can be
removed from $S$ and this will not change the classification of any
point in the plane (see \figref{voronoi}.b).   We say that these
points \emph{do not contribute} to the decision boundary, and the
remaining points \emph{contribute} to the decision boundary.


\begin{figure}
\begin{center}\begin{tabular}{cc}
%\includegraphicsFit{2.2in}\includegraphics{voronoi-bw} & \includegraphicsFit{2.2in}\includegraphics{voronoi-condense-bw} \\
\includegraphics{boundary-a} & \includegraphics{boundary-b} \\
(a) & (b)
\end{tabular}\end{center}
\caption{The \voronoi\ diagram (a)~before \voronoi\ condensing
and (b)~after \voronoi\ condensing.  Note that the decision boundary
(in bold) is unaffected by \voronoi\ condensing. Note: In this
figure, and all other figures, red points are denoted by white circles 
and blue points are denoted by black disks.}
\figlabel{voronoi}
\end{figure}

The preceding discussion suggests that one approach to reducing the
size of the training set $S$ is to simply compute the \voronoi\
diagram of $S$ and remove any points of $S$ whose \voronoi\ cells are
surrounded by \voronoi\ cells of the same colour.  Indeed, this method
is referred to as \emph{\voronoi\ condensing} \cite{tbp84}.  There are
several $O(n\log n)$ time algorithms for computing the \voronoi\
diagram a set of points in the plane, so \voronoi\ condensing can be
implemented to run in $O(n\log n)$ time.\footnote{Historically, the
first efficient algorithm for specifically computing the
nearest-neighbour decision boundary is due to Dasarathy and White
\cite{dw78} and runs in $O(n^4)$ time.  The first $O(n\log n)$ time
algorithm for computing the \voronoi\ diagram of a set of $n$ points
in the plane is due to Shamos \cite{s75}.}  However, in this paper we
show that we can do significantly better when the number of points
that contribute to the decision boundary is small.  Indeed, we show
how to do \voronoi\ condensing in $O(n\log k)$ time, where $k$ is the
number of points that contribute to the decision boundary (i.e., the
number of points of $S$ that remain after \voronoi\ condensing).  We
also show that the same result holds even if there are $c>2$ colour
classes.
Algorithms, like these, in which the size of the input and
the size of the output play a role in the running time are referred to
as \emph{output-sensitive} algorithms. 

Readers familiar with the literature on output-sensitive convex hull
algorithms may recognize the expression $O(n\log k)$ as the running
time of optimal algorithms for computing convex hulls of $n$ point
sets with $k$ extreme points, in 2 or 3 dimensions
\cite{bs97,c96,csy97,ks86,w97}. This is no coincidence.  Given a set
of $n$ points in $\mathbb{R}^2$, we can colour them all red and add
three blue points at infinity (see \figref{convexhull}).  In this set,
the only points that contribute to the nearest-neighbour decision
boundary are the three blue points and the red points on the convex
hull of the original set.  Thus, identifying the points that
contribute to the nearest-neighbour decision boundary is at least as
difficult as computing the extreme points of a set.

\begin{figure}
\centerline{\includegraphics{convexhull-bwx}}
\caption{The relationship between convex hulls and decision
boundaries. Each vertex of the convex hull of $R$ contributes 
to the decision boundary.}
\figlabel{convexhull}
\end{figure}

Observe that, once the size of the training set has been reduced by
\voronoi\ condensing, the condensed set can be preprocessed in $O(k\log
k)$ time to answer nearest neighbour queries in $O(\log k)$ time per
query.  This makes it possible to do nearest-neighbour classifications
in $O(\log k)$ time.  Alternatively, the algorithm we describe for
computing the nearest neighbour decision boundary actually produces
the \voronoi\ diagram of the condensed set (which has size $O(k)$)
that can be preprocessed in $O(k)$ time by Kirkpatrick's
point-location algorithm \cite{k83} to allow nearest neighbour
classification in $O(\log k)$ time.

The remainder of this paper is organized as follows: In \secref{one-d}
we describe an algorithm for computing the nearest-neighbour decision
boundary of points on a line that runs in $O(n\log k)$ time.  In
\secref{two-d} we present an algorithm for points in the plane that
also runs in $O(n\log k)$ time.  Finally, in \secref{conclusions} we
summarize and conclude with open problems.

%%%%%%%%%%%%%%%%%%%%%%%%%%%%%%%%%%%%%%%%%%%%%%%%%%%%%%%%%%%%%%%%%%%%%%%%
\section{A 1-Dimensional Algorithm}\seclabel{one-d}

In the 1-dimensional version of the nearest-neighbour decision
boundary problem, the input set $S$ consists of $n$ real numbers.
Imagine sorting $S$, so that $S=\{s_1,\ldots,s_n\}$ where
$s_i<s_{i+1}$ for all $1\le i< n$.  The decision boundary consists of
all pairs $(s_i,s_{i+1})$ where $s_i$ is red and $s_{i+1}$ is blue, or
\emph{vice-versa}.  Thus, this problem is solvable in linear-time if
the points of $S$ are sorted.  Since sorting the elements of $S$ can
be done using any number of $O(n\log n)$ time sorting algorithms, this
immediately implies an $O(n\log n)$ time algorithm.  Next, we give an
algorithm that runs in $O(n\log k)$ time and is similar in spirit to
Hoare's quicksort \cite{h61}.

To find the decision boundary in $O(n\log k)$ time, we begin by
computing the median element $m=s_{\ceil{n/2}}$ in $O(n)$ time using
any one of the existing linear-time median finding algorithms (see
\cite{bfprt73}).  Using an additional $O(n)$ time, we split $S$ into
the sets $S_1=\{s_1,\ldots,s_{\ceil{n/2}-1}\}$ and
$S_2=\{s_{\ceil{n/2}+1},\ldots,s_n\}$ by comparing each element of $S$
to the median element $m$.  At the same time we also find
$s_{\ceil{n/2}-1}$ and $s_{\ceil{n/2}+1}$ by finding the maximum and
minimum elements of $S_1$ and $S_2$, respectively.  We then check if
$(s_{\ceil{n/2}-1},m)$ and/or $(m,s_{\ceil{n/2}+1})$ are part of the
decision boundary and report them if necessary.  

At this point, a standard divide-and-conquer algorithm would recurse
on both $S_1$ and $S_2$ to give an $O(n\log n)$ time algorithm.
However, we can improve on this by observing that it is not necessary
to recurse on a subproblem if it contains only elements of one colour,
since it will not contribute a pair to the decision boundary.
Therefore, we recurse on each of $S_1$ and $S_2$ only if they contain
at least one red element and one blue element.

The correctness of the above algorithm is clear.  To analyze its
running time we observe that the running time is bounded by the
recurrence
\[
   T(n,k) \le O(n) + T\left(n/2,l\right) 
   	+ T\left(n/2,k-l\right) \enspace
  , 
\]
where $l$ is the number of points that contribute to the decision
boundary in  $S_1$ and where $T(1,k)=O(1)$ and $T(n,0)=O(n)$.  An
easy inductive argument that uses the concavity of the logarithm shows
that this recurrence is maximized when $l=k/2$, in which case the
recurrence solves to $O(n\log k)$ \cite{csy97}.

\begin{thm}
The nearest-neighbour decision boundary of a set of $n$ red and blue
real numbers can be computed in $O(n\log k)$ time, where $k$ is the
number of elements that contribute to the decision boundary.
\end{thm}



%%%%%%%%%%%%%%%%%%%%%%%%%%%%%%%%%%%%%%%%%%%%%%%%%%%%%%%%%%%%%%%%%%%%%%%%
\section{A 2-Dimensional Algorithm}\seclabel{two-d}

In the 2-dimensional nearest-neighbour decision boundary problem the
\voronoi\ cells of $S$ are (possibly unbounded) convex polygons and
the goal is to find all \voronoi\ edges that bound two cells whose
defining points have different colours.  Throughout this section we
will assume that the points of $S$ are in \emph{general position} so
that no four points of $S$ lie on a common circle.  This assumption is
not very restrictive, since general position can be simulated using
infinitesmal perturbations of the input points.

It will be more convenient to present our algorithm using the
terminology of Delaunay triangulations.  A \emph{Delaunay triangle} in
$S$ is a triangle whose vertices $(v_1,v_2,v_3)$ are in $S$ and such
that the circle with $v_1$, $v_2$ and $v_3$ on its boundary does not
contain any point of $S$ in its interior.  A \emph{Delaunay
triangulation} of $S$ is a partitioning of the convex hull of $S$ into
Delaunay triangles.  Alternatively, a \emph{Delaunay edge} is a line
segment whose vertices $(v_1,v_2)$ are in $S$ and such that there
exists a circle with $v_1$ and $v_2$ on its boundary that does not
contain any point of $S$ in its interior.  When $S$ is in general
position, the Delaunay triangulation of $S$ is unique and contains all
triangles whose edges are Delaunay edges (see \cite{ps85}).  It is
well known that the Delaunay triangulation and the \voronoi\ diagram are
dual in the sense that two points of $S$ are joined by an edge in the
Delaunay triangulation if and only if their \voronoi\ cells share an
edge.  

We call a Delaunay triangle or Delaunay edge \emph{bichromatic} if its
set of defining vertices contains at least one red and at least one
blue point of $S$. Thus, the problem of computing the
nearest-neighbour decision boundary is equivalent to the problem of
finding all bichromatic Delaunay edges. 

%=======================================================================
\subsection{The High Level Algorithm}\seclabel{outline}

In the next few sections, we will describe an algorithm that, given a
value $\kappa\ge k$, finds the set of all bichromatic Delaunay
triangles in $S$ in $O((\kappa^2 + n)\log \kappa)$ time, which
for $\kappa\le\sqrt{n}$ simplifies to $O(n\log\kappa)$.  To obtain an
algorithm that runs in $O(n\log k)$ time, we repeatedly guess the
value of $\kappa$, run the algorithm until we find the entire decision
boundary or until it determines that $\kappa< k$ and, in the latter
case, restart the algorithm with a larger value of $\kappa$.  If we
ever reach a point where the value of $\kappa$ exceeds $\sqrt{n}$ then
we stop the entire algorithm and run an $O(n\log n)$ time algorithm to
compute the entire Delaunay triangulation of $S$.

The values of $\kappa$ that we use are $\kappa=2^{2^i}$ for
$i=0,1,2,\ldots,\ceil{\log\log n}$.  Since the algorithm will
terminate once $\kappa\ge k$ or $\kappa\ge\sqrt{n}$, the total cost of
all runs of the algorithm is therefore \[ T(n,k) =
\sum_{i=0}^{\ceil{\log\log k}} O(n\log 2^{2^i}) =
\sum_{i=0}^{\ceil{\log\log k}} O(n{2^i}) = O(n\log k) \enspace , \] as
required.


%=======================================================================
\subsection{Pivots}

A key subroutine in our algorithm is the \emph{pivot}\footnote{The
term pivot comes from linear programming.  The relationship between a
(polar dual) linear programming pivot and the circular pivot described
here is evident when we consider the parabolic lifting that transforms
the problem of computing a 2-dimensional Delaunay triangulation to
that of computing a 3-dimensional convex hull of a set of points on
the paraboloid $z=x^2+y^2$.  In this case, the circle is the
projection of the intersection of a plane with the paraboloid.}
operation illustrated in \figref{pivot}. A pivot in the set of points
$S$ takes as input a ray and reports the largest circle whose center
is on the ray, has the origin of the ray on its boundary and has no
point of $S$ in its interior.  We will make use of the following data
structuring result, due to Chan \cite{c96}.  For completeness, we also
include a proof.

\begin{figure}
\begin{center}\begin{tabular}{c@{\hspace{1.5cm}}c}
\includegraphics{pivot-a-bw}
\end{tabular}\end{center}
\caption{A pivot operation.}
\figlabel{pivot}
\end{figure}


\begin{lem}[Chan 1996]\lemlabel{chan} Let $S$ be a set of $n$ points
in $\mathbb{R}^2$.  Then, for any integer $1\le m\le n$, there exists a
data structure of size $O(n)$ that can be constructed in $O(n\log m)$
time, and that can perform pivots in $S$ in $O(\frac{n}{m}\log m)$
time per pivot.  \end{lem}

\begin{proof} 
Dobkin and Kirkpatrick \cite{dk83,dk85} show how to preprocess a set
$S$ of $n$ points in $O(n\log n)$ time to answer pivot queries in
$O(\log n)$ time per query.  Chan's data structure simply partitions
$S$ into $n/m$ groups each of size $m$ and then uses the
Dobkin-Kirkpatrick data structure on each group.  The time to build
all $n/m$ data structures is $\frac{n}{m}\times O(m\log m)=O(n\log
m)$.  To perform a query, we simply query each of the $n/m$ data
structures in $O(\log m)$ time per data structure and report the
smallest circle found, for a query time of $\frac{n}{m}\times O(\log
m)=O(\frac{n}{m}\log m)$.  
\end{proof}


In the following, we will be using \lemref{chan} with a value of
$m=\kappa^2$, so that the time to construct the data structure is
$O(n\log\kappa)$ and the query time is
$O(\frac{n}{\kappa^2}\log\kappa)$.  We will use two such data
structures, one for performing pivots in the set $R$ of red points
and one for performing pivots in the set $B$ of blue points.

%=======================================================================
\subsection{Finding the First Edge}

The first step in our algorithm is to find a single bichromatic edge
of the Delaunay triangulation.  Refer to \figref{first}.  To do this,
we begin by choosing any red point $r$ and any blue point $b$.  We
then perform a pivot in the set $B$ along the ray with origin $r$ that
contains $b$.  This gives us a circle $C$ that has no blue points in
its interior and has $r$ as well as some blue point $b'$ (possibly
$b=b'$) on its boundary.  Next, we perform a pivot in the set $R$
along the ray originating at $b'$ and passing through the center of
$C$.  This gives us a circle $C_1$ that has no point of $S$ in its
interior and has $b'$ and some red point $r'$ (possibly $r=r'$) on its
boundary.  Therefore, $(r',b')$ is a bichromatic edge in the Delaunay
triangulation of $S$.

\begin{figure}
\begin{center}\begin{tabular}{c@{\hspace{1.5cm}}c}
\includegraphics{first-a-bw} & \includegraphics{first-b-bw} \\
(a) & (b)
\end{tabular}\end{center}
\caption{The (a)~first and (b)~second pivot used to find a
bichromatic edge $(r',b')$.}
\figlabel{first}
\end{figure}

The above argument shows how to find  a bichromatic Delaunay edge
using only 2 pivots, one in $R$ and one in $B$.  The second part of
the argument also implies the following useful lemma.

\begin{lem}\lemlabel{finding} 
If there is a circle with a red point $r$ and a blue point $b$ on
its boundary, and no red (respectively, blue) points in its interior,
then $r$ (respectively, $b$) contributes to the decision boundary.  
\end{lem}

%=======================================================================
\subsection{Finding More Points}

Let $Q$ be the set of points that contribute to the decision
boundary, i.e., the set of points that are the vertices of bichromatic
triangles in the Delaunay triangulation of $S$.  Suppose that we have
already found a set $P\subseteq Q$ and we wish to either (1)~find a
new point $p\in Q\setminus P$ or (2)~verify that $P=Q$.  

To do this, we will make use of the \emph{augmented Delaunay
triangulation} of $P$ (see \figref{augmented}).  This is the
Delaunay triangulation of $P\cup \{v_1,v_2,v_3\}$, where $v_1$,
$v_2$, and $v_3$ are three \emph{black} points ``at infinity'' (see
\figref{augmented}). For any triangle $t$, we use the notation $C(t)$
to denote the circle whose boundary contains the three vertices of $t$
(note that if $t$ contains a black point then $C(t)$ is a halfplane).
The following lemma allows us to tell when we have found the entire
set of points $Q$ that contribute to the decision boundary.

\begin{figure}
\centerline{\includegraphics{augmented-bw}}
\caption{The augmented Delaunay triangulation of
$S$.}\figlabel{augmented}
\end{figure}


\begin{lem}\lemlabel{stopping} 
Let $\emptyset\neq P\subseteq Q$.  The following statements are equivalent:
\begin{enumerate}

\item For every triangle $t$ in the augmented Delaunay triangulation
of $P$, if $t$ has a blue (respectively, red) vertex then $C(t)$
does not have a red (respectively, blue) point of $S$ in its
interior.  

\item $P=Q$.
\end{enumerate}
\end{lem}

\begin{proof} 
First we show that if Statement~1 of the lemma is not true, then
Statement~2 is also not true, i.e., $P\neq Q$.  Suppose there is some
triangle $t$ in the augmented Delaunay triangulation of $P$ such that
$t$ has a blue vertex $b$ and $C(t)$ contains a red point of $S$ in
its interior.  Pivot in $R$ along the ray originating at $b$ and
passing through the center of $C(t)$ (see \figref{witness}).  This
will give a circle $C$ with $b$ and some red point $r\notin P$ on its
boundary and with no red points in its interior.  Therefore, by
\lemref{finding}, $r$ contributes to the decision boundary and is
therefore in $Q$, so $P\neq Q$.  A symmetric argument applies when $t$
has a red vertex $r$ and $C(t)$ contains a blue vertex in its
interior.

\begin{figure}
\centerline{\includegraphics{monochrome-bw}}
\caption{If Statement~1 of \lemref{stopping} is not true then
$P\neq Q$.}
\figlabel{witness}
\end{figure}

Next we show that if Statement~2 of the lemma is not true then
Statement~1 is not true.  Suppose that $P\neq Q$.  Let $r$ be a point
in $Q\setminus P$ and, without loss of generality, assume $r$ is a red
point.  Since $r\in Q$, there is a circle $C$ with $r$ and some
other blue point $b$ on its boundary and with no points of $S$ in its
interior.  We will use $r$ and $b$ to show that the augmented Delaunay
triangulation of $P$ contains a triangle $t$ such that either (1)~$b$
is a vertex of $t$ and $C(t)$ contains $r$ in its interior, or
(2)~$C(t)$ contains both $r$ and $b$ in its interior.  In either case,
Statement~1 of the lemma is not true because of triangle $t$.

Refer to \figref{stopping} for what follows.  Consider the largest
circle $C_1$ that is concentric with $C$ and that contains no point of
$P$ in its interior (this circle is at least as large as $C$).  The
circle $C_1$ will have at least one point $p_1$ of $P$ on its boundary
(it could be that $p_1=b$, if $b\in P$).  Next, perform a pivot in $P$
along the ray originating at $p_1$ and containing the center of $C_1$.
This will give a circle $C_2$ that contains $C_1$ and with two points
$p_1$ and $p_2$ of $P\cup\{v_1,v_2,v_3\}$ on its boundary and with no
points of $P\cup\{v_1,v_2,v_3\}$ in its interior.  Therefore,
$(p_1,p_2)$ is an edge in the augmented Delaunay triangulation of $P$. 

\begin{figure}
\begin{center}\begin{tabular}{cc}
\includegraphics{stopping-11a-bw} & \includegraphics{stopping-21a-bw} \\[2ex] 
\includegraphics{stopping-11b-bw} & \includegraphics{stopping-21b-bw} \\[2ex]
\includegraphics{stopping-11c-bw} & \includegraphics{stopping-21c-bw} \\[1ex]
(1) & (2)
\end{tabular}\end{center}
\caption{If $P\neq Q$ then Statement~1 of \lemref{stopping} is not
true. 
The left column (1) corresponds to the case where $b\not\in
P$ and the right column (2) corresponds to the case where $b\in
P$.}
\figlabel{stopping}
\end{figure}

The edge $(p_1,p_2)$ partitions the interior of $C_2$ into two pieces,
one that contains $r$ and one that does not.  It is possible to move
the center of $C_2$ along the perpendicular bisector of $(p_1,p_2)$
maintaining $p_1$ and $p_2$ on the boundary of $C_2$.  There are two
directions in which the center of $C_2$ can be moved to accomplish
this.  In one direction, say $\overrightarrow{d}$, the part of the
interior that contains $r$ only increases, so move the center in this
direction until a third point $p_3\in P\cup\{v_1,v_2,v_3\}$ is on the
boundary of $C_2$.  The resulting circle has the points $p_1$, $p_2$,
and $p_3$ on its boundary and no points of $P$ in its interior, so
$p_1$, $p_2$ and $p_3$ are the vertices of a triangle $t$ in the
augmented Delaunay triangulation of $P$.  The circumcircle $C(t)$
contains $r$ in its interior and contains $b$ either in its interior
or on its boundary.  In either case, $t$ contradicts Statement~1, as
promised.  
\end{proof}

Note that the first paragraph in the proof of \lemref{stopping} gives
a method of testing whether $P=Q$, and when this is not the case, of
finding a point in $Q\setminus P$.  For each triangle $t$ in the
Delaunay triangulation of $P$, if $t$ contains a blue vertex $b$ then
perform a pivot in $R$ along the ray originating at $b$ and passing
through $C(t)$.  If the result of this pivot is $C(t)$, then do
nothing.  Otherwise, the pivot finds a circle $C$ with no red points
in its interior and that has one blue point $b$ and one red point
$r\notin P$ on its boundary.  By \lemref{finding}, the point $r$ must
be in $Q$.  If $t$ contains a red vertex, repeat the above procedure
swapping the roles of red and blue.  If both pivots (from the red
point and the blue point) find the circle $C(t)$, then we have
verified Statement~1 of \lemref{stopping} for the triangle $t$.

The above procedure performs at most two pivots for each triangle $t$
in the augmented Delaunay triangulation of $P$.  Therefore, this
procedure performs $O(|P|)=O(\kappa)$ pivots.  Since we repeat this
procedure at most $\kappa$ times before deciding that $\kappa<k$, we
perform $O(\kappa^2)$ pivots, at a total cost of
$O(\kappa^2\times\frac{n}{\kappa^2}\log\kappa)=O(n\log\kappa)$.  The
only other work done by the algorithm is that of recomputing the
augmented Delaunay triangulation of $P$ each time we add a new vertex
to $P$.  Since each such computation takes $O(|P|\log|P|)$ time and
$|P|\le\kappa$, the total amount of work done in computing all these
triangulations is $O(\kappa^2\log\kappa)$.

In summary, we have an algorithm that given $S$ and $\kappa$ decides
whether the condensed set $Q$ of points in $S$ that contribute to the
decision boundary has size at most $\kappa$, and if so, computes $Q$.
This algorithm runs in $O((\kappa^2 + n)\log\kappa)$ time.  By trying
increasingly large values of $\kappa$ as described in \secref{outline}
we obtain our main theorem.

\begin{thm}\thmlabel{main}
The nearest-neighbour decision boundary of a set of $n$ red and blue
points in $\mathbb{R}^2$ can be computed in $O(n\log k)$ time, where
$k$ is the number of points that contribute to the decision boundary.
\end{thm}


\noindent\textbf{Remark:} In the pattern-recognition community pattern
classification rules are often implemented as neural networks.  In the
terminology of neural networks, \thmref{main} states that it is
possible, in $O(n\log k)$ time, to design a simple one-layer neural
network that implements the nearest-neighbour decision rule and uses
only $k$ McCulloch-Pitts neurons (threshold logic units).

\subsection{More than 2 Color Classes}

\thmref{main} extends easily to the case where there are $c>2$ colour
classes and our goal is to find all \voronoi\ edges bounding two
\voronoi\ cells of different colour.  In this case we build a pivot
data structure for each of the $c$ colour classes.  When performing
pivots from a point in some colour class $R$, we perform queries in
all the data structures except the one for colour class $R$.  (This is
equivalent to a pivot in the set $S\setminus R$.)  This increases the
cost of pivot operations to $O(\frac{cn}{m}\log m)$.

The only other modifications to the algorithm are the tuning of a few
parameters. Note that $c$ is a lower bound on $k$ because each colour
class contributes at least 1 point to the decision boundary.
Therefore, we may assume that $c\le \kappa\le n^{1/3}$ since we can
always start our algorithm with $\kappa=2^{2^{\ceil{\log\log c}}}$ and
we can use an $O(n\log n)$ time algorithm when $\kappa>n^{1/3}$.  In
the previous section, we chose $m=\kappa^2$, but could just as easily
have chosen $m=\kappa^3$ without affecting the analysis of the running
time (this is where the assumption that $\kappa\le n^{1/3}$ is used).
With this choice of $m$, the time to perform a pivot becomes \[
O\left(\frac{cn}{m}\log m\right) = O\left(\frac{n}{\kappa^2}\log
m\right) \enspace .  \] Each round of the algorithm performs
$O(\kappa^2)$ queries. Therefore, each round can be implemented in
$O((\kappa^3+n)\log m)=O(n\log\kappa)$ time.  The rest of the analysis
remains unaffected and the entire algorithm runs in $O(n\log k)$ time.

\begin{thm}\thmlabel{multicolour}
The nearest-neighbour decision boundary of a set of $n$ points from
$c$ different colour classes in $\mathbb{R}^2$ can be computed in
$O(n\log k)$ time, where $k$ is the number of points that contribute
to the decision boundary.
\end{thm}

%%%%%%%%%%%%%%%%%%%%%%%%%%%%%%%%%%%%%%%%%%%%%%%%%%%%%%%%%%%%%%%%%%%%%%%%
\section{Conclusions}\seclabel{conclusions}

We have given $O(n\log k)$ time algorithms for computing
nearest-neighbour decisions boundaries of bichromatic point sets in 1
and 2 dimensions, where $k$ is the number of points that contribute to
the decision boundary.  A standard application of Ben-Or's lower-bound
technique \cite{bo83} shows that even the 1-dimensional algorithm is
optimal in the algebraic decision tree model of computation. Simple
variants of these algorithms give solutions for computing the
nearest-neighbour decision boundary of points sets with $c>2$ colours
that run in the same time bounds.

We have not studied algorithms for dimensions $d\ge 3$. In this case,
it is not even clear what the term ``output-sensitive'' means.  Should
$k$ be the number of points that contribute to the decision boundary,
or should $k$ be the complexity of the decision boundary?  In the
first case, $k\le n$ for any dimension $d$, while in the second case,
$k$ could be as large as $\Omega(n^{\lceil d/2\rceil})$.  To the best
of our knowledge, both are open problems.

\section*{Acknowledgements}

This research wash initiated at the McGill Worshop on Instance-Based
Learning at Bellairs Marine Biology Institute, Jan.~21--Feb.~6, 2003.
The authors would like to thank the other workshop participants,
namely, Greg~Aloupis, Jit~Bose, Vida~Dujmovi\'c, Ferran~Hurtado,
Danny~Krizanc, Henk~Meijer, Mark~Overmars, Tom~Shermer,
Sue~Whitesides, and David~Wood for helpful discussions and for
providing a stimulating working environment.  The authors are also
grateful to an anonymous referee for showing us how to remove a $\log
c$ factor from the running time in \thmref{multicolour}.



\bibliographystyle{plain}
\bibliography{boundary}

\end{document}

