\documentclass[11pt]{patmorin}
\usepackage{fullpage}
\usepackage{amsfonts}
\usepackage{amsthm}
\usepackage{graphicx}
\usepackage{algorithmic}
 
%\usepackage{amsthm}

\newcommand{\centeripe}[1]{\begin{center}\Ipe{#1}\end{center}}
\newcommand{\comment}[1]{}

\newcommand{\centerpsfig}[1]{\centerline{\psfig{#1}}}

\newcommand{\seclabel}[1]{\label{sec:#1}}
\newcommand{\Secref}[1]{Section~\ref{sec:#1}}
\newcommand{\secref}[1]{\mbox{Section~\ref{sec:#1}}}

\newcommand{\alglabel}[1]{\label{alg:#1}}
\newcommand{\Algref}[1]{Algorithm~\ref{alg:#1}}
\newcommand{\algref}[1]{\mbox{Algorithm~\ref{alg:#1}}}

\newcommand{\applabel}[1]{\label{app:#1}}
\newcommand{\Appref}[1]{Appendix~\ref{app:#1}}
\newcommand{\appref}[1]{\mbox{Appendix~\ref{app:#1}}}

\newcommand{\tablabel}[1]{\label{tab:#1}}
\newcommand{\Tabref}[1]{Table~\ref{tab:#1}}
\newcommand{\tabref}[1]{Table~\ref{tab:#1}}

\newcommand{\figlabel}[1]{\label{fig:#1}}
\newcommand{\Figref}[1]{Figure~\ref{fig:#1}}
\newcommand{\figref}[1]{\mbox{Figure~\ref{fig:#1}}}

\newcommand{\eqlabel}[1]{\label{eq:#1}}
\newcommand{\eqref}[1]{(\ref{eq:#1})}

\newtheorem{thm}{Theorem}{\bfseries}{\itshape}
\newcommand{\thmlabel}[1]{\label{thm:#1}}
\newcommand{\thmref}[1]{Theorem~\ref{thm:#1}}

\newtheorem{lem}{Lemma}{\bfseries}{\itshape}
\newcommand{\lemlabel}[1]{\label{lem:#1}}
\newcommand{\lemref}[1]{Lemma~\ref{lem:#1}}

\newtheorem{cor}{Corollary}{\bfseries}{\itshape}
\newcommand{\corlabel}[1]{\label{cor:#1}}
\newcommand{\corref}[1]{Corollary~\ref{cor:#1}}

\newtheorem{obs}{Observation}{\bfseries}{\itshape}
\newcommand{\obslabel}[1]{\label{obs:#1}}
\newcommand{\obsref}[1]{Observation~\ref{obs:#1}}

\newtheorem{assumption}{Assumption}{\bfseries}{\rm}
\newenvironment{ass}{\begin{assumption}\rm}{\end{assumption}}
\newcommand{\asslabel}[1]{\label{ass:#1}}
\newcommand{\assref}[1]{Assumption~\ref{ass:#1}}

\newcommand{\proclabel}[1]{\label{alg:#1}}
\newcommand{\procref}[1]{Procedure~\ref{alg:#1}}

\newtheorem{rem}{Remark}
\newtheorem{op}{Open Problem}

\newcommand{\etal}{\emph{et al}}

\newcommand{\voronoi}{Vorono\u\i}
\newcommand{\ceil}[1]{\left\lceil #1 \right\rceil}
\newcommand{\floor}[1]{\left\lfloor #1 \right\rfloor}



\newcommand{\rank}{\mathrm{rank}}
\newcommand{\Z}{\mathbb{Z}}
\newcommand{\N}{\mathbb{N}}
%\newcommand{\defeq}{\stackrel{\mathrm{def}}{=}}
\newcommand{\defeq}{=}

\title{\MakeUppercase{Realizing Partitions Respecting Full 
	and Partial Order Information}}
\author{
	Erik D. Demaine \and
	Jeff Erickson \and
	Danny Kri\c{z}anc \and
	Henk Meijer \and
	Pat Morin \and
	Mark Overmars \and
	Sue Whitesides
}
\date{}

%\DeclareMathOperator{\op}{op}
\newcommand{\op}{\diamond}
\newcommand{\opij}{\op_{ij}}

\pagestyle{empty}
\begin{document}
\maketitle
\thispagestyle{empty}

\begin{abstract} 
For $n\in\N$, we consider the problem of partitioning the interval
$[0,n)$ into $k$ subintervals of positive integer lengths
$\ell_1,\ldots,\ell_k$ such that the lengths satisfy a set of simple
constraints of the form $\ell_i \opij \ell_j$ where $\opij$ is one of
$<$, $>$, or $=$.  In the \emph{full information} case, $\opij$ is
given for all $1\le i,j\le k$.  In the \emph{sequential information}
case, $\opij$ is given for all $1< i< k$ and $j=i\pm 1$.
That is, only the relations between the lengths of consecutive
intervals are specified.  The \emph{cyclic information} case is an
extension of the sequential information case in which the 
relationship $\op_{1k}$ between $\ell_1$ and $\ell_k$ is also given.
We show that all three versions of the problem can be solved in time
polynomial in $k$ and $\log n$.
\end{abstract}

\noindent\textbf{Keywords:} Integer partitions, integer sequences,
subset-sum, rhythm pattern, rhythm perception, modular arithmetic

\section{Introduction}
We consider problems of realizing a sequence having restrictions on
its sum and the relative sizes of its terms.  In particular, we
consider the following problem:  Given positive integers $n$ and $k$,
partition $[0,n)$ into $k$ subintervals of positive integer lengths
$\ell_1,\ldots,\ell_k$ such that the lengths satisfy a set of simple
constraints of the form $\ell_i \opij \ell_j$ where $\opij$ is one of
$<$, $>$, or $=$.  In the \emph{full information} case, $\opij$ is
given for all $1\le i,j\le k$.  In the \emph{sequential information}
information case, $\opij$ is given for all $1\le i\le k$ and $j=i\pm
1$.  The \emph{cyclic information} case is an extension of the
sequential information case in which the relationship $\op_{1k}$
between $\ell_1$ and $\ell_k$ is also given. 

For an example of the full information case observe that, for $n=12$,
the comparison matrix in \figref{example} is satisfied by the
sequences $\langle\ell_1,\ldots,\ell_4\rangle\in
\{\langle1,1,8,2\rangle, \langle1,1,7,3\rangle, \langle1,1,6,4\rangle,
\langle2,2,5,3\rangle\}$.  On the other hand, for $n=6$ no solution is
possible since the smallest natural sequence satisfying the comparison
matrix is $\langle1,1,3,2\rangle$ and $1+1+3+2=7$.

\begin{figure}
\begin{center}
\begin{tabular}{r|cccc}
$i\backslash j$ & 1 & 2 & 3 & 4 \\ \hline
1 & $=$ & $=$ & $<$ & $<$ \\
2 & $=$ & $=$ & $<$ & $<$ \\
3 & $>$ & $>$ & $=$ & $>$ \\
4 & $>$ & $>$ & $<$ & $=$ \\
\end{tabular}
\end{center}
\caption{A comparison matrix for the full information case.}
\figlabel{example}
\end{figure}

The motivation for studying these types of problems comes from the
study of the perception of musical rhythm.  Mathematically, a rhythm
is a partition  of $[0,n)$ into $k$ open intervals called
\emph{off-sets} and $k$ integer points called \emph{on-sets} (see
References~\cite{dfgrt04,t02,t03a,t03b,t04}).  Musically, we interpret
the on-sets as points in time (modulo $n$) when a percussion
instrument is to be struck.  Experimental evidence shows that humans
often do not distinguish between different rhythms with the same
\emph{rhythmic contour}, i.e.  the sequence that specifies whether one
off-set is longer than, shorter than or equal to the previous off-set
(see References~\cite{d78,ftrkp04,kcgv00,l96}).  It then becomes a
natural question to ask whether and how a given rhythmic contour can
be realized.

In this paper, we give polynomial (in $k$ and $\log n$) time
algorithms for all three versions of the problem under study.  In
particular, for the full information case we give an algorithm that
runs in $O(k^2 + \log^cn)$ time, for the sequential information case
we give an algorithm that runs in $O(k^4+\log^c n)$ time, and for the
cyclic information case we give an algorithm that runs in
$O(k^5+\log^c n)$ time.  The exponent $c$ is given by the time it
takes to compute the residue $n\bmod k$ and is not more than $2$.

All versions of this problem reduce to special cases of
$\textsc{Subset-Sum}$ with multiplicity, where there are special
constraints on the allowable multiplicities.  The efficiency and
correctness of our algorithms for solving these problems rely
primarily on properties of modular arithmetic. Throughout this paper,
we use some standard number-theoretical notations: $\Z_k\defeq
\{0,\ldots,k-1\}$, $\N_k\defeq\Z_k\setminus\{0\}$,
$\Z\defeq\Z_\infty$, $\N\defeq\N_\infty$, and $\Z_k^+$ is the group
whose elements are $\Z_k$ and whose operator is addition modulo $k$.

The remainder of the paper is organized as follows.  In
\secref{full-information} we given an algorithm for the full
information case.  In \secref{sequential-information} we given an
algorithm for the sequential information case.  In
\secref{cyclic-information} we give an algorithm for the cyclic
information case. Finally, \secref{conclusions} summarizes our results
and concludes with directions for future research.

%%%%%%%%%%%%%%%%%%%%%%%%%%%%%%%%%%%%%%%%%%%%%%%%%%%%%%%%%%%%%%%%%%%%%%%
\section{Full Information}\seclabel{full-information}

In this section we consider the full information case in which $n$ and
$k$ are given and, for each $1\le i,j \le k$ we are told either that
$\ell_i< \ell_j$, $\ell_i > \ell_j$ or $\ell_i = \ell_j$.  We assume
that this information is given (implicitly or explicitly) in the form
of a comparison matrix $\op$ so that we can determine in constant time
which of the three cases applies to $\ell_i$ and $\ell_j$.  The
algorithm we describe will either find a sequence
$\ell_1,\ldots,\ell_k\in\N$ such that $\sum_{i=1}^k\ell_i = n$ and
$\ell_i\opij\ell_j$ for all $1\le i,j\le k$ or the algorithm will
conclude that no such sequence exists.

We first observe that, because we are given the entire comparison
matrix $\op$, we can run any reasonable sorting algorithm to partition
$1,\ldots,k$ into  $m\le k$ equivalence classes $C_1,\ldots,C_{m}$
where (1)~$\ell_i = \ell_j$ if $i$ and $j$ belong to the same class and
(2)~$\ell_i < \ell_j$ if $i\in C_{i'}$, $j\in C_{j'}$ and $i'< j'$.

Refer to \figref{graphical}.  Now our problem is to find
$v_1,\ldots,v_{m}\in\N$ such that $v_{i'} < v_{i'+1}$, for all $1\le
i' <
m$ and $\sum_{i'=1}^m{v_{i'}|C_{i'}|} = n$.  Then by assigning $l_i=v_{i'}$
for all $i\in C_{i'}$ we obtain a solution to the original problem.
The restriction $v_{i'} < v_{i'+1}$ is slightly inconvenient and we
can remove it with a rewording of the problem.  Let $t_1 = k$,
 and let $t_i=t_{i-1}-|C_{i-1}|$ for $1\le i\le m$.  Then
it suffices to find $w_1,\ldots,w_m\in\N$ such that
\begin{equation}
       \sum_{i=1}^m w_it_i = n \enspace . \eqlabel{wis}
\end{equation}
From $w_1,\ldots, w_m$ we can compute the value of $v_i$ as 
$v_i=\sum_{j=1}^i w_j$.  That is, each value $w_i$
represents the increase from $v_{i-1}$ to $v_i$.

\begin{figure}
\begin{center}\includegraphics{graphical}\end{center}
\caption{The relationship between $v_1,\ldots,v_k$ and
$w_1,\ldots,w_k$. The area under the
curve is $n$.} \figlabel{graphical}
\end{figure}

At this point it is tempting to apply dynamic programming immediately
to solve \eqref{wis} directly.  However, this would lead to an
algorithm with running time $O(k^a n^b)$, for some constants $a$ and
$b$.  In general, this is superpolynomial in the input size since the
input is a $k\times k$ comparison matrix and an integer $n$, all of
which can be encoded in $O(k^2 + \log n)$ bits.  In the following, we
describe a representation that allows us to reduce the dependence
on $n$.

Let 
\[ S_i = \left\{\sum_{j=1}^i w_jt_j : 
        w_1,\ldots,w_i\in \mathbb{N} \right\} 
\enspace .
\]
Our problem is to determine whether $n\in S_m$.
To solve this problem we use dynamic programming to compute each $S_i$ for
$i=1,\ldots,m$.  However, since the set $S_i$ has infinite size, we
require a compact representation for it.  To obtain a nice
representation, we observe that, because $t_1=k$, if $S_i$ contains
$x$ then $S_i$ also contains $x+k$, $x+2k$, $x+3k$, and so on.  Thus,
we can represent $S_i$ by storing, for each $y\in \Z_k$,
the value 
\[
   D_{i}(y) = \min (\{\infty\}\cup\{x: \mbox{$x\equiv y\pmod k$ and
$x\in S_i$} \}) \enspace . 
\]
\begin{lem}\lemlabel{firstalg}
Given $D_{i-1}$, $D_i$ can be computed in $O(k)$ time.
\end{lem}

\begin{proof}
We show that a careful reordering of the elements of $\Z_k$ allows us
to compute $D_i$ by a sequence of $k/r$ lower envelope computations
each taking time $O(r)$; here, $r=k/\gcd(k,\ell_i)$ is the length of
the \emph{orbit} of $\ell_i$ in the group $\mathbb{Z}_k^+$.  The
example of 
the lower envelope in
\figref{envelope} may be useful in what follows.

\begin{figure}
\begin{center}
\includegraphics{envelope}
\end{center}
\caption{A possible lower envelope used for the set $q_{0}$, with $k=20$
and $\ell_i=6$.   The empty circles show values in $D_{i-1}$ and
the filled disks show values in $D_i$.}\figlabel{envelope}
\end{figure}

Define $q_{c,j} = (c + jt_i)\bmod k$.  We will show how to compute
$D_{i}(y)$ for all $y$ in 
\[ 
    q_{0}=\{q_{0,1}, q_{0,2},q_{0,3},\ldots,q_{0,r}\}
\]
in $O(r)$ time.  The same algorithm can be used for the sets $q_{1},
q_{2}, \ldots, q_{k/r-1}$ to give a total running time of
$O(r)\times k/r= O(k)$.
The main observation we use is that
\begin{equation}
    f(j) \defeq D_{i}(q_{0,j}) = \min\{D_{i-1}(q_{0,(j-x)\bmod k})+xt_i :
x\in \N_k\} \enspace .
\end{equation}

That is, the univariate function $f(j)$ is the lower
envelope of $k$ half-lines, where the $x$th half-line is given by the
equation $y=(D_{i-1}(q_{0,x}) + (j-x)t_i)\bmod k$, $j\ge 1$.  Since
the slope, $t_i$, of all $k$ half-lines is identical and positive and
their left endpoints are sorted (by $j$) the lower envelope can easily
be computed in $O(r)$ time by scanning from left to right and keeping
track of the current minimum line.  This completes the proof.
\end{proof}

Note that the algorithm implied by \lemref{firstalg} is actually very
simple, and is given by the following pseudocode

\begin{algorithmic}[1]
\STATE{$r\gets\gcd(k,t_i)$}
\FOR{$c = 0,\ldots,k/r-1$}
  \STATE{$m \gets \infty$}
  \FOR{$j = 1,\ldots,r$}
    \STATE{$m \gets \min\{m, D_{i-1}((c-jt_i)\bmod k)+jt_i\}$}
  \ENDFOR
  \FOR{$j = 0,\ldots, r-1$}
    \STATE{$D_i((c+jt_i)\bmod k)\gets m$}
    \STATE{$m \gets\min\{m, D_{i-1}((c+jt_i)\bmod k) \} + t_i$}
  \ENDFOR
\ENDFOR
\end{algorithmic}


Once we have computed $D_{m}$, we can test if $n$ is in the set $S_m$ by
checking if $D_m(n\bmod k) \le n$.  This completes the proof of our
first result:

\begin{thm}
The realization problem with full information can be solved in
$O(k^2+\log^c n)$ time.
\end{thm}


\comment{
\begin{lem}
Let $t_1,\ldots,t_m$ be natural numbers not exceeding $k$ and suppose
there exists integers $w_1,\ldots,w_m$ such that $\sum_{i=1}^m
w_it_i\equiv x \pmod{k}$.  Then, there exists non-negative integers
$w_1',\ldots,w_m'$ such that $\sum_{i=1}^m
w_i't_i\equiv x \pmod{k}$ and $\sum_{i=1}^m
w_i' \le k/\gcd(k,t_1,\ldots,t_m)$. 
\end{lem}

\begin{proof}
See my notes.
\end{proof}

Unfortunately, this beautiful lemma is useless to us,  because the
$w_i$ might be zero, which is not valid for us.
}

%%%%%%%%%%%%%%%%%%%%%%%%%%%%%%%%%%%%%%%%%%%%%%%%%%%%%%%%%%%%%%%%%%%%%%%
\section{Sequential Information}\seclabel{sequential-information}

Next we consider the realization problem given only sequential
information.  That is, for each $i\in\{1,\ldots,k-1\}$ we are told
only that $\ell_i > \ell_{i+1}$, $\ell_i < \ell_{i+1}$ or
$\ell_i=\ell_{i+1}$.  Our approach is similar to that of the full
information case.  By scanning for $\op_{i,i+1}$ for $i=1,\ldots,k-1$
we determine a set of $m\le k$ equivalences classes $C_1,\ldots,C_m$
over $1,\ldots,k$ such that (1)~$l_i=l_j$ if $i$ and $j$ belong to the
same class and (2)~if $i\in C_{i'}$ and $j\in C_{i'+1}$ then either
$l_i<l_j$ or $l_i > l_j$, as indicated by $\op$. 

Let $t_1=k$, and let $t_i = t_{i-1}-|C_{i-1}|$ for $i>1$.  Let
$s_1=+1$ and, for $i>1$, let $s_i=+1$ if the elements in $C_{i}$
should be greater than the elements in $C_{i-1}$ and let $s_{i}=-1$ if
the elements in $C_{i}$ should be less than the elements in $C_{i-1}$.
Then, our problem is to find $w_1,\ldots,w_m\in\N$ such
that
\begin{equation}
\sum_{j=1}^m w_js_jt_j = n \eqlabel{sum-n}
\end{equation}
and
\begin{equation}
\sum_{j=1}^i w_js_j \ge 1 \mbox{ for all $i\in\{1,\ldots,m\}$.}
\eqlabel{admissible}
\end{equation}
We say that $w_1,\ldots,w_m$ are \emph{admissible} if they satisfy
\eqref{admissible}.

Given $w_1,\ldots,w_m$ satisfying \eqref{sum-n} and
\eqref{admissible}, we can compute the value of $\ell_i\in C_{i'}$ as 
$\ell_i =\sum_{j=1}^{i'} w_js_j$.  That is, the value $w_j$ represents the
difference in the values in $C_{j-1}$ and $C_j$, this difference being
an increase if $s_j=+1$ and a decrease if $s_j=-1$.

As before, because $t_1=k$ and $s_1=+1$, we can implicitly represent
the set 
\[
   S_i = \left\{\sum_{j=1}^i w_js_jt_j : 
	\mbox{$w_1,\ldots,w_i\in\mathbb{N}$ and  
		$w_1,\ldots,w_i$ are admissible} \right\}
\]
by maintaining, for each $y\in \Z_k$ the value 
\[
   D_i(y) = \min\left\{x : \mbox{$x\in S_i$ and $x\equiv y\pmod{k}$} \right\}
    \enspace .
\]
However, unlike the case for full information, the function $D_{i-1}$
is not sufficient for computing the function $D_{i}$.  In particular,
which values of $w_i$ are admissible depends on
$\sum_{j=1}^{i-1}w_js_j$, which can be different for each value of $y$.
Instead, we maintain a two-dimensional table
\[
   D_i(y,h) = \min\left(\{\infty\}\cup \left\{
    \begin{array}{ll}
          & \mbox{there exists admissible $w_1,\ldots,w_i$ s.t.} \\
        \mbox{$x\equiv y\pmod{k}$} : & \mbox{$\sum_{j=1}^i w_js_jt_j = x$ and} \\
          & \mbox{$\sum_{j=1}^i w_js_j = h$} \\
          
    \end{array}
    \right\}\right)
    \enspace .
\]

Next we consider exactly how much information must be stored in order
to maintain the table $D_i$. Since $y\in \mathbb{Z}_k$ we know that
the first dimension ($y$) of the table is of size $k$. The following
lemma shows that the second dimension ($h$) is also not too big.

\begin{lem}\lemlabel{lowheight}
Let $H=k^2+1$.  If $h\ge H$ then $D_i(y,h) - kt_i \ge D_i(y,h-k)$.
\end{lem}

\begin{proof}
Let $w_1,\ldots,w_i$ be any admissible sequence that defines
$D_i(y,h)$.  That is,  $\sum_{j=1}^iw_js_j = h$ and $\sum_{j=1}^i w_js_jt_j
= D_i(y,h)$.  Let $i'\le i$ be the largest index such that
$w_{i'}\ge k+1$ and $s_{i'}=+1$.  The existence of $i'$ is guaranteed
by the pigeonhole principle and the
assumption that $h>H$.  Consider the sequence $w_1',\ldots,w_i'$ where
\[
     w_j' = \left\{ \begin{array}{ll}
            w_j-k & \mbox{if $j=i'$} \\
            w_j & \mbox{otherwise}  \end{array} \right.
\]
Then
\[
  \sum_{j=1}^i w_j's_j = \sum_{j=1}^i w_js_j - k = h-k
\]
and
\[
   \sum_{j=1}^i w_j's_jt_j = \sum_{j=1}^i w_js_jt_j - kt_{i'} 
    \le \sum_{j=1}^i w_js_jt_j - kt_{i} \enspace . 
\]
Thus, $D_i(y,h-k) < D_i(y,h) - kt_i$ provided that $w_1',\ldots,w_i'$
is admissible.  To see that $w_1',\ldots,w_i'$ is admissible we
observe that, if this were not the case, then there must exist some
index $r>i'$ such that $\sum_{j=1}^r w_js_j \le k$.  But then, by the
pigeonhole principle there
must exist some index $i''>r>i'$ such that $w_{i''}>k$ and
$s_{i''}=+1$.  But this is not possible since $i'$ was chosen to be
the largest index with this property.
\end{proof}

\lemref{lowheight} shows that in computing $D_m$ we need only consider
values of $h\le H$.  This is because for any value $x$ that appears as
$x=D_m(y,h)$ for $h>H$, there is a value $z< x$ that appears as
$z=D_m(y,h')$ with $h'\le H$.  Since $D_m(y,h')$ implicitly represents the
set $\{z, z+k, z+2k,\ldots\}$, the value $x$ is represented by
$D_m(y,h')$.

Thus, to obtain our final answer, we need only compute a
table $D_m$ containing $Hk$ entries.  However, a small technicality
occurs because computing $D_m$ from $D_{m-1}$ requires (as we shall
see) looking up table entries of the form $D_{m-1}(y,h)$ where
$H<h<H+k$.  The easiest way to deal with this is to use a table of
size $(H+k)k$ to store $D_{m-1}$.  But then to compute $D_{m-1}$ from
$D_{m-2}$ we require table entries of the form $D_{m-2}(y,h)$ where
$H<h<H+2k$, and so on.  In general, the table $D_i$ will have
$(H+k(m-i))k$ entries so that we can lookup any value $D_i(y,h)$ with
$y\in Z_k$ and $1\le h\le H+(m-i)k$.  Note that this only increases
the sizes of the tables by a constant factor, and the following lemma
shows that we can compute these tables in time proportional to their
size.

\begin{lem}
Given $D_{i-1}$, $D_i$ can be constructed in $O(Hk)$ time.
\end{lem}

\begin{proof}
We first describe the algorithm for the case $s_i=+1$. The algorithm
for the case $s_i=-1$ is similar except for a small modification
described at the end of the proof.

As in the proof of \lemref{firstalg} we reduce the problem to a 
sequence of lower-envelope
computations.   As before, we begin by
splitting the elements of $\Z_k$ into the sets
$q_{0},\ldots,q_{k/r}$ where $r=\gcd(k,t_i)$ and
$q_{c,j}=(c+jt_i)\bmod k$.
Using exactly the same scanning algorithm used in \lemref{firstalg} we
can compute $D_i(q_{0,j},j)$ for all $1\le j\le H+(m-i)k$ in $O(H)$ time.
Again, this is because the univariate function
\[ f(j) = D_i(q_{0,j},j) = \min\{ D_{i-1}(q_{0,j-x},j-x) + xt_i : 1\le
x\le j\} 
\]
is the lower envelope of $H+(m-i)k$ parallel half-lines.
By repeated applications of the above procedure we can compute
$D_i(q_{0,j},j+c)$ for all $1\le j\le H$ and all
$0\le c< r$ in $O(Hr)$ time.  Finally, by repeating that
procedure $k/r$ times we compute entire table $D_i(y,h)$, for all
$y\in Z_k$ and all $1\le h\le H$ in $O(Hk)$ time, as required. 

The case $s_{i}=-1$ is handled in a symmetric manner except that now
the function $f$ is defined as
\[
  f(j) = D_{i}(q_{0,j},j) = \min\{D_{i-1}(q_{0,j+x},j+x) - xt_i :
1\le x\le \infty\} \enspace .
\]
The difficulty with this formulation is that $f(j)$ is the lower
envelope of an infinite number of lines.  However, it follows immediately from \lemref{lowheight} that 
\[
  f(j) = D_{i}(q_{0,j},j) = \min\{D_{i-1}(q_{0,j+x},j+x) - xt_i :
1\le x\le H-j+(m-i+1)k\} \enspace .
\]
Thus, we can compute $D_i$ by taking the lower envelope of
$H+(m-i+1)k$ parallel half-lines.  This completes the proof.
\end{proof}

We have just shown that we can incrementally construct the sets
$S_1,\ldots,S_m$ in $O(Hk)=O(k^3)$ time per set.  This yields our
second theorem:

\begin{thm}
The realization problem with sequential information can be solved in
$O(k^4+\log^c n)$ time.
\end{thm}


%%%%%%%%%%%%%%%%%%%%%%%%%%%%%%%%%%%%%%%%%%%%%%%%%%%%%%%%%%%%%%%%%%%%%%%
\section{The Cyclic Information Case}\seclabel{cyclic-information}

In this section we consider the cyclic version of the realization
problem.  The cyclic version is identical to the sequential version
except that one additional constraint, namely the relationship between
$\ell_1$ and $\ell_k$, is given.  We show that the cyclic version of
the problem can be solved using $O(k)$ applications of the algorithm
for the sequential version of the problem.  

Let $t_1,\ldots,t_m$ and $s_1,\ldots,s_m$ be defined as in the
previous section and suppose that there exists $w_1,\ldots,w_m\in\N$ such
that
\[
   \sum_{j=1}^m w_js_jt_j = n \enspace ,
\]
\[
  \sum_{j=1}^i w_js_j \ge 1 \enspace \mbox{for all $i\in \{1,\ldots,m\}$}
   \enspace ,
\]
\[
  w_1s_1 \le \sum_{j=1}^i w_js_j
     \enspace\mbox{for all $i\in\{ 3,\ldots m-1\}$}
     \enspace ,
\] 
and
\[
  w_1s_1 < \sum_{j=1}^i w_js_j
     \enspace\mbox{for all $i\in\{ 2,m\}$}
     \enspace .
\]
Then, rearranging the above equations we get the equivalent statements
\begin{equation}
\sum_{j=2}^m w_js_jt_j = n - w_1t_1 = n-w_1k  \eqlabel{sum-n-2}
\enspace ,
\end{equation}
\begin{equation}
   \sum_{j=2}^i w_js_j \ge 0
       \enspace\mbox{for all $i\in\{3,\ldots,m-1\}$}
       \enspace  , 
\end{equation}
and
\begin{equation}
   \sum_{j=2}^i w_js_j \ge 1
       \enspace\mbox{for all $i\in\{2,m\}$}
       \enspace .  \eqlabel{admissible-2}
\end{equation}

Note that Equations~(\ref{eq:sum-n-2})--\eqref{admissible-2} are
almost identical to Equations~(\ref{eq:sum-n}) and \eqref{admissible}
and that the existence of $w_2,\ldots,w_m$ satisfying these equations
can be tested in $O(k^4+\log^c n)$ time using the algorithm from the
previous section.  This means that, if there exists a solution to our
cyclic information problem in which the elements of class $C_1$ are
assigned a value not exceeding any value assigned to any other class
$C_i$, $i\neq 1$, then we can find this solution in $O(k^4+\log^c n)$
time.  However, if there exists any solution then at least one of the
classes $C_i$ must be assigned a minimum value in this solution.
Thus, by running the algorithm from the previous section $m$ times we
can determine if there exists any solution.

\begin{thm}
The realization problem with cyclic information can be solved in
$O(k^5+\log^c n)$ time.
\end{thm}


%%%%%%%%%%%%%%%%%%%%%%%%%%%%%%%%%%%%%%%%%%%%%%%%%%%%%%%%%%%%%%%%%%%%%%%
\section{Conclusions}\seclabel{conclusions}

We have considered the problem of partitioning the interval $[0,n)$
into $k$ positive integer length subintervals satisfying some simple
order requirements.  The types of requirements we have considered
include full information, in which the relative length of each pair of
subintervals is given, sequential information, in which only the
relative lengths of consecutive subintervals is given, and cyclic
information in which the relationships between consecutive
subintervals and the first and last subinterval are given.  Our
algorithms run in $O(k^2+\log^c n)$, $O(k^4+\log^c n)$ and
$O(k^5+\log^cn)$ time, respectively. The exponent $c$ is given
by the time it takes to compute the residue $n\bmod k$ and is not more
than $2$.

The most general version of this class of problems is as follows:
Given any subset of the order matrix $\op$, find a sequence
$l_1,\ldots,l_k\in\N$ that respects all relations in this matrix and
whose sum is $n$. This remains an open problem.

Another problem, whose solution would be useful in performing
perceptual tests on rhythms, is that of selecting uniformly at random
from all partitions of $[0,n)$ that satisfy some sequential, cyclic or
total information constraints.  Such an algorithm would be useful for
testing hypotheses of the form:  ``Most rhythms of length $n$ and
having $k$ onsets that satisfy some set of constraints sound alike to
many listeners.''

The sequential and cyclic information problems we study are motivated
by the 3-level ($+-0$) contour representation studied by Dowling
\cite{d78}.  This representation has been generalized to multi-level
contours \cite{l96,kcgv00} where we are given, for each $l_i$, a range
relative to $l_{i-1}$.  For example, we may be told that $l_{i}\in
[l_{i-1} + 50, l_{i-1}+100]$.  The problem is then to find
$l_1,\ldots,l_k$ that satisfy all these constraints and whose sum is
$n$.

\section*{Acknowledgements}

This work was initiated at the Bellairs Winter Workshop on
Computational Geometry for Music Information Retrieval, January 28 to
February 4th, 2005.  The authors are grateful to Godfried Toussaint
for organizing the workshop and presenting the open problems which
lead to the current paper.  The authors are also grateful to the other
workshop participants, namely 
Greg~Aloupis,
David~Bremner,
Justin~Colannino,
Mirela~Damian, 
Vida~Dujmovi\'c,
Francisco~Gomez,
Ferran~Hurtado, 
John~Iacono,
Stefan~Langerman,
Erin~McLeish,
Suneeta~Ramaswami,
David~Rappaport,
Diane~Souvaine,
Ileana~Streinu,
Perouz~Taslakian,
Remco~Veltcamp,
and David~Wood,
for providing a
stimulating working environment.

\bibliographystyle{plain}
\bibliography{realizing}

\end{document}

