\documentclass[lotsofwhite]{patmorin}
\usepackage{html}
\usepackage{pat}
\usepackage{graphicx}
\usepackage{paralist}
\definecolor{linkblue}{named}{Blue}
\hypersetup{colorlinks=true, linkcolor=linkblue,  anchorcolor=linkblue,
citecolor=linkblue, filecolor=linkblue, menucolor=linkblue, pagecolor=linkblue,
urlcolor=linkblue, pdfcreator=Me, pdfproducer=Me} \setlength{\parskip}{1ex}
\usepackage{algorithmic}

\usepackage{titlesec}


\DeclareMathOperator{\spn}{span}
\DeclareMathOperator{\tp}{top}
\DeclareMathOperator{\bttom}{bottom}
\DeclareMathOperator{\tl}{t}
\DeclareMathOperator{\bl}{b}
\DeclareMathOperator{\tr}{tr}
\DeclareMathOperator{\br}{br}

\newcommand{\iters}{2k^2}


\title{\MakeUppercase{Notes on Growing a Spanning Tree}}
\author{Banff Workshop on Random Geometric Graphs}

\begin{document}
\maketitle

\section{Notes}


\subsection{A Ball in Urns Lemma}
\seclabel{balls-in-bins}

For some integer $w\in\N$, consider the following balls in urns process.
We have some number $k$ of urns labelled $1,\ldots,n$, and each urn
$i$ has a weight $w_i\in\{0,\ldots,w\}$.  Define $W=\sum_{i=1}^n w_i$.
We place $\lceil W/w\rceil$ balls into these urns using the following
process:

\noindent
\begin{algorithmic}
  \FOR{$j\in\{1,\ldots,\lceil W/w\rceil\}$}
    \STATE{Select a random urn $i$ using the distribution $\Pr\{i\}=w_i/W$}
    \STATE{Add a ball to urn $i$}
    \STATE{Increase $w_i$ by some value $x_j\in\{-1,0,1,\ldots,w\}$}
   \ENDFOR
\end{algorithmic}
\noindent
Let $b_i$ be the number of balls in urn $i$ at the end of this process.

\begin{lem}\lemlabel{balls-in-bins}
   For any $w_1,\ldots,w_n$ and any $x_1,\ldots,x_{\lceil W/w\rceil}$,
   $\E[\sum_{i=1}^n \binom{b_i+1}{2}] \in O(W/w)$
\end{lem}

\begin{proof}
Easy enough (for Luc).
\end{proof}

\subsection{Random Spanning Trees}

Consider the following process for building an $n$-vertex spanning tree
of an undirected graph $G$ with a special vertex $v_0$.

\noindent
\begin{algorithmic}
  \STATE{$T_1\gets (\{v_0\},\emptyset)$ \COMMENT{$T_1$ has one vertex and no edges}}
  \FOR{$k\in\{1,\ldots,|V(G)|-1$}
    \STATE{Select a uniformly random edge from $\{uv\in E(G):u\in V(T_k)\wedge v\not\in V(T_k)\}$} 
    \STATE{$T_{k+1}\gets (V(T_k)\cup\{v\}, E(T_k)\cup\{uv\})$}
  \ENDFOR
\end{algorithmic}

We aim to study the (average) depth of a node in the tree $T_n$ that
results from this process.  The following theorem shows how this can
be done when the graph $G$ has an isoperimetric inequality.


\begin{thm}
   If $G$ is the $\sqrt{n}\times\sqrt{n}$ toroidal grid, then the 
   expected average height of a node in $T_n$ is $O(\sqrt{n})$.
\end{thm}

\begin{proof}
   We use the isoperimetic inequality for the toroidal grid, which states
   that for any $A\subseteq V(G)$,
  \begin{equation}
        |N_G(A)| \ge f(|A|,n)=\begin{cases}
          1 & \text{if $|A|=n-1$} \\
          \min\{4\sqrt{|A|}, 4(\sqrt{n-|A|}-1)\} & \text{otherwise}
        \end{cases} 
     \eqlabel{isogrid}
  \end{equation}

%  Rather than track the height of $T_k$, we will track the average depth,
%  $D_k$, of the $p(k)$ deepest nodes in $T_k$.  Note that $p(n-1)=1$,
%  so $D_{n-1}$ is the maximum depth of a node in $T_{n-1}$.  The height
%  of $T_n$ is therefore at most $D_{n-1}+1$.

  We break the process of creating $T_n$ into \emph{phases} defined
  as follows: Phase 1 contains $T_1$.  In the general case, if phase
  $j-1$ finishes with the tree $T_k$ which has $|N_G(V(T_k))|=\tau$, then phase $j$ contains trees
  $T_{k+1},\ldots,T_{k+\lceil\tau/3\rceil}$.

  Now, consider what happens during phase $j$: We have a tree $T_k$
  with $k$ vertices. Each of the vertices $i\in V(T_k)$ is incident
  to $w_i\in\{0,1,2,3\}$ edges of $G$ with one endpoint not in $V(T)$.
  The isoperimetric inequality \eqref{isogrid} implies that $W=\sum_{i\in
  V(T)} w_i \ge 3p(k)-1$.  We now draw a parallel between the process of
  growing $T_{k+1},\ldots,T_{k+f(k)}$ and the balls and urns process
  described in \secref{balls-in-bins}.

  Each vertex $i$ of $T_k$ represents an urn.  The weight $w_i$ of vertex
  $i$ is the weight of urn $i$.  Let $v_t$ denote the unique vertex of
  $T_{k+t}$ that is not in $T_{k+t-1}$.  Then the $t$th ball is assigned
  to bin $i$ if there is a path in $T_{k+t}$ to $i$ that does not use
  any vertex of $T_{k}$ except $i$.  Now note that, under these rules,
  this process obeys exactly the rules of the balls in urns process
  described in \secref{balls-in-bins} with $w=3$.

  Now, consider the average depth of a node in $T_{k}$, which is
  \[
     D_k = \left(\frac{1}{k}\right)\sum_{i\in T(k)} d(i)
  \]
  Letting $B_i$ denote the set of vertices assigned to bin $i$,
  we can write the average depth of a node in $T_{k+p(k)}$ as
  \begin{align*}
      D_{k+p(k)} 
         & = \left(\frac{1}{k+p(k)}\right)\sum_{i\in T_k}\left(d(i) + \sum_{v\in B_i} d(v) \right) \\
         & = \left(\frac{1}{k+p(k)}\right)\sum_{i\in T_k}\left(d(i) + b_id(i) + \sum_{v\in B_i} (d(v)-d(i)) \right) \\
         & \le \left(\frac{1}{k+p(k)}\right)\sum_{i\in T_k}\left(d(i) + b_id(i) + \binom{d_i+1}{2}\right)  \\
         & < D_k + \left(\frac{1}{k+p(k)}\right)\sum_{i\in T_k}\left(b_id(i) + \binom{d_i+1}{2}\right)  \\
         & \le D_k + \left(\frac{1}{k+p(k)}\right)\left(p(k)\max\{d(i):i\in T_k\} + \sum_{i\in T_k}\binom{d_i+1}{2}\right)  \\
  \end{align*} 
  where the last inequality is only tight if the vertices in each $B_i$
  form a path.


  Next, consider the difference in average height between $T_{k}$
  and $T_{k+f(k)}$.  The total height increases because of the addition
  of nodes $v_1,\ldots,v_{f(k)}$.  If $b_i$ of these nodes are assigned to vertex $i$ of depth $d(i)$, then these $b_i$ nodes contribute at most
  \[  \sum_{r=1}^{b_i} \left(h+r\right) =  b_i h + \binom{b_{i}+1}{2} \]
  to the total height of $T_{k+f(k)}$.  Thus, the total depth of $T_{k+f(k)}$
  is at most
  \[
      \sum_{i\in V(T_k)} d(i) + f(k)
  \]

\end{proof}


\end{document}
