\documentclass[lotsofwhite]{patmorin}
\usepackage{amsthm}

\newcommand{\CH}{\mathit{CH}}
\newcommand{\depth}{\mathrm{depth}}
\newcommand{\argmax}{\mathrm{argmax}}
\newcommand{\drop}{\!\!\downarrow\!\!}
\newcommand{\dual}{\varphi}

 
%\usepackage{amsthm}

\newcommand{\centeripe}[1]{\begin{center}\Ipe{#1}\end{center}}
\newcommand{\comment}[1]{}

\newcommand{\centerpsfig}[1]{\centerline{\psfig{#1}}}

\newcommand{\seclabel}[1]{\label{sec:#1}}
\newcommand{\Secref}[1]{Section~\ref{sec:#1}}
\newcommand{\secref}[1]{\mbox{Section~\ref{sec:#1}}}

\newcommand{\alglabel}[1]{\label{alg:#1}}
\newcommand{\Algref}[1]{Algorithm~\ref{alg:#1}}
\newcommand{\algref}[1]{\mbox{Algorithm~\ref{alg:#1}}}

\newcommand{\applabel}[1]{\label{app:#1}}
\newcommand{\Appref}[1]{Appendix~\ref{app:#1}}
\newcommand{\appref}[1]{\mbox{Appendix~\ref{app:#1}}}

\newcommand{\tablabel}[1]{\label{tab:#1}}
\newcommand{\Tabref}[1]{Table~\ref{tab:#1}}
\newcommand{\tabref}[1]{Table~\ref{tab:#1}}

\newcommand{\figlabel}[1]{\label{fig:#1}}
\newcommand{\Figref}[1]{Figure~\ref{fig:#1}}
\newcommand{\figref}[1]{\mbox{Figure~\ref{fig:#1}}}

\newcommand{\eqlabel}[1]{\label{eq:#1}}
\newcommand{\eqref}[1]{(\ref{eq:#1})}

\newtheorem{thm}{Theorem}{\bfseries}{\itshape}
\newcommand{\thmlabel}[1]{\label{thm:#1}}
\newcommand{\thmref}[1]{Theorem~\ref{thm:#1}}

\newtheorem{lem}{Lemma}{\bfseries}{\itshape}
\newcommand{\lemlabel}[1]{\label{lem:#1}}
\newcommand{\lemref}[1]{Lemma~\ref{lem:#1}}

\newtheorem{cor}{Corollary}{\bfseries}{\itshape}
\newcommand{\corlabel}[1]{\label{cor:#1}}
\newcommand{\corref}[1]{Corollary~\ref{cor:#1}}

\newtheorem{obs}{Observation}{\bfseries}{\itshape}
\newcommand{\obslabel}[1]{\label{obs:#1}}
\newcommand{\obsref}[1]{Observation~\ref{obs:#1}}

\newtheorem{assumption}{Assumption}{\bfseries}{\rm}
\newenvironment{ass}{\begin{assumption}\rm}{\end{assumption}}
\newcommand{\asslabel}[1]{\label{ass:#1}}
\newcommand{\assref}[1]{Assumption~\ref{ass:#1}}

\newcommand{\proclabel}[1]{\label{alg:#1}}
\newcommand{\procref}[1]{Procedure~\ref{alg:#1}}

\newtheorem{rem}{Remark}
\newtheorem{op}{Open Problem}

\newcommand{\etal}{\emph{et al}}

\newcommand{\voronoi}{Vorono\u\i}
\newcommand{\ceil}[1]{\left\lceil #1 \right\rceil}
\newcommand{\floor}[1]{\left\lfloor #1 \right\rfloor}



%\title{An Optimal Randomized Algorithm for $d$-Variate Zonoid Depth}
\title{\MakeUppercase{An Optimal Randomized Algorithm
	for} $d$\MakeUppercase{-Variate Zonoid Depth}}
\author{Pat Morin \\ 
	School of Computer Science \\
	Carleton University \\
	\texttt{morin@scs.carleton.ca}}
\date{}

\begin{document}
\maketitle
\begin{abstract}
A randomized linear expected-time algorithm for computing the 
zonoid depth of a point with respect to a fixed dimensional point set is
presented.
\end{abstract}

\section{Introduction}

Let $S$ be a set of $n$ points in $\R^d$.
For a real number $k\ge 1$, the \emph{$k$-zonoid} of $S$ is defined as 
\[
      Z_k(S) = \left\{\sum_{p\in S}\lambda_p p
	: \mbox{$0\le \lambda_p \le 1/k$ for all $p\in S$  
	   and $\sum_{p\in S}\lambda_p = 1$}  \right\} \enspace .
\] 
Notice that, for $k=1$ the $1$-zonoid of $S$ is the convex hull of
$S$,  i.e., $Z_1(S)=\CH(S)$.
As $k$ increases, $Z_k(S)$ becomes smaller and smaller until
the limiting case $k=n$, for which $Z_n(S)$ consists of a single point,
the mean of $S$.  The \emph{zonoid depth} of a point
$p\in\CH(S)$ with respect to $S$ is defined as
\[
     \depth(p,S) = \sup\{k : p\in Z_k(S) \} \enspace ,
\]
and, for $p\in\CH(S)$, is a real number in the interval $[1,n]$.

Gopala and Morin \cite{gm06} consider algorithms for bivariate ($d=2$)
zonoid depth and give an $O(n)$ expected time algorithm for computing,
$\depth(p,S)$ when $p$ and $S$ are in $\R^2$.  The current
paper extends their results by giving an $O(n)$ time algorithm to compute
$\depth(p,S)$ when $p$ and $S$ are in $\R^d$ for any constant
dimension $d$. 

In the following, all points, vectors, and hyperplanes are assumed to
live in $\R^d$ and $\Hy^d$ denotes the set of all hyperplanes in
$\R^d$.  The notation $x_i$ denotes the $i$th coordinate of the point
$x$.  We use the $\cdot$ notation to denote the inner-product of two
points/vectors $x\cdot y=\sum_{i=1}^d x_iy_i$.

For a set $S$ of $n$ points and a non-zero vector $r$,
$S_1^r,\ldots,S_n^r$ is the sequence of elements of $S$ ordered by
decreasing projections onto $r$, i.e., $S_i^r\cdot r \ge
S_{i+1}^r\cdot r$ for all $1\le r\le n-1$.

For a point $x$
and a hyperplane $h$, let $x\drop h$ denote the $d$th coordinate of
the vertical projection of $x$ onto $h$.  For a set $H$ of $n$
hyperplanes, let $H_i^x$ be the $i$th hyperplane in $H$ encountered by
a downward vertical ray originating at $(x_1,\ldots,x_{d-1},\infty)$.
For ease of notation we use the shorthand $H_{-i}^x=H_{|H|-i+1}^x$.
For $i>n$ we use the convention that $H_{i}^x$
(respectively $H_{-i}^x$) is the ``horizontal hyperplane at infinity''
$\{x:x_d=-\infty\}$ (respectively,
$\{x:x_d=+\infty\}$). 

\section{Chan's Generalized Optimization Technique}

Chan \cite{c2004} used the following theorem to provide an $O(n\log
n)$ time algorithm for maximum Tukey depth.  In the following, and
throughout the remainder of the paper, we use the shorthand $\cap S$
to denote $\bigcap_{s\in S}s$.

\begin{thm}[Chan 2004]\thmlabel{chan}
Let $\mathcal{H}$ denote the set of all halfspaces in $\R^d$,
let $\mathcal{P}$ denote the set of all possible inputs to some problem, 
let $f:\mathcal{P}\mapsto 2^{\mathcal{H}}$ be any function mapping 
problem inputs to sets of halfspaces in $\R^d$,
let $g:\R^d\mapsto \R$ be
any linear objective function, 
and let $D(n)=\Omega(n^{\epsilon})$, for some $\epsilon>0$, be a non-decreasing function of
$n$.  Suppose that $f$ and $w$ satisfy:
\begin{enumerate}
\setcounter{enumi}{-1}
\item Given inputs $P_1,\ldots,P_d\in\mathcal{P}$ each of constant
size, a point $p\in\cap (f(P_1)\cup\cdots\cup f(P_d))$ maximizing
$g(p)$ can be found in constant time.

\item Given a point $p\in\R^d$ and an input $P\in\mathcal{P}$
of size $n$, there exists a $D(n)$ time algorithm to determine whether
$p\in\cap f(P)$.

\item There exists constants $\alpha < 1$ and $r$ such that, for any
input $P\in\mathcal{P}$ of size $n$, it is possible to compute, in
$D(n)$ time, inputs $P_1,\ldots,P_r$, each of size at most
$\ceil{\alpha n}$, and such that $\cap f(P) =
\cap(f(P_1)\cup\cdots\cup f(P_r))$.

\end{enumerate}
Then there exists an $O(D(n))$ expected time randomized algorithm to
compute, for any input $P\in\mathcal{P}$ of size $n$ a point
$p\in\cap f(P)$ that maximizes $w(p)$.
\end{thm}

It is worth noting that the codomain of the function $f$ may contain
infinite sets.  That is, it is acceptable (and common) to have inputs
$P\in\mathcal{P}$ that generate an infinite number of constraints,
i.e., $|f(P)|=\infty$.


\section{Properties of Primal and Dual Zonoids}

The $k$-zonoid $Z_k(S)$ is a convex polytope.  The extreme-most vertex
of $Z_k(S)$ in direction $x$ can be obtained as a convex combination of the
$\ceil{k}$ extreme-most points of $S$ in direction $x$.  More
precisely,
\begin{equation} 
\argmax\{p\cdot x: p\in Z_k(S)\} =
        \left(\sum_{i=1}^{\floor{k}} \frac{1}{k}S_i^x\right) +
          (1-\floor{k}/k)S_{\ceil{k}}^x  
          \eqlabel{extreme}
\end{equation}
\cite{beXX,gmXX}.  Intuitively, we assign the maximum allowable
coefficient ($1/k$) to each of the $\floor{k}$ extreme-most vertices
and the ``leftover'' is assigned to the next vertex.

We wish to arrive at a situation in which we can apply \thmref{chan}
and this is best done by working in the dual.
Consider the point-hyperplane duality function
$\dual$ given by 
\[
    \dual(x)=\{y\in\R^d : y_d = x_1y_1 +\cdots +x_{d-1}y_{d-1} - x_d \}
\] 
when $x$ is a point in $\R^d$ and
\[
     \dual(X) = \{\dual(x) : x\in X\}
\]
when $S$ is a subset of $\R^d$.  See Edelsbrunner's book \cite{eXX}
for properties of this duality. 

Let $H=\dual(S)$.  Then,
under this duality, the \emph{dual $k$-zonoid} $\dual(Z_k(S))$ is the set 
of all hyperplanes in $\R^d$
that do not intersect either of two convex sets $A_k(S)$ and $B_k(S)$.
That is,
\[
     \dual(Z_k(S)) = \{ h\in\Hy^d : h\cap(A_k(S)\cup B_k(S)) = \emptyset \} \enspace ,
\]
where
\begin{equation}
   A_k(S) = \left\{x\in\R^d: x_d \ge 
\left(\sum_{i=1}^{\floor{k}} \frac{1}{k}(x\drop H_i^x)\right) +
          (1-\floor{k}/k)(x\drop H_{\ceil{k}}^x) \right\}  \eqlabel{ra}
\end{equation} 
and
\begin{equation}
   B_k(S) = \left\{x\in\R^d: x_d \le 
\left(\sum_{i=1}^{\floor{k}} \frac{1}{k}(x\drop H_{-i}^x)\right) +
          (1-\floor{k}/k)(x\drop H_{-\ceil{k}}^x) \right\} \enspace .
\end{equation}
The definitions of $A_k(S)$ and $B_k(S)$ follow from \eqref{extreme}
and the duality $\dual(\cdot)$.
The two sets $A_k(S)$ and $B_k(S)$ are convex, unbounded from above,
respectively, below, and piecewise linear.  Indeed, the linear pieces
of $A_k(S)$ (respectively $B_k(S)$) are in one to one correspondence
with the linear pieces of the $\ceil{k}$-level (respectively
the $(n-\floor{k}+1)$-level) of the hyperplanes in $H$.  Thus,
$A_k(S)$ and $B_k(s)$ are convex polytopes that are implicitly defined
by the hyperplanes in $H$ and it is these implicit ``linear programs''
that will ultimately allow us to apply \thmref{chan}.


\section{The Decision Algorithm}

Next we consider the following decision problem:  Given a point set
$S$ and an integer $k$, is the origin contained in $Z_k(S)$?  By
translation, a solution to this problem allows us to test if an
arbitrary point $p\in\R^d$ is contained in $Z_k(S)$. One approach to
solving this problem is to compute the intersection of $Z_k(S)$ with
the vertical line $\{x\in\R^d:x_0=x_1=\cdots=x_{d-1}=0\}$ through the origin
and then check if this intersection contains the origin \cite{gm04}. 

Under the duality $\dual$, the above strategy is equivalent to finding
the lowest point on $A_k(S)$ and the highest point on $B_k(S)$ and
checking that each of these points is above, respectively, below, the
hyperplane $\{x\in\R^d:x_d=0\}$.  In the remainder, we focus on
determining the lowest point in $A_k(S)$.  Finding the highest point
in $B_k(S)$ is done in a symmetric manner.  However, before we can
proceed, we need to define a slightly more general problem involving
weights.

Let $S$ be a set of $n$ points in $\R^d$ and let $w:S\mapsto\N$ be a
function assigning positive integer weights to the elements of $S$.
We denote by $S^w$, the multiset in which each element $p\in S$ occurs
$w(p)$ times.  The \emph{$w$-weighted zonoid} $Z_k(S,w)$ is simply the
$k$-zonoid of the multiset $S^w$, i.e., $Z_k(S,w)=Z_k(S^w)$.  As with
standard zonoids, the weighted zonoid $Z_k(S,w)$ dualizes to the set
of all lines that do not intersect either of two convex regions
$A_k(S,w)$ and $B_k(S,w)$, where $A_k(S,w)=A_k(S^w)$ and
$B_k(S,w)=B_k(S^w)$.

This definition of weighted zonoids allows us to naturally define
subproblems.  For a subset $C\subseteq S$, define the \emph{total
weight}
\[
       w(C)=\sum_{p\in C}w(p)
\]
and the \emph{weighted mean}
\[ 
       \mu(C)=\frac{1}{w(C)}\sum_{p\in C} p\times w(p) \enspace .
\]
The \emph{contraction} of
$(S,w)$ by $C$ is obtained by replacing the points of $C$ by their
weighted average, $\mu(C)$.  More precisely, the contraction of
$(S,w)$ by $C$ is the pair $(R,v)$ where 
\[ R = (S\setminus C) \cup \{ \mu(C) \} \] 
and 
\[ v(p) = \left\{\begin{array}{ll} 
        w(p) & \mbox{if $p\in S\setminus C$} \\ 
        w(C) & \mbox{if $p=\mu(C)$} \end{array}\right.
\]

The following lemma shows that contraction results in strictly smaller
zonoids:

\begin{lem}\lemlabel{contraction}
If $(R,v)$ is a contraction of $(S,w)$ by $C$ then $Z_k(R,v)
\subseteq Z_k(S,w)$.
\end{lem}

\begin{proof}
Let $x$ be any point in $Z_k(R,v)$.  Then, by the definition of
zonoids:
\begin{eqnarray*}
    x &=& \sum_{p\in R^v} \lambda_pp \\
      &=& \left( \sum_{p\in (R\setminus \{\mu(C)\})^v} \lambda_pp \right)
          + \left( \sum_{p\in \{\mu(C)\}^v}\lambda_{\mu(C)}p \right) \\
      &=& \left(\sum_{p\in (S\setminus C)^w} \lambda_pp \right)
          + \left(\sum_{p\in C^w}\lambda_{\mu(C)} p\right) \\
      &\in& Z_k(S,w)
\end{eqnarray*}
as required.
\end{proof}

We now have all the tools required to apply \thmref{chan} to solve our
decision problem.

\begin{thm}\thmlabel{decision}
Given a set $S$ of $n$ points in $\R^d$ and a 
function $w:S\mapsto\N$ that is computable in constant
time, the point $x\in A_k(S,w)$ such
that $x_d$ is minimum can be found in $O(n)$ expected
time.
\end{thm}

\begin{proof}
Let $f$ be the function that maps the pair $(S,w)$ onto a set of
halfspaces whose intersection is $A_k(S,w)$ and let the objective
function $g(p)=p_d$.  We need to show that the function $f$ satisfies
the Conditions~0--2 of \thmref{chan}.

To satisfy Condition 0 of \thmref{chan} we can enumerate all the
linear constraints generated by each of the $d$ subproblems and use
any linear programming algorithm to find a point $x$ that satisfies
all constraints and such that $x_d$ is minimum.  There are only $d$
subproblems, each of constant size, so this step takes constant time,
as required.

To satisfy Condition 1 of \thmref{chan}
we observe that testing if $x\in A_k(S,w)$
simply involves checking if $x$ satisfies \eqref{ra}.  Let
$H=\dual(S)$.  This check can be accomplished by using a $D(n)=O(n)$
time weighted selection algorithm \cite{X} to compute the smallest
index $t$ and the hyperplanes $H_{1}^x,\ldots,H_{t}^x$ such that
$\sum_{i=1}^tw(\dual(H_{i}^x)) \ge k$.  Once this is done we need only
check \eqref{ra} which, in the weighted setting, becomes 
\[
     x \ge \left(\sum_{i=1}^{t-1} \frac{1}{k}(x\drop
H_i^x)\times w(\dual(H_i^x))\right) 
   + \frac{1}{k}(x\drop H_t^x) \times \left(k-\sum_{i=1}^{t-1} v(\dual(H_{i}^x)) \right)
\enspace .
\]

To satisfy Condition~2 of \thmref{chan}
we make use of \emph{cuttings} \cite{X}.  In
particular, we use the fact that, in $O(n)$ time, it is possible to
partition $\R^d$ into $r=O(1)$ simplices
$\Delta_1,\ldots,\Delta_r$ such that the interior of each simplex is
intersected by at most $n/2$ of the hyperplanes in $\dual(S)$.  For
each simplex $\Delta_i$ we create a subproblem $(S_i,w_i)$ as follows:
Let $C_i\subseteq S$ contain every point $p\in S$ such that $\dual(p)$
is above the interior of $\Delta_i$.  We first construct the pair
$(T_i,w_i)$ by contracting $(S,w)$ by $C_i$.  Next, we obtain $S_i$ by
removing from $T_i$ every point $p$ such that $\dual(p)$ is below the
interior of $\Delta_i$.  The subproblems $(S_i,v_i)$ for $1\le i\le r$
that we obtain in this manner are each of size at most $n/2+2$.

It follows from \lemref{contraction} (the contraction step) 
and the definition of $Z_k(S,w)$ (the deletion step)
that $Z_k(S_i,{w_i})\subseteq Z_k(S,w)$.  In the dual, this means that
$A_k(S_i,{w_i})\supseteq A_k(S,w)$.  To satisfy Condition~2 of
\thmref{chan} we must
show that $\bigcap_{i=1}^r A_k(S_i^{w_i}) = A_k(S^w)$.  To do this,
consider any point $x$ on the boundary of $A_k(S,w)$.  It is
sufficient to show that $x$ is also on the boundary of at least one
region $A_k(S_i,{w_i})$ for $1\le i\le r$.  The point $x$ is defined
by $\ceil{k}$ hyperplanes $h_1,\ldots,h_{\ceil{k}}\in\dual(S^w)$ in
the sense that 
\[
   x_d=\left(\sum_{i=1}^{\floor{k}} \frac{1}{k}(x\drop h_i)\right) 
       + (1-\floor{k}/k)(x\drop h_{\ceil{k}}) \enspace .
\]
Let $q=x\drop h_{\ceil{k}}$.  There
is some simplex $\Delta_i$ that contains $q$.  Observe that
each of $h_1,\ldots,h_{\ceil{k}-1}$ is either completely above
the interior of $\Delta_i$ or intersects $\Delta_i$.  Furthermore, 
any hyperplane in
$\dual(S)$ that is completely above $\Delta_i$ is one of
$h_1,\ldots,h_{\ceil{k}-1}$.  Therefore, the subproblem $(S_i,w_i)$
is obtained from $(S,w)$ by contracting
$C_i\subseteq\{\dual(h_1),\ldots,\dual(h_{\ceil{k}-1})\}$ and then deleting 
some subset of
$S\setminus\{\dual(h_1),\ldots,\dual(h_{\ceil{k}-1})\}$.  Let
$I=\dual(S_i^{w_i})$.  Then, every point $x$ in $A_k(S_i,w_i)$ must
satisfy
\begin{eqnarray*}
  x_d & \ge & \left(\sum_{i=1}^{\floor{k}} \frac{1}{k}(x\drop I_i^x)\right) 
       + (1-\floor{k}/k)(x\drop I_{\ceil{k}}^x) \\
   & = & \left(\sum_{i=1}^{\floor{k}} \frac{1}{k}(x\drop h_i)\right) 
       + (1-\floor{k}/k)(x\drop h_{\ceil{k}}) \enspace .
\end{eqnarray*}
Thus $x$ is on the boundary of $A_k(S_i,w_i)$, as required.  We have
now satisfied all three conditions necessary to apply \thmref{chan},
completing the proof.
\end{proof}

\section{The Optimization Algorithm}

In the previous section we showed, given $p$, $S$ and $k$, how to
answer the question:  Is $p\in Z_k(S)$?  In this section we consider
the optimization problem, given $p$ and $S$: What is the maximum value
of $k$ such that $p\in Z_k(S)$?  For this problem, we can apply
\thmref{chan} again, this time on a problem in $\R^{d+1}$.

Consider the point set
\[
    Z(S)=\{p\in\R^{d+1} : (p_1,\ldots,p_d)\in Z_{p_{d+1}}(S)  \} \enspace .
\]
As shown by Bern and Eppstein \cite{beXX} the set $Z(S)$ is a zonotope
(the Minkowsky sum of line segments) and is therefore a convex
polytope.  Dualizing $Z(S)$ as before gives two regions
$A(S)$ and $B(S)$.
We are interested in the hyperplane $h=\{x\in\R^{d+1}:x_d=0\}$.
In particular, the value $k$ we are searching for is the minimum of
$k_A$ and $k_B$ where
\[
   k_A = \min\{p_{d+1}: p\in h\cap A(S)\}
\]
and
\[
   k_B = \min\{p_{d+1}: p\in h\cap B(S)\}
\]
In words, we want the minimum value of $k$ such that the $A_k(S)$
(respectively, $B_k(S)$)
intersects the hyperplane $\{x\in\R^d: x_d=0\}$.

\begin{thm}
Given a set $S$ of $n$ points in $\R^d$ and a point $p\in\R^d$, the
maximum value $k$ such that $p\in Z_k(S)$ can be found in $O(n)$
expected time.
\end{thm}

\begin{proof}[Proof Sketch]
The proof is another application of \thmref{chan} to find the values
$k_A$ and $k_B$ described above.  We focus only on finding $k_A$, as
finding $k_B$ is a symmetric problem.  The details are much the same
as in \thmref{decision} so we only sketch them.  As before we
generalize $A(S)$ and $B(S)$ to the weighted setting using multisets
and let $f(S,w)$ be the function that maps $(S,w)$ on to the set of
linear constraints that define $h\cap A(S^w)$.

As before, $S$ satisfies Condition~0 of \thmref{chan} since, for
constant size subproblems we can explictly generate the polytopes
$Z(S_1,w_1),\ldots,Z(S_{d+1},w_{d+1})$ and find a point in the
intersection maximizing the objective function $g(p)=p_{d+1}$. 

The decision problem we must solve to satisfy Condition~1 of
\thmref{chan} is the problem of testing whether a point $p\in
\R^{d+1}$ is contained in $A(S)\cap h$.  But this is simply the
question of whether $(p_1,\ldots,p_d)$ is $A_{p_{d+1}}(S)$, a problem
for which we described an $O(n)$ time algorithm in the proof of
\thmref{decision}.

The partitioning into subproblems required to satisfy Condition~2 of
\thmref{chan} can be done in exactly the same manner as described in
the proof of \thmref{decision}.  To see that this partitioning works
in the current case we need only observe that the paritioning makes no
use of the value $k$ and the argument used to show its correctness
holds for all values of $k$.  This completes the proof.
\end{proof}

\end{document}

