\documentclass[lotsofwhite]{patmorin}
\usepackage{amsmath}
\usepackage{url}

\title{\MakeUppercase{On the False-Positive Rate of Bloom Filters}%
	\thanks{School of Computer Science,
		Carleton University,
		\texttt{\{jit,hguo2,kranakis,maheshwa,morin,morrison,michiel,y\_tang\}@scs.carleton.ca}}}

\author{Prosenjit Bose 
	\and Hua Guo 
	\and Evangelos Kranakis 
	\and Anil Maheshwari 
	\and Pat Morin
	\and Jason Morrison
	\and Michiel Smid 
	\and Yihui Tang
}

\date{May 30, 2004}

 
%\usepackage{amsthm}

\newcommand{\centeripe}[1]{\begin{center}\Ipe{#1}\end{center}}
\newcommand{\comment}[1]{}

\newcommand{\centerpsfig}[1]{\centerline{\psfig{#1}}}

\newcommand{\seclabel}[1]{\label{sec:#1}}
\newcommand{\Secref}[1]{Section~\ref{sec:#1}}
\newcommand{\secref}[1]{\mbox{Section~\ref{sec:#1}}}

\newcommand{\alglabel}[1]{\label{alg:#1}}
\newcommand{\Algref}[1]{Algorithm~\ref{alg:#1}}
\newcommand{\algref}[1]{\mbox{Algorithm~\ref{alg:#1}}}

\newcommand{\applabel}[1]{\label{app:#1}}
\newcommand{\Appref}[1]{Appendix~\ref{app:#1}}
\newcommand{\appref}[1]{\mbox{Appendix~\ref{app:#1}}}

\newcommand{\tablabel}[1]{\label{tab:#1}}
\newcommand{\Tabref}[1]{Table~\ref{tab:#1}}
\newcommand{\tabref}[1]{Table~\ref{tab:#1}}

\newcommand{\figlabel}[1]{\label{fig:#1}}
\newcommand{\Figref}[1]{Figure~\ref{fig:#1}}
\newcommand{\figref}[1]{\mbox{Figure~\ref{fig:#1}}}

\newcommand{\eqlabel}[1]{\label{eq:#1}}
\newcommand{\eqref}[1]{(\ref{eq:#1})}

\newtheorem{thm}{Theorem}{\bfseries}{\itshape}
\newcommand{\thmlabel}[1]{\label{thm:#1}}
\newcommand{\thmref}[1]{Theorem~\ref{thm:#1}}

\newtheorem{lem}{Lemma}{\bfseries}{\itshape}
\newcommand{\lemlabel}[1]{\label{lem:#1}}
\newcommand{\lemref}[1]{Lemma~\ref{lem:#1}}

\newtheorem{cor}{Corollary}{\bfseries}{\itshape}
\newcommand{\corlabel}[1]{\label{cor:#1}}
\newcommand{\corref}[1]{Corollary~\ref{cor:#1}}

\newtheorem{obs}{Observation}{\bfseries}{\itshape}
\newcommand{\obslabel}[1]{\label{obs:#1}}
\newcommand{\obsref}[1]{Observation~\ref{obs:#1}}

\newtheorem{assumption}{Assumption}{\bfseries}{\rm}
\newenvironment{ass}{\begin{assumption}\rm}{\end{assumption}}
\newcommand{\asslabel}[1]{\label{ass:#1}}
\newcommand{\assref}[1]{Assumption~\ref{ass:#1}}

\newcommand{\proclabel}[1]{\label{alg:#1}}
\newcommand{\procref}[1]{Procedure~\ref{alg:#1}}

\newtheorem{rem}{Remark}
\newtheorem{op}{Open Problem}

\newcommand{\etal}{\emph{et al}}

\newcommand{\voronoi}{Vorono\u\i}
\newcommand{\ceil}[1]{\left\lceil #1 \right\rceil}
\newcommand{\floor}[1]{\left\lfloor #1 \right\rfloor}



\newcommand{\stirling}[2]{\genfrac{\{}{\}}{0pt}{}{#1}{#2}}
\newcommand{\blah}{\frac{k}{p}\sqrt{\frac{\ln m-2k\ln p}{m}}}

\begin{document}
\maketitle
%\begin{center}
%\end{center}

\begin{abstract}
Bloom filters are a randomized data structure for membership queries
dating back to 1970.  Bloom filters sometimes give erroneous answers
to queries, called \emph{false positives}.  Bloom analyzed the probability of
such erroneous answers, called the \emph{false-positive rate}, and
Bloom's analysis has appeared in many publications throughout the
years.  We show that Bloom's analysis is incorrect and give a correct
analysis.
\end{abstract}

\paragraph{Keywords:} Data structures, analysis of algorithms

\section{Introduction}

A Bloom filter \cite{b70} is an extremely simple randomized data
structure for testing membership in a set and has found applications
in many areas including (sequential, distributed and parallel)
databases \cite{g82,m83,r82}, computer networks \cite{bm02}, social
networks \cite{loaf}, and cryptography \cite{bc04,g03}. A Bloom filter
represents an $n$ element set $S$ using a bit-vector
$B=B_1,\ldots,B_{m}$ of length $m$.  Initially all the bits of $B$
are set to 0.  It is assumed that each element $x$ to be stored or
searched for comes with a sequence of $k$ random\footnote{Here, as
usual, the term ``random'' means chosen uniformly at random and
independently of any other ``random'' choices.} hash values
$x_1,\ldots,x_k\in \{1,\ldots,m\}$.  To store the element $x$ one
simply sets the bits $B_{x_1}=B_{x_2}=\cdots=B_{x_k}=1$.  To query a
Bloom filter for an element $y$ one simply checks if
$B_{y_1},\ldots,B_{y_k}$ are all set to 1.  If so, the filter outputs
``yes'' otherwise it outputs ``no.''

If $y$ is stored in the filter then, by definition,
$B_{y_1},\ldots,B_{y_k}$ are all set to 1, so the query algorithm
correctly outputs yes.  However, the converse is not true.  It is
possible that $y$ is not stored in the filter but (by coincidence)
$B_{y_1},\ldots,B_{y_k}$ are all set to 1.  This situation is called a
\emph{false positive} and the probability that this occurs is called
the \emph{false-positive rate}.  Bloom derives the false-positive rate
in the following way: The probability that any particular bit $B_i$ is
equal to 0 is $(1-1/m)^{kn}$, since the value $i$ must be avoided by
all $kn$ hash values.   Therefore, the probability that a particular
bit is set to 1 is 
\[
	p\stackrel{\mathrm{def}}{=}\Pr\{B_i=1\}=1-(1-1/m)^{kn} \enspace .
\]
Now, in order for $y$ to result in a false positive, each of the $k$
hash values $y_1,\ldots,y_k$ must be the index of a bit that is set to
1.  The probability that this happens is 
\[
	p_{k,n,m} \stackrel{\mathrm{def}}{=} 
	\Pr\{\mbox{$B_{y_1}=1$ and $B_{y_2}=1$ and \ldots and $B_{y_k}=1$}\}
\]
which is claimed to be
\begin{equation}         
	p^k = (1-(1-1/m)^{kn})^k \enspace .  \eqlabel{mistake}
\end{equation}
This proof, which has appeared in many papers
throughout the years, is not quite correct. The error occurs in deriving
\eqref{eq:mistake}, where there is an implicit assumption that the
event ``$B_{y_i}=1$'' and the event ``$B_{y_1}=B_{y_2}=\cdots
=B_{y_{i-1}}=1$'' are independent.  At first glance, this seems to be
true, since $y_1,\ldots,y_i$ are independent.  However, a simple
counterexample to this proof can be obtained by considering the case
$n=1$, $k=2$, $m=2$.  In this case, by simply enumerating the 16
possible situations\footnote{There are two elements involved, the
element $x$ stored in the table and some element $y$ not stored in the
table.  Whether or not $y$ is a false positive depends only on $x_1$,
$x_2$, $y_1$ and $y_2$.} one finds that the false-positive rate is
$5/8$, whereas $p^k$ evaluates to $9/16=4.5/8$.  Thus,
Bloom's bound underestimates the false-positive rate in this case.

Mitzenmacher \cite{m02} gives a concentration result on the number of
1 bits in Bloom filter that somewhat justifies the standard analysis
of Bloom filters for certain common choices of parameters.  In this
paper, we perform a more detailed analysis of the false-positive rate
of Bloom filters.  The analysis of the false-positive rate is not
nearly as straightforward as one would hope. The contributions of this
paper are as follows: \begin{enumerate}

\item We give an exact formula for the false-positive rate of Bloom filters.
Unfortunately, this formula is not anywhere near closed form, but
could be useful for small values of $k$, $n$ and $m$.

\item We give an upper bound on the false-positive rate that converges to
 $p^k$ for most common choices of the parameters $k$ and $m$.

\item We show that, rather than being an upper-bound,
$p^k$ is actually a strict lower bound on the
false-positive rate for any $k\ge 2$.  That is, for any choice of $k$,
$m$, and $n$ with $k\ge 2$, the false-positive rate of the resulting
Bloom filter is greater than $p^k$
\end{enumerate}

To the best of our knowledge, this is the first paper to point out
this error in the analysis of Bloom filters and give a corrected
analysis.  The only similar result we know of is a paper
by Carter \etal\ \cite{cfgmw78} in which they define a data structure
(Approximate Membership Tester 1) that, under their assumptions, is
equivalent to a Bloom filter.  They show that the expected number of
bits set to 1 in the filter is $mp$ so the probability of
a false positive ``is at most about'' $p^k$.  Their use
of the qualifier ``about'' indicates that they realize this is not the
exact probability, but they do not elaborate any further.

The remainder of this paper is organized as follows:  In
\secref{exact} we derive an exact formula for the false-positive rate.  In
\secref{asymptotic} we give tight upper and lower bounds on the
false-positive rate.  In \secref{conclusions} we summarize and
conclude.

\section{An Exact Formula}\seclabel{exact}

We model the problem of determining the false-positive rate as a
problem on balls and urns.  We are given $m$ urns. We throw $kn$
white balls at random into these urns.  We call an urn \emph{white}
if it contains at least one white ball.  Next we throw $k$ black balls
in the urns.  Let $A$ be the event that each black ball is in a white
urn.  We want to evaluate $\Pr\{A\}$.  To see that this correctly
models Bloom filters, treat the urns as the bits $B_1,\ldots,B_m$ and
use the convention that $B_i=1$ if and only if urn $i$ is white.
Thus, the event $A$ corresponds to a false positive ($k$ randomly
chosen bits are all set to 1).  Thus, the false-positive rate
$p_{k,n,m}$ is equal to $\Pr\{A\}$.

Observe that the set of white urns can be represented as a subset of
$\{ 1,\ldots,m \}$.  For any $I \subseteq \{ 1,\ldots,m \}$, let
$E_I$ be the event that $I$ is the set of white urns. Observe that $1
\leq |I| \leq m$. Using conditional probabilities, we get 
\[ \Pr\{A\} = \sum_{I \subseteq \{ 1,\ldots,m \}} 
            \Pr\{ A \mid E_I \} \cdot \Pr\{ E_I \}. 
\] 
If $I$ is fixed then 
\[ \Pr\{ A \mid E_I \} = \left(\frac{|I|}{m}\right)^k ,
\] 
whereas $\Pr\{ E_I \}$ is the quotient of 
\begin{itemize} 
\item the number of surjections from a set of size $kn$ onto a set of
size $i$, and
\item the number of functions from a set of size $kn$ to a set of size
$m$.
\end{itemize} 


The number of surjections from a set of size $kn$ onto a set of size
$i$ is given by $i!\stirling{kn}{i}$ where 
\[ 
 \stirling{kn}{i} = 
     \frac{1}{i!}\sum_{j=0}^i (-1)^j \binom{i}{j} j^{kn} 
\] is called a
\emph{Stirling number of the second kind} \cite[Section~6.1]{gkp94}.
 
The number of functions from a set of size $kn$ to a set of size $m$
is equal to $m^{kn}$. 

Putting everything together, we obtain 
\begin{eqnarray*} 
 \Pr\{ A \} & = & \sum_{I \subseteq \{ 1,\ldots,m \}}
              \left(\frac{|I|}{m}\right)^k \times
              \frac{|I|!\stirling{kn}{|I|}}{m^{kn}} \\
  & = & \frac{1}{m^{k(n+1)}}\sum_{i=1}^{m} 
           i^ki!\binom{m}{i}
          \stirling{kn}{i} \enspace .
\end{eqnarray*} 

\begin{thm}\thmlabel{exact}
Let $p_{k,n,m}$ be the false-positive rate for a Bloom filter that stores $n$
elements in a bit-vector of size $m$ using $k$ hash functions.  Then,
\[
  p_{k,n,m} = \frac{1}{m^{k(n+1)}}\sum_{i=1}^{m} 
           i^ki!\binom{m}{i}
          \stirling{kn}{i} \enspace .
\]
\end{thm}



\section{Asymptotic Bounds}\seclabel{asymptotic}

Unfortunately, the formula for $p_{k,n,m}$ given by \thmref{exact} is
not very enlightening.  In particular, it is not easy to compare it
directly with $p^k$, the value derived by Bloom.  In this section, we
use probability theory to study the asymptotics of $p_{k,n,m}$, and
give closed-form upper and lower bounds.  We make use of the following
result on balls and urns due to Kamath \etal\ \cite{kmps94} (see also
Motwani and Raghavan~\cite[Theorem~4.18]{mr95}):

\begin{thm}[Kamath, Motwani, Palem, Spirakis 1994]\thmlabel{ballsurns}
Let $W$ denote the number of white urns after throwing $kn$ white balls into
$m$ urns. Then
\[
    E[W] = m\left(1-\left(1-\frac{1}{m}\right)^{kn}\right)
\]
and for $\lambda > 0$
\[
    \Pr\{|W - E[W]| \ge \lambda\} \le 2\exp\left(
        -\frac{\lambda^2(m-1/2)}{m^2-E[W]^2}\right) \le
2\exp(-\lambda^2/(2m)) \enspace .
\]
\end{thm}

Again, let $A$ be the event ``every black ball is contained in a white
urn.'' We want to compute upper and lower bounds on $\Pr\{A\}$.  

\subsection{The Upper Bound}

In this section we give an upper-bound on $p_{k,n,m}$. However, it is
awkward (and not very useful) to give an upper bound that holds for
all possible choices of $k$, $n$ and $m$.  Our upper bound requires
the condition that
\begin{equation}
	\blah \le c  \eqlabel{blah}
\end{equation}
for some constant $c < 1$.  The reasons for this will become apparent
in the analysis.  To see that this assumption is justified, note that,
in nearly all applications of Bloom filters, the parameter $k$ is
chosen (as a function of $m$ and $n$) so that $p=1-(1-1/m)^{kn}$ is a
constant, usually close to $1/2$. Under these conditions, $k = \Theta(m/n)$
and \eqref{eq:blah} becomes
\[
        \blah = O\left(\frac{m}{n}\sqrt{\frac{m/n+\ln m}{m}}\right) < c
\enspace .
\]
For sufficiently large values of $m$, this is satisfied as long as $m
= o(n^{3/2})$.  Again, this is true in all applications of Bloom
filters since, if we are willing to use $m=\Theta(n\log n)$ bits of
storage, hash tables are a better alternative since they offer
constant time searches with no false positives.


We obtain the upper bound by conditioning on the value of $W$ which,
according to \thmref{ballsurns} is strongly concentrated around its
expected value.  Recall the definition $p=1-(1-1/m)^{kn}$ and let
$j=E[W]+ \sqrt{m(\ln m-2k\ln p)}$. Then
\begin{eqnarray}
\Pr\{A\} & = & 
 \Pr\{W \le j\}\times\Pr\{A\mid W\le j\} +
	\Pr\{W >j\} \times \Pr\{A\mid W > j\} \\
  &\le&  1\times\Pr\{A\mid W= j\} +
	\Pr\{W >j\} \times 1 \\
  &\le& \left(\frac{E[W]+\sqrt{m(\ln m-2k\ln p)}}{m}\right)^k + 
       2\exp\left(-\frac{m(\ln m-2k\ln p)}{2m}\right) \\
   &=& \left(p + \sqrt{\frac{\ln m-2k\ln p}{m}}\right)^k+ 
       2p^k/\sqrt{m}  \\  \eqlabel{leap-one}
   &\le& 
	\sum_{i=0}^kp^{k-i}\left(k\sqrt{\frac{\ln m-2k\ln p}{m}}\right)^i 
		+ 2p^k/\sqrt{m} \\
   &\le& p^k\times\left( 
	\sum_{i=0}^k\left(\blah\right)^i 
		+ 2/\sqrt{m}\right) \\
   &=& p^k \times \left(
	\frac{1-\left(\blah\right)^{k+1}}
		{1-\blah}
	+ 2/\sqrt{m} \right) \\
   &\le& p^k \times \left(
	\frac{1}{1-\blah}
	+ 2/\sqrt{m} \right) \\
   &=& p^k \times \left(1+
	\frac{\blah}{1-\blah}
	+ 2/\sqrt{m} \right) \\
   &=& p^k \times \left(1+ O\left(\frac{k}{p}\sqrt{\frac{\ln m-k\ln
p}{m}}\right)\right)
\end{eqnarray}
where \eqref{eq:leap-one} uses the inequality (easily verified by induction
on $k$) which states that
\[
   (a+b)^k \le a^k + kb(a+b)^{k-1}\enspace\mbox{, valid for $a,b \ge 0$} 
\eqlabel{binomial}
\]
and that, when iterated $k+1$ times gives
\[
(a+b)^k \le \sum_{i=0}^k a^{k-i}b^ik!/(k-i)! \le \sum_{i=0}^k a^{k-i}(bk)^i
 \enspace\mbox{, valid for $a,b \ge 0$.}
\]

\subsection{The Lower Bound}
For the lower bound, we use a very different argument.  Let
$b_1,\ldots,b_k$ be the urns in which the $k$ black balls are thrown.
We will show that, for $2\le i\le k$,
\begin{equation}
	\Pr\{\mbox{$b_i$ is white} \mid \mbox{$b_1,\ldots,b_{i-1}$ are white}\}
           > \Pr\{\mbox{$b_i$ is white}\} = p \enspace .
	\eqlabel{conditional}
\end{equation}
Therefore,
\[
	\Pr\{A\}=\prod_{i=1}^k \Pr\{\mbox{$b_i$ is white} \mid
	\mbox{$b_1,\ldots,b_{i-1}$ are white}\} > p^k \enspace .
\]

Note that this lower-bound is strict, so the actual false-positive
rate of a Bloom filter is strictly greater than $p^k$
whenever $k\ge 2$. To finish the proof,
all that remains is to justify \eqref{eq:conditional}.  Recall that
$b_1,\ldots,b_{i-1}$ are just randomly chosen urns.  For any
$j\ge 2$, the following is obvious
\begin{equation}
	\Pr\{\mbox{$b_1,\ldots,b_{i-1}$ are white} \mid W\ge j \}
		> \Pr\{\mbox{$b_1,\ldots,b_{i-1}$ are white}\} \enspace .
	\eqlabel{conditionb}
\end{equation}
We say that this is obvious because, for example, the case in which
all white balls land in one urn is excluded.  From
the definition of conditional probability, \eqref{eq:conditionb} is
equivalent to
\[
	\Pr\{W\ge j \mid \mbox{$b_1,\ldots,b_{i-1}$ are white}\}
	> \Pr\{W \ge j \} \enspace .
\]
The above statement says that the random variable $W$ conditioned on
``$b_1,\ldots,b_{i-1}$ are white'' \emph{stochastically dominates} the
random variable $W$ (conditioned on nothing).  Note that, if a random
variable $X$ stochastically dominates a random variable $Y$ then
$E[X]> E[Y]$.  Therefore,
\[
	E[W\mid \mbox{$b_1,\ldots,b_{i-1}$ are white}] > E[W] \enspace .
\]
Consider the random variable $W/m$ and observe that
$E[W/m]=\Pr\{\mbox{$b_i$ is white}\}$.  Therefore, we have
\begin{eqnarray*}
   \Pr\{\mbox{$b_i$ is white} \mid \mbox{$b_1,\ldots,b_{i-1}$ are
	white}\}
   &=& E[W/m\mid \mbox{$b_1,\ldots,b_{i-1}$ are white}] \\
   &>& E[W/m] \\
   &=&\Pr\{\mbox{$b_i$ is white}\}
\end{eqnarray*}
as required for \eqref{eq:conditional}.

This completes the proof of
\begin{thm}
Let $p_{k,n,m}$ be the false-positive rate for a Bloom filter that
stores $n$ elements in a bit-vector of size $m$ using $k$ hash
functions, where $k\ge 2$ and  $k$, $n$ and $m$ satisfy
\eqref{eq:blah}. Let $p=1-(1-1/m)^{kn}$.  Then,
\[
    p^k < p_{k,n,m} \le 
	p^k \times \left(1+ O\left(\frac{k}{p}
	\sqrt{\frac{\ln m-k\ln p}{m}}\right)\right)
\]
\end{thm}

\section{Conclusions} \seclabel{conclusions}

We have shown that the analysis of Bloom filters originally given by
Bloom, and repeated in many subsequent papers, is incorrect.  The
actual false-positive rate is strictly larger than
$p^k=(1-(1-1/m)^{kn})^k$.  We have also given bounds on how much
larger the false-positive rate can be.  Our upper bounds show that,
for large enough values of $m$ with small values of $k$, the
difference between $p^k$ and the actual false-positive rate is
negligible.

Mullin \cite{m83} and Gremillion \cite{g82} both observe that the
false-positive rate of Bloom filters in their database applications
are slightly higher than $p^k$.  However, they attribute this to poor
quality pseudorandom numbers.  Our results offer another possible
explanation: the actual  false-positive rate is higher than $p^k$,
even if perfect random numbers are available.

\section*{Acknowledgement}

The authors are grateful to Michael Mitzenmacher for bringing his
paper \cite{m02} to our attention.

\bibliographystyle{plain}
\bibliography{bloom}

\end{document}
