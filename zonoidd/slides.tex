\documentclass{beamer}
\usepackage{amsopn}

\DeclareMathOperator{\conv}{conv}
\DeclareMathOperator{\average}{average}

\usepackage{beamerthemesplit}
 
%\usepackage{amsthm}

\newcommand{\centeripe}[1]{\begin{center}\Ipe{#1}\end{center}}
\newcommand{\comment}[1]{}

\newcommand{\centerpsfig}[1]{\centerline{\psfig{#1}}}

\newcommand{\seclabel}[1]{\label{sec:#1}}
\newcommand{\Secref}[1]{Section~\ref{sec:#1}}
\newcommand{\secref}[1]{\mbox{Section~\ref{sec:#1}}}

\newcommand{\alglabel}[1]{\label{alg:#1}}
\newcommand{\Algref}[1]{Algorithm~\ref{alg:#1}}
\newcommand{\algref}[1]{\mbox{Algorithm~\ref{alg:#1}}}

\newcommand{\applabel}[1]{\label{app:#1}}
\newcommand{\Appref}[1]{Appendix~\ref{app:#1}}
\newcommand{\appref}[1]{\mbox{Appendix~\ref{app:#1}}}

\newcommand{\tablabel}[1]{\label{tab:#1}}
\newcommand{\Tabref}[1]{Table~\ref{tab:#1}}
\newcommand{\tabref}[1]{Table~\ref{tab:#1}}

\newcommand{\figlabel}[1]{\label{fig:#1}}
\newcommand{\Figref}[1]{Figure~\ref{fig:#1}}
\newcommand{\figref}[1]{\mbox{Figure~\ref{fig:#1}}}

\newcommand{\eqlabel}[1]{\label{eq:#1}}
\newcommand{\eqref}[1]{(\ref{eq:#1})}

\newtheorem{thm}{Theorem}{\bfseries}{\itshape}
\newcommand{\thmlabel}[1]{\label{thm:#1}}
\newcommand{\thmref}[1]{Theorem~\ref{thm:#1}}

\newtheorem{lem}{Lemma}{\bfseries}{\itshape}
\newcommand{\lemlabel}[1]{\label{lem:#1}}
\newcommand{\lemref}[1]{Lemma~\ref{lem:#1}}

\newtheorem{cor}{Corollary}{\bfseries}{\itshape}
\newcommand{\corlabel}[1]{\label{cor:#1}}
\newcommand{\corref}[1]{Corollary~\ref{cor:#1}}

\newtheorem{obs}{Observation}{\bfseries}{\itshape}
\newcommand{\obslabel}[1]{\label{obs:#1}}
\newcommand{\obsref}[1]{Observation~\ref{obs:#1}}

\newtheorem{assumption}{Assumption}{\bfseries}{\rm}
\newenvironment{ass}{\begin{assumption}\rm}{\end{assumption}}
\newcommand{\asslabel}[1]{\label{ass:#1}}
\newcommand{\assref}[1]{Assumption~\ref{ass:#1}}

\newcommand{\proclabel}[1]{\label{alg:#1}}
\newcommand{\procref}[1]{Procedure~\ref{alg:#1}}

\newtheorem{rem}{Remark}
\newtheorem{op}{Open Problem}

\newcommand{\etal}{\emph{et al}}

\newcommand{\voronoi}{Vorono\u\i}
\newcommand{\ceil}[1]{\left\lceil #1 \right\rceil}
\newcommand{\floor}[1]{\left\lfloor #1 \right\rfloor}



\title{An Optimal Randomized Algorithm \\ 
	for $d$-Variate Zonoid Depth}
\author{Pat Morin}
\institute{Carleton University}
\date{October 12, 2006 \\ ULB}

\begin{document}

\frame{\titlepage}

\section[Outline]{}
\frame{\tableofcontents}

\section{Introduction}
\subsection{Depth Functions}
\frame
{
  \frametitle{Depth Functions}
  \begin{itemize}
  \item<1-> $S$: A set of $n$ points in $\R^d$
  \item<2-> A \emph{depth function} assigns a value $D(p,S)$ to each point in $\R^d$
%  \item<3-> Generalizes 1-dimensional \emph{order statistics} 
  \item<3-> If $D(p,S)$ is big then $p$ is ``central'' with respect to $S$
  \item<4-> Many depth functions have been defined, including Tukey depth,
projection depth, Delaunay depth, Fermat-Weber depth, simplicial
depth, convex-hull peeling depth,\ldots,
  \item<5->zonoid depth
  \end{itemize}
}
    
\subsection{Zonoid Depth}
\frame
{
   \frametitle{Zonoid Depth}
   \begin{itemize}
   \item<1-> Recall the definition of the \emph{convex hull}
    \[ \conv(S) = \left\{\sum_{p\in S} p\lambda_p : 
         \mbox{$0\le\lambda_p\le 1$ and $\sum_{p\in S}\lambda_p = 1$} 
        \right\} 
    \]
   \item<2-> Change it slightly to get the \emph{$k$-zonoid}
    \[ Z_k(S) = \left\{\sum_{p\in S} p\lambda_p : 
         \mbox{$0\le\lambda_p\le 1/k$ and $\sum_{p\in S}\lambda_p = 1$} 
        \right\} 
    \]
   \item<3-> Note that $Z_1(S)=\conv(S)$ and $Z_n(S)=\{\average(S)\}$ 
   \item<4-> The \emph{zonoid depth} of $p$ is 
     \[ Z(p,S)=\max\{k : p\in Z_k(S)\} \]
   \end{itemize}
}

\frame
{
    \frametitle{Zonoid Depth}
     \only<1>{\centerline{\includegraphics{zonoid-eg-ps}} 
	$S$} 
     \only<2>{\centerline{\includegraphics{zonoid-eg-1}}
	$S,Z_1(S)$}
     \only<3>{\centerline{\includegraphics{zonoid-eg-2}}
	$S,Z_1(S),Z_2(S)$}
     \only<4>{\centerline{\includegraphics{zonoid-eg-3}}
	$S,Z_1(S),Z_2(S),Z_3(S)$}
     \only<5>{\centerline{\includegraphics{zonoid-eg-4}}
	$S,Z_1(S),Z_2(S),Z_3(S),Z_4(S)$}
     \only<6>{\centerline{\includegraphics{zonoid-eg-5}}
	$S,Z_1(S),Z_2(S),Z_3(S),Z_4(S),Z_5(S)$}
     \only<7>{\centerline{\includegraphics{zonoid-eg-6}}
	$S,Z_1(S),Z_2(S),Z_3(S),Z_4(S),Z_5(S),Z_6(S)$}
}

\frame
{
   \frametitle{Applications of (Zonoid) Depth}
   \begin{itemize}
    \item<1-> Exploratory data analysis (visualization)
    \item<2-> Multivariate statistics
    \item<3-> Clustering
    \item<4-> Outlier removal
    \item<5-> Machine learning (classification)
    \item<6-> Soft-margin support vector machines
   \end{itemize}
}


\subsection{Zonoid Depth Computation}
\frame
{
  \frametitle{Zonoid Depth Computation}
  \begin{itemize}
  \item<1-> Given $p$ and $S$ compute the zonoid depth of $p$
w.r.t. $S$
  \item<2-> A necessary subroutine for most applications
  \item<3-> Known algorithms:
     \begin{itemize}  
       \item<4-> Polynomial: Solve a linear program in $\lambda_p$ 
		variables ($n$ variables and $O(n)$ inequalities)
       \item<5-> Bern and Eppstein 2001:  $O(n(Ld\log n)^c)$ time where $L$
            is the bit precision of the input  (uses ellipsoid method)
       \item<6-> Gopala and Morin 2004: $O(n)$ expected time, but only
            for $d=2$
       \item<7-> Here: $O(n)$ expected time for any constant $d$ (but
            with a superpolynomial dependence on $d$)
     \end{itemize}
     \item<8-> In this talk we focus mainly on the \emph{decision 
          problem}:  \\ ``Is $p\in Z_k(S)$?''
  \end{itemize}
}


\section{The Algorithm}

\subsection{Finding a Zonoid Vertex}

\frame{
   \frametitle{Finding a Zonoid Vertex}
   To Find the extreme vertex of $Z_k(S)$ in direction $\overrightarrow{w}$:
   \only<1>{\includegraphics{project-1}}
   \only<2>{\includegraphics{project-2}}
   \only<3>{\includegraphics{project-3}}
   \only<4>{\includegraphics{project-4}}
   \begin{enumerate}
     \item<2->Project points of $S$ onto $\overrightarrow{w}$
     \item<3->In $O(n)$ time, find the largest $k$ points in the projection 
     \item<4->Take the average of these $k$ points ($\lambda_p=1/k$)
   \end{enumerate}
}

\frame
{
   \frametitle{The Example Again}
   \begin{center}
    \includegraphics{zonoid-eg-6}
   \end{center}
}



\frame
{
   \frametitle{Going from Optimization to Decision}
 \begin{itemize}
   \item<1->\textbf{Recall:} We want to test if $p\in Z_k(S)$
   \item<1->Finding the extreme vertex of $Z_k(S)$ in any direction
	is easy
   \item<2->Bern and Eppstein use this fact in combination with the 
	ellipsoid method to test if 
	$p\in Z_k(S)$ by trapping $p$
	\begin{center}\includegraphics{trap}\end{center}
   \item<3-> We use this fact to apply a very powerful
	geometric optimization technique\ldots
  \end{itemize}
}

\subsection{Chan's Optimization Theorem}
\frame
{
  \frametitle{Chan's Optimization Theorem (2004)}
  Suppose:  
  \begin{enumerate}
   \item<1-> $\mathcal{P}$ is a computational problem
   \item<2-> $f:\mathcal{P}\mapsto C$ maps problem instances onto closed 
         convex subsets of $\R^d$
   \item<3-> $w:\R^d\mapsto \R$ is a linear objective function
   \item<5-> For any $q\in\R^d$ and any $P\in\mathcal{P}$ of size $n$ we
         can test if $q\in f(P)$ in $D(n)$ time
   \item<6-> For any $P\in\mathcal{P}$ of size $n$ we can find
         $P_1,\ldots, P_r\in\mathcal{P}$ such that $|P_i|\le\alpha n$ 
         and $f(P)=\bigcap_{i=1}^r f(P_i)$ \hfill{[$r=O(1)$ and $\alpha<1$]}
  \end{enumerate}
  Then:
  \begin{itemize}
    \item<4-> For any $P\in\mathcal{P}$ of size $n$ we can find the point
    $q\in f(P)$ that maximizes $w(q)$ in $O(D(n))$ time
  \end{itemize}
} 

\frame
{
   \frametitle{Chan's Technique}

   \begin{center}
   \begin{tabular}{ccc}
    $P$ & $\stackrel{f}{\Rightarrow}$ & \includegraphics[scale=.4]{chan-1} \\
    $\Downarrow$ \\
    $P_1$,$P_2$,$P_3$,$P_4$ & $\stackrel{f}{\Rightarrow}$ 
	& \includegraphics[scale=.2]{chan-2} \\
   \end{tabular}
   \end{center}
}

\subsection{Geometric Duality}
\frame
{ \frametitle{Geometric Duality}
   To apply Chan's Theorem we work in dual-space:
   \only<1>{\centerline{\includegraphics{dual-7}}
	Points become\ldots}
   \only<2>{\centerline{\includegraphics{dual-8}}
	Points become hyperplanes}
    \only<3>{\centerline{\includegraphics{dual-9}} 
	A polytope becomes\ldots}
    \only<4>{\centerline{\includegraphics{dual-0}} 
	A polytope becomes a set of hyperplanes}
}

\subsection{Dual Zonoids}
\frame
{
   \frametitle{Dual Zonoids}
   \begin{itemize}
     \item<1-> \textbf{Recall:} We want to test if $p\in Z_k(S)$
      \begin{center}
	\only<1>{\includegraphics[scale=.8]{dual-zonoid-1}}
	\only<2>{\includegraphics[scale=.8]{dual-zonoid-2}}
	\only<3->{\includegraphics[scale=.8]{dual-zonoid-3}}
      \end{center}
     \item<2-> Equivalent to asking if $\mathrm{dual}(p)$ separates 
	$A_k(S)$ and $B_k(S)$.
     \item<3-> Assuming $p=(0,0,\ldots,0)$ implies $\mathrm{dual}(p)$ 
	is the flat hyperplane through the origin
     \item<4-> Find the lowest point in $A_k(S)$ and the highest point
	in $B_k(S)$ 
   \end{itemize}
}

\subsection{Overview of Algorithm}
\frame
{
     \frametitle{Review}
     \begin{itemize}
     \item<1-> To test if $p\in Z_k(S)$:
     \begin{enumerate}
      \item<2-> Translate input so that $p=(0,0,\ldots,0)$
      \item<3-> Find the lowest point $a$ in $A_k(S)$
      \item<4-> Find the highest point $b$ in $B_k(S)$
      \item<5-> If $a_d < 0$ or $b_d >0$
	then return \textbf{false}, else return \textbf{true}
     \end{enumerate}
     \item<6->Our problem reduces to two linear optimization problems over
	the implicit polytopes $A_k(S)$ and $B_k(S)$ 
     \item<7->We handle these problems using Chan's Optimization Theorem
    \end{itemize} 
}

\subsection{Applying Chan's Optimization Theorem}

\frame{
     \frametitle{Applying Chan's Method}
     To find the lowest point in $A_k(S)$:
     \begin{enumerate}
        \item<1-> Problem instance: $P=(S,k)$
        \item<2-> Constraint function: $f(S,k) = A_k(S)$
        \item<3-> Objective function: $w(x_1,\ldots,x_d)=-x_d$
     \end{enumerate}
}

\frame
{
     \frametitle{Testing if $q\in f(P)$}

     \begin{enumerate}
	\setcounter{enumi}{3}
     \item To test if $q\in A_k(S)$:
     \begin{center}
	\only<1>{\includegraphics[scale=.9]{decision-0}}
	\only<2>{\includegraphics[scale=.9]{decision-1}}
	\only<3>{\includegraphics[scale=.9]{decision-2}}
	\only<4>{\includegraphics[scale=.9]{decision-3}}
	\only<5->{\includegraphics[scale=.9]{decision-4}}
     \end{center}
     \end{enumerate}
     \begin{itemize}
     \item<6->{Testing if $q\in A_k(S)$ can be done in
	$D(n)=O(n)$ time}
     \end{itemize}
} 


\frame{
   \frametitle{Partitioning into Subproblems}
   
   \begin{enumerate}
   \setcounter{enumi}{4}
   \item To partition $P=(S,k,p)$ into subproblems we use cuttings
   \begin{itemize}
   \item<2->A \emph{cutting} partitions $\mathbb{R}^d$ into $O(1)$ simplices
	$\Delta_1,\ldots,\Delta_r$ such that no simplex intersects
        more than $\ceil{n/2}$ lines of $\mathrm{dual}(S)$.
   \begin{center}
	\only<2>{\includegraphics[scale=.8]{cutting-4}}
	\only<3>{\includegraphics[scale=.8]{cutting-3}}
	\only<4->{\includegraphics[scale=.8]{cutting-2}}
   \end{center}
   \item<3-> We get subproblem $P_i$ by combining all the lines above 
	$\Delta_i$ into a single \emph{multiline} and removing all 
	the lines below $\Delta_i$ 
   \end{itemize}
   \end{enumerate}
}

\frame
{
   \frametitle{Correctness of Cutting}
   \begin{itemize}
     \item<1-> We need to show that $f(P)=\bigcap_{i=1}^r f(P_i)$
     \item<2-> For each $P_i$ the associated polytope $f(P_i)$ contains
          $f(P)=A_k(S)$,  so $f(P) \subseteq \bigcap_{i=1}^r f(P_i)$
          \begin{center}\includegraphics{correctness}\end{center}
     \item<3-> For each point $x$ on the boundary of $f(P)$ there is a
	subproblem $P_i$ such that $x$ is on the boundary of $f(P_i)$,
	so $\bigcap_{i=1}^r f(P_i)\subseteq f(P)$
     \item<4-> Done!
   \end{itemize}
}

\section{Summary and Open Problems}

\frame
{
   \frametitle{Summary and Open Problems}
   \begin{itemize}

     \item<1-> We give an $O(n)$ time expected-time algorithm for
computing the zonoid depth of a point $p$ with respect to a set $S$ in
$\mathbb{R}^d$, for any constant $d$.

     \item<2-> \textbf{Open Problem:} A deterministic algorithm
     \item<3-> \textbf{Open Problem:} A linear-in-$n$, polynomial-in-$d$
         algorithm (with no dependence on $L$)
     \item<4-> \textbf{Open Problem Family:} The computational complexity
	of other data depth definitions (see Aloupis 2006 for a survey)
   \end{itemize}
}



\end{document}

