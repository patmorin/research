%% -*- LaTeX -*- This is LaTeX2e code


\newcommand{\LU}{\ensuremath{<_{y.\mathrm{upper}}}}
\newcommand{\LY}{\ensuremath{<_{y}}}

\section{Orthogonal Line-Segment Intersection}\label{sec:olsi}

In this appendix, we describe an algorithm that solves the
orthogonal line segment intersection problem in
\Oh{n \log n +k} time using \Oh{1} extra space, where $n$
is the total number of line segments and $k$ is the total number
of intersections. We assume that the input is given in the form
of two arrays ${H}[0\ldots n_h-1]$ and
${V}[0\ldots n_v-1]$ of horizontal and vertical line
segments, respectively, where $n = n_h + n_v$.

We will describe the algorithm as a recursive algorithm. Since it
follows the general framework of Section~\ref{sec:tools}, however, we
can make it space-efficient so that it uses only \Oh{1} extra space.

Consider a subarray ${V}[0\ldots e_v-1]$, and let
$m:=2^{\lfloor \log_2 ((e_v)/2) \rfloor}$. Let $i$ be the index
such that the $X$-coordinate of $V[i]$ is the $m$-th smallest among
all $X$-coordinates of the line segments in this subarray.
Our algorithm will use $V[i]$ to first partition the subarray into a
subarray of length $m$ (we call this ``partitioning the left'') and then
into a subarray of length $e_v-m$ (we call this ``partitioning
the right''). These partitioning algorithms are given as
Algorithms~\ref{algPreVP} and~\ref{algMidVP}. Both of them can be 
undone, see Algorithms~\ref{algUndoPreVP} and~\ref{algUndoMidVP}. 

\begin{algorithm}

  \caption{\textsc{PreVerticalPartition}(${V}$,$e_v$) Partition
  the vertical segments before the first recursive call (Partitioning
  the left)}
  
  \label{algPreVP}
  \begin{algorithmic}[1]
    \REQUIRE ${V}[0\ldots e_v-1]$ is sorted according to
    \LU, i.e., $Y$-coordinates of the upper endpoints. 
    \ENSURE That $e_v:=m$ and the resulting array 
    ${V}[0\ldots e_v-1]$ contains all vertical segments not to the 
    right of the $X$-median, and is sorted according to \LU. 
    \STATE $m \leftarrow 2^{\lfloor \log_2 ((e_v)/2) \rfloor}$
    %% \COMMENT{$m=e_v/2 \iff e_v\neq |V|$}
    \STATE $i \leftarrow 
           \textsc{RestoringSelect}({V},0,e_v,m,\textsc{Select})$. 
    \STATE Using \textsc{SortedSubsetSelection}, move all elements
    less than or equal to ${V}[i]$ to ${V}[0\ldots m-1]$.
    \STATE $\textsc{Push}(S,\textsc{Left})$.
    \STATE $e_v \leftarrow m$
    \STATE Return $e_v$
 \end{algorithmic}
\end{algorithm}

\begin{algorithm}

  \caption{\textsc{UndoPreVerticalPartition}(${V}$,$e_v$)
  Undo the pre-partitioning of the vertical segments after returning
  from the first recursive call.}
  \label{algUndoPreVP}
  \begin{algorithmic}[1]
    \REQUIRE The first recursion has ended, and we are given an array 
    ${V}[0\ldots e_v-1]$ that is sorted according to \LU.
    \ENSURE  The variable $e_v$ has been reset to its original value and 
    the resulting array ${V}[0\ldots e_v-1]$ is sorted 
    according to \LU.
    \STATE $\textsc{Pop}(S)$.
    \IF{$2e_v \leq |V|$}
      \STATE $e_v^{orig} \leftarrow 2e_v$.
    \ELSE
      \STATE $e_v^{orig} \leftarrow |V|$.
    \ENDIF
    \STATE $\textsc{UndoSubsetSelection}({V},0,e_v^{orig},e_v)$
    \STATE $e_v \leftarrow e_v^{orig}$
    \STATE Return $e_v$
 \end{algorithmic}
\end{algorithm}

\newcounter{RestoreRight}

\begin{algorithm}
  \caption{\textsc{MidVerticalPartition}(${V}$,$e_v$) Partition
  the vertical segments before the second recursive call.} 
  \label{algMidVP}
  \begin{algorithmic}[1]
    \REQUIRE ${V}[0\ldots e_v-1]$ is sorted according to \LU.
    \ENSURE That $e_v := e_v - m$ and that the resulting array 
    ${V}[0\ldots e_v]$ contains all vertical segments to the right of 
    the $X$-median and is sorted according to \LU.
    \STATE $m \leftarrow 2^{\lfloor \log_2 (e_v/2) \rfloor}$
    %% \COMMENT{$m=e_v/2 \iff e_v\neq |V|$}
    \STATE $i \leftarrow 
         \textsc{RestoringSelect}({V},0,e_v,m,\textsc{Select})$. 
    \STATE Using \textsc{SortedSubsetSelection}, move all elements
    larger than ${V}[i]$ to ${V}[0\ldots e_v - m-1 ]$.
    \STATE $\textsc{Push}(S,\textsc{Right})$.
    \STATE $e_v \leftarrow e_v-m$
    \STATE Return $e_v$
 \end{algorithmic}
\end{algorithm}

\begin{algorithm}

  \caption{\textsc{UndoMidVerticalPartition}(${V}$,$e_v$)
  Undo the partitioning of the vertical segments after returning from
  the second recursive call}
  \label{algUndoMidVP}
   
  \begin{algorithmic}[1]
    \REQUIRE The last recursive call has ended, and we are
    given an array ${V}[0\ldots e_v]$ that is
    sorted  according to \LU.
    \ENSURE The variable $e_v$ has been reset to its original value and 
    the resulting array ${V}[0\ldots e_v-1]$ is sorted 
    according to \LU.  
    \STATE $\textsc{Pop}(S)$.
    \STATE Let $d$ be the number of elements on the stack $S$.
    \STATE $e_v^{orig} \leftarrow 2^{\lceil\log_2 |V|\rceil - (d+1)} +
    e_v$. \setcounter{RestoreRight}{\value{ALC@line}} 
    \STATE $\textsc{UndoSubsetSelection}({V},0,e_v^{orig},e_v)$
    \STATE $e_v \leftarrow e_v^{orig}$
    \STATE Return $e_v$
 \end{algorithmic}
\end{algorithm}

The observation that shows the correctness of the formula for
restoring the value of $e_v$ (Line~\theRestoreRight\ in 
Algorithm~\ref{algUndoMidVP}) is that the left
subtree of the current node is a complete binary tree. The height of
the tree is the height of the recursion tree (which is $\lceil\log_2
|V|\rceil$) minus the depth of the current node.
%% , that is minus the depth of the ``recursion'' at present. 
The number of leaves in the
left subtree, and hence the number of elements on which the first 
recursive call took place, is then $2^{\lceil\log_2 |V|\rceil -
  (d+1)}$, which is also the index of the split point.

\newcounter{StopElement}
\begin{algorithm}
  \caption{\textsc{PreHorizontalPartition}(${H}$,$e_h$)
  Partition the horizontal segments before the first recursive call.} 
   \label{algPreHP}
  \begin{algorithmic}[1]
    \REQUIRE ${H}[0\ldots e_h-1]$ is sorted according to
    \LY. The current slab boundaries as well as the median for
    splitting the slab are known.
    \ENSURE That $e_h := m$ and the resulting array 
    ${H}[0\ldots e_h-1]$ contains all horizontal segments to be 
    passed to the first recursion sorted according to \LY.
    \STATE Let $m$ be the number of elements in ${H}[0\ldots
    e_h-1]$ that intersect the left sub-slab and do not cross the
    current slab.
    \IF{$m < e_h$}
      \STATE $\textsc{Push}(T,1)$. \COMMENT{At least one segment will not
        move.}
      \STATE Synchronously go back in stack $T$ and stack $S$ and find
      the most recent recursion (except for the current) where at
      least one segment was not moved. Let $R$ be the type of this
      recursion. 
      \IF{$R=\textsc{Right}$}
        \STATE Let $h$ be the segment of those crossing the current
        slab with the leftmost left endpoint. 
      \ELSE
        \STATE Let $h$ be the segment of those crossing the current
        slab with the rightmost right endpoint. 
      \ENDIF
      \IF{$h$ is undefined}
        \STATE Let $h$ be ${H}[m]$. \COMMENT{Don't do anything.}
      \ENDIF
      \STATE Move $h$ to ${H}[m]$. \setcounter{StopElement}{\value{ALC@line}}
      \STATE Using \textsc{SortedSubsetSelection}, move all elements
      except for those that (a) avoid the left sub-slab or (b)
      cross the current slab to ${H}[0\ldots m-1]$. 
    \ELSE
      \STATE $\textsc{Push}(T,0)$.
    \ENDIF
    \STATE $e_h \leftarrow m$
    \STATE Return $e_h$
 \end{algorithmic}
\end{algorithm}

\begin{algorithm}

  \caption{\textsc{UndoPreHorizontalPartition}(${H}$,$e_h$)
  Undo the partitioning of the horizontal segments after returning from
  the first recursive call.} 
    \label{algUndoPHP}
    
  \begin{algorithmic}[1]
    \REQUIRE The first recursive call has ended, and we are
    given an array ${H}[0\ldots e_h-1]$ that is
    sorted  according to \LY. The current slab boundaries as well as
    the median for splitting the slab is known.
    \ENSURE The variable $e_h$ have been reset to its original value and
    the resulting array ${H}[0\ldots e_h-1]$ is sorted 
    according to \LY. 
    \STATE $i \leftarrow \textsc{Pop}(T)$.
    \IF{$i=0$}
      \STATE \COMMENT{No partitioning needs to be reverted.}
    \ELSE
      \STATE $h \leftarrow {H}[e_h]$.
      \IF{$h$ crosses the current slab}
        \STATE Synchronously go back in stack $T$ and stack $S$ and find
        the most recent recursion (except for the current) where at
        least one segment was not moved. Let $R$ be the type of this
        recursion. 
        \IF{$R=\textsc{Right}$}
          \STATE Starting at ${H}[e_h+1]$, scan to find the first
          element that either is right of the current slab or which
          crosses the current slab and whose left endpoint is left of
          $h$'s left endpoint.  
        \ELSE
          \STATE Starting at ${H}[e_h+1]$, scan to find the first
          element that either is right of the current slab or which
          crosses the current slab and whose right endpoint is right of
          $h$'s right endpoint. 
        \ENDIF
      \ELSE
        \STATE Starting at ${H}[e_h+1]$, scan to find the
        first element whose right endpoint is right of the right slab
        boundary or the first element which crosses the current slab.
      \ENDIF
      \STATE Let $e_h^{orig}$ be the index of the element just found.
      \STATE $\textsc{UndoSubsetSelection}({H},0,e_h^{orig},e_h)$
      \STATE Move $h$ to its proper position.
      \STATE $e_h \leftarrow e_h^{orig}$
      \STATE Return $e_h$
    \ENDIF
 \end{algorithmic}
\end{algorithm}

After a subarray ${V}[0\ldots e_v-1]$ has been ``partitioned
to the left'', the vertical slab spanned by
${V}[0\ldots e_v-1]$ (this is the \emph{current slab})
has been partitioned into a \emph{left sub-slab} and a
\emph{right sub-slab}. At this moment, we use these sub-slabs to
partition the corresponding subarray ${H}[0\ldots e_h-1]$
of horizontal line segments. To be more precise, in ``partitioning the 
left'' (see Algorithm~\ref{algPreHP}), the initial part 
of the subarray of ${H}$ contains all $m$ horizontal line segments
in the subarray that intersect the left sub-slab and do not cross the
current slab. A problem arises when we want to undo this partitioning
(in Algorithm~\ref{algUndoPHP}): At that moment, we do not know the 
original value of $e_h$. The solution is to store a ``special'' horizontal 
segment (\emph{viz}.\ the segment $h$ in Line~\theStopElement\ 
of Algorithm~\ref{algPreHP}) in ${H}[m]$.

This segment is used to distinguish horizontal segments crossing the
left sub-slab from horizontal segments crossing a slab corresponding
to a recursive call higher up in the recursion tree. These segments
may be stored in cells ${H}[m]$ and higher and may make it
impossible to re-obtain the original index $e_h$ that is needed in the
restoration process. If there is no segment crossing the current
slab, but at least one segment did not move, we can easily re-obtain
the original index $e_h$ by searching for the first segment that is either
to the right of the current slab or completely crosses the current
slab.

Because of the special role of the horizontal line segment $h$,
we use a variant of \textsc{SortedSubsetSelection} in
Algorithm~\ref{algPreHP} and a variant of \textsc{UndoSubsetSelection}
in Algorithm~\ref{algUndoPHP}. These variants skip over the line 
segment $h$. 
%entry
%${H}[e_h]$.

We have only described how to partition the subarray
${H}[0\ldots e_h-1]$ ``to the left''.
In a completely symmetric way, this subarray can be partitioned
``to the right''.

Having these subroutines at hand, the algorithm solving the orthogonal 
line segment intersection problem is given as Algorithm~\ref{algIOLSI}.

\begin{algorithm}

  \caption{\textsc{ImprovedOLSI}(${V}$,$e_v$,${H}$,$e_h$)
  Solving the Orthogonal Line Segment Intersection Problem.} 
   \label{algIOLSI}
  
  \begin{algorithmic}[1]
    \STATE Scan ${V}[0\ldots e_v-1]$ to compute the
    boundaries of the current slab (min/max values of
    $X$-coordinates). 
    \STATE Using \textsc{SortedSubsetSelection}, move all horizontal 
    segments spanning the current slab to ${H}[0\ldots \ell -1]$. 
    \STATE Perform a top-down sweep over ${H}[0\ldots \ell -1]$
    and ${V}[0\ldots e_v-1]$ to find all intersections.
    \STATE \textsc{UndoSubsetSelection} on
    ${H}[0\ldots e_h-1]$.
    \STATE $e_v \leftarrow$ \textsc{PreVerticalPartition}(${V}$,$e_v$)
    \STATE $e_h \leftarrow$ \textsc{PreHorizontalPartition}(${H}$,$e_h$)
    \STATE \textsc{ImprovedOLSI}(${V}$,$e_v$,${H}$,$e_h$)
    \STATE $e_v \leftarrow$ \textsc{UndoPreVerticalPartition}(${V}$,$e_v$)
    \STATE \textsc{UndoPreHorizontalPartition}(${H}$,$e_h$,$m_h$)
    \STATE $e_v \leftarrow$ \textsc{MidVerticalPartition}(${V}$,$e_v$)
    \STATE $e_h \leftarrow$ \textsc{MidHorizontalPartition}(${H}$,$e_h$)
    \STATE
    \textsc{ImprovedOLSI}(${V}$,$e_v$,${H}$,$e_h$)
    \STATE $e_v \leftarrow$ \textsc{UndoMidVerticalPartition}(${V}$,$e_v$)
    \STATE $e_h \leftarrow$ \textsc{UndoMidHorizontalPartition}(${H}$,$e_h$)
  \end{algorithmic}
\end{algorithm}

\begin{theorem}
Given a set of $n$ horizontal and vertical line segments, all $k$ 
intersections among them can be reported in \Oh{n\log n + k} time 
using \Oh{1} extra space. 
\end{theorem}


%%% Local Variables:
%%% mode: latex
%%% TeX-master: "paper"
%%% End:
