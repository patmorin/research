\documentclass{patmorin}
\usepackage{amsthm,amsmath,graphicx,stmaryrd,amsopn}
\usepackage{pat}
 
\newcommand{\eps}{\varepsilon}

\title{\MakeUppercase{A Crossing Number Inequality for 3D Grid Drawings}}
\author{Vida Dujmovi\'c,\thanks{Department of Mathematics and Statistics
and Department of Systems and Computer Engineering, Carleton University}\, Pat Morin,\thanks{School of Computer Science, Carleton University}\, and Howard Wolowitz}


\DeclareMathOperator{\x}{x}
\DeclareMathOperator{\y}{y}
\DeclareMathOperator{\z}{z}
\DeclareMathOperator{\crs}{cr}
\DeclareMathOperator{\len}{len}
\DeclareMathOperator{\skp}{skip}
\DeclareMathOperator{\ncs}{G}

\begin{document}
\maketitle

\begin{abstract}
   We study crossing number inequalities drawings of graphs on
   $d$-dimensional grids.
\end{abstract}

\section{Introduction}

The study of crossings in graph drawing has a long history.  \emph{Euler's
Formula} states that the maximum number of edges in a planar graph with
$n$ vertices is $3n-6$.  Using Euler's Formula and careful counting,
Ajtai \etal\ \cite{S} showed that any plane drawing of a graph with $n$
vertices and $m\ge 4n$ edges has at least $c m^3/n^2$ crossings, for
some constant $c$.  Using this result, the same authors showed that the
maximum number of different planar graphs that can be embedded any set
of $n$ points is $2^{O(n)}$

The lower bound, $cm^3/n^2$, on the number of crossings in a plane drawing
has since become known as the ``the \emph{Crossing Lemma}'' or ``the Crossing
Number Inequality.''  Szekely \cite{s} showed that this inequality
can be used to give very simple proofs of many results in incidence
geometry, including a proof of the Szemeredi-Trotter Theorem on point-line
incidences.  Szekely's method has since been used for many combinatorial
geometry problems; the most famous of these applications is probably the
result of Dey \cite{d98} on the maximum number $k$-sets of a point set.

Ajtai \etal's proof of the Crossing Lemma uses the probabilistic method
in the sense of Chvatal \cite{cXX}: The proof works by counting the
number of crossings in two different ways.  More recently, a ``from the
book'' proof of the Crossing Lemma that uses a more obvious application
of the probabilistic method and has a better constant was discovered
by Chazelle, Sharir, and Welzl (See Aigner and Ziegler \cite{azXX}).
Pushing this technique even further, Pach \etal\ currently hold the
record for the largest constant, $c$.

Our goal in this paper is to extend the results of Ajtai \etal\ to graph
drawings in higher dimensions and, in particular, to graphs drawn on
the grid.  The $d$-dimensional \emph{$X_1\times\cdots\times X_d$ grid}
is the point set
\[  \{(x_1,\ldots,x_d): 
      \mbox{$x_i\in\{1,\ldots,X_i\}$ for all $i\in\{1,\ldots,d\}$}\}
	\enspace .\]
The \emph{volume} of the $X_1\times\cdots\times X_d$
grid is $\prod_{i=1}^d X_i$, i.e,. the number of points in the grid.
A \emph{(proper) drawing}, $\varphi:V(G)\rightarrow R$, of a graph, $G$,
on a grid, $R$, is a one-to-one mapping such that, for every edge $uw\in
E(G)$ and every vertex $x\in V(G)$, the open line segment with endpoints
$\varphi(u)$ and $\varphi(w)$ does not contain $x$.

We extend the notation $\varphi$ to edges, so that $\varphi(uw)$ denotes
the open line segment with endpoints $\varphi(u)$ and $\varphi(w)$.
Two edges $uw$ and $xy$ in a drawing \emph{cross} if $\varphi(uw)$
and $\varphi(xy)$ have a point in common.  When this happens, we
say that $uw$ and $xy$ generate a \emph{crossing}.  For a graph $G$,
define $\crs_d(G,n)$ to be the minimum number of crossings in a drawing
of $G$ on a $d$-dimensional grid of volume at most $n$.  Define
\[ g_d(n,m)=\min\{\crs_d(G,n):\text{$G$ is a graph with $m$ edges}\} \enspace .
\]
Define $\ncs_d(n)$ as the maximum number of non-crossing graphs that
can be drawn on any $d$-dimensional grid of volume at most $n$.

The results of Ajtai \etal\ \cite{S} on crossings in 2-dimensions all
have implications for $g_2(n,m)$ and $G_2(n)$. Euler's Formula implies
that implies that $g_2(n,3n-5)\ge 1$.  The Crossing Lemma implies that
$g_2(n,m)\in\Omega(m^3/n^2)$ for $m\ge 4n$.  The bound of $2^{O(n)}$ on
the number of planar graphs that can be drawn on any point set implies
that $\ncs_2(n)\in 2^{O(n)}$.

Czyzowicz \etal\ \cite{cXX} show that the maximum number of
edges in a non-crossing $d$-dimensional grid drawing is at most
$(2^d-1)n-\Theta(n^{(d-1)/d})$. This result is analagous to Euler's
Formula in that it shows that such graphs have a linear number of edges.
It also implies, for example, that $g_2(n,(2^d-1)n)\ge 1$.

Since Euler's Formula is the main property used by Ajtai \etal\ to
prove their results, it seems reasonable that similar bounds should
hold for $d$-dimensional grid drawings.  In the current paper, we study
$g_d(n,m)$ and $\ncs_d(n)$ for $d\ge 3$ and prove the results shown
in \tabref{results}.  Talk some more about these results\ldots


%In this paper, we show that there exists constant $c_1,c_2,c_3>0$ such that, 
%for all $m> c_3n$
%\[
%   c_1(m^2/n)\log\log(m/n) \le g(n,m) \le c_2(m^2/n)\log(m/n) \enspace .
%\]
%The proof of the lower bound is given in \secref{lower-bound} and the
%upper bound is given in \secref{upper-bound}.

\begin{table}
  \begin{center}
    \begin{tabular}{rlccc}
      $d$ & $g_d(n,m)$ & $\ncs(n)$ & Refs \\ \hline
      1 & --- & $2^{n-1}$ & \\
      2 & $\Theta(m^{3}/n^2)$ & $2^{\Theta(n)}$ & \cite{aiisjsXX} \\
      3 & $\Omega((m^2/n)\log\log(m/n))$ & ?  & here \\
      3 & $O((m^2/n)\log(m/n))$ & ?  & here \\
      $\ge 4$ & $\Theta(m^{2}/n)$ & $2^{\Theta(n\log n)}$ & here \\
    \end{tabular}
  \end{center}
  \caption{Old and new results on crossings in grid drawings.}
  \tablabel{results}
\end{table}

\section{3-Dimensional Grid Drawings}

In this section, we present upper and lower bounds on $g_3(n,m)$.

\subsection{The Lower Bound}
\seclabel{3d-lower-bound}

\begin{thm}
  $g(n,m) \in \Omega((m^2/n)\log\log (m/n))$.
\end{thm}

\begin{proof}
Let $G$ be a drawing of a graph on the $X\times Y\times Z$ grid, with
$n=XYZ$, that has $m$ edges and for which $\crs(G,n)=g(n,m)$. ($G$
has the minimum number of crossings over all drawings of graphs with
$m$ edges.)  We may assume, without loss of generality, that no edge of
$G$ contains any point of the $X\times Y\times Z$ grid in its interior;
such an edge could be replaced with a shorter edge without introducing
any additional crossings.

For a prime number, $p$, define the $X\times Y\times Z$ \emph{$p$-grid}
as the set of points
\[
  \{(x/p,y/p,z/p): x\in\{1,\ldots,pX\},\, y\in\{1,\ldots,pY\},\,
  z\in\{1,\ldots,pZ\}\} \enspace .
\]
Observe that, for two distinct primes $p$ and $q$, the intersection
of the $p$-grid and the $q$-grid is exactly the $X\times Y\times Z$
grid. (This is because the only integers $a$ and $b$ that satisfy the
equation $a/p = b/q$ have $a\equiv 0 \pmod{p}$ and $b\equiv 0\pmod q$,
so $a/p$ and $b/q$ are each integers.)

Note that the number of points in the $X\times Y\times Z$ $p$-grid is
no more than $np^3$.  Note, furthermore, that each edge $uw$ of $G$
contains the $p$-grid points
\[
    P_{uw}^p = \{ (i/p)(u-w) : i\in\{1,\ldots,p-1\} \} \enspace .
\]
Next, we claim that $P_{uw}^p$ does not contain any $q$-grid points for
any prime $q\neq p$.  In fact, $P_{uw}^p$ does not contain any point of
the $X\times Y\times Z$ grid.  This follows from the assumption that
no edge contains a grid point in its interior, so $\gcd(\x(u)-\x(w),
\y(u)-\y(w), \z(u)-\z(w))=1$.  In particular, at least one of
$\x(u)-\x(w)$, $\y(u)-\y(w)$, $\z(u)-\z(w)$ does not have $p$ as a
divisor.  If, say, $\x(u)-\x(w)$ does not have $p$ as a divisor then
$i(\x(u)-\x(w))/p$ is not an integer and therefore $(i/p)(u-w)$ is not
on the $X\times Y\times Z$ grid.

Therefore, there are at least $(p-1)m$ incidences between edges of $G$
and points of the $p$-grid that are not points of any $q$-grid for $q\neq p$.
Any point of the $p$ grid incident
on $k$ edges contributes $\binom{k}{2}$ to $\crs(G)$.  Therefore, the
total contribution of the $p$-grid to $\crs(G)$ is at least
\[
    np^3\binom{m(p-1)/np^3}{2} \ge cm^2/np
\]
for $c<blah$ and $m(p-1)/np^3 \ge 2$.  This latter condition can be rewritten
as $p \le c'(m/n)^{1/2}$ for a sufficiently small $c'>0$.
Finally, we finish by summing over primes $p$ smaller than $c'p^{1/2}$:
\[
   \crs(G) \ge \sum_{p} cm^2/np = (cm^2/n) \sum_{p} 1/p = ((cm^2/n)(\ln\ln(m/n)-O(1))
\]
where the last inequality is Euler's result on the sums of reciprocals
of primes. (This sum is also known as a harmonic series of primes.  See,
e.g., Graham, Knuth and Patashnik \cite[pages~22--25]{gkpXX}.)
\end{proof}

%\To see this, notice that 
%\begin{thm}
%  $g(n,m,q)=\Omega(m^3/q^2)$ 
%\end{thm}
%
%\begin{proof}
%The weaker inequality ,
%\begin{equation}
%  g(n,m,q)\ge m-7q \enspace ,\eqlabel{weak}
%\end{equation}  
%is easily shown by induction
%on $m$, using the result of Cyzowicz \etal\ \cite{X} which states that
%$g(n,m,q) > 0$ if $m> 7q$ (any drawing with more than $7q$ edges contains
%at least one crossing).
%
%Next we show that, using induction on $n$, that 
%\[
%   g(n,m,q) \ge Cm^3/q^2
%\]
%for any $m\ge 8q$.  Note that the case 
%
%
%First, note that the result is easily established
%using \eqref{weak} for any $m$ such that $8q\le m\le 9q$ for $C\le 1/81$.
%
%Next, consider the case $m> 9q$.  Consider some drawing, $\varphi$, of
%$G$ with $\crs(G,q)$ crossings.  Count the pairs $(c,w)$ where $c$ is
%a crossing and $w$ is a vertex not incident to either of the two edges
%involved in $c$.  Each crossing, $c$, appears in $n-4$ such pairs so
%the total number of those pairs is $(n-4)\crs(G,q)$.  On the other hand,
%each vertex, $w$, occurs in at least $\crs(G\setminus w,q)$ such pairs.
%Therefore,
%\[
%   (n-4)\crs(G,q) \ge \sum_{w\in V(G)} \crs(G\setminus w, q) \enspace .
%\]
%The number, $m_w$, of edges in $G\setminus w$ is at least $m-(n-1)
%\ge m-q\ge 8q$ so, by induction, $\crs(G\setminus w,q)\ge Cm_w^3/q^2$.
%Putting everything together, we have
%\begin{align*}
%  (n-4)\crs(G,q) 
%    & \ge  \sum_{w\in V(G)} \crs(G\setminus w, q) \\
%    & \ge  C\sum_{w\in V(G)} m_w^3/q^2  \\
%    & \ge  Cn((n-2)m/n)^3/q^2 & \left(\mbox{since $\sum_{w\in V(G)} m_w^3\ge n((n-2)m/n)^2$}\right) \\
%    & =  (Cm^3/q^2)\cdot(n(n-2)^3/n^3)  \\
%\end{align*}
%Damn! The second factor in the last line is less than 1 and I don't see any
%way to fix it.
%\end{proof}
%

\subsection{The Upper Bound}
\seclabel{3d-upper-bound}

In this section, we prove the following result:

\begin{thm}\thmlabel{upper-bound}
  $g(n,m) \in O((m^2/n)\log (m/n))$.
\end{thm}

The proof of \thmref{upper-bound} follows easily from the following lemma:
\begin{lem}\lemlabel{2-layer}
  There exists a drawing of the complete bipartite graph $K_{k^2,k^2}$
  on the $k\times k\times 2$ grid with $O(k^6\log k)$ crossings.
\end{lem}

Before proving the lemma, we first show how it implies
\thmref{upper-bound}: For simplicity, in what follows, assume $\sqrt{n}$,
$\sqrt{m/n}$, and $\sqrt{m}/n$ are each integers.  Apply \lemref{2-layer},
with $k=\sqrt{m/n}$ and tile the $\sqrt{n}\times\sqrt{n}\times 2$ grid
with $n/k^2$ copies of this drawing. The resulting geometric graph has
$2n$ vertices, $nk^4/k^2=m$ edges and
\[ O((n/k^2)k^6\log k) = O(nk^4\log k) = O((m^2/n)\log(m/n))
\] 
crossings, as required by \thmref{upper-bound}.

\begin{proof}[Proof of \lemref{2-layer}]
The drawing of $K_{k^2,k^2}$ is the obvious one; each point with
$z$-coordinate 1 is connected by an edge to every point with
$z$-coordinate 2.  We shall denote the resulting geometric graph
by $G_{k^2}$.

We start by considering some edge $uw$ with $\z(u)=1$ and counting
the number of edges that intersect $uw$.  Let $\pi_1,\ldots,\pi_k$ be
the planes that contain $uw$ and at least one additional grid point.
Observe that each plane $\pi_j$ contains, and is uniquely determined by,
a line in the plane $z=1$ that passes through some point, $p_j$ of the
integer lattice $\N^2$ and such that $up_j$ does not contain any other
lattice points, i.e., $\gcd(\x(u)-\x(p_j),\y(u)-\y(p_j))=1$ (see \figref{planes}).
\begin{figure}
  \centering{\includegraphics{planes}}
  \caption{A plane $\pi_j$, defined by a point $p_j$, with $\skp(\pi_j)=2$. ($\z$-coordinates are exaggerated.)}
  \figlabel{planes}
\end{figure}
Define the \emph{skip} of $\pi_j$ as
\[
   \skp(\pi_j)=\max\{|\x(u)|-\x(p_j)|,|\y(u)-\y(p_j)|\} \enspace .
\]
Observe that, if $\skp(\pi_j)=r$, then $\pi_j$ contains at most $2k/r$
points other than $u$ and $w$ and therefore contains at most $(k/r)^2$
edges that cross $uw$.  Furthermore, the number of planes $\pi_j$ such
that $\skp(\pi_j)=r$ is at most $4r+1$.  Therefore, the total number
of edges that cross $uw$ is at most
\begin{equation}
   \sum_{r=1}^k (4r+1)(k/r)^2 \le 4k^2\ln k + O(k) \enspace . \eqlabel{critical}
\end{equation}
Since this is true for each of the $k^4$ edges, $uw$, we conclude that
the total number of crossings in this drawing of $K_{k^2,k^2}$ is at
most $4k^6\ln k+O(k^3)\in O(k^6\log k)$, as required.
\end{proof}

%We finish this section with a strengthening of \thmref{upper-bound}
%that will be used to prove a lower bound on the number of non-crossing
%graphs that can be drawn on the $\sqrt{n}\times\sqrt{n}\times 2$ grid.
%Let $G_n$ denote the drawing of the bipartite graph $K_{n,n}$ on the
%$\sqrt{n}\times\sqrt{n}\times 2$ grid, as described in the proof of
%\thmref{upper-bound}.
%Define the \emph{length} of an edge $uw$ in $G_n$ as
%\[
%   \len(uw) = \max\{|\x(u)-x(w)|, |y(u)-y(w)|\} \enspace .
%\]
%Note that this definition of length defines a partial order on the edges
%of $G_n$. Extend this to a total order by breaking ties arbitrarily so
%that for any two edges we can refer to the longer or the shorter of the
%two without ambiguity.  For any integer $M\in\{0,\ldots,n^2\}$,
%let $G_{n,M}$ denote the subgraph of $G_n$ that contains only the $M$
%shortest edges of $G_n$.
%
%\begin{lem}\lemlabel{gross-upper-bound}
%  Each edge of $G_{n,M}$ crosses at most $O((n/M)\log(n/M))$ other edges.
%\end{lem}
%
%\begin{proof}[Proof sketch]
%The proof is a straightforward generalization of the proof of
%\lemref{2-layer}.  The number of edges in $G_n$ whose length is at
%most $r$ is $\Omega(nr^2)$.  Therefore, the longest edge in $G_{n,M}$
%has length $O(\sqrt{M/n})$.
%
%Observe that, if two crossing edges are contained in a plane $\pi_j$
%with $\skp(\pi_j)=r$, then at least one of the two edges has length at
%least $r$.  Therefore, when counting crossings in $G_{n,M}$,
% it is unnecessary to consider any plane $\pi_j$ with $\skp(\pi_j)>
% c\sqrt{M/n}$.  In this way the sum, \eqref{critical}, which counts the
% maximum number of edges intersecting any other edge, becomes
%\[
%   \sum_{r=1}^{c\sqrt{M/n}} (4r+1)(k/r)^2 \le 4c(M/n)\ln(M/n) + O(M/n)  \enspace .
%\qedhere
%\]
%\end{proof}

\section{Higher Dimensions}

\begin{thm}
  For all $m\ge 2(2^d-1)n$, $g_d(n,m)\in\Omega(m^2/n)$.
\end{thm}

\begin{proof}
  The proof is similar to the simplified proof of the Crossing Lemma
  due to Chazelle, Sharir, and Welzl \cite{S}.  We begin by arguing that
  the preliminary inequality 
  \begin{equation}
    g_d(n,m) \ge m - (2^d-1)n  \eqlabel{prelim}
  \end{equation}
  holds for all $m \ge 0$.  To see this, recall that Czyzowicz \etal\
  \cite{s} showed that the maximum number of edges in a non-crossing
  drawing on a $d$-dimensional grid of volume $n$ is no more than
  $(2^d-1)n$.  Therefore, in any drawing of a graph with $m> (2^d-1)n$
  edges, we can always remove an edge that eliminates at least one
  crossing.  This can be repeated $m-(2^d-1)n$ times, which proves that
  the number of crossings is at least $m-(2^d-1)n$.

  Next, let $G$ be any graph with $m$ edges and consider the drawing
  of $G$ that attains $\crs_d(G)$.  Randomly sample each edge of $G$
  independently with probability $p$ to obtain a random subgraph $G'$
  of $G$.  The expected number of crossings in the resulting drawing of
  $G'$ is then
  \[  p^2\crs_d(G) \ge \E[\crs_d(G')] \ge pm - (2^d-1)n \enspace ,\] 
  where the second inequality follows from the preliminary bound
  in \eqref{prelim}.  Rewriting this, we obtain
  \[  \crs_d(G) \ge m/p - (2^d-1)n/p^2 \enspace . \]
  Setting $p=2(2^{d}-1)n/m$ yields
  \[  \crs_d(G) \ge \frac{m^2}{4(2^{d}-1)n} \enspace .  \qedhere \]
\end{proof}


\begin{thm}
  $g_d(n,m)\in O(m^2/n)$.
\end{thm}

\begin{proof}
The proofs is like the proof of \thmref{3d-upper-bound}, but different.
\end{proof}

\section{The Number of Non-Crossing Graphs}

Next we show that the number of non-crossing subgraphs that can be drawn 
on the $n^{1/d}\times \cdots\times n^{1/d}\times 2$ grid is at
least $2^{\Omega(n\log n)}$.  A corresponding upper bound on the number of
non-crossing graphs that can be drawn on any grid of volume $n$ follows
from the fact that any such graph has at most $(2^d-1)n$ edges (Czyzowicz
\etal\ \cite[Lemma Y]{X}); so any non-crossing graph corresponds to one
of the ways of choosing at most $(2^d-1)n$ edges from among the $\binom{n}{2}$
possible edges.  Therefore, the number of such graphs is at most
\[
    2^{(2^d-1)n}\binom{\binom{n}{2}}{(2^d-1)n} \le 2^{(2^d-1)n} (n^{2})^{(2^d-1)n} = 2^{2(2^d-1)n\log n+(2^d-1)n}
\]

We now prove a lower-bound of the same form:

\begin{thm}
  There are $2^{\Omega(n\log\log n)}$ subgraphs of $G_n$ that have
  no crossings.
\end{thm}

\begin{proof}
For a geometric graph $G$, let $\ncs(G)$ denote the number of non-crossing
subgraphs of $G$ and define
\[
   f(m,M) = \min\{ \ncs(G) : \mbox{$G$ is a subgraph of $G_{n,M}$ with $m$ edges} \}
   \enspace . 
\]
Our goal is to lower-bound $f(n^2,n^2)$.  In order to do this, we first establish some preliminary recurrence inequalities and base cases.

For our base cases, we have 
\[ 
   f(m,M)\ge 1 \enspace ,
\]
for all $m\in\{0,\ldots,cN\}$ since the subgraph of $G_{n,M}$ with no
edges is non-crossing.

Fix any geometric subgraph, $G$, of $G_{n,M}$ that has $m> cn$
edges and select an edge $e$ from $G$. Then there are at least
$f(m-1,M)$ subgraphs of $G$ that do not include $e$.  Furthermore, by
\lemref{gross-upper-bound}, $e$ intersects at most $c(M/n)\log(M/n)$
edges of $G$, so there are at least $f(m-c(M/n)\log(M/n),M)$ subgraphs
of $G$ that include $e$.  Therefore,
\[  f(m,M) \ge f(m-1,M) + f(m-c(M/n)\log(M/n)) \enspace ,
\]
for $m\ge c(M/n)\log(M/n)$.  Repeatedly expanding the first term gives:
\begin{eqnarray*}
f(m,M) & \ge & f(m-1,M) + f(m-c(M/n)\log(M/n), M) \\
       & \ge & f(m-1,M) + f(m-2c(M/n)\log(M/n), M) \\
       & \ge & f(m-2,M) + 2f(m-2c(M/n)\log(M/n), M) \\
       & \ge & f(m-3,M) + 3f(m-2c(M/n)\log(M/n), M) \\
       & \vdots & \\
       & \ge & (c(M/n)\log(M/n))\times f(m-2c(M/n)\log(M/n), M) \enspace ,
\end{eqnarray*}
for $m\ge 2c(M/n)\log(M/n)$.

For an integer $t$, we can iterate the preceding $t$ times to obtain
\begin{equation}
   f(m,M) \ge (c(M/n)\log(M/n))^t\times f(m-2ct(M/n)\log(M/n), M)) \enspace ,
   \eqlabel{recur-a}
\end{equation} 
for $m\ge 2ct(M/n)\log(M/n)$.

Our next recurrence inequality is trivial: fix any subgraph, $G$,
of $G_{n,M}$ and observe that the subgraph, $G\cap G_{n,M/2}$, of $G$
that contains only the edges that appear in $G_{n,M/2}$ is a subgraph
of $G_{n,M}$ and this subgraph contains at most $M/2$ fewer edges.  Thus
\begin{equation}
  f(m,M) \ge f(m-M/2,M/2) \enspace .
  \eqlabel{recur-b}
\end{equation}

Our strategy is to prove that $f(n^2,n^2) \ge 2^{\Omega(n\log\log n)}$ by
repeatedly applying \eqref{recur-a} and \eqref{recur-b}.  Each application of
these inequalities gives reduced values of $m$ and $M$.  Let $m_i$ and $M_i$ be
these values, respectively, after applying these inequalities $i$ times each.  Initially $m_0=M_0=n^2$.  After $i$ iterations, we will have $m_i=n^2/i2^i$ and $M_i=n^2/2^i$.  The first thing to understand is the decrease in the value of $m$ from one round to the next (slow and careful):
\[
    m_1 - m_0 = n^2/2
\]
and, for $i \ge 2$, 
\begin{align*}
  m_{i-1}-m_i
      & = \frac{n^2}{(i-1)2^{i-1}} - \frac{n^2}{i2^{i}} \\
      & = \frac{2n^2}{(i-1)2^{i}} - \frac{n^2}{i2^{i}} \\
      & = \frac{2n^2}{(i-1)2^{i}} - \frac{n^2}{i2^{i}} \\
      & = \frac{2in^2}{i(i-1)2^{i}} - \frac{(i-1)n^2}{i(i-1)2^{i}} \\
      & = \frac{n^2(i+1)}{i(i-1)2^{i}} \\
      & = n^2/i2^i + \Theta(n^2/i^22^i)\\
\end{align*}
The difference $m_i-m_{i-1}$ comes from two parts.  One part, $n^2/2^i$,
comes from applying \eqref{recur-b}.  The second part,
\[
     n^2/i2^i 
\]

\end{proof}

%\begin{proof}
%Consider the following the method of enumerating some of the subgraphs
%of $G_n$, which takes as input a subgraph, $G_{n,M}$, of $G_n$.
%We select an 
%
%
% that has $m$
%edges and which .  We select the longest edge
%
%
%We get the recurrence
%\[ f(m) \ge  
% \begin{cases} 1 & \text{if $0< m < Cn$} \\
%               f(m-2) + f(m-c(m/n)\log(m/n)) & \text{otherwise} 
% \end{cases}
%\]
%If we focus on the second case, we can expand the first term
%$c(m/n)\log(m/n)$ times:
%\begin{align*}
%   f(m) & \ge f(m-2) + f(m-2c(m/n)\log(m/n)) \\
%        & \ge f(m-4) + 2f(m-2c(m/n)\log(m/n)) \\
%        & \ge f(m-6) + 3f(m-2c(m/n)\log(m/n)) \\
%        & \,\,\vdots \\
%        & \ge (cm/n)\log(m/n)\times f(m-2c(m/n)\log(m/n))
%\end{align*}
%By iterating this process $x=n/(2c\log(m/n))$ times, we obtain
%\[
%   f(m) \ge ((cm/n)\log(m/n))^x\times f(m/2)
%\]
%Next, notice that
%\begin{align*}
%((cm/n)\log(m/n))^x 
%   &= (c\log(m/n))^x2^{\log(m/n)x} \\
%   &= (c\log(m/n))^x2^{n/2c} \\
%   & \ge 2^{n/2c}
%\end{align*}
%for $m>n2^{1/c}$.  Thus far, we have
%\[
%   f(m) \ge 2^{n/2c}\times f(m/2) \enspace .
%\]
%We can now iterate this process $\alpha\log(m/n)$ times to obtain
%\[
%   f(m) \ge 2^{\alpha n\log(m/n)/2c}\times f(O(n)) 
%        \ge 2^{\alpha n\log(m/n)/2c} 
%        = 2^{\Omega(n\log n)}
%\]
%for $m=n^2$, as required.
%\end{proof}
%
%
%BLAAAHAAHAHAHAHAHAH
%
%Now we prove, by induction on $m$ that 
%\[  \log(f(m)) \ge \alpha(m/n)\log(m/n) \]
%for some constant $\alpha > 0$.  The base cases occur when $m<cn$, in
%which case $\log(f(m))=1>\alpha c\log c$ for sufficiently small $\alpha$.
%For the inductive step, we have
%\begin{align}
%  \log(f(m)) 
%  & \ge \log(cm/2n)\log(m/n))+\log(f(m-(2cm/n)\log(m/n))) \\
%  & \ge \log(cm/2n)\log(m/n))+\alpha\left(\frac{m-\frac{2cm\log(m/n)}{n}}{n}\right)\log\left(\frac{m-\frac{2cm\log(m/n)}{n}}{n}\right) \\
%  & = \log(2cm/2n)\log(m/n))
%        +\alpha\left(\frac{m}{n} 
%                      - \frac{2cm\log(m/n)}{n^2}\right)
%        \log\left(\frac{m-\frac{2cm\log(m/n)}{n}}{n}\right) 
%\end{align}

\section{Summary}
\seclabel{summary}


Open problem: our lower-bound argument fails for $d\ge 4$.

\section*{Acknowledgement}

The work of Pat Morin was partly funded by NSERC and CFI.

\bibliographystyle{plain}
\bibliography{template}


\end{document}


