\documentclass[12pt]{elsarticle}

\newcommand{\etal}{et al.}


\begin{document}
\begin{frontmatter}
\title{Guest Editor's Introduction}
\author{Pat Morin}
\end{frontmatter}

I am pleased to introduce the selected papers from the Twentieth Canadian
Conference on Computational Geometry (CCCG~2008), which took place
August~13--15, 2008.    Like the first and tenth CCCG, the twentieth
CCCG was held in Montreal at McGill University, with invited talks by
Michael~Shamos and Dmitry~Tymoczko and a Paul Erd\"os Memorial Lecture
by Ron~Graham.

The conference received 72 submissions, of which 53 were accepted for
presentation at the conference, and (after a lengthy review process)
4 were welcomed into this special issue.  These four papers represent
a nice cross-section of results from the field:

\begin{enumerate}
\item Gupta \etal\ study a new class of range searching problems that
ask for concise summary information, such as the diameter, the width,
or the closest pair, about the points in a query range.

\item Bhattacharya and Shi show that a facility location problem on
graphs can be transformed into a classic computational geometry problem:
Klee's measure problem.

\item Aloupis \etal\ marry two time-honoured research areas, triangulations and guarding, to show that triangulating a simple polygon can be done with a simple algorithm if that polygon can be guarded with few guards.

\item Aloupis and a different \etal\ study a fun problem that any child with a bath toy can understand:  Given a polygon that has been filled with water, where should holes be placed so that, after some rotations, we can get all the water out.
\end{enumerate}
I hope that the reader has as much fun reading this special issue as I
had preparing it.



\end{document}


