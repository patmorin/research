\documentclass[12pt]{article}
\usepackage{amsopn,amsthm,amsmath}

\newtheorem{thm}{Theorem}
\DeclareMathOperator{\cn}{cr}

\begin{document}
\begin{thm}
Let $S$ be a set of $r$ points and let $H_0,\ldots,H_{t-1}$ be a
set of lines such that the orthogonal projection of $S$ onto $H_{j}$
gives exactly $i+j$ distinct values, for each $j\in\{0,\ldots,t-1\}$.
Then $t \le 276.15i+1$.
\end{thm}

\begin{proof}
The proof is by contradiction.  We will assume that $t > ci$ for some
constant $c$ to be defined later.
First we observe that the number of points, $r$, satisfies 
\[   r \ge i+t-1 
\]
since the points of $S$ project onto $i+t-1$ distinct values in $H_{t-1}$.
Furthermore, the projections of $S$ onto $H_0$ and $H_1$ imply that the
points of $S$ are contained in the intersection of two sets of parallel
lines $L_0$ and $L_1$, with $|L_0|=i$ and $|L_1|=i+1$, so
\[
   r \le i(i+1) \enspace .
\]
Each projection direction $H_j$ defines a set $L_j$ of $i+j$ parallel lines,
each of which contains at least one point of $S$.  In total, this defines
a set $L=L_0\cup\cdots\cup L_{t-1}$ of
\[
   \ell(t) = \sum_{j=0}^{t-1} (i+j) = ti + t(t-1)/2 \le ti + t^2/2 \enspace .
\]
lines.  The total number of incidences between the these lines and
the points in $S$ is $tr$.

Define a geometric graph $G$ having $r+2$ vertices consisting
of the points in $S$ plus two points ``at infinity,'' as in
Figure~\ref{fig:G}. Observe that $G$ is a graph with no self-loops and no
parallel edges. The number of
edges in $G$ is exactly $tr + \ell(t)$.
The Crossing Lemma \cite{XX} then implies that, either
\[
  tr +\ell(t) \le 7.5(r+2)
\]
or that
\begin{equation}
  \cn(G) \ge \gamma\cdot\frac{(tr+\ell(t))^3}{(r+2)^2} \enspace . \label{eq:cn}
\end{equation}
In the former case, we readily establish that $t \le i$,
so we may assume that Eq.~(\ref{eq:cn}) holds.

Observe that we have a drawing of $G$ so that the only crossings between
edges occur where lines in $L$ intersect each other.  The total number of
intersecting pairs of lines in $L$ is 
\[
  \begin{aligned}
    \chi(t) 
      & = \sum_{j=1}^{t-1}(i+j)\cdot\sum_{k=0}^{j-1}(i+k) \\
      & \le \sum_{j=1}^{t-1}(i+j)(ij + j^2/2) \\
      & \le \sum_{j=1}^{t-1}(i^2j+3ij^2/2 + j^3/2) \\
      & \le i^2t^2/2 + it^3/2 + t^4/8 \enspace .
  \end{aligned}
\]
%However, each point of $S$ eliminates $\binom{t}{2}$ of these intersecting
%pairs.  Therefore, the crossing number $\cn(G)$ of $G$ satisfies
%\[
%  \cn(G) \le \chi(t) - \binom{t}{2}r = i^2t^2/2 + it^3/2 + t^4/8 - \binom{t}{2}r \enspace .
%\]
Combining this with Eq.~\ref{eq:cn} and rewriting, we obtain 
\[
  \chi(t)  \ge \gamma\cdot\frac{(tr+\ell(t))^3}{(r+2)^2} \enspace .
\]
Let $G_{s}$, $s\in\{1,\ldots,t\}$, be defined like $G$, but only with
respect to the set of lines in $L_0,\ldots,L_{s-1}$.  Then, by the same reasoning as above,
\[
  \chi(s) \ge \gamma\cdot\frac{(sr+\ell(s))^3}{(r+2)^2} 
\]
for all $s\in\{0,\ldots,t-1\}$.

Select $s=di$ and $t=ci$ for some constants $1 < d < c$ to be determined
later.
Rewriting, we get
\[
  \begin{aligned}
  s 
   &\le \frac{\chi(s)^{1/3} (r+2)^{2/3} - \ell(s)}{\gamma^{1/3}r} \\ 
   &\le \frac{\chi(s)^{1/3} (r^{2/3}+2^{2/3}) - \ell(s)}{\gamma^{1/3} r} \\
   & = \frac{\chi(s)^{1/3}r^{2/3}+\chi(s)^{1/3}2^{2/3} - \ell(s)}{\gamma^{1/3} r}  \\
   & = \frac{\chi(s)^{1/3}r^{2/3}+(2\ell(s))^{2/3} - \ell(s)}{\gamma^{1/3} r} 
    & \mbox{ (since $\chi(s) \le \ell(s)^2$)} \\
   &\le \frac{\chi(s)^{1/3}}{(\gamma r)^{1/3}}
    & \mbox{ (for sufficiently large $\ell(s)$)} \\
   &\le \frac{(i^2s^2/2 + is^3/2 + s^4/8)^{1/3}r^{2/3}}{\gamma^{1/3} r}  \\
   &\le \frac{((is)^{2/3}/2^{1/3} + i^{1/3}s/2^{1/3} + s^{4/3}/2)}{(\gamma r)^{1/3}}  \\
   &\le \frac{(d^{2/3}/2^{1/3}+d/2^{1/3}+d^{4/3}/2)i^{4/3}}{(\gamma r)^{1/3}}  
    & \mbox{ (since $s=di$)} \\
   & =  \frac{(d^{2/3}/2^{1/3}+d/2^{1/3}+d^{4/3}/2)i^{4/3}}{(\gamma(i+t-1))^{1/3}} \\ 
   &\le  \frac{(d^{2/3}/2^{1/3}+d/2^{1/3}+d^{4/3}/2)i^{4/3}}{(\gamma(1+c-1/i)i)^{1/3}}  
    & \mbox{ (since $t=ci$)} \\
   & =  \frac{(d^{2/3}/2^{1/3}+d/2^{1/3}+d^{4/3}/2)i}{(\gamma(1+c-1/i))^{1/3}} \\ 
   & < di = s
  \end{aligned}
\]
for any $c > ((d^{2/3}/2^{1/3}+d/2^{1/3}+d^{4/3}/2)/d)^3/\gamma-1+1/i$.
Selecting $d=2$ and $c=276.15+1/i$ completes the proof.
\end{proof}



\end{document}
