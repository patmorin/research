\documentclass{article}
\usepackage{fullpage}

\newcommand{\comment}[1]{}

\setlength{\parskip}{1.5ex}
\newenvironment{comm}{\noindent\textbf{Referee's Comment:}}{}
\newenvironment{resp}{\noindent\textbf{Authors' Response:}}{}

\title{Revision Document:\\
  Space-Efficient Planar Convex Hull Algorithms}

\begin{document}
\maketitle

\section{Referee C}

\noindent\textbf{Note:} Referee C's comments have been paraphrased
since I am unable to cut and paste them from a postscript file.

\begin{comm}
You need to define the convex hull problem in such a way that Shamos'
lower bound holds.
\end{comm} 

\begin{resp}
We now define the convex hull problem as that of outputting the
convex hull vertices in the order in which they appear on the hull.
\end{resp} 

\begin{comm}
It is conceivable that an algorithm produces the ouput point by point
and prints it to a file.
\end{comm} 

\begin{resp} 
This is true, but I'm not sure what I should say about it.
\end{resp} 

\begin{comm}
$S[h-1]$ should be $S[h-2]$ in Figure 1.
\end{comm} 

\begin{resp}
Corrected.
\end{resp}
 
\begin{comm}
Lines 2 and 3 of Graham-InPlaceHull can implemented a bit more cheaply
by shifting and moving $a$.
\end{comm} 

\begin{resp} 
This is true, but we wrote all our algorithms in terms of swaps for
the sake of consistency and brevity.  It should be clear to anyone who
implements this algorithm that they can do this.
\end{resp} 

\begin{comm}
``further reduce'' This is strange because so far no constants have
been reduced.
\end{comm} 

\begin{resp} 
In the previous paragraph we show how to reduce the constants in the
algorithm by first partitioning the set into upper hull and lower hull
candidates.  In fact, the previous paragraph begins with the sentence
``We can reduce the constants in Graham-InPlace-Hull by\ldots
\end{resp} 


\begin{comm}
The definitions of $S^i$ are different from those in the original
paper.  
\end{comm} 

\begin{resp}
The definitions of the $S^i$ have been changed so that we can have
$S^0$, $S^1$ and $S^2$ appear in that order in the output array.  If
we stuck with Chan's definition we would have $S^0$ followed by $S^2$
followed by $S^1$.  Since this is more a matter of taste than anything
else we did not change to Chan's original notation.
\end{resp}


\begin{comm}
The notation $(a,b)$ is confusing; is it the line segment from
$a$ to $b$ or the line through $a$ and $b$?
\end{comm} 

\begin{resp} 
We now define the notation in the previous paragraph.
\end{resp} 

\begin{comm}
This is just an incarnation of the Dutch National Flag problem and can be solved much more simply.
\end{comm} 

\begin{resp} 
Referee A also noticed this (but didn't tell us the name Dutch
National Flag).  We updated the write-up with a simple algorithm and
reference to the DNF problem.
\end{resp} 

\begin{comm}
Why $3n/4$ and not $n/2$?
\end{comm} 

\begin{resp}
This is a good observation.  We wrote $S^0\le 3n/4$ but the referee is correct in pointing out that we actually have the stronger condition $S^0\le n/2$.  This has been updated.
\end{resp} 

\begin{comm}
I don't see where in-place comes from.  Megiddo's algorithm uses
median finding which takes $O(\log n)$ extra space (so is not
in-place).
\end{comm} 

\begin{resp}
There are in-place median finding algorithms.  We
now cite one of these (Lai and Wood).
\end{resp} 

\begin{comm}
Clarkson's algorithm looks like it is in-place and finds the $h$
extreme points in $O(nh)$ time.
\end{comm} 

\begin{resp} 
We now discuss Clarkson's algorithm in the conclusion.
\end{resp} 

\comment{

\section{Referee A}

\textbf{Note:} There are some typos in referee A's comments due to
cutting and pasting from a PDF file.

\begin{comm}
Graham s scan is basically already in-place, and I am sure some of the
existing implementations are in-place (e.g.,
http://cm.bell-labs.com/cm/cs/who/clarkson/2dch.c), so this section is
mostly a review . 
\end{comm} 

\begin{resp} 
We have added a sentence in the introduction observing that Graham's
scan is inherently in-place (once the sorting is done).
\end{resp} 

\begin{comm}
Concerning the issue on how to divide into/combine
upper and lower hulls (where the paper presents two options), here is
a third option: don t! The original version (or the version in Cormen,
Leiserson, and Rivest, instead of Andrew s paper) sorts by angle
instead of x-coordinates and computes the entire convex hull at
once. The main disadvantage is that the number of right turn tests is
\end{comm}

\begin{resp}
We have added a paragraph discussing this option at the end of Section~2.
\end{resp}

\begin{comm}
First, in the equation, it should be mentioned that the notation                    !  refers to the line segment between the two points, not the line (this causes some confusion on my first reading).
\end{comm}

\begin{resp}
Done.
\end{resp}

\begin{comm}
Second, I may have missed a subtle point, but I wonder whether the
partitioning into can be done more simply. 
\end{comm}

\begin{resp}
This is an excellent idea that we missed.  It is much simpler than the
method we proposed.  The old algorithm has been replaced with this
method.
\end{resp}

\begin{comm}
Third paragraph: Unlike the previous algorithm, partitioning . . . is
trivial.  Are we using quicksort s partition twice here?
\end{comm}

\begin{resp}
Two applications of quicksort's partition would work. Another option
is to use two scans to first compute the sizes ofthe three groups and
then to move the elements to their correct locations.
\end{resp}

\begin{comm}
Next paragraph: Period after a question mark (finally a typo!).
\end{comm}

\begin{resp}
Fixed.
\end{resp}

\begin{comm}
As a matter of personal taste, it seems better to stay in primal space
for the bridge algorithm, rather than switching to the dual. One
reason is that the parallelism between CSY and KS is currently lost
(that both CSY and Megiddo uses pairing is no coincidence). Otherwise,
at least change 2D linear programming to 2D separated linear
programming in Theorem 5 (it probably holds for general 2D linear
programming, but hasn t been fully demonstrated).
\end{comm}

\begin{resp}
The statement of Theorem 5 has been changed to separated 2D linear
programming.
\end{resp}

\begin{comm}
7th paragraph: The point ( �)+*, -� is on the convex hull of ( if and
only if ( .)/*102 is on the convex hull of ( and the triangle (
�)/*304�5 ( .)6*, -7 ( 82 is oriented clockwise.  Generally, I am a
little unclear as to what the overall test looks like. This particular
sentence is confusing. By definition of ) , ( �)9*1 - is always on the
convex hull. Rather, if I understand correctly, this sentence refers
to testing whether a given index is equal to ):*; . Secondly, the
condition ( .)<*304 is on the convex hull of ( is not intended to be
tested recursively, I hope.
\end{comm}

\begin{resp}
This entire paragraph has been rewritten in a cleaner way and a figure
has been added that illustrates the test.
\end{resp}

\begin{comm}
 Experimentally, the speedup of the in-place versions
 . . . varies. . . For Graham s scan, we typically find a speedup of
 roughly 15\% compared to the non-in-place version.  I was a little
 intrigued by this statement. What non-in-place version of Graham s
 scan are we comparing against? To me, most sufficiently careful
 implementation of Graham s scan is already in-place. Are we storing
 the stack externally? Is the extra cost due to copying (which causes
 only a difference in the linear term)? Or is the extra cost a result
 of the =>�@?A -line trick of Section 2.2 (in which case the speedup
 is not so much due to in-place vs. non-in-place)? The reason for
 picking on this statement is that generally I would expect a
 nontrivial in-place algorithm not to have a speedup but a slow-down
 (because we are trading off space for time). For example, the in-situ
 CSY algorithm (which uses in-place selection) would probably be
 slower.

I was unable to locate [3] for the timing results.
\end{comm}

\begin{resp}
The discussion on timing results has been removed.
\end{resp}

\begin{comm}
In the last sentence, smallest enclosing disks can be solved by Welzl
s variant on Seidel s algorithm, which would appear to be in-place.
\end{comm}

\begin{resp}
Of course.  Smallest enclosing disk was removed from the list of open
problems.
\end{resp}

\section*{Referee B}

\begin{comm}
Last line of abstract: I love wit as much as anyone, but the abstract
is not the place for it.
\end{comm}

\begin{resp}
The last line of the abstract was removed.
\end{resp}

\begin{comm}
p 1: mention that the model of computation must allow the evaluation
of quadratic polynomials (not needed for sorting).
\end{comm}

\begin{resp}
Done.
\end{resp}

\begin{comm}
p 2: any relevance to I/O efficient, external-memory algorithms?
\end{comm}

\begin{resp}
One might hope that, because space-efficient algorithms only store a
small amount of extra data (including pointers) that they would
naturally be IO efficient. Unfortunately, space-efficient algorithm
may do pointer arithmetic that eliminates any locality of reference
one might hope for (e.g., Chan's $O(n\log h)$ algorithm).
\end{resp}

\begin{comm}
p 5: discuss the relevance of stability in partitioning.
\end{comm}

\begin{resp}
Done.
\end{resp}

\begin{comm}
p 8: 2 observations -> two observations
\end{comm}

\begin{resp}
This section has been removed.
\end{resp}

\begin{comm}
p 9: in to -> into
\end{comm}

\begin{resp}
Changed.
\end{resp}

\begin{comm}
Why is in situ in italics? 
\end{comm}

\begin{resp}
Because it's Latin. In my dictionary (The Oxford Canadian) it is also
printed that way.
\end{resp}

\section*{Referee C}

\begin{comm}
In the statement of the Theorem 3 you say that no 
lexicographic comparisons are used, but to obtain
this result you write in the previous paragraph that you use
a partitioning algorithm which performs O(n) comparisons.
It could be helpful for some readers if you write that 
these comparisons are performed as right turn tests. 
\end{comm}

\begin{resp}
Done.
\end{resp}
}

\end{document}
