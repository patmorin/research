\documentclass[ccfonts,lotsofwhite]{patmorin}

 
%\usepackage{amsthm}

\newcommand{\centeripe}[1]{\begin{center}\Ipe{#1}\end{center}}
\newcommand{\comment}[1]{}

\newcommand{\centerpsfig}[1]{\centerline{\psfig{#1}}}

\newcommand{\seclabel}[1]{\label{sec:#1}}
\newcommand{\Secref}[1]{Section~\ref{sec:#1}}
\newcommand{\secref}[1]{\mbox{Section~\ref{sec:#1}}}

\newcommand{\alglabel}[1]{\label{alg:#1}}
\newcommand{\Algref}[1]{Algorithm~\ref{alg:#1}}
\newcommand{\algref}[1]{\mbox{Algorithm~\ref{alg:#1}}}

\newcommand{\applabel}[1]{\label{app:#1}}
\newcommand{\Appref}[1]{Appendix~\ref{app:#1}}
\newcommand{\appref}[1]{\mbox{Appendix~\ref{app:#1}}}

\newcommand{\tablabel}[1]{\label{tab:#1}}
\newcommand{\Tabref}[1]{Table~\ref{tab:#1}}
\newcommand{\tabref}[1]{Table~\ref{tab:#1}}

\newcommand{\figlabel}[1]{\label{fig:#1}}
\newcommand{\Figref}[1]{Figure~\ref{fig:#1}}
\newcommand{\figref}[1]{\mbox{Figure~\ref{fig:#1}}}

\newcommand{\eqlabel}[1]{\label{eq:#1}}
\newcommand{\eqref}[1]{(\ref{eq:#1})}

\newtheorem{thm}{Theorem}{\bfseries}{\itshape}
\newcommand{\thmlabel}[1]{\label{thm:#1}}
\newcommand{\thmref}[1]{Theorem~\ref{thm:#1}}

\newtheorem{lem}{Lemma}{\bfseries}{\itshape}
\newcommand{\lemlabel}[1]{\label{lem:#1}}
\newcommand{\lemref}[1]{Lemma~\ref{lem:#1}}

\newtheorem{cor}{Corollary}{\bfseries}{\itshape}
\newcommand{\corlabel}[1]{\label{cor:#1}}
\newcommand{\corref}[1]{Corollary~\ref{cor:#1}}

\newtheorem{obs}{Observation}{\bfseries}{\itshape}
\newcommand{\obslabel}[1]{\label{obs:#1}}
\newcommand{\obsref}[1]{Observation~\ref{obs:#1}}

\newtheorem{assumption}{Assumption}{\bfseries}{\rm}
\newenvironment{ass}{\begin{assumption}\rm}{\end{assumption}}
\newcommand{\asslabel}[1]{\label{ass:#1}}
\newcommand{\assref}[1]{Assumption~\ref{ass:#1}}

\newcommand{\proclabel}[1]{\label{alg:#1}}
\newcommand{\procref}[1]{Procedure~\ref{alg:#1}}

\newtheorem{rem}{Remark}
\newtheorem{op}{Open Problem}

\newcommand{\etal}{\emph{et al}}

\newcommand{\voronoi}{Vorono\u\i}
\newcommand{\ceil}[1]{\left\lceil #1 \right\rceil}
\newcommand{\floor}[1]{\left\lfloor #1 \right\rfloor}



\title{\MakeUppercase{New Data Structures for Mass Finding}}
\author{Pat Morin \\
	\emph{School of Computer Science, Carleton University} \\
	\texttt{morin@cs.carleton.ca}}
\date{}

\begin{document}
\maketitle
\begin{abstract}
Given a sequence $W=w_1,\ldots,w_n$ of positive real numbers, we show
how to preprocess $W$ so that, given a query value $M$ we can
determine if there exists indices $i$ and $j$, $1\le i\le j\le n$ such
that $M=\sum_{k=i}^j w_k$.  We describe a simple data structure that
uses $O(n^{1.465})$ space and answers queries in $O(\log n)$ time.  We
show that this generalizes to a data structure that uses
$O(n^{1+\epsilon})$ space and $O(\log n)$ query time, for any constant
$\epsilon > 0$.  
\end{abstract}

\section{Introduction}

We consider the following problem, which seems to have been introduced
by Cieliebak \etal\ \cite{cels02,cels01b} in the context of
computational biology.  Given a sequence $W=w_1,\ldots,w_n$ of
positive real numbers, preprocess $S$ so that, given a query value $M$
one can quickly determine if there exists indices $i$ and $j$, $1\le
i\le j\le n$ such that $M=\sum_{k=i}^j w_k$.

The quality of a solution to this problem is measured by three
quantities: The \emph{storage requirement} $S(n)$ measures the amount
of memory required by the preprocessing and query algorithms and the
\emph{query time} $Q(n)$ measures how long it takes to answer a query.

Cieliebak \etal\ point out two trivial solutions to this problem.  The
first has $S(n)=O(n)$ and $Q(n)=O(n)$. That is, the algorithm simply
tries to find a subsequence $w_i,\ldots,w_j$ whose sum is $M$ using a
linear time algorithm.  The second trivial solution is simply to
compute all the $n\choose 2$ possible sums obtained from contiguous
subseqences of $S$ and store these in dictionary.  In this way, we
obtain $S(n)=O(n^2)$ and $Q(n)=O(\log n)$.  The challenge, therefore,
is to find algorithms in which $S(n)=o(n^2)$ and $Q(n)=o(n)$.

In the context of computational biology, the Cieliebak argue that the
number of distinct values $m$ appearing in $W$ is $O(1)$.  This is
because these values represent the weights of amino acids, of which
only 20 or so are known to exist.  Under this assumption, the authors
give a solution with $S(n)=O(n)$ and $Q(n)=O(n/\log n)$.  For the
special case of binary alphabets, i.e., when the number of distinct
values in $W$ is $2$, the same authors give a solution with
$S(n)=O(n)$ and $Q(n)=O(\log n)$.

\begin{table}
\begin{center}
\begin{tabular}{|l|l|l|l|l|}\hline
$S(n)$ & $Q(n)$ & Restriction on $m$ & Reference \\ \hline\hline
$O(n)$ & $O(n)$ & none & \cite{cels02b} \\
$O(n^2)$ & $O(\log n)$ & none & \cite{cels02b} \\
$O(n)$ & $O(n/\log n)$ & $m=O(1)$ & \cite{cels02b} \\
$O(n)$ & $O(\log n)$ & $m=2$ & \cite{cels02b} \\
$O(n^{1.465})$ & $O(\log n)$ & none & here \\ \hline
\end{tabular}
\end{center}
\tablabel{results}
\caption{Summary of old and new results.}
\end{table}

\tabref{results} summarizes the results of Cieliebak \etal\
\cite{cels02b,cels01b} as well as the new result obtained here.  Here
we give a solution with $S(n)=O(n^{\log_3 5})=o(n^{1.465})$ and
$Q(n)=O(\log n)$.  Thus, we show that logarithmic query time is
possible using subquadratic space, even when the number is distinct
weights is unbounded.  Furthermore, the algorithm is quite simple and
the big-Oh notation doesn't hide large constant factors.



\section{A Data Structure for Mass Finding}

\section{Conclusions}

\bibliography{mass}
\bibliographystyle{plain}


\end{document}