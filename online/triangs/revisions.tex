\documentclass{article}
\usepackage{fullpage}

\newcommand{\comment}[1]{}

\setlength{\parskip}{1.5ex}
\newenvironment{comm}{\noindent\textbf{Referee's Comment:}}{}
\newenvironment{resp}{\noindent\textbf{Authors' Response:}}{}

\title{Revision Document:\\
  Online Routing in Triangulations}

\begin{document}
\maketitle

\begin{comm}
In section 2, they claim that any determinisitc
algorithm is defeated by one of the graphs in figure 2.1.
This is hardly clear, and the authors must provide
a argument to back this up.
\end{comm} 

\begin{resp}
We modified this to make a smaller example and provided a short proof.
\end{resp} 

\begin{comm}
Last sentence of page 2: "in two ways".  I see no
reason why these two ways are singled out.
Why isn't the first way just a special case
of the second?  Is there any other possibility
besides the second way?
\end{comm} 

\begin{resp}
It is true that the first way is a specialization of the second, except
that usually, a set of 2 vertices isn't considered to be a simple
cycle.  We now point out that the first way is a special case of the second.
\end{resp} 

\begin{comm}
Page 3, Last paragraph: "initially believe".
I am not sure this remark is useful.  It seems to
me that the graphs in Figure 2.2 can already
be adapted to defeat compass routing.  If this is
true, I no longer see the need for Figure 2.4 (despite
the greater ingenuity needed for its construction).
\end{comm} 

\begin{resp}
We don't see this. Compass routing never fails to route to a vertex
that is on the convex hull of a triangulation (like the destination in
figure 2.2) since, in this case, it is actually an instance of the
simplex method for linear programming in 3d.
\end{resp}
 
\begin{comm}
Page 4, paragraph before Lemma 2.3: the definition
of the triangle uv seems to be incomplete or wrong.
\end{comm} 

\begin{resp} 
This is a good observation.  We forgot to mention that the triangle in
question has $uv$ as one of its edges.  This has been corrected.
\end{resp} 

\begin{comm}
Page 5, two sentences before Theorem 2.4: this
sentence refers to obscuring with respect to
some viewpoints.  But the definition of "obscures"
never refer to the concept of viewpoints?
\end{comm} 

\begin{resp}
We have modified the definition of ``obscures'' to specify that it is
always with respect to some viewpoint.
\end{resp} 


\begin{comm}
Page 6, sentence before Theorem 2.6: claiming that additional
memory is only "O(1)" is wrong, as explained earlier.
\end{comm} 

\begin{resp}
In the introduction, we state that the algorithm requires only enough
memory to store a constant number of vertex positions and then note
that we refer to such algorithms as using $O(1)$ memory.
\end{resp}


\end{document}
