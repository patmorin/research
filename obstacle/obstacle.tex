\documentclass{patmorin}
\usepackage{pat}
\usepackage{amsopn}

\DeclareMathOperator{\obs}{obs}

\title{On Obstacle Numbers}
\author{Bellairs Geometry and Graphs 2013}

\begin{document}
\maketitle

\begin{abstract}
The obstacle number of a graph is an interesting new graph parameter
introduced by So and So (200X).  Pach \etal\ (20XX) show that there
exist graphs with $n$ vertices having obstacle number in $\Omega(n/\log^2
n)$. In this paper we up this lower bound to $\Omega(n/\log n)$.
\end{abstract}

\section{Introduction}

The obstacle number of a graph is a new graph parameter introduced by
X and Y \cite{xy}.

Let $G=(V,E)$ be a graph, let $\varphi:V\to \R^2$ be a one-to-one
mapping of the vertices of $G$ onto $\R^2$, and let $S$ be a set of
connected subsets of $\R^2$.  The pair $(\varphi,S)$ is an \emph{obstacle
representation} of $G$ iff, for every pair of vertices $u,w\in V$, the
edge $\{u,w\}\in E$ if and only if the open line segment with endpoints
$\varphi(u)$ and $\varphi(w)$ does not intersect any \emph{obstacle}
in $S$.  The \emph{obstacle number} of a graph $G$ is the minimum number
of obstacles in any obstacle representation of $G$.

Pach \etal\ \cite{paXX} show that there exist graphs with obstacle
number $\Omega(n/\log^2 n)$.  In this paper, we show that there exist
graphs with obstacle number $\Omega(n/\log n)$.

\section{The Proof}

A \emph{quad} $(a,b,c,d)\in(\R^2)^4$ is a sequence of points such that
the polygon whose vertices, in counterclockwise order, are $(a,b,c,d)$
is simple.  The \emph{vertices} of a quad $q=(a,b,c,d)$ are the points
$a$, $b$, $c$, and $d$ and the \emph{edges} of $q$ are the 6 pairs
$\{x,y\}\in\binom{\{a,b,c,d\}}{2}$.  When clear from context, we will
sometimes treat a quad interchangeably with the simple polygon defined
by its vertices.

NOTE: Degenerate quads (with 3 or 4 collinear points need to be argued
about differently.  They still work (even better) because, for these,
it's impossible to have the edges $ab$, $bc$, $cd$, and $da$ without
having at least one of $ac$ or $bd$.

For a set $P\subset\R^2$, a \emph{quadset} of $P$ is a set
$Q=\{q_1,\ldots,q_k\}$ of quads whose vertices are points in $P$,
such that no two quads have a common edge, and such that, for each
$q_i,q_j\in \binom{Q}{2}$, the polygons determined by $q_i$ and $q_j$
have disjoint interiors.

\begin{lem}\lemlabel{quadsets}
  Let $P$ be any set of $n>4$ points in general position.  Then, for
  each $k\in\{1,\ldots,\lfloor n/4\rfloor\}$, there exists a set of $k^2$
  quadsets $Q_1,\ldots,Q_{k^2}$ such that
  \begin{enumerate}
    \item $|Q_i| \ge n/k$ for each $i\in\{1,\ldots,k^2\}$;
    \item For any two quads $p,q\in \bigcup_{i=1}^{k^2} Q_i$, $p$ and $q$
      have no edges in common.
  \end{enumerate}
\end{lem}

\begin{proof}[Proof Sketch]
  Without loss of generality, assume that no two points of $P$ have the
  same $x$-coordinate and denote the points of $P$ by $p_1,\ldots,p_n$
  in order of increasing $x$ coordinate.  Observe that we can obtain a
  quadset of size $\lfloor (n-1)/3\rfloor$ by using the sets
  \[
     \{p_1,p_2,p_3,p_4\}, \{p_4,p_5,p_6,p_7\},
        \ldots,\{p_{n-3},p_{n-2},p_{n-1},p_{n}\}
  \]
  and that these can be partitioned into $\approx k/3$ quadsets each of size at
  least $n/k$.  We call these the \emph{slab quadsets} of $P$.

  For two integers $i$ and $j$, let $P_{i,j}$ denote the subset of $P$
  given by
  \[
    P_{i,j} = \{ p_{ti+j} : t\in \{1,\ldots,\lfloor(n-j)/i\rfloor\} \}
  \]
  The set $P_{i,j}$ has size $\lfloor(n-j)/i\rfloor$ and therefore has
  $\approx k/3i$ slab quadsets.  To obtain the $k^2$ quadsets required
  by the lemma we take the slab quadsets for the point sets $P_{i,j}$,
  for sufficiently many values of $i$ that are not multiples of 2
  or 3.  For each choice of $i$, we use the slab quadsets from $P_{i,j}$
  for every $j\in\{0,\ldots,i-1\}$.  That is, the values of $i$
  are taken from the sequence $\langle 1,5,7,11,13,17,\ldots\rangle$
  of integers congruent to 1 or 5 modulo 6.  Thus, for each $i$, we obtain
  $\approx k/3$ quadsets.

  All that remains is to show that the quadsets obtained this way satisfy
  the second property (edge disjointness) required by the lemma.  To see
  why this property is satisified, define the \emph{rank} of an edge
  $p_xp_y$ as $|x-y|$.  Next, observe that the quadsets obtained from
  $P_{i,j}$ only have edges of ranks in $\{i,2i,3i\}$.  This implies
  that the quads generated from $P_{i',j'}$ for any $i'>i$ do not have
  any edges in common with the quads generated from $P_{i,j}$, since
  $\{i,2i,3i\}\cap\{i',2i',3i'\}=\emptyset$ (recall that neither $i$
  nor $i'$ is a multiple of 2 or 3).

  The only remaining possibility is that the edges of quads obtained
  from $P_{i,j}$ overlap with the edges of quads obtained from $P_{i,j'}$
  for some $j'\neq j$.  But this is certainly not possible since the point
  sets $P_{i,j}$ and $P_{i,j'}$ are disjoint.
\end{proof}

\begin{lem}
  Let $G=(V,E)$ be an Erd\"os-Renyi random graph $G_{n,\frac{1}{2}}$,
  let $P\subset\R^2$ be a set of $n$ points in general position, let
  $\varphi:V\rightarrow P$ be a bijection between $V$ and $P$ that is
  independent of the choices of edges in $G$, and let $(\varphi, S)$ be
  an obstacle representation of $G$ using the minimum number of obstacles
  (subject to $G$ and $\varphi$).  Then
  \[
     \Pr\{|S| < n/128k\} \le e^{-nk/512}
  \]
\end{lem}

\begin{proof}
  Let $Q_1,\ldots,Q_{k^2}$ be the quadsets of $P$ guaranteed by
  \lemref{quadsets} and let $G'$ be the geometric graph defined by
  the embedding, $\varphi$, of $G$ onto $P$.  

  We say that a quad, $q=(a,b,c,d)$ \emph{survives} in $G$ if the edges
  $ab$, $bc$, $cd$, and $da$ are in $G'$ and the edges $ac$ and $bd$ are
  not in $G'$.  Observe that, in any obstacle representation $(\varphi,S)$
  of $G$, there must be an obstacle contained in the interior of $q$:
  Some obstacle, $s$, must intersect the interior of $q$ since, otherwise
  at least one of the segments $ac$ or $bd$ do not intersect any obstacle;
  the obstacle $s$ must be contained in the interior $q$, since otherwise
  at least one of the segments $ab$, $bc$, $cd$, or $da$ intersects $s$.

  Consider some quadset, $Q_i$ and recall that $|Q_i|\ge n/k$.
  The probability that a particular quad in $Q_i$ survives in $G$ is
  exactly $1/64$ and, since no two quads share an edge, the number of
  quads in $Q_i$ that survive is a binomial$(|Q_i|,1/64)$ random variable.
  Let $\mathcal{E}_i$ denote the event ``less than $n/128k$ quads
  of $Q_i$ survive in $Q$.''  By Chernoff's Bound on the head of the
  binomial distribution,
  \[
    \Pr\{\mathcal{E}_i\}
      \le e^{-|Q_i|/512} \le e^{-n/512k} \enspace .
  \]

  If the event $\mathcal{E}_i$ does not occur then the minimum number of
  obstacles in $S$ is at last $n/128k$.  Finally, observe that the
  events $\mathcal{E}_1,\ldots,\mathcal{E}_{k^2}$ are indepedent, since
  none of the quads in $\bigcup_{i=1}^{k^2} Q_i$ have any edges in common.
  Therefore
  \begin{align*}
    \Pr\{|S|\le (\alpha/128)n/k\} 
      & \le \Pr\{ \mathcal{E}_1\cap \mathcal{E}_2\cap\cdots\cap\mathcal{E}_{k^2}\} \\
      & \le \left(e^{-n/512k}\right)^{k^2} \\
      & = e^{-nk/512} \enspace . \qedhere
  \end{align*}
\end{proof}

\begin{thm}
  Let $G=(V,E)$ be an Erd\"os-Renyi random graph $G_{n,\frac{1}{2}}$, then
  \[
    \Pr\left\{\obs(G) \le \frac{n}{65536(c+5)\ln n}\right\} 
         \le e^{-cn\ln n + o(cn\ln n)} \enspace .
  \]
  In particular, for any sufficiently large constant $c$, this probability
  is less than one, so there exists graphs having obstacle number
  $\Omega(n/log n)$.
\end{thm}

\begin{proof}
    The number of distinct order types of sets of $n$ points in the plane
  is $n^{4n+o(n)}$.  For each such order type the number of mappings of 
  the points of $G$ onto points of the order type is $n!$.  Therefore, the
  total number of order type/mapping pairs is at most
  \[
    n^{4n+o(n)}n! < n^{5n+o(n)} = e^{5n\ln n+o(n\ln n)} \enspace .
  \]
  On the other hand, \lemref{quadsets} states that the probability that
  any particular order type/mapping pair yields an obstacle representation
  with fewer than $n/128k$ obstacles is at most
  \[
     e^{-kn/512} \enspace .
  \]
  Therefore, by the union bound, the probability that there exists any
  obstacle representation of $G$ using fewer than $n/128k$ obstacles is
  at most
  \[
     e^{5n\ln n+o(n\ln n)-kn/512} .
  \]
  Taking $k=512(c+5)n\ln n$ completes proof.
\end{proof}

NOTE: The constant 65536 can be improved considerably.  By arguing about
quads a little more carefully and using $G_{n,\frac{4}{5}}$,
the number of quads becomes a binomial random variable with probability
$4^4/5^5\approx 1/12$.

NOTE: Using order types in the proof of the preceding theorem is overkill.
A simpler form of order-type that defines only the orientations important
for our quadsets could be used instead.  There are only $e^{O(nk)}$ such
order types.  Unfortunately, this doesn't get us above $\Omega(n/\log
n)$ because there are still $n!$ ways of embedding the graph onto the
order type.

NOTE: Pach \etal\ prove their result by showing that the number of
graphs with obstacle number at most $h$ is at most $2^{O(hn\log^2 n)}$.
Does our proof improve this bound?


\end{document}



