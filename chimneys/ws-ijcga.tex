%%%%%%%%%%%%%%%%%%%%%%%%%%%%%%%%%%%%%%%%%%%%%%%%%%%%%%%%%%%%%%%%%%%%%%%%%%%
%% Trim Size: 9.75in x 6.5in
%% Text Area: 8in (include Runningheads) x 5in
%% ws-ijcga.tex   :   19-10-2007
%% Class file to use with ws-ijcga.cls written in Latex2E. 
%% The content, structure, format and layout of this style file is the 
%% property of World Scientific Publishing Co. Pte. Ltd. 
%% Copyright 1995, 2002 by World Scientific Publishing Co. 
%% All rights are reserved.
%%%%%%%%%%%%%%%%%%%%%%%%%%%%%%%%%%%%%%%%%%%%%%%%%%%%%%%%%%%%%%%%%%%%%%%%%%%%
%

\documentclass{ws-ijcga}

\begin{document}

\markboth{Authors' Names}
{Instructions for Typing Camera-Ready Manuscripts (Paper's Title)}

\catchline

\title{INSTRUCTIONS FOR TYPESETTING CAMERA-READY \\
MANUSCRIPTS USING \TeX\ OR \LaTeX\footnote{For the title, try not to use 
more than 3 lines. Typeset the title in 10 pt roman, uppercase and
boldface.}
}

\author{FIRST AUTHOR\footnote{
Typeset names in 8 pt roman, uppercase. Use the footnote to indicate the
present or permanent address of the author.}}

\address{University Department, University Name, Address\\
City, State ZIP/Zone,Country\,\footnote{State completely without 
abbreviations, the affiliation and mailing address, 
including country. Typeset in 8 pt italic.}\\
e-mail address
}

\author{SECOND AUTHOR}

\address{Group, Laboratory, Address\\
City, State ZIP/Zone, Country\\
e-mail address
}

\maketitle

\pub{Received (received date)}{Revised (revised date)}
{Communicated by (Name)}

\begin{abstract}
The abstract should summarize the context, content
and conclusions of the paper in less than 200 words. It should
not contain any references or displayed equations. Typeset the
abstract in 8 pt roman with baselineskip of 10 pt, making
an indentation of 1.5 pica on the left and right margins.

\keywords{Keyword1; keyword2; keyword3.}
\end{abstract}

\section{General Appearance}	%) A SECTION HEADING

Contributions to the {\it International Journal of Computational Geometry 
\& Applications} will be reproduced by photographing the author's
submitted typeset manuscript. It is therefore essential that the
manuscript be in its final form, and of good appearance because
it will be printed directly without any editing. The manuscript
should also be clean and unfolded. The copy should be evenly
printed on a high resolution printer (600 dots/inch or higher).
If typographical errors cannot be avoided, use cut and paste
methods to correct them. Smudged copy, pencil or ink text
corrections will not be accepted. Do not use cellophane or
transparent tape on the surface as this interferes with the
picture taken by the publisher's camera.

\section{The Main Text}

Contributions are to be in English. Authors are encouraged to
have their contribution checked for grammar. American spelling
should be used. Abbreviations are allowed but should be spelt
out in full when first used. Integers ten and below are to be
spelt out. Italicize foreign language phrases (e.g.~Latin,
French).

The text is to be typeset in 10 pt roman, single spaced
with baselineskip of 13~pt. Text area (including copyright block)  
is 8 inches high and 5 inches wide for the first page.  
Text area (excluding running title) is 7.7 inches high and 
5 inches wide for subsequent pages.  Final pagination and 
insertion of running titles will be done by the publisher.

\section{Major Headings}

Major headings should be typeset in boldface with the first
letter of important words capitalized.

\subsection{Sub-headings}

Sub-headings should be typeset in boldface italic and capitalize
the first letter of the first word only. Section number to be in
boldface roman.

\subsubsection{Sub-subheadings}

Typeset sub-subheadings in medium face italic and capitalize the
first letter of the first word only. Section numbers to be in
roman.

\subsection{Numbering and spacing}

Sections, sub-sections and sub-subsections are numbered in
Arabic.  Use double spacing before all section headings, and
single spacing after section headings. Flush left all paragraphs
that follow after section headings.

\subsection{Lists of items}

Lists may be laid out with each item marked by a dot:
\begin{itemlist}
 \item item one,
 \item item two.
\end{itemlist}
Items may also be numbered in lowercase roman numerals:
\begin{romanlist}
\item item one
\item item two 
	\begin{romanlist}[(b)]
	\item Lists within lists can be numbered with lowercase 
              roman letters,
	\item second item. 
	\end{romanlist}
\end{romanlist}

\section{Equations}

Displayed equations should be numbered consecutively,
with the number set flush right and enclosed in parentheses
\begin{equation}
\mu(n, t) = {\sum^\infty_{i=1} 1(d_i < t, N(d_i) 
= n)}{\int^t_{\sigma=0} 1(N(\sigma) = n)d\sigma}\,.
\label{eq:jaa}
\end{equation}

Equations should be referred to in abbreviated form,
e.g.~``Eq.~(\ref{eq:jaa})'' or ``(2)''. In multiple-line
equations, the number should be given on the last line.

Displayed equations are to be centered on the page width.
Standard English letters like x are to appear as $x$
(italicized) in the text if they are used as mathematical
symbols. Punctuation marks are used at the end of equations as
if they appeared directly in the text.

\section{Theorem environments}

\begin{theorem} \label{theo1}
Theorems, lemmas, etc. are to be numbered consecutively in the
paper. Use double spacing before and after theorems, lemmas, etc.
\end{theorem}

The labels cited in the theorem environments text can 
cross link to the body text e.g. Theorem~\ref{theo1} 
and Lemma~\ref{lemm1}.

\begin{lemma}  \label{lemm1}
Theorems, lemmas, etc. are to be numbered consecutively in the
paper. Use double spacing before and after theorems, lemmas, etc.
\end{lemma}

\begin{proof}
Proofs should end with
\end{proof}

\section{Illustrations and Photographs}

\begin{figure}[b]
\centerline{\psfig{file=ijcgaf1.eps,width=5cm}}
\vspace*{8pt}
\caption{A schematic illustration of dissociative recombination. The
direct mechanism, 4m$^2_\pi$ is initiated when the
molecular ion S$_{\rm L}$ captures an electron with 
kinetic energy. \label{fig1}}
\end{figure}

Figures are to be inserted in the text nearest their first
reference.  Figure~\ref{fig1} placements can be either top or 
bottom. Softcopies of illustrations are to be in either EPS, PS
or TIF format, preferably on a PC platform. Please prepare in 
300 dpi for line drawings (black and white); 
300 dpi for halftones (gray scale); 
300 dpi for colour images. Must be in CMYK (Cyan, Magenta,
Yellow and Black) for colour separation. If the author requires the
publisher to reduce the figures, ensure that the figures (including
letterings and numbers) are large enough to be clearly seen after
reduction. If photographs are to be used, only black and white ones 
are acceptable.

Figures are to be sequentially numbered in Arabic numerals. The
caption must be placed below the figure. Typeset in 8 pt roman
with baselineskip of 10~pt. Use double spacing between a
caption and the text that follows immediately.

Previously published material must be accompanied by written
permission from the author and publisher.

\section{Tables}

Tables should be inserted in the text as close to the point of
reference as possible. Some space should be left above and below
the table.

Tables should be numbered sequentially in the text in Arabic
numerals. Captions are to be centralized above the tables.
Typeset tables and captions in 8 pt roman with
baselineskip of 10 pt.

\begin{table}[h]
\tbl{Comparison of acoustic for frequencies for piston-cylinder problem.}
{\begin{tabular}{@{}cccc@{}} \toprule
Piston mass & Analytical frequency & TRIA6-$S_1$ model &
\% Error \\
& (Rad/s) & (Rad/s) \\ \colrule
1.0\hphantom{00} & \hphantom{0}281.0 & \hphantom{0}280.81 & 0.07 \\
0.1\hphantom{00} & \hphantom{0}876.0 & \hphantom{0}875.74 & 0.03 \\
0.01\hphantom{0} & 2441.0 & 2441.0\hphantom{0} & 0.0\hphantom{0} \\
0.001 & 4130.0 & 4129.3\hphantom{0} & 0.16\\ \botrule
\end{tabular}}
\begin{tabnote}
Table notes
\end{tabnote}
\end{table}

If tables need to extend over to a second page,
the continuation of the table should be preceded by a caption,
e.g.~``{\it Table 2.} $(${\it Continued}$)$''

\section{Running Heads}

Please provide a shortened runninghead (not more than eight words) for
the title of your paper. This will appear on the top right-hand side
of your paper.

\section{Footnotes}

Footnotes should be numbered sequentially in superscript
lowercase roman letters.\footnote{Footnotes should be
typeset in 8 pt roman at the bottom of the page.}

\section*{Acknowledgements}

This section should come before the References. Funding
information may also be included here.

\appendix

\section{Appendices}

Appendices should be used only when absolutely necessary. They
should come after the References. If there is more than one
appendix, number them alphabetically. Number displayed equations
occurring in the Appendix in this way, e.g.~(\ref{appeqn}), (A.2),
etc.
\begin{equation}
\mu(n, t) = {\sum^\infty_{i=1} 1(d_i < t, N(d_i) 
= n)}{\int^t_{\sigma=0} 1(N(\sigma) = n)d\sigma}\,.
\label{appeqn}
\end{equation}

\section*{References}

References are to be listed in the order cited in the text in Arabic
numerals.  They can be typed in superscripts after punctuation marks,
e.g.~``$\ldots$ in the statement.\cite{dolve}'' or used directly,
e.g.~``see Ref.~5 for examples.'' Please list using the style shown in
the following examples.  For journal names, use the standard
abbreviations.  Typeset references in 9 pt roman.

\begin{thebibliography}{0}
\bibitem{lorentz}
R. Lorentz and D. B. Benson, Deterministic and nondeterministic
flow-chart interpretations, {\it J. Comput. System Sci.}
{\bf 27} (1983) 400--433.

\bibitem{beeson}
M. J. Beeson, {\it Foundations of Constructive Mathematics}
(Springer, Berlin, 1985).

\bibitem{clark}
K. L. Clark, Negations as failure, in {\it Logic and Data
Bases}, eds. H. Gallaire and J.~Winker (Plenum Press, New York,
1973) 293--306.

\bibitem{joliat}
M. Joliat, A simple technique for partial elimination of unit
productions from LR({\it k}) parsers, {\it IEEE Trans.
Comput.} {\bf 27} (1976) 753--764.

\bibitem{dolve}
D. Dolve, Unanimity in an unknown and unreliable environment, in
{\it Proc. 22nd Annual Symp. Foundations of Computer
Science}, Nashville, TN (Oct. 1981) pp.~159--168.

\bibitem{tamassia}
R. Tamassia, C. Batini and M. Talamo, An algorithm for automatic
layout of entity relationship diagrams, in {\it
Entity-Relationship Approach to Software Engineering, Proc. 3rd
Int.  Conf. Entity-Relationship Approach}, eds. C. G. Davis,
S. Jajodia, P. A. Ng and R. T. Yeh (North-Holland, Amsterdam,
1983) pp.~421--439.

\bibitem{gewirtz}
W. L. Gewirtz, Investigations in the theory of descriptive
complexity, Ph. D. Thesis, New York University (1974).
\end{thebibliography}

\end{document}




